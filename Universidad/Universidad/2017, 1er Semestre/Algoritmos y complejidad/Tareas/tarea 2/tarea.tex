\documentclass{article}
\usepackage[utf8]{inputenc}
\usepackage{graphicx}
\usepackage{amsmath}
\usepackage{amsfonts}
\renewcommand{\labelitemii}{$\star$}
\usepackage[top = 2.5cm, bottom = 2cm, left = 2cm, right = 2cm]{geometry}

\title{Tarea \#2 Algoritmos y Complejidad}
\author{Juan Pablo León, 201473047-0 }
\date{Abril 2017}

\begin{document}

\maketitle

\section*{Pregunta 1}
Ver archivos "Proceso Gram-Schmidt.py" y "ortogonalidad.py" donde se desarrolla y verifica la solución respectivamente. Sugerencia: los cálculos para comprobar la ortogonalidad pueden tardar varios minutos.

\section*{Pregunta 2}
Ver archivo "raices.py", se utiliza un comando "solve" para obtener las soluciones fácilmente.

\section*{Pregutna 3}
Siguiendo la regla de cuadratura gausseana se utiliza una productoria
dentro de la integral en "coeficientes.py" para obtener cada
coeficiente pedido.

\end{document}

% curvas-elipticas.tex
%
% Copyright (c) 2012-2014 Horst H. von Brand
% Derechos reservados. Vea COPYRIGHT para detalles

\subsection{Curvas elípticas}
\label{sec:curvas-elipticas}
\index{curva eliptica@curva elíptica|textbfhy}

  Una \emph{curva elíptica}
  está definida por una ecuación de la forma:
  \begin{equation*}
    \label{eq:elliptic-curve}
    y^2
      = x^3 + a x + b
  \end{equation*}
  \begin{figure}
    \centering
    \subfloat[\(y^2 = x^3 - x\)]{
      \pgfimage{images/curva-eliptica-1}
    }%
    \hspace{1em}%
    \subfloat[\(y^2 = x^3 - x + 1\)]{
      \pgfimage{images/curva-eliptica-2}
    }
    \caption{Curvas elípticas}
    \label{fig:curvas-elipticas}
  \end{figure}
  que no tiene puntos aislados,
  no se intersecta a sí misma y no tiene cuernos.
  Algebraicamente,
  el discriminante \(\Delta = - 16 (4 a^3 + 27 b^2) \ne 0\).%
    \index{discriminante}
  El gráfico de la curva tiene dos componentes si \(\Delta > 0\)
  y uno solo si \(\Delta < 0\),
  ver la figura~\ref{fig:curvas-elipticas}.
  El nombre no tiene relación con la forma de la curva,
  sino con el hecho que se requieren funciones elípticas
  para representarlas paramétricamente.

  \begin{figure}
    \centering
    \subfloat[Suma de \(\mathtt{P}_1 = (x_1, y_1)\)
	      y \(\mathtt{P}_2 = (x_2, y_2)\)]{
      \pgfimage{images/suma-curva-eliptica}
      \label{subfig:suma-curva-eliptica}
    }%
    \hspace{1em}%
    \subfloat[Doble de \(\mathtt{P} = (x, y)\)]{
      \pgfimage{images/doble-curva-eliptica}
      \label{subfig:doble-curva-eliptica}
    }
    \caption{Sumas en curvas elípticas}
    \label{fig:sumas-curva-eliptica}
  \end{figure}
  Dados dos puntos \(\mathtt{P}_1 = (x_1, y_1)\)
  y \(\mathtt{P}_2 = (x_2, y_2)\) sobre una curva elíptica
  podemos definir la suma
  como el punto donde la recta entre los puntos corta la curva
  reflejado en el eje \(x\),
  véase~\ref{subfig:suma-curva-eliptica} para un ejemplo.
  Esto hace que si \(\mathtt{P}_1\),
  \(\mathtt{P}_2\) y \(\mathtt{P}_3\)
  son puntos sobre la curva,
  es \(\mathtt{P}_1 + \mathtt{P}_2 + \mathtt{P}_3 = 0\),
  donde 0 es el punto en el infinito.
  En caso que \(x_1 = x_2\)
  hay dos posibilidades:
  Si \(y_1 = - y_2\),
  (incluyendo el caso en que los puntos coinciden),
  la suma se define como \(0\)
  (el punto en el infinito).
  Tenemos así para \(\mathtt{P} = (x, y)\)
  que \(- \mathtt{P} = (x, -y)\).
  En caso contrario
  definimos \(\mathtt{P}_1 + \mathtt{P}_2 = \mathtt{P}_3\)
  con \(\mathtt{P}_3 = (x_3, y_3)\)
  mediante:
  \begin{align}
    s
      &= \frac{y_2 - y_1}{x_2 - x_1} \notag \\
    x_3
      &= s^2 - x_1 - x_2 \label{eq:suma-curva-eliptica} \\
    y_3
      &= y_1 + s (x_3 - x_1) \notag
  \end{align}
  Para sumar el punto \(\mathtt{P} = (x, y)\) consigo mismo
  corresponde usar la tangente a la curva,
  ver la figura~\ref{subfig:doble-curva-eliptica},
  lo que da \(\mathtt{P}_2 = (x_2, y_2)\):
  \begin{align}
    s
      &= \frac{3 x + a}{2 y} \notag \\
    x_2
      &= s^2 - 2 x \label{eq:doble-curva-eliptica} \\
    y_2
      &= y + s (x_2 - x) \notag
  \end{align}
  Es rutina verificar que esto define un grupo abeliano.%
    \index{curva eliptica@curva elíptica!grupo}%
    \index{grupo!abeliano}

  Lo interesante es que las relaciones
  que definen la suma en curvas elípticas
  valen en cualquier campo,
  por lo que podemos considerar
  el grupo definido por la curva elíptica sobre un campo cualquiera.
  Si la característica del campo \(F\) no es \(2\) ni \(3\)
  (vale decir,
   no es \(2 x = 0\) ni \(3 x = 0\) para todo \(x \in F\);
   la discusión formal deberá esperar
   al capítulo~\ref{cha:campos-finitos}),
  toda curva elíptica puede escribirse en la forma:
  \begin{equation*}
    y^2
      = x^3 - p x - q
  \end{equation*}
  tal que el lado derecho no tiene ceros repetidos.
  Interesan los puntos con coordenadas en \(F\).
  El teorema de Hasse~%
    \cite{hasse36:_EC-I, hasse36:_EC-II, hasse36:_EC-III}
  da las cotas para el número \(N\) de elementos en curvas elípticas
  sobre el campo finito de \(q\) elementos:
  \begin{equation*}
    \lvert N - (q + 1) \rvert
      \le 2 \sqrt{q}
  \end{equation*}

  Las curvas elípticas son importantes en teoría de números,
  y tienen bastantes aplicaciones prácticas,
  particularmente se están haciendo muy populares en criptografía.%
    \index{criptografia@criptografía}%
  El sistema \texttt{PARI/GP}~\cite{PARI:2.7.2}%
    \index{PARI/GP@\texttt{PARI/GP}}
  incluye soporte para operar en los grupos respectivos.
  El paquete GAP~%
    \cite{GAP:4.7.5}%
    \index{GAP@\texttt{GAP}}
  tiene extenso soporte para trabajar con grupos,
  incluyendo grupos de curvas elípticas.

%%% Local Variables:
%%% mode: latex
%%% TeX-master: "clases"
%%% End:

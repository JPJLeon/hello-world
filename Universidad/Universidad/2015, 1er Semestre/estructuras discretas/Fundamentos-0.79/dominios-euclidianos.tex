% dominios-euclidianos.tex
%
% Copyright (c) 2012-2014 Horst H. von Brand
% Derechos reservados. Vea COPYRIGHT para detalles

\section{Dominios euclidianos}
\label{sec:dominios-euclidianos}

  A un dominio integral \(D\)
  equipado con una \emph{función euclidiana}
  (a veces llamada \emph{función grado} o simplemente \emph{grado})
  \(g \colon D \smallsetminus \{0\} \rightarrow \mathbb{N}\)
  tal que si \(a, b \in D\),
  con \(b \ne 0\),
  hay \(q, r \in D\)
  tales que \(a = q b + r\)
  con \(r = 0\) o \(g(r) < g(b)\)
  se le llama \emph{dominio euclidiano}.
  Estas estructuras tienen mucho en común con \(\mathbb{Z}\)
  (en particular,
   toman su nombre porque es aplicable el algoritmo de Euclides
   para calcular máximo común divisor,
   y tenemos el equivalente
   de la identidad de \foreignlanguage{french}{Bézout}).
  En el caso de los polinomios,
  el grado sirve como función euclidiana.

  Tenemos algunos resultados simples:
  \begin{theorem}
    \label{theo:ED:g-minimo=>unidad}
    Sea \(D\) un dominio euclidiano con función euclidiana \(g\).
    El valor \(g(a)\) es mínimo si \(a\) es una unidad.
  \end{theorem}
  \begin{proof}
    Tomemos \(a \ne 0\) en \(D\) tal que \(g(a)\) es mínimo.
    Por el algoritmo de división,
    teorema~\ref{theo:F[x]:division-algorithm},
    tenemos \(1 = q a + r\) con \(r = 0\) o \(g(r) < g(a)\).
    Pero \(g(a)\) es mínimo,
    por lo que debe ser \(r = 0\)
    y \(a\) es una unidad.
  \end{proof}
  También tenemos las propiedades
  (ver Rogers~%
    \cite{rogers71:_axiom_euclidean_domain}
   y Samuel~%
    \cite{samuel71:_about_euclidean_rings}):
  \begin{theorem}
    \label{theo:ED:propiedades-f}
    Sea \(D\) un dominio euclidiano
    con función euclidiana \(g\).
    La función definida por:
    \begin{equation*}
      f(a)
	= \min_{x \in D \smallsetminus \{0\}} g(a x)
    \end{equation*}
    es una función euclidiana,
    y cumple:
    \begin{enumerate}[label=(\alph{*})]
    \item
      \label{en:fe:a}
      \(f(a) \le f(a b)\) si \(a b \ne 0\)
    \item
      \label{en:fe:b}
      \(f(a) \le g(a)\) para todo \(a \in D \smallsetminus \{0\}\)
    \item
      \label{en:fe:c}
      \(f(a u) = f(a)\) si y solo si \(u \in D^\times\)
    \end{enumerate}
  \end{theorem}
  \begin{proof}
    Por la definición de \(f\)
    los puntos~\ref{en:fe:a} y~\ref{en:fe:b} son obvios.

    Para demostrar que \(f\) es euclidiana,
    consideremos elementos \(a, b\)
    cualquiera en \(D \smallsetminus \{0\}\).
    Debemos demostrar que si \(b = q a + r\)
    entonces \(r = 0\) o \(f(r) < f(a)\).

    El caso \(r = 0\) es trivial.
    Supongamos entonces \(r \ne 0\).
    Por definición
    es \(f(a) = g(a c)\)
    para algún \(c \in D \smallsetminus \{0\}\).
    De la definición de \(r\) tenemos que \(g(r) < g(a)\)
    por ser \(g\) euclidiana.
    De \(b c = q a c + r c\),
    por ser \(g\) euclidiana es \(g(r c) < g(a c) = f(a)\);
    y por la definición de \(f\) es también \(f(r) \le g(r c)\).
    Uniendo las anteriores queda \(f(r) < f(a)\),
    y \(f\) es euclidiana.

    Para~\ref{en:fe:c} demostramos implicancia en ambas direcciones.
    Primero,
    sea \(u \in D^\times\).
    Por el punto~\ref{en:fe:a}
    es \(f(a) \le f(a u) \le f((a u) u^{-1}) = f(a)\).
    Por otro lado,
    si \(f(a c) = f(a)\),
    escribimos \(a = q a c + r\) con \(f(r) < f(a c) = f(a)\);
    siendo \(r = a (1 - c q)\),
    por la parte~\ref{en:fe:a}
    si \(r \ne 0\) es \(f(r) \ge f(a)\),
    lo que es absurdo.
    Así \(r = 0\) y \(c\) es una unidad.
  \end{proof}
  Supondremos una función euclidiana
  tal que \(f(a) \le f(a b)\) para todo \(a, b \in D\)
  desde ahora,
  ya que simplifica mucha de la discusión que sigue.
  Nótese que en particular el grado de polinomios cumple esto.
  Como la función euclidiana no es única,
  no la incluimos en la definición del dominio.
  \begin{definition}
    Sea \(R\) un dominio integral.
    Si podemos escribir \(m = b c\),
    decimos que \(b\) \emph{divide a} \(m\),
    y anotamos \(b \mid m\).
  \end{definition}
  Esto también se expresa
  diciendo que \(b\) es un \emph{factor} de \(m\),
  o que \(m\) es un \emph{múltiplo} de \(b\).
  \begin{definition}
    Sea \(R\) un dominio integral.
    Dos elementos \(a, b \in R\) se dicen \emph{asociados}
    si \(a = u b\),
    donde \(u\) es una unidad.
    Se anota \(a \sim b\).
  \end{definition}
  Es fácil ver que \(\sim\) es una relación de equivalencia.
  \begin{definition}
    Sea \(R\) un dominio integral.
    Un elemento \(e \in R \smallsetminus R^\times\)
    se llama \emph{irreductible} si siempre que
    \(e = u \cdot v\),
    \(u\) o \(v\) es una unidad.
    En caso contrario,
    decimos que \(e\) es \emph{reductible}.
  \end{definition}
  \begin{definition}
    Sea \(R\) un dominio integral,
    \(p \in R \smallsetminus R^\times\).
    Si \(p \mid a b\) implica que \(p \mid a\) o \(p \mid b\),
    se dice que \(p\) es \emph{primo}.
  \end{definition}
  Vemos que si un elemento es primo,
  es irreductible:
  \begin{lemma}
    \label{lem:prime=>irreducible}
    Sea \(R\) un dominio integral.
    Si \(p \in R\) es primo,
    entonces es irreductible.
  \end{lemma}
  \begin{proof}
    Por contradicción.
    Supongamos \(p\) primo pero no irreductible.
    Así podemos escribir \(p = u v\),
    donde \(u, v \notin R^\times\).
    Pero entonces \(p \mid u v\),
    y por la definición de primo
    es \(p \mid u\) o \(p \mid v\).
    Sin pérdida de generalidad
    podemos suponer \(u = a p\),
    con lo que:
    \begin{align*}
      p
	&= a p v \\
      0
	&= p (1 - a v)
    \end{align*}
    Como no hay divisores de cero distintos de cero en \(R\),
    debe ser \(a v = 1\)
    y \(v\) es una unidad,
    lo que contradice su elección.
  \end{proof}
  El recíproco del lema~\ref{lem:prime=>irreducible}
  no siempre se cumple.
  Consideremos el dominio integral
  \(\mathbb{Z}[\sqrt{-5}]\)
  (ver la sección~\ref{sec:anillos-cuadraticos},
   solo que este es un subanillo de \(\mathbb{C}\)).
  Si \(3 = u \cdot v\),
  debe ser \(N(u) \cdot N(v) = N(3) = 9\),
  con lo que las normas posibles para \(u\) y \(v\)
  son los divisores de 9.
  Si \(N(u) = 1\),
  \(u\) es una unidad.
  Si \(N(u) = 3\),
  con \(u = u_1 + u_2 \sqrt{-5}\)
  es \(u_1^2 + 5 u_2^2 = 3\).
  Esto claramente es imposible con \(u_1\), \(u_2\) enteros.
  Así \(3\) es irreductible en \(\mathbb{Z}[\sqrt{-5}]\).
  Por el otro lado:
  \begin{align*}
    &(2 + \sqrt{-5}) \cdot (2 - \sqrt{-5})
      = 9 \\
  \intertext{con lo que}
    &3
      \mid (2 + \sqrt{-5}) \cdot (2 - \sqrt{-5})
  \end{align*}
  Claramente \(3 \centernot\mid 2 \pm \sqrt{-5}\),
  o sea 3 no es primo en \(\mathbb{Z}[\sqrt{-5}]\).

  \begin{definition}
    Sea \(R\) un dominio integral,
    y sea \(a \in R\) con \(a \ne 0\).
    Entonces se dice
    que \emph{\(a\) tiene factorización única en irreductibles}
    si hay una unidad \(u\)
    e irreductibles \(p_i\) tales que \(a = u p_1 p_2 \dotsm p_r\),
    y además,
    si \(a = v q_1 q_2 \dotsm q_s\) para una unidad \(v\)
    e irreductibles \(q_i\),
    entonces \(r = s\)
    y \(p_i = u_i q_i\)
    para unidades \(u_i\) salvo reordenamiento.
  \end{definition}
  \begin{definition}
    Se dice que \(R\) es un \emph{dominio de factorización única}
    (en inglés
     \emph{\foreignlanguage{english}{Unique Factorization Domain}}.
     abreviado \emph{UFD})
    si todo elemento de \(R\)
    tiene factorización única en irreductibles.
  \end{definition}

  En vista del algoritmo de división en el dominio euclidiano,
  tenemos:
  \begin{theorem}
    \label{theo:ED=>PID}
    Sea \(D\) un dominio euclidiano con función euclidiana \(f\)
    y \(a, b \in D\).
    Entonces el conjunto \(I = \{u a + v b \colon u, v \in D\}\)
    consta de todos los múltiplos de un elemento \(m\).
  \end{theorem}
  La demostración es muy similar
  a la discusión sobre máximo común divisor
  en el capítulo~\ref{cha:teoria-numeros}.
  \begin{proof}
    Si \(a = b = 0\),
    claramente \(I = \{0\}\),
    y lo aseverado se cumple.
    Supongamos entonces que al menos uno
    de \(a\), \(b\) es diferente de 0,
    en cuyo caso \(I\) contiene elementos diferentes de 0.
    Elijamos uno de ellos con \(f\) mínimo,
    llamémosle \(m\).
    Tomemos ahora \(n \in I\) cualquiera.
    Si \(n = 0\),
    se cumple \(m \mid n\),
    y estamos listos.
    Si \(n \ne 0\),
    podemos aplicar el algoritmo de división y escribir:
    \begin{equation*}
      n = q m + r
    \end{equation*}
    donde \(r = 0\) o \(f(r) < f(m)\).
    Dado que hay \(u, v, u', v'\) tales que \(n = u a + v b\)
    y \(m = u' a + v' b\) resulta:
    \begin{align*}
      r
	&= n - q m \\
	&= (u - q u') a + (v - q v') b
    \end{align*}
    con lo que \(r \in I\).
    Pero no puede ser \(f(r) < f(m)\),
    hemos elegido \(m\) precisamente por ser \(f(m)\) mínimo.
    En consecuencia,
    \(r = 0\) y \(m \mid n\).
  \end{proof}
  Conjuntos como \(I\)
  que aparece en la demostración del teorema~\ref{theo:ED=>PID}
  son muy importantes.
  Podemos definir \(m\)
  (o uno de sus asociados,
   que también son parte de \(I\);
   \(m\) no necesariamente es único)
  como un máximo común divisor de \(a\) y \(b\)
  (``máximo'' en el sentido de la función euclidiana \(f\)).
  \begin{definition}
    Sea \(R\) un anillo conmutativo.
    Un \emph{ideal} de \(R\)
    es un conjunto \(I \subseteq R\)
    tal que:
    \begin{enumerate}
    \item
      \((I, +)\) es un subgrupo de \((R, +)\)
    \item
      Para todo \(x \in I\)
      y para todo \(r \in R\)
      se cumple \(r \cdot x \in I\)
    \end{enumerate}
  \end{definition}
  Los ideales son casi subanillos de \(R\)
  (solo falta el elemento \(1\)).
  Hay quienes definen anillos sin \(1\),
  para ellos los ideales son subanillos.
  \begin{definition}
    Sea \(R\) un anillo conmutativo,
    y \(\{x_1, x_2, \dotsc, x_n\} \subseteq R\).
    Al ideal
      \(\{\sum_{1 \le k \le n} u_k x_k \colon u_k \in R\}\)
    se le llama
    el \emph{ideal generado por \(\{x_1, x_2, \dotsc, x_n\}\)},
    que se suele anotar \((x_1, x_2, \dotsc, x_n)\).
    Por la convención que sumas vacías son cero,
    \(\{0\}\) es generado por \(\varnothing\).
    A un ideal generado por un único elemento \(x_1\),
    anotado \((x_1)\),
    se le llama \emph{ideal principal}.
    Si en \(R\) todos los ideales son principales,
    se dice que \(R\) es un \emph{dominio de ideal principal}
    (en inglés
     \emph{\foreignlanguage{english}{Principal Ideal Domain}},
     abreviado \emph{PID}).
  \end{definition}
  En estos términos,
  el teorema~\ref{theo:ED=>PID} asevera que todo dominio euclidiano
  es un dominio de ideal principal.

  \begin{lemma}
    \label{lem:ED:irreductible-coprime}
    Sea \(p\) irreductible en un dominio euclidiano \(D\),
    y \(a\) otro elemento de \(D\).
    Si \(p\) no divide a \(a\),
    entonces \(1\) es un máximo común divisor entre \(a\) y \(p\).
  \end{lemma}
  \begin{proof}
    Sea \(m\) un máximo común divisor de \(a\) y \(p\).
    Por el teorema~\ref{theo:ED=>PID}
    existen \(x, y \in D\) tales que:
    \begin{equation*}
      m = x a + y p
    \end{equation*}
    Como \(m\) divide a \(p\),
    que es irreductible,
    \(m\) es una unidad o \(m \sim p\).
    Si \(m\) es una unidad,
    \(1\) es un máximo común divisor de \(a\) y \(p\)
    y estamos listos.
    En el otro caso,
    por ser \(m\) divisor de \(a\)
    es \(a = c m\) para algún \(c \in D\)
    y como a su vez \(m = u p\) para una unidad \(u\),
    entonces \(a = c u p\) y \(p \mid a\).
  \end{proof}
  Esto nos permite demostrar
  el recíproco del lema~\ref{lem:prime=>irreducible}
  en dominios euclidianos,
  como ya lo hicimos en el teorema~\ref{theo:Z:irreductible=>prime}
  para los enteros:
  \begin{theorem}
    \label{theo:ED:irreducible=>prime}
    En un dominio euclidiano,
    si \(p\) es irreductible
    entonces \(p\) es primo.
  \end{theorem}
  \begin{proof}
    Supongamos que el irreductible \(p\) divide a \(a b\).
    Debemos demostrar que \(p\) divide a \(a\) o a \(b\)
    (o a ambos).
    Si \(p \mid a\),
    estamos listos.
    En caso contrario,
    como \(p \mid a b\),
    hay un \(c \in D\) tal que \(a b = c p\).
    Por el lema~\ref{lem:ED:irreductible-coprime}
    tenemos que \(1\) es un máximo común divisor de \(a\) y \(p\),
    y por el teorema~\ref{theo:ED=>PID} podemos escribir:
    \begin{align*}
      1
	&= u p + v a \\
      b
	&= b u p + v a b \\
	&= (b u + v c) p
    \end{align*}
    con lo que \(p \mid b\).
  \end{proof}
  \begin{lemma}
    \label{lem:ED:primo-divide-producto}
    Si el primo \(p\) divide al producto \(x_1 x_2 \dotsm x_n\),
    entonces \(p \mid x_i\) para algún \(i\).
  \end{lemma}
  \begin{proof}
    Por inducción sobre \(n\).
    Si \(n = 1\),
    no hay nada que demostrar.
    \begin{description}
    \item[Base:]
      Para \(n = 2\),
      por la definición de primo tenemos que si \(p \mid x_1 x_2\),
      entonces \(p \mid x_1\) o \(p \mid x_2\).
    \item[Inducción:]
      Por la hipótesis de inducción,
      si \(p \mid x_1 x_2 \dotsm x_n\)
      entonces \(p \mid x_i\) para \(1 \le i \le n\).
      Si ahora \(p \mid x_1 x_2 \dotsm x_n x_{n + 1}\)
      por el caso \(n = 2\) significa que ya sea
      \(p \mid x_1 x_2 \dotsm x_n\)
      (lo que implica \(p \mid x_i\) para \(1 \le i \le n\))
      o \(p \mid x_{n + 1}\).
      En conjunto,
      \(p \mid x_i\) para \(1 \le i \le n + 1\).
    \end{description}
    Por inducción,
    vale para todo \(n \in \mathbb{N}\).
  \end{proof}
  Así tenemos:
  \begin{theorem}
    \label{theo:PID=>UFD}
    Si \(D\) es un dominio de ideal principal,
    entonces es un dominio de factorización única.
  \end{theorem}
  \begin{proof}
    Por contradicción.
    Consideremos un elemento \(a \in D \smallsetminus D^\times\)
    (distinto de \(0\))
    con \(f(a)\) mínimo
    y que no tiene factorización en irreductibles.
    Entonces \(a\) no es irreductible
    (sería el producto de un irreductible),
    por lo que podemos escribir \(a = b c\),
    con \(b, c \notin D^\times\).
    Por el teorema~\ref{theo:ED:propiedades-f}
    resulta \(f(b) < f(a)\) y \(f(c) < f(a)\).
    Pero entonces \(b\) y \(c\) son producto de irreductibles,
    con lo que lo es \(a\).
    Vale decir,
    tal \(a\) no existe.

    Para demostrar factorización única usamos reducción al absurdo.
    Sea \(a\) un elemento de mínimo \(f\)
    que tiene dos factorizaciones esencialmente diferentes:
    \begin{equation*}
      a
	= u p_1 p_2 \dotsm p_m
	= v q_1 q_2 \dotsm q_n
    \end{equation*}
    donde los \(p_i\) son primos
    (no necesariamente diferentes),
    y similarmente los \(q_i\),
    y \(u\) y \(v\) son unidades.
    Por el lema~\ref{lem:ED:primo-divide-producto},
    esto significa que \(p_1\) divide a \(q_i\) para algún \(i\),
    o sea \(q_i = u_i p_i\) para alguna unidad \(u_i\),
    con lo que:
    \begin{equation*}
      a / p_1
	= u p_2 \dotsm p_m
	= v u_i q_1 \dotsm q_{i -1} q_{i + 1} \dotsm q_n
    \end{equation*}
    tendría dos factorizaciones diferentes,
    pero \(f(a / p_1) < f(a)\),
    lo que contradice la elección de \(a\) como uno de mínimo \(f\)
    con dos factorizaciones.
  \end{proof}
  Esto viene a ser el equivalente
  del teorema fundamental de la aritmética
  (teorema~\ref{theo:fundamental-aritmetica}):
  En un dominio euclidiano
  todo elemento \(a\) es el producto de un número finito de primos.
  Además,
  si tenemos factorizaciones en primos \(p_i\) y \(q_i\):
  \begin{equation*}
    a
      = p_1 p_2 \dotsm p_m
      = q_1 q_2 \dotsm q_n
  \end{equation*}
  entonces cada \(p\) es el asociado de uno de los \(q\).
  En particular,
  \(m = n\).

%%% Local Variables:
%%% mode: latex
%%% TeX-master: "clases"
%%% End:

% campos-finitos.tex
%
% Copyright (c) 2010, 2012-2014 Horst H. von Brand
% Derechos reservados. Vea COPYRIGHT para detalles

\chapter{Campos finitos}
\label{cha:campos-finitos}
\index{campo (algebra)@campo (álgebra)!finito|textbfhy}

  Gran parte de las matemáticas
  giran alrededor del álgebra abstracta,%
    \index{algebra abstracta@álgebra abstracta}
  que hemos conocido en los grupos%
    \index{grupo}
  y anillos.%
    \index{anillo}
  Pero sin duda las estructuras algebraicas más importantes
  son los campos,%
    \index{campo (algebra)@campo (álgebra)}
  que definimos en el capítulo~\ref{cha:teoria-numeros}
  como anillos conmutativos en los que todos los elementos
  (salvo \(0\))
  tienen inverso multiplicativo.
  Implícitamente usamos el campo de los reales \(\mathbb{R}\),%
    \index{R (números reales)@\(\mathbb{R}\) (números reales)}
  ocasionalmente nos aventuramos a los números complejos \(\mathbb{C}\).%
    \index{C (numeros complejos)@\(\mathbb{C}\) (números complejos)}
  Pero en realidad nuestros cálculos
  casi siempre son con aproximaciones racionales en \(\mathbb{Q}\).%
    \index{Q (números racionales)@\(\mathbb{Q}\) (números racionales)}
  De particular interés son los campos finitos,
  con una bonita teoría y abundantes aplicaciones prácticas.

\section{Propiedades básicas}
\label{sec:propiedades-campo}

  El orden aditivo de \(1\) en el campo \(F\)
  determina en gran medida la estructura del campo,
  y se le llama la \emph{característica del campo}.%
    \index{campo (algebra)@campo (álgebra)!caracteristica@característica}
  Se dice
  \(\chr(F) = n\) si el orden es \(n\)
  y \(\chr(F) = 0\) si es infinito.

  Si \(F\) es un campo,
  y \(K\) es un subcampo de \(F\)%
    \index{campo (algebra)@campo (álgebra)!subcampo}
  (cosa que se anota \(K \le F\))
  se dice que \(F\) es un \emph{campo extensión} de \(K\);%
    \index{campo (algebra)@campo (álgebra)!extension@extensión}
  si \(K \ne F\)
  decimos que \(K\) es un \emph{subcampo propio} de \(F\)%
    \index{campo (algebra)@campo (álgebra)!subcampo!propio}
  (se anota \(K < F\) en este caso).
  Se le llama \emph{subcampo primo} de \(F\)%
    \index{campo (algebra)@campo (álgebra)!subcampo!primo}
  a la intersección entre todos los subcampos de \(F\).
  El subcampo primo no tiene subcampos a su vez.
  Anotamos \(K \cong F\)
  si los campos \(K\) y \(F\) son isomorfos%
    \index{campo (algebra)@campo (álgebra)!isomorfo}
  \begin{theorem}
    \label{theo:prime-subfield}
    Sea \(K\) el subcampo primo de \(F\).
    Entonces:
    \begin{enumerate}[label = (\roman*), ref = (\roman*)]
    \item
      \label{en:prime-subfield-Q}
      Si \(F\) tiene característica \(0\),
      entonces \(K \cong \mathbb{Q}\)
    \item
      \label{en:prime-subfield-Zp}
      Si \(F\) tiene característica \(p\),
      entonces \(p\) es primo y \(K \cong \mathbb{Z}_p\)
    \end{enumerate}
  \end{theorem}
  \begin{proof}
    Llamaremos \(0_K\) y \(1_K\) a los elementos neutros de \(K\).
    \begin{enumerate}[label = (\roman*), ref = (\roman*)]
    \item
      En este caso para \(a \in \mathbb{Z}\)
      tenemos que \(a \cdot 1_K \in K\),
      y como \(K\) es un campo para \(b \in \mathbb{N}\)
      también \((a \cdot 1_K) (b \cdot 1_K)^{-1} \in K\),
      y esto es isomorfo al campo \(\mathbb{Q}\).
      Como \(K\) no tiene subcampos,
      \(K \cong \mathbb{Q}\).
    \item
      Nuevamente para \(a \in \mathbb{Z}\)
      tenemos que \(a \cdot 1_K \in K\),
      pero \(p \cdot 1_K = 0_K\),
      con lo que hay un subcampo de \(K\)
      que es isomorfo a \(\mathbb{Z}_p\),
      y como \(K\) es mínimo es \(K \cong \mathbb{Z}_p\).
      Pero \(\mathbb{Z}_p\) es campo si y solo si \(p\) es primo.
    \qedhere
    \end{enumerate}
  \end{proof}
  Si el campo finito \(F\) no es isomorfo a ningún \(\mathbb{Z}_p\),
  habrá algún elemento que no pertenece a su subcampo primo \(K\),
  llamémosle \(\alpha\).
  Pero en tal caso,
  los elementos \(\{k \alpha \colon k \in K\}\)
  deben ser todos distintos entre sí,
  y salvo \(0\) ninguno pertenece a \(K\).
  Así tenemos elementos
    \(\{k_0 + k_1 \alpha \colon k_0, k_1 \in K\}\).
  Debemos además incluir las potencias de \(\alpha\).
  Si la primera potencia de \(\alpha\)
  que pertenece a \(K\) es \(\alpha^m\),
  tendremos elementos
    \(k_0 + k_1 \alpha + \dotsb + k_{m - 1} \alpha^{m - 1}\),
  todos diferentes.
  Cualquier elemento \(\beta\) aún no considerado
  dará lugar a una construcción similar sobre los anteriores.
  Repitiendo este proceso,
  vemos que hay
  una colección de \(n\) elementos
    \(\alpha_i \in F \smallsetminus K\)
  (elementos como \(\alpha\) y \(\beta\) mencionados arriba,
   sus potencias,
   y productos de ellas)
  tales que eligiendo adecuadamente los \(k_i \in K\)
  podemos representar cualquier elemento \(f \in F\)
  mediante la expresión:
  \begin{equation*}
    f
      = \sum_{1 \le i \le n} k_i \alpha_i
  \end{equation*}
  Por la construcción anterior,
  cada elección de los \(k_i\)
  da lugar a un elemento diferente de \(F\),
  con lo que concluimos que si la característica del campo es \(p\),
  y el campo es finito,
  su orden debe ser \(p^n\) para \(n \in \mathbb{N}\).
  Para discutir este fenómeno se requieren conceptos adicionales,
  para mayores detalles véase por ejemplo el texto de Strang~%
    \cite{strang09:_intr_linear_algebra}.

% espacios-vectoriales.tex
%
% Copyright (c) 2012-2014 Horst H. von Brand
% Derechos reservados. Vea COPYRIGHT para detalles

\section{Espacios vectoriales}
\label{sec:espacios-vectoriales}
\index{espacio vectorial}

  Una estructura algebraica común es el espacio vectorial.
  Es aplicable a una gran variedad de situaciones,
  algunas bastante inesperadas.
  \begin{definition}
    Sea \(F\) un campo
    (sus elementos los llamaremos \emph{escalares})%
      \index{espacio vectorial!escalar|textbfhy}
    y \(V\) un conjunto
    (los \emph{vectores},%
      \index{espacio vectorial!vector|textbfhy}
     que por convención anotaremos en negrita).
    Hay operaciones \emph{suma de vectores}%
      \index{espacio vectorial!operaciones}
    (anotada \(+\))
    y \emph{producto escalar}
    entre un escalar y un vector
    (anotada \(\cdot\)).
    Se dice que \(V\) es un \emph{espacio vectorial sobre \(F\)}
    si cumple con los siguientes axiomas,%
      \index{espacio vectorial!axiomas}%
      \index{axioma!espacio vectorial}
    donde \(\alpha, \beta, \dotsc \in F\),
    y \(\boldsymbol{v}_1, \boldsymbol{v}_2, \dotsc \in V\).
    \begin{enumerate}[label=\textbf{V\arabic{*}:}, ref=V\arabic{*}]
    \item\label{ax:V:associative}
      \((\boldsymbol{v}_1 + \boldsymbol{v}_2) + \boldsymbol{v}_3
	   = \boldsymbol{v}_1
	       + (\boldsymbol{v}_2
	       + \boldsymbol{v}_3)\)
    \item\label{ax:V:neutral}
      Hay un elemento \(\boldsymbol{0} \in V\)
      tal que para todo \(\boldsymbol{v} \in V\)
      se cumple
      \(\boldsymbol{v} + \boldsymbol{0} = \boldsymbol{v}\)
    \item\label{ax:V:inverse}
      Para cada \(\boldsymbol{v} \in V\)
      hay \(- \boldsymbol{v} \in V\)
      tal que \(\boldsymbol{v} + (- \boldsymbol{v})
		  = \boldsymbol{0}\)
    \item\label{ax:V:commutative}
       \(\boldsymbol{v}_1 + \boldsymbol{v}_2
	   = \boldsymbol{v}_2 + \boldsymbol{v}_1\)
    \item\label{ax:V:scalar(vectorsum)}
      \(\alpha \cdot ( \boldsymbol{v}_1 + \boldsymbol{v}_2 )
	  = \alpha \cdot \boldsymbol{v}_1
	      + \alpha \cdot \boldsymbol{v}_2\)
    \item\label{ax:V:(scalarsum)vector}
      \((\alpha + \beta) \cdot \boldsymbol{v}
	  = \alpha \cdot \boldsymbol{v}
	      + \beta \cdot \boldsymbol{v}\)
    \item\label{ax:V:scalar-scalar-vector}
      \(\alpha \cdot (\beta \cdot \boldsymbol{v})
	  = (\alpha \beta) \cdot \boldsymbol{v}\)
     \item\label{ax:V:1-vector}
       Si \(1\) es el neutro multiplicativo de \(F\),
       \(1 \cdot \boldsymbol{v} = \boldsymbol{v}\)
    \end{enumerate}
  \end{definition}
  \noindent
  En resumen,
  \((V, +)\) es un grupo abeliano%
    \index{grupo!abeliano}
  (axiomas~\ref{ax:V:associative} a~\ref{ax:V:commutative}),
  junto con el campo \(F\) y multiplicación escalar que cumple
  los axiomas adicionales~\ref{ax:V:scalar(vectorsum)}
  a~\ref{ax:V:1-vector}.
  Normalmente indicaremos la multiplicación escalar
  por simple yuxtaposición.
  Dejamos de ejercicio
  demostrar que \(0 \cdot \boldsymbol{v} = \boldsymbol{0}\)
  y que \((- \alpha) \cdot \boldsymbol{v}
	    = - ( \alpha \cdot \boldsymbol{v})\).
  \begin{definition}
    Sea \(V\) un espacio vectorial sobre el campo \(F\).
    Si para el conjunto de vectores \(B\)
    es:
    \begin{equation*}
      \sum_{\boldsymbol{b} \in B}
	\alpha_{\boldsymbol{b}} \boldsymbol{b}
	= \boldsymbol{0}
    \end{equation*}
    solo si \(\alpha_{\boldsymbol{b}} = 0\)
    para todo \(\boldsymbol{b} \in B\)
    se dice que esos vectores
    son \emph{linealmente independientes}.%
      \index{espacio vectorial!independencia lineal}
  \end{definition}
  Si un conjunto de vectores no es linealmente independiente
  se dice que son \emph{linealmente dependientes}.
  Nótese que \(\boldsymbol{0}\)
  nunca pertenece
  a un conjunto de vectores linealmente independientes,
  ya que al multiplicarlo
  por cualquier escalar obtenemos \(\boldsymbol{0}\).
  \begin{definition}
    Sea \(V\) un espacio vectorial sobre \(F\),
    y \(B \subseteq V\) un conjunto de vectores.
    El \emph{espacio vectorial generado por \(B\)}
    es el conjunto:
    \begin{equation*}
      \langle B \rangle
	= \left\{
	    \sum_{\boldsymbol{b} \in B}
	      \alpha_{\boldsymbol{b}} \boldsymbol{b}
	       \colon \alpha_{\boldsymbol{b}} \in F
	  \right\}
    \end{equation*}
    Si \(V = \langle B \rangle\),
    se dice que \(B\) \emph{abarca} \(V\).
  \end{definition}
  En particular:
  \begin{definition}
    \index{espacio vectorial!base|textbfhy}
    Una \emph{base} del espacio vectorial \(V\)
    es un conjunto linealmente independiente de vectores \(B\)
    que abarca \(V\).
  \end{definition}
  La representación de \(\boldsymbol{v} \in V\)
  en términos de la base \(B\)
  es única,
  ya que si hubieran dos representaciones diferentes
  darían una dependencia lineal en \(B\).
  Para el vector:
  \begin{equation*}
    \boldsymbol{v}
      = \sum_{\boldsymbol{b} \in B}
	  \alpha_{\boldsymbol{b}} \boldsymbol{b}
  \end{equation*}
  a los coeficientes \(\alpha_{\boldsymbol{b}}\)
  se les llama \emph{componentes} de \(\boldsymbol{v}\)%
    \index{espacio vectorial!componentes (de un vector)|textbfhy}
  (en la base \(B\)).
  \begin{definition}
    \index{espacio vectorial!dimension@dimensión|textbfhy}
    Al número de vectores en una base de \(V\)
    se le llama la \emph{dimensión} de \(V\),
    anotada \(\dim V\).
    Un espacio vectorial abarcado por un conjunto finito de vectores
    se dice de \emph{dimensión finita},
    en caso contrario es de \emph{dimensión infinita}.
    Al espacio vectorial \(\{\boldsymbol{0}\}\)
    se le asigna dimensión cero.
    Se anota \([V : F]\)
    para la dimensión de \(V\) sobre el campo \(F\).
  \end{definition}
  En el caso de espacios vectoriales de dimensión finita
  es simple demostrar que todas las bases
  tienen la misma cardinalidad,
  con lo que nuestra definición de dimensión tiene sentido.
  \begin{theorem}
    \label{theo:espacio-vectorial-li}
    Si \(V\) es un espacio vectorial
    con base
      \(B = \{\boldsymbol{b}_1, \boldsymbol{b}_2,
	       \dotsc, \boldsymbol{b}_n\}\),
    y \(A = \{\boldsymbol{a}_1, \boldsymbol{a}_2,
	       \dotsc, \boldsymbol{a}_r\}\)
    es un conjunto linealmente independiente de vectores en \(V\),
    entonces \(r \le n\).
  \end{theorem}
  \begin{proof}
    Como \(B\) abarca \(V\),
    \(B \cup \{\boldsymbol{a}_1\}\) también abarca \(V\).
    Como \(\boldsymbol{a}_1 \ne \boldsymbol{0}\)
    (\(A\) es linealmente independiente),
    podemos expresar \(\boldsymbol{a}_1\)
    como combinación lineal de los \(B\),
    y en ella algún \(\boldsymbol{b}_t\)
    tendrá coeficiente diferente de 0.
    Ese \(\boldsymbol{b}_t\) puede expresarse en términos
    de \(B_1 = \{\boldsymbol{a}_1, \boldsymbol{b}_1,
		   \boldsymbol{b}_2,
		   \dotsc, \boldsymbol{b}_{t - 1},
		   \boldsymbol{b}_{t + 1},
		 \dotsc, \boldsymbol{b}_n\}\).
    Como todo \(\boldsymbol{v} \in V\)
    puede escribirse como combinación lineal
    de los \(B\),
    también puede escribirse como combinación lineal de los \(B_1\)
    (substituyendo la combinación
     de \(B_1\) que da \(\boldsymbol{b}_t\)
     en la combinación lineal para \(\boldsymbol{v}\)
     se obtiene una nueva combinación lineal).
    Este proceso puede repetirse
    intercambiando un \(A\) por uno de los \(B\),
    manteniendo siempre \(B_k\) como base,
    finalmente llegando
    a \(B_r = \{\boldsymbol{a}_1, \dotsc, \boldsymbol{a}_r,
		\boldsymbol{b}_{m_1}, \boldsymbol{b}_{m_2},
		\dotsc, \boldsymbol{b}_{m_s}\}\)
    (posiblemente no queden \(\boldsymbol{b}_{m_k}\) en \(B_r\)).
    No pueden quedar \(A\) si se acaban los \(B\),
    ya que si fuera así un \(\boldsymbol{a}_i\) sobrante
    no podría representarse como combinación lineal
    de los \(B\),
    y \(B\) no sería una base.
    Tenemos \(A \subseteq B_r\),
    y claramente
      \(\lvert A \rvert \le \lvert B_r \rvert = \lvert B \rvert\).
  \end{proof}
  Esto justifica la definición de la dimensión
  en el caso de espacios vectoriales de dimensión finita:
  \begin{corollary}[Teorema de dimensión de espacios vectoriales]
    \label{cor:espacio-vectorial-dimension}
    Si \(A\) y \(B\) son bases
    de un espacio vectorial de dimensión finita,
    entonces \(\lvert A \rvert = \lvert B \rvert\).
  \end{corollary}
  \begin{proof}
    La base \(A\) es linealmente independiente,
    con lo que por el teorema~\ref{theo:espacio-vectorial-li}
    es \(\lvert A \rvert \le \lvert B \rvert\).
    Por el mismo argumento,
    intercambiando los roles de \(A\) y \(B\),
    \(\lvert B \rvert \le \lvert A \rvert\),
    con lo que \(\lvert A \rvert = \lvert B \rvert\).
  \end{proof}
  Esto nos lleva a:
  \begin{theorem}
    \label{theo:espacio-vectorial-isomorfos}
    Todos los espacios vectoriales de la misma dimensión finita
    sobre \(F\) son isomorfos.
  \end{theorem}
  \begin{proof}
    Sean \(U\) y \(V\) espacios vectoriales
    de la misma dimensión finita,
    con bases \(\{\boldsymbol{a}_k\}_{1 \le k \le n}\)
    y \(\{\boldsymbol{b}_k\}_{1 \le k \le n}\),
    respectivamente.
    Podemos representar todos los vectores \(\boldsymbol{u} \in U\)
    y \(\boldsymbol{v} \in V\)
    mediante:
    \begin{equation*}
      \boldsymbol{u}
	= \sum_{1 \le k \le n} a_k \boldsymbol{a}_k \hspace{2em}
      \boldsymbol{v}
	= \sum_{1 \le k \le n} b_k \boldsymbol{b}_k
    \end{equation*}
    Definimos la biyección \(f \colon U \rightarrow V\)
    mediante:
    \begin{equation*}
      f \colon \sum_{1 \le k \le n} a_k \boldsymbol{a}_k
	\mapsto \sum_{1 \le k \le n} a_k \boldsymbol{b}_k
    \end{equation*}
    Demostrar que la suma vectorial
    y el producto escalar se preservan
    es rutinario.
  \end{proof}
  \noindent
  En vista de la demostración
  del teorema~\ref{theo:espacio-vectorial-isomorfos},
  en un espacio vectorial de dimensión finita
  basta elegir una base,
  cada vector puede representarse
  mediante la secuencia de los coeficientes en \(F\).
  La suma vectorial es sumar componente a componente,
  el producto escalar es multiplicar cada componente por el escalar.
  Es por esta representación que a secuencias de largo fijo
  les llaman vectores.

  Lo anterior solo cubre una peueña parte
  de la extensa teoría relacionada con operaciones lineales.
  Para profundizar en ella recomendamos el texto de Treil~%
    \cite{treil14:_linear_algeb_done_wrong}.

%%% Local Variables:
%%% mode: latex
%%% TeX-master: "clases"
%%% End:


\section{Estructura de los campos finitos}
\label{sec:estructura-campos-finitos}
\index{campo (algebra)@campo (álgebra)!finito!estructura|textbfhy}

  Profundizaremos nuestro estudio de los campos finitos,
  apoyados ahora en lo que sabemos de espacios vectoriales.%
    \index{espacio vectorial}
  \begin{lemma}
    \label{lem:unique-irreductible-root}
    Sea \(K\) el subcampo primo de \(F\),%
      \index{campo (algebra)@campo (álgebra)!subcampo!primo}
    y sea \(\alpha \in F\) el cero de un polinomio en \(K[x]\).
    Entonces hay un polinomio mónico%
      \index{polinomio!monico@mónico}
    único de grado mínimo en \(K[x]\)
    con \(\alpha\) de cero.
  \end{lemma}
  \begin{proof}
    Es simple
    demostrar que \(I = \{f \in K[x] \colon f(\alpha) = 0\}\)
    es un ideal de \(K[x]\).
    Como \(K[x]\) es un dominio de ideal principal,%
      \index{dominio de ideal principal}
    \(I = (h)\) para algún \(h \in K[x]\),
    donde \(h\) es mónico
    y de mínimo grado entre los elementos de \(I\),
    y es único con estas características.
  \end{proof}
  Al polinomio \(h\)
  de la demostración del lema~\ref{lem:unique-irreductible-root}
  se le llama el \emph{polinomio mínimo de \(\alpha\) sobre \(K\)}.%
    \index{polinomio!minimo (de un elemento)@mínimo (de un elemento)}
  Un elemento \(\alpha\) que es cero de un polinomio en \(K[x]\)
  se dice \emph{algebraico sobre \(K\)}.
    \index{anillo!elemento algebraico}
  \begin{theorem}
    \label{theo:minimal-polynomial-divides}
    Sea \(\alpha \in F\) el cero de un polinomio en \(K[x]\)
    y sea \(g\) el polinomio mínimo de \(\alpha\).
    Entonces
    \begin{enumerate}[label = (\roman*), ref = (\roman*)]
    \item
      \(g\) es irreductible en \(K[x]\)%
	\index{anillo!elemento irreductible}
    \item
      \(f(\alpha) = 0\)
      si y solo si \(g \mid f\)
    \end{enumerate}
  \end{theorem}
  \begin{proof}
    Por turno.
    \begin{enumerate}[label = (\roman*), ref = (\roman*)]
    \item
      Como \(g\) tiene un cero en \(F\),
      \(\deg(g) \ge 1\).
      Demostramos que \(g\) es irreductible por contradicción.
      Supongamos que podemos expresar \(g = h_1 h_2\) en \(K[x]\)
      con \(1 \le \deg(h_i) < \deg(g)\)
      para \(i = 1, 2\).
      Entonces \(g(\alpha) = h_1(\alpha) h_2(\alpha) = 0\),
      por lo que \(h_1(\alpha) = 0\) o \(h_2(\alpha) = 0\);
      o sea uno de los polinomios está en el ideal \(I\)%
	\index{anillo!ideal}
      de la demostración
	del lema~\ref{lem:unique-irreductible-root}.
      Al ser \(K[x]\) un dominio de ideal principal,%
	\index{dominio de ideal principal}
      ese ideal es el conjunto de los múltiplos de \(g\),
      con lo que \(g \mid h_1\) o \(g \mid h_2\),
      lo que es imposible porque sus grados son menores al de \(g\).
    \item
      Esto sigue de la definición de \(g\)
      como generador del ideal \(I\) del mencionado lema.
    \qedhere
    \end{enumerate}
  \end{proof}
  Antes de continuar,
  demostraremos que todos los campos finitos de orden \(q\)
  son isomorfos
  (ya sabemos que \(q = p^n\) para un primo \(p\)).
  De partida:
  \begin{theorem}[Polinomio universal]
    \index{polinomio!universal|textbfhy}
    \label{theo:universal-polynomial}
    Sea \(F\) un campo finito de orden \(q\).
    Entonces todos los elementos \(a \in F\)
    cumplen la ecuación:
    \begin{equation*}
      x^q - x = 0
    \end{equation*}
  \end{theorem}
  \begin{proof}
    Por el teorema de Lagrange,%
      \index{Lagrange, teorema de}
    si \(a \ne 0\) el orden multiplicativo de \(a\)
    divide a \(q - 1\);
    en particular:
    \begin{equation*}
      a^{q - 1} - 1
	= 0
    \end{equation*}
    Si multiplicamos esta ecuación por \(a\),
    resulta que para todo \(a \in F\):
    \begin{equation*}
      a^q - a
	= 0
      \qedhere
    \end{equation*}
  \end{proof}
  Nótese que el teorema~\ref{theo:universal-polynomial}
  dice que los elementos del campo finito \(F\)
  de orden \(q\)
  son todas los ceros del \emph{polinomio universal} \(x^q - x\).
  En particular,
  los polinomios mínimos de los elementos de \(F\)
  dividen a \(x^q - x\).
  \begin{theorem}
    \label{theo:finite-field-unique}
    Sean \(F\) y \(F'\) campos finitos de orden \(q\).
    Entonces \(F \cong F'\).
  \end{theorem}
  \begin{proof}
    Sabemos que si \(\lvert F \rvert = \lvert F' \rvert = p^n\),
    la característica de ambos campos es \(p\),
    en particular,
    el campo primo de ambos es isomorfo a \(\mathbb{Z}_p\).

    Sabemos que \(F^\times\) es cíclico
    (teorema~\ref{theo:F*-ciclico}),
    elijamos un generador \(\pi\) de \(F^\times\),
    y sea \(m(x)\) el polinomio mínimo de \(\pi\),
    que por la observación anterior
    (teorema~\ref{theo:minimal-polynomial-divides})
    con \(q = p^n\) en \(F\) cumple:
    \begin{equation*}
      m(x) \mid x^{q - 1} - 1
    \end{equation*}
    Consideremos el polinomio \(m(x)\) en \(F'\) ahora,
    donde también divide a \(x^q - x\)
    (los coeficientes y las operaciones al dividir
     son estrictamente en el subcampo primo,
     serán las mismas en \(F\) y \(F'\)).
    Acá podemos escribir:
    \begin{equation*}
      x^{q - 1} - 1
	= \prod_{a' \in F'^\times} (x - a')
    \end{equation*}
    por lo que \(m(x)\) se factoriza completamente en \(F'\):
    \begin{equation*}
      m(x)
	= (x - a_1') (x - a_2') \dotsm (x - a_d')
    \end{equation*}
    Elijamos un cero cualquiera de \(m(x)\) en \(F'\),
    digamos \(\pi' = a_1'\).
    Observamos primeramente que \(\pi'\) genera \(F'^\times\),
    ya que si su orden fuera \(d < q - 1\),
    cumpliría:
    \begin{equation*}
      x^d - 1
	= 0
    \end{equation*}
    Pero como \(m(x)\) es un polinomio irreductible,%
      \index{polinomio!irreductible}
    debe ser su polinomio mínimo,%
      \index{polinomio!minimo (de un elemento)@mínimo (de un elemento)}
    y en \(F'\):
    \begin{equation*}
      m(x) \mid x^d - 1
    \end{equation*}
    Volviendo a \(F\),
    esto significa que \(\pi\) también satisface esta ecuación,
    y tiene orden \(d < q - 1\)
    (o sea,
     no sería generador de \(F^\times\)).

    Hay un isomorfismo de grupo obvio%
      \index{grupo!isomorfo}
    entre \((F^\times, \cdot)\) y \((F'^\times, \cdot)\):
    \begin{equation*}
      \Theta(\pi^k)
	= \pi'^k
    \end{equation*}
    Podemos extenderlo a una biyección entre \(F\) y \(F'\)
    definiendo:
    \begin{equation*}
      \Theta(0)
	= 0
    \end{equation*}
    Resta demostrar que \(\Theta\) es un isomorfismo para la suma.
    Sean \(a, b \in F\),
    debemos mostrar que:
    \begin{equation*}
      \Theta(a + b)
	= \Theta(a) + \Theta(b)
    \end{equation*}
    Si \(a = 0\) o \(b = 0\),
    el resultado es inmediato,
    así que en lo que sigue \(a \ne 0\) y \(b \ne 0\).
    Debemos considerar los dos casos \(a + b = 0\)
    y \(a + b \ne 0\).
    Veamos primero el segundo,
    más general.
    Sean:
    \begin{equation*}
      a = \pi^i \quad b = \pi^j \quad a + b = \pi^k
    \end{equation*}
    Entonces en \(F\):
    \begin{equation*}
      \pi^i + \pi^j
	= \pi^k
    \end{equation*}
    Vale decir,
    \(\pi\) satisface la ecuación:
    \begin{equation*}
      x^i + x^j - x^k
	= 0
    \end{equation*}
    Por el teorema~\ref{theo:minimal-polynomial-divides}
    en \(F\):
    \begin{equation*}
      m(x) \mid x^i + x^j - x^k
    \end{equation*}
    Pero en tal caso esto también se cumple en \(F'\):
    \begin{equation*}
      \pi'^i + \pi'^j
	= \pi'^k
    \end{equation*}
    Esto es precisamente:
    \begin{equation*}
      \Theta(a + b)
	= \Theta(a) + \Theta(b)
    \end{equation*}

    Resta el caso \(a + b = 0\).
    Si la característica de los campos es \(2\),
    esto significa \(a = b\),
    y en consecuencia como \(\Theta(a) = \Theta(b)\)
    es:
    \begin{equation*}
      \Theta(a + b)
	= \Theta(0)
	= 0
	= \Theta(a) + \Theta(b)
    \end{equation*}
    Si la característica de \(F\) es impar,
    notamos que \(-1\) es el único elemento de orden \(2\),
    ya que:
    \begin{equation*}
      x^2 - 1
	= (x - 1) (x + 1)
    \end{equation*}
    no puede tener más de dos ceros.
    En efecto,
    como \(F\) tiene \(q\) elementos
    (y así \(F^\times\) tiene \(q - 1\) elementos),
    debe ser:
    \begin{equation*}
      -1 = \pi^{\frac{q - 1}{2}}
    \end{equation*}
    dado que el elemento al lado derecho tiene el orden correcto.
    Escribamos \(a = \pi^i\) y \(b = \pi^j\),
    donde podemos suponer sin pérdida de generalidad que \(i > j\),
    con lo que:
    \begin{align*}
      &\pi^i + \pi^j
	= 0 \\
      &\pi^i
	= - \pi^j \\
      &\pi^{i - j}
	= -1 \\
      &i - j
	= \frac{q - 1}{2} \\
      &\pi'^{(i - j)}
	= -1 \\
    \intertext{De acá, aplicando lo anterior en reversa en \(F'\):}
      &\pi'^i + \pi'^j
	= 0
    \end{align*}
    Vale decir:
    \begin{equation*}
      \Theta(a + b)
	= \Theta(a) + \Theta(b)
    \end{equation*}
    En resumen,
    la biyección \(\Theta\) preserva suma y multiplicación,
    es un isomorfismo entre los campos.
  \end{proof}
  Al campo finito de orden \(q\) se le anota \(\mathbb{F}_q\)
  (en la literatura más antigua se suele encontrar la notación
   \(\mathrm{GF}(q)\),
   por la abreviatura de \emph{campo de Galois}%
     \index{Galois, campo de|see{campo (álgebra)!finito}}%
     \index{Galois, Evariste}
   en honor a quien comenzó su estudio).
  Resulta curioso que todo polinomio irreductible de grado \(n\)
  da el mismo campo.
  Resta demostrar que tales campos existen para todo primo \(p\)
  y todo \(n\).

  Vimos
  (teorema~\ref{theo:a-invertible})
  que \(\mathbb{Z}_m = \mathbb{Z} / (m)\)
  es un campo solo cuando \(m\) es primo.
  Hay notables similitudes entre el anillo \(\mathbb{Z}\)
  y los anillos de polinomios \(K[x]\) sobre un campo \(K\)%
    \index{anillo!polinomios}
  -- particularmente si \(K\) es finito.
  Ya vimos un ejemplo de esto:
  En los números enteros
  hay infinitos primos,
  y pueden expresarse en forma esencialmente única
  como producto de primos
  (y un signo,
   vale decir multiplicar por una unidad),
  el teorema fundamental de la aritmética.%
    \index{teorema fundamental de la aritmetica@teorema fundamental de la aritmética}
  Para polinomios
  tenemos el teorema~\ref{theo:fundamental-arithmetic-polynomials}.%
    \index{polinomio!teorema fundamental de la aritmetica@teorema fundamental de la aritmética}
  Los primos en \(\mathbb{Z}\)
  corresponden a los polinomios mónicos irreductibles en \(K[x]\).
  Igual que la relación de congruencia entre enteros,%
    \index{polinomio!congruencia}
  podemos definirla para polinomios
  \(f(x), g(x), m(x) \in K[x]\):
  \begin{equation*}
    f(x)
      \equiv g(x) \pmod{m(x)}
  \end{equation*}
  siempre que para algún \(q(x) \in K[x]\):
  \begin{equation*}
    g(x) - f(x)
      = m(x) q(x)
  \end{equation*}
  Es rutinario verificar que es equivalencia en \(K[x]\),
  y que las clases de equivalencia
  forman un anillo \(K[x] / (m(x))\),
  el \emph{anillo de polinomios sobre \(K\) módulo \(m(x)\)}.
  Este anillo contiene el campo \(K\)
  como los polinomios constantes.
  \begin{theorem}
    \label{theo:polynomial-ring-modulo-m=field}
    El anillo cociente \(K[x] / (m(x))\)
    es un campo si y solo si \(m(x)\) es irreductible.
  \end{theorem}
  \begin{proof}
    Demostramos implicancia en ambas direcciones.
    Para el directo,
    sea \(f(x) \in K[x]\) tal que \(m(x) \centernot\mid f(x)\).
    Sabemos
    (sección~\ref{sec:dominios-euclidianos})%
      \index{dominio euclidiano}
    que los polinomios sobre un campo son un dominio euclidiano,
    es aplicable la identidad de Bézout
    y tenemos un inverso de \(f\).%
      \index{Bezout, identidad de@Bézout, identidad de}

    Para el recíproco,
    usamos contradicción.%
      \index{demostracion@demostración!contradiccion@contradicción}
    Supongamos que \(m(x)\) no es irreductible,
    vale decir:
    \begin{equation*}
      m(x)
	= a(x) \cdot b(x)
    \end{equation*}
    donde \(a(x)\) y \(b(x)\) no son constantes.
    Las clases de equivalencia correspondientes no son cero,
    pero:
    \begin{equation*}
      [a(z)] \cdot [b(x)]
	= [a(x) \cdot b(x)]
	= [m(x)]
	= 0
    \end{equation*}
    Al haber divisores de cero,
    no es campo.
  \end{proof}
  Considerando \(K[x] / (m(x))\) como espacio vectorial,%
    \index{espacio vectorial}
  es una extensión de \(K\):%
    \index{campo (algebra)@campo (álgebra)!extension@extensión}
  \begin{definition}
    \label{def:extension-degree}
    A la dimensión de \(F\) como espacio vectorial sobre \(K\)
    la anotamos \([F : K]\),
    y la llamamos el \emph{grado de la extensión}.%
      \index{campo (algebra)@campo (álgebra)!extension@extensión!grado|textbfhy}
  \end{definition}
  \begin{lemma}
    \label{lem:irreducible-equivalence-classes}
    Si \(m(x)\) es un polinomio irreductible de grado \(d\),
    entonces las clases de equivalencia:
    \begin{equation*}
      [1], [x], \dotsc, [x^{d - 1}]
    \end{equation*}
    forman una base para \(K[x] / (m(x))\).
    En particular:
    \begin{equation*}
      \dim_K K[x] / (m(x)) = \deg m(x)
    \end{equation*}
  \end{lemma}
  \begin{proof}
    Demostramos por contradicción
    que las clases son linealmente independientes.%
      \index{espacio vectorial!independencia lineal}
    Supongamos:
    \begin{equation*}
      c_0 [1] + c_1 [x] + \dotsb + c_{d - 1} [x^{d - 1}]
	= 0
    \end{equation*}
    Esto significa:
    \begin{equation*}
      m(x) \mid c_0 + c_1 x + \dotsb + c_{d - 1} x^{d - 1}
    \end{equation*}
    Esto solo es posible si todos los \(c_i = 0\),
    ya que el grado de \(m\) es \(d\).

    Para demostrar que las clases de equivalencia
    abarcan \(K[x] / (m(x))\),
    elijamos \(f(x) \in K[x]\) cualquiera.
    Podemos dividir:%
      \index{polinomio!algoritmo de division@algoritmo de división}
    \begin{equation*}
      f(x) = m(x) \cdot q(x) + r(x)
      \qquad
      \deg r(x) < \deg m(x)
    \end{equation*}
    Así:
    \begin{equation*}
      f(x) \equiv r(x) \pmod{m(x)}
    \end{equation*}
    O sea,
    si:
    \begin{equation*}
      r(x)
	= r_0 + r_1 x + \dotsb + r_{d - 1} x^{d - 1}
    \end{equation*}
    entonces:
    \begin{equation*}
      [f(x)]
	= [r(x)]
	= r_0 [1] + r_1 [x] + \dotsb + r_{d - 1} [x^{d - 1}]
      \qedhere
    \end{equation*}
  \end{proof}
  \begin{corollary}
    \label{cor:fielp-p^n}
    Si \(m(x) \in \mathbb{F}_q [x]\) es irreductible
    de grado \(n\),
    entonces \(\mathbb{F}_q[x] / (m(x))\) es de orden \(q^n\).
  \end{corollary}
  \begin{proof}
    Inmediato del teorema~\ref{theo:polynomial-ring-modulo-m=field}:
    \(\mathbb{F}_q[x] / (m(x))\)
    es un espacio vectorial de dimensión \(n\)
    sobre \(\mathbb{F}_q\),
    con lo que contiene \(q^n\) elementos.
  \end{proof}
  En lo anterior construimos un campo \(K[x] / (m(x))\)
  partiendo de un campo \(K\) y un polinomio irreductible sobre él.
  Vimos también que todos los campos finitos del mismo orden
  son isomorfos.
  Ahora la construcción inversa,
  buscando la relación entre un campo \(F\) y sus subcampos.
  \begin{definition}
    \label{def:extension-field}
    Sean \(\alpha_1, \alpha_2, \dotsc, \alpha_n \in F\),
    y sea \(K\) un subcampo de \(F\).
    Al mínimo subcampo de \(F\)
    que contiene \(\alpha_1, \alpha_2, \dotsc, \alpha_n\) y \(K\)
    se anota \(K(\alpha_1, \alpha_2, \dotsc, \alpha_n)\).
    A tales campos se les llama \emph{extensiones} de \(K\).%
      \index{campo (algebra)@campo (álgebra)!extension@extensión}
    Si \(F = K(\alpha)\),
    se dice que \(F\) es una \emph{extensión simple} de \(K\).%
      \index{campo (algebra)@campo (álgebra)!extension@extensión!simple}
  \end{definition}
  Nótese la similitud entre la definición~\ref{def:extension-field}
  y la noción de anillos cuadráticos%
    \index{anillo!cuadratico@cuadrático}
  vistos en la sección~\ref{sec:anillos-cuadraticos}.
  Allá usamos la notación \(\mathbb{Z}[\sqrt{2}]\) para el anillo,
  acá hablamos del campo \(\mathbb{Q}(\sqrt{2})\).

  Por la discusión anterior \(K(\alpha_1, \dotsc, \alpha_n)\)
  es un espacio vectorial sobre \(K\).%
    \index{espacio vectorial}
  \begin{theorem}
    \label{theo:extension-isomorphic-minimal-polynomial}
    Sea \(F\) una extensión del campo \(K\),
    y sea \(\alpha \in F\) el cero de un polinomio en \(K[x]\),
    con polinomio mínimo \(g\).%
      \index{anillo!polinomio minimo@polinomio mínimo}
    Entonces:
    \begin{enumerate}[label = (\roman*), ref = (\roman*)]
    \item
      \label{en:K(alpha)-g-1}
      \(K(\alpha)\) es isomorfo a \(K[x] / (g(x))\)
    \item
      \label{en:K(alpha)-g-2}
      \([K(\alpha) : K] = \deg(g)\)
      y \(\{1, \alpha, \alpha^2, \dotsc, \alpha^{n - 1}\}\)
      es una base de \(K(\alpha)\) sobre \(K\)
    \item
      \label{en:K(alpha)-g-3}
      Si \(\beta \in K(\alpha)\)
      es el cero de un polinomio en \(K[x]\),
      el grado del polinomio mínimo de \(\beta\)
      divide al grado de \(g\)
    \end{enumerate}
  \end{theorem}
  \begin{proof}
    Para el punto~\ref{en:K(alpha)-g-1},
    por el lema~\ref{lem:irreducible-equivalence-classes}
    la clase \([x]\) de \(K[x] / (g(x))\)
    satisface la ecuación:
    \begin{equation*}
      g(x) = 0
    \end{equation*}
    En consecuencia,
    \(K(\alpha) \cong K[x] / (g(x))\)
    ya que son campos finitos del mismo orden.
    El punto~\ref{en:K(alpha)-g-2} es inmediato de lo anterior.

    Para~\ref{en:K(alpha)-g-3},
    que \(K(\alpha)\) es un espacio vectorial sobre \(K(\beta)\),
    con lo que el grado
    del polinomio mínimo de \(\alpha\) sobre \(K(\beta)\)
    da la condición de divisibilidad prometida.
  \end{proof}
  De lo anterior tenemos directamente:
  \begin{corollary}
    \label{cor:[H:F]=[H:G][G:F]}
    Sean \(F \le G \le H\) campos finitos.
    Entonces:
    \begin{equation*}
      [H : F]
	= [H : G] \cdot [G : F]
    \end{equation*}
  \end{corollary}

  Lo anterior muestra un cero de cada polinomio irreductible,
  pero nos interesan todas los ceros.
  Al efecto,
  definimos:
  \begin{definition}
    Sea \(f \in K[x]\) un polinomio de grado positivo,
    y \(F\) una extensión de \(K\).
    Decimos que \(f\) \emph{se divide} en \(F\)
    si hay \(a \in K\) y \(\alpha_i \in F\) para \(1 \le i \le n\)
    tales que podemos escribir:
    \begin{equation*}
      f(x)
	= a (x - \alpha_1) (x - \alpha_2) \dotsm (x - \alpha_n)
    \end{equation*}
    El campo \(F\) se llama \emph{campo divisor}
    de \(f\) sobre \(K\)
    si \(f\) se divide en \(F\)
    y además es \(F = K(\alpha_1, \alpha_2, \dotsc, \alpha_n)\).
  \end{definition}

  El siguiente resultado es una pieza angular
  de la teoría de campos.
  \begin{theorem}[Kroneker]
    \index{Kronecker, teorema de}
    \index{Kronecker, Leopold}
    \label{theo:Kroneker}
    Para todo polinomio irreductible \(f(x)\) sobre el campo \(F\)
    hay una extensión en la cual \(f(x)\) tiene un cero.
  \end{theorem}
  \begin{proof}
    Sea \(f(x) = a_0 + a_1 x + \dotsb + a_n x^n \in F[x]\)
      irreductible.
    Consideremos el elemento \([x]\) en el campo \(F[x] / (f(x))\).
    Entonces en \(F[x] / (f(x))\):
    \begin{align*}
      f([x])
	&= [a_0] + [a_1] [x] + \dotsb + [a_n] [x^n] \\
	&= [a_0 + a_1 x + \dotsb + a_n x^n] \\
	&= 0
    \end{align*}
    Así \(F[x] / (f(x))\) es campo divisor,
    y en el \([x]\) es cero.
  \end{proof}
  \noindent
  Esto parece ser solo jugar con la notación,
  pero es más profundo:
  Hay que distinguir entre \(x\)
  (el símbolo usado para describir polinomios formales)
  y \([x]\)
  (la clase de equivalencia del polinomio \(x\) módulo \(f(x)\)
   sobre \(F\)).
  Además usamos las definiciones y propiedades
  de las operaciones entre clases de congruencia.
  También vemos
  que al aplicar repetidas veces el teorema de Kroneker
  obtenemos finalmente el campo divisor de cualquier polinomio.

  \begin{theorem}
    \index{polinomio!campo divisor}
    \label{theo:splitting-field-roots}
    Sea \(F\) un campo,
    \(f\) un polinomio irreductible sobre el campo \(K\)
    con ceros \(\alpha, \beta \in F\).
    Entonces \(K(\alpha) \cong K(\beta)\),
    con un isomorfismo que mantiene fijos los elementos de \(K\)%
      \index{campo (algebra)@campo (álgebra)!isomorfo}
    e intercambia los ceros \(\alpha\) y \(\beta\).
  \end{theorem}
  \begin{proof}
    Por el teorema de Kroneker,
    ambos son isomorfos a \(K[x] / (f(x))\),
    dado que el irreductible \(f\)
    es el polinomio mínimo de \(\alpha\) y \(\beta\).
    El isomorfismo claramente mantiene fijos los elementos de \(K\),
    \(\beta\) se expresa
    como una combinación lineal en \(K(\alpha)\)
    y similarmente \(\alpha\) en \(K(\beta)\).
  \end{proof}

  Lo siguiente básicamente recoge resultados previos.
  \begin{theorem}[Existencia y unicidad del campo divisor]
    \label{theo:E!-splitting-field}
    Todo polinomio tiene campo divisor único:
    \begin{enumerate}[label = (\roman*), ref = (\roman*)]
    \item
      Si \(K\) es un campo
      y \(f\) un polinomio de grado positivo en \(K[x]\),
      entonces existe un campo divisor de \(f\) sobre \(K\).
    \item
      Cualquier par de campos divisores de \(f\) sobre \(K\)
      son isomorfos bajo un isomorfismo
      que mantiene fijos los elementos de \(K\)
      y permuta ceros de \(f\).
    \end{enumerate}
  \end{theorem}
  Así podemos hablar
  de \emph{el} campo divisor de \(f\) sobre \(K\),
  que se obtiene
  adjuntando un número finito de elementos algebraicos a \(K\),
  y es una extensión finita de \(K\).

  \begin{theorem}[Existencia y unicidad de campos finitos]
    \label{theo:E!-FF}
    Para cada primo \(p\) y natural \(n\)
    hay un campo finito de orden \(p^n\).
    Todo campo finito de orden \(q = p^n\)
    es isomorfo al campo divisor
    de \(x^q - x\) sobre \(\mathbb{Z}_p\).
  \end{theorem}
  \begin{proof}
    Sea \(F\) el campo divisor de \(f(x) = x^q - x\)
    en \(\mathbb{Z}_p[x]\).
% Fixme: Traducción de "Splitting field"
    Como \(f'(x) = q x^{q - 1} - 1 = -1\) sobre \(\mathbb{Z}_p\),
    por el lema~\ref{lem:repeated-roots}
    \(f(x)\) no tiene factores repetidos,
    con lo que \(f(x)\) tiene \(q\) ceros en \(F\),
    exactamente los \(q\) elementos de \(F\).

    Por el teorema~%
     \ref{theo:extension-isomorphic-minimal-polynomial},
    el campo divisor es único.
  \end{proof}
  El \emph{polinomio universal} \(U_n (x) = x^{p^n} - x\)%
    \index{anillo!polinomio universal}
  tiene como ceros todos los elementos de \(\mathbb{F}_{p^n}\).
  Resulta que \(U_m (x) \mid U_n (x)\) si y solo si \(m \mid n\),
  pero curiosamente es más fácil demostrar
  algo bastante más general:
  \begin{theorem}
    \label{theo:U-gcd}
    Sobre\/ \(\mathbb{F}_p\):
    \begin{equation*}
      \gcd(U_m (x), U_n (x))
	= U_{\gcd(m, n)} (x)
    \end{equation*}
  \end{theorem}
  \begin{proof}
    Si \(m = n\) no hay nada que demostrar.
    Usamos inducción fuerte sobre \(n\) para \(m < n\).%
      \index{demostracion@demostración!induccion@inducción}
    \begin{description}
    \item[Base:]
      Cuando \(n = 1\),
      no hay nada que demostrar.
    \item[Inducción:]
      Sea \(r = n - m\)
      y consideremos:
      \begin{equation*}
	\left( U_m (x) \right)^p
	  = x^{p^{m + 1}} - x^p
      \end{equation*}
      ya que al aplicar el teorema del binomio%
	\index{binomio, teorema del}
      los términos intermedios
      se anulan por ser divisibles por \(p\).
      Aplicando lo anterior \(r\) veces resulta:
      \begin{align*}
	\left( U_m (x) \right)^{p^r}
	  &= x^{p^{m + n - m}} - x^{p^r} \\
	  & = x^{p^n} - x^{p^r} \\
	  &= U_n (x) - U_r (x)
      \end{align*}
      de lo que obtenemos:
      \begin{equation*}
	\gcd(U_m (x), U_n (x))
	  = \gcd(U_r (x), U_m (x))
      \end{equation*}
      Por inducción:
      \begin{align*}
	\gcd(U_r (x), U_m (x))
	  &= U_{\gcd(r, m)} (x) \\
	  &= U_{\gcd(m, n)} (x)
      \end{align*}
      ya que \(\gcd(r, m) = \gcd(n - m, m) = \gcd(m, n)\).
    \end{description}
    Por inducción
    lo prometido vale para todo \(m, n \in \mathbb{N}\).
  \end{proof}
  Así:
  \begin{corollary}
    \label{cor:U-divides}
    Sobre\/ \(\mathbb{F}_p\) es
    \(U_m (x) \mid U_n (x)\) si y solo si \(m \mid n\).
  \end{corollary}
  \begin{proof}
    Recurrimos a una cadena de equivalencias.
    Es claro que \(U_m (x) \mid U_n (x)\)
    si y solo si \(\gcd(U_m (x), U_n (x)) = U_m (x)\).
    Por el teorema~\ref{theo:U-gcd}
    esto es si y solo si \(\gcd(m, n) = m\),
    que es si y solo si \(m \mid n\).
  \end{proof}

  \begin{theorem}
    \label{theo:finite-field-extension=simple-extension}
    Sea \(F_q\) un campo finito
    y \(F_r\) una extensión finita de \(F_q\).
    Entonces
    \begin{enumerate}[label = (\roman*), ref = (\roman*)]
    \item
      \label{en:tffes-1}
      \(F_r\) es una extensión simple de \(F_q\),
      vale decir,
      hay \(\beta \in F_r\) tal que \(F_r = F_q(\beta)\)
    \item
      \label{en:tffes-2}
      Cualquier elemento primitivo de \(F_r\)
      sirve como elemento definidor \(\beta\)
    \end{enumerate}
  \end{theorem}
  \begin{proof}
    Para~\ref{en:tffes-1},
    sea \(\alpha\) un elemento primitivo de \(F_r\),
    con lo que \(F_q(\alpha) \subseteq F_r\).
    Por otro lado,
    \(F_q(\alpha)\) contiene a \(0\)
    y todas las potencias de \(\alpha\),
    que son los elementos de \(F_r^\times\);
    con lo que \(F_r \subseteq F_q(\alpha)\).
    En consecuencia \(F_r = F_q(\alpha)\).
    El punto~\ref{en:tffes-2} es inmediato de lo anterior.
  \end{proof}
  Así tenemos
  \begin{corollary}
    \label{cor:E-irreducible-degrees}
    Para cada primo \(p\)
    hay polinomios irreductibles
    de todo grado \(n \ge 1\) sobre \(\mathbb{Z}_p\).
  \end{corollary}
  \begin{proof}
    Por el teorema~%
      \ref{theo:extension-isomorphic-minimal-polynomial}
    toda extensión de \(\mathbb{Z}_p\) es isomorfa
    a algún \(\mathbb{Z}_p(\alpha) \cong \mathbb{Z}_p[x] / (g(x))\)
    donde \(g(x)\) es el polinomio mínimo de \(\alpha\),
    por el teorema~\ref{theo:minimal-polynomial-divides}
    el polinomio mínimo es irreductible.
    Por el teorema~\ref{theo:E!-FF}
    hay campos finitos de \(p^n\) elementos
    para todo primo \(p\) y natural \(n\).
    En consecuencia,
    hay polinomios irreductibles
    de todos los grados sobre \(\mathbb{Z}_p\).
  \end{proof}
  Podemos hacer más:
  \begin{theorem}
    \index{polinomio!irreductible!numero@número}
    \label{theo:number-irreducible-polynomials}
    Sea \(N_n\) el número de polinomios irreductibles de grado \(n\)
    sobre\/ \(\mathbb{F}_q\).
    Entonces:
    \begin{equation}
      \label{eq:N-irreducible-polynomials-degree-n}
      N_n
	= \frac{1}{n} \, \sum_{d \mid n} \mu(n / d)  \, q^d
    \end{equation}
  \end{theorem}
  \begin{proof}
    En \(\mathbb{F}_{q^n}\)
    cada elemento es cero de su polinomio mínimo.
    Tal polinomio mínimo de grado \(d\)
    es irreductible sobre \(\mathbb{F}_q\)
    y tiene \(d\) ceros distintos
    en \(\mathbb{F}_{q^n}\).
    Contabilizando los elementos de \(\mathbb{F}_{q^n}\)
    como los ceros de sus polinomios mínimos:
    \begin{equation*}
      \sum_{d \mid n} d N_d
	= q^n
    \end{equation*}
    Aplicando inversión de Möbius%
      \index{Mobius, inversion de@Möbius, inversión de}
    (teorema~\ref{theo:Moebius-inversion})
    obtenemos lo anunciado.
  \end{proof}
  Esto da otra demostración
  de que hay polinomios irreductibles de grado \(n\)
  sobre \(\mathbb{Z}_q\) para todo \(n \in \mathbb{N}\):
  Para \(n = 1\),
  todos los polinomios son irreductibles.
  Si \(n \ge 2\),
  en la suma~\eqref{eq:N-irreducible-polynomials-degree-n}
  el término \(q^n\) es mayor que la suma de los demás,
  ya que como \(\lvert \mu(x) \rvert \le 1\) podemos acotar:
  \begin{equation*}
    \left\lvert
      \sum_{\substack{
	      d \mid n \\
	      d < n
	   }} \mu(n / d) q^d
    \right\rvert
      \le \sum_{\substack{
		  d \mid n \\
		  d < n
	       }} q^d
      \le \sum_{0 \le d \le n - 1} q^d
      = \frac{q^n - 1}{q - 1}
      < q^n
  \end{equation*}
  Así la suma en~\eqref{eq:N-irreducible-polynomials-degree-n}
  nunca se anula si \(n \ge 2\).
  Uniendo este resultado con el caso \(n = 1\),
  \(N_n > 0\) para todo \(n \in \mathbb{N}\).
  En vista del corolario~\ref{cor:U-divides},
  podemos obtener todos los polinomios irreductibles
  sobre \(\mathbb{F}_p\)
  de grado hasta \(n\) como factores de \(U_n(x)\).

% codigo-deteccion-errores.tex
%
% Copyright (c) 2012-2014 Horst H. von Brand
% Derechos reservados. Vea COPYRIGHT para detalles

\section{Códigos de detección y corrección de errores}
\label{sec:codigos-errores}
\index{codigo@código}

  Consideremos \emph{mensajes} de \(m\)~bits de largo%
    \index{mensaje}
  que se transmiten por algún medio
  (podría ser simplemente que se almacenan y se recuperan luego).
  En este proceso pueden ocurrir errores,
  que interesa detectar o corregir.
  El tema fue estudiado inicialmente por Hamming~%
    \cite{hamming50:_error_detec_correc_codes}.%
    \index{Hamming, Richard}
  Para ello usamos \emph{palabras de código}
  de \(n\) bits de largo,%
    \index{codigo@código!palabra}
  donde obviamente \(n \ge m\),
  usando los bits adicionales para detectar o corregir errores,
  usando solo \(2^m\) de las \(2^n\) palabras posibles.
  Si se recibe una palabra errada
  (que no corresponde al código),
  una estrategia obvia es suponer que el código correcto
  es el más cercano al recibido,
  vale decir,
  el que difiere en menos bits del recibido.
  Al número de bits en que difieren dos palabras se les llama
  la \emph{distancia de Hamming}%
    \index{Hamming, distancia de|textbfhy}
  entre ellas.
  Por ejemplo,
  la distancia de Hamming entre \(10111011\) y \(10010100\)
  es \(5\).
  A la distancia de Hamming mínima entre dos palabras de un código
  se le conoce como \emph{distancia de Hamming del código}.%
    \index{codigo@código!Hamming, distancia de|textbfhy}
  Lo que interesa entonces es hallar códigos
  de distancia de Hamming máxima
  en forma uniforme
  (nos interesa que a cada código correcto
   le corresponda un número similar
   de palabras erróneas)
  y por el otro lado hallar formas eficientes
  de determinar si la palabra es correcta
  (solo detectar errores)%
     \index{codigo@código!deteccion de errores@detección de errores}
  o la más cercana a la recibida
  (para corregirlos).%
    \index{codigo@código!correccion de errores@corrección de errores}
  Para detectar \(d\) errores
  se requiere que la distancia de Hamming
  del código sea mayor a \(d + 1\)
  (así la palabra errada nunca coincide con una correcta,
   está al menos a un bit de distancia),
  para corregir \(d\) errores la distancia debe ser \(2 d + 1\)
  (la palabra errada estará a distancia a lo más \(d\)
   de la correcta,
   la siguiente más cercana estará
   a la distancia al menos \(d + 1\)).

\subsection{Códigos de Hamming}
\label{sec:codigos-Hamming}
\index{codigo@código!Hamming}

  Hamming~%
    \cite{hamming50:_error_detec_correc_codes}
  halló una manera de construir códigos de detección
  y corrección de errores de distancia~\(3\)
  (capaces de detectar \(2\) errores y corregir \(1\)).
  Suponiendo \(n = 2^w\)
  y contando los bits de \(1\) a \(2^w\),
  se usan los bits en las posiciones \(k = 1, 2, \dotsc, 2^{w - 1}\)
  como bits de paridad%
    \index{paridad, bits de}
  y los demás como bits de datos.
  La idea es calcular el bit en la posición \(2^k\)
  de forma que los bits en las posiciones
  escritas en binario que tienen \(1\) en la posición~\(k\)
  tengan un número par
  de unos,
  como muestra el cuadro~\ref{tab:paridad-Hamming}.
  \begin{table}[ht]
    \centering
    \begin{tabular}{|l@{\hspace{0.5em}}>{\(}r<{\)}|r*{7}{@{, }r}|}
      \hline
      \multicolumn{2}{|c|}{\textbf{Paridad\rule[-0.7ex]{0pt}{3ex}}} &
	\multicolumn{8}{c|}{\textbf{Posiciones}} \\
      \hline\rule[-0.7ex]{0pt}{3.2ex}%
      0 & 2^0 = \phantom{0}1 &
	 \phantom{0}1 &	 \phantom{0}3 &
	 5 &  7 &  9 & 11 & 13 & 15 \\
      1 & 2^1 = \phantom{0}2 &
	 2 &  3 &  6 &	7 & 10 & 11 & 14 & 15 \\
      2 & 2^2 = \phantom{0}4 &
	 4 &  5 &  6 &	7 & 12 & 13 & 14 & 15 \\
      3 & 2^3 = \phantom{0}8 &
	 8 &  9 & 10 & 11 & 12 & 13 & 14 & 15 \\
      \hline
    \end{tabular}
    \caption{Paridades para el código de Hamming $(15, 4)$}
    \label{tab:paridad-Hamming}
  \end{table}
  Para determinar el bit errado,
  lo que se hace es calcular los bits de paridad correspondientes
  a los datos recibidos,
  si son iguales a lo calculado,
  no se detectan errores;
  en caso de haber diferencias
  considerar los bits de paridad
  como un número binario da el bit errado.
  Por ejemplo,
  si \(w = 15\),
  el código tendrá \(4\)~bits de paridad
  (posiciones \(1\), \(2\), \(4\) y~\(8\))
  y \(11\)~bits de mensaje
  (en las posiciones \(3\), \(5\) a~\(7\) y \(9\) a~\(15\)).
  Se recibe \lstinline[language=C]!0x6A6A!,
  en binario \(0110\,1010\,0110\,1010\),
  revisamos los bits respectivos,
  lo que da \(3\) para \(0\),
  \(6\) para \(1\),
  \(6\) para \(2\)
  y \(4\) para \(3\).
  Esto corresponde a \(0001\),
  que significa que hay un error en la posición \(1\).
  El código correcto es \(0110\,1010\,0110\,1011\)
  o \lstinline[language=C]!0x6A6B!,
  y los bits de mensaje son \(110\,1010\,1100\),
  o \lstinline[language=C]!0x6AC!.
  El código de Hamming es óptimo,
  en el sentido que tiene la máxima distancia de Hamming
  para el número de bits dado.

\subsection{Verificación de redundancia cíclica}
\label{sec:CRC}
\index{codigo@código!redundancia ciclica@redundancia cíclica}
\index{CRC@\emph{CRC}|ver{código!redundancia cíclica}}

  Una manera de construir códigos de detección de errores
  simples de analizar matemáticamente
  fue descubierta por Peterson~%
    \cite{peterson61:_CRC}.
  La técnica tiene además la ventaja
  de poder implementarse en circuitos sencillos y rápidos,
  como veremos luego.
  Se les llama \emph{verificación de redundancia cíclica}
  (en inglés,
   \emph{\foreignlanguage{english}{Cyclic Redundancy Check}},
   o CRC)
  dado que se agregan bits de verificación
  (\emph{\foreignlanguage{english}{check}} en inglés)
  que son redundantes
  (no aportan información)
  según un código cíclico.

  Consideremos un mensaje binario de \(m\) bits,
  \(M = M_{m - 1} M_{m - 2} \dotso M_0\).
  Podemos considerarlo como un polinomio sobre \(\mathbb{Z}_2\):%
    \index{polinomio!campo finito}
  \begin{equation*}
    M(x)
      = M_{m - 1} x^{m - 1}
	 + M_{m - 2} x^{m - 2}
	 + \dotsb
	 + M_1 x
	 + M_0
  \end{equation*}
  Sea además un polinomio \(G(x)\) de grado \(n - 1\).
  Si calculamos:
  \begin{align*}
    r(x)
      &= M(x) x^n \bmod G(x) \\
    T(x)
      &= M(x) x^n - r(x)
  \end{align*}
  Es claro que \(T(x)\) es divisible por \(G(x)\).
  La estrategia es entonces tomar el mensaje,%
    \index{mensaje}
  añadirle \(n - 1\) bits cero al final,
  calcular el resto de esto al dividir por \(G(x)\)
  (un polinomio de grado \(n - 1\))
  y substituir los ceros añadidos por el resto
  (en \(\mathbb{Z}_2[x]\) suma y resta son la misma operación).
  A estos bits agregados los llamaremos \emph{bits de paridad}.
  El resultado es el polinomio \(T(x)\),
  que se trasmite.
  Al recibirlo,
  se calcula el resto de la división con \(G(x)\);
  si el resto es cero,
  el dato recibido es correcto.
  Al polinomio \(G(x)\) se le llama \emph{generador} del código.%
    \index{codigo@código!polinomio generador}%
    \index{polinomio!generador de un codigo@generador de un código|see{código!polinomio generador}}
  Esto es similar a la prueba de los nueves
  que discutimos en el capítulo~\ref{cha:estructura-Zm}.

  Si se transmite \(T\) y se recibe \(R\),
  el error
  (las posiciones de bit erradas)
  es simplemente la diferencia entre los polinomios respectivos:%
    \index{error}
  \begin{equation*}
    E(x)
      = T(x) - R(x)
  \end{equation*}
  Para que nuestra técnica detecte el error,
  debe ser que \(G(x)\) no divida a \(E(x)\).
  Nos interesa entonces estudiar bajo qué condiciones \(G(x)\)
  divide a \(E(x)\) en \(\mathbb{Z}_2[x]\),
  de manera de obtener criterios
  que den buenos polinomios generadores
  (capaces de detectar clases de errores de interés).

  Claramente no todo \(G(x)\) sirve,
  usaremos la teoría desarrollada antes
  para poner algunas condiciones.
  De partida,
  el término constante de \(G(x)\)
  no debe ser cero,
  de otra forma se desperdician bits de paridad:
  \begin{equation*}
    \left( x^n M(x) \right) \bmod \left( x^k p(x) \right)
      = x^{n - k} \left( M(x) \bmod p(x) \right)
  \end{equation*}

  Si consideramos cómo se multiplican polinomios
  en \(\mathbb{Z}_2[x]\),
  vemos que si \(G(x)\) tiene un número par de coeficientes uno
  lo mismo ocurrirá con el producto \(p(x) G(x)\),
  con lo que si \(G(x)\) tiene un número par de coeficientes uno
  detectará todos los errores que cambian un número impar de bits.

  Por otro lado,
  si \(G(x)\) es de grado \(n - 1\)
  no puede dividir a polinomios de grado menor.
  Vale decir,
  será capaz de detectar todos los errores
  que cambian bits en un rango contiguo de menos de \(n\) bits.

  Una \emph{ráfaga}%
    \index{error!rafaga@ráfaga}
  es un bloque de bits cambiados,
  con lo que \(E = 0 0 \dotsm 0 1 1 \dotsm 1 1 0 \dotsm 0\).
  Si se cambian \(r\) bits,
  esto significa que para algún \(k\):
  \begin{align*}
    E(x)
      &= (x^{r - 1} + x^{r - 2} + \dotsb + 1) x^k \\
      &= \frac{(x^r - 1) x^k}{x - 1}
  \end{align*}
  Esto es divisible por \(G(x)\) si lo es \(x^r - 1\).
  De la teoría precedente sobre campos finitos
  sabemos que si \(G(x)\)
  es el polinomio mínimo de un generador de \(\mathbb{F}_{2^n}\)
  (lo que llaman un \emph{polinomio primitivo})
  dividirá a \(x^r - 1\) solo si \(r \ge 2^n - 1\).
  Lamentablemente
  en \(\mathbb{Z}_2[x]\) el polinomio \(x + 1\) es primitivo
  y divide a todos los polinomios con un número par de términos
  (porque \(x - 1\) siempre divide a \(x^k - 1\)).

  Analicemos ahora cómo armar circuitos
  que calculen el resto de la división
  de polinomios en \(\mathbb{Z}_2[x]\).
  Requeriremos memorias de un bit,
  que al pulso de una línea de reloj
  (que no se muestra)
  aceptan un nuevo bit y entregan el anterior.
  La suma en \(\mathbb{Z}_2\)
  es la operación lógica \emph{o exclusivo},
  comúnmente anotada \(\oplus\).
  \begin{figure}[ht]
    \centering
    \subfloat[Registro]{\pgfimage{images/register}
       \label{subfig:register}}
    \qquad
    \subfloat[O exclusivo]{\pgfimage{images/xor}
       \label{subfig:xor}}
    \caption{Elementos de circuitos lógicos}
    \label{fig:circuitos-logicos}
  \end{figure}
  Los elementos de circuito que emplearemos
  se ilustran en la figura~\ref{fig:circuitos-logicos}.
  El amable lector verificará
  (por ejemplo dividiendo \(x^8 + x^5 + x^4 + x^2 + 1\)
   por \(x^4 + x + 1\))
  que el proceso para obtener el resto
  puede describirse de la siguiente forma:
  Si el primer bit del dividendo actual es \(1\),
  sume los términos de menor exponente
  a partir del segundo término del dividendo;
  en caso que el primer bit del dividendo sea \(0\),
  no haga nada.
  Luego descarte el primer bit del dividendo.
  Esto es lo mismo que sumar el primer bit del dividendo actual
  en ciertas posiciones,
  y luego correr todo en una posición.
  En términos de nuestros elementos,
  para el polinomio primitivo \(x^8 + x^4 + x^3 + x^2 + 1\)
  resulta el circuito de la figura~\ref{fig:LFSR-11d}.
  \begin{figure}[ht]
    \centering
    \pgfimage{images/LFSR-11d}
    \caption{Circuito para $x^8 + x^4 + x^3 + x^2 + 1$}
    \label{fig:LFSR-11d}
  \end{figure}
  La operación es la siguiente:
  Inicialmente se cargan ceros en los registros,
  luego se van ingresando los bits del dividendo
  (partiendo por el más significativo)
  al circuito.
  El resto queda en los registros.
  Es claro que interesan polinomios primitivos
  con el mínimo número de términos
  (ya que esto minimiza la circuitería requerida).

  Otro uso interesante resulta de inicializar los registros
  con un valor diferente de cero,
  y luego alimentar el circuito con una corriente de ceros
  (lo que puede lograrse simplemente obviando
   la primera operación a la izquierda en la figura)
  Como hay un número finito de posibilidades
  para los valores de los registros,
  en algún momento se repetirán.
  Si el valor inicial es \(p(x)\),
  lo que estamos haciendo
  es calcular sucesivamente \(x^k p(x) \bmod G(x)\).
  Si \(G(x)\) es primitivo,
  la repetición ocurrirá cuando \(k = 2^n\),
  por lo que los valores en los registros
  habrán recorrido todas las combinaciones de \(n\) bits
  (salvo solo ceros).
  El resultado es un contador simple
  (si solo interesa obtener valores diferentes,
   no necesariamente en orden),
  y la salida del circuito
  (que en nuestra aplicación anterior descartamos)
  es una corriente de números aleatorios
  si se toman de a \(n\) bits.
  Si \(G(x)\) no es primitivo,
  la teoría precedente indica que habrá un \(k\) menor a \(n\)
  que hace que \(x^k p(x) \equiv p(x) \pmod{G(x)}\),
  y nuevamente hay ciclos.
  El lector interesado determinará los posibles ciclos
  para algún polinomio no primitivo,
  como \((x^4 + x + 1) (x + 1) = x^5 + x^4 + x^2 +1\).

%%% Local Variables:
%%% mode: latex
%%% TeX-master: "clases"
%%% End:


%%% Local Variables:
%%% mode: latex
%%% TeX-master: "clases"
%%% End:

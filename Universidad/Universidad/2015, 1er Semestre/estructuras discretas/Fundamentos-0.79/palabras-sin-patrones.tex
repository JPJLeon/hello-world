% palabras-sin-patrones.tex
%
% Copyright (c) 2013-2014 Horst H. von Brand
% Derechos reservados. Vea COPYRIGHT para detalles

\subsection{Palabras que no contienen un patrón dado}
\label{sec:strings-excluding-pattern}
\index{palabra!numero@número}

  Nos interesan secuencias que no contengan un patrón dado.%
    \index{palabra!patron@patrón}
  Un ejemplo simple es secuencias binarias sin ceros seguidos.
  Llamemos \(\mathcal{B}_{00}\) a esta clase.
  Un elemento de \(\mathcal{B}_{00}\) puede ser vacío o \(0\),
  o es \(1\) o \(01\)
  seguido por un elemento de \(\mathcal{B}_{00}\).
  O sea:
  \begin{equation*}
    \mathcal{B}_{00}
      = \mathcal{E} + \{0\} + \{1, 01\} \times \mathcal{B}_{00}
  \end{equation*}
  Si \(z\) marca cada símbolo,
  para la respectiva función generatriz ordinaria \(B_{00}(z)\):%
    \index{generatriz}
  \begin{align*}
    B_{00}(z)
      &= 1 + z + (z + z^2) B_{00}(z) \\
  \intertext{Despejando:}
    B_{00}(z)
      &= \frac{1 + z}{1 - z - z^2}
  \end{align*}
  Resulta ser
    \(\left[ z^n \right] B_{00}(z) = F_n + F_{n + 1} = F_{n + 2}\),
  un número de Fibonacci.%
    \index{Fibonacci, numeros de@Fibonacci, números de}

  Si ahora buscamos que no contenga \(k\) ceros seguidos,
  podemos expresar:
  \begin{equation*}
    \mathcal{B}_{0^k}
      = \mathcal{P}_{< k}
	  + \mathcal{P}_{< k}
	      \times \{ 1 \} \times \mathcal{B}_{0^k}
  \end{equation*}
  Acá \(\mathcal{P}_{< k}\)
  es la clase de secuencias de menos de \(k\) ceros:
  \begin{equation*}
    \mathcal{P}_{< k}
      = \mathcal{E} + \{0\} + \{0\}^2 + \dotsb + \{0\}^{k - 1}
  \end{equation*}
  Las respectivas funciones generatrices ordinarias cumplen:
  \begin{align}
    P_{< k} (z)
      &= 1 + z + z^2 + \dotsb + z^{k - 1} \notag \\
      &= \frac{1 - z^k}{1 - z} \notag \\
    B_{0^k} (z)
      &= \frac{1 - z^k}{1 - z} (1 + z B_{0^k} (z)) \notag \\
  \intertext{Despejando:}
    B_{0^k} (z)
      &= \frac{1 - z^k}{1 - 2 z + z^{k + 1}}
	   \label{eq:B0^k}
  \end{align}
  Podemos extraer información adicional de acá.
  Los coeficientes de \(B_{0^k}(z)\)
  son el número de \foreignlanguage{english}{strings}
  que no contienen \(0^k\),
  el coeficiente de \(z^n\)
  dividido por \(2^n\) es la proporción del total:
  \begin{align*}
    B_{0^k}(z)
      &= \sum_{n \ge 0}
	   \text{\{\# de largo \(n\) sin \(0^k\)\}} z^n \\
    B_{0^k}(z / 2)
      &= \sum_{n \ge 0}
	   \text{\{\# de largo \(n\) sin \(0^k\)\}} / 2^n z^n \\
    B_{0^k}(1 / 2)
      &= \sum_{n \ge 0}
	   \text{\{\# de largo \(n\) sin \(0^k\)\}} / 2^n \\
      &= \sum_{n \ge 0}
	   \Pr( \text{No hay \(0^k\) en los primeros \(n\)}) \\
      &= \sum_{n \ge 0}
	   \Pr( \text{Primer \(0^k\)
		      termina después de \({} > n\)} ) \\
      &= \text{Posición esperada
	       del fin de los primeros \(k\) ceros}
  \end{align*}
  A esto se le llama \emph{tiempo de espera}%
    \index{palabra!tiempo de espera|textbfhy}
  (en inglés,
   \emph{\foreignlanguage{english}{waiting time}}).%
     \index{palabra!waiting time@\emph{\foreignlanguage{english}{waiting time}}|see{palabra!tiempo de espera}}
  Resulta:
  \begin{theorem}
    \label{theo:waiting-time-2^k}
    El tiempo de espera para los primeros \(k\) ceros
    en un \foreignlanguage{english}{string} binario al azar es:
    \begin{equation}
      \label{eq:waiting-time-2^k}
      B_{0^k}(1/2)
	= 2^{k + 1} - 2
    \end{equation}
  \end{theorem}
  O sea,
  en promedio hay que esperar \(30\)~bits hasta hallar \(0000\).
  La pregunta obvia es si esto vale también para otros patrones
  de largo cuatro,
  por ejemplo \(0001\).
  Resulta que no es así.
  Consideremos la primera vez que aparece \(000\).
  Es igualmente probable que continúe \(0000\) o \(0001\).
  Si \(0000\) no calza es \(0001\),
  y para \(0000\) debemos esperar al menos \(4\)~bits más.
  Si \(0001\) no calza,
  es porque es \(0000\) y el próximo bit puede completar \(0001\).

  Consideremos un patrón \(p\) de largo \(k\) arbitrario
  tomados entre \(s\) símbolos entonces.
  Sea \(\mathcal{B}_p\)
  la clase de \foreignlanguage{english}{strings}
  que no contienen \(p\),
  y sea \(\mathcal{T}_p\)
  la clase de \foreignlanguage{english}{strings}
  que terminan en \(p\),
  pero en los cuales \(p\) no aparece nunca antes del final.
  Es claro que \(\mathcal{B}_p\) y \(\mathcal{T}_p\) son disjuntos.
  Si agregamos un símbolo a un \foreignlanguage{english}{string}
  en \(\mathcal{B}_p\),
  el resultado es un \foreignlanguage{english}{string} no vacío
  en \(\mathcal{B}_p\) o en \(\mathcal{T}_p\).
  O sea:%
    \index{metodo simbolico@método simbólico}
  \begin{equation*}
    \mathcal{B}_p + \mathcal{T}_p
      = \mathcal{E} + \mathcal{B}_p \times \{ 0, 1, \dotsc, s - 1 \}
  \end{equation*}
  Esto nos da la ecuación funcional
  para las respectivas funciones generatrices ordinarias:%
    \index{generatriz}
  \begin{equation}
    \label{eq:B+T-fe}
    B_p(z) + T_p(z)
      = 1 + s z B_p(z)
  \end{equation}
  Hace falta determinar \(\mathcal{T}_p\).
  Es similar a \(\mathcal{B}_p \times \{ p \}\),
  pero debemos considerar que un elemento de \(\mathcal{B}_p\)
  puede terminar en ``casi'' \(p\),
  con lo que solo le falta una cola.

  El desarrollo que sigue
  es básicamente de Odlyzko~\cite{odlyzko85:_enum_strings}.
  Escribiremos \(\lvert x \rvert\)
  para el largo del \foreignlanguage{english}{string} \(x\)
  (el número de símbolos que lo componen).
  Para describir la manera
  en que dos \foreignlanguage{english}{strings} se traslapan
  definimos la \emph{correlación}%
    \index{palabra!correlacion@correlación|textbfhy}
  entre los \foreignlanguage{english}{string} \(x\) e \(y\)
  (posiblemente de distinto largo)
  como el polinomio \(c_{x y}(t)\) de grado \(\lvert x \rvert - 1\)
  tal que el coeficiente de \(t^k\)
  se determina ubicando \(y\) bajo \(x\)
  de manera que el primer caracter de \(y\)
  cae bajo el \(k\)\nobreakdash-ésimo caracter de \(x\).
  El coeficiente es \(1\) si ambos son iguales donde traslapan,
  \(0\) en caso contrario.
  Por ejemplo,
  si \(x = \mathtt{c a b c a b c}\)
  e \(y = \mathtt{a b c a b c d e}\),
  resulta \(c_{x y}(t) = t^4 + t\),
  como muestra el cuadro~\ref{tab:cxy}.
  \begin{table}[ht]
    \centering
    \begin{tabular}{>{\(}l<{\)}*{16}{>{\(\mathtt}c<{\)}}>{\(}r<{\)}}
      x: & c & a & b & c & a & b & c &	 &   &	 &   &	 &   &	 &   \\
      y: & a & b & c & a & b & c & d & e &   &	 &   &	 &   &	 & 0 \\
	 &   & a & b & c & a & b & c & d & e &	 &   &	 &   &	 & 1 \\
	 &   &	 & a & b & c & a & b & c & d & e &   &	 &   &	 & 0 \\
	 &   &	 &   & a & b & c & a & b & c & d & e &	 &   &	 & 0 \\
	 &   &	 &   &	 & a & b & c & a & b & c & d & e &   &	 & 1 \\
	 &   &	 &   &	 &   & a & b & c & a & b & c & d & e &	 & 0 \\
	 &   &	 &   &	 &   &	 & a & b & c & a & b & c & d & e & 0
    \end{tabular}
    \caption{Cálculo de $c_{x y} (t) = t^4 + t$}
    \label{tab:cxy}
  \end{table}
  Nótese que en general \(c_{x y}(t) \ne c_{y x}(t)\)
  (en el ejemplo es \(c_{y x}(t) = 0\)).
  De particular interés
  es la \emph{autocorrelación} \(c_x(t) = c_{x x}(t)\),
  la correlación de un \foreignlanguage{english}{string}
  consigo mismo.%
    \index{palabra!autocorrelacion@autocorrelación|textbfhy}
  En el ejemplo,
  \(c_x (t) = t^6 + t^3 + 1\).

  Fijemos un patrón \(p\) de largo \(k\),
  y escribamos:
  \begin{align*}
    B_p(z)
      &= \sum_{n \ge 0} b_n z^n \\
    T_p(z)
      &= \sum_{n \ge 0} t_n z^n
  \end{align*}
  Consideremos uno de los \(b_n\)
  \foreignlanguage{english}{strings} de largo \(n\)
  que no terminan en \(p\),
  y adosemos \(p\) al final.
  Sea \(n + r\) la posición
  en la cual por primera vez termina \(p\) en el resultado,
  donde \(0 < r \le k\).
  Como \(p\) también aparece al final,
  deben coincidir el prefijo de largo \(k - r\) de \(p\)
  y el sufijo de largo \(k - r\) de \(p\),
  o sea,
  \(\left[ t^{k - r} \right] c_p(t) = 1\).

  Para un ejemplo,
  sea el patrón \(p = \mathtt{a a b a}\)
  y el \foreignlanguage{english}{string}
    \(x = \mathtt{a b a b b a a b} \in \mathcal{B}_p\).
  Es \(k = \lvert p \rvert = 4\)
  y \(n = \lvert x \rvert = 8\).
  Vemos que
    \(x p = \mathtt{a b a b b
		    \textcolor{red}{a a b a}
		    \textcolor{blue}{a b a}}\),
  o sea,
  \(r = 1\)
  (hay un traslapo de \(k - r = 4 - 1 = 3\)
   entre el principio del patrón
   y el final del \foreignlanguage{english}{string}).
  Tenemos un miembro de \(\mathcal{T}_p\) de largo \(n + r = 9\)
  y una cola de largo \(k - r = 3\),
  determinados en forma única por \(x\) y \(p\).
  Esta descomposición solo es posible cuando
  \(\left[ t^{k - r} \right] c_p (t) = 1\).

  Nos interesa contar estos \foreignlanguage{english}{string}.
  Hay \(t_{n + r}\) de ellos,
  la descomposición descrita es una biyección.
  Vale decir,
  como los coeficientes de \(c_p\) son cero o uno:
  \begin{equation}
    \label{eq:bn=tnc}
    b_n
      = \sum_{0 < r \le k}
	  t_{n + r} \left[ t^{k - r} \right] c_p (t)
  \end{equation}
  Multiplicando~\eqref{eq:bn=tnc}
  por \(z^{n + k}\) y sumando para \(n \ge 0\)
  (recordar que \(k = \lvert p \rvert \ge \deg(c_p(t)) + 1\))
  da:
  \begin{align}
    B_p(z) z^k
      &= \sum_{n \ge 0}
	   z^{n + k}
	   \sum_{0 < r \le k}
	     t_{n + r}
	     \left[ t^{k - r} \right] c_p (t)
		 \notag \\
  \intertext{Esta es la convolución de \(T_p(z)\) con \(c_p(z)\):}
    B_p(z) z^k
      &= T_p(z) c_p(z) \label{eq:T-fe}
  \end{align}

  Uniendo las piezas anteriores
  tenemos un resultado de Solov'ev~%
    \cite{solovev66:_combin_ident_its_applic_probl}:
  \begin{theorem}
    \label{theo:Bp-gf}
    Sea \(p\) un patrón de largo \(k\) formado por \(s\) símbolos,
    con autocorrelación \(c_p(z)\).
    Entonces
    el número de \foreignlanguage{english}{strings} de largo \(n\)
    que no contienen el patrón \(p\)
    tiene función generatriz ordinaria:
    \begin{equation}
      \label{eq:Bp-gf}
      B_p(z)
	= \frac{c_p(z)}{(1 - s z) c_p(z) + z^k}
    \end{equation}
    El tiempo de espera para el patrón \(p\)%
      \index{palabra!tiempo de espera}
    está dado por:
    \begin{equation}
      \label{eq:WT-p}
      W_p
	= s^k c_p (1 / s)
    \qedhere
    \end{equation}
  \end{theorem}
  \begin{proof}
    La ecuación~\eqref{eq:Bp-gf}
    es la solución del sistema de ecuaciones~\eqref{eq:B+T-fe}
    y~\eqref{eq:T-fe}.
    El tiempo de espera
    es como se discutió antes para el patrón \(0^k\),
    la expresión dada
    resulta de substituir \(z = 1 / s\) en~\eqref{eq:Bp-gf}.
  \end{proof}
  Incidentalmente,
  al ser \(c_p\)
  un polinomio de coeficientes enteros de grado menor a \(k\),
  por~\eqref{eq:WT-p} el tiempo de espera siempre es un entero.

  Completando la discusión previa,
  tenemos \(c_{0000} (z) = 1 + z + z^2 + z^3\)
  y \(c_{0001} (z) = 1\).
  El tiempo de espera para el patrón \(p\)
  está dado por \(2^4 c_p(1/2)\).
  Para nuestros dos patrones,
  por las fórmulas desarrolladas antes:
  \begin{align}
    B_{0000}(z)
      &= \frac{1 - z^4}{1 - 2 z + z^5}
	      \label{B0000-fg} \\
    W_{0000}
      &= 30   \label{B0000-wt} \\
    B_{0001}(z)
      &= \frac{1}{1 - 2 z + z^4}
	      \label{B0001-fg} \\
    W_{0001}
      &= 16   \label{B0001-wt}
  \end{align}
  Por técnicas similares se pueden manejar conjuntos de patrones.

%%% Local Variables:
%%% mode: latex
%%% TeX-master: "clases"
%%% End:

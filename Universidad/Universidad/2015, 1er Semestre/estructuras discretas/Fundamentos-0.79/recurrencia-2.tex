% recurrencia-2.tex
%
% Copyright (c) 2013-2014 Horst H. von Brand
% Derechos reservados. Vea COPYRIGHT para detalles

\section{Otras recurrencias de dos índices}
\label{sec:recurrencia-2-indices}

  Consideremos el problema de calcular
  cuántos subconjuntos de \(k\) elementos de \(\{1, 2, \dotsc, n\}\)
  hay tal que no contienen números consecutivos.
  Llamemos \(s(n, k)\) a este valor.
  Para construir una recurrencia para ellos,
  aplicamos el método general de ver qué pasa al incluir o excluir \(n\).
  \begin{description}
  \item[\boldmath No incluye \(n\):\unboldmath]
    Esto es simplemente elegir \(k\) de entre los primeros \(n - 1\),
    o sea \(s(n - 1, k)\).
  \item[\boldmath Incluye \(n\):\unboldmath]
    Quedan por agregar \(k - 1\) elementos,
    que no pueden incluir a \(n - 1\),
    o sea corresponde a \(s(n - 2, k - 1)\).
  \end{description}
  Esto nos da la recurrencia:
  \begin{equation*}
    s(n, k)
      = s(n - 1, k) + s(n - 2, k - 1)
  \end{equation*}
  Ajustando índices:
  \begin{equation}
    \label{eq:recurrence-s-1}
    s(n + 2, k + 1)
      = s(n + 1, k + 1) + s(n, k)
  \end{equation}
  Es claro que para \(n \ge 1\):
  \begin{equation}
    s(n, 1)
      = n
	   \label{eq:recurrence-s(n,1)-boundary}
  \end{equation}
  Requeriremos
  los valores \(s(0, 0)\), \(s(n, 0)\), \(s(0, k)\), \(s(1, k)\).
  De la recurrencia,
  con \(n \ge 0\):
  \begin{align*}
    s(n + 2, 1)
      &= s(n + 1, 1) + s(n, 0) \\
    n + 2
      &= n + 1 + s(n, 0)
  \end{align*}
  Por tanto,
  definimos \(s(n, 0) = 1\).
  Similarmente:
  \begin{equation*}
    s(2, k + 1)
      = s(1, k + 1) + s(0, k)
  \end{equation*}
  Para \(k \ge 1\) resulta \(s(0, k) = 0\),
  con lo que \(s(0, k) = [k = 0]\).
  Además,
  uniendo los casos \(s(1, 0) = s(1, 1) = 1\)
  con \(s(1, k) = 0\) para \(k > 1\):
  \begin{equation*}
    s(1, k)
      = [ 0 \le k \le 1 ]
  \end{equation*}

  Definamos la función generatriz:
  \begin{equation}
    \label{eq:recurrence-s-GF}
    S(x, y)
      = \sum_{n, k \ge 0} s(n, k) x^n y^k
  \end{equation}
  Para aplicar nuestra técnicas de solución de recurrencias,
  necesitaremos las sumas:
  \begin{align*}
    x^2 y \sum_{n, k \ge 0} s(n + 2, k + 1) x^n y^k
      &= S(x, y)
	   - \sum_{k \ge 0} s(0, k) y^k
	   - \sum_{k \ge 0} s(1, k) x y^k
	   - \sum_{n \ge 0} s(n, 0) x^n \\
      &\hspace{6em}
	   + s(0, 0)
	   + s(1, 0) x \\
      &= S(x, y)
	   - 1
	   - x (1 + y)
	   - \frac{1}{1 - x}
	   + 1
	   + x \\
      &= S(x, y) - x y - \frac{1}{1 - x} \\
    x y \sum_{n, k \ge 0} s(n + 1, k + 1) x^n y^k
      &= S(x, y)
	   - \sum_{k \ge 0} s(0, k) y^k
	   - \sum_{n \ge 0} s(n, 0) x^n
	   + s(0, 0) \\
      &= S(x, y)
	   - 1
	   - \frac{1}{1 - x}
	   + 1 \\
      &= S(x, y) - \frac{1}{1 - x}
  \end{align*}
  Los términos que se suman se han restado dos veces,
  y deben reponerse.
  Esto con la recurrencia da:
  \begin{equation*}
    \frac{S(x, y)
	    - x (1 + y)
	    - \frac{1}{1 - x}}
	 {x^2 y}
      = \frac{S(x, y) - \frac{1}{1 - x}}{x y}
	  + S(x, y)
  \end{equation*}
  Despejando:
  \begin{equation}
    \label{eq:GF-s}
    S(x, y)
      = \frac{1 + x y}{1 - x - x^2 y}
  \end{equation}
  Podemos escribir~\eqref{eq:GF-s} como:
  \begin{align*}
    S(x, y)
      &= \frac{1 + x y}{1 - x (1 + x y)} \\
      &= \sum_{r \ge 0} x^r (1 + x y)^{r + 1} \\
      &= \sum_{r \ge 0} x^r
	   \sum_{s \ge 0} \binom{r + 1}{s} \, x^s y^s \\
      &= \sum_{r, s \ge 0} \binom{r + 1}{s} \, x^{r + s} y^s
  \end{align*}
  De aquí:
  \begin{equation}
    \label{eq:s(n,k)}
    s(n, k)
      = \left[ x^n y^k \right] \, S(x, y)
      = \binom{n - k + 1}{k}
  \end{equation}

  Para \(y = 1\) la función generatriz~\eqref{eq:GF-s} da:
  \begin{equation*}
    S(x, 1)
      = \frac{1 + x}{1 - x - x^2}
      = \frac{F(x) - x}{x^2}
  \end{equation*}
  Acá \(F(x)\) es la función generatriz
  de los números de Fibonacci~\eqref{eq:gf-Fibonacci}.
  Esto concuerda con sumar la recurrencia~\eqref{eq:recurrence-s-1}
  sobre todo \(k\),
  como \(s(0, 0) = 1 = F_2\) y \(s(1, 0) + s(1, 1) = 2 = F_3\).

  Otro caso de interés son los números de Delannoy~%
    \cite{delannoy95:_quest_probab}
  (ver también la discusión de Banderier y Schwer~%
    \cite{banderier05:_why_delannoy_numbers}).
  Se define \(D(m, n)\)
  como el número de caminos entre \((0, 0)\) y \((m, n)\)
  en una cuadrícula,
  si se permiten únicamente pasos hacia el norte, el nordeste o este.
  De la definición es clara la recurrencia:
  \begin{equation}
    \label{eq:Delannoy-recurrence}
    D(m, n)
      = D(m - 1, n) + D(m, n - 1) + D(m - 1, n - 1)
    \qquad D(0, 0) = 1
  \end{equation}
  Para aplicar nuestra técnica requeriremos:
  \begin{equation}
    \label{eq:Delannoy-boundary}
    D(m, 0) = D(0, n) = 1
  \end{equation}
  Definiendo la función generatriz ordinaria:
  \begin{equation}
    \label{eq:Delannoy-GF-definition}
    d(x, y)
      = \sum_{m, n \ge 0} D(m, n) x^m y^n
  \end{equation}
  obtenemos la ecuación funcional:
  \begin{equation}
    \label{eq:Delannoy-FE}
    \frac{d(x, y) - d(0, y) - d(y, 0) + d(0, 0)}{x y}
      = \frac{d(x, y) - d(0, y)}{x} + \frac{d(x, y) - d(x, 0)}{y} + d(x, y)
  \end{equation}
  Las condiciones de contorno dan:
  \begin{equation}
    \label{eq:Delannoy-boundary-GF}
    d(0, y)
      = \frac{1}{1 - y}
    \hspace{3em}
    d(x, 0)
      = \frac{1}{1 - x}
  \end{equation}
  Substituyendo~\eqref{eq:Delannoy-boundary-GF} en~\eqref{eq:Delannoy-FE}
  y despejando \(d(x, y)\) resulta:
  \begin{equation}
    \label{eq:Delannoy-GF}
    d(x, y)
      = \frac{1}{1 - x - y - x y}
  \end{equation}
  El lector interesado verificará que expandir como serie geométrica
  y extraer los coeficientes respectivos resulta en:
  \begin{equation}
    \label{eq:Delannoy}
      D(m, n)
	= \sum_t \binom{m + n - t}{n} \, \binom{n}{t}
  \end{equation}
  La asimetría de~\eqref{eq:Delannoy}
  ofende las sensibilidades del autor.
  Puede escribirse en forma simétrica en términos de coeficientes trinomiales,
  eso sí.

  La siguiente idea da una expansión más simétrica:
  \begin{align*}
    d(x, y)
      &= \frac{1}{(1 - x) (1 - y) - 2 x y} \\
      &= \frac{(1 - x) (1 - y)}{1 - \frac{2 x y}{(1 - x) (1 - y)}} \\
      &= (1 - x) (1 - y)
	   \sum_{r \ge 0} \left( \frac{2 x y}{(1 - x) (1 - y)} \right)^r \\
      &= \sum_{r \ge 0}
	  \frac{2^r x^r y^r}{(1 - x)^{r - 1} \, (1 - y)^{r - 1}} \\
      &= \sum_{r \ge 0} 2^r \,
	   \sum_{s \ge 0} \binom{r + s}{s} \, x^{r + s}
	   \sum_{t \ge 0} \binom{r + t}{t} \, y^{r + t}
  \end{align*}
  Al extraer el coeficiente de \(x^m y^n\)
  solo sobreviven los términos con \(r + s = m\) y \(r + t = n\),
  aprovechando la simetría de los coeficientes binomiales:
  \begin{equation}
    \label{eq:Delannoy-alt}
    D(m, n)
      = \sum_{r \ge 0} 2^r \, \binom{m}{r} \, \binom{n}{r}
  \end{equation}
  En todo caso,
  ya habíamos demostrado esta identidad como~\eqref{eq:so:binomial-identity}
  usando aceite de serpiente.

%%% Local Variables:
%%% mode: latex
%%% TeX-master: "clases"
%%% End:

% prefacio.tex
%
% Copyright (c) 2012-2014 Horst H. von Brand
% Derechos reservados. Vea COPYRIGHT para detalles

\chapter*{Prefacio}
\label{cha:prefacio}

  Este documento presenta
  (y extiende substancialmente)
  la materia de los ramos \emph{Fundamentos de Informática I},
  \emph{Fundamentos de Informática II}
  y \emph{Estructuras Discretas}
  como dictados durante los años 2009 a~2014
  en la Casa Central de Universidad Técnica Federico Santa María
  por el autor.
  El tratamiento de algunos temas es definitivamente no tradicional,
  y en algunas áreas el autor sigue líneas de razonamiento
  que le parecen interesantes,
  aún si no son directamente parte del curso.
  Así hay material adicional
  a la materia oficial del curso en estos apuntes,
  que no se vio en clase.
  Se notan resultados que están fuera del temario
  donde ayudan a iluminar los temas tratados.
  Se ha hecho el intento de juntar todo al material relevante,
  en forma accesible para el no especialista.
  Intentamos también seguir la exhortación de Knuth de no perderse
  en abstracción excesiva.

  Veremos una colección de temas
  que en conjunto se conocen
  bajo el nombre de \emph{matemáticas discretas}.
  Trataremos de razonamiento matemático,
  y nos ocuparemos más que nada de fenómenos discretos,
  en contraposición de lo continuo que es el ámbito del cálculo.
  Siendo un área menos conocida,
  encontraremos en ella resultados sorprendentes
  y técnicas ingeniosas.
  La importancia en la informática
  es que en computación no se tratan fenómenos continuos.
  El aprender a razonar en el ámbito de objetos discretos,
  y las técnicas que veremos durante el curso de estos ramos,
  serán útiles a la hora de diseñar sistemas
  y evaluar su desempeño.
  El análisis complejo ofrece herramientas poderosas,
  particularmente para derivar estimaciones asintóticas
  de muchas de las cantidades de interés.
  Al no ser materia cubierta
  en el currículum tradicional de las carreras de ingeniería,
  se incluye una breve reseña de los resultados requeridos.

  La razón de fondo de preocuparse de cómo razonar,
  en particular en el ámbito de la informática,
  es que cada día dependemos más de sistemas informáticos,
  que han dejado de ser accesorios
  para transformarse en parte indispensable
  de nuestra vida diaria.
  Fallas en tales sistemas pueden tener consecuencias desastrosas.
  Ejemplos de estas situaciones abundan,
  lamentablemente.
  Particularmente preocupantes son cuando errores lógicos
  son los causantes.
  En este sentido,
  parte del objetivo
  de los presentes ramos en el currículum de informática
  es entrenar en el arte
  de enfrentar problemas en forma estructurada,
  y de reconocer cuándo se tiene una solución correcta.

%%% Local Variables:
%%% mode: latex
%%% TeX-master: t
%%% End:

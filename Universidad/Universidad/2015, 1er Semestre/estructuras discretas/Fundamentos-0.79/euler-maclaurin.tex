% euler-maclaurin.tex
%
% Copyright (c) 2010-2014 Horst H. von Brand
% Derechos reservados. Vea COPYRIGHT para detalles

\chapter{La fórmula de Euler-Maclaurin}
\label{cha:euler-maclaurin}

  Es común que interese el valor de alguna suma,
  particularmente alguna suma infinita.
  En muchos casos de interés la suma converge muy lentamente,
  y resulta indispensable
  contar con alguna técnica que permita acelerarla.
  En otros casos una expresión simple para un valor aproximado
  de una suma finita
  resulta mucho más útil que el valor exacto.
  Una de las técnicas principales para aproximar sumas infinitas
  es la fórmula de Euler-Maclaurin.
  En su desarrollo toman lugar central
  los polinomios y números de Bernoulli,
  que a su vez aparecen inesperadamente en muchas situaciones combinatorias.
  Entre otras aplicaciones,
  la fórmula de Euler-Maclaurin
  permite obtener aproximaciones simples para factoriales
  y números harmónicos,
  valores que a su vez son ubicuos en la combinatoria
  (y por tanto el análisis de algoritmos).

\section{Relación entre suma e integral}
\label{sec:relacion-suma-integral}

  Conceptualmente la suma y la integral
  están íntimamente relacionadas,
  ambas podemos representarlas como áreas bajo curvas
  como en la figura~\ref{fig:fz}.
  \begin{figure}[htbp]
    \centering
    \pgfimage{images/fz}
    \caption{Suma e integral como áreas}
    \label{fig:fz}
  \end{figure}
  Pareciera ser que la integral (área bajo la curva)
  y la suma (área bajo la escalera)
  tienden a tener una diferencia constante.
  Esto es exactamente lo que asegura nuestro siguiente teorema.
  \begin{theorem}[Maclaurin-Cauchy]
    \index{Maclaurin-Cauchy, teorema de|textbfhy}
    \label{theo:maclaurin-Cauchy}
    Sea \(f(z)\) una función continua,
    positiva
    y que tiende monotónicamente a cero.
    Entonces existe la constante de Euler:%
      \index{Euler, constante de (para una funcion)@Euler, constante de (para una función)}
    \begin{equation*}
      \gamma_f
	= \lim_{n \rightarrow \infty}
	    \left(
	      \sum_{1 \le k < n} f(k) - \int_1^n f(z)
					  \, \mathrm{d} z
	    \right)
    \end{equation*}
  \end{theorem}
  \begin{proof}
    Como \(f\) es continua,
    la integral existe para todo \(n \in \mathbb{N}\).
    Por ser decreciente:
    \begin{equation*}
      f(\lceil z \rceil)
	\le f(z)
	\le f(\lfloor z \rfloor)
    \end{equation*}
    Entonces:
    \begin{align*}
      \int_1^n	f(\lceil z \rceil) \, \mathrm{d} z
	&\le \int_1^n f(z) \, d z
	\le \int_1^n f( \lfloor z \rfloor ) \, \mathrm{d} z \\
      \sum_{1 \le k < n} f(k + 1)
	&\le \int_1^n f(z) \, \mathrm{d} z
	\le \sum_{1 \le k < n}	f(k) \\
      \sum_{2 \le k < n + 1} f(k)
	&\le \int_1^n f(z) \, \mathrm{d} z
	\le \sum_{1 \le k < n} f(k)
    \end{align*}
    Así,
    la diferencia:
    \begin{equation*}
      a_n
	= \sum_{1 \le k < n} f(k) - \int_1^n f(z) \, \mathrm{d} z
    \end{equation*}
    satisface \(0 \le f(n) \le a_n \le f(1)\).
    Además:
    \begin{equation*}
      a_{n + 1} - a_n
	= f(n + 1) - \int_n^{n + 1} f(z) \, \mathrm{d} z
	\le 0
    \end{equation*}
    O sea,
    la secuencia \(a_n\) es decreciente y acotada,
    y por tanto converge.
  \end{proof}
  Nótese que la demostración da cotas precisas:
  \(0 \le \gamma_f \le f(1)\).
  Consideremos la situación geométrica:
  La diferencia entre la línea continua y la escalera
  de la figura~\ref{fig:fz}
  es una serie de ``triángulos''
  de base \(1\) cuyas alturas suman \(f(1)\),
  con lo que su área total es aproximadamente \(f(1) / 2\),
  y esto suele no ser tan mala aproximación de \(\gamma_f\).

\section{Desarrollo de la fórmula}
\label{sec:desarrollo-euler-maclaurin}

  Con la intención de tomar límites \(b \rightarrow \infty\) luego,
  para calcular la suma entre \(1\) y \(a\)
  escribimos:
  \begin{align*}
    \sum_{1 \le k < b} f(k) - \int_1^b f(z) \, \mathrm{d} z
      &= \sum_{1 \le k < a} f(k) - \int_1^a f(z) \, \mathrm{d} z
	   + \sum_{a \le k < b} f(k) - \int_a^b f(z) \, \mathrm{d} z
  \end{align*}
  Nos falta aproximar los últimos términos.

  Tomemos el tramo entre un entero y el siguiente,
  para simplificar el rango entre \(0\) y \(1\).
  Nuestro resultado final
  se obtendrá sumando sobre los diferentes tramos.
  Integrando por partes:
  \begin{align*}
    \int_0^1 f(z) \, \mathrm{d} z
      &= z f(z) \, \biggr|_0^1
	  - \int_0^1 z f'(z) \, \mathrm{d} z \\
      &= z f(z) \, \biggr|_0^1
	   -  \frac{1}{2} \, z^2 f'(z) \, \biggr|_0^1
	   + \int_0^1 \frac{1}{2} \, z^2 f''(z) \, \mathrm{d} z \\
      &= z f(z) \, \biggr|_0^1
	   -  \frac{1}{2} \, z^2 f'(z) \, \biggr|_0^1
	   +  \frac{1}{2 \cdot 3} \, z^3 f''(z) \, \biggr|_0^1
	   - \int_0^1 \frac{1}{2 \cdot 3} \, z^3 f'''(z)
	       \, \mathrm{d} z
  \end{align*}
  Están apareciendo las derivadas sucesivas de \(f\)
  multiplicadas por polinomios.
  Si queremos polinomios mónicos,
  aparecerán divididos por factoriales.
  Llamemos \(B_n(z)\) al polinomio mónico de grado \(n\),
  partiendo con \(B_0(z) = 1\).
  Integrando por partes tenemos la relación básica:
  \begin{equation*}
    \int_0^1 B_n(z) f^{(n)}(z) \, \mathrm{d} z
      = \frac{B_{n + 1} (z)}{n + 1} \, f^{(n)} (z) \, \biggr|_0^1
	  - \int_0^1 \frac{B_{n + 1} (z)}{n + 1} \, f^{(n + 1)} (z)
	      \, \mathrm{d} z
  \end{equation*}
  De acá:%
    \index{Bernoulli, polinomios de!recurrencia|textbfhy}
  \begin{align}
    B_0(z)
      &= 1 \label{eq:B0} \\
    B'_{n + 1} (z)
      &= (n + 1) B_n(z)	 \quad n \ge 0
	    \label{eq:Bn}
  \end{align}
  Queda por definir la constante de integración en~\eqref{eq:Bn}.
  Tenemos primeramente:
  \begin{equation}
    \label{eq:Euler-Maclaurin-1}
    \int_0^1 f(z) \, \mathrm{d} z
      = \sum_{0 \le k \le n}
	   \frac{(-1)^k B_{k + 1}(z)}{(k + 1)!} \, f^{(k)}(z)
	     \, \biggr|_0^1
	 - \int_0^1 (-1)^n \frac{B_{n + 1}(z)}{(n + 1)!} \,
			     f^{(n + 1)}(z) \, \mathrm{d} z
  \end{equation}
  Interesa sumar la expresión~\eqref{eq:Euler-Maclaurin-1}
  para \([a, a + 1]\), \([a + 1, a + 2]\), \ldots, \([b - 1, b]\),
  conviene que se cumpla:
  \begin{equation}
    \index{Bernoulli, polinomios de|textbfhy}
    \label{eq:Bernoulli-polynomial-ends}
    B_n(0)
      = B_n(1)
  \end{equation}
  de forma que los términos intermedios se cancelen.
  Como \(B_0(z) = 1\),
  \(B_1(z)\) es una función lineal
  que solo si fuera constante cumpliría \(B_1(0) = B_1(1)\).
  La relación~\eqref{eq:Bernoulli-polynomial-ends}
  es válida siempre que \(n \ne 1\).
  Definimos en general:
  \begin{equation}
    \index{Bernoulli, numeros de@Bernoulli, números de|textbfhy}
    \label{eq:Bernoulli-number-definition}
    B_n
      = B_n(1)
  \end{equation}
  A los polinomios \(B_n(x)\)
  se les conoce como \emph{polinomios de Bernoulli},
  y las constantes \(B_n\) como \emph{números de Bernoulli},
  por razones
  que discutiremos en la sección~\ref{sec:suma-potencias}.
  Los números y polinomios de Bernoulli
  aparecen en una amplia gama de situaciones.
  Debe tenerse cuidado,
  hay autores que definen la secuencia
  (bajo el mismo nombre e incluso con la misma notación)
  de forma que todos los elementos son cero o positivos.

  En vista de la recurrencia~\eqref{eq:Bn},%
    \index{Bernoulli, polinomios de!recurrencia}
  si \(n \ge 2\) la condición~\eqref{eq:Bernoulli-polynomial-ends}
  puede expresarse también como:%
    \index{Bernoulli, polinomios de!integral}
  \begin{equation}
    \label{eq:Bernoulli-polynomial-integral}
    \int_0^1 B_n(z) \, \mathrm{d} z = 0
  \end{equation}
  Por el proceso que los produce,
  todos los coeficientes
  de los polinomios \(B_n(z)\) son racionales,
  por lo que también lo son las constantes \(B_n\).
  Los primeros polinomios y constantes
  registra el cuadro~\ref{tab:Bernoulli}.
  \begin{table}[htbp]
    \centering
    \begin{align*}
      \begin{array}{l@{\hspace*{2em}}l@{${} = {}$}c}
	\displaystyle
	  B_0(z) = 1 &
	  B_0 & \displaystyle \phantom{-}1 \\[1.5ex]
	\displaystyle
	  B_1(z) = z - \frac{1}{2} &
	  B_1 & \displaystyle -\frac{1}{2} \\[1.5ex]
	\displaystyle
	  B_2(z) = z^2 - z + \frac{1}{6} &
	  B_2 & \displaystyle \phantom{-}\frac{1}{6} \\[1.5ex]
	\displaystyle
	  B_3(z) = z^3 - \frac{3}{2} z^2 + \frac{1}{2} z &
	  B_3 & \displaystyle \phantom{-}0 \\[1.5ex]
	\displaystyle
	  B_4(z) = z^4 - 2 z^3 + z^2 - \frac{1}{30} &
	  B_4 & \displaystyle -\frac{1}{30} \\[1.5ex]
	\displaystyle
	  B_5(z) = z^5 - \frac{5}{2} z^4 + \frac{5}{3} z^3
		    - \frac{1}{6} z &
	  B_5 & \displaystyle \phantom{-}0 \\[1.5ex]
	\displaystyle
	  B_6(z) = z^6 - 3 z^5 + \frac{5}{2} z^4 - \frac{1}{2} z^2
		    + \frac{1}{42} &
	  B_6 & \displaystyle \phantom{-}\frac{1}{42} \\[1.5ex]
	\displaystyle
	  B_7(z) = z^7 - \frac{7}{2} z^6 + \frac{7}{2} z^5
		    - \frac{7}{6} z^3 + \frac{1}{6} z &
	  B_7 & \displaystyle \phantom{-}0 \\[1.5ex]
	\displaystyle
	  B_8(z) = z^8 - 4 z^7 + \frac{14}{3} z^6 - \frac{7}{3} z^4
		    + \frac{2}{3} z^2 - \frac{1}{30} &
	  B_8 & \displaystyle -\frac{1}{30} \\[1.5ex]
	\displaystyle
	  B_9(z) = z^9 - \frac{9}{2} z^8 + 6 z^7 - \frac{21}{5} z^5
		    + 2 z^3 -\frac{3}{10} z &
	  B_9 &	 \displaystyle \phantom{-}0 \\[1.5ex]
	\displaystyle
	  B_{10}(z) = z^{10} - 5 z^9 + \frac{15}{2} z^8
		    - 7 z^6 + 5 z^4 - \frac{3}{2} z^2
		    + \frac{5}{66} &
	  B_{10} &  \displaystyle \phantom{-} \frac{5}{66}
      \end{array}
    \end{align*}
    \caption[Polinomios y números de Bernoulli]
	    {Polinomios y números de Bernoulli~\cite{DLMF}}
    \label{tab:Bernoulli}
    \index{Bernoulli, polinomios de!cuadro}
    \index{Bernoulli, numeros de@Bernoulli, números de!cuadro}
  \end{table}
  Se observa que salvo \(B_1\)
  los valores de \(B_n\) para \(n\) impar
  son cero,
  y que los \(B_{2 n}\) alternan signo.
  Esto lo demostraremos en general más adelante.

  Para simplificar la derivación que sigue,
  definimos una extensión periódica del polinomio \(B_n(z)\):
  \begin{equation*}
    \widetilde{B}_n(z)
      = B_n(z - \lfloor z \rfloor)
  \end{equation*}
  La función \(\widetilde{B}_n(z)\) es continua
  dado que definimos \(B_n(0) = B_n(1) = B_n\)
  (salvo cuando \(n = 1\)).
  Así tenemos:
  \begin{align}
    \int_a^b f(z) \, \mathrm{d} z
      &= \sum_{1 \le k \le n}
	   \frac{(-1)^k \widetilde{B}_k(z)}{k!} \, f^{(k - 1)}(z)
	     \, \biggr|_a^b
	  + (-1)^{n + 1}
	      \int_a^b \frac{\widetilde{B}_{n + 1}(z)}{(n + 1)!} \,
				    f^{(n + 1)}(z)
		\, \mathrm{d} z \notag \\
      &= \frac{1}{2} f(a)
	   + \sum_{a < r < b} f(r)
	   + \frac{1}{2} f(b) \notag \\
      &\hspace{3.5em}
	    + \sum_{2 \le k \le n}
		\frac{(-1)^k B_k}{k!} \, f^{(k - 1)}(z)
		  \, \biggr|_a^b
	    + (-1)^{n + 1} \int_a^b
			     \frac{\widetilde{B}_{n + 1}(z)}
				  {(n + 1)!} \,
			     f^{(n + 1)}(z) \, \mathrm{d} z
	    \notag \\
      &= \sum_{a \le r < b} f(r)
	   + \sum_{1 \le k \le n}
	      \frac{(-1)^k B_k}{k!} \, f^{(k - 1)}(z)
		 \, \biggr|_a^b
	 + (-1)^{n + 1}
	     \int_a^b \frac{\widetilde{B}_{n + 1}(z)}{(n + 1)!} \,
				   f^{(n + 1)}(z)
	       \, \mathrm{d} z
	   \label{eq:Euler-Maclaurin-2}
  \end{align}
  Acá aprovechamos que \(B_1 = - 1 / 2\),
  absorbimos el término \(f(a)\)
  en la primera sumatoria y reorganizamos.

  Dividiendo el rango de la suma en~\eqref{eq:Euler-Maclaurin-2}
  \begin{multline}
    \label{eq:Euler-Maclaurin-3}
    \sum_{1 \le k < b} f(k) - \int_1^b f(z) \, \mathrm{d} z
      = \sum_{1 \le k < a} f(k) - \int_1^a f(z) \, \mathrm{d} z \\
	   - \sum_{1 \le k \le n}
	       \frac{(-1)^k B_k}{k!} \, f^{(k - 1)}(z)
		  \, \biggr|_a^b
	   + (-1)^n \int_a^b \frac{\widetilde{B}_{n + 1}(z)}
				  {(n + 1)!} \,
				     f^{(n + 1)}(z)
		      \, \mathrm{d} z
  \end{multline}
  Haciendo ahora \(b \rightarrow \infty\),
  y reorganizando~\eqref{eq:Euler-Maclaurin-3}
  bajo el entendido que:
  \begin{equation*}
    \lim_{z \rightarrow \infty} f^{(n)} (z) = 0
  \end{equation*}
  Recordando que salvo \(B_1\) todos los \(B_{2 k + 1} = 0\),
  obtenemos la fórmula de Euler-Maclaurin:
  \begin{equation}
    \index{Euler-Maclaurin, formula de@Euler-Maclaurin, fórmula de|textbfhy}
    \label{eq:Euler-Maclaurin}
    \sum_{1 \le k < a} f(k)
      = \int_1^a f(z) \, \mathrm{d} z
	  + \gamma_f
	  + B_1 f(a)
	  + \sum_{1 \le k \le n}
	       \frac{B_{2 k}}{(2 k)!} f^{(2 k - 1)}(a)
	  + R_n(f; a)
  \end{equation}
  En esto hemos escrito:
  \begin{align}
    \gamma_f
      &= \lim_{b \rightarrow \infty}
	   \left(
	     \sum_{1 \le k \le b} f(k) - \int_1^b f(z)
					   \, \mathrm{d} z
	   \right)
	      \label{eq:gamma-f} \\
    R_n(f; a)
      &= \int_a^\infty
	   \frac{\widetilde{B}_{2 n + 1}(z)}{(2 n + 1)!} \,
	     f^{(2 n + 1)}(z) \, \mathrm{d} z
	      \label{eq:euler-maclaurin-residue}
  \end{align}
  Resta encontrar mejores maneras
  de determinar los polinomios \(B_n(z)\),
  los coeficientes \(B_n = B_n(0)\),
  y finalmente acotar el resto \(R_n(f; a)\).
  Esto lo haremos en la sección~\ref{sec:resto-Euler-Maclaurin}.
  Lamentablemente,
  los \(B_n\) crecen muy rápidamente
  y~\eqref{eq:Euler-Maclaurin} rara vez converge,
  por lo que la constante \(\gamma_f\)
  debe determinarse de alguna otra forma.
  Las cotas que daremos indican que el error cometido
  es a lo más el último término incluido,
  la fórmula igual es útil para obtener valores numéricos precisos.

\section{Suma de potencias}
\label{sec:suma-potencias}

  Una aplicación obvia de la fórmula de Euler-Maclaurin%
    \index{Euler-Maclaurin, formula de@Euler-Maclaurin, fórmula de}
  es calcular las sumas:
  \begin{equation*}
    S_m(n)
      = \sum_{1 \le k \le n - 1} k^m
  \end{equation*}
  Acá tenemos:
  \begin{align*}
    f(z)
      &= z^m \\
    f^{(k)}(z)
      &= m^{\underline{k}} z^{m - k}
  \end{align*}
  La fórmula de Euler-Maclaurin sumando hasta el término \(m - 1\)
  (el resto es cero en este caso;
   en realidad estamos aplicando~\eqref{eq:Euler-Maclaurin-2},
   no hay constante \(\gamma\) porque es parte del resto)
  y tomando la suma desde \(0\) para simplificar
  da:
  \begin{align*}
    S_m(n)
      &= \int_0^n z^m \, \mathrm{d} z
	   + \sum_{0 \le k \le m - 1}
	       \frac{B_{k + 1}}{(k + 1)!} \,
		 m^{\underline{k}} \, n^{m - k} \\
      &= \frac{n^{m + 1}}{(m + 1)}
	   + \sum_{0 \le k \le m - 1}
	       B_{k + 1} \,
		  \frac{m^{\underline{k}}}{(k + 1)!} \,
		  n^{m - k} \\
      &= \frac{1}{m + 1}
	   \left(
	     n^{m + 1}
	       + \sum_{0 \le k \le m - 1} \binom{m + 1}{k + 1} \,
		   B_{k + 1} \, n^{m - k}
	   \right)
  \end{align*}
  Pero como \(B_0 = 1\),
  podemos incorporar el primer término a la suma,
  y luego de ajustar índices queda:%
    \index{suma!potencias|textbfhy}
  \begin{equation}
    \label{eq:Bernoulli-Smn}
    S_m(n)
      = \frac{1}{m + 1} \,
	  \left(
	    \sum_{0 \le k \le m}
	      \binom{m + 1}{k} \, B_k n^{m + 1 - k}
	  \right)
  \end{equation}
  Jakob Bernoulli%
    \index{Bernoulli, Jakob}
  había notado esta expansión,
  e incluso la usó para calcular \(S_{10}(1\,000)\).
  Es en honor a su descubrimiento
  que llevan su nombre estos números.

  La fórmula~\eqref{eq:Bernoulli-Smn}
  a veces se atribuye erróneamente a Faulhaber,%
    \index{Faulhaber, Johann}
  quien desarrolló fórmulas eficientes
  para expresar \(S_{2 m + 1}(n)\)
  en términos de \(n (n + 1)\).
  Una discusión detallada de sus resultados,
  reconstrucción de sus posibles métodos
  y una variedad de extensiones presenta Knuth~%
    \cite{knuth93:_johann_faulhaber_sums_powers}.

\section{Números harmónicos}
\label{sec:em-harmonicos}
\index{numeros harmonicos@números harmónicos}

  Usemos ahora nuestro nuevo juguete
  para aproximar los números harmónicos.
  Tenemos primeramente:
  \begin{equation*}
    \mathrm{D}^k z^{-1}
       = (-1)^{\underline{k}} \, z^{- k - 1}
       = (-1)^k k! z^{- k - 1}
  \end{equation*}
  Las derivadas tienden a cero cuando \(z \rightarrow \infty\),
  así que vamos bien.
  La fórmula de Euler-Maclaurin da:%
    \index{Euler-Maclaurin, formula de@Euler-Maclaurin, fórmula de}%
    \index{numeros harmonicos@números harmónicos!aproximacion@aproximación}
  \begin{align}
    H_n
      &= \frac{1}{n} + \sum_{1 \le r < n} \frac{1}{r} \notag \\
      &= \frac{1}{n} + \int_1^n \frac{\mathrm{d} z}{z}
	   + \gamma
	   + B_1 \cdot (-1) 1! n^{-1}
	   + \sum_{1 \le k \le s}
	       \frac{B_{2 k}}{(2 k)!}
		  \cdot (2 k - 1)! n^{- 2 k - 2}
	   + R_s(n) \notag \\
      &= \ln n + \frac{1}{n}
	   + \gamma
	   - \frac{1}{2 n}
	   + \sum_{1 \le k \le s}
	       \frac{B_{2 k}}{2 k}
		  \cdot n^{- 2 k - 2}
	   + R_s(n)
	      \label{eq:Hn-asy-Bn} \\
      &= \ln n + \gamma + \frac{1}{2 n}
	   - \frac{1}{12 n^2}
	   + \frac{1}{120 n^4}
	   - \frac{1}{252 n^6}
	   + O(n^{-8})
	      \label{eq:Hn-asy-numbers}
  \end{align}
  Por el teorema de Maclaurin-Cauchy,%
    \index{Maclaurin-Cauchy, teorema de}
  existe la constante:
  \begin{align*}
    \index{Euler, constante de|textbfhy}
    \index{\(\gamma\) (constante de Euler)|see{Euler, constante de}}
    \gamma
      &= \lim_{n \rightarrow \infty}
	   \left(
	     \sum_{1 \le k < n}
	       \frac{1}{k} - \int_1^n \frac{\mathrm{d} z}{z}
	   \right) \\
      &= \lim_{n \rightarrow \infty}
	   \left( H_n - \ln n \right) \\
      &\approx 0,5772156649
  \end{align*}
  La aproximación simple obtenida
  luego del teorema de Maclaurin-Cauchy
  da \(\gamma \approx 1 / 2\).
  Nada mal.

  Euler%
    \index{Euler, Leonhard}
  en 1736 obtuvo el valor de \(\gamma\) con \(16\) dígitos
  usando \(8\) términos de la expansión:
  \begin{equation*}
    \gamma
      = H_{10} - \ln 10 - \frac{1}{20} + \frac{1}{1200} - \dotsb
  \end{equation*}
  Para calcular el número de términos requeridos
  para una precisión similar directamente
  podemos usar la aproximación que derivamos.
  Para la precisión que obtuvo Euler
  requeriríamos \(n\) tal que:
  \begin{equation*}
    \left\lvert \gamma - (H_n - \ln n) \right\rvert
      \approx \frac{1}{2 n}
      < 5 \cdot 10^{-17} \\
  \end{equation*}
  Resulta \(10^{16}\) términos.

  Al número \(\gamma\)
  se le conoce como \emph{constante de Euler-Mascheroni}%
    \index{Euler-Mascheroni, constante de|see{Euler, constante de}}
  o simplemente como \emph{constante de Euler}.
  Gourdon y Sebah~%
    \cite{gourdon03:_euler_const}
  incluso lo consideran
  el tercer número más importante de la matemática,
  después de \(\pi\) y \(e\).
  Determinar si \(\gamma\) es racional,%
    \index{numero@número!irracional!\(\gamma\)}
  algebraico o trascendente
  es un problema abierto famoso.

\section{Fórmula de Stirling}
\label{sec:em-Stirling}

  Veamos cómo podemos aproximar factoriales
  con esta herramienta.
  Primero tenemos:
  \begin{equation*}
    \ln n!
      = \sum_{1 \le k < n} \ln k + \ln n
  \end{equation*}
  El teorema de Maclaurin-Cauchy no sirve si
  (como acá)
  tenemos una función creciente.
  Pero en caso que la función \(f(z)\) sea monótona creciente
  es claro que:
  \begin{align*}
    \int_1^n \lfloor f(z) \rfloor \, \mathrm{d} z
       \le \int _1^n f(z) \, \mathrm{d} z
      &\le \int_1^n \lceil f(z) \rceil \, \mathrm{d} z \\
    \sum_{1 \le k < n} f(k)
       \le \int _1^n f(z) \, \mathrm{d} z
      &\le \sum_{1 \le k < n} f(k + 1) \\
      &= \sum_{1 \le k < n} f(k) + f(n) - f(1)
  \end{align*}
  Promediando ambas cotas
  (la diferencia entre la curva y las escaleras son casi triángulos)
  queda:
  \begin{equation*}
    \sum_{1 \le k < n} f(k)
      \approx \int_1^n f(z) \, \mathrm{d} z
		- \frac{1}{2} \, (f(n) - f(1))
  \end{equation*}
  Hasta acá podemos decir que,
  burdamente:
  \begin{align*}
    \ln n!
      &= \sum_{1 \le k < n} \ln n + \ln n \\
      &\approx \int_1^n \ln z \, \mathrm{d} z
	  - \frac{1}{2} \, (\ln n - \ln 1)
	  + \ln z \\
      &= n \ln n - n + 1 + \frac{1}{2} \, \ln n \\
    n!
      &\approx e \sqrt{n} \, \left(\frac{n}{e}\right)^n
  \end{align*}
  Esto nos hace albergar la esperanza de obtener algo útil.

  Para una mejor aproximación usamos la fórmula de Euler-Maclaurin:
  \begin{align*}
    \int_1^n \ln z \, \mathrm{d} z
      &= n \ln n - n + 1 \\
    \mathrm{D}^k \ln z
      &= (-1)^{k - 1} (k - 1)! z^{-k}
  \end{align*}
  Si suponemos que existe la constante \(\ln \sigma\)
  (\emph{constante de Stirling}):%
    \index{Stirling, constante de}
  \begin{align}
    \ln n!
      &= \ln n + \sum_{1 \le k < n} \ln k \\
      &= \ln n + \int_1^n \ln z \, \mathrm{d} z + \ln \sigma
	   + B_1 \ln n \notag \\
      &\qquad
	   + \sum_{1 \le k \le s}
	       \frac{B_{2 k}}{(2 k)!}
		 \cdot (-1)^{2 k - 2} (2 k - 2)!
		 \cdot n^{- 2 k - 1}
	   + R_s(n) \notag \\
      &= \ln \sigma + (n + 1) \ln n - n
	   - \frac{1}{2} \ln n
	   - \sum_{1 \le k \le s}
	       \frac{B_{2 k}}{(2 k - 1) 2 k} \, n^{- 2 k - 1}
	   + R_s(n) \notag \\
      &= \ln \sigma + n \ln n + \frac{1}{2} \, \ln n - n
	   + \frac{1}{12 n}
	   - \frac{1}{360 n^3}
	   + \frac{1}{1260 n^5}
	   + R_3(n)
	      \label{eq:Stirling-1}
  \end{align}
  La constante \(\ln \sigma\)
  en~\eqref{eq:Stirling-1}
  queda determinada por el siguiente límite,
  si existe:
  \begin{align*}
    \ln \sigma
      &= \lim_{n \rightarrow \infty}
	   \left(
	     \sum_{1 \le k \le n}
	       \ln k - \int_1^n \ln z \, \mathrm{d} z
	   \right) \\
      &= \lim_{n \rightarrow \infty}
	   \left(
	     \ln n! - n \ln n - n
	   \right)
  \end{align*}
  Veremos más adelante que \(\sigma = \sqrt{2 \pi}\),
  usando este valor en~\eqref{eq:Stirling-1},
  y expandiendo la exponencial:
  \begin{equation}
    \label{eq:Stirling-asymptotic}
    n!
      = \sqrt{2 \pi n} \, \left( \frac{n}{e} \right)^n
	  \cdot \left(1
			+ \frac{1}{12 n}
			+ \frac{1}{288 n^2}
			- \frac{139}{51840 n^3}
			- \frac{571}{2488320 n^4}
			+ O(n^{-5})
		\right)
  \end{equation}
  Truncado en el primer término,
  queda:
  \begin{equation}
    \index{Stirling, formula de@Stirling, fórmula de|textbfhy}
    \label{eq:Stirling}
    n!
     \approx \sqrt{2 \pi n} \, \left( \frac{n}{e} \right)^n
  \end{equation}
  La ecuación~\eqref{eq:Stirling}
  es la fórmula de Stirling para el factorial.
  La aproximación dada antes dice que
  ya para \(n \ge 8\) el error es de cerca de \(1\)\%,
  cosa que cálculo directo confirma.

  Nótese que \(\sigma = \sqrt{2 \pi} = 2,5066\),
  nuestra burda aproximación \(\sigma \approx e\) no era tan mala.

  Falta el valor de \(\ln \sigma\).
  Por el producto de Wallis%
    \index{Wallis, producto de}
  (lo demostraremos más adelante):
  \begin{align*}
    \frac{\pi}{2}
      &= \prod_{k \ge 1}
	   \frac{2 k}{2 k - 1} \cdot \frac{2 k}{2 k + 1} \\
      &= \lim_{n \rightarrow \infty}
	   \prod_{1 \le k \le n}
	     \left(
	       \frac{2 k}{2 k - 1} \cdot \frac{2 k}{2 k}
	     \right) \cdot
	     \left(
	       \frac{2 k}{2 k} \cdot \frac{2 k}{2 k + 1}
	     \right) \\
      &= \lim_{n \rightarrow \infty}
	   \frac{(2 n!)^4}{(2 n)! (2 n + 1)!} \\
      &= \lim_{n \rightarrow \infty}
	   \frac{1}{2 n + 1} \cdot
	     \frac{2^{4 n} \, n!^4}{(2 n)!^2}
  \end{align*}
  Substituyendo la aproximación~\eqref{eq:Stirling-1}
  para los factoriales:
  \begin{align*}
    \frac{\pi}{2}
      &= \lim_{n \rightarrow \infty}
	   \frac{1}{2 n + 1} \cdot
	     \frac{2^{4 n} \sigma^4 n^2 \left( n / e \right)^{4 n}}
		  {\sigma^2 (2 n) \left( 2 n / e \right)^{4 n}} \\
    \sigma^2
      &= 2 \pi
  \end{align*}

  Para completar,
  demostraremos el producto de Wallis,
  siguiendo a Lynn~%
    \cite{lynn:_wallis_product}.
  \begin{theorem}[Producto de Wallis]
    \index{Wallis, producto de|textbfhy}
    \label{theo:producto-Wallis}
    Tenemos:
    \begin{equation*}
      \frac{\pi}{2}
	= \prod_{k \ge 1}
	    \frac{2 k}{2 k - 1} \cdot \frac{2 k}{2 k + 1}
    \end{equation*}
  \end{theorem}
  \begin{proof}
    Definamos:
    \begin{equation*}
      a_n
	= \int_0^{\pi / 2} \sin^n z \, \mathrm{d} z
    \end{equation*}
    Como \(0 \le \sin z \le 1\) en el rango \(0 \le z \le \pi / 2\),
      \(\left\langle a_n \right\rangle_{n \ge 0}\)
    es una secuencia positiva y monótona decreciente.
    Integrando por partes:
    \begin{align*}
      a_n
	&= -\sin^{n - 1} z \, \cos z \,
	   \biggr|_0^{\pi / 2}
	   + \int_0^{\pi / 2}
	       (n - 1) \sin^{n - 2} z \, \cos^2 z \, \mathrm{d} z \\
	&= \int_0^{\pi / 2}
	     (n - 1) \sin^{n - 2} z \, (1 - \sin^2 z)
	     \, \mathrm{d} z
    \end{align*}
    O sea \(a_n = (n - 1) a_{n - 2} - (n - 1) a_n\),
    que resulta en:
    \begin{equation}
      \label{eq:Wallis-recurrence}
      a_n
	= \frac{n - 1}{n} \, a_{n - 2}
    \end{equation}
    Por el otro lado,
    directamente tenemos \(a_0 = \pi / 2\) y \(a_1 = 1\).
    Con estos puntos de partida en~\eqref{eq:Wallis-recurrence}:
    \begin{align}
      a_{2 n}
	&= \frac{\pi}{2} \cdot \frac{1}{2} \cdot \dotsm
	     \cdot \frac{2 n - 1}{2 n} \label{eq:Wallis-an-even} \\
      a_{2 n + 1}
	&= \frac{2}{3} \cdot \frac{4}{5} \cdot \dotsm
	     \cdot \frac{2 n}{2 n + 1} \label{eq:Wallis-an-odd}
    \end{align}
    Como la secuencia \(a_n\) es decreciente,
    \(a_{2 n + 1} \le a_{2 n} \le a_{2 n - 1}\),
    y de la recurrencia~\eqref{eq:Wallis-recurrence}:
    \begin{equation*}
      1 \le \frac{a_{2 n}}{a_{2 n + 1}}
	\le \frac{a_{2 n - 1}}{a_{2 n + 1}}
	= 1 + \frac{1}{2 n}
    \end{equation*}
    En consecuencia:
    \begin{equation}
      \label{eq:Wallis-limit}
      \lim_{n \rightarrow \infty} \frac{a_{2 n}}{a_{2 n + 1}}
	= 1
    \end{equation}
    y por~\eqref{eq:Wallis-limit}
    con~\eqref{eq:Wallis-an-even} y~\eqref{eq:Wallis-an-odd}:
    \begin{equation*}
      \lim_{n \rightarrow \infty}
	\frac{\pi}{2} \cdot
	  \frac{1 \cdot 3 \dotsm (2 n - 1)}
	       {2 \cdot 4 \dotsm (2 n)} \cdot
	  \frac{3 \cdot 5 \dotsm (2 n + 1)}
	       {2 \cdot 4 \dotsm (2 n)}
	= 1
    \end{equation*}
    que es equivalente a lo planteado.
  \end{proof}
  Una muy bonita demostración alternativa
  (originalmente de Euler,%
     \index{Euler, Leonhard}
   quien de forma similar
   obtuvo muchos otros resultados sorprendentes)
  es la siguiente.
  Tiene el problema de basarse en la fórmula de Euler para el seno,
  que es muy sugestiva pero no es sencilla de demostrar.
  \begin{proof}
    La fórmula de Euler para el seno
    resulta de considerar esta función impar
    como un ``polinomio infinito'' con ceros \(0\) y \(\pm n \pi\),
    como también:
    \begin{equation*}
      \lim_{z \rightarrow 0} \frac{\sin z}{z}
	= 1
    \end{equation*}
    el coeficiente de \(z\) debe ser \(1\),
    por lo que puede expresarse:%
      \index{Euler, formula para seno@Euler, fórmula para seno}
    \begin{align*}
      \frac{\sin z}{z}
	&= \left( 1 - \frac{z^2}{\pi^2} \right)
	     \left( 1 - \frac{z^2}{4 \pi^2} \right)
	     \left( 1 - \frac{z^2}{9 \pi^2} \right) \dotsm \\
	&= \prod_{k \ge 1}
	     \left( 1 - \frac{z^2}{k^2 \pi^2} \right)
    \end{align*}
    Notamos que para \(z = \pi / 2\):
    \begin{align*}
      \frac{2}{\pi}
	&= \prod_{k \ge 1} \left( 1 - \frac{1}{4 k^2} \right) \\
      \frac{\pi}{2}
	&= \prod_{k \ge 1} \left( \frac{4 k^2}{4 k^2 - 1} \right) \\
	&= \prod_{k \ge 1} \frac{(2 k) (2 k)}{(2 k - 1) (2 k + 1)}
      \qedhere
    \end{align*}
  \end{proof}

\section{Propiedades de los polinomios
       y números de Bernoulli}
\label{sec:propiedades-Bernoulli}

  Derivaremos algunas propiedades adicionales
  de los polinomios y números de Bernoulli.%
    \index{Bernoulli, polinomios de}%
    \index{Bernoulli, numeros de@Bernoulli, números de}
  Algunas ya las usamos,
  otras las necesitaremos más adelante.
  En el proceso mostraremos algunas técnicas útiles
  para obtener información sobre los coeficientes de una serie.%
    \index{serie de potencias!coeficientes}

  De la recurrencia~\eqref{eq:Bn}:
  \begin{equation*}
    B'_n(z)
     = n B_{n - 1}(z)
  \end{equation*}
  Con esto,
  à la Maclaurin en \(y\):
  \begin{align}
    B_n(z + y)
      &= B_n(z)
	  + n B_{n - 1}(z) \, y
	  + \frac{n (n - 1)}{2} B_{n - 2}(z) \, y^2
	  + \dotsb \notag \\
      &= \sum_{0 \le k \le n} \binom{n}{k} B_{n - k}(z) \, y^k
	    \label{eq:Bn(z+y)}
  \end{align}
  Si ahora hacemos \(z = 0\) en~\eqref{eq:Bn(z+y)},
  recordamos \(B_n(0) = B_n\)
  y cambiamos variables \(y \leadsto z\) en el resultado:
  \begin{equation}
    \label{eq:Bn(z)-expansion}
    B_n(z)
      = \sum_{0 \le k \le n} \binom{n}{k} B_{n - k} \, z^k
  \end{equation}
  Para \(z = 1\),
  como \(B_n(1) = B_n\)
  salvo para \(n = 1\),
  resulta:
  \begin{equation}
    \label{eq:Bn-expansion}
    B_n
      = \sum_{0 \le k \le n} \binom{n}{k} B_k
  \end{equation}
  Si en~\eqref{eq:Bn-expansion}
  interpretamos \(\mathbf{B}^n\) como \(B_n\)
  tenemos la linda fórmula:%
    \index{Bernoulli, numeros de@Bernoulli, números de!linda formula@linda fórmula}
  \begin{equation}
    \label{eq:B-linda-formula}
    \mathbf{B}^n
      = (1 + \mathbf{B})^n
  \end{equation}
  En la linda fórmula~\eqref{eq:B-linda-formula}
  para \(B_{n + 1}\) se cancelan los \(B_{n + 1}\),
  y puede despejarse \(B_n\)
  dando la relación válida para \(n \ge 1\):
  \begin{equation*}
    B_n
      = - \frac{1}{n + 1}
	    \sum_{0 \le k \le n - 1} \binom{n + 1}{k} \, B_k
  \end{equation*}

  Por los factoriales definamos una función generatriz exponencial
  para los polinomios \(B_n(x)\)
  (estamos trabajando con series sobre el anillo \(\mathbb{Q}[x]\)):
  \begin{equation}
    \label{eq:definicion-B(x,z)}
    B(x, z)
      = \sum_{n \ge 0} B_n(x) \, \frac{z^n}{n!}
  \end{equation}
  De la recurrencia~\eqref{eq:Bn} para los polinomios:
  \begin{align}
    \frac{\partial B(x, z)}{\partial x}
      &= \sum_{n \ge 1} B'_n(x) \, \frac{z^n}{n!} \notag \\
      &= z \sum_{n \ge 1}
	     B_{n - 1}(x) \, \frac{z^{n - 1}}{(n - 1)!} \notag \\
      &= z B(x, z) \label{eq:B(x,z)}
  \end{align}
  La ecuación~\eqref{eq:B(x,z)} indica
  que para alguna función \(c(z)\) que no depende de \(x\):
  \begin{equation}
    \label{eq:B(x,z)-2}
    B(x, z)
      = c(z) \mathrm{e}^{x z}
  \end{equation}
  Usamos ahora la otra condición sobre los polinomios.
  Debe cumplirse:
  \begin{align*}
    \int_0^1 B(x, z) \, \mathrm{d} x
      &= \sum_{n \ge 0}
	   \frac{z^n}{n!} \, \int_0^1 B_n(x) \, \mathrm{d} x \\
      &= 1
  \end{align*}
  De nuestra expresión~\eqref{eq:B(x,z)-2} para \(B(x, z)\):
  \begin{align*}
    \int_0^1 B(x, z) \, \mathrm{d} x
      &= c(z) \int_0^1 \mathrm{e}^{x z} \, \mathrm{d} x \\
      &= c(z) \frac{1}{z} \, \left( \mathrm{e}^z - 1 \right)
  \end{align*}
  Comparando ambas expresiones para la integral obtenemos \(c(z)\),
  y finalmente:
  \begin{equation}
    \index{Bernoulli, polinomios de!generatriz}
    \label{eq:B(x,z)-3}
    B(x, z)
      = \sum_{n \ge 0} B_n(x) \, \frac{z^n}{n!}
      = \frac{z \mathrm{e}^{x z}}{\mathrm{e}^z - 1}
  \end{equation}
  Para justificar algunas de las manipulaciones que siguen,
  debemos asegurarnos
  que esta serie converge uniformemente para \(0 \le z \le 1\)%
    \index{convergencia uniforme}
  (no hay problemas con \(x\)).
  La función~\eqref{eq:B(x,z)-3} en \(z = 0\)
  tiene una singularidad removible,%
    \index{singularidad!removible}
  y tiene polos en \(z = \pm 2 n \pi \mathrm{i}\) para \(n \ge 1\),%
    \index{polo}
  por lo que el radio de convergencia es \(2 \pi > 1\).%
    \index{serie de potencias!radio de convergencia}
  Para detalles de estos conceptos
  véase el capítulo~\ref{cha:analisis-complejo}.%
    \index{analisis complejo@análisis complejo}

  Con \(x = 0\) obtenemos la función generatriz
  de los coeficientes \(B_n = B_n(0)\):%
    \index{Bernoulli, numeros de@Bernoulli, números de!generatriz}
  \begin{equation}
    \label{eq:B(0,z)}
    B(0, z)
      = \sum_{n \ge 0} B_n \, \frac{z^n}{n!}
      = \frac{z}{\mathrm{e}^z - 1}
  \end{equation}

  De los valores dados antes
  pareciera ser que los valores para índices impares son todos cero,
  salvo \mbox{\(B_1 = - 1 / 2\)}.
  Consideremos la función:
  \begin{align*}
    \frac{z}{\mathrm{e}^z - 1} + \frac{z}{2}
      &= \frac{z}{2}
	   \cdot \frac{\mathrm{e}^z + 1}{\mathrm{e}^z - 1} \\
      &= \frac{z}{2} \, \coth \frac{z}{2}
  \end{align*}
  Esta función es par,
  confirmando nuestra sospecha.

  Anotamos el resultado siguiente
  en términos de la función \(\zeta\) de Riemann:%
    \index{Riemann, funcion \(\zeta\) de@Riemann, función \(\zeta\) de|textbfhy}%
    \index{\(\zeta\) de Riemann|see{Riemann, función \(\zeta\) de}}
  \begin{equation}
    \label{eq:zeta}
    \zeta(z)
      = \sum_{n \ge 1} n^{-z}
  \end{equation}
  Uno de los resultados más sensacionales de Euler
  fue la solución en 1734 del problema de Basilea,%
    \index{Basilea, problema de|textbfhy}%
    \index{Euler, Leonhard}
  que venía siendo un tema recurrente desde 1650.
  Se buscaba el valor de la serie:
  \begin{equation}
    \label{eq:zeta(z)}
    \zeta(2)
      = \sum_{n \ge 1} \frac{1}{n^2}
  \end{equation}
  Hizo mucho más que esto,
  hallando los valores de \(\zeta(2 k)\) para \(k\) hasta \(13\).
  Luego halló la fórmula general,
  para la que hay hermosas demostraciones
  (ver Aigner y Ziegler~%
    \cite{aigner14:_proof_the_book}
   o Kalman~%
    \cite{kalman93:_six_ways_sum_series}).
  Podemos seguir uno de los razonamientos de Euler
  (Dunham~\cite{dunham09:_when_euler_met_lhopital}
   da otras de las demostraciones
   y algo del sabor del trabajo original)
  como sigue.

  Por la fórmula de Euler para la exponencial de un complejo:%
    \index{Euler, formula de (exponencial complejo)@Euler, fórmula de (exponencial complejo)}
  \begin{align}
    \cot z
      &= \frac{\cos z}{\sin z} \notag \\
      &= \mathrm{i} \, \frac{\mathrm{e}^{\mathrm{i} z}
			     + \mathrm{e}^{- \mathrm{i} z}}
		{\mathrm{e}^{\mathrm{i} z}
		  - \mathrm{e}^{- \mathrm{i} z}}
	     \notag \\
      &= \mathrm{i} + \frac{2 \mathrm{i}}
			   {\mathrm{e}^{2 \mathrm{i} z} - 1}
	     \notag \\
    \frac{z}{2} \, \cot \frac{z}{2}
      &= \frac{\mathrm{i} z}{2}
	   + \frac{\mathrm{i} z}{\mathrm{e}^{\mathrm{i} z} - 1}
	     \notag \\
    \intertext{Manejado los primeros dos términos
	       en forma especial queda:}
    \frac{z}{2} \, \cot \frac{z}{2}
      &= 1 + \sum_{k \ge 2} B_k \frac{(\mathrm{i} z)^k}{k!}
	   \label{eq:zcotz-1}
  \end{align}
  Por el otro lado tenemos la fórmula de Euler para el seno:%
    \index{Euler, formula para seno@Euler, fórmula para seno}
  \begin{equation}
    \label{eq:Euler-seno}
    \sin z
      = z \prod_{n \ge 1} \left( 1 - \frac{z^2}{n^2 \pi^2} \right)
  \end{equation}
  Aplicando \(z \mathrm{D} \log\) a~\eqref{eq:Euler-seno}:
  \begin{equation*}
    z \, \frac{\mathrm{d}}{\mathrm{d} z} \, \ln \sin z
      = z \, \frac{\cos z}{\sin z}
      = z \cot z
  \end{equation*}
  Con esto tenemos:
  \begin{align}
    z \cot z
      &= 1 - \sum_{n \ge 1} \frac{2 z^2}{1 - (z^2 / n^2 \pi^2)}
	  \notag \\
      &= 1 - 2 \sum_{n \ge 1}
		 \sum_{k \ge 0} \frac{z^{2 k + 2}}
				     {n^{2 k} \pi^{2 k}} \notag \\
      &= 1 - 2 \sum_{k \ge 0}
		 \frac{z^{2 k + 2}}{\pi^{2 k}}
		    \cdot \sum_{n \ge 1} \frac{1}{n^{2 k}} \notag \\
      &= 1 - 2 \sum_{k \ge 0} \frac{z^{2 k + 2} \zeta(2 k)}
				   {\pi^{2 k}}
						    \notag \\
    \frac{z}{2} \cot \frac{z}{2}
      &= 1 - 2 \sum_{k \ge 0}
		 \frac{z^{2 k + 2} \zeta(2 k)}
		      {2^{2 k + 2} \pi^{2 k}}
		    \label{eq:zcotz-2}
  \end{align}
  Comparando coeficientes de \(z^{2 k}\)
  entre~\eqref{eq:zcotz-1} y~\eqref{eq:zcotz-2} resulta:
  \begin{equation*}
    \zeta(2 k)
      = \frac{(-1)^{k + 1} 4^k \pi^{2 k} B_{2 k}}{2 (2 k)!}
  \end{equation*}
  Incidentalmente,
  esto demuestra que los números \(B_{2 k}\) alternan signo,
  ya que \(\zeta(2 k)\) claramente es positivo.
  Como \(\zeta(2 k) \sim 1\),
  usando la fórmula de Stirling~\eqref{eq:Stirling}:%
    \index{Stirling, formula de@Stirling, fórmula de}%
    \index{Bernoulli, numeros de@Bernoulli, números de!asintotica@asintótica|textbfhy}
  \begin{align}
    B_{2 k}
      &\sim (-1)^{k + 1} \frac{2 (2 k)!}{4^k \pi^{2 k}} \notag \\
      &\sim (-1)^{k + 1} \, 4 \sqrt{k \pi}
	      \left( \frac{k}{\pi e} \right)^{2 k}
		    \label{eq:Bernoulli-approximation}
  \end{align}
  Con este crecimiento de \(B_{2 k}\)
  las derivadas de \(f\) deben disminuir muy rápidamente
  para que la fórmula de Euler-Maclaurin converja.%
    \index{Euler-Maclaurin, formula de@Euler-Maclaurin, fórmula de!convergencia}

  La figura~\ref{fig:Bernoulli}
  grafica algunos polinomios de Bernoulli
  en el rango que nos interesa.
  \begin{figure}[htbp]
    \newlength{\ten}
    \settowidth{\ten}{\(10\)}
    \begin{tabular}{>{\raggedleft}m{\ten}@{\hspace{0.5ex}}
		     *{4}{@{}m{0.232\linewidth}}}
	     & \multicolumn{1}{c}{\(m\)}
	     & \multicolumn{1}{c}{\(m + 1\)}
	     & \multicolumn{1}{c}{\(m + 2\)}
	     & \multicolumn{1}{c}{\(m + 3\)} \\
       \(2\) & \pgfimage[width=\linewidth]{images/B2}
	     & \pgfimage[width=\linewidth]{images/B3}
	     & \pgfimage[width=\linewidth]{images/B4}
	     & \pgfimage[width=\linewidth]{images/B5} \\
       \(6\) & \pgfimage[width=\linewidth]{images/B6}
	     & \pgfimage[width=\linewidth]{images/B7}
	     & \pgfimage[width=\linewidth]{images/B8}
	     & \pgfimage[width=\linewidth]{images/B9} \\
      \(10\) & \pgfimage[width=\linewidth]{images/B10}
	     & \pgfimage[width=\linewidth]{images/B11}
	     & \pgfimage[width=\linewidth]{images/B12}
	     & \pgfimage[width=\linewidth]{images/B13}
    \end{tabular}
    \caption{Polinomios de Bernoulli en $[0, 1]$
	     (escalados de mínimo a máximo)}
    \label{fig:Bernoulli}%
    \index{Bernoulli, polinomios de!grafica@gráfica}
  \end{figure}
  Pareciera ser que \(B_{2 k}(z)\)
  es simétrica alrededor de \(1 / 2\),
  mientras \(B_{2 k + 1}(z)\) es antisimétrica.
  Para demostrar estos hechos
  consideramos:
  \begin{align*}
    B \left( 1 / 2 + u, z \right)
	+ B \left( 1 / 2 - u, z \right)
      &= \frac{z \mathrm{e}^{z (1 / 2 + u)}}{\mathrm{e}^z - 1}
	   + \frac{z \mathrm{e}^{z (1 / 2 - u)}}
		  {\mathrm{e}^z - 1} \\
      &= \frac{z \mathrm{e}^{z / 2}}{\mathrm{e}^z - 1}
	   \cdot \left(
		   \mathrm{e}^{z u} + \mathrm{e}^{- z u}
		 \right)
  \end{align*}
  Esta expresión es par en \(z\):
  \begin{align*}
    \frac{-z \mathrm{e}^{-z / 2}}{\mathrm{e}^{-z} - 1}
       \cdot \frac{\mathrm{e}^z}{\mathrm{e}^z}
      &= \frac{-z \mathrm{e}^{z / 2}}{1 - \mathrm{e}^z} \\
      &= \frac{z \mathrm{e}^{z / 2}}{\mathrm{e}^z - 1}
  \end{align*}
  Esto significa que los términos para \(z^{2 k + 1}\) se anulan:
  \begin{equation*}
    B_{2 k + 1}(1 / 2 - u)
      = - B_{2 k + 1}(1 / 2 + u) \\
  \end{equation*}
  En particular, \(B_{2 k + 1}(1 / 2) = 0\).

  De forma similar:
  \begin{align*}
    B \left( 1 / 2 + u, z \right)
	- B \left( 1 / 2 - u, z \right)
      &= \frac{z \mathrm{e}^{z (1 / 2 + u)}}{\mathrm{e}^z - 1}
	   - \frac{z \mathrm{e}^{z (1 / 2 - u)}}
		  {\mathrm{e}^z - 1} \\
      &= \frac{z \mathrm{e}^{z / 2}}{\mathrm{e}^z - 1}
	   \cdot \left(
		   \mathrm{e}^{z u} - \mathrm{e}^{- z u}
		 \right)
  \end{align*}
  Esta expresión es impar en \(z\),
  lo que significa que ahora se anularon los términos pares:
  \begin{equation*}
    B_{2 k}(1 / 2 - u)
      = B_{2 k}(1 / 2 + u)
  \end{equation*}

  De las gráficas~\ref{fig:Bernoulli}
  en el rango \([0, 1]\)
  se ve que el polinomio \(B_{2 k}(z)\) tiene dos ceros,
  mientras \(B_{2 k + 1}(z)\) tiene tres
  (0 y \(\pm 1\)).
  Esto vale en general.%
    \index{Bernoulli, polinomios de!ceros}
  \begin{theorem}
    \label{theo:Bernoulli-zeros}
    Para \(k > 0\),
    en el rango \([0, 1]\) el polinomio \(B_{2 k}(z)\)
    tiene exactamente dos ceros,
    mientras \(B_{2 k + 1} (z)\) tiene exactamente tres
    (\(0\), \(1 / 2\) y	 \(1\)).
  \end{theorem}
  \begin{proof}
    Demostramos por inducción que para \(k \ge 1\) en \([0, 1 / 2]\)
    el polinomio \(B_{2 k} (z)\) tiene exactamente un cero,%
      \index{demostracion@demostración!induccion@inducción}
    y que \(B_{2 k + 1}(z)\) no cambia de signo
    y se anula únicamente en los extremos.
    De partida,
    ninguno de los polinomios es idénticamente cero.
    \begin{description}
    \item[Base:]
      Para \(k = 1\) tenemos \(B_2(z) = z^2 - z + 1 / 6\)
      con ceros \(1 / 2 \pm \sqrt{3} / 6\)
      (estos dos están en el rango  \([0, 1]\),
       hay uno en \([0, 1/2]\)),
      y \(B_3(z) = z^3 - 3 z^2 / 2 + z / 2\) con ceros \(0, \pm 1\).
    \item[Inducción:]
      Supongamos que \(B_{2 k}(z)\)
      tiene un único cero en \([0, 1 / 2]\),
      y que \(B_{2 k + 1}(z)\)
      se anula únicamente en \(0\) y \(1 /2\).
      Debemos demostrar que vale para \(B_{2 k + 2}(z)\)
      y \(B_{2 k +3}(z)\) también.

      Como \(B_{2 k + 1}(z)\) no cambia de signo en el rango,
      \(B_{2 k + 2}(z)\) es monótona
      y por tanto puede tener a lo más un cero.
      Pero pusimos como condición
      que la integral de \(B_{2 k + 2}(z)\)
      de \(0\) a \(1\) se anule;
      como \(B_{2 k + 2}(z)\)
      es simétrica alrededor de \(z = 1 / 2\)
      se anula la integral de \(0\) a \(1 / 2\),
      por lo que deben haber valores positivos y negativos
      en el rango,
      y hay exactamente un cero en él.

      Por la recurrencia \(B'_n (z) = n B_{n - 1} (z)\),
      al tener un único cero \(B_{2 k + 2} (z)\),
      \(B_{2 k + 3}(z)\) tiene un único mínimo o máximo
      en el rango \([0, 1/2]\),
      y \(B_{2 k + 3}(z)\) puede tener a lo más dos ceros allí
      y conocemos dos (0 y \(1 / 2\)).
    \end{description}
    Por inducción vale para todo \(k \ge 1\).
  \end{proof}

  Como \(B'_{2 k}(1 / 2) = 2 k B_{2 k - 1} (1 / 2) = 0\),
  sabemos que \(B_{2 k}(z)\)
  tiene un mínimo o máximo en \(z = 1 / 2\).
  Tenemos:
  \begin{align*}
    B(0, z / 2)
      &= \frac{z}{2 (\mathrm{e}^{z / 2} - 1)} \\
      &= \frac{1}{2} \cdot \frac{z}{\mathrm{e}^{z / 2} - 1}
	   \cdot \frac{\mathrm{e}^{z / 2} + 1}
		      {\mathrm{e}^{z / 2} + 1} \\
      &= \frac{1}{2}
	   \cdot \frac{z (\mathrm{e}^{z / 2} + 1)}
		      {\mathrm{e}^z - 1} \\
      &= \frac{1}{2} \,
	   \left(
	     \frac{z \mathrm{e}^{z / 2}}{\mathrm{e}^z - 1}
	       - \frac{z}{\mathrm{e}^z - 1}
	   \right) \\
      &= \frac{1}{2} \,
	   \left(B(1 / 2, z) - B(0, z) \right)
  \end{align*}
  Comparando coeficientes de \(z^k\):
  \begin{align*}
    \frac{B_k}{2^k}
      &= \frac{1}{2} \, \left( B_k(1 / 2) - B_k \right) \\
    B_k(1 / 2)
      &= - \left( 1 - 2^{1 - k} \right) \, B_k
  \end{align*}
  Como \(z = 1 / 2\) es el único máximo (mínimo) de \(B_{2 k} (z)\)
  en el rango \([0, 1]\) por ser monótona en \([0, 1 / 2]\),
  al ser simétrica alrededor de \(1 / 2\) los mínimos (máximos)
  se dan en los extremos,
  y en este rango:
  \begin{equation*}
    \lvert B_{2 k}(z) \rvert
      \le \lvert B_{2 k} \rvert
  \end{equation*}

\section{El resto}
\label{sec:resto-Euler-Maclaurin}

  Nuestra fórmula maestra es:
  \begin{equation*}
    \sum_{1 \le k < a} f(k)
      = \int_1^a f(z) \, \mathrm{d} z
	  + \gamma_f
	  + B_1 f(a)
	  + \sum_{1 \le k \le n}
	       \frac{B_{2 k}}{(2 k)!} \, f^{(2 k - 1)}(a)
	  + \int_a^\infty
	      \frac{\widetilde{B}_{2 n + 1}(z)}{(2 n + 1)!} \,
			      f^{(2 n + 1)}(z) \, \mathrm{d} z
  \end{equation*}
  Interesa acotar la integral que determina el resto,
  la llamaremos \(R_n(f; a)\).
  Podemos volver a integrar por partes:
  \begin{equation*}
    R_n(f; a)
      = \int_a^\infty \frac{\widetilde{B}_{2 n + 2}(z)}
			   {(2 n + 2)!} \,
			      f^{(2 n + 2)}(z) \, \mathrm{d} z
  \end{equation*}
  Suponiendo que \(f^{(2 n + 2)}(z)\) y \(f^{(2 n + 1)}(z)\)
  tienden monótonamente a cero
  (con lo que en particular no cambian signo),
  la integral queda acotada
  por el valor extremo de \(\widetilde{B}_{2 n + 2}(z)\)
  y la integral del segundo factor.
  Para el valor extremo de \(\widetilde{B}_{2 n + 2}(z)\)
  tenemos la cota \(\lvert B_{2 n + 2} \rvert\):%
    \index{Euler-Maclaurin, formula de@Euler-Maclaurin, fórmula de!resto|textbfhy}
  \begin{align*}
    \lvert R_n(f; a) \rvert
      &\le \frac{\lvert B_{2 n + 2} \rvert}{(2 n + 2)!}
	      \cdot \left\lvert
		      \int_a^\infty f^{(2 n + 2)}(z)
			\, \mathrm{d} z
		    \right\rvert \\
      &= \frac{\lvert B_{2 n + 2} \rvert}{(2 n + 2)!}
	   \cdot \lvert f^{(2 n + 1)}(a) \rvert
  \end{align*}
  Esto es del orden del primer término omitido,
  como se indicó.

%%% Local Variables:
%%% mode: latex
%%% TeX-master: "clases"
%%% End:

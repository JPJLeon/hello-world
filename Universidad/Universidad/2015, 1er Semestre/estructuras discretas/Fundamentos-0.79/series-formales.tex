% series-formales.tex
%
% Copyright (c) 2009, 2012-2014 Horst H. von Brand
% Derechos reservados. Vea COPYRIGHT para detalles

\chapter{Series formales de potencias}
\label{cha:series-formales}
\index{serie formal|textbfhy}

  Nuestro interés en las series de potencias%
    \index{serie de potencias}
  no es en su capacidad de definir funciones,
  sino simplemente como una representación compacta
  de una secuencia infinita.
  El desarrollo de la teoría de series formales
  nació de la observación que ciertas manipulaciones de series
  ``como si fueran polinomios''
  entregaban resultados correctos,
  incluso cuando una revisión más detallada demostraba
  que las operaciones no tenían validez.
  Resumiremos los resultados más importantes del área,
  que corroboran nuestras manipulaciones,
  a primera vista irresponsables y sin justificación,
  en los capítulos anteriores.
  Incluso veremos que las manipulaciones pueden justificarse
  si los coeficientes de la serie pertenecen a un anillo,
  no necesariamente son números reales o complejos.
  Esto es notable,
  estamos usando sumas infinitas
  en ámbitos en los cuales el concepto de límite
  necesario para justificar convergencia
  no es aplicable directamente.

\section{Un primer ejemplo}
\label{sec:un-primer-ejemplo}

  Si dejamos de lado el requerimiento de que la serie converja
  (y defina una función),
  podemos darle sentido incluso a series como:
  \begin{equation}
    \label{eq:gf-n!}
    f(z)
      =\sum_{n \ge 0} n! z^n
  \end{equation}
  que solo convergen para \(z = 0\),
  y para las que el análisis no tiene ningún uso.
  Podemos considerar la serie~\eqref{eq:gf-n!}
  como la función generatriz ordinaria de los factoriales.%
    \index{generatriz!ordinaria}%
    \index{factorial}
  Así:
  \begin{align*}
    z f(z)
      &= \sum_{n \ge 0} n! z^{n + 1} \\
    \mathrm{D}(z f(z))
      &= \sum_{n \ge 0} (n + 1)! z^n
       = \frac{f(z) - 1}{z} \\
  \intertext{Por el otro lado, derivando el producto:}
    \mathrm{D}(z f(z))
      &= f(z) + z f'(z) \\
    z f'(z)
      &= \frac{f(z) - 1}{z} - f(z) \\
  \intertext{El lado derecho es la función generatriz de \((n + 1)! - n!\),
	     e invita a sumar
	     (dividir por \(1 - z\) en funciones generatrices):}
    \frac{z f'(z)}{1 - z}
      &= \frac{1}{1 - z} \,
	   \left(
	     \frac{f(z) - 1}{z} - (f(z) - 1) - 1
	   \right) \\
      &= \frac{1}{1 - z} \,
	   \left(
	     (f(z) - 1) \, \left( \frac{1}{z} - 1 \right)
	   \right)
	   - \frac{1}{1 - z} \\
      &= \frac{1}{1 - z} \,
	   \left(
	     (f(z) - 1) \, \frac{1 - z}{z}
	   \right)
	   - \frac{1}{1 - z} \\
      &= \frac{f(z) - 1}{z} - \frac{1}{1 - z}
  \end{align*}
  Como:
  \begin{align*}
    z f'(z)
      &\ogf \langle n \, n! \rangle_{n \ge 0} \\
    \frac{z f'(z)}{1 - z}
      &\ogf \left\langle
	      \sum_{0 \le k \le n} k \cdot k!
	    \right\rangle_{n \ge 0}
  \end{align*}
  Aplicando nuevamente las propiedades,
  vemos que:
  \begin{equation}
    \label{eq:sn}
    \sum_{\mathclap{0 \le k \le n}} k \cdot k!
      = (n + 1)! - 1
  \end{equation}
  A pesar de su espeluznante derivación
  (no falta un paso en que no hagamos operaciones al menos dudosas
   con series infinitas que solo para \(z = 0\) convergen)
  la relación~\eqref{eq:sn} es correcta,
  cosa que el lector escéptico demostrará por inducción.%
    \index{demostracion@demostración!induccion@inducción}
  Resulta que estas operaciones pueden justificarse rigurosamente,
  tema que nos ocupará en este capítulo.

  Parte de lo que sigue viene de Shoup~%
    \cite{shoup09:_comput_introd_number_theor_algeb},
  las justificaciones siguen a Kauers~%
    \cite{kauers11:_concr_tetrah}.
  Operaciones con series pueden efectuarse con paquetes de álgebra simbólica,%
    \index{algebra simbolica@álgebra simbólica}
  como \texttt{maxima}~\cite{maxima14b:_computer_algebra},%
    \index{maxima@\texttt{maxima}}
  o aún mejor con sistemas especializados como \texttt{PARI/GP}~%
    \cite{PARI:2.7.2}.%
    \index{PARI/GP@\texttt{PARI/GP}}
  La biblioteca GiNaC~%
    \cite{bauer02:_ginac_fram_symbol_comput, GiNaC:1.6.2}%
    \index{GiNaC@\texttt{GiNaC}}
  permite manipular expresiones simbólicas,
  incluyendo series formales,
  y numéricas
  directamente en \cplusplus.%
    \index{C++ (lenguaje de programacion)@\cplusplus{} (lenguaje de programación)}

\section{Definición de serie formal}
\label{sec:serie-formal-def}

  Sea la serie:
  \begin{equation*}
    \sum_{n \ge 0} a_n z^n
  \end{equation*}
  donde los elementos \(a_n\) pertenecen a un anillo \(R\).%
    \index{anillo}

  Acá como en polinomios formales
  (capítulo~\ref{cha:anillos-polinomios})
  \(z\) es simplemente un \emph{símbolo}
  (también llamado \emph{indeterminada} o \emph{variable}).
  La consideramos como una construcción puramente formal,
  sin darle sentido a \(z\)
  ni preocuparse por convergencia.%
    \index{serie de potencias!convergencia}
  La única restricción que impone esto
  es que toda vez que se calcula un coeficiente
  deben efectuarse un número finito de operaciones
  (en un anillo arbitrario no son aplicables
   las ideas de límite y convergencia,
   necesarias para darle sentido a un número infinito de operaciones).
  Trataremos el caso en que \(R\) es un campo,%
    \index{campo (algebra)@campo (álgebra)}
  o al menos un dominio integral%
    \index{dominio integral}
  (un anillo conmutativo sin divisores de cero distintos de cero).
  Para evitar tener que mencionarlo infinidad de veces,
  usaremos la convención que \(R\) es un anillo general,
  \(D\) es un dominio integral
  y \(F\) es un campo.

  Definimos la igualdad entre series formales sobre el anillo \(R\):%
    \index{serie formal!igualdad}
  \begin{equation*}
    \sum_{n \ge 0} a_n z^n
       = \sum_{n \ge 0} b_n z^n \quad
	     \text{cuando \(a_n = b_n\) para todo \(n \ge 0\)}
  \end{equation*}
  Definimos además las operaciones:%
    \index{serie formal!operaciones}
  \begin{align*}
    \alpha \sum_{n \ge 0} a_n z^n
       &= \sum_{n \ge 0} (\alpha a_n) z^n
	    \qquad \text{para \(\alpha \in R\) o \(\alpha \in \mathbb{Z}\)} \\
    \sum_{n \ge 0} a_n z^n + \sum_{n \ge 0} b_n z^n
       &= \sum_{n \ge 0} (a_n + b_n) z^n \\
    \biggl( \, \sum_{r \ge 0} a_r z^r \biggr) \cdot
      \biggl( \, \sum_{s \ge 0} b_s z^s \biggr)
       &= \sum_{\substack{
		  r \ge 0 \\
		  s \ge 0
	       }} a_r b_s z^{r + s} \\
       &= \sum_{\mathclap{\substack{
			    n \ge 0 \\
			    0 \le k \le n
	       }}} a_k b_{n - k} z^n \\
       &= \sum_{n \ge 0} \biggl( \,
			    \sum_{0 \le k \le n} a_k b_{n - k}
			  \biggr) z^n
  \end{align*}
  Notar que en particular,
  para constantes \(\alpha\) y \(\beta\):
  \begin{equation*}
    \alpha \sum_{n \ge 0} a_n z^n + \beta \sum_{n \ge 0} b_n z^n
      = \sum_{n \ge 0} (\alpha a_n + \beta b_n) z^n
  \end{equation*}
  Las series formales como generalmente usadas hasta acá
  son un espacio vectorial de dimensión infinita%
    \index{espacio vectorial}
  sobre el campo \(\mathbb{R}\)
  (con base \(\{z^k\}_{k \ge 0}\)),
  con la operación adicional de multiplicación.

  Es rutina verificar que las series formales de potencias
  sobre el dominio integral \(D\) con variable \(z\)
  son un dominio integral,
  con:
  \begin{align*}
    0 &= \sum_{n \ge 0} 0 \, z^n \\
    1 &= \sum_{n \ge 0} [n = 0] \, z^n
  \end{align*}
  Al anillo de series formales sobre el anillo \(R\)
  lo llamaremos \(R \llbracket z \rrbracket\)
  (recuérdese que llamamos \(R[z]\)
   al anillo de polinomios en \(z\) sobre \(R\)).
  Para evitar distinciones inútiles
  consideramos \(R[z]\)
  subanillo de \(R\llbracket z \rrbracket\) de la forma natural.

\section{Unidades y recíprocos}
\label{sec:series:unidades-reciprocos}
\index{serie formal!unidad}
\index{serie formal!reciproco@recíproco}

  Sea \(F\) un campo.
  En el anillo de series formales \(F \llbracket z \rrbracket\)
  hay unidades que no son simplemente constantes
  (como ocurre en el correspondiente anillo de polinomios formales).
  Por ejemplo,
  en \(\mathbb{C}\llbracket z \rrbracket\):
  \begin{equation*}
    (1 - z) \cdot (1 + z + z^2 + z^3 + \dotsb)
      = (1 + z + z^2 + \dotsb) - (z + z^2 + z^3 + \dotsb)
      = 1
  \end{equation*}
  Si \(a_0 = 0\) la serie no puede tener recíproco,
  ya que no hay forma
  de crear un término constante del producto en tal caso.

  Por otro lado,
  supongamos que \(a_0 \ne 0\):
  \begin{align*}
    \biggl( \, \sum_{n \ge 0} a_n z^n \biggr) \cdot
      \biggl( \, \sum_{n \ge 0} b_n z^n \biggr)
       &= 1 \\
    \sum_{n \ge 0} \biggl( \,
		     \sum_{0 \le k \le n} a_{n -k} b_k
		   \biggr) z^n
       &= 1
  \end{align*}
  Para \(n = 0\) debe ser \(a_0 b_0 = 1\),
  o sea,
  \(b_0 = 1 / a_0\);
  luego para \(n > 0\) las sumas deben anularse:
  \begin{equation*}
    b_n
      = -\frac{1}{a_0} \, \sum_{0 \le k \le n - 1} a_{n - k} b_k
  \end{equation*}
  Con esto último se obtiene la secuencia de todos los \(b_n\),
  que es la secuencia de coeficientes del recíproco de la serie \(A(z)\).

\section{Secuencias de series}
\label{sec:series-secuencias}
\index{serie formal!secuencia}

  Para justificar rigurosamente el operar con series formales
  debemos desarrollar el marco adecuado.
  En términos generales,
  las operaciones son válidas siempre que el cálculo de cada coeficiente
  involucre un número finito de operaciones.
  Por ejemplo,
  en la sección~\ref{sec:series:unidades-reciprocos}
  vimos que el \(n\)\nobreakdash-ésimo coeficiente
  del recíproco de una serie cuyo término constante no es cero
  es una combinación lineal de los coeficientes \(0\) al \(n - 1\)
  de la serie,
  una suma finita.
  En contraste,
  ``evaluar'' la serie \(A(z)\) en algún ``punto'' implica una suma infinita.
  Usaremos \(A(0)\) como una notación cómoda
  para \(\left[ z^0 \right] A(z)\),
  eso sí.

  Por otro lado,
  sí tiene sentido reemplazar \(z\)
  por \(z + z^2\) en la serie formal \(A(z)\)
  (note que en la serie que estamos reemplazando
   el coeficiente de \(z^0\) se anula).
  Observamos que:
  \begin{equation*}
    (z + z^2)^n
      = z^n (z + 1)^n
      = z^n \sum_{0 \le k \le n} \binom{n}{k} \, z^k
  \end{equation*}
  Para:
  \begin{equation*}
    A(z)
      = \sum_{n \ge 0} a_n z^n
  \end{equation*}
  resulta:
  \begin{equation*}
    A(z + z^2)
      = \sum_{n \ge 0} a_n z^n \sum_{0 \le k \le n} \binom{n}{k} \, z^k
      = \sum_{n \ge 0}
	  \left(
	    \sum_{0 \le k \le n} \binom{n - k}{k} \, a_{n - 1}
	  \right) z^n
  \end{equation*}
  Como cada coeficiente de la nueva serie
  se calcula con un número finito de operaciones,
  tenemos una serie formal perfectamente definida.

  Para generalizar esto,
  requerimos definir
  la noción de límite de secuencias en \(R \llbracket z \rrbracket\).
  Informalmente,
  consideramos dos series como ``cercanas'' si coinciden sus primeros términos.
  \begin{definition}
    El \emph{orden} de una serie,%
      \index{serie formal!orden|textbfhy}
    \(\ord A(z)\),
    es el índice del primer coeficiente no cero.
  \end{definition}
  \begin{definition}
    La secuencia de series
    \(\left\langle A_k(z) \right\rangle_{k \ge 0}\)
    \emph{converge} a la serie \(A(z)\) si%
      \index{serie formal!secuencia!convergencia|textbfhy}
    \begin{equation*}
      \lim_{k \rightarrow \infty} \ord (A(z) - A_k(z))
	= \infty
    \end{equation*}
    En tal caso
    escribimos \(\lim_{k \rightarrow \infty} A_k(z) =  A(z)\).
  \end{definition}
  Si \(a_{n k} = \left[ z^n \right] A_k(z)\),
  hay un número finito de \(a_{n k}\)
  que difiere de \(\left[ z^n \right] A(z)\),
  y en un número finito de operaciones podemos calcular \(a_n\).

  Algunas consecuencias son las siguientes.
  \begin{theorem}
    \label{theo:series-operaciones}
    Sean \(\left\langle A_n(z) \right\rangle_{n \ge 0}\)
    y \(\left\langle B_n(z) \right\rangle_{n \ge 0}\) secuencias de series
    que convergen a \(A(z)\) y \(B(z)\) en \(R \llbracket z \rrbracket\),
    respectivamente.
    Entonces:
    \begin{enumerate}
    \item
     \(\displaystyle \left\langle A_n(z) + B_n(z) \right\rangle_{n \ge 0}\)
     converge,
     y \(\displaystyle \lim_{n \rightarrow \infty} A_n(z) + B_n(z)
	    = A(z) + B(z)\)
   \item
     \(\displaystyle
	  \left\langle A_n(z) \cdot B_n(z) \right\rangle_{n \ge 0}\)
     converge,
     y \(\displaystyle \lim_{n \rightarrow \infty} A_n(z) \cdot B_n(z)
	    = A(z) \cdot B(z)\)
    \end{enumerate}
  \end{theorem}
  \begin{proof}
    La demostración es aplicar hechos simples como:
    \begin{align*}
      \ord (A(z) + B(z))
	&\ge \min \{ \ord A(z), \ord B(z) \} \\
      \ord (A(z) \cdot B(z))
	&\ge \ord A(z) + \ord B(z)
    \end{align*}
    Omitiremos los detalles.
  \end{proof}

  Para series en \(R \llbracket z \rrbracket\):
  \begin{equation*}
    A(z)
      = \sum_{n \ge 0} a_n z^n
    \hspace{3em}
    B(z)
      = \sum_{n \ge 0} b_n z^n
  \end{equation*}
  Si \(b_0 = 0\),
  \(\ord (B(z))^k \ge k\).
  Consideremos la secuencia:
  \begin{align*}
    C_0(z)
      &= a_0 \\
    C_1(z)
      &= a_0 + a_1 B(z) \\
    C_2(z)
      &= a_0 + a_1 B(z) + a_2 (B(z))^2 \\
      &\vdots \\
    C_k(z)
      &= a_0 + a_1 B(z) + a_2 (B(z))^2 + \dotsb + a_k (B(z))^k \\
  \end{align*}
  Los coeficientes de \(C_k(z)\) coinciden hasta el de orden \(k\)
  con todos los sucesores en la secuencia,
  luego esta converge a una serie \(C(z)\).%
    \index{serie formal!secuencia!convergencia}
  \begin{definition}
    \index{serie formal!composicion@composición|textbfhy}
    Definimos la \emph{composición} de las series \(A(z)\) y \(B(z)\),
    donde \(b_0 = 0\),
    como:
    \begin{equation*}
      A(B(z))
	= \sum_{n \ge 0} a_n (B(z))^n
	= \lim_{k \rightarrow \infty} C_k(z)
    \end{equation*}
  \end{definition}

  Un teorema importante es:
  \begin{theorem}
    \label{theo:series-composicion-homomorfismo}
    \index{anillo!homomorfismo}
    Para todo \(U(z) \in R \llbracket z \rrbracket\)
    con \(U(0) = 0\),
    el mapa
      \(\Phi_U \colon
	  R \llbracket z \rrbracket \rightarrow R \llbracket z \rrbracket\)
    definido mediante \(\Phi_U(A(z)) = A(U(z))\)
    es un homomorfismo de anillo.
  \end{theorem}
  \begin{proof}
    Sean:
    \begin{equation*}
      A(z)
	= \sum_{n \ge 0} a_n z^n
      \hspace{3em}
      B(z)
	= \sum_{n \ge 0} b_n z^n
    \end{equation*}
    Demostrar que \(\Phi_U(A(z) + B(z)) = \Phi_U(A(z)) + \Phi_U(B(z))\)
    es simple,
    y queda de ejercicio.

    Para la multiplicación:
    \begin{align*}
      \Phi_U(A(z) \cdot &B(z)) - \Phi_U(A(z)) \cdot \Phi_U(B(z)) \\
	&= \sum_{n \ge 0}
	     \left(
	       \sum_{0 \le k \le n} a_k b_{n - k}
	     \right) U^n
	    - \left(
		\sum_{n\ge 0} a_n U^n
	      \right) \cdot
		\left(
		  \sum_{n\ge 0} b_n U^n
		\right) \\
	&= \lim_{N \rightarrow \infty}
	     \sum_{0 \le n \le N}
	       \left(
		 \sum_{0 \le k \le n} a_k b_{n - k}
	       \right) U^n \\
	&\hspace{3em}
	    - \lim_{N \rightarrow \infty}
		\left(
		  \sum_{0 \le n \le N} a_n U^n
		\right) \cdot
		  \lim_{N \rightarrow \infty}
		    \left(
		      \sum_{0 \le n \le N} b_n U^n
		    \right) \\
	 &= \lim_{N \rightarrow \infty}
	      \left(
		\sum_{0 \le n \le N}
		  \left(
		    \sum_{0 \le k \le n} a_k b_{n - k}
		  \right) U^n
		- \left(
		    \sum_{0 \le n \le N} a_n U^n
		  \right) \cdot
		    \left(
		      \sum_{0 \le n \le N} b_n U^n
		    \right)
	      \right) \\
	 &= - \lim_{N \rightarrow \infty}
		\sum_{N + 1 \le n \le 2 N}
		  \left(
		    \sum_{0 \le k \le n} a_k b_{n - k}
		  \right) U(z)^n
    \end{align*}
    En esto hemos usado el teorema~\ref{theo:series-operaciones}.
    Resta demostrar que el último límite es infinito:
    \begin{equation*}
      \ord
	\left(
	  \sum_{N + 1 \le n \le 2 N}
	    \left(
	      \sum_{0 \le k \le n} a_k b_{n - k}
	    \right) U(z)^n
	\right)
	\ge \ord U(z)^{N + 1}
    \end{equation*}
    Esto tiende a infinito cuando \(N \rightarrow \infty\),
    y es \(\Phi_U(A(z) \cdot B(z)) = \Phi_U(A(z)) \cdot \Phi_U(B(z))\).
  \end{proof}
  O sea,
  operar con las series \(A(z)\) y \(B(z)\)
  es lo mismo que operar con las series \(A(U(z))\) y \(B(U(z))\).
  La importancia del teorema~\ref{theo:series-composicion-homomorfismo}
  radica en que expresiones como:
  \begin{equation*}
    \frac{1}{1 - z - z^2}
  \end{equation*}
  son ambiguas:
  ¿Es el recíproco de la serie \(1 - z - z^2\),
  o es tal vez la composición de \(1 / (1 - u)\) con \(u = z + z^2\)?
  El teorema asegura que ambas son la misma serie.

  Con las mismas herramientas se pueden justificar sumas y productos infinitos
  de series formales.
  \begin{definition}
    Sea la secuencia \(\left\langle A_k(z) \right\rangle_{k \ge 0}\)
    de series formales en \(R \llbracket z \rrbracket\).
    Decimos que la suma infinita
    \begin{equation*}
      \sum_{k \ge 0} A_k(z)
    \end{equation*}
    \emph{converge} si la secuencia
    \begin{equation*}
      \sum_{0 \le k \le N} A_k(z)
    \end{equation*}
    converge en el sentido definido antes cuando \(N \rightarrow \infty\).
    Igualmente,
    decimos que el producto infinito
    \begin{equation*}
      \prod_{k \ge 0} A_k(z)
    \end{equation*}
    \emph{converge} si la secuencia
    \begin{equation*}
      \prod_{0 \le k \le N} A_k(z)
    \end{equation*}
    converge cuando \(N \rightarrow \infty\).
  \end{definition}
  Con esto:
  \begin{theorem}
    \label{theo:formal-series-convergence:sum+product}
    Sea \(\left\langle A_k(z) \right\rangle_{k \ge 0}\)
    una secuencia en \(R \llbracket z \rrbracket\).
    Entonces las siguientes son equivalentes:
    \begin{enumerate}
    \item\label{item:lfs:infty}
      \(\displaystyle \lim_{k \rightarrow \infty} \ord A_k(z) = \infty\)
    \item\label{item:lfs:sum}
      \(\displaystyle \sum_{k \ge 0} A_k(z)\) converge
    \item\label{item:lfs:prod}
      \(\displaystyle \prod_{k \ge 0} (1 + A_k(z))\) converge
    \end{enumerate}
  \end{theorem}
  \begin{proof}
    Demostraremos que (\ref{item:lfs:infty}) equivale a (\ref{item:lfs:sum}),
    la demostración que (\ref{item:lfs:infty}) equivale a (\ref{item:lfs:prod})
    es similar y se omite.

    Definamos:
    \begin{equation*}
      C_n(z)
	= \sum_{0 \le k \le n} A_k(z)
    \end{equation*}
    Primero demostramos
    (\ref{item:lfs:infty})~\(\implies\)~(\ref{item:lfs:sum}).
    Tenemos:
    \begin{equation*}
      \forall n \exists k_0 \forall k \ge k_0 \colon \ord A_k(z) > n
    \end{equation*}
    Esto es equivalente a decir:
    \begin{equation*}
      \forall n \exists k_0 \forall k \ge k_0
	\colon \ord (C_{k + 1}(z) - C_k(z)) > n
    \end{equation*}
    O sea,
    para cada \(n\) hay \(k_0\) tal que
    para todo \(k \ge k_0\) el valor de \(\left[ z^n \right] C_k(z)\)
    es fijo,
    llamémosle \(c_n\).
    Consideremos:
    \begin{equation*}
      C(z)
	= \sum_{r \ge 0} c_r z^r
    \end{equation*}
    Por construcción:
    \begin{equation*}
      \forall n \exists k_0 \forall k \ge k_0
	\colon \ord (C_k(z) - C(z)) > n
    \end{equation*}
    y la secuencia \(C_k(z)\) converge.

    Ahora demostramos
    (\ref{item:lfs:sum})~\(\implies\)~(\ref{item:lfs:infty}).
    Si \(C_k(z)\) converge a \(C(z)\),
    entonces:
    \begin{equation*}
      \forall n \exists k_0 \forall k \ge k_0
	\colon \ord (C_k(z) - C(z)) > n
    \end{equation*}
    Para tales \(n\) y \(k\)
    tenemos \(C_k(z) - C(z) > n\),
    y:
    \begin{equation*}
      \left[ z^n \right] A_k(z)
	= \left[ z \right] (C_{k + 1}(z) - C_k(z))
	= \left[ z \right] C_{k + 1}(z) - \left[ z \right] C_k(z))
	= 0
    \end{equation*}
    lo que es equivalente a:
    \begin{equation*}
      \forall n \exists k_0 \forall k \ge k_0
	\colon \ord A_k(z) > n
    \end{equation*}
    que es lo que había que demostrar.
  \end{proof}
  Nótese que en el ámbito de los reales \(a_n \rightarrow 0\)
  no asegura que \(\sum a_n\) ni \(\prod (1 + a_n)\) converjan.

\section{El principio de transferencia}
\label{sec:series-principio-transferencia}
\index{serie formal!principio de transferencia|textbfhy}
\index{serie de potencias}

  Cuando \(R = \mathbb{R}\) o \(R = \mathbb{C}\),
  es obvio preguntarse sobre la relación entre la serie formal
  y la función definida por la serie de potencias.
  Es claro que no toda serie formal corresponde a una función analítica,
  por ejemplo la serie
  \begin{equation*}
    \sum_{n \ge 0} n! z^n
  \end{equation*}
  converge únicamente para \(z = 0\).
  Pero como \(0! = 1\),
  tiene un recíproco como serie formal:
  \begin{equation*}
    \left( \sum_{n \ge 0} n! z^n \right)^{-1}
      = 1 - z - z^2 - 3 z^3 - 13 z^4 - 71 z^5 - 461 z^6 - \dotsb
  \end{equation*}
  Razonamientos válidos para series formales
  no necesariamente tienen sentido para series de potencias.

  Por el otro lado,
  si dos series de potencias son idénticas como funciones analíticas
  lo son como series formales:
  \begin{theorem}[Principio de transferencia]
    \label{theo:series-principio-transferencia}
    Sean:
    \begin{equation*}
      A(z)
	= \sum_{n \ge 0} a_n z^n
      \hspace{3em}
      B(z)
	= \sum_{n \ge 0} b_n z^n
    \end{equation*}
    funciones reales o complejas,
    analíticas en un vecindario \(\mathcal{U}\) no vacío de cero.%
      \index{funcion@función!analitica@analítica}
    Si \(A(z) = B(z)\) para todo \(z \in \mathcal{U}\),
    entonces \(a_n = b_n\) para todo \(n \in \mathbb{N}_0\).
  \end{theorem}
  \begin{proof}
    Bajo las suposiciones,
    \(C(z) = A(z) - B(z)\) es analítica
    e idénticamente \(0\) en \(\mathcal{U}\).
    Por el teorema de Taylor,%
      \index{Taylor, teorema de}
    esto significa que todos los coeficientes de \(C(z)\) son cero:
    \begin{equation*}
      \left[ z^n \right] C(z)
	= \left[ z^n \right] (A(z) - B(z))
	= \left[ z^n \right] A(z) - \left[ z^n \right] B(z)
	= a_n - b_n
	= 0
      \qedhere
    \end{equation*}
  \end{proof}
  Esto permite demostrar algunas identidades en forma simple.
  Por ejemplo,
  tenemos las expansiones:
  \begin{align*}
    \mathrm{e}^z
      &= \sum_{n \ge 0} \frac{z^n}{n!}
	   && \text{para todo \(z \in \mathbb{C}\)} \\
    \ln (1 + z)
      &= \sum_{n \ge 1} \frac{z^n}{n}
	   && \text{para \(\lvert z \rvert < 1\)}
  \end{align*}
  Como para las funciones analíticas respectivas
  \(\exp(\ln (1 + z)) = 1 + z\),
  esto vale para las series formales.
  Verificarlo en forma directa involucra largos y complicados cálculos.

\section{Derivadas e integrales formales}
\label{sec:series:derivadas-integrales}

  Definimos:%
    \index{serie formal!derivada}
  \begin{equation*}
    \frac{\mathrm{d}}{\mathrm{d} z} \,
      \biggl( \, \sum_{n \ge 0} a_n z^n \biggr)
      = \sum_{n \ge 1} n a_n z^{n - 1}
      = \sum_{n \ge 0} (n + 1) a_{n + 1} z^n
  \end{equation*}
  Esta es una definición puramente formal,
  no intervienen límites ni el significado de la serie como función.
  Definimos además para una serie formal \(f(z)\):
  \begin{align*}
    f^{(0)}(z)
      &= f(z) \\
    f^{(n + 1)}(z)
      &= \frac{\mathrm{d}}{\mathrm{d} z} \, f^{(n)}(z)
  \end{align*}
  Una notación alternativa útil es:
  \begin{equation*}
    \mathrm{D} f(z)
      = \frac{\mathrm{d}}{\mathrm{d} z} \, f(z)
  \end{equation*}
  bajo el entendido \(\mathrm{D}^n f(z) = f^{(n)}(z)\).
  Para las primeras derivadas se suele usar:
  \begin{equation*}
    f'(z)
      = \frac{\mathrm{d} f}{\mathrm{d} x}
    \hspace{3em}
    f''(z)
      = \frac{\mathrm{d}^2 f}{\mathrm{d} x^2}
    \hspace{3em}
    f'''(z)
      = \frac{\mathrm{d}^3 f}{\mathrm{d} x^3}
  \end{equation*}

  Tenemos también:
  \begin{theorem}
    \label{theo:derivadas-series}
    Sean \(f(z)\) y \(g(z)\) series formales.
    Entonces:
    \begin{align*}
      \mathrm{D}^n \left( a f(z) + b g(z)\right)
	&= a f^{(n)}(z) + b g^{(n)}(z) \\
      \mathrm{D}^n \left(f(z) \cdot g(z)\right)
	&= \sum_{0 \le k \le n} \binom{n}{k} \, f^{(k)}(z) \cdot g^{(n - k)}(z)
    \end{align*}
  \end{theorem}
  La fórmula para las derivadas de un producto
  se conoce bajo el nombre de Leibniz.
  Las demostraciones son rutinarias,
  y quedan de ejercicio.

  Para la composición de series definida antes
  \begin{theorem}[Regla de la cadena]
    \label{theo:cadena-series}
    Sean \(f(z)\) y \(g(z)\) series formales,
    con \(g(0) = 0\).
    Entonces:
    \begin{equation*}
      \frac{\mathrm{d}}{\mathrm{d} z} f(g(z))
	= f'(g(z)) \cdot g'(z)
    \end{equation*}
  \end{theorem}
  Para demostrarlo,
  primeramente se demuestra la derivada de una potencia de una serie,
  y usando esto se aplica término a término.
  Nuevamente es rutina,
  y nos ahorraremos los detalles.

  De la misma manera que obtenemos derivadas término a término,
  podemos calcular integrales:%
    \index{serie formal!integral}
  \begin{equation*}
    \int_0^z f(t) \, \mathrm{d} t
      = \sum_{n \ge 1} \frac{a_{n - 1}}{n} \, z^n
  \end{equation*}
  Es claro que se cumplen las relaciones fundamentales:
  \begin{equation*}
    \frac{\mathrm{d}}{\mathrm{d} z} \, \int_0^z f(t) \, \mathrm{d} t
      = \int_0^z \frac{\mathrm{d}}{\mathrm{d} t} \, f(t) \, \mathrm{d} t
      = f(z)
  \end{equation*}
  Podemos anotar para la antiderivada:
  \begin{equation*}
    \mathrm{D}^{-1} f(z)
      = \int_0^z f(z) \, \mathrm{d} z
  \end{equation*}

\section{Series en múltiples variables}
\label{sec:series-multivariables}
\index{serie formal!multivariada}

  Podemos también considerar series en más de una variable,
  si las variables son \(x\) e \(y\)
  anotaremos \(R\llbracket x, y \rrbracket\).
  Esto es,
  por ejemplo:
  \begin{equation*}
    A(x, y)
      = \sum_{\substack{
		r \ge 0 \\
		s \ge 0
	     }} a_{r, s} x^r y^s
  \end{equation*}
  Es simple
  (aunque engorroso)
  demostrar que \(R \llbracket x, y \rrbracket\)
  es isomorfo a \(R \llbracket x \rrbracket \llbracket y \rrbracket\)%
    \index{anillo!homomorfismo}
  y a \(R \llbracket y \rrbracket \llbracket x \rrbracket\),
  pero no nos detendremos en tales detalles.

  Para desarrollar una teoría de secuencias en series multivariables,
  definimos el \emph{orden total}%
    \index{serie formal!orden total|textbfhy}
  del término \(x_1^{n_1} x_2^{n_2} x_3^{n_3} \dotso x_m^{n_m}\)
  como \(n_1 + n_2 + \dotsb + n_m\).
  El orden (total) de la serie \(\ord A(x_1, x_2, \dotsc, x_m)\)
  es el orden total del término
  de orden total mínimo en \(A(x_1, x_2, \dotsc, x_m)\).
  Con esta definición
  si para una secuencia
  de series formales multivariadas \(A_k(x_1, x_2, \dotsc, x_n)\):
  \begin{equation*}
    \lim_{k \rightarrow \infty} \ord A_k (x_1, x_2, \dotsc, x_n)
      = \infty
  \end{equation*}
  sabemos que solo en un número finito de los \(A_k(x_1, x_2, \dotsc, x_n)\)
  el coeficiente de \(x_1^{m_1} x_2^{m_2} \dotsm x_n^{m_n}\)
  difiere,
  y lo podemos calcular en un número finito de pasos.
  Omitimos los detalles
  de teoremas análogos a los para el caso univariado,
  solo notamos que esto permite justificar
  en \(R \llbracket x, y \rrbracket\)
  series como:
  \begin{equation*}
    \exp(x + y)
      = \sum_{n \ge 0} \frac{(x + y)^n}{n!}
  \end{equation*}
  Esto no resulta del caso univariado,
  en \(R \llbracket x \rrbracket \llbracket y \rrbracket\)
  la serie \(x + y\) tiene término constante \(x\),
  que no es cero.
  Podemos definir derivadas (parciales) e integrales
  de forma similar al caso univariado.
  Usaremos la notación \(\mathrm{D}_x f\) para la derivada parcial
  respecto de \(x\) de la serie \(f\),
  también \(f_{x y}\) para la derivada respecto de \(x\) e \(y\).

  \begin{theorem}[Funciones implícitas]
    \index{serie formal!funcion implicita@función implícita}
    \label{theo:series-funciones-implicitas}
    Sea \(A(x, y) \in F \llbracket x, y \rrbracket\)
    tal que \(A(0, 0) = 0\)
    y \(\mathrm{D}_y A(0, 0) \ne 0\).
    Entonces existe
    una única serie formal \(f(x) \in F \llbracket x \rrbracket\)
    con \(f(0) = 0\)
    tal que \(A(x, f(x)) = 0\).
  \end{theorem}
  \begin{proof}
    Escribamos
    \begin{equation*}
      A(x, y)
	= \sum_{n \ge 0} a_n(x) y^n
	= \sum_{\substack{n \ge 0 \\ k \ge 0}} a_{n k} x^k y^n
    \end{equation*}
    donde \(a_n(x) \in F \llbracket x \rrbracket\).
    Las condiciones sobre \(A(x, y)\) resultan en
    \(a_0(x) = 0\) y \(a_1(x) \ne 0\).
    Mostraremos cómo calcular sucesivamente los coeficientes \(f_n\) de
    \begin{equation*}
      f(x)
	= \sum_{n \ge 0} f_n x^n
    \end{equation*}
    Como \(f(0) = 0\),
    ya tenemos \(f_0 = 0\).
    Enseguida:
    \begin{equation*}
      \left[ x \right] \sum_{n \ge 0} a_n(x) f(x)^n
	= 0
    \end{equation*}
    hace que baste considerar
    \begin{equation*}
      \left[ x \right] ( a_0(x) + a_1(x) f(x) )
	= \left[ x \right]
	     (a_{0 0} + a_{0 1} x + a_{1 0} f_1 x + a_{1 1} f_0 x + \dotsb)
	= 0
    \end{equation*}
    de donde despejamos
    \begin{equation*}
      f_1
	= - \frac{a_{0 1}}{a_{1 1}}
    \end{equation*}
    Esto es válido,
    ya que \(a_{1 1} = \mathrm{D}_y A(0, 0) \ne 0\).
    Continuamos:
    \begin{equation*}
      \left[ x^k \right] \sum_{n \ge 0} a_n(x) f(x)^n
	= 0
    \end{equation*}
    donde \(f(0) = 0\) permite truncar la suma,
    y se traduce en
    \begin{equation*}
      \left[ x^k \right]
	\sum_{0 \le n \le k} a_n(x) f(x)^n
    \end{equation*}
    Observamos que el único término en esto que depende de \(f_k\)
    viene de \(a_1(0) f_k\),
    todos los demás solo involucran \(f_0\), \ldots, \(f_{k - 1}\).
    Despejando,
    tenemos \(f_k\) en términos de los coeficientes anteriores,
    y \(f(x)\) queda determinada mediante un proceso convergente.
  \end{proof}

  Junto con el teorema~\ref{theo:series-composicion-homomorfismo},
  el teorema~\ref{theo:series-funciones-implicitas}
  nos dice que la substitución
  \(z \leadsto U(z)\) es un isomorfismo
  de \(F \llbracket z \rrbracket\) a sí mismo
  (un \emph{automorfismo})%
    \index{anillo!automorfismo}
  si \(\ord U(z) = 1\).
  Es claro que los coeficientes de tales funciones implícitas
  normalmente resultan bastante locos.

  Por ejemplo,
  podemos definir:
  \begin{equation*}
    W(z) \mathrm{e}^{W(z)}
      = z
  \end{equation*}
  con \(W(0) = 0\).
  Para esto tomamos
    \(A(x, y) = y \mathrm{e}^y - x \in \mathbb{R} \llbracket x, y \rrbracket\),
  como \(A(0, 0) = 0\) y \(\mathrm{D}_y A(0, 0) = 1 \ne 0\),
  se cumplen las condiciones del teorema~\ref{theo:series-funciones-implicitas}
  y tal serie de potencias \(W(z)\) existe.
  La demostración da una receta para obtener los coeficientes.

  Es relativamente sencillo el caso particular de ecuaciones de la forma
  \begin{equation*}
    u = t \phi(u)
  \end{equation*}
  donde \(\phi\) es una función dada de \(u\).
  Esta relación define \(u\) en función de \(t\),
  y ``estamos despejando \(u\) en términos de \(t\)''.
  Fue demostrada por Lagrange%
    \index{Lagrange, inversion de@Lagrange, inversión de|textbfhy}%
    \index{Lagrange-Burmann, inversion de@Lagrange-Bürmann, inversión de|see{Lagrange, inversión de}}
  y casi simultáneamente por Bürmann,%
    \index{Burmann, Hans Heinrich@Bürmann, Hans Heinrich}
  la forma dada acá es la de Bürmann.
  \begin{theorem}[Fórmula de inversión de Lagrange]
    \label{theo:LIF}
    Sean \(f(u)\) y \(\phi(u)\) series formales de potencias en \(u\)
    sobre un campo \(F\),
    con \(\phi(0) = 1\).
    Entonces hay una única serie formal \(u = u(t)\) que cumple:
    \begin{equation*}
      u = t \phi(u)
    \end{equation*}
    Además,
    el valor \(f(u(t))\) de \(f\) en el cero \(u = u(t)\),
    expandido en serie alrededor de \(t = 0\),
    cumple:
    \begin{equation*}
      \left[ t^n \right] \left\{ f(u(t)) \right\}
	 = \frac{1}{n} \, \left[ u^{n - 1} \right] \,
			    \left\{ f'(u) \phi(u)^n \right\}
    \end{equation*}
  \end{theorem}
  Dadas \(f\) y \(\phi\),
  esta fórmula da los coeficientes de \(f(u(t))\) en bandeja.
  No demostraremos este resultado,
  ya que nos llevaría demasiado fuera del rango de este ramo.
  La demostración del teorema puede encontrarse en el texto de Wilf~%
    \cite{wilf06:_gfology}.

  La razón del nombre es que si \(t = A(u)\),
  esta fórmula da \(u = u(t)\) mediante:
  \begin{equation*}
    u = t \frac{u}{A(u)}
  \end{equation*}

  Una aplicación entretenida provee la función de Cayley,%
    \index{Cayley, funcion de@Cayley, función de}
  definida por:
  \begin{equation}
    \label{eq:Cayley-function}
    C(z)
      = z \mathrm{e}^{C(z)}
  \end{equation}
  Con \(f(u) = u\)
  y \(\phi(u) = \mathrm{e}^u\) tenemos directamente los coeficientes:
  \begin{align}
    [z^n] C(z)
      &= \frac{1}{n} [u^{n - 1}] \mathrm{e}^{n u} \notag \\
      &= \frac{1}{n} \frac{n^{n - 1}}{(n - 1)!} \notag \\
      &= \frac{n^{n - 1}}{n!}
	    \label{eq:Cayley-coeff}
  \end{align}
  Por otro lado,
  con \(f(u) = \mathrm{e}^{\gamma u}\)
  y \(\phi(u) = \mathrm{e}^u\) resulta:
  \begin{align*}
    [z^n] \mathrm{e}^{\gamma C(z)}
      &= \frac{1}{n} [u^{n - 1}]
	   \left(
	     \frac{\mathrm{d}}{\mathrm{d} u} \mathrm{e}^{\gamma u}
	       \cdot \mathrm{e}^{n u}
	   \right) \\
      &= \frac{\gamma (\gamma + n)^{n - 1}}{n!}
  \end{align*}
  Comparando coeficientes de:
  \begin{equation*}
    \mathrm{e}^{(\alpha + \beta) C(z)}
      = \mathrm{e}^{\alpha C(z)} \cdot \mathrm{e}^{\beta C(z)}
  \end{equation*}
  se obtiene la fórmula binomial de Abel~\cite{abel26:_identity}:%
    \index{Abel, formula binomial de@Abel, fórmula binomial de}
  \begin{equation}
    \label{eq:Abel-binomial}
    (\alpha + \beta) (\alpha + \beta + n)^{n - 1}
      = \alpha \beta
	  \sum_{0 \le k \le n}
	    \binom{n}{k}
	      (\alpha + k)^{k - 1}
	      (\beta + n - k)^{n - k - 1}
  \end{equation}

%%% Local Variables:
%%% mode: latex
%%% TeX-master: "clases"
%%% End:

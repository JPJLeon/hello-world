% analisis-complejo.tex
%
% Copyright (c) 2013-2015 Horst H. von Brand
% Derechos reservados. Vea COPYRIGHT para detalles

\chapter{Introducción al análisis complejo}
\label{cha:analisis-complejo}
\index{analisis complejo@análisis complejo|textbfhy}
\index{analisis complejo@análisis complejo|seealso{\(\mathbb{C}\) (números complejos)}}

  Nahin~\cite{nahin10:_imaginary_tale}
  narra la larga y variada historia de los números complejos.
  En forma similar al análisis con los reales
  se puede desarrollar análisis en el ámbito complejo.
  Muchos resultados son simples de obtener para los complejos,
  y algunos fenómenos misteriosos en el análisis real
  se explican al observar desde esta óptica.
  La teoría tiene su propio encanto.

  Algunas de nuestras maniobras
  son en extremo engorrosas
  usando solo las técnicas del análisis real.
  Daremos acá los resultados más importantes del análisis complejo,
  que ayuda enormemente al simplificar integrales definidas
  y sumas.
  Entrega además herramientas útiles
  para construir aproximaciones asintóticas
  a muchos valores de interés.
  Textos introductorios más completos son por ejemplo los de
  Ash y Novinger~\cite{ash07:_complex_variables},
  Beck, Marchesi, Pixton y Sabalka~%
    \cite{beck12:_first_course_compl_analysis},
  Cain~\cite{cain01:_compl_analy}
  y~Chen~\cite{chen08:_intro_complex_anal}.
  Una visión detallada y bastante completa dan Stein y~Shakarchi~%
    \cite{stein10:_compl_analy}.

\section{Aritmética}
\label{sec:aritmetica-complejos}
\index{C (numeros complejos)@\(\mathbb{C}\) (números complejos)!operaciones}

  Sean \(w = u + \mathrm{i} v\) y \(z = x + \mathrm{i} y\) complejos
  (con la \emph{unidad imaginaria}\, \(\mathrm{i} = \sqrt{-1}\)%
    \index{C (numeros complejos)@\(\mathbb{C}\) (números complejos)!unidad imaginaria}
   y \(u, v, x, y \in \mathbb{R}\) ).
  La \emph{parte real} de \(z\) es \(\Re z = x\),%
    \index{C (numeros complejos)@\(\mathbb{C}\) (números complejos)!parte real}
  la \emph{parte imaginaria} de \(z\) es \(\Im z = y\).%
    \index{C (numeros complejos)@\(\mathbb{C}\) (números complejos)!parte imaginaria}
  La suma y multiplicación
  se calculan como polinomios en \(\mathbb{R}\)
  en la variable \(\mathrm{i}\),
  para luego considerar \(\mathrm{i}^2 = -1\).
  Vale decir:
  \begin{align*}
    w + z
      &= (u + x) + \mathrm{i} (v + y) \\
    w \cdot z
      &= (u x - v y) + \mathrm{i} (v x + u y)
  \end{align*}
  Con estas operaciones los números complejos
  son un campo,%
    \index{campo (algebra)@campo (álgebra)}
  que anotamos \(\mathbb{C}\).

  Podemos identificar los complejos con parte imaginaria \(0\)
  con los reales,
  y representar el complejo \(z = x + \mathrm{i} y\)
  por el punto \((x, y)\) del plano \(\mathbb{R}^2\).
  A esta forma de representarlos
  se le llama \emph{forma rectangular}%
    \index{C (numeros complejos)@\(\mathbb{C}\) (números complejos)!forma rectangular|see{\(\mathbb{C}\) (números complejos)!forma cartesiana}}
  o \emph{forma cartesiana}.%
    \index{C (numeros complejos)@\(\mathbb{C}\) (números complejos)!forma cartesiana}
  Llamamos \emph{eje real} al eje~\(X\)
  y \emph{eje imaginario} al eje~\(Y\).
  La suma de complejos
  es simplemente la suma de los vectores correspondientes.
  \begin{figure}[ht]
    \centering
    \subfloat[Suma]{
      \pgfimage{images/complex-sum}
      \label{subfig:complex-sum}
    }%
    \hspace*{3em}%
    \subfloat[Producto]{
      \pgfimage{images/complex-product}
      \label{subfig:complex-product}
    }
    \caption{Operaciones entre complejos}
    \label{fig:complex-operations}
  \end{figure}
  Vea la figura~\ref{subfig:complex-sum}.

  Como alternativa a dar las coordenadas del vector,
  podemos describirlo mediante su largo
  y el ángulo que forma con el eje~\(X\).
  Para \(z = x + \mathrm{i} y\)
  el \emph{valor absoluto}
  (también \emph{módulo}) \(r = \lvert z \rvert\) se define como:%
    \index{C (numeros complejos)@\(\mathbb{C}\) (números complejos)!modulo@módulo}
  \begin{equation}
    \label{eq:complex-modulus}
    \lvert z \rvert
      = \sqrt{x^2 + y^2}
  \end{equation}
  Un \emph{argumento} de \(z\),%
    \index{C (numeros complejos)@\(\mathbb{C}\) (números complejos)!argumento}
  anotado \(\arg z\),
  es un número real \(\phi\) tal que
  \begin{equation}
    \label{eq:complex-argument}
    x = r \cos \phi \qquad y = r \sin \phi
  \end{equation}
  Nótese que todo número complejo tiene infinitos argumentos.
  En el caso excepcional \(z = 0\)
  el módulo es \(0\) y cualquier ángulo sirve de argumento.
  En caso \(z \ne 0\)
  vemos que si \(\phi\) es un argumento de \(z\),
  también lo es \(\phi + 2 \pi k\),
  para todo \(k \in \mathbb{Z}\).
  La representación del número complejo como módulo y argumento
  se conoce como \emph{representación polar}.%
    \index{C (numeros complejos)@\(\mathbb{C}\) (números complejos)!forma polar}
  El \emph{valor principal} del argumento se anota \(\Arg z\),%
    \index{C (numeros complejos)@\(\mathbb{C}\) (números complejos)!valor principal}
  es el ángulo restringido al rango \(-\pi < \phi \le \pi\).
  A veces resulta útil restringir el ángulo a otro rango,
  usaremos \(\arg_\alpha z\)
  para el ángulo en el rango \(\alpha \le \arg_\alpha z < \alpha + 2 \pi\).

  La representación polar
  da una bonita interpretación de la multiplicación.
  Sean números complejos \(z_1 = x_1 + \mathrm{i} y_1\)
  y \(z_2 = x_2 + \mathrm{i} y_2\),
  respectivamente con módulos \(r_1\) y \(r_2\)
  y argumentos \(\phi_1\) y \(\phi_2\).
  Entonces:
  \begin{align*}
    (x_1 + \mathrm{i} y_1) \cdot (x_2 + \mathrm{i} y_2)
      &= (r_1 \cos \phi_1 + \mathrm{i} r_1 \sin \phi_1)
	   \cdot (r_2 \cos \phi_2 + \mathrm{i} r_2 \sin \phi_2) \\
      &= r_1 r_2
	   \left(
	     (\cos \phi_1 \cos \phi_2 - \sin \phi_1 \sin \phi_2)
	       + \mathrm{i}
		   (\sin \phi_1 \cos \phi_2 + \cos \phi_1 \sin \phi_2)
	   \right) \\
      &= r_1 r_2 (\cos (\phi_1 + \phi_2)
		    + \mathrm{i} \, \sin(\phi_1 + \phi_2))
  \end{align*}
  Vea la figura~\ref{subfig:complex-product} para un ejemplo.

  Deberemos manipular expresiones de la forma
    \(\cos \phi + \mathrm{i} \, \sin \phi\)
  con bastante frecuencia,
  rindiéndonos a la flojera
  (con la excusa de ahorrar papel,
   tinta,
   etc.)
  escribimos:
  \begin{equation}
    \index{C (numeros complejos)@\(\mathbb{C}\) (números complejos)!exponencial}
    \label{eq:imaginary-exponential}
    \exp(\mathrm{i} \, \phi)
      = \mathrm{e}^{\mathrm{i} \, \phi}
      = \cos \phi + \mathrm{i} \, \sin \phi
  \end{equation}
  Por ahora~\eqref{eq:imaginary-exponential}
  es simplemente una abreviatura cómoda,
  más adelante demostraremos que es consistente
  con la función exponencial
  del cálculo real.
  También es común la notación:
  \begin{equation*}
    \operatorname{cis} \phi
      = \cos \phi + \mathrm{i} \, \sin \phi
  \end{equation*}

  El siguiente lema recoge algunas propiedades salientes,
  alentamos al lector interesado demostrarlas.
  \begin{lemma}
    \label{lem:imaginary-exponential}
    Para cualquier \(\phi, \phi_1, \phi_2 \in \mathbb{R}\)
    y todo \(k \in \mathbb{Z}\):
    \begin{enumerate}[label=(\roman*), ref=(\roman*)]
    \item
      \(\lvert \mathrm{e}^{\mathrm{i} \, \phi} \rvert
	  = 1\)
    \item
      \(\mathrm{e}^{\mathrm{i} \, \phi_1}
	\mathrm{e}^{\mathrm{i} \, \phi_2}
	  = \mathrm{e}^{\mathrm{i} (\phi_1 + \phi_2)}\)
    \item
      \(1 / \mathrm{e}^{\mathrm{i} \, \phi}
	  = \mathrm{e}^{- \mathrm{i} \, \phi}\)
    \item
      \(\mathrm{e}^{\mathrm{i} (\phi + 2 k \pi)}
	  = \mathrm{e}^{\mathrm{i} \, \phi}\)
    \end{enumerate}
  \end{lemma}
  Con esta notación,
  el número complejo de módulo \(r\) y argumento \(\phi\)
  puede escribirse:
  \begin{equation}
    \label{eq:complex-exponential-notation}
    x + \mathrm{i} y
      = r \mathrm{e}^{\mathrm{i} \, \phi}
  \end{equation}

  El cuadrado del valor absoluto de \(z\) tiene la bonita propiedad:
  \begin{equation}
    \label{eq:complex-modulus-conjugates}
    \lvert z \rvert^2
      = x^2 + y^2
      = (x + \mathrm{i} y) \cdot (x - \mathrm{i} y)
  \end{equation}
  Esto hace útil la operación de \emph{conjugado}:%
    \index{C (numeros complejos)@\(\mathbb{C}\) (números complejos)!conjugado}
  \begin{equation}
    \label{eq:complex-conjugate}
    \overline{x + \mathrm{i} y}
      = x - \mathrm{i} y
  \end{equation}
  En el plano cartesiano
  corresponde a reflejar el vector en el eje~\(X\).
  Tenemos algunas propiedades,
  que nuevamente animamos a demostrar.
  \begin{lemma}
    \label{lem:complex-conjugate}
    Para todo \(z, z_1, z_2 \in \mathbb{C}\),
    y para todo \(\phi \in \mathbb{R}\):
    \begin{enumerate}[label=(\roman*), ref=(\roman*)]
    \item
      \(\overline{z_1 \pm z_2} = \overline{z_1} \pm \overline{z_2}\)
    \item
      \(\overline{z_1 \cdot z_2}
	   = \overline{z_1} \cdot \overline{z_2}\)
    \item
      \(\overline{z_1 / z_2} = \overline{z_1} / \overline{z_2}\)
    \item
      \(\overline{\overline{z}} = z\)
    \item
      \(\lvert \overline{z} \rvert = \lvert z \rvert\)
    \item
      \label{lem:complex-conjugate:part-f}
      \(\lvert z \rvert^2 = z \overline{z}\)
    \item
      \(\Re z = \frac{1}{2} \, ( z + \overline{z})\)
      \qquad
      \(\Im z = \frac{1}{2 \mathrm{i}} \, ( z - \overline{z})\)
    \item
      \(\overline{\mathrm{e}^{\mathrm{i} \, \phi}}
	  = \mathrm{e}^{- \mathrm{i} \, \phi}\)
    \end{enumerate}
  \end{lemma}
  La parte~\ref{lem:complex-conjugate:part-f}
  del lema~\ref{lem:complex-conjugate}
  da una fórmula limpia para el recíproco de un complejo no cero:
  \begin{equation}
    \index{C (numeros complejos)@\(\mathbb{C}\) (números complejos)!reciproco@recíproco}
    \label{eq:complex-inverse}
    \frac{1}{z}
      = \frac{\overline{z}}{\lvert z \rvert^2}
  \end{equation}

\section{Un poquito de topología del plano}
\label{sec:topologia}
\index{topologia@topología}

  Al considerar funciones en \(\mathbb{R}\)
  la situación es bastante simple,
  basta hablar de intervalos.
  Incluso en \(\mathbb{R}^n\)
  el tratamiento puede seguir esencialmente
  variable a variable
  y restringirse a intervalos adecuados.
  En \(\mathbb{C}\) esto no es satisfactorio,
  áreas del plano
  pueden tener relaciones mucho más complicadas entre sí
  que los simples intervalos.
  Las definiciones siguientes
  serán usadas con mucha frecuencia en lo que sigue,
  es importante familiarizarse con ellas.

  Requeriremos alguna terminología
  para tratar con subconjuntos de \(\mathbb{C}\).
  Si \(w, z \in \mathbb{C}\),
  entonces \(\lvert z - w \rvert\) es la distancia en el plano
  entre esos puntos.
  Si fijamos un número complejo \(a\) y un real positivo \(r\),
  el conjunto \(\lvert z - a \rvert = r\)
  es la circunferencia de radio \(r\) alrededor de \(a\).
  Al interior del círculo
  se le llama el \emph{disco abierto} de radio \(r\)%
    \index{disco abierto}
  alrededor de \(a\),
  que anotaremos \(D_r(a)\).
  Más precisamente,
  \(D_r(a) = \{z \in \mathbb{C} \colon \lvert z - a \rvert < r\}\).
  Nótese que esto no incluye la circunferencia.

  \begin{definition}
    Sea \(E\) un subconjunto cualquiera de \(\mathbb{C}\).
    \begin{enumerate}[label=(\roman*), ref=(\roman*)]
    \item
      \index{punto interior}
      Un punto \(a\) es un \emph{punto interior} de \(E\)
      si hay algún disco abierto \(D_r(a)\)
      que está completamente en \(E\)
    \item
      \index{punto frontera}
      Un punto \(b\) es un \emph{punto frontera} de \(E\)
      si todo disco abierto \(D_r(b)\) contiene un punto de \(E\)
      y un punto que no pertenece a \(E\).
    \item
      \index{punto de acumulacion@punto de acumulación}
      Un punto \(c\) es un \emph{punto de acumulación} de \(E\)
      si todo disco abierto \(D_r(c)\) contiene un punto de \(E\)
      diferente de \(c\)
    \item
      \index{punto aislado}
      Un punto \(d\) es un \emph{punto aislado}	 de \(E\)
      si pertenece a \(E\) y algún disco abierto \(D_r(d)\)
      no contiene ningún punto de \(E\) excepto \(d\)
    \end{enumerate}
  \end{definition}
  En lo anterior \(a\) pertenece a \(E\),
  pero \(b\) y \(c\) no necesariamente pertenecen a \(E\).
  Desde un punto interior
  podemos movernos un poco en cualquier dirección
  sin salir de \(E\),
  de un punto frontera moviéndonos un poco quedamos dentro de \(E\),
  pero otros movimientos arbitrariamente pequeños
  nos dejan al exterior.
  Como el nombre indica,
  un punto aislado está desconectado del resto del conjunto,
  hay un entorno de él que no contiene otros puntos del conjunto.
  \begin{definition}
    \index{conjunto!abierto|see{topología}}
    \index{conjunto!cerrado|see{topología}}
    Un conjunto es \emph{abierto} si todos sus puntos son internos,
    y es \emph{cerrado} si incluye todos sus puntos frontera.
  \end{definition}
  Como	ejemplos,
  para \(r > 0\) y \(z_0 \in \mathbb{C}\)
  los conjuntos
    \(\{z \in \mathbb{C} \colon \lvert z - z_0 \rvert < r\}\),
  \(\{z \in \mathbb{C} \colon \lvert z - z_0 \rvert > r\}\)
  y \(\{x + \mathrm{i} y \in \mathbb{C} \colon -1 < x < 1\}\)
  son abiertos.
  El conjunto
    \(\{ x + \mathrm{i} y \in \mathbb{C}
	   \colon -1 \le x \le 1 \wedge -5 \le y \le 5\}\)
  es cerrado.
  Los conjuntos \(\varnothing\) y \(\mathbb{C}\) son abiertos,
  pero también son cerrados
  (no tienen puntos frontera,
   con lo que vacuamente incluyen sus fronteras).
  El conjunto
    \(\{ x + \mathrm{i} y \colon 0 \le x \le 1 \wedge 0 < y < 1 \}\)
  no es abierto ni cerrado.
  \begin{definition}
    La \emph{frontera} de \(E\),%
      \index{frontera|see{topología}}
    anotada \(\partial E\),
    es el conjunto de los puntos frontera de \(E\).
    La \emph{clausura} de \(E\),%
      \index{clausura|see{topología}}
    anotada \(\overline{E}\),
    es el conjunto \(E\) junto con su frontera.
  \end{definition}
  Para el disco abierto
    \(D_r(z_0)
	= \{ z \in \mathbb{C} \colon\lvert z - z_0 \rvert < r \}\)
  la frontera es
    \(\partial D_r(z_0)
	= \{ z \in \mathbb{C} \colon \lvert z - z_0 \rvert = r \}\),
  y su clausura es
    \(\overline{D}_r(z_0)
	= \{ z \in \mathbb{C}
	     \colon \lvert z - z_0 \rvert \le r \}\).
  Un tema un tanto sutil en los complejos
  es la idea de \emph{conectividad}.%
      \index{conectividad|see{topología}}
  Intuitivamente,
  un conjunto es conexo si es ``una sola pieza''.
  En los reales un conjunto es conexo
  exactamente si es un único intervalo,
  lo que no tiene mucho interés.
  En un plano hay gran variedad de conjuntos conexos,
  y se requiere una definición precisa.
  \begin{definition}
    Dos conjuntos \(X, Y \subseteq \mathbb{C}\)
    se dicen \emph{separados}
    si hay conjuntos abiertos \(A\) y \(B\) disjuntos
    tales que \(X \subseteq A\) y \(Y \subseteq B\).
    El conjunto \(D \subseteq \mathbb{C}\) es \emph{conexo}
    si es imposible hallar conjuntos abiertos disjuntos
    tales que \(D\) es su unión.
    Una \emph{región} es un conjunto conexo abierto.
  \end{definition}
  Por ejemplo,
  los intervalos \([0, 1)\) y \((1, 2]\) en el eje real
  están separados
  (hay infinitas posibilidades para \(X\) e \(Y\) de la definición,
   por ejemplo \(X = D_1(0)\) e \(Y = D_1(2)\)).

  Un tipo de conjunto conexo
  que usaremos con frecuencia es la curva.
  \begin{definition}
    \index{camino|see{topología}}
    \index{curva|see{topología}}
    Un \emph{camino} o \emph{curva} en \(\mathbb{C}\)
    es la imagen de una función continua
      \(\gamma \colon [a, b] \rightarrow \mathbb{C}\),
    donde \([a, b]\) es un intervalo cerrado en \(\mathbb{R}\).
    Acá la continuidad
    se refiere a que \(t \mapsto x(t) + \mathrm{i} y(t)\),
    y que tanto \(x\) como \(y\) son continuas.
    La curva se dice \emph{suave}
    si ambas componentes son diferenciables.
  \end{definition}
  Decimos que la curva está \emph{parametrizada} por \(\gamma\),
  y en un abuso común de la notación usaremos \(\gamma\)
  para referirnos a la curva.
  La curva se dice \emph{cerrada} si \(\gamma(a) = \gamma(b)\),
  y es una \emph{curva simple cerrada} si \(\gamma(a) = \gamma(b)\)%
    \index{curva simple cerrada}
  y \(\gamma(s) = \gamma(t)\) solo si \(s = t\),
  \(s = a\) y \(t = b\),
  o \(s = b\) y \(t = a\).
  Vale decir,
  la curva no se intersecta a sí misma,
  solo coinciden los puntos inicial y final.

  Lo siguiente es intuitivamente obvio,
  pero requiere ahondar bastante para demostrarse:
  \begin{theorem}
    \label{theo:curve=connected}
    Toda curva en \(\mathbb{C}\) es conexa.
  \end{theorem}
  Es claro que \(\gamma \colon [0, 1] \rightarrow \mathbb{C}\)
  con \(\gamma(t) = z_0 + t (z_1 - z_0)\)
  define un segmento de una recta en \(\mathbb{C}\)
  que va de \(z_0\) a \(z_1\).
  Al segmento de la recta \(z_0 z_1\) así parametrizada
  la anotaremos \([z_0, z_1]\).
  Podemos definir una curva formada
  por los segmentos \(z_0 z_1\),
  \(z_1 z_2\),
  \ldots,
  \(z_{n - 1} z_n\),
  que anotaremos \([z_0, z_1, \dotsc, z_n]\).
  A tales curvas las llamaremos \emph{poligonales}.
  La parametrización quedará a cargo del amable lector.
  Un teorema intuitivamente obvio,
  pero cuya demostración tiene sus sutilezas,
  es el siguiente:
  \begin{theorem}
    \label{theo:connected=curve-inside}
    Si \(D\) es un subconjunto de \(\mathbb{C}\)
    tal que cualquier par de puntos en \(D\)
    pueden conectarse mediante una curva en \(D\)
    entonces \(D\) es conexo.
    Por el otro lado,
    si \(D\) es un subconjunto abierto conexo de \(\mathbb{C}\)
    entonces cualquier par de puntos de \(D\)
    pueden conectarse mediante una curva en \(D\),
    incluso es posible conectarlos mediante una curva poligonal.
  \end{theorem}
  Un teorema central,
  bastante difícil de demostrar en su generalidad
  (ver Jordan~%
    \cite[páginas~587-594]{jordan87:_cours_analyse}
   para la demostración original)
  es el siguiente:
  \begin{theorem}[Jordan]
    \label{theo:Jordan}
    \index{Jordan, teorema de}
    Sea \(\gamma\) una curva suave simple cerrada.
    Entonces el complemento de \(\gamma\)
    consiste exactamente de dos componentes conexos.
    Uno de estos componentes es acotado
    (el \emph{interior} de \(\gamma\)),
    el otro no es acotado
    (el \emph{exterior} de \(\gamma\)).
  \end{theorem}
  Por este teorema comúnmente se les llama \emph{curvas de Jordan}%
    \index{Jordan, curva de|see{curva simple cerrada}}
  a las curvas simples cerradas.

\section{Límites y derivadas}
\label{sec:limites-derivadas}

  Si \(z\) es una variable compleja,
  y \(f(z)\) alguna función de la misma,
  el límite se define igual que en los reales.%
    \index{C (numeros complejos)@\(\mathbb{C}\) (números complejos)!limite@límite}
  Decimos que
  \begin{equation*}
    \lim_{z \rightarrow z_0} f(z) = \omega
  \end{equation*}
  si para todo \(\epsilon > 0\) existe \(\delta > 0\) tal que:
  \begin{equation}
    \label{eq:complex-limit}
    0 < \lvert z - z_0 \rvert < \delta
      \implies \lvert f(z) - \omega \rvert < \epsilon
  \end{equation}
  Formalmente es idéntica a la definición para los reales,
  pero debe tenerse presente
  que acá \(\lvert z - z_0 \rvert < \delta\)
  representa un círculo alrededor de \(z_0\).
  Esto suele describirse diciendo que el límite debe ser el mismo,
  independiente del camino que siga \(z = x + \mathrm{i} y\)
  para acercarse a \(z_0 = x_0 + \mathrm{i} y_0\).
  Definimos que \(f(z)\) es \emph{continua} en \(z_0\) si
  \begin{equation*}
    \lim_{z \rightarrow z_0} f(z)
      = f(z_0)
  \end{equation*}
  Si \(f\) es continua en todos los puntos en que está definida
  decimos simplemente que es continua.
  Si \(z = x + \mathrm{i} y\),
  \(z_0 = x_0 + \mathrm{i} y_0\)
  y \(f(z) = u(x, y) + \mathrm{i} v(x, y)\)
  (como por ejemplo \(f(z) = z^2 = x^2 - y^2 + 2 x y \mathrm{i}\)) ,
  es fácil ver que \(f(z)\) es continua si y solo si
  lo son \(u(x, y)\) y \(v(x, y)\).
  \begin{lemma}
    \label{lem:complex-limits}
    Si \(\lim_{z \rightarrow z_0} f(z)\)
    y \(\lim_{z \rightarrow z_0} g(z)\)
    existen,
    tenemos las siguientes propiedades de los límites:
    \begin{enumerate}[label=(\roman*), ref=(\roman*)]
    \item
      \(\displaystyle \lim_{z \rightarrow z_0} c f(z)
	= c \lim_{z \rightarrow z_0} f(z)\)
    \item
      \(\displaystyle \lim_{z \rightarrow z_0} (f(z) + g(z))
	= \lim_{z \rightarrow z_0} f(z)
	    + \lim_{z \rightarrow z_0} g(z)\)
    \item
      \(\displaystyle \lim_{z \rightarrow z_0} (f(z) \cdot g(z))
	= \lim_{z \rightarrow z_0} f(z)
	    \cdot \lim_{z \rightarrow z_0} g(z)\)
    \item
      Siempre que \(\lim_{z \rightarrow z_0} g(z) \ne 0\)
      es:\\[0.3ex]
      \(\displaystyle \lim_{z \rightarrow z_0} \frac{f(z)}{g(z)}
	= \frac{\lim_{z \rightarrow z_0} f(z)}
	       {\lim_{z \rightarrow z_0} g(z)}\)
    \end{enumerate}
  \end{lemma}
  La demostración es simple,
  y quedará de ejercicio.
  De acá es inmediato que la suma, diferencia, producto y cociente
  de funciones continuas son continuas
  (siempre que no tengamos un denominador cero,
   claro está).

  Dada una función compleja \(f(z)\)
  definimos su \emph{derivada} en \(z = z_0\)%
    \index{C (numeros complejos)@\(\mathbb{C}\) (números complejos)!derivada}
  como el siguiente límite,
  si existe:
  \begin{equation}
    \label{eq:complex-derivative}
    f'(z_0)
      = \lim_{h \rightarrow 0} \frac{f(z_0 + h) - f(z_0)}{h}
  \end{equation}
  Una notación alternativa común es:
  \begin{equation*}
    \frac{\mathrm{d} f}{\mathrm{d} z}
      = f'(z)
  \end{equation*}
  La condición que el límite sea el mismo,
  independiente del camino seguido por \(h\)
  para aproximarse a cero,
  hace que para
    \(f(x + \mathrm{i} y) = u(x, y) + \mathrm{i} v(x, y)\)
  con \(\Delta x\) y \(\Delta y\) reales deba ser:
  \begin{align}
    \lim_{\Delta x \rightarrow 0}
       \frac{f(z_0 + \Delta x) - f(z_0)}{\Delta x}
      &= \lim_{\Delta y \rightarrow 0}
	   \frac{f(z_0 + \mathrm{i} \, \Delta y) - f(z_0)}
		{\mathrm{i} \, \Delta y}
	     \notag \\
    \frac{\partial u}{\partial x}
      + \mathrm{i} \, \frac{\partial v}{\partial x}
      &= - \mathrm{i} \, \frac{\partial u}{\partial y}
	     + \frac{\partial v}{\partial y}
	     \label{eq:complex-derivative-uv}
  \end{align}
  Las expresiones~\eqref{eq:complex-derivative-uv}
  expresan la derivada compleja en términos de las coordenadas.

  Igualando partes reales y complejas,
  resultan las \emph{ecuaciones de Cauchy-Riemann}:
  \begin{equation}
    \label{eq:Cauchy-Riemann}
    \index{Cauchy-Riemann, ecuaciones de|textbfhy}
    \begin{split}
      \frac{\partial u}{\partial x}
	&= \frac{\partial v}{\partial y} \\
      \frac{\partial u}{\partial y}
	&= - \frac{\partial v}{\partial x}
    \end{split}
  \end{equation}
  Esto ya demuestra que hay condiciones fuertes
  para que una función tenga derivada en un punto.
  Resulta que las ecuaciones~\eqref{eq:Cauchy-Riemann}
  junto con continuidad de las derivadas
  son condiciones necesarias y suficientes
  para que \(f(x + \mathrm{i} y) = u(x, y) + \mathrm{i} v(x, y)\)
  tenga derivada en \(z_0 = x_0 + \mathrm{i} y_0\).

  Si la función \(f\) tiene derivada en \(z_0\),
  se dice que es \emph{diferenciable} en \(z_0\).
  A una función diferenciable en todo punto en una región abierta
  se le llama \emph{holomorfa}%
    \index{C (numeros complejos)@\(\mathbb{C}\) (números complejos)!funcion holomorfa@función holomorfa|textbfhy}
  (mucha literatura erróneamente se refiere a ellas
   como \emph{funciones analíticas},
   un concepto relacionado).
  Una función holomorfa sobre todo \(\mathbb{C}\)
  se llama \emph{entera}.%
    \index{C (numeros complejos)@\(\mathbb{C}\) (números complejos)!funcion entera@función entera|textbfhy}

  Las siguientes propiedades de la derivada
  se demuestran básicamente
  cambiando \(x\) por \(z\)
  en las demostraciones respectivas para los reales:
  \begin{lemma}
    \label{lem:complex-derivative-rules}
    Sean \(f\) y \(g\) diferenciables en \(z \in \mathbb{C}\),
    sea \(c \in \mathbb{C}\),
    sea \(n \in \mathbb{Z}\),
    y sea \(h\) diferenciable en \(g(z)\).
    Entonces:
    \begin{enumerate}[label=(\roman*), ref=(\roman*)]
    \item
      \((c f(z))'
	  = c f'(z)\)
    \item
      \((f(z) + g(z))'
	  = f'(z) + g'(z)\)
    \item
      \((f(z) \cdot g(z))'
	  = f'(z) \cdot g(z) + f(z) \cdot g'(z)\)
    \item
      Siempre que \(g(z) \ne 0\) es
      \((f(z) / g(z))'
	  = (f'(z) \cdot g(z) - f(z) \cdot g'(z)) / g^2 (z)\)
    \item
      \((z^n)'
	  = n z^{n - 1}\)
    \item
      \((h(g(z)))'
	  = h'(g(z)) \cdot g'(z)\)
    \end{enumerate}
  \end{lemma}
  Un último resultado se refiere a funciones inversas.
  \begin{lemma}
    \label{lem:complex-derivative-inverse}
    Sean \(G\) y \(H\) conjuntos abiertos en \(\mathbb{C}\),
    \(f \colon G \rightarrow H\) una biyección
    con inversa \(g \colon H \rightarrow G\),
    y sea \(z_0 \in H\).
    Si \(f\) es diferenciable en \(g(z_0)\)
    con \(f'(g(z_0)) \ne 0\),
    y \(g\) es continua en \(z_0\),
    entonces \(g\) es diferenciable en \(z_0\),
    y:
    \begin{equation*}
      g'(z_0)
	= \frac{1}{f'(g(z_0))}
    \end{equation*}
  \end{lemma}
  \begin{proof}
    Por definición:
    \begin{equation*}
      g'(z_0)
	= \lim_{z \rightarrow z_0} \frac{g(z) - g(z_0)}{z - z_0}
	= \lim_{z \rightarrow z_0}
	    \frac{g(z) - g(z_0)}{f(g(z)) - f(g(z_0))}
	= \lim_{z \rightarrow z_0}
	    \frac{1}{\frac{f(g(z)) - f(g(z_0))}{g(z) - g(z_0)}}
    \end{equation*}
    Dado que \(g\) es continua en \(z_0\),
    \(g(z) \rightarrow g(z_0)\) cuando \(z \rightarrow z_0\),
    lo que da:
    \begin{equation*}
      g'(z_0)
	= \lim_{w \rightarrow g(z_0)}
	    \frac{1}{\frac{f(w) - f(g(z_0))}{w - g(z_0)}}
    \end{equation*}
    El denominador es continuo y diferente de cero,
    por el lema~\ref{lem:complex-limits} tenemos lo prometido.
  \end{proof}
  Un resultado importante es:
  \begin{theorem}
    \label{theo:complex-zero-derivative}
    Si \(f'(z) = 0\) para todo \(z\) en una región \(D\),
    entonces \(f(z)\) es constante en \(D\).
  \end{theorem}
  Elegimos un punto fijo \(z_0 \in D\),
  y conectaremos un punto arbitrario \(z \in D\) con \(z_0\)
  mediante una poligonal,
  y demostraremos que \(f\) es constante sobre esa poligonal.
  Como \(z\) es arbitrario,
  \(f\) es constante en \(D\).
  \begin{proof}
    Sean \(z_0, z \in D\).
    Por el teorema~\ref{theo:connected=curve-inside}
    hay una poligonal \([z_0, z_1, \dotsc, z]\) en \(D\)
    que los conecta.
    Sea \(f(z) = u + \mathrm{i} v\),
    si \(f'(z) = 0\) de las ecuaciones de Cauchy-Riemann
    es:
    \begin{equation*}
      \frac{\partial u}{\partial x}
	= \frac{\partial u}{\partial y}
	= \frac{\partial v}{\partial x}
	= \frac{\partial v}{\partial y}
	= 0
    \end{equation*}
    Considerando uno de los tramos,
    digamos \([z_k, z_{k + 1}]\)
    vemos que esta recta
    queda descrita por una parametrización de la forma
    \(z_k + (t - t_k) (z_{k + 1} - z_k) / (t_{k + 1} - t_k)
       = \alpha t + \beta\).
    Resulta que la derivada de \(f\) respecto a \(t\)
    a lo largo de \([z_k, z_{k + 1}]\)
    se anula:
    \begin{equation*}
      \frac{f(\alpha (t + h) + \beta)
	     - f(\alpha t + \beta)}
	   {h}
	= \frac{\partial u}{\partial x}
	     \frac{\mathrm{d} x}{\mathrm{d} t}
	     + \mathrm{i} \,
		 \frac{\partial v}{\partial x}
		 \frac{\mathrm{d} y}{\mathrm{d} t}
	= 0
    \end{equation*}
    Acá usamos la fórmula~\eqref{eq:complex-derivative-uv}
    para la derivada compleja.
    Podemos aplicar el teorema del valor medio para funciones reales
    (componente a componente)
    a \(f\) como función de \(t\).
    Como la derivada se anula,
    esto nos dice que no hay cambios:
    \begin{equation*}
      f(z_{k + 1}) - f(z_k)
	= 0
    \end{equation*}
    Esto es lo que queríamos demostrar.
  \end{proof}

\section{Funciones elementales}
\label{sec:complex-elementary-functions}
\index{C (numeros complejos)@\(\mathbb{C}\) (números complejos)!funciones elementales}

  Es claro que los polinomios son funciones enteras,
  y si se restringen a argumentos reales
  son simplemente las funciones conocidas.
  Un poquito más delicado es el caso de funciones racionales,
  como:
  \begin{equation*}
    \frac{z^3 - 3 z + 2}{z^2 + 1}
  \end{equation*}
  Esta función es holomorfa
  salvo en los puntos \(z = \pm \mathrm{i}\),
  donde el denominador se anula.
  Como función real es continua y tiene derivada en todas partes.

  Veamos la función exponencial.
  Parece razonable extender la propiedad básica
    \(\mathrm{e}^{x_1} \cdot \mathrm{e}^{x_2}
	= \mathrm{e}^{x_1 + x_2}\)
  a argumentos complejos,
  lo que para \(x, y \in \mathbb{R}\) da:
  \begin{equation*}
    \mathrm{e}^{x + \mathrm{i} y}
      = \mathrm{e}^x \cdot \mathrm{e}^{\mathrm{i} y}
  \end{equation*}
  Esto,
  con la convención~\eqref{eq:imaginary-exponential},
  resulta en:
  \begin{definition}
    Para \(x, y \in \mathbb{R}\) definimos:
    \begin{equation}
      \index{C (numeros complejos)@\(\mathbb{C}\) (números complejos)!exponencial|textbfhy}
      \label{eq:complex-exponential}
      \mathrm{e}^{x + \mathrm{i} y}
	= \mathrm{e}^x ( \cos y + \mathrm{i} \, \sin y )
    \end{equation}
  \end{definition}
  Escribiendo:
  \begin{equation*}
    \mathrm{e}^z
      = u(x, y) + \mathrm{i} v(x, y)
  \end{equation*}
  vemos que se satisfacen
  las ecuaciones de Cauchy-Riemann~\eqref{eq:Cauchy-Riemann}%
    \index{Cauchy-Riemann, ecuaciones de}
  y que las derivadas son continuas para todo \(z \in \mathbb{C}\).
  Esta función es entera.
  Vemos también que:
  \begin{equation}
    \label{eq:complex-exponential-derivative}
    \frac{\mathrm{d}}{\mathrm{d} z} \mathrm{e}^z
      = \frac{\partial u}{\partial x}
	  + \mathrm{i} \, \frac{\partial v}{\partial x}
      = \mathrm{e}^x \cos y + \mathrm{i} \, \mathrm{e}^x \sin y
      = \mathrm{e}^z
  \end{equation}
  Ya vimos que
    \(\mathrm{e}^{\mathrm{i} y_1} \cdot \mathrm{e}^{\mathrm{i} y_2}
	= \mathrm{e}^{\mathrm{i} (y_1 + y_2)}\),
  y
    \(\mathrm{e}^{x_1} \cdot \mathrm{e}^{x_2}
	= \mathrm{e}^{x_1 + x_2}\),
  que en conjunto hacen que
    \(\mathrm{e}^{z_1} \cdot \mathrm{e}^{z_2}
	= \mathrm{e}^{z_1 + z_2}\),
  como buscábamos.
  Notamos que:
  \begin{equation}
    \label{eq:complex-exponential-modulus}
    \left\lvert \mathrm{e}^z \right\rvert
      = \left\lvert \mathrm{e}^x \right\rvert \cdot
	  \left\lvert \cos y + \mathrm{i} \, \sin y \right\rvert
      = \left\lvert \mathrm{e}^x \right\rvert \cdot
	  \left( \cos^2 y + \sin^2 y \right)^{1/2}
      = \mathrm{e}^x
  \end{equation}
  Como \(\mathrm{e}^x\) nunca se anula para \(x \in \mathbb{R}\),
  y \(\cos y\) y \(\sin y\) no se anulan juntas,
  \(\mathrm{e}^z \ne 0\) para todo \(z \in \mathbb{C}\).

  En nuestra lista siguen las funciones trigonométricas.%
    \index{C (numeros complejos)@\(\mathbb{C}\) (números complejos)!funciones trigonometricas@funciones trigonométricas}
  La convención~\eqref{eq:imaginary-exponential}
  para \(\pm \phi\) da:
  \begin{equation*}
    \mathrm{e}^{\mathrm{i} \, \phi}
      = \cos \phi + \mathrm{i} \, \sin \phi
    \qquad
    \mathrm{e}^{- \mathrm{i} \, \phi}
      = \cos \phi - \mathrm{i} \, \sin \phi
  \end{equation*}
  De este sistema de ecuaciones:
  \begin{equation*}
    \cos \phi
      = \frac{\mathrm{e}^{\mathrm{i} \, \phi}
		+ \mathrm{e}^{- \mathrm{i} \, \phi}}
	     {2}
    \qquad
    \sin \phi
      = \frac{\mathrm{e}^{\mathrm{i} \, \phi}
		- \mathrm{e}^{- \mathrm{i} \, \phi}}
	     {2 \mathrm{i}}
  \end{equation*}
  lo que sugiere definir:
  \begin{align}
    \cos z
      &= \frac{\mathrm{e}^{\mathrm{i} z} + \mathrm{e}^{- \mathrm{i} z}}{2}
	    \label{eq:complex-cos} \\
    \sin z
      &= \frac{\mathrm{e}^{\mathrm{i} z} - \mathrm{e}^{- \mathrm{i} z}}
	      {2 \mathrm{i}}
	    \label{eq:complex-sin}
  \end{align}
  Es claro que estas funciones son enteras,
  y cumplen las identidades trigonométricas conocidas
  para los reales.
  Cuidado,
  estas funciones no son acotadas en \(\mathbb{C}\).
  El lector escéptico podrá entretenerse
  demostrando algunas identidades,
  como \(\cos^2 z + \sin^2 z = 1\)
  o las fórmulas para sumas de ángulos.

  De las relaciones~\eqref{eq:complex-cos} y~\eqref{eq:complex-sin}
  vemos que para las funciones hiperbólicas:%
    \index{C (numeros complejos)@\(\mathbb{C}\) (números complejos)!funciones hiperbolicas@funciones hiperbólicas}
  \begin{align}
    \cosh z
      &= \frac{\mathrm{e}^z + \mathrm{e}^{- z}}{2}
       = \cos \mathrm{i} z
	    \label{eq:complex-cosh} \\
    \sinh z
      &= \frac{\mathrm{e}^z  - \mathrm{e}^{- z}}{2}
       = - \mathrm{i} \, \sin \mathrm{i} z
	    \label{eq:complex-sinh}
  \end{align}
  Estas funciones también son enteras.
  Podemos definir las demás funciones trigonométricas
  e hiperbólicas usando las mismas definiciones que para los reales.

\section{Logaritmos y potencias}
\label{sec:complex-logarithm}

  Entre los reales,
  el logaritmo es simplemente el inverso de la exponencial.
  Esto está perfectamente bien definido en ese caso.
  Entre los complejos,
  sin embargo,
  la exponencial es una función periódica
  (el período es \(2 \pi \mathrm{i}\)).
  Como hay infinitas soluciones a la ecuación \(\mathrm{e}^z = w\)
  siempre que \(w \ne 0\),
  no podemos esperar definir una función análoga,
  deberemos dar algunos rodeos.
  En particular,
  para \(x \in \mathbb{R}\) con \(x > 0\)
  realmente es \(\log x = \ln x + 2 k \pi \mathrm{i}\),%
    \index{C (numeros complejos)@\(\mathbb{C}\) (números complejos)!logaritmo}
  ya que debemos considerar los posibles argumentos.
  Acá \(\ln x\) es el familiar logaritmo natural de los reales.
  En los reales
  simplemente dejamos de lado la componente imaginaria.

  Para \(z \ne 0\) definimos:
  \begin{equation}
    \label{eq:log-definition}
    \log z
      = \ln \lvert z \rvert + \mathrm{i} \, \arg z
  \end{equation}
  Con esto tenemos el caso emblemático:
  \begin{equation*}
    \log(-1)
      = \ln 1 + \mathrm{i} \, \arg(-1)
      = (2 k + 1) \pi \mathrm{i}
  \end{equation*}
  Esto cumple el familiar:
  \begin{equation*}
    \mathrm{e}^{\log z}
      = \mathrm{e}^{\ln \lvert z \rvert + \mathrm{i} \, \arg z}
      = \mathrm{e}^{\ln \lvert z \rvert}
	  \cdot \mathrm{e}^{\mathrm{i} \, \arg z}
      = z
  \end{equation*}
  Pero aparece una complicación.
  Con el ya tradicional \(z = x + \mathrm{i} y\) tenemos:
  \begin{equation*}
    \log \left( \mathrm{e}^z \right)
      = \ln \mathrm{e}^x + \mathrm{i} \, \arg \mathrm{e}^z
      = x + (y + 2 k \pi) \mathrm{i}
      = z + 2 k \pi \mathrm{i}
  \end{equation*}
  Acá \(k\) es un entero cualquiera.
  De la misma manera,
  para un entero cualquiera \(k\):
  \begin{align*}
    \log (w z)
      &= \ln (\lvert w \rvert \cdot \lvert z \rvert)
	   + \mathrm{i} \, \arg (w z)
       = \ln \lvert w \rvert + \mathrm{i} \, \arg w
	   + \ln \lvert z \rvert + \mathrm{i} \, \arg z
	   + 2 k \pi \mathrm{i} \\
      &= \log w + \log z + 2 k \pi \mathrm{i}
  \end{align*}
  Definimos la \emph{rama principal} del logaritmo mediante:%
    \index{C (numeros complejos)@\(\mathbb{C}\) (números complejos)!logaritmo!rama principal}
  \begin{equation}
    \label{eq:log-principal-branch}
    \Log z
      = \ln \lvert z \rvert + \mathrm{i} \, \Arg z
  \end{equation}
  Si \(z = x\),
  un real positivo,
  es:
  \begin{equation*}
    \Log x
      = \ln x + \mathrm{i} \, \Arg x
      = \ln x
  \end{equation*}
  Vemos que la nueva función es una extensión del logaritmo real.
  La definición del argumento principal con este rango,
  en vez del mucho más natural \(0 \le \phi < 2 \pi\),
  es precisamente para asegurar esta coincidencia
  sin estar equilibrándonos en el borde del abismo
  a lo largo de la línea real.

  La función \(\Log\) es holomorfa en muchas partes.
  No está definida para \(z = 0\),
  y tiene un corte en la línea real negativa.
  Sea \(z_0 = x_0 + \mathrm{i} y_0\)
  tal que \(\Log z_0\) esté definido,
  y veamos su derivada:
  \begin{equation*}
    \lim_{z \rightarrow z_0} \frac{\Log z - \Log z_0}{z - z_0}
      = \lim_{z \rightarrow z_0}
	  \frac{\Log z - \Log z_0}
	       {\mathrm{e}^{\Log z} - \mathrm{e}^{\Log z_0}}
  \end{equation*}
  Deberemos restringirnos a trabajar en una región
  que no incluya los puntos conflictivos mencionados.
  Con \(w = \Log z\) y \(w_0 = \Log z_0\),
  notando que \(w \rightarrow w_0\)
  cuando \(z \rightarrow z_0\),
  esto es:
  \begin{align*}
    \lim_{z \rightarrow z_0} \frac{\Log z - \Log z_0}{z - z_0}
      &= \lim_{w \rightarrow w_0}
	   \frac{w - w_0}{\mathrm{e}^w - \mathrm{e}^{w_0}}
       = \frac{1}{e^{w_0}} \\
      &= \frac{1}{z_0}
  \end{align*}

  \begin{figure}[ht]
    \centering
    \pgfimage{images/log-domain}
    \caption{Dominio alternativo de $\log z$}
    \label{fig:log-domain}
  \end{figure}
  Nótese que la restricción del argumento es un tanto arbitraria,
  si nos interesa trabajar en la región \(D\)
  de la figura~\ref{fig:log-domain},
  podemos restringir el argumento
  al rango \([\pi / 4,	9 \pi / 4)\).

  Ahora estamos en condiciones
  de definir potencias arbitrarias de \(z\).%
    \index{C (numeros complejos)@\(\mathbb{C}\) (números complejos)!potencia}
  La definición obvia es:
  \begin{equation*}
    z^c
      = \mathrm{e}^{c \log z}
  \end{equation*}
  Hay muchos valores de \(\log z\),
  con lo que pueden haber muchos valores de \(z^c\).
  Al lector atento no le extrañará
  que se le llame el \emph{valor principal} de \(z^c\)
  a \(\mathrm{e}^{c \Log z}\).%
    \index{C (numeros complejos)@\(\mathbb{C}\) (números complejos)!potencia!rama principal}
  En caso que \(c = n\),
  un entero,
  la definición da:
  \begin{align*}
    z^n
      &= \mathrm{e}^{n \log z}
       = \mathrm{e}^{n (\Log z + 2 k \pi \mathrm{i})}
       = \mathrm{e}^{n \Log z}
	   \cdot \mathrm{e}^{2 n k \pi \mathrm{i}} \\
      &= \mathrm{e}^{n \Log z} \\
      &= \lvert z \rvert^n \cdot \mathrm{e}^{\mathrm{i} n \Arg z}
  \end{align*}
  Exactamente como debiera ser.
  El lector interesado verificará que si \(c\) es un racional,
  la fórmula da todos los valores esperados.

  Esto introduce un nuevo problema.
  Tenemos una definición de potencias que aplicada a \(z = e\)
  puede dar infinitos valores para \(\mathrm{e}^c\).
  Hasta acá hemos asumido simplemente
  que para \(z = x + \mathrm{i} y\)
  es:
  \begin{equation*}
    \mathrm{e}^z
      = \exp(z)
      = \mathrm{e}^x (\cos y + \mathrm{i} \, \sin y)
  \end{equation*}
  Equivalentemente,
  \(\mathrm{e}^z\) se refiere al valor principal de esta expresión.
  Esta es la convención que adoptaremos,
  que por lo demás es universal.

\section{Integrales}
\label{sec:complex-integrals}
\index{C (numeros complejos)@\(\mathbb{C}\) (números complejos)!integral}

  La integral en los reales
  es simplemente a lo largo de la línea real.
  Al integrar en los complejos
  hay muchos caminos distintos que podemos seguir.
  De todas formas,
  una definición natural para la integral de la función \(f\)
  a lo largo del camino suave
    \(\gamma \colon [a, b] \rightarrow \mathbb{C}\)
  es seguir la definición de la integral de Riemann en los reales.
  Imaginemos una subdivisión del rango \([a, b]\) en \(n\) tramos
  \([t_{k - 1}, t_k]\),
  donde \(1 \le k \le n\).
  Vemos que al tramo \([t_{k - 1}, t_k]\)
  corresponde un arco \([z_{k - 1}, z_k]\) del camino \(\gamma\),
  un punto \(t_k^*\) en el tramo \([t_{k - 1}, t_k]\)
  da un punto \(z_k^*\) en el arco correspondiente.
  Supongamos ahora dado \(\epsilon > 0\) cualquiera.
  Dada una partición \(P\) tal que se cumple para todo tramo
  que \(\lvert z_k - z_{k - 1} \rvert < \epsilon\),
  eligiendo puntos \(t_k^*\) en cada tramo
  calculamos la suma:
  \begin{equation*}
    S(P)
      = \sum_{1 \le k \le n} f(z_k^*) (z_k - z_{k - 1})
  \end{equation*}
  Si estas sumas tienden al valor \(L\)
  cuando \(\epsilon \rightarrow 0\),
  llamamos a este límite el valor de la integral,
  y anotamos:
  \begin{equation*}
    \int_\gamma f(z) \, \mathrm{d} z
      = L
  \end{equation*}
  Podemos expresar la suma en términos de \(t\):
  \begin{align*}
    S(p)
      &= \sum_{1 \le k \le n}
	   f(\gamma(t_k^*)) (\gamma(t_k^*) - \gamma(t_{k - 1}^*)) \\
      &= \sum_{1 \le k \le n}
	   f(\gamma(t_k^*)) (\gamma(t_k^*) - \gamma(t_{k - 1}^*)) \\
      &= \sum_{1 \le k \le n}
	   f(\gamma(t_k^*)) \,
	     \frac{\gamma(t_k^*) - \gamma(t_{k - 1}^*)}
		  {t_k^* - t_{k - 1}^*}
	     \cdot (t_k^* - t_{k - 1}^*)
  \end{align*}
  Si \(\epsilon \rightarrow 0\),
  vemos que esto tiende a:
  \begin{equation}
    \label{eq:complex-integral-parametrized}
    \int_\gamma f(z) \, \mathrm{d} z
      = \int_a^b f(\gamma(t)) \gamma'(t) \, \mathrm{d} t
  \end{equation}
  Esta misma fórmula indica que el valor de la integral
  es independiente de la parametrización de la curva.

  Una cota que usaremos frecuentemente es la siguiente.
  \begin{lemma}[Cota para integrales complejas]
    \label{lem:complex-integral-bound}
    Supóngase que hay un número \(M\) tal que \(\lvert f(z) \rvert \le M\)
    para todo \(z\) en la curva suave
      \(\gamma \colon [a, b] \rightarrow \mathbb{C}\),
    y sea \(l_\gamma\) el largo de la curva \(\gamma\).
    Entonces:
    \begin{equation}
      \label{eq:complex-integral-bound}
      \left\lvert \int_\gamma f(z) \, \mathrm{d} z \right\rvert
	\le M l_\gamma
    \end{equation}
  \end{lemma}
  \begin{proof}
    Usando la desigualdad triangular
    sobre la definición de la integral,%
      \index{desigualdad triangular}
    vemos que:
    \begin{equation*}
      \left\lvert \int_\gamma f(z) \, \mathrm{d} z \right\rvert
	= \left\lvert \int_a^b f(z) \gamma'(t)
	    \, \mathrm{d} t \right\rvert
	\le \int_a^b
	      \lvert f(z) \rvert
		\cdot	\lvert \gamma'(t) \rvert \, \mathrm{d} t
	\le M \int_a^b \lvert \gamma'(t) \rvert \, \mathrm{d} t
    \end{equation*}
    Si describimos \(\gamma(t) = x(t) + \mathrm{i} y(t)\) tenemos:
    \begin{equation*}
      \int_a^b \lvert \gamma'(t) \rvert \, \mathrm{d} t
	= \int_a^b
	    \left(
	      \left( x'(t) \right)^2 + \left( y'(t) \right)^2
	    \right)^{1/2} \, \mathrm{d} t
    \end{equation*}
    que reconocemos como el largo \(l_\gamma\) de la curva.
  \end{proof}

\subsection{Integrales y antiderivadas}
\label{sec:complex-integral-antiderivative}

  Supongamos nuevamente
  un camino suave \(\gamma\) entre \(a\) y \(b\),
  una función \(g(z)\) diferenciable en \(\gamma\),
  y consideremos \(t \in [a, b]\).
  Veamos cuál es la derivada de \(g(\gamma(t))\).
  Resulta ser exactamente como nos imaginamos.
  Primero,
  con \(g(x + \mathrm{i} y) = u(x, y) + \mathrm{i} v(x, y)\)
  y \(\gamma(t) = x(t) + \mathrm{i} y(t)\),
  es:
  \begin{equation*}
    g(\gamma(t))
      = u(x(t), y(t)) + \mathrm{i} v(x(t), y(t))
  \end{equation*}
  Enseguida:
  \begin{align*}
    \frac{\mathrm{d}}{\mathrm{d} t} \, g(\gamma(t))
      &= \frac{\partial u}{\partial x} \,
	    \frac{\mathrm{d} x}{\mathrm{d} t}
	    + \frac{\partial u}{\partial y} \,
		\frac{\mathrm{d} y}{\mathrm{d} t}
	  + \mathrm{i}
	      \left(
		\frac{\partial v}{\partial x} \,
		  \frac{\mathrm{d} x}{\mathrm{d} t}
		  + \frac{\partial v}{\partial y} \,
		      \frac{\mathrm{d} y}{\mathrm{d} t}
	      \right) \\
  \intertext{Usando las ecuaciones de Cauchy-Riemann:%
	       \index{Cauchy-Riemann, ecuaciones de}}
    \frac{\mathrm{d}}{\mathrm{d} t} \, g(\gamma(t))
      &= \frac{\partial u}{\partial x} \,
	   \frac{\mathrm{d} x}{\mathrm{d} t}
	     - \frac{\partial v}{\partial x} \,
		 \frac{\mathrm{d} y}{\mathrm{d} t}
	   + \mathrm{i} \,
	       \left(
		\frac{\partial v}{\partial x} \,
		  \frac{\mathrm{d} x}{\mathrm{d} t}
		  + \frac{\partial u}{\partial x} \,
		      \frac{\mathrm{d} y}{\mathrm{d} t}
	       \right) \\
      &= \left(
	   \frac{\partial u}{\partial x}
	     + \mathrm{i} \, \frac{\partial v}{\partial x}
	 \right)
	   \cdot \left(
		   \frac{\mathrm{d} x}{\mathrm{d} t}
		     + \mathrm{i} \, \frac{\mathrm{d} y}
					  {\mathrm{d} t}
		 \right) \\
      &= g'(\gamma(t)) \gamma'(t)
  \end{align*}

  Volvamos a las integrales ahora.
  Sea una región \(D\)
  y una función \(F \colon D \rightarrow \mathbb{C}\)
  tal que en \(D\) tenemos \(F'(z) = f(z)\).
  Supongamos un camino suave \(\gamma \colon [a, b] \rightarrow D\).
  Sabemos de arriba que:
  \begin{equation*}
    \frac{\mathrm{d}}{\mathrm{d} t} \, F(\gamma(t))
      = F'(\gamma(t)) \gamma'(t)
      = f(\gamma(t)) \gamma'(t)
  \end{equation*}
  Entonces:
  \begin{equation}
    \label{eq:complex-integral-antiderivative}
    \int_\gamma f(\zeta) \, \mathrm{d} \zeta
      = \int_a^b f(\gamma(t)) \gamma'(t) \, \mathrm{d} t
      = \int_a^b \frac{\mathrm{d}}{\mathrm{d} t} \, F(\gamma(t))
	  \, \mathrm{d} t
      = F(\gamma(b)) - F(\gamma(a))
  \end{equation}
  La última relación
  resulta del teorema fundamental del cálculo integral.

  Muy agradable,
  la integral depende únicamente
  de los puntos inicial y final del camino.
  En particular,
  si el camino es cerrado,
  la integral es cero.

  El recíproco ahora.
  Supongamos que la integral de la función continua \(f\)
  no depende del camino,
  vale decir podemos tomar un punto \(z_0 \in D\)
  y definir una función:
  \begin{equation*}
    F(z)
      = \int_{\gamma_z} f(\zeta) \, \mathrm{d} \zeta
  \end{equation*}
  donde el camino \(\gamma_z\)
  comienza en \(z_0\) y termina en \(z\),
  sin salir de \(D\).
  Sabemos que de existir tales caminos para cada elección de \(z\)
  la región \(D\) debe ser conexa.
  Evaluemos la derivada de \(F\):
  \begin{equation*}
    \lim_{h \rightarrow 0}
      \frac{F(z + h) - F(z)}{h}
      = \lim_{h \rightarrow 0}
	  \frac{1}{h} \int_{\gamma_h} f(\zeta) \, \mathrm{d} \zeta
  \end{equation*}
  Acá \(\gamma_h\)
  es un camino que comienza en \(z\) y termina en \(z + h\).
  Vemos también que:
  \begin{align*}
    \int_{\gamma_h} \, \mathrm{d} \zeta
      &= h \\
    \int_{\gamma_h} f(z) \, \mathrm{d} \zeta
      &= h f(z)
  \end{align*}
  Con esto:
  \begin{equation*}
    \lim_{h \rightarrow 0}
      \frac{F(z + h) - F(z)}{h}
	- f(z)
      = \lim_{h \rightarrow 0}
	  \frac{1}{h} \, \int_{\gamma_h} (f(\zeta) - f(z))
	  \, \mathrm{d} \zeta
  \end{equation*}
  Ahora bien,
  como las integrales no dependen del camino
  podemos calcularla sobre la recta de \(z\) a \(z + h\):
  \begin{equation*}
    \left\lvert
      \frac{1}{h} \, \int_{\gamma_h} (f(\zeta) - f(z))
	\, \textrm{d} \zeta
    \right\rvert
      \le \left\lvert \frac{1}{h} \right\rvert
	    \cdot \lvert h \rvert
	    \cdot \max \left\{ \lvert f(\zeta) - f(z) \rvert
			   \colon \zeta \in [z, z + h] \right\}
  \end{equation*}
  Como \(f\) es continua,
  cuando \(h \rightarrow 0\) esto tiende a cero,
  y \(F'(z) = f(z)\),
  como esperábamos.

  En resumen:
  \begin{theorem}
    \label{theo:complex-integral=antiderivative}
    Sea \(D\) una región conexa,
    y sea \(f \colon D \rightarrow \mathbb{C}\) continua.
    Entonces \(f\) tiene antiderivada en \(D\) si y solo si
    la integral entre dos puntos de \(D\)
    es independiente del camino.
    El valor de la integral es la diferencia
    entre los valores de la antiderivada.
  \end{theorem}
  Pero también hemos demostrado:
  \begin{theorem}[Morera]
    \label{theo:Morera}
    Sea \(f\) continua en \(D\) tal que
    para toda curva suave cerrada simple \(\gamma \subset D\):
    \begin{equation*}
      \int_\gamma f(z) \, \mathrm{d} z
	= 0
    \end{equation*}
    Entonces \(f\) es holomorfa en \(D\).
  \end{theorem}

\subsection{El teorema de Cauchy}
\label{sec:Cauchy-theorem}

  Nos interesa evaluar integrales sobre caminos cerrados,
  en particular demostrar que tales integrales
  no dependen del detalle del camino.
  Para ello requeriremos algunas herramientas adicionales.
  \begin{definition}
    \index{curva cerrada!homotopica@homotópica|textbfhy}
    Sean \(\gamma_0\) y \(\gamma_1\) curvas cerradas
    en el conjunto abierto \(D \subseteq \mathbb{C}\),
    parametrizadas por \(\gamma_0 \colon [0, 1] \rightarrow D\)
    y \(\gamma_1 \colon [0, 1] \rightarrow D\),
    respectivamente.
    Decimos que \(\gamma_0\)
    es \emph{\(D\)-homotópica} a \(\gamma_1\),
    en símbolos \(\gamma_0 \sim_D \gamma_1\),
    si hay una función continua \(h \colon [0, 1]^2 \rightarrow D\)
    tal que:
    \begin{align*}
      h(t, 0)
	&= \gamma_0(t) \\
      h(t, 1)
	&= \gamma_1(t) \\
      h(0, s)
	&= h(1, s)
    \end{align*}
    Si \(D\) es conexa
    y tal que toda curva cerrada simple es homotópica
    a un punto,
    se dice que \(D\) es \emph{conexa simple}.%
      \index{C (numeros complejos)@\(\mathbb{C}\) (números complejos)!region conexa simple@región conexa simple}
  \end{definition}
  \begin{figure}[ht]
    \centering
    \pgfimage{images/homotopy}
    \caption{Ejemplos de homotopía}
    \label{fig:homotopy}
  \end{figure}
  La idea es que \(h(t, s)\) es una curva en \(D\),
  la última condición asegura que sea siempre cerrada.
  Cambiando \(s\) cambia la curva,
  que va en forma continua de \(\gamma_0\) a \(\gamma_1\).
  Nótese también que las curvas se recorren todas
  en dirección de \(t\) creciente,
  arbitrariamente definimos la dirección positiva
  como aquella en que el interior encerrado por la curva
  queda a su izquierda.
  Es simple ver que la homotopía
  es una relación de equivalencia entre curvas.
  Frecuentemente consideraremos un punto aislado como una ``curva''
  de largo cero.
  Una región conexa simple no tiene ``agujeros''.

  Mucho de lo que viene a continuación
  se basa en el siguiente teorema.
  La demostración es de Beck, Marchesi, Pixton y Sabalka~%
    \cite{beck12:_first_course_compl_analysis}.
  \begin{theorem}[Cauchy]
    \index{Cauchy, teorema de}
    Sea \(D \subseteq \mathbb{C}\) una región abierta,
    \(f\) holomorfa en \(D\),
    y \(\gamma_0 \sim_D \gamma_1\) vía una homotopía
    con segundas derivadas continuas
    y que coinciden para \(s = 0\) y \(s = 1\).
    Entonces:
    \begin{equation*}
      \int_{\gamma_0} f(z) \, \mathrm{d} z
	= \int_{\gamma_1} f(z) \, \mathrm{d} z
    \end{equation*}
  \end{theorem}
  La condición de suavidad de la homotopía puede relajarse bastante,
  pero la demostración se hace muy compleja.
  Para las aplicaciones de nuestro interés
  esta condición se cumple.
  \begin{proof}
    Sea \(h(t, s)\) la homotopía de \(\gamma_0\) a \(\gamma_1\),
    y definamos \(\gamma_s\) como la curva definida por \(h\)
    para \(s\).
    Consideremos la función:
    \begin{equation*}
      I(s)
	= \int_{\gamma_s} f(z) \, \mathrm{d} z
	= \int_0^1 f(h(t, s)) \frac{\partial h}{\partial t}
	    \, \mathrm{d} t
    \end{equation*}
    Esta expresión
    resulta de~\eqref{eq:complex-integral-parametrized}.
    Demostraremos que \(I(s)\) es constante,
    con lo que se cumple lo prometido como \(I(0) = I(1)\).
    Por la regla de Leibnitz:%
      \index{Leibnitz, regla de}
    \begin{equation*}
      \frac{\mathrm{d}}{\mathrm{d} s} \, I(s)
	= \frac{\mathrm{d}}{\mathrm{d} s}
	    \int_0^1 f(h(t, s)) \,
	      \frac{\partial h}{\partial t} \, \mathrm{d} t
	= \int_0^1 \frac{\partial}{\partial s} \,
	    \left(
	      f(h(t, s)) \frac{\partial h}{\partial t}
	    \right)
	    \, \mathrm{d} t
    \end{equation*}
    Usando el menú completo
    de propiedades de las derivadas parciales:
    \begin{align*}
      \frac{\mathrm{d}}{\mathrm{d} s} I(s)
	&= \int_0^1
	     \left(
	       f'(h(t, s))
		 \frac{\partial h}{\partial s}
		     \frac{\partial h}{\partial t}
		 + f(h(t, s))
		     \frac{\partial^2 h}{\partial s \partial t}
	     \right)
	     \, \mathrm{d} t \\
	&= \int_0^1
	     \left(
	       f'(h(t, s))
		 \frac{\partial h}{\partial t}
		    \frac{\partial h}{\partial s}
		 + f(h(t, s))
		    \frac{\partial^2 h}{\partial t \partial s}
	     \right)
	     \, \mathrm{d} t \\
	&= \int_0^1 \frac{\partial}{\partial t}
	     \left(
	       f(h(t, s)) \frac{\partial h}{\partial s}
	     \right)
	     \, \mathrm{d} t
    \end{align*}
    Aplicando el teorema fundamental del cálculo integral%
      \index{calculo integral, teorema fundamental del@cálculo integral, teorema fundamental del}
    por separado a las componentes real e imaginaria,
    y recordando la condición \(h(0, s) = h(1, s)\)
    y que las respectivas derivadas coinciden:
    \begin{equation*}
      \frac{\mathrm{d}}{\mathrm{d} s} I(s)
	= f(h(1, s)) \, \frac{\partial h}{\partial s} (1, s)
	   - f(h(0, s)) \, \frac{\partial h}{\partial s} (0, s)
	= 0
    \end{equation*}
    Si la derivada compleja es cero,
    lo son las derivadas parciales
    de las componentes real e imaginaria,
    y así la función es constante.
  \end{proof}
  Una consecuencia inmediata es que si \(D\)
  es una región conexa simple
  en la cual la función \(f\) es holomorfa
  entonces para todo camino suave cerrado \(\gamma \subset D\):
  \begin{equation*}
    \int_\gamma f(z) \, \mathrm{d} z
      = 0
  \end{equation*}
  Esto porque en este caso
  cualquier curva \(\gamma \subset D\) es homotópica con un punto,
  y claramente la integral para un punto es cero.
  Por el teorema~\ref{theo:complex-integral=antiderivative}
  en tales regiones la integral es independiente del camino
  y existe una antiderivada.

\subsection{La fórmula integral de Cauchy}
\label{sec:Cauchy-integral-formula}

  Tenemos el siguiente resultado notable:
  \begin{theorem}[Fórmula integral de Cauchy]
    \index{Cauchy, formula integral de@Cauchy, fórmula integral de|textbfhy}
    \label{theo:Cauchy-integral-formula}
    Sea \(f\) holomorfa en la región \(D\)
    que contiene el camino cerrado simple \(\gamma\),
    con la orientación habitual que el interior está a la izquierda,
    y suponga que \(z_0\) está al interior de \(\gamma\).
    Entonces:
    \begin{equation*}
      f(z_0)
	= \frac{1}{2 \pi \mathrm{i}} \,
	    \int_\gamma	 \frac{f(z)}{z - z_0} \, \mathrm{d} z
    \end{equation*}
  \end{theorem}
  \begin{proof}
    Sea \(\epsilon > 0\) arbitrario.
    Sabemos que \(f\) es continua en \(z_0\),
    por lo que existe \(\delta > 0\) tal que
    \(\lvert f(z) - f(z_0) \rvert < \epsilon\)
    siempre que \(\lvert z - z_0 \rvert < \delta\).
    Sea ahora \(r > 0\) tal que \(r < \delta\)
    y la circunferencia
      \(C_0 = \{ z \colon \lvert z - z_0 \rvert = r\}\)
    está dentro de \(\gamma\).
    Entonces \(f(z) / (z - z_0)\) es holomorfa
    en la región entre \(\gamma\) y \(C_0\),
    por lo que del teorema de Cauchy:
    \begin{equation*}
      \int_\gamma \frac{f(z)}{z - z_0} \, \mathrm{d} z
	= \int_{C_0} \frac{f(z)}{z - z_0} \, \mathrm{d} z
    \end{equation*}
    La integral siguiente es fácil de evaluar
    si parametrizamos \(C_0\)
    como \(z_0 + r \mathrm{e}^{\mathrm{i} t}\):
    \begin{equation*}
      \int_{C_0} \frac{1}{z - z_0} \, \mathrm{d} z
	= \int_0^{2 \pi} \frac{1}{r}
			   \cdot r \mathrm{i}
			     \, \mathrm{e}^{\mathrm{i} t}
			   \, \mathrm{d} t
	= 2 \pi \mathrm{i}
    \end{equation*}
    Con esto:
    \begin{equation*}
      \int_{C_0} \frac{f(z)}{z - z_0} \, \mathrm{d} z
	- 2 \pi \mathrm{i} f(z_0)
	= \int_{C_0} \frac{f(z)}{z - z_0} \, \mathrm{d} z
	    - \int_{C_0} \frac{f(z_0)}{z - z_0} \, \mathrm{d} z
	= \int_{C_0} \frac{f(z) - f(z_0)}{z - z_0} \, \mathrm{d} z
    \end{equation*}
    Sobre \(C_0\) tenemos:
    \begin{equation*}
      \left\lvert \frac{f(z) - f(z_0)}{z - z_0} \right\rvert
	= \frac{\lvert f(z) - f(z_0) \rvert}{\lvert z - z_0 \rvert}
	\le \frac{\epsilon}{r}
    \end{equation*}
    De nuestra cota~\eqref{eq:complex-integral-bound}
    para integrales:
    \begin{equation*}
      \int_{C_0} \frac{f(z) - f(z_0)}{z - z_0} \, \mathrm{d} z
	\le \frac{\epsilon}{r} \cdot 2 \pi r
	= 2 \pi \epsilon
    \end{equation*}
    Pero \(\epsilon\) es un número positivo arbitrario,
    la integral debe ser cero.
  \end{proof}
  Esto es realmente notable:
  Si \(f\) es holomorfa
  al interior del camino cerrado simple \(\gamma\)
  y conocemos los valores de \(f\) sobre \(\gamma\),
  los conocemos en todo su interior.

  Incluso da una manera sencilla de evaluar ciertas integrales.
  Por ejemplo,
  evaluemos la integral:
  \begin{equation*}
    \int_0^\infty \frac{1}{x^2 + 1} \, \mathrm{d} x
  \end{equation*}
  Primero,
  el integrando es par,
  con lo que:
  \begin{equation*}
    \int_0^\infty \frac{1}{x^2 + 1} \, \mathrm{d} x
      = \frac{1}{2} \,
	  \int_{- \infty}^\infty \frac{1}{x^2 + 1} \, \mathrm{d} x
  \end{equation*}
  El integrando tiene problemas en \(\pm \mathrm{i}\),
  es holomorfo
  en \(\mathbb{C} \smallsetminus \{- \mathrm{i}, \mathrm{i} \}\).
  \begin{figure}[ht]
    \centering
    \pgfimage{images/example-integral-contour}
    \caption{Curva para integral ejemplo}
    \label{fig:example-integral-contour}
  \end{figure}
  La curva \(\gamma\)
  de la figura~\ref{fig:example-integral-contour}
  se descompone en un arco \(A\) de radio \(R\)
  y la línea \(L\) de \(-R\) a \(R\) a lo largo del eje~\(X\).
  Si hacemos tender \(R \rightarrow \infty\),
  la integral sobre \(L\) es el resultado que nos interesa.
  Debemos evaluar la integral sobre el arco.
  Como nos interesa \(R \rightarrow \infty\),
  perfectamente podemos concentrarnos en \(R > 1\).
  Tenemos:
  \begin{equation*}
    \left\lvert
      \int_A \frac{1}{z^2 + 1} \, \mathrm{d} z
    \right\rvert
      \le \int_A \frac{1}{\lvert z^2 + 1 \rvert} \, \mathrm{d} z
      \le \int_A \frac{1}{\lvert z \rvert^2 - 1} \, \mathrm{d} z
      =	  \frac{1}{R^2 - 1} \cdot \pi R
  \end{equation*}
  Esto tiende a cero cuando \(R \rightarrow \infty\),
  como esperábamos.
  Acá usamos:
  \begin{equation*}
    \lvert z \rvert^2
      =	  \lvert (z^2 + 1) - 1 \rvert
      \le \lvert z^2 + 1 \rvert + 1
  \end{equation*}
  Por otro lado,
  por la fórmula de Cauchy%
    \index{Cauchy, formula integral de@Cauchy, fórmula integral de}
  podemos escribir para \(z_0 = \mathrm{i}\):
  \begin{equation*}
    \int_\gamma \frac{1 / (z + \mathrm{i})}{z - \mathrm{i}}
	\, \mathrm{d} z
      = \left.
	  2 \pi \mathrm{i} \, \frac{1}{z + \mathrm{i}}
	\right\rvert_{z = \mathrm{i}}
      = \pi
  \end{equation*}
  y nuestra integral original es:
  \begin{equation*}
    \int_0^\infty \frac{1}{x^2 + 1} \, \mathrm{d} x
      = \frac{\pi}{2}
  \end{equation*}
  Esta integral es simple de evaluar en forma tradicional,
  pero esta técnica es aplicable en forma mucho más amplia.

  \begin{theorem}
    \label{theo:Cauchy-formula-f-prime}
    Sea \(f\) holomorfa en la región \(D\).
    Entonces \(f'\) es holomorfa en \(D\).
  \end{theorem}
  \begin{proof}
    Sea \(C\) una circunferencia centrada en \(z\) de radio \(r\)
    dentro de \(D\),
    y \(z + h\) un punto dentro de \(C\).
    Es rutina verificar que:
    \begin{equation*}
      \frac{1}{h} \,
	\left(
	  \frac{1}{\zeta - z - h} - \frac{1}{\zeta - z}
	\right)
	= \frac{1}{(\zeta - z)^2}
	    + \frac{h}{(\zeta - z)^2 (\zeta - z - h)}
    \end{equation*}
    Calculamos:
    \begin{align*}
      \frac{f(z + h) - f(z)}{h}
	&= \frac{1}{2 \pi h \mathrm{i}} \,
	     \int_C \frac{f(\zeta)}{\zeta - z - h}
		 \, \mathrm{d} \zeta
	     - \frac{1}{2 \pi h \mathrm{i}}
		 \int_C \frac{f(\zeta)}{\zeta - z}
		    \, \mathrm{d} \zeta \\
	&= \frac{1}{2 \pi \mathrm{i}} \,
	     \int_C \frac{f(\zeta)}{(\zeta - z)^2}
		 \, \mathrm{d} \zeta
	       + \frac{h}{2 \pi \mathrm{i}} \,
		   \int_C \frac{f(\zeta)}
			       {(\zeta - z)^2 (\zeta - z - h)}
		     \, \mathrm{d} \zeta
    \end{align*}
    Si \(\lvert h \rvert < r / 2\),
    por la desigualdad triangular
    (teorema~\ref{theo:desigualdad-triangular})%
      \index{desigualdad triangular}
    para todo \(\zeta \in C\) es:
    \begin{equation*}
      \lvert \zeta - z - h \rvert
	\ge \lvert \zeta - z \rvert - \lvert h \rvert
	> r - \frac{r}{2}
	= \frac{r}{2}
    \end{equation*}
    Por el otro lado,
    \(f\) es continua sobre \(C\),
    por lo que hay \(M\) tal que \(\lvert f(\zeta) \rvert \le M\)
    para \(\zeta \in C\).
    De la estimación~\eqref{eq:complex-integral-bound} tenemos:
    \begin{equation*}
      \left\lvert
	\frac{h}{2 \pi \mathrm{i}} \,
	  \int_C \frac{f(\zeta)}{(\zeta - z)^2 (\zeta - z - h)}
		   \, \mathrm{d} \zeta
      \right\rvert
	\le \frac{\lvert h \rvert}{2 \pi}
	       \, \frac{2 M}{r^3} \, 2 \pi r
	= \frac{2 M \lvert h \rvert}{r^2}
    \end{equation*}
    Cuando \(h \rightarrow 0\) esto tiende a cero.
    Esto incluso da una fórmula explícita para \(f'(z)\):
    \begin{equation*}
      f'(z)
	= \frac{1}{2 \pi \mathrm{i}}
	    \int_C \frac{f(\zeta)}{(\zeta - z)^2}
	       \, \mathrm{d} \zeta
    \end{equation*}
  \end{proof}
  Aplicando el mismo argumento,
  obtenemos \(f''(z)\),
  con lo que \(f'\) es holomorfa.%
    \index{C (numeros complejos)@\(\mathbb{C}\) (números complejos)!funcion holomorfa@función holomorfa}
  Continuando tenemos derivadas de todos los órdenes.
  Hemos demostrado:
  \begin{theorem}
    \label{theo:holomorphic=>f-(n)}
    Sea \(f\) holomorfa en la región \(D\).
    Entonces para todo \(n \in \mathbb{N}\)
    la función \(f^{(n)}\) es holomorfa en \(D\).
  \end{theorem}
  Esto es notable,
  en los reales la existencia de la primera derivada
  nada dice de las derivadas superiores,
  acá la existencia de la primera derivada
  asegura que hay derivadas de todos los órdenes.
  Aún más:
  \begin{theorem}[Fórmula integral de Cauchy generalizada]
    \label{theo:Cauchy-formula-f-(n)}
    Sea \(f\) holomorfa en la región \(D\),
    y \(\gamma\) un camino cerrado simple al interior de \(D\).
    Entonces:
    \begin{equation}
      \label{eq:Cauchy-formula-f-(n)}
      f^{(n)}(z)
	= \frac{n!}{2 \pi \mathrm{i}} \,
	    \int_\gamma \frac{f(\zeta)}{(\zeta - z)^{n + 1}}
	      \, \mathrm{d} \zeta
    \end{equation}
  \end{theorem}
  \begin{proof}
    Por el teorema~\ref{theo:holomorphic=>f-(n)}
    sabemos que \(f^{(n)}\)
    es holomorfa en \(D\).
    La fórmula integral de Cauchy permite escribir:%
      \index{Cauchy, formula integral de@Cauchy, fórmula integral de}
    \begin{equation*}
      f^{(n)}(z)
	= \frac{1}{2 \pi \mathrm{i}} \,
	    \int_\gamma \frac{f^{(n)}(\zeta)}{\zeta - z}
	      \, \mathrm{d} \zeta
    \end{equation*}
    Integrando por partes \(n\) veces entrega lo prometido.
  \end{proof}
  Hay más consecuencias de interés.
  \begin{theorem}[Liouville]
    \index{Liouville, teorema de}
    \label{theo:Liouville}
    Si una función entera es acotada en valor absoluto,
    es constante.
  \end{theorem}
  \begin{proof}
    Por hipótesis hay una constante \(M\)
    tal que \(\lvert f(z) \rvert  \le M\)
    para todo \(z \in \mathbb{C}\).
    Demostramos por contradicción
    que \(f'(z) = 0\) en todo \(\mathbb{C}\),
    con lo que por el teorema~\ref{theo:complex-zero-derivative}
    \(f\) es constante.

    Supongamos que para algún \(z\) es \(f'(z) \ne 0\).
    Elija \(R\) de manera que \(M / R < \lvert f'(z) \rvert\).
    Sea \(C\) la circunferencia de radio \(R\) alrededor de \(z\).
    Entonces:
    \begin{equation*}
      \frac{M}{R}
	<   \lvert f'(z) \rvert
	=   \left\lvert
	      \frac{1}{2 \pi \mathrm{i}} \,
		\int_C \frac{f(\zeta)}{(\zeta - w)^2}
		  \, \mathrm{d} \zeta
	    \right\rvert
	\le \frac{1}{2 \pi} \, \frac{M}{R^2} \, 2 \pi R
	= \frac{M}{R}
    \end{equation*}
    Esta contradicción muestra que tal \(z\) no existe.
  \end{proof}
  Podemos aprovechar esto inmediatamente,
  básicamente como lo hizo Gauß en su disertación
  (aunque Liouville es bastante posterior).
  Pese a su nombre,
  poco tiene que ver con el álgebra actual
  y no es particularmente fundamental.
  \begin{theorem}[Teorema fundamental del álgebra]
    \index{algebra, teorema fundamental del@álgebra, teorema fundamental del}
    \index{polinomio!cero}
    \label{theo:fundamental-algebra}
    Todo polinomio no constante con coeficientes complejos
    tiene un cero complejo.
  \end{theorem}
  \begin{proof}
    Por contradicción.
    Sea \(p(z)\) un polinomio no constante sin ceros complejos.
    Entonces \(1 / p(z)\) es entera y acotada
    (cuando \(z \rightarrow \infty\)
     también \(\lvert p(z) \rvert \rightarrow \infty\),
     y la función \(1 / p(z)\) es acotada en todo \(\mathbb{C}\)),
    y por el teorema de Liouville es constante.
    Esto contradice el que \(p\) no es constante.
  \end{proof}
  La manera tradicional
  de expresar el teorema~\ref{theo:fundamental-algebra}
  es diciendo que todo polinomio no constante
  de coeficientes reales
  se puede factorizar en factores lineales
  o cuadráticos sin ceros reales,
  o que si tiene grado \(n\)
  tiene \(n\) ceros reales o complejos conjugados
  (contando multiplicidades).

\section{Secuencias y series}
\label{sec:complex-sequences-series}

  Las definiciones de secuencias y series complejas
  son esencialmente las mismas que para los reales.
  Anotamos \(\langle a_n \rangle_{n \ge 0}\)
  para la secuencia de los \(a_n\)
  (formalmente,
   es una función
     \(a \colon \mathbb{N}_0 \rightarrow \mathbb{C}\)).%
     \index{C (numeros complejos)@\(\mathbb{C}\) (números complejos)!secuencia}
  El número \(L\) se llama el \emph{límite} de la secuencia
  si para cualquier \(\epsilon > 0\) que se elija
  hay un entero \(n_\epsilon\),
  dependiente de \(\epsilon\),
  tal que siempre que \(n \ge n_\epsilon\)
  es \(\lvert L - a_n \rvert < \epsilon\).
  Esto lo anotamos \(\lim a_n = L\).%
     \index{C (numeros complejos)@\(\mathbb{C}\) (números complejos)!secuencia!limite@límite}
  Es fácil ver que si \(a_n = u_n + \mathrm{i} v_n\)
  y la secuencia \(\langle a_n \rangle_{n \ge 0}\) converge a \(L\),
  tenemos \(\lim u_n = \Re L\) y \(\lim v_n = \Im L\).
  Al revés,
  si las secuencias reales \(\langle u_n \rangle_{n \ge 0}\)
  y \(\langle v_n \rangle_{n \ge 0}\) convergen,
  converge la secuencia compleja
    \(\langle u_n + \mathrm{i} v_n \rangle_{n \ge 0}\).
  Se desprenden todas las familiares propiedades de los límites.
  Una condición necesaria y suficiente
  para la convergencia
  de la secuencia \(\langle a_n \rangle_{n \ge 0}\)
  es el \emph{criterio de Cauchy}:%
    \index{Cauchy, criterio de}
  Dado \(\epsilon > 0\)
  hay un entero \(n_\epsilon\)
  tal que \(\lvert a_m - a_n \rvert < \epsilon\)
  siempre que \(m, n \ge n_\epsilon\).

  Es obvio considerar secuencias de funciones en una región \(D\).
  Para cada \(z \in D\) tenemos una secuencia ordinaria
  \(\langle f_n(z) \rangle_{n \ge 0}\).
  Si estas secuencias convergen,
  la secuencia \emph{converge punto a punto}%
     \index{C (numeros complejos)@\(\mathbb{C}\) (números complejos)!secuencia!convergencia}
  a la función \(f(z) = \lim f_n(z)\).
  Se dice que la secuencia de funciones
  converge \emph{uniformemente}%
    \index{convergencia uniforme}
  sobre el conjunto \(S\) si dado un \(\epsilon > 0\)
  hay un entero \(n_\epsilon\)
  tal que \(\lvert f(z) - f_n(z) \rvert < \epsilon\)
  para todo \(n \ge n_\epsilon\) y todo \(z \in S\).
  El punto de la convergencia uniforme
  es que el mismo \(n_\epsilon\) sirve para todos los \(z \in S\).

  \begin{figure}[ht]
    \centering
    \pgfimage{images/sequence-continuous-limit-discontinuous}
    \caption{Secuencia de funciones continuas con límite discontinuo}
    \label{fig:sequence-continuous-limit-discontinuous}
  \end{figure}
  Note que una secuencia de funciones continuas
  puede converger a una función discontinua.
  Considere por ejemplo la secuencia de funciones
  definidas para \(n \ge 1\) por:
  \begin{equation}
    \label{eq:sequence-continuous-functions}
    f_n(x)
      = \begin{cases}
	  0	    & x \le - 1 / n \\
	  1 + n x   & -1 / n < x \le 0 \\
	  1 - n x   & 0 < x \le 1 / n \\
	  0	    & 1 / n < x
	\end{cases}
  \end{equation}
  La figura~\ref{fig:sequence-continuous-limit-discontinuous}
  grafica algunas de las funciones~%
    \eqref{eq:sequence-continuous-functions}.
  Es claro que:
  \begin{equation}
    \label{eq:continuous-functions-discontinuous-limit}
    \lim_{n \rightarrow \infty} f_n(x)
      = \begin{cases}
	  0  & x \ne 0 \\
	  1  & x = 1
	\end{cases}
  \end{equation}
  Este comportamiento es imposible si la convergencia es uniforme.
  Porque suponga que \(\langle f_n(z) \rangle_{n \ge 0}\)
  converge uniformemente a \(f\) en la región \(D\),
  sea \(z_0 \in D\) y \(\epsilon > 0\).
  Demostraremos que hay \(\delta\)
  tal que \(\lvert f(z_0) - f(z) \rvert < \epsilon\)
  siempre que \(\lvert z_0 - z \rvert < \delta\).
  Elija \(n_\epsilon\)
  tal que \(\lvert f_{n_\epsilon}(z) - f(z) \rvert < \epsilon / 3\).
  Por convergencia uniforme
  también es
     \(\lvert f_{n_\epsilon}(z_0) - f(z_0) \rvert < \epsilon / 3\).
  Ahora elija \(\delta\)
  de forma que
    \(\lvert f_{n_\epsilon}(z_0) - f_{n_\epsilon}(z) \rvert
	< \epsilon / 3\)
  siempre que \(\lvert z_0 - z \rvert < \delta\).
  Esto es posible ya que \(f_{n_\epsilon}\) es continua.
  Si \(\lvert z_0 - z \rvert < \delta\) resulta
  para todo \(n > n_\epsilon\):
  \begin{align*}
    \lvert f(z_0) - f(z) \rvert
      &=   \lvert f(z_0) - f_n(z_0)
		    + f_n(z_0) - f_n(z)
		    + f_n(z) - f(z) \rvert \\
      &\le \lvert f(z_0) - f_n(z_0) \rvert
	     + \lvert f_n(z_0) - f_n(z) \rvert
	     + \lvert f_n(z) - f(z) \rvert \\
      &<   \frac{\epsilon}{3}
	     + \frac{\epsilon}{3}
	     + \frac{\epsilon}{3} \\
      &=   \epsilon
  \end{align*}

  En los reales hay secuencias de funciones diferenciables
  que convergen uniformemente a funciones que no son diferenciables.
  La función símbolo que no es diferenciable en \(0\)
  es \(\lvert x \rvert\),
  interesa construir una secuencia de funciones
  que se parecen a las ramas del valor absoluto,
  ``suavizando'' la esquina
  por ejemplo con una parábola \(y = a x^2\)
  entre \(\pm 1 / n\).
  Las ramas serán rectas de pendiente \(\pm 1\),
  digamos \(y = \pm x + b\);
  queremos que los valores
  y las derivadas coincidan en \(\pm 1 / n\):
  \begin{equation*}
    f_n	 \left( \pm \frac{1}{n} \right)
      = \frac{1}{n} + b
    \hspace{4em}
    f_n' \left( \pm \frac{1}{n} \right)
      = \pm 2 a \frac{1}{n}
      = \pm 1
  \end{equation*}
  De acá:
  \begin{equation*}
    a = \frac{n}{2}
    \hspace{3em}
    b = - \frac{1}{2 n}
  \end{equation*}
  Nuestra función es:
  \begin{equation*}
    f_n(x)
      = \begin{cases}
	  - x - 1 / 2 n	      & -1 \le x < 1 / n \\
	  n x^2 / 2	      & - 1 / n < x \le 1 / n \\
	  x - 1 / 2 n	      & 1 / n < x \le 1
	\end{cases}
  \end{equation*}
  Por la forma que la construimos,
  \(f_n\) es diferenciable en \([-1, 1]\).
  Difiere de \(\lvert x \rvert\) a lo más en \(1 / 2 n\),
  con lo que la convergencia es uniforme.
  Pero \(\lim f_n(x) = \lvert x \rvert\),
  que no es diferenciable en \(x = 0\).

  Pero también:
  \begin{theorem}
    \label{theo:int_f_n->int_f}
    Sea \(\gamma\) una curva suave,
    sobre la cual las funciones \(f_n\) son continuas
    y convergen uniformemente a \(f\).
    Entonces:
    \begin{equation}
      \label{eq:int_f_n->int_f}
      \lim_{n \rightarrow \infty} \int_\gamma f_n(z) \, \mathrm{d} z
	= \int_\gamma f(z) \, \mathrm{d} z
    \end{equation}
  \end{theorem}
  Este resultado tiene múltiples consecuencias,
  que veremos más adelante.
  La demostración es rutina:
  \begin{proof}
    Podemos acotar:
    \begin{equation*}
      \left\lvert
	\int_\gamma f_n(z) \, \mathrm{d} z
	  - \int_\gamma f(z) \, \mathrm{d} z
      \right\rvert
	= \left\lvert
	    \int_\gamma (f_n(z) - f(z)) \, \mathrm{d} z
	  \right\rvert
	\le \max_{z \in \gamma} \lvert f_n(z) - f(z) \rvert
	      \cdot l_\gamma
    \end{equation*}
    Por convergencia uniforme
    podemos hacer el primer factor de la cota
    tan pequeño como deseemos.
  \end{proof}
  Incluso más:
  \begin{theorem}
    \label{theo:holomorphic_f_n->holomorphic_f}
    Sea una secuencia de funciones holomorfas
    \(\langle f_n(z) \rangle_{n \ge 0}\) que en \(D\)
    convergen uniformemente a \(f(z)\).
    Entonces \(f\) es holomorfa en \(D\).
  \end{theorem}
  Nótese que el resultado no se cumple para reales.
  \begin{proof}
    Sea \(\gamma \subset D\) una curva cerrada simple.
    Del teorema de Cauchy sabemos:%
      \index{Cauchy, teorema integral de}
    \begin{equation*}
      \int_\gamma f_n(z) \, \mathrm{d} z
	= 0
    \end{equation*}
    Por convergencia uniforme:
    \begin{equation*}
      \int_\gamma f(z) \, \mathrm{d} z
	= 0
    \end{equation*}
    El teorema de Morera,
    teorema~\ref{theo:Morera},
    nos dice que
    la función \(f\) es holomorfa en \(D\).
  \end{proof}

\subsection{Series}
\label{sec:complex-series}

  Una serie es simplemente
  la secuencia \(\langle s_n \rangle_{n \ge 0}\)
  resultante de sumar los elementos
  de una secuencia \(\langle a_n \rangle_{n \ge 0}\),
  vale decir,
  \(s_n = a_0 + a_1 + \dotsb + a_n\).
  Si la serie converge,
  debe ser \(\lim a_n = 0\).
  Para el límite de la serie
  anotamos según nuestra convención sobre sumas:
  \begin{equation*}
    \sum_{n \ge 0} a_n
  \end{equation*}
  o el más familiar:
  \begin{equation*}
    \sum_{n = 0}^\infty a_n
  \end{equation*}

  Igual que en el caso de series en los reales,
  es útil distinguir series
  que \emph{convergen en valor absoluto},%
    \index{serie!convergencia absoluta}
  vale decir la secuencia:
  \begin{equation*}
    \sum_{0 \le k \le n} \lvert a_k \rvert
  \end{equation*}
  converge.
  Es simple demostrar que si la serie converge en valor absoluto,
  converge la serie original;
  pero la convergencia de la serie original
  no asegura convergencia en valor absoluto.
  Por ejemplo,
  tenemos la serie harmónica alternante
  (el valor lo justificaremos más adelante):
  \begin{equation}
    \label{eq:alternating-harmonic-series}
    \sum_{k \ge 1} \frac{(-1)^{k + 1}}{k}
      = \ln 2
  \end{equation}
  pero la contraparte de valores absolutos es la serie harmónica,
  que no converge.
  Incluso más:
  \begin{theorem}[Reordenamiento de Riemann]
    \index{Riemann, teorema de reordenamiento de}
    \label{theo:Riemann-rearrangement}
    Sea una serie real que converge pero no absolutamente.
    Entonces sus términos
    pueden reordenarse para dar cualquier suma,
    e incluso diverger a \(\pm \infty\)
    o no tener límite.
  \end{theorem}
  \begin{proof}
    Si la serie converge,
    pero no absolutamente,
    tiene infinitos términos positivos cuya suma diverge
    y de la misma forma tiene infinitos términos negativos
    cuya suma diverge.
    Fijemos un valor \(L\) cualquiera;
    consideraremos el caso en que \(L \ge 0\),
    el caso \(L < 0\) es similar.
    Ordenamos los términos como sigue:
    \begin{itemize}
    \item
      Elegimos términos positivos
      hasta que la suma sobrepase a \(L\).
      Como la suma de los términos positivos diverge,
      esto puede hacerse.
    \item
      Elegimos luego términos negativos
      hasta que la suma sea menor a \(L\).
      Nuevamente,
      como los términos negativos divergen esto puede hacerse.
    \end{itemize}
    Repitiendo este proceso obtenemos un ordenamiento
    de los términos de la serie que converge a \(L\).
    La diferencia entre la suma y el valor elegido va disminuyendo,
    como la serie original converge
    sabemos que los términos de la serie reordenada
    disminuyen en valor absoluto.

    Siendo suficientemente tacaños con los términos negativos
    (respectivamente positivos)
    logramos que diverja;
    tomando dos valores podemos hacer oscilar los valores de la suma
    alrededor de ellos.
  \end{proof}
  De tales series se dice que \emph{convergen condicionalmente}.%
    \index{serie!convergencia condicional}
  Un ejemplo de este fenómeno
  es escribir la serie harmónica alternante~%
    \eqref{eq:alternating-harmonic-series}
  como:
  \begin{equation}
    \label{eq:alternating-harmonic-series-reordered-value}
    \sum_{k \ge 1}
      \left(
	\frac{1}{2 k - 1} - \frac{1}{2 (2 k - 1)} - \frac{1}{4 k}
      \right)
      = \sum_{k \ge 1}
	  \left(
	    \frac{1}{2 (2 k - 1)} - \frac{1}{2 \cdot 2 k}
	  \right)
      = \frac{1}{2}
	  \sum_{k \ge 1} \frac{(-1)^{k + 1}}{k}
      = \frac{1}{2} \ln 2
  \end{equation}
  Este es un reordenamiento correcto,
  aparecen los recíprocos de todos los impares con signo positivo
  y los recíprocos de todos los pares con signo negativo.
  La mitad
  del valor original~\eqref{eq:alternating-harmonic-series}.

  En forma análoga podemos considerar series de funciones:
  \begin{equation*}
    \sum_{k \ge 0} f_k(z)
  \end{equation*}
  Tales series pueden converger para ciertos valores de \(z\)
  y no para otros.
  Un criterio útil de convergencia es el siguiente:
  \begin{theorem}[Prueba \(M\) de Weierstraß]
    \index{Weierstrass, prueba \(M\) de@Weierstraß, prueba \(M\) de}
    \label{theo:Weierstrass-M}
    Sea \(\langle M_k \rangle_{k \ge 0}\)
    una secuencia de números reales,
    que hay \(K\) tal que \(M_k \ge 0\) para todo \(k > K\),
    y suponga que la secuencia
    \begin{equation*}
      \left\langle \sum_{0 \le k \le n} M_k \right\rangle_{n \ge 0}
    \end{equation*}
    converge.
    Si para todo \(z \in D\) es
    \(\lvert f_k(z) \rvert \le M_k\) para \(k \ge K\),
    la serie
    \begin{equation*}
      \sum_{k \ge 0} f_k(z)
    \end{equation*}
    converge uniformemente en valor absoluto en \(D\).
  \end{theorem}
  \begin{proof}
    Sea \(\epsilon > 0\) cualquiera,
    y elegimos \(N > K\) tal que:
    \begin{equation*}
      \sum_{m \le k \le n} M_k
	< \epsilon
    \end{equation*}
    para todo \(m, n > N\)
    (esto resulta del criterio de Cauchy).
    Por la desigualdad triangular,
    teorema~\ref{theo:desigualdad-triangular},
    es:
    \begin{equation*}
      \left\lvert
	\sum_{m \le k \le n} f_k(z)
      \right\rvert
	\le \sum_{m \le k \le n} \lvert f_k(z) \rvert
	\le \sum_{m \le k \le n} M_k
	< \epsilon
    \end{equation*}
    La serie converge.
    Para convergencia uniforme,
    observe que para todo \(z \in D\) y \(n > m > N\):
    \begin{equation*}
      \left\lvert
	\sum_{m \le k \le n} f_k(z)
      \right\rvert
	= \left\lvert
	    \sum_{0 \le k \le n} f_k(z)
	      - \sum_{0 \le k \le m - 1} f_k(z)
	  \right\rvert
	< \epsilon
    \end{equation*}
    En consecuencia:
    \begin{equation*}
      \lim_{n \rightarrow \infty}
	\left\lvert
	  \sum_{m \le k \le n} f_k(z)
	\right\rvert
	= \left\lvert
	    \sum_{k \ge 0} f_k(z)
	      - \sum_{0 \le k \le m - 1} f_k(z)
	  \right\rvert
	\le \epsilon
    \end{equation*}
    y la convergencia es uniforme y en valor absoluto.
  \end{proof}

  El caso más interesante es el de series de potencias:
  \begin{equation*}
    s_n(z)
      = \sum_{0 \le k \le n} c_k (z - z_0)^k
  \end{equation*}
  Una serie de potencias
  podrá tener un límite para ciertos valores de \(z\)
  y no para otros.
  Claramente siempre tiene límite si \(z = z_0\).
  \begin{theorem}[Cauchy-Hadamard]
    \index{Cauchy-Hadamard, teorema de}
    \index{serie de potencias!radio de convergencia}
    \label{theo:convergence-root}
    Sea la serie:
    \begin{equation*}
      \sum_{0 \le k \le n} c_k (z - z_0)^k
    \end{equation*}
    Sea:
    \begin{equation}
      \label{eq:convergence-radius-root}
      \lambda
	= \limsup_{k \rightarrow \infty} \sqrt[k]{\lvert c_k \rvert}
    \end{equation}
    Sea \(R = \lambda^{-1}\),
    donde diremos que \(R = \infty\) si \(\lambda = 0\)
    y que \(R = 0\) si \(\lambda = \infty\).
    Entonces la serie converge uniformemente en valor absoluto
    para todo \(\lvert z - z_0 \rvert < r < R\)
    y diverge para todo \(\lvert z - z_0 \rvert > R\).
  \end{theorem}
  \begin{proof}
    Primero demostramos que la serie no converge
    para \(\lvert z - z_0 \rvert > R\).
    Sea \(L\) tal que:
    \begin{equation*}
      \frac{1}{\lvert z - z_0 \rvert}
	< L
	< \frac{1}{R}
	= \lambda
    \end{equation*}
    Hay un número infinito de \(c_k\)
    tales que \(\sqrt[k]{\lvert c_k \rvert} > L\),
    ya que de lo contrario el límite superior sería menor a \(L\).
    Para cada uno de ellos tenemos:
    \begin{equation*}
      \lvert c_k (z - z_0)^k \rvert
	= \left(
	    \sqrt[k]{\lvert c_k \rvert} \cdot \lvert z - z_0 \rvert
	  \right)^k
	> \left( L \lvert z - z_0 \rvert \right)^k
	> 1
    \end{equation*}
    y la serie no puede converger.

    Enseguida demostramos que converge uniformemente
    para todo \(\lvert z - z_0 \rvert < r < R\).
    Sea \(L\) tal que:
    \begin{equation*}
      \lambda
	= \frac{1}{R}
	< L
	< \frac{1}{r}
    \end{equation*}
    Para \(k\) suficientemente grande
    es \(\sqrt[k]{\lvert c_k \rvert} < L\),
    con lo que si \(\lvert z - z_0 \rvert \le r\):
    \begin{equation*}
      \lvert c_k (z - z_0)^k \rvert
	= \left(
	    \sqrt[k]{\lvert c_k \rvert} \cdot \lvert z - z_0 \rvert
	  \right)^k
	< (L \lvert z - z_0 \rvert)^k
	< (L r)^k
    \end{equation*}
    La serie geométrica de los \((L r)^k\) converge,
    y la prueba de \(M\) da convergencia uniforme en valor absoluto.
  \end{proof}
  Nótese que hemos demostrado que toda serie de potencias
  converge uniformemente en valor absoluto
  en el disco abierto \(D_R(z_0)\),
  su \emph{región de convergencia};
  a \(R\) se le llama el \emph{radio de convergencia} de la serie.

  Ya que estamos en eso,
  explicitemos lo que el teorema~\ref{theo:int_f_n->int_f}
  dice para series de potencias.
  La función \(g\) que introduciremos nos vendrá bien más adelante.
  \begin{corollary}
    \label{cor:power-series-integrate-termwise}
    Suponga una serie de potencias
      \(\sum_{k \ge 0} c_k (z - z_0)^k\)
    con radio de convergencia \(R\),
    y una curva suave \(\gamma \subset D_R(z_0)\),
    y sea \(g(z)\) continua
    sobre \(\gamma\).
    Entonces:
    \begin{equation*}
      \int_\gamma g(z) \sum_{k \ge 0} c_k (z - z_0)^k
	  \, \mathrm{d} z
	= \sum_{k \ge 0} c_k \int_\gamma g(z) (z - z_0)^k
	    \, \mathrm{d} z
    \end{equation*}
    En particular,
    si \(\gamma\) es cerrada:
    \begin{equation*}
      \int_\gamma \sum_{k \ge 0} g(z) c_k (z - z_0)^k
	  \, \mathrm{d} z
	= 0
    \end{equation*}
  \end{corollary}
  \begin{proof}
    Sea \(\epsilon > 0\),
    y sea \(M\) el máximo de \(g\) sobre \(\gamma\),
    y \(l_\gamma\) el largo de la curva.
    Entonces hay un entero \(N\) tal que para \(n > N\):
    \begin{equation*}
      \left\lvert
	\sum_{k \ge n} c_k (z - z_0)^k
      \right\rvert
	< \frac{\epsilon}{M l_\gamma}
    \end{equation*}
    de donde:
    \begin{equation*}
      \left\lvert
	\int_\gamma g(z) \sum_{k \ge n} c_k (z - z_0)^k
	  \, \mathrm{d} z
      \right\rvert
	< M l_\gamma \, \frac{\epsilon}{M l_\gamma}
	= \epsilon
    \end{equation*}
    Con esto:
    \begin{equation*}
      \left\lvert
	\int_\gamma g(z) \sum_{k \ge n} c_k (z - z_0)^k
	  \, \mathrm{d} z
	  - \sum_{0 \le k \le n - 1}
	      c_k \int_\gamma g(z) (z - z_0)^k \, \mathrm{d} z
      \right\rvert
	= \left\lvert
	    \int_\gamma g(z) \sum_{k \ge n} c_k (z - z_0)^k
	      \, \mathrm{d} z
	  \right\rvert
	< \epsilon
    \end{equation*}
    Esto es lo que prometimos.
  \end{proof}
  Si elegimos \(g(z) = 1\),
  tenemos que
  podemos integrar término a término
  dentro del radio de convergencia,
  y la serie define una función holomorfa dentro de \(D_R(z_0)\).
  Si analizamos los términos de la serie,
  tenemos el siguiente:
  \begin{corollary}
    \label{cor:ratio-test-1}
    Sean \(c_k\) los coeficientes de una serie de potencias
    con radio de convergencia \(R\).
    Entonces para \(r < R\):
    \begin{equation*}
      \lim_{k \rightarrow \infty} \lvert c_k \rvert r^k
	= 0
    \end{equation*}
    mientras para \(r > R\):
    \begin{equation*}
      \lim_{k \rightarrow \infty} \lvert c_k \rvert r^k
	= \infty
    \end{equation*}
  \end{corollary}
  Nótese que esto nada dice sobre la convergencia o divergencia
  en la frontera de la región de convergencia.
  La manoseada serie:
  \begin{equation}
    \label{eq:logarithmic-series}
    \sum_{k \ge 1} \frac{z^k}{k}
  \end{equation}
  tiene radio de convergencia \(1\).
  Si partimos con:
  \begin{equation}
    \label{eq:geometric-series}
    \frac{1}{1 - z}
      = \sum_{k \ge 0} z^k
  \end{equation}
  cuyo radio de convergencia es \(1\),
  obtenemos que su integral
  (vale decir,
   \eqref{eq:logarithmic-series})
  tiene el mismo radio de convergencia.
  Dentro del radio de convergencia converge a \(\Log (1 - z)\);
  como converge para \(z = -1\),
  por continuidad converge a \(\ln 2\) para \(z = -1\).
  Diverge para \(z = 1\),
  donde resulta la serie harmónica.

  Una forma más sencilla
  de usar del corolario~\ref{cor:ratio-test-1}
  es la siguiente:
  \begin{corollary}
    \label{cor:ratio-test}
    Sean \(c_k\) los coeficientes de una serie de potencias,
    con \(c_k \ne 0\) para \(k\) suficientemente grande.
    Entonces su radio de convergencia es:
    \begin{equation}
      \label{eq:convergence-radius-ratio}
      R
	= \lim_{k \rightarrow \infty}
	    \frac{\lvert c_{k + 1} \rvert}{\lvert c_k \rvert}
    \end{equation}
  \end{corollary}

  Vimos
  (teorema~\ref{theo:holomorphic_f_n->holomorphic_f})
  que si una secuencia de funciones holomorfas
  converge uniformemente
  en una región,
  su límite es holomorfo.
  Aplicando esto a series de potencias:
  \begin{corollary}
    \label{cor:holomorphic=>analytic}
    Suponga \(f(z) = \sum_{k \ge 0} c_k (z - z_0)^k\)
    tiene radio de convergencia positivo \(R\).
    Entonces \(f\) es holomorfa en \(D_R(z_0)\).
  \end{corollary}
  Series de potencias con radio de convergencia infinito
  representan funciones enteras.
  Podemos derivar término a término:
  \begin{theorem}
    \label{theo:complex-series-differentiate-termwise}
    Suponga \(f(z) = \sum_{k \ge 0} c_k (z - z_0)^k\)
    tiene radio de convergencia positivo \(R\).
    En \(D_R(z_0)\) tenemos:
    \begin{equation*}
      f'(z)
	= \sum_{k \ge 0} k c_k (z - z_0)^{k - 1}
    \end{equation*}
  \end{theorem}
  \begin{proof}
    Sea \(z\) un punto dentro de la región de convergencia,
    y sea \(C\) una circunferencia orientada positivamente
    centrada en \(z\) al interior de la región de convergencia.
    Defina:
    \begin{equation*}
      g(\zeta)
	= \frac{1}{2 \pi \mathrm{i} (\zeta - z)^2}
    \end{equation*}
    Aplicamos el corolario~\ref{cor:power-series-integrate-termwise}
    para concluir:
    \begin{align*}
      \int_C g(\zeta) f(\zeta) \, \mathrm{d} \zeta
	&= \sum_{k \ge 0}
	     c_k \int_C g(\zeta) (\zeta - z_0)^k
	       \, \mathrm{d} \zeta \\
      \frac{1}{2 \pi \mathrm{i}} \,
	\int_C \frac{f(\zeta)}{(\zeta - z)^2} \, \mathrm{d} \zeta
	&= \sum_{k \ge 0} c_k \frac{1}{2 \pi \mathrm{i}} \,
	     \int_C \frac{(\zeta - z_0)^k}{(\zeta - z)^2}
	       \, \mathrm{d} \zeta \\
    \intertext{La fórmula integral de Cauchy,
	       teorema~\ref{theo:Cauchy-formula-f-prime},
	       da:}
      f'(z)
	&= \sum_{k \ge 0} k c_k (z - z_0)^k
    \end{align*}
    La expansión es válida
    dentro de la región de convergencia original.
  \end{proof}

\section{Series de Taylor y Laurent}
\label{sec:Taylor-Laurent-series}

  Veamos ahora cómo construir series
  para una función holomorfa dada.

\subsection{Serie de Taylor}
\label{sec:Taylor-series}
\index{Taylor, serie de}

  \begin{theorem}[Serie de Taylor]
    \label{theo:Taylor}
    Suponga \(f\) holomorfa en el disco abierto \(D_R(z_0)\).
    Entonces:
    \begin{equation}
      \label{eq:Taylor-series}
      f(z)
	= \sum_{k \ge 0}
	    \frac{f^{(k)}(z_0)}{k!} (z - z_0)^k
    \end{equation}
    Esta serie converge en \(D_R(z_0)\).
  \end{theorem}
  \begin{proof}
    Sea \(z \in D_R(z_0)\),
    y sea \(C\)
    una circunferencia de radio \(r\) alrededor de \(z\)
    dentro del disco abierto.
    Entonces \(0 < r < R\).
    Para \(\zeta \in C\) podemos escribir:
    \begin{equation*}
      \frac{1}{\zeta - z}
	= \frac{1}{(\zeta - z_0) - (z - z_0)}
	= \frac{1}
	       {\zeta - z_0}
		 \cdot \frac{1}{1 - \frac{z - z_0}{\zeta - z_0}}
	= \sum_{k \ge 0} \frac{(z - z_0)^k}{(\zeta - z_0)^{k + 1}}
    \end{equation*}
    Esto es válido ya que \((z - z_0) / (\zeta - z_0) < 1\),
    y la convergencia es uniforme.
    Podemos integrar:
    \begin{align*}
      \int_C \frac{f(\zeta)}{\zeta - z_0} \, \mathrm{d} \zeta
	&= \sum_{k \ge 0}
	     \left(
	       \int_C \frac{f(\zeta)}{(\zeta - z_0)^{k + 1}}
		 \, \mathrm{d} \zeta
	     \right) \, (z - z_0)^k \\
      f(z)
	&= \frac{1}{2 \pi \mathrm{i}} \,
	     \int_C \frac{f(\zeta)}{\zeta - z} \, \mathrm{d} \zeta
	 = \sum_{k \ge 0}
	     \left(
	       \frac{1}{2 \pi \mathrm{i}} \,
		 \int_C \frac{f(\zeta)}{(\zeta - z_0)^{k + 1}}
		   \, \mathrm{d} \zeta
	     \right) \, (z - z_0)^k \\
	&= \sum_{k \ge 0} \frac{f^{(k)}(z_0)}{k!} \, (z - z_0)^k
      \qedhere
    \end{align*}
  \end{proof}
  A una función que puede representarse localmente
  (en un disco abierto centrado en \(z_0\))
  mediante una serie de potencias convergente
  se le llama \emph{analítica}
  (en \(z_0\)).
  Vemos que es muy similar al concepto de holomorfismo
  en los complejos,
  lo que explica que comúnmente se usen intercambiablemente.

  Vale la pena tomar un poco de distancia
  y recapitular lo que hemos demostrado.
  \begin{itemize}
  \item
    Si la función \(f\) es derivable
    en una región \(D\) de \(\mathbb{C}\),
    tiene derivadas de todos los órdenes en \(D\).
    Igualmente tiene antiderivadas de todos los órdenes en \(D\).
  \item
    La integral de \(f\) es independiente del camino
    al interior de la región simple \(D\),
    y el valor de la integral
    es la diferencia de los valores de una antiderivada
    en los puntos inicial y final.
  \item
    El valor de la función en un punto interior de \(D\)
    queda determinado por los valores en la frontera de \(D\).
  \item
    Podemos representar la función \(f\) mediante
    su serie de Taylor~\eqref{eq:Taylor-series}
    sobre un disco abierto
      \(\{z \colon \lvert z - z_0 \rvert < R\}\).
  \item
    La serie de Taylor de \(f\)
    converge uniformemente en valor absoluto
    a \(f\) para \(\{z \colon \lvert z - z_0 \rvert \le r < R\}\),
    donde el radio de convergencia \(R\)
    está dado por~\eqref{eq:convergence-radius-root}
    o~\eqref{eq:convergence-radius-ratio}.
  \item
    Una serie de potencias
    puede derivarse e integrarse término a término,
    resultando series con el mismo radio de convergencia.
    Las series resultantes
    igualmente convergen uniformemente en valor absoluto.
  \item
    La serie de potencias para \(f\) es única
    (si tenemos dos series de potencias
     que representan la misma función
     centradas en el mismo punto,
     tienen los mismo coeficientes).
  \end{itemize}

\subsection{Singularidades}
\label{sec:singularities}
\index{C (numeros complejos)@\(\mathbb{C}\) (números complejos)!singularidad|textbfhy}
\index{singularidad}

  Requeriremos un poco de jerga adicional.
  Si la función \(f\) es tal que:
  \begin{equation*}
    f(z_0)
      = 0
  \end{equation*}
  decimos que \(z_0\) es un \emph{cero} de \(f\).%
    \index{cero|textbfhy}
  Si \(m \in \mathbb{N}\) es tal que:
  \begin{equation*}
    \lim_{z \rightarrow z_0} \frac{f(z)}{(z - z_0)^{m - 1}}
      = 0
    \text{\ y\ }
    \lim_{z \rightarrow z_0} \frac{f(z)}{(z - z_0)^m}
      \ne 0
  \end{equation*}
  decimos que es un \emph{cero de multiplicidad} \(m\).
  En el caso \(m = 1\) se habla de \emph{cero simple}.

  Veamos cómo extender la representación de funciones
  para abarcar puntos en los que dejan de ser holomorfas,
  \emph{singularidades} en las que la función no está definida.
  Estamos interesados en \emph{singularidades aisladas},
  vale decir,
  hay un disco abierto perforado
  (en inglés, \emph{\foreignlanguage{english}{punctured disk}})
  \(\{ z \in \mathbb{C} \colon 0 < \lvert z - z_0 \rvert < R \}\)
  sobre el que la función \(f\) es holomorfa.
  Tenemos los siguientes casos,
  que detallaremos en lo que sigue:
  \begin{description}
  \item[Singularidad removible:]
    Si existe:
    \begin{equation*}
      \lim_{z \rightarrow z_0} f(z)
    \end{equation*}
    la singularidad es removible.
    Basta definir \(f(z_0)\) como el límite del caso,
    y asunto resuelto.%
      \index{singularidad!removible}
    Véase el teorema~\ref{theo:Riemann-removable-singularity}.
  \item[Polo:]\index{C (numeros complejos)@\(\mathbb{C}\) (números complejos)!polo}
    Si hay \(m \in \mathbb{N}\)
    tal que la función \(g\) definida por:
    \begin{equation*}
      g(z)
	= (z - z_0)^m f(z)
    \end{equation*}
    cumple
    \begin{equation*}
      \lim_{z \rightarrow z_0} g(z) \ne 0
    \end{equation*}
    y \(g\) es holomorfa en un entorno de \(z_0\),
    decimos que \(f\)
    tiene un \emph{polo} de orden \(m\) en \(z_0\).%
      \index{singularidad!polo}
    Si \(m = 1\),
    también se le llama \emph{polo simple}.

    Es claro que si \(f\) tiene un cero de multiplicidad \(m\)
    en \(z_0\),
    entonces \(1 / f(z)\) tiene un polo de multiplicidad \(m\)
    en ese punto.
  \item[Singularidad esencial:]
    Si \(f\) tiene una singularidad en \(z_0\)
    que no es removible ni es un polo,
    es una \emph{singularidad esencial}.%
      \index{singularidad!esencial}
  \end{description}
  Una definición que será útil más adelante es la siguiente:
  \begin{definition}
    A una función \(f\) que es holomorfa sobre una región \(D\)
    salvo polos se le llama \emph{meromorfa}.%
      \index{C (numeros complejos)@\(\mathbb{C}\) (números complejos)!funcion meromorfa@función meromorfa}
  \end{definition}

  Nótese que al conocer todas las derivadas de una función
  en un punto cualquiera,
  podemos expandirla en serie de Taylor alrededor de ese punto,
  y esta serie converge en el disco abierto centrado en ese punto
  y que llega a la singularidad más cercana.
  De esta forma,
  podemos \emph{extender analíticamente} la función%
      \index{C (numeros complejos)@\(\mathbb{C}\) (números complejos)!extension analitica@extensión analítica}
  a nuevas áreas del plano,
  salvo que las singularidades formen un conjunto denso
  en alguna curva cerrada.

  Algunos resultados al respecto son los siguientes:
  \begin{theorem}[De Riemann sobre singularidades removibles]
    \label{theo:Riemann-removable-singularity}
    Suponga que \(f\) es holomorfa sobre el disco abierto perforado
    \(\{ z \colon 0 < \lvert z - z_0 \rvert < R \}\),
    y que
    \begin{equation*}
      \lim_{z \rightarrow z_0} (z - z_0) f(z)
	= 0
    \end{equation*}
    Entonces \(f\) tiene una singularidad removible en \(z_0\).
  \end{theorem}
  \begin{proof}
    Sea \(z\) un punto en el disco perforado
    \(\{ z \colon 0 < \lvert z - z_0 \rvert < R \}\),
    y sean \(r_1\) y \(r_2\) tales que
    \(0 < r_1 < \lvert z \rvert < r_2\),
    y sean \(C_1\) y \(C_2\) circunferencias
    de radios \(r_1\) y \(r_2\) respectivamente
    centradas en \(z_0\),
    orientadas ambas en sentido positivo.
    La función:
    \begin{equation*}
      g(\zeta)
       = \begin{cases}
	   \displaystyle f'(z)
				   & \zeta =   z \\
	   \\
	   \displaystyle \frac{f(\zeta) - f(z)}{\zeta - z}
				   & \zeta \ne z
	 \end{cases}
    \end{equation*}
    claramente es holomorfa
    en \(\{ \zeta \colon \lvert \zeta - z_0 \rvert < R \}\).
    Por el teorema de Cauchy:
    \begin{equation*}
      \int_{C_1} g(\zeta) \, \mathrm{d} \zeta
	= \int_{C_2} g(\zeta) \, \mathrm{d} \zeta
    \end{equation*}
    De la definición de \(g\) esto significa:
    \begin{equation*}
      \int_{C_1} \frac{f(\zeta)}{\zeta - z} \, \mathrm{d} \zeta
	-  f(z) \int_{C_1} \frac{1}{\zeta - z} \, \mathrm{d} \zeta
	= \int_{C_2} \frac{f(\zeta)}{\zeta - z} \, \mathrm{d} \zeta
	    -  f(z) \int_{C_2} \frac{1}{\zeta - z}
		      \, \mathrm{d} \zeta
    \end{equation*}
    En la región
      \(\{ \zeta \colon \lvert \zeta - z_0 \rvert
	   < \lvert z - z_0 \rvert \}\)
    (que incluye la curva \(C_1\) y su interior,
     pero excluye \(z\))
    es holomorfa la función:
    \begin{equation*}
      \frac{1}{\zeta - z}
    \end{equation*}
    y por lo tanto:
    \begin{equation*}
      \int_{C_1} \frac{1}{\zeta - z} \, \mathrm{d} \zeta
	= 0
    \end{equation*}
    Por el otro lado,
    la fórmula integral de Cauchy
    aplicada a la función \(1 = 1(z)\) da:%
      \index{Cauchy, formula integral de@Cauchy, fórmula integral de}
    \begin{equation*}
      \int_{C_2} \frac{1}{\zeta - z} \, \mathrm{d} \zeta
	= 2 \pi \mathrm{i}
    \end{equation*}
    Por hipótesis,
    dado \(\epsilon > 0\) hay \(\delta > 0\)
    tal que si \(0 < \lvert \zeta - z_0 \rvert < \delta\)
    entonces \(\lvert (\zeta - z_0) f(\zeta) \rvert < \epsilon\).
    Sin perder generalidad podemos suponer:
    \begin{equation*}
      \delta < \frac{1}{2} \, \lvert z - z_0 \rvert
    \end{equation*}
    Si elegimos \(r_1 = \delta\),
    por~\eqref{eq:complex-integral-bound}:
    \begin{equation*}
      \left\lvert
	\int_{C_1} \frac{f(\zeta)}{\zeta - z} \, \mathrm{d} \zeta
      \right\rvert
	\le \left\lvert
	      \int_{C_1} \frac{(\zeta - z_0) f(\zeta)}
			      {(\zeta - z_0) \, (\zeta - z)}
			 \, \mathrm{d} \zeta
	    \right\rvert
	\le \frac{\epsilon}
		 {\delta ( \lvert z - z_0 \rvert - \delta )}
	      \cdot 2 \pi \delta
	\le \frac{4 \pi \epsilon}{\lvert z - z_0 \rvert}
    \end{equation*}
    Como \(\epsilon\) es arbitrario,
    la integral se anula.
    Concluimos:
    \begin{equation*}
      f(z)
	= \frac{1}{2 \pi \mathrm{i}} \,
	    \int_{C_2} \frac{f(\zeta)}{\zeta - z}
	      \, \mathrm{d} \zeta
    \end{equation*}
    Esta definición vale sobre el disco perforado
    \(\{ z  \colon 0 < \lvert z - z_0 \rvert < R \}\),
    pero la integral define una función holomorfa
    en \(\{z \colon \lvert z - z_0 \rvert < R \}\),
    con lo que definiendo \(f(z_0)\) mediante:
    \begin{equation*}
      f(z_0)
	= \frac{1}{2 \pi \mathrm{i}} \,
	    \int_{C_2} \frac{f(\zeta)}{\zeta - z_0}
	      \, \mathrm{d} \zeta
    \end{equation*}
    la función \(f\) es holomorfa
    en \(\{z \colon \lvert z - z_0 \rvert < R \}\),
    como prometimos.
  \end{proof}
  La condición del teorema~\ref{theo:Riemann-removable-singularity}
  se cumple si \(\lvert f(z) \rvert\) es acotada.
  Como una función holomorfa es continua,
  si la singularidad es removible basta definir:
  \begin{equation*}
    f(z_0)
      = \lim_{z \rightarrow z_0} f(z)
  \end{equation*}

  Cerca de singularidades esenciales las funciones
  se comportan en forma descontrolada.
  El gran teorema de Picard~%
    \cite{picard79:_sur_propr_fonct_entier}
  (para demostración ver textos de análisis complejo,
   como el de Conway\cite{conway78:_funct_one_compl_variab_i})
  muestra que en todo entorno de la singularidad
  la función toma todos los valores posibles,%
    \index{Picard, teorema de}%
    \index{Picard, Charles Emile@Picard, Charles Émile}
  salvo posiblemente uno.
  El resultado siguiente es mucho más restringido,
  pero también más fácil de demostrar:
  \begin{theorem}[Casorati-Weierstraß]
    \index{Casorati-Weierstrass, teorema de@Casorati-Weierstraß, teorema de}
    \label{theo:Casorati-Weierstrass}
    Sea \(f\) holomorfa en el disco perforado
    \(\{ z \colon 0 < \lvert z - z_0 \rvert < R \}\),
    con una singularidad esencial en \(z_0\).
    Entonces dado cualquier \(w \in \mathbb{C}\)
    y números reales arbitrarios
    \(\epsilon > 0\) y \(\delta > 0\)
    hay \(z\) en el disco perforado tal que
    \begin{equation*}
      0 < \lvert z - z_0 \rvert < \delta
      \text{\ y\ }
      \lvert f(z) - w \rvert < \epsilon
    \end{equation*}
  \end{theorem}
  \begin{proof}
    Por contradicción.%
      \index{demostracion@demostración!contradiccion@contradicción}
    Suponga que hay \(w \in \mathbb{C}\),
    \(\epsilon > 0\) y	\(\delta > 0\) tales que
    \(\lvert f(z) - w \lvert \ge \epsilon\)
    siempre que \(0 < \lvert z - z_0 \rvert < \delta\).
    Entonces la función
    \begin{equation*}
      g(z)
	= \frac{1}{f(z) - w}
    \end{equation*}
    es holomorfa y acotada en el disco perforado
    \(\{ z \colon 0 < \lvert z - z_0 \rvert < \delta \}\),
    con una singularidad removible en \(z_0\)
    (teorema~\ref{theo:Riemann-removable-singularity}).
    Definiendo \(g(z_0)\) apropiadamente,
    \(g(z)\) es holomorfa
    en \(\{ z \colon \lvert z - z_0 \rvert < \delta \}\),
    y no es idénticamente cero en este disco.
    Note que:
    \begin{equation*}
      f(z)
	= w + \frac{1}{g(z)}
    \end{equation*}
    con lo que si \(g(z_0) \ne 0\),
    \(f\) es derivable en \(z_0\);
    si \(g(z_0) = 0\) entonces \(f\) tiene un polo en \(z_0\).
    Esto contradice la hipótesis
    de que \(f\) tiene una singularidad esencial
    en \(z_0\).
  \end{proof}

  Un par de ejemplos servirán para clarificar el tema.
  La función \(\mathrm{e}^{1 / z}\)
  es holomorfa en todo \(\mathbb{C}\)
  salvo en \(z = 0\),
  que es una singularidad aislada.
  Si nos restringimos al eje real,
  vemos que:
  \begin{equation*}
    \lim_{x \rightarrow 0^+} \mathrm{e}^{1 / x}
      = \lim_{u \rightarrow \infty} \mathrm{e}^u
      = \infty
  \end{equation*}
  y la singularidad no es removible.
  Por otro lado,
  para \(n \in \mathbb{N}\):
  \begin{equation*}
    \lim_{x \rightarrow 0^+} x^n \mathrm{e}^{1 / x}
      = \lim_{u \rightarrow \infty} \frac{\mathrm{e}^u}{u^n}
      = \infty
  \end{equation*}
  y tampoco es un polo.
  En consecuencia,
  es una singularidad esencial.

  Analicemos ahora la función:
  \begin{equation*}
    f(z)
      = \frac{\mathrm{e}^z - 1}{z (z - 1)}
  \end{equation*}
  Es claro que tiene singularidades aisladas
  en \(z = 0\) y \(z = 1\).
  En \(z = 1\) tiene un polo simple,
  ya que:
  \begin{equation*}
    g(z)
      = (z - 1) f(z)
      = \frac{\mathrm{e}^z - 1}{z}
  \end{equation*}
  y \(g(1) = \mathrm{e} - 1 \ne 0\).
  La singularidad en \(z = 0\) es removible,
  ya que por la regla de l'Hôpital:%
    \index{Hopital, regla de@l'Hôpital, regla de}
  \begin{equation*}
    \lim_{z \rightarrow 0} \frac{\mathrm{e}^z - 1}{z (z - 1)}
      = \lim_{z \rightarrow 0} \frac{\mathrm{e}^z}{2 z - 1}
      = -1
  \end{equation*}
  Podemos analizar lo que ocurre en \(\infty\)
  considerando \(f(1 / z)\),
  que tiene una singularidad aislada en \(0\).
  Según el comportamiento,
  diremos que \(f\) tiene una singularidad removible,
  un polo o una singularidad esencial en \(\infty\).
  Primeramente,
  el límite:
  \begin{equation*}
    \lim_{\lvert z \rvert \rightarrow \infty}
      \frac{\mathrm{e}^z - 1}{z (z - 1)}
  \end{equation*}
  no existe,
  ya que:
  \begin{equation*}
    \lim_{x \rightarrow \infty} \frac{\mathrm{e}^x - 1}{x (x - 1)}
      = \infty
    \qquad
    \lim_{x \rightarrow -\infty} \frac{\mathrm{e}^x - 1}{x (x - 1)}
      = 0
  \end{equation*}
  O sea,
  la singularidad no es removible.
  Enseguida,
  para \(n \in \mathbb{N}\) cualquiera,
  no existe el límite:
  \begin{equation*}
    \lim_{\lvert z \rvert \rightarrow \infty}
      \frac{\mathrm{e}^z - 1}{z^{n + 1} (z - 1)}
  \end{equation*}
  porque:
  \begin{equation*}
    \lim_{x \rightarrow \infty}
	\frac{\mathrm{e}^x - 1}{x^{n + 1} (x - 1)}
      = \infty
    \qquad
    \lim_{x \rightarrow -\infty}
	\frac{\mathrm{e}^x - 1}{x^{n + 1} (x - 1)}
      = 0
  \end{equation*}
  Por tanto,
  al no ser removible ni un polo,
  es una singularidad esencial.

  Es simple demostrar que si
  \(\lim_{\lvert z \rvert \rightarrow \infty}
	\lvert f(z) \rvert = \infty\),
  entonces \(f\) es entera si y solo si es un polinomio.
  O sea,
  una función entera que no es un polinomio
  tiene una singularidad esencial en \(\infty\).
  Supongamos que \(f\) tiene un polo de orden \(m\) en \(\infty\).
  Restando un polinomio \(p\) de \(f\),
  de ser necesario,
  podemos arreglar que \(f(z) - p(z)\)
  tenga un cero de orden \(m\) en \(0\).
  Claramente \(p\) tiene grado a lo más \(m\).
  La función:
  \begin{equation*}
    g(z)
      = \frac{f(z) - p(z)}{z^m}
  \end{equation*}
  es entera y acotada,
  con lo que por el teorema de Liouville
  (teorema~\ref{theo:Liouville})
  es constante.
  Entonces \(f\) es un polinomio.

  La función tangente se define como:
  \begin{equation*}
    \tan z
      = \frac{\sin z}{\cos z}
  \end{equation*}
  Esta función tiene singularidades no removibles
  en los ceros de \(\cos z\),
  que son \((k + 1 / 2) \pi\) para \(k \in \mathbb{Z}\).
  Por l'Hôpital vemos que:%
    \index{Hopital, regla de@l'Hôpital, regla de}
  \begin{equation*}
    \lim_{z \rightarrow \left( k + \frac{1}{2} \right) \, \pi}
       \frac{\cos z}{z - \left( k + \frac{1}{2} \right) \, \pi}
      = \lim_{z \rightarrow
	    \left( k + \frac{1}{2} \right) \, \pi} - \sin z
      = -1
  \end{equation*}
  Como los ceros de \(\cos z\) son simples,
  las singularidades mencionadas son polos simples.
  La singularidad en \(\infty\) no es aislada,
  ya que todo entorno de \(z = 0\) de \(\tan 1 / z\)
  contiene infinitas singularidades
  (no hay \(R\) tal que \(\tan 1 / z\) es holomorfa
   en \(\{ z \colon R < \lvert z \rvert < \infty \}\)).

\subsection{Series de Laurent}
\label{sec:Laurent-series}

  Supongamos la función \(f\) holomorfa en el disco perforado
  \(\{ z \colon 0 < \lvert z - z_0 \rvert < R \}\),
  con un polo de orden \(m\) en \(z_0\).
  De la discusión precedente esto significa que la función siguiente
  es holomorfa
  en \(\{ z \colon \lvert z - z_0 \rvert < R \}\):
  \begin{equation*}
    g(z)
      = (z - z_0)^m f(z)
  \end{equation*}
  Como es holomorfa,
  tiene una serie de Taylor,
  lo que significa que podemos escribir:
  \begin{equation}
    \label{eq:Laurent-series-1}
    f(z)
      = \sum_{k \ge 0} \frac{g^{(k)}(z_0)}{k!} (z - z_0)^{k - m}
  \end{equation}
  A la expresión:
  \begin{equation}
    \label{eq:Laurent-series-pp}
    \frac{g(z_0)}{(z - z_0)^m}
      + \frac{g'(z_0)}{(z - z_0)^{m - 1}}
      + \frac{g''(z_0)}{2! (z - z_0)^{m - 1}}
      + \dotsb
      + \frac{g^{(m - 1)}(z_0)}{(m - 1)! (z - z_0)}
  \end{equation}
  se le llama la \emph{parte principal}%
    \index{parte principal}
  de la serie~\eqref{eq:Laurent-series-1},
  que se anota \(\pp(f; z_0)\).

  Nos interesa formalizar y extender lo anterior.
  \begin{theorem}
    \label{theo:complex-f1+f2-singularity}
    Sea \(f\) una función holomorfa
    en el disco perforado
      \(\{ z \colon 0 < \lvert z - z_0 \rvert < R \}\),
    con una singularidad aislada en \(z_0\).
    Entonces existen funciones únicas \(f_1\) y \(f_2\) tales que:
    \begin{enumerate}[label=(\alph*), ref=(\alph*)]
    \item
      \label{theo:complex-f1+f2-singularity:a}
      \(f(z) = f_1(z) + f_2(z)\)
      en \(\{ z \colon 0 < \lvert z - z_0 \rvert < R \}\)
    \item
      \label{theo:complex-f1+f2-singularity:b}
      \(f_1\) es holomorfa en \(\mathbb{C}\),
      excepto posiblemente en \(z_0\)
    \item
      \label{theo:complex-f1+f2-singularity:c}
      \(f_1(z) \rightarrow 0\)
	 cuando \(\lvert z \rvert \rightarrow \infty\)
    \item
      \label{theo:complex-f1+f2-singularity:d}
      \(f_2\) es holomorfa
	 en \(\{ z \colon \lvert z - z_0 \rvert < R \}\)
    \end{enumerate}
  \end{theorem}
  \begin{proof}
    Comenzamos como la demostración
    del teorema~\ref{theo:Riemann-removable-singularity}.
    Sea \(z\) un punto en el disco perforado
    \(\{ z \colon 0 < \lvert z - z_0 \rvert < R \}\),
    y sean \(0 < r_1 < \lvert z - z_0 \rvert < r_2 < R\).
    Sean además \(C_1\) y \(C_2\)
    las circunferencias
    de radios \(r_1\) y \(r_2\) respectivamente
    alrededor de \(z_0\).
    Obtenemos:
    \begin{equation*}
      f(z)
	= \frac{1}{2 \pi \mathrm{i}} \,
	    \int_{C_2} \frac{f(\zeta)}{\zeta - z}
	      \, \mathrm{d} \zeta
	      - \frac{1}{2 \pi \mathrm{i}} \,
		  \int_{C_1} \frac{f(\zeta)}{\zeta - z}
		    \, \mathrm{d} \zeta
    \end{equation*}
    Defina entonces:
    \begin{equation*}
      f_1(z)
	= - \frac{1}{2 \pi \mathrm{i}} \,
	      \int_{C_1} \frac{f(\zeta)}{\zeta - z}
		\, \mathrm{d} \zeta
     \qquad
      f_2(z)
	= \frac{1}{2 \pi \mathrm{i}} \,
	    \int_{C_2} \frac{f(\zeta)}{\zeta - z}
	      \, \mathrm{d} \zeta
    \end{equation*}
    La parte~\ref{theo:complex-f1+f2-singularity:a} es inmediata.
    Para la parte~\ref{theo:complex-f1+f2-singularity:d},
    la función \(f_2\) definida por la integral es holomorfa
    en el disco \(\{ z \colon \lvert z - z_0 \rvert < r_2 \}\)
    (como en el teorema~\ref{theo:Riemann-removable-singularity}).
    Para~\ref{theo:complex-f1+f2-singularity:b},
    la función \(f_2\) es holomorfa en el anillo
    \(\{ z \colon r_1 < \lvert z - z_0 \rvert < r_2 \}\).
    Como \(f\) y \(f_2\) no dependen de \(r_1\),
    tampoco depende de \(r_1\)
    la función \(f_1(z) = f(z) - f_2(z)\).
    Tampoco depende de \(r_2\) la función \(f_2\),
    por un razonamiento similar.
    Se ve que:
    \begin{equation*}
      \lim_{\lvert z \rvert \rightarrow \infty}
	\int_{C_1} \frac{f(\zeta)}{\zeta - z} \, \mathrm{d} \zeta
	= 0
    \end{equation*}
    que es la parte~\ref{theo:complex-f1+f2-singularity:c}.
    Para demostrar que \(f_1\) y \(f_2\) son únicas,
    suponga funciones \(g_1\) y \(g_2\) con las mismas propiedades
    en el disco perforado
      \(\{ z \colon 0 < \lvert z - z_0 \rvert < R \}\),
    con lo que allí:
    \begin{equation*}
      f_1(z) - g_1(z)
	= g_2(z) - f_2(z)
    \end{equation*}
    Defina:
    \begin{equation*}
      F(z)
	= \begin{cases}
	    g_2(z) - f_2(z) & \lvert z - z_0 \rvert < R \\
	    g_1(z) - f_1(z) & \lvert z - z_0 \rvert \ge R
	  \end{cases}
    \end{equation*}
    Entonces \(F\) es entera.
    Pero por la parte~\ref{theo:complex-f1+f2-singularity:c}
    \(F(z) \rightarrow 0\)
    cuando \(\lvert z \rvert \rightarrow \infty\),
    con lo que \(F\) es acotada.
    Por el teorema de Liouville,%
      \index{Liouville, teorema de}
    \(F(z)\) es constante,
    igual al límite mencionado antes;
    con lo que \(F(z) = 0\) para todo \(z \in \mathbb{C}\),
    y las funciones coinciden.
  \end{proof}
  Estamos en condiciones de completar nuestro ejemplo inicial.
  \begin{theorem}[Serie de Laurent]
    \label{theo:Laurent-series}
    Sea \(f\) holomorfa en el disco perforado
    \(\{ z \colon 0 < \lvert z - z_0 \rvert < R \}\),
    con una singularidad aislada en \(z_0\).
    Sea \(C\) una circunferencia de radio \(r\),
    donde \(0 < r < R\),
    y para cada \(n \in \mathbb{Z}\) sea
    \begin{equation}
      \label{eq:Laurent-series-coefficient}
      a_n
	= \frac{1}{2 \pi \mathrm{i}} \,
	    \int_C \frac{f(z)}{(z - z_0)^{n + 1}} \, \mathrm{d} z
    \end{equation}
    Entonces la serie
    \begin{equation}
      \label{eq:Laurent-series}
      f(z)
	= \sum_n a_n (z - z_0)^n
    \end{equation}
    converge en el disco perforado
      \(\{ z \colon 0 < \lvert z - z_0 \rvert < R \}\).
    Más aún,
    la convergencia es uniforme en todo anillo
      \(\{ z \colon r_1 < \lvert z - z_0 \rvert < r_2 \}\),
    donde \(0 < r_1 < r_2 < R\).
  \end{theorem}
  Nótese que la convergencia uniforme de la serie en el anillo
    \(\{ z \colon r_1 < \lvert z - z_0 \rvert < r_2 \}\)
  significa que para todo \(\epsilon > 0\)
  existe \(N_0 = N_0(\epsilon, r_1, r_2)\),
  independiente de \(z\),
  tal que si \(N_1 \ge N_0\) y \(N_2 \ge N_0\):
  \begin{equation*}
    \left\lvert
      f(z)
	- \sum_{-N_1 \le n \le N_2} a_n (z - z_0)^n
    \right\rvert
      < \epsilon
  \end{equation*}
  A la serie~\eqref{eq:Laurent-series} se la llama
  la \emph{serie de Laurent} de \(f\) alrededor de \(z_0\).%
    \index{Laurent, serie de|textbfhy}
  \begin{proof}
    El primer paso es demostrar
    que la serie~\eqref{eq:Laurent-series}
    converge uniformemente a \(f(z)\)
    sobre la circunferencia \(C\)
    centrada en \(z_0\) de radio \(r\),
    donde \(0 < r < R\),
    y los coeficientes
    están dados por~\eqref{eq:Laurent-series-coefficient}.
    Suponga \(n \in \mathbb{Z}\) dado y fijo.
    Para cualquier \(\epsilon > 0\)
    podemos elegir \(N_1\) y \(N_2\)
    suficientemente grandes para que
    \(-N_1 \le n \le N_2\)
    y tal que para todo \(z \in C\):
    \begin{equation*}
      \left\lvert
	f(z)
	  - \sum_{-N_1 \le k \le N_2} a_k (z - z_0)^k
      \right\rvert
	< \epsilon
    \end{equation*}
    Por la cota~\eqref{eq:complex-integral-bound} es:
    \begin{equation*}
      \left\lvert
	\frac{1}{2 \pi \mathrm{i}} \,
	  \int_C
	    \left(
	      f(z)
	      - \sum_{-N_1 \le k \le N_2} a_k (z - z_0)^k
	    \right) \,
	      \frac{1}{(z - z_0)^n}
	      \, \mathrm{d} z
      \right\rvert
	<\frac{\epsilon}{r^n}
    \end{equation*}
    Como:
    \begin{equation*}
      \frac{1}{2 \pi \mathrm{i}}
	\, \int_C (z - z_0)^k \, \mathrm{d} z
	= [k = -1]
    \end{equation*}
    resulta:
    \begin{equation*}
      \frac{1}{2 \pi \mathrm{i}} \,
	\int_C
	  \left(
	    \sum_{-N_1 \le k \le N_2} a_k (z - z_0)^k
	  \right) \,
	    \frac{1}{(z - z_0)^{n + 1}}
	    \, \mathrm{d} z
       = a_n
    \end{equation*}
    Esto permite simplificar:
    \begin{equation*}
      \left\lvert
	\frac{1}{2 \pi \mathrm{i}} \,
	  \int_C \frac{f(z)}{(z - z_0)^{n + 1}} \, \mathrm{d} z
	  - a_n
      \right\rvert
	<\frac{\epsilon}{r^n}
    \end{equation*}
    Como \(\epsilon\) es arbitrario,
    resulta~\eqref{eq:Laurent-series-coefficient}.

    Queda por demostrar
    que la representación~\eqref{eq:Laurent-series}
    vale en el disco perforado
    \(\{ z \colon 0 < \lvert z - z_0 \rvert < R \}\),
    y que la convergencia es uniforme en todo anillo
    \(\{ z \colon r_1 < \lvert z - z_0 \rvert < r_2 \}\)
    con \(0 < r_1 < r_2 < R\).
    Suponga \(r_1 < r < r_2\).
    Por el teorema~\ref{theo:complex-f1+f2-singularity}
    podemos escribir \(f(z) = f_1(z) + f_2(z)\)
    donde \(f_1\) y \(f_2\) son únicas
    y cumplen las condiciones~\ref{theo:complex-f1+f2-singularity:a}
    a~\ref{theo:complex-f1+f2-singularity:d}.
    Como \(f_2\) es holomorfa en el disco
      \(\{ z \colon \lvert z - z_0 \rvert < R \}\),
    su serie de Taylor:
    \begin{equation*}
      f_2(z)
	= \sum_{n \ge 0} A_n (z - z_0)^n
    \end{equation*}
    converge
    en el disco \(\{ z \colon \lvert z - z_0 \rvert < R \}\),
    uniformemente
    en el disco \(\{ z \colon \lvert z - z_0 \rvert < r_2 \}\).
    Para estudiar \(f_1\),
    efectuamos el cambio de variables:
    \begin{equation*}
      w = \frac{1}{z - z_0}
      \qquad
      z = z_0 + \frac{1}{w}
    \end{equation*}
    La función:
    \begin{equation*}
      f_1(z)
	= f_1 \left( \frac{1}{w} + z_0 \right)
    \end{equation*}
    es entera en \(w\),
    con lo que la serie de Taylor:%
      \index{Taylor, serie de}
    \begin{equation*}
      f_1 \left( \frac{1}{w} + z_0 \right)
	= \sum_{m \ge 1} B_m w^m
    \end{equation*}
    converge en \(\mathbb{C}\),
    uniformemente en el disco cerrado
      \(\{ w \colon \lvert w \rvert \le 1 / r_1 \}\).
    El coeficiente \(B_0\) se anula.
    Es el valor de \(f_1\) en \(w = 0\),
    vale decir \(z = \infty\);
    y por la parte~\ref{theo:complex-f1+f2-singularity:c}
    del teorema~\ref{theo:complex-f1+f2-singularity}
    esto se anula.
    Combinando las series para \(f_1\) en términos de \(z - z_0\)
    y para \(f_2\)
    resulta lo prometido.
  \end{proof}
  El tipo de singularidad aislada
  es sencillo de ver de los coeficientes
  de la serie de Laurent~\eqref{eq:Laurent-series}:
  \begin{corollary}
    \label{cor:Laurent-singularity}
    Sea \(f\) holomorfa en el disco perforado
      \(\{ z \colon 0 < \lvert z - z_0 \rvert < R \}\),
    y serie de Laurent dada por~\eqref{eq:Laurent-series}
    con coeficientes \(a_n\)
    como en~\eqref{eq:Laurent-series-coefficient}.
    Entonces:
    \begin{enumerate}[label=(\alph*), ref=(\alph*)]
    \item
      \label{cor:Laurent-singularity:a}
      La función \(f\) es diferenciable en \(z_0\)
      o tiene una singularidad removible si y solo si
      \(a_n = 0\) para todo \(n < 0\)
    \item
      \label{cor:Laurent-singularity:b}
      La función \(f\) tiene un polo
      si y solo si un número finito pero no nulo
      de coeficientes \(a_n\) con \(n\) negativo
      son diferentes de cero
    \item
      \label{cor:Laurent-singularity:c}
      La función \(f\) tiene una singularidad esencial
      si y solo si un número infinito de coeficientes \(a_n\)
      con \(n\) negativo son diferentes de cero
    \end{enumerate}
  \end{corollary}
  \begin{proof}
    Para la parte~\ref{cor:Laurent-singularity:a},
    sabemos que si \(f\)
    tiene una singularidad removible en \(z_0\),
    podemos hacerla holomorfa definiendo \(f(z_0)\) adecuadamente.
    Y la función holomorfa tiene serie de Taylor,
    vale decir,
    sin términos de índice negativo.

    Para la parte~\ref{cor:Laurent-singularity:b},
    si solo un número finito de coeficientes de índice negativo
    no se anulan,
    habrá \(m > 0\) tal que \(a_{-m} \ne 0\)
    pero \(a_n = 0\) para todo \(n < -m\).
    En tal caso:
    \begin{equation*}
      f(z)
	= \sum_{n  \ge -m} a_n (z - z_0)^n
    \end{equation*}
    de forma que:
    \begin{equation*}
      g(z)
	= (z - z_0)^m f(z)
    \end{equation*}
    es holomorfa en un entorno de \(z_0\),
    y \(g(z_0) = a_{-m} \ne 0\),
    y \(f\) tiene un polo de orden \(m\) en \(z_0\).

    La parte~\ref{cor:Laurent-singularity:c}
    resulta por no ser~\ref{cor:Laurent-singularity:a}
    ni ~\ref{cor:Laurent-singularity:b}.
  \end{proof}
  Como la serie de Laurent es única,
  podemos usar técnicas alternativas
  a la fórmula~\eqref{eq:Laurent-series-coefficient}
  para obtener los coeficientes.
  Por ejemplo,
  de la substitución \(w = 1 / z\) en la serie para \(\mathrm{e}^w\)
  tenemos directamente:
  \begin{equation*}
    \mathrm{e}^{1/z}
      = \sum_{n \ge 0} \frac{1}{n! z^n}
  \end{equation*}
  Nuevamente concluimos
  que la singularidad en \(z = 0\) de esta función
  es esencial.

\subsection{Residuos}
\label{sec:residues}

  Al calcular la integral sobre una curva cerrada simple
  que encierra una región en la que el integrando no es holomorfo,
  el resultado no siempre es cero.
  Primeramente tenemos:
  \begin{lemma}
    \label{lem:integral-residue}
    Sea \(f\) una función holomorfa
    en la región conexa simple \(D\),
    excepto una singularidad aislada en \(z_0\),
    y sea:
    \begin{equation*}
      f_1(z)
	= \sum_{n \le -1} a_n (z - z_0)^n
    \end{equation*}
    la parte principal de la serie de Laurent de \(f\) en \(z_0\).
    Sea también \(\gamma \subset D\) una curva cerrada simple suave
    que se sigue en la dirección positiva,
    que no pasa por \(z_0\).
    Entonces:
    \begin{equation*}
      \frac{1}{2 \pi \mathrm{i}}
	\, \int_\gamma f(\zeta) \, \mathrm{d} \zeta
	= \begin{cases}
	    a_{-1}
	       & \text{si \(z_0\) está en el interior
		       de \(\gamma\)} \\
	    0
	       & \text{si \(z_0\) está en el exterior de \(\gamma\)}
	  \end{cases}
   \end{equation*}
  \end{lemma}
  \begin{proof}
    Si \(z_0\) está al exterior de \(\gamma\),
    vemos que \(\gamma\) es holomorfa a un punto en \(D\),
    y la integral es cero.

    En caso que \(z_0\) está al interior de \(\gamma\),
    la integral no es más que el coeficiente \(a_{-1}\)
    de la serie de Laurent
    según~\eqref{eq:Laurent-series-coefficient}.
  \end{proof}
  Al coeficiente \(a_{-1}\)
  se le llama el \emph{residuo} de \(f\) en \(z_0\),
  y se anota \(a_{-1} = \res(f, z_0)\).

  Con esto podemos demostrar una forma simple
  del teorema de residuos de Cauchy.
  \begin{theorem}[De residuos de Cauchy]
    \index{Cauchy, teorema de residuos de|textbfhy}
    \label{theo:Cauchy-residues}
    Sea \(f\) holomorfa,
    salvo singularidades aisladas \(z_1, \dotsc, z_n\),
    en la región simple conexa \(D\),
    y \(\gamma \subset D\) una curva cerrada simple
    trazada en dirección positiva.
    Entonces:
    \begin{equation}
      \label{eq:Cauchy-integral-residues}
      \int_\gamma f(z) \, \mathrm{d} z
	= \sum_{\substack{1 \le k \le n \\
			  \text{\(z_k\) interior a \(\gamma\)}}}
	    \res(f, z_k)
    \end{equation}
  \end{theorem}
  \begin{proof}
    Sea \(f_k(z)\) la parte principal de \(f(z)\) en \(z_k\).
    Por el teorema~\ref{theo:complex-f1+f2-singularity}
    sabemos que \(f_k\) es holomorfa excepto en \(z_k\),
    por lo que la función:
    \begin{equation*}
      g(z)
	= f(z) - \sum_{1 \le k \le n} f_k(z)
    \end{equation*}
    es holomorfa en \(D\).
    Aplicando el lema~\ref{lem:integral-residue} resulta:
    \begin{equation*}
      \int_\gamma f(z) \, \mathrm{d} z
	= \int_\gamma g(z) \, \mathrm{d} z
	    + \sum_{1 \le k \le n}
		\int_\gamma f_k(z) \, \mathrm{d} z
	= 0 + \sum_{\substack{1 \le k \le n \\
			  \text{\(z_k\) interior a \(\gamma\)}}}
		\res(f_k, z_k)
	= \sum_{\substack{1 \le k \le n \\
			  \text{\(z_k\) interior a \(\gamma\)}}}
	    \res(f, z_k)
    \qedhere
    \end{equation*}
  \end{proof}
  Para que esto resulte útil,
  requerimos formas de calcular el residuo de \(f\)
  en una singularidad \(z_0\).

  Si la singularidad es removible,
  \(f\) tiene una serie de Taylor alrededor de \(z_0\),%
    \index{Taylor, serie de}
  y el residuo es cero.

  Enseguida,
  si \(z_0\) es un polo simple de \(f\),
  entonces tiene serie de Laurent:%
    \index{Laurent, serie de}
  \begin{equation*}
    f(z)
      = \frac{a_{-1}}{z - z_0} + g(z)
  \end{equation*}
  La función \(g(z)\) está representada por una serie de Taylor,
  con lo que es holomorfa en un disco centrado en \(z_0\),
  y \(\lim_{z \rightarrow z_0} (z - z_0) g(z) = 0\).
  O sea:
  \begin{equation*}
    a_{-1}
      = \lim_{z \rightarrow z_0} (z - z_0) f(z)
  \end{equation*}
  Una forma alternativa útil es la siguiente:
  Suponga que \(f(z) = g(z) / h(z)\),
  donde \(g\) y \(h\) son holomorfas,
  \(g(z_0) \ne 0\)
  y \(h\) tiene un cero simple en \(z_0\).
  Entonces,
  como \(h(z_0) = 0\):
  \begin{equation*}
    \lim_{z \rightarrow z_0} (z - z_0) \frac{g(z)}{h(z)}
      = \frac{\lim_{z \rightarrow z_0} g(z)}
	     {\lim_{z \rightarrow z_0} \frac{h(z) - h(z_0)}
					    {z - z_0}}
      = \frac{g(z_0)}{h'(z_0)}
  \end{equation*}

  Si \(z_0\) es un polo de orden \(m\) de \(f\),
  tenemos:
  \begin{equation*}
    f(z)
      = \frac{a_{-m}}{(z - z_0)^m}
	  + \frac{a_{-m + 1}}{(z - z_0)^{m - 1}}
	  + \dotsb
	  + \frac{a_{-1}}{z - z_0}
	  + g(z)
  \end{equation*}
  Entonces es holomorfa:
  \begin{equation*}
    (z - z_0)^m f(z)
      = a_{-m}
	 + a_{- m + 1} (z - z_0)
	 + \dotsb
	 + a_{-1} (z - z_0)^{m - 1}
	 + (z - z_0)^m g(z)
  \end{equation*}
  Derivando \(m - 1\) veces:
  \begin{equation*}
    \frac{\mathrm{d}^{m - 1}}{\mathrm{d} z^{m - 1}} \,
      ((z - z_0)^m f(z))
      = a_{-1} (m - 1)!
	  + \frac{\mathrm{d}^{m - 1}}{\mathrm{d} z^{m - 1}} \,
	      ((z - z_0)^m f(z))
  \end{equation*}
  Como \(g\) es holomorfa,
  el segundo término tiende a 0 cuando \(z \rightarrow z_0\):
  \begin{equation*}
    a_{-1}
      = \frac{1}{(m - 1)!} \,
	  \lim_{z \rightarrow z_0}
	    \frac{\mathrm{d}^{m - 1}}{\mathrm{d} z^{m - 1}} \,
	      ((z - z_0)^m f(z))
  \end{equation*}

  Un resultado que requeriremos más adelante es el siguiente.
  Para \(0 < \alpha < 1\) tenemos la integral real:
  \begin{equation}
    \label{eq:integral-Gamma(z)Gamma(1-z)}
    \int_{-\infty}^\infty
      \frac{\mathrm{e}^{\alpha x}}{1 + \mathrm{e}^x}
	 \, \mathrm{d} x
      = \frac{\pi}{\sin \pi \alpha}
  \end{equation}
  Usamos como contorno
  el rectángulo con vértices en
  \(-R, R, R + 2 \pi \mathrm{i}, -R + 2 \pi \mathrm{i}\),
  luego haremos tender \(R \rightarrow \infty\).
  El numerador del integrando es una función entera,
  el denominador tiene un único cero simple
  en \(z = \pi \mathrm{i}\)
  dentro de la curva:
  \begin{equation*}
    \res \left(
	   \frac{\mathrm{e}^{\alpha z}}{1 + \mathrm{e}^z},
	   \pi \mathrm{i}
	 \right)
      = \frac{\mathrm{e}^{\alpha \pi \mathrm{i}}}
	     {\mathrm{e}^{\pi \mathrm{i}}}
      = - \mathrm{e}^{\alpha \pi \mathrm{i}}
  \end{equation*}
  Veamos las integrales sobre los lados del rectángulo.
  Sea \(I_R\) la integral que nos interesa,
  a lo largo del eje real desde \(-R\) a \(R\);
  y similarmente \(I\) la integral de interés.
  La integral a lo largo del lado superior del rectángulo
  (recordar que estamos integrando de derecha a izquierda)
  es:
  \begin{equation*}
    - \mathrm{e}^{2 \pi \mathrm{i} \, \alpha} I_R
  \end{equation*}
  Finalmente,
  para el lado derecho
    \(V_R = \{ R + \mathrm{i} t \colon 0 \le t \le 2\pi \}\)
  resulta:
  \begin{equation*}
    \left\lvert
      \int_{V_R}
	\frac{\mathrm{e}^{\alpha z}}{1 + \mathrm{e}^z}
	\, \mathrm{d} z
    \right\rvert
      \le \int_0^{2 \pi}
	    \left\lvert
	      \frac{\mathrm{e}^{\alpha (R + \mathrm{i} t)}}
		   {1 + \mathrm{e}^{R + \mathrm{i} t}}
	      \, \mathrm{d} t
	    \right\rvert
      \le \int_0^{2 \pi}
	    \mathrm{e}^{R (\alpha - 1) t}
	      \cdot \lvert \mathrm{e}^{(\alpha - 1) \mathrm{i} t}
	    \rvert
	    \, \mathrm{d} t
      \le C \mathrm{e}^{R (\alpha - 1)}
  \end{equation*}
  Como \(\alpha < 1\),
  esto tiende a \(0\) al tender \(R\) a infinito.
  El lado izquierdo es casi lo mismo.
  Por el teorema de residuos:
  \begin{align*}
    I - \mathrm{e}^{2 \alpha \pi \mathrm{i}} I
      &= - 2 \pi \mathrm{i} \, \mathrm{e}^{\alpha \pi \mathrm{i}} \\
    I
      &= - 2 \pi \mathrm{i} \,
	   \frac{\mathrm{e}^{\alpha \pi \mathrm{i}}}
		{1 - \mathrm{e}^{2 \alpha \pi \mathrm{i}}}
       = \frac{2 \pi \mathrm{i}}
	      {\mathrm{e}^{\alpha \pi \mathrm{i}}
		 - \mathrm{e}^{- \alpha \pi \mathrm{i}}}
       = \frac{\pi}{\sin \alpha \pi}
  \end{align*}

\subsection{Principio del argumento}
\label{sec:argument-principle}

  Aplicado adecuadamente,
  el teorema de residuos de Cauchy
  (teorema~\ref{theo:Cauchy-residues})
  permite calcular el número de ceros y de polos
  de funciones meromorfas.
  \begin{lemma}
    \label{lem:zero-residue}
    Sea \(f\) holomorfa en un entorno de \(z_0\),
    donde tiene un cero de multiplicidad \(m\).
    Entonces la función \(f'(z) / f(z)\)
    es holomorfa en un entorno perforado de \(z_0\),
    con un polo simple en \(z_0\) con residuo \(m\).
  \end{lemma}
  \begin{lemma}
    \label{lem:pole-residue}
    Sea \(f\) holomorfa en un entorno de \(z_0\),
    donde tiene un polo de multiplicidad \(m\).
    Entonces la función \(f'(z) / f(z)\)
    es holomorfa en un entorno perforado de \(z_0\),
    con un polo simple en \(z_0\) con residuo \(-m\).
  \end{lemma}
  \begin{proof}[Demostración del lema~\ref{lem:zero-residue}]
    Podemos escribir \(f(z) = (z - z_0)^m g(z)\),
    con \(g\) holomorfa en un entorno de \(z_0\) y \(g(z_0) \ne 0\).
    Entonces:
    \begin{equation*}
      \frac{f'(z)}{f(z)}
	= \frac{m (z - z_0)^{m - 1} g(z) + (z - z_0)^m g'(z)}
	       {(z - z_0)^m g(z)}
	= \frac{m}{z - z_0} + \frac{g'(z)}{g(z)}
    \end{equation*}
    El segundo término es holomorfo en el entorno de \(z_0\)
    y tenemos lo prometido.
  \end{proof}
  \begin{proof}[Demostración del lema~\ref{lem:pole-residue}]
    Podemos escribir \(f(z) = (z - z_0)^{-m} g(z)\),
    con \(g\) holomorfa en un entorno de \(z_0\) y \(g(z_0) \ne 0\).
    Entonces:
    \begin{equation*}
      \frac{f'(z)}{f(z)}
	= \frac{-m (z - z_0)^{-m - 1} g(z) + (z - z_0)^{-m} g'(z)}
	       {(z - z_0)^{-m} g(z)}
	= \frac{-m}{z - z_0} + \frac{g'(z)}{g(z)}
    \end{equation*}
    El segundo término es holomorfo en el entorno de \(z_0\)
    y tenemos lo prometido.
  \end{proof}
  Uniendo los lemas~\ref{lem:zero-residue} y~\ref{lem:pole-residue}
  resulta:
  \begin{theorem}[Principio del argumento]
    \index{principio del argumento|textbfhy}
    \label{theo:argument-principle}
    Sea \(f\) meromorfa en la región \(D\),
    y \(\gamma \subset D\) una curva cerrada simple
    trazada en dirección positiva,
    y tal que no hayan ceros ni polos de \(f\) sobre \(\gamma\).
    Sea \(N\) el número de ceros de \(f\) al interior de \(\gamma\),
    contados con sus multiplicidades;
    y análogamente \(P\)
    el número de polos de \(f\) al interior de \(\gamma\),
    contados con sus multiplicidades.
    Entonces:
    \begin{equation}
      \label{eq:argument-principle}
      \frac{1}{2 \pi \mathrm{i}} \,
	\int_\gamma \frac{f'(z)}{f(z)} \, \mathrm{d} z
	= N - P
    \end{equation}
  \end{theorem}
  El nombre del teorema~\eqref{theo:argument-principle}
  viene de lo siguiente:
  \begin{equation*}
    \int_\gamma \frac{f'(z)}{f(z)} \, \mathrm{d} z
  \end{equation*}
  es la variación del logaritmo de \(f(z)\) al trazar \(\gamma\),
  como \(\gamma\) es cerrada
  esto es únicamente la variación del argumento de \(f\)
  en unidades de \(2 \pi \mathrm{i}\).

  Para aplicación concreta del principio del argumento
  al contar ceros
  el resultado siguiente permite obviar el principio mismo,
  o al menos aplicarlo a una función más simple.
  \begin{theorem}[Rouché]
    \index{Rouche, teorema de@Rouché, teorema de|textbfhy}
    \label{theo:Rouche}
    Sean \(f\) y \(g\) holomorfas en \(D\),
    y sea \(\gamma \subset D\) una curva conexa simple.
    Suponga además que \(\lvert f(z) \rvert > \lvert g(z) \rvert\)
    sobre \(\gamma\).
    Entonces el número de ceros de \(f\) y \(f + g\)
    al interior de \(\gamma\) es el mismo.
  \end{theorem}
  La demostración que daremos se debe a Chen~%
      \cite{chen08:_intro_complex_anal}.
  \begin{proof}
    Para \(\tau \in [0, 1]\) defina:
    \begin{equation*}
      N(\tau)
	= \int_\gamma \frac{f'(z) + \tau g'(z)}{f(z) + \tau g(z)}
	    \, \mathrm{d} z
    \end{equation*}
    Como en \(\gamma\)
    es \(\lvert f(z) \rvert > \lvert g(z) \rvert\),
    esto asegura:
    \begin{equation*}
      \lvert f(z) + \tau g(z) \rvert
	\ge \lvert f(z) \rvert - \tau \lvert g(z) \rvert
	\ge \lvert f(z) \rvert - \lvert g(z) \rvert
	> 0
    \end{equation*}
    de forma que \(f + \tau g\) no tiene ceros sobre \(\gamma\),
    con lo que \(N(\tau)\) es continua.
    Siendo entero el valor,
    la única posibilidad es que \(N(\tau)\) sea constante.
    Pero \(N(0)\) es el número de ceros de \(f\)
    al interior de \(\gamma\),
    y \(N(1)\) el número de ceros de \(f + g\).
  \end{proof}

\section{Aplicaciones discretas}
\label{sec:complex-discrete-applications}

  Para ir a nuestro tema,
  veremos algunas aplicaciones discretas del análisis complejo.

\subsection{Sumas infinitas}
\label{sec:complex-infinite-sums}

  Para mostrar la técnica general,
  veamos un par de ejemplos.

  Consideremos la función:
  \begin{equation*}
    f(z)
      = \frac{\pi \cot \pi z}{z^2}
  \end{equation*}
  Esta función tiene singularidades en \(z = k\)
  para todo \(k \in \mathbb{Z}\)
  (\(\sin \pi z\) tiene ceros simples allí).
  En \(z = 0\) es un polo de orden 3,
  los demás son polos simples.
  En el origen:
  \begin{equation*}
    \res(f, 0)
      = \frac{1}{2!} \,
	  \lim_{z \rightarrow 0}
	    \frac{\mathrm{d}^2}{\mathrm{d} z^2} z^3 f(z)
      = - \frac{\pi^2}{3}
  \end{equation*}
  Para los demás polos:
  \begin{equation*}
    \res(f, k)
      = \frac{\pi \cos k \pi}{k^2 \pi \cos k \pi + 2 k \sin k \pi}
      = \frac{1}{k^2}
  \end{equation*}
  Nuestra estrategia será hallar una curva
  que encierre todas las singularidades de interés de \(f\),
  y sobre la cual sea sencillo calcular la integral
  (o demostrar que dicha integral tiende a cero).
  Elegir adecuadamente la curva es un arte,
  no hay recetas simples.
  Con eso tenemos la suma.

  Una curva simple de manejar es \(\gamma_n\),
  el cuadrado
  con esquinas en \(\pm (n + 1/2) \pm \mathrm{i}(n + 1/2)\).
  Es importante mantenerse alejado de los polos,
  ya que en ellos la función tiende a infinito
  y acotar la integral sobre un camino que pase cerca de un polo
  será complicado.

  Otras alternativas populares
  son circunferencias centradas en el origen
  cuyo radio tiende a infinito,
  y también semicircunferencias
  (en el semiplano imaginario positivo o negativo)
  completadas con el eje \(x\).

  Consideremos la función \(\cot z\) para \(z = x + \mathrm{i} y\).
  Podemos expresar:
  \begin{equation}
    \label{eq:cot(z)}
    \cot z
      = \frac{\cos x \cosh y - \mathrm{i} \, \sin x \sinh y}
	     {\sin x \cosh y + \mathrm{i} \, \cos x \sinh y}
  \end{equation}
  En los lados verticales del cuadrado es \(\cos x = 0\),
  con lo que \(\sin x = 1\) y~\eqref{eq:cot(z)} queda:
  \begin{align}
    \cot z
      &= \frac{- \mathrm{i} \, \sinh y}{\cosh y}
       = - \mathrm{i} \, \tanh y \notag \\
    \lvert \cot z \rvert
      &= \lvert \tanh y \rvert
       \le 1 \label{eq:complex-cot-vertical-bound}
  \end{align}
  Obtenemos la cota:
  \begin{equation*}
    \left\lvert
	\int_\text{vertical} f(z) \, \mathrm{d} z
    \right\rvert
      \le \max( \lvert f(z) \rvert) \cdot (2 n + 1)
      \le \frac{1}{(n + 1/2)^2} \cdot (2 n + 1)
  \end{equation*}

  En los lados horizontales escribimos:
  \begin{equation*}
    \lvert \cot z \rvert^2
      =	 \frac{\cos^2 x \cosh^2 y + \sin^2 x \sinh^2 y}
	      {\sin^2 x \cosh^2 y + \cos^2 x \sinh^2 y}
  \end{equation*}
  Por~\eqref{eq:cot(z)} podemos expresar:
  \begin{align*}
    \lvert \cot z \rvert^2
      &= \frac{\cos^2 x \cosh^2 y + \sin^2 x \sinh^2 y}
	      {\sin^2 x \cosh^2 y + \cos^2 x \sinh^2 y} \\
      &= \frac{\cos^2 x \cosh^2 y + (1 - \cos^2 x) (\cosh^2 y - 1)}
	      {(1 - \cos^2 x) \cosh^2 y
		+ \cos^2 x (\cosh^2 y - 1)} \\
      &= 1 + \frac{2 \cos^2 x - 1}{\cosh^2 y - \cos^2 x}
  \end{align*}
  Como cuando \(y \rightarrow \pm \infty\)
  también \(\cosh y \rightarrow \infty\),
  para \(y\) suficientemente grande es:
  \begin{equation}
    \label{eq:complex-cot-horizontal-bound}
    \lvert \cot z \rvert
      \le 2
  \end{equation}
  Con este entendido resulta la cota:
  \begin{equation*}
    \left\lvert \int_\text{horizontal} f(z) \, \mathrm{d} z \right\rvert
      \le \max(\lvert f(z) \rvert) \cdot (2 n + 1)
      \le \frac{2}{(n + 1/2)^2} \cdot (2 n + 1)
  \end{equation*}
  Uniendo las anteriores
  tenemos la cota para la integral sobre \(\gamma_n\):
  \begin{equation*}
    \left\lvert \int_{\gamma_n} f(z) \, \mathrm{d} z \right\rvert
      \le 2 \cdot \frac{1}{(n  + 1/2)^2} \cdot (2 n + 1)
	   + 2 \cdot \frac{2}{(n + 1/2)^2} \cdot (2 n + 1)
      = \frac{24}{2 n + 1}
  \end{equation*}
  Esto tiende a cero cuando \(n \rightarrow \infty\).
  Por lo tanto:
  \begin{align*}
    \lim_{n \rightarrow \infty} \int_{\gamma_n} f(z) \, \mathrm{d} z
      &= 0 \\
    \sum_{k \in \mathbb{Z}} \res(f, k \pi)
      &= 0
       = - \frac{\pi^2}{3} + 2 \sum_{k \ge 1} \frac{1}{k^2}
  \end{align*}
  De acá:%
    \index{Basilea, problema de}
  \begin{equation}
    \label{eq:sum-reciprocal-squares}
    \sum_{k \ge 1} \frac{1}{k^2}
      = \frac{\pi^2}{6}
  \end{equation}
  Otra solución al problema de Basilea.
  El mismo método entrega valores
  para \(\zeta(2 n)\) para todo \(n\).

  Consideremos ahora la función:
  \begin{equation*}
    f(z)
      = \frac{\pi \csc \pi z}{z^2}
  \end{equation*}
  Nuevamente tenemos singularidades aisladas
  en \(z \in \mathbb{Z}\).
  En el origen es un polo de orden tres:
  \begin{equation*}
    \res(f, 0)
      = \lim_{z \rightarrow 0} \frac{1}{2!} \,
	  \frac{\mathrm{d}^2}{\mathrm{d} z^2} f(z)
      = \frac{\pi^2}{6}
  \end{equation*}
  Para los demás polos,
  que son todos simples:
  \begin{equation*}
    \res(f, k)
      = \frac{\pi}{\pi k^2 \cos k \pi + 2 k \sin k \pi}
      = \frac{(-1)^k}{k^2}
  \end{equation*}
  Veamos \(\csc z\) para \(z = x + \mathrm{i} y\)
  sobre el mismo cuadrado \(\gamma_n\).
  Primero:
  \begin{equation*}
    \csc z
      = \frac{1}{\sin x \cosh y + \mathrm{i} \, \cos x \sinh y}
  \end{equation*}
  En los lados verticales,
  donde \(\cos x = 0\) y \(\sin x = 1\):
  \begin{equation}
    \label{eq:complex-csc-vertical-bound}
    \lvert \csc z \rvert
      = \frac{1}{\cosh y}
      \le 1
  \end{equation}
  Para los horizontales interesa una cota superior para \(\cot z\),
  que es lo mismo que una cota inferior para \(\sin z\):
  \begin{align*}
    \lvert \sin z \rvert^2
      &= \lvert \sin^2 x \, \cosh^2 y
	    + \cos^2 x \, \sinh^2 y \rvert \\
      &= \lvert (1 - \cos^2 x) (1 + \sinh^2 y)
		   + \cos^2 x \, \sinh^2 y \rvert \\
      &= \lvert 1 - \cos^2 x + \sinh^2 y \rvert
  \end{align*}
  Para \(y \rightarrow \pm \infty\)
  tenemos \(\sinh^2 y \rightarrow \infty\),
  por lo que para \(y\) suficientemente grande
    \(\lvert \csc z \rvert \le 1\).
  Similar a antes:
  \begin{equation*}
    \left\lvert \int_{\gamma_n} f(z) \, \mathrm{d} z \right\rvert
      \le 2 \cdot \frac{1}{(n  + 1/2)^2} \cdot (2 n + 1)
	   + 2 \cdot \frac{1}{(n + 1/2)^2} \cdot (2 n + 1)
      = \frac{16}{2 n + 1}
  \end{equation*}
  Resulta:
  \begin{align*}
    \lim_{n \rightarrow \infty} \int_{\gamma_n} f(z) \, \mathrm{d} z
      &= 0 \\
    \sum_{k \in \mathbb{Z}} \res(f, k \pi)
      &= 0
       = - \frac{\pi^2}{6} + 2 \sum_{k \ge 1} \frac{(-1)^k}{k^2}
  \end{align*}
  De acá:
  \begin{equation}
    \label{eq:sum-alternating-reciprocal-squares}
    \sum_{k \ge 1} \frac{(-1)^k}{k^2}
      = \frac{\pi^2}{12}
  \end{equation}

  Si se revisa el desarrollo,
  es claro que para cualquier función meromorfa \(g(z)\)
  par en \(\mathbb{R}\) y tal que:
  \begin{equation*}
    \lim_{\lvert z \rvert \rightarrow \infty} z g(z) = 0
  \end{equation*}
  podemos aplicar las mismas técnicas para evaluar las sumas:
  \begin{equation*}
    \sum_{k \ge 1} g(k)
    \qquad
    \sum_{k \ge 1} (-1)^k g(k)
  \end{equation*}

  Podemos usar el mismo método para evaluar series.
  Por ejemplo,
  si nos interesa evaluar
  la siguiente serie para \(w \in \mathbb{C}\):
  \begin{equation*}
    \sum_{k \ge 1} \frac{w}{k^2 - w^2}
  \end{equation*}
  Si consideramos \(w\) como una constante,
  podemos aplicar la idea precedente con:
  \begin{equation*}
    f(z)
      = \frac{\pi \cot \pi z}{z^2 - w^2}
  \end{equation*}
  Esta función tiene polos simples en \(z \in \mathbb{Z}\)
  y en \(z = \pm w\).
  Igual que antes,
  en las últimas singularidades:
  \begin{equation*}
    \res(f, \pm w)
      = \lim_{z \rightarrow \pm w}
	  \frac{\pi \cot \pi z}{z \pm w}
      = \pm \frac{\pi \cot (\pm \pi w)}{2 w}
      = \frac{\pi \cot \pi w}{2 w}
  \end{equation*}
  En \(z = k \in \mathbb{Z}\):
  \begin{equation*}
    \res(f, k)
      = \lim_{z \rightarrow k}
	  \frac{\pi \cos \pi z}
	{(z^2 - w^2) \, \pi \cos \pi z + 2 z \sin \pi z}
      = \frac{1}{k^2 - w^2}
  \end{equation*}
  Igual que arriba,
  la integral sobre el cuadrado se anula:
  \begin{align*}
    \sum_{k \in \mathbb{Z}} \frac{1}{k^2 - w^2}
      + 2 \cdot \frac{\pi \cot (\pi w)}{2 w}
      &= 0 \\
    2 \sum_{k \ge 1} \frac{1}{k^2 - w^2}
      - \frac{1}{w^2}
      + \frac{\pi \cot (\pi w)}{w}
      &= 0
  \end{align*}
  \begin{align*}
    \sum_{k \ge 1} \frac{1}{k^2 - w^2}
      &= \frac{1}{2 w^2} - \frac{\pi \cot (\pi w)}{2 w} \\
    \sum_{k \ge 1} \frac{w}{k^2 - w^2}
      &= \frac{1}{2 w} - \frac{\pi}{2} \cot \pi w
  \end{align*}

\subsection{Números de Fibonacci}
\label{sec:complex-Fibonacci-numbers}

  Los números de Fibonacci quedan definidos por la recurrencia
  (ver sección~\ref{sec:Fibonacci}):%
     \index{Fibonacci, numeros de@Fibonacci, números de}
  \begin{equation}
    \label{eq:Fibonacci-recurrence-2}
    F_{n + 2}
      = F_{n + 1} + F_n
    \qquad
    F_0 = 0, F_1 = 1
  \end{equation}
  Sabemos que la función generatriz ordinaria
  es~\eqref{eq:Fibonacci-explicit}:
  \begin{equation}
    \label{eq:Fibonacci-explicit}
    F(z)
      = \sum_{k \ge 0} F_k z^k
      = \frac{z}{1 - z - z^2}
  \end{equation}
  Los ceros del denominador de~\eqref{eq:Fibonacci-explicit}
  son \((-1 \pm \sqrt{5}) / 2\),
  siendo \((-1 + \sqrt{5}) / 2\) el más cercano al origen.
  Esto determina el radio de convergencia
  de la serie.
  Recordamos las definiciones:
  \begin{equation*}
    \tau
      = \frac{1 + \sqrt{5}}{2}
    \qquad
    \phi
      = \frac{1 - \sqrt{5}}{2}
  \end{equation*}
  con lo que:
  \begin{equation*}
    1 - z - z^2
      = - (z + \tau) (z + \phi)
  \end{equation*}
  y también:
  \begin{equation}
    \label{eq:phi*barphi}
    \tau \phi
      = -1
  \end{equation}
  Nuestro operador de extracción de coeficientes es:
  \begin{equation*}
    \left[ z^n \right] F(z)
      = \res \left( \frac{F(z)}{z^{n + 1}}, 0 \right)
      = F_n
  \end{equation*}
  Las otras singularidades son polos simples:
  \begin{align*}
    \res \left( \frac{F(z)}{z^{n + 1}}, -\tau \right)
      &= - \lim_{z \rightarrow -\tau}
	     \frac{1}{z^n (z + \phi)}
      = - \frac{1}{(-\tau)^n (-\tau + \phi)}
      = \frac{\phi^n}{\tau - \phi} \\
    \res \left( \frac{F(z)}{z^{n + 1}}, -\phi \right)
      &= - \lim_{z \rightarrow -\phi}
	     \frac{1}{z^n (z + \tau)}
      = - \frac{1}{(-\phi)^n (-\phi + \tau)}
      = - \frac{\tau^n}{\tau - \phi}
  \end{align*}
  Acá usamos la ecuación~\eqref{eq:phi*barphi} para los recíprocos
  de \(\tau\) y \(\phi\).
  Integremos sobre \(C_R\),
  la circunferencia de radio \(R\) centrada en el origen,
  donde \(R > \tau\).
  Por la desigualdad triangular,%
    \index{desigualdad triangular}
  teorema~\ref{theo:desigualdad-triangular}:
  \begin{equation*}
    \lvert z \rvert^2 - \lvert z \rvert - 1
      \le \lvert z^2 + z - 1 \rvert
  \end{equation*}
  Para \(\lvert z \rvert\) suficientemente grande:
  \begin{equation*}
    \lvert z \rvert^2
       \left(
	 1 - \frac{\lvert z \rvert + 1}{\lvert z \rvert^2}
       \right)
       \ge \frac{1}{2} \lvert z \rvert^2
  \end{equation*}
  Usando esto en la integral:
  \begin{equation*}
    \left\lvert
      \int_{C_R} \frac{z}{z^{n + 1}(1 - z - z^2)} \, \mathrm{d} z
    \right\rvert
      \le \frac{2}{R^{n + 1} R^2}
	    \cdot 2 \pi R
      = \frac{4 \pi}{R^{n + 2}}
  \end{equation*}
  Esto tiende a cero al tender \(R\) a infinito:
  \begin{equation*}
    F_n
      = \frac{1}{\tau - \phi} \,
	  \left(
	    \tau^n - \phi^n
	  \right)
      = \frac{1}{\sqrt{5}} \,
	  \left(
	    \left( \frac{1 + \sqrt{5}}{2} \right)^n
	      - \left( \frac{1 - \sqrt{5}}{2} \right)^n
	  \right)
  \end{equation*}
  De nuevo la fórmula de Binet~\eqref{eq:Binet-Fibonacci}.%
    \index{Binet, formula de@Binet, fórmula de}

% gamma.tex
%
% Copyright (c) 2013-2014 Horst H. von Brand
% Derechos reservados. Vea COPYRIGHT para detalles

\section[La función \texorpdfstring{$\Gamma$}{gamma}]
	{\protect\boldmath
	    La función $\Gamma$
	 \protect\unboldmath}
\label{sec:gamma-function}
\index{\(\Gamma\)|textbfhy}

  Tendremos ocasión de usar la función \(\Gamma\)
  (gamma mayúscula),
  interesa deducir sus propiedades elementales.
  Para \(z\) complejo,
  se define:
  \begin{equation}
    \label{eq:Gamma-definition}
    \Gamma(z)
      = \int_0^\infty t^{z - 1} \mathrm{e}^{-t} \, \mathrm{d} t
  \end{equation}
  Para \(t > 0\) fijo el integrando de~\eqref{eq:Gamma-definition}
  es holomorfo en \(z\),
  con lo que esto define una función holomorfa en el semiplano
  \(\Re z > 1\).%
    \index{C (numeros complejos)@\(\mathbb{C}\) (números complejos)!funcion holomorfa@función holomorfa}
% Fixme: Completar
  Integrando por partes:
  \begin{equation*}
    \Gamma(z)
      = \int_0^\infty t^{z - 1} \mathrm{e}^{-t} \, \mathrm{d} t
      = \frac{1}{z}
	  \, \int_0^\infty \mathrm{e}^{-t} \, \mathrm{d} t^z
      = \left. \frac{1}{z} \, t^z \mathrm{e}^{-t} \right|_0^\infty
	  + \frac{1}{z}
	      \, \int_0^\infty t^z \mathrm{e}^{-t} \, \mathrm{d} t
      = \frac{1}{z} \, \Gamma(z + 1)
  \end{equation*}
  Tenemos así la \emph{fórmula de reducción}:
  \begin{equation}
    \label{eq:Gamma-reduction}
    \index{\(\Gamma\)!formula de reduccion@fórmula de reducción}
    \Gamma(z + 1)
      = z \, \Gamma(z)
  \end{equation}
  También tenemos \(\Gamma(1) = 1\),
  lo que para \(n \in \mathbb{N}\) por la fórmula de reducción da:
  \begin{equation}
    \label{eq:Gamma-factorial}
    \index{\(\Gamma\)!factorial}
    \Gamma(n)
      = (n - 1)!
  \end{equation}
  Incidentalmente,
  como para \(z \in - \mathbb{N}_0\)
  resulta \(1 / \Gamma(z) = 0\)
  es consistente nuestra convención~\eqref{eq:1/k!-convention}
  que \(1 / n! = 0\) si \(n\) es un entero negativo.

  Si \(\Re z > 1\),
  tenemos el valor de~\eqref{eq:Gamma-definition}.
  Fijemos entonces \(z\) con \(\Re z \le 1\),
  y sea \(n\) tal que \(\Re (z + n) > 0\).
  En un entorno de \(z + n\) la función \(\Gamma\) es holomorfa,
  y por~\eqref{eq:Gamma-reduction} tenemos:
  \begin{equation*}
    \Gamma(z + n)
      = z^{\underline{n + 1}} \, \Gamma(z)
  \end{equation*}
  Vale decir,
  si \(\Re z < 0\),
  tiene sentido definir con \(n = \lceil - \Re z \rceil\):%
    \index{potencia!factorial}
  \begin{equation}
    \label{eq:Gamma-reduction-n}
    \Gamma(z)
      = \frac{\Gamma(z + n)}{z^{\underline{n + 1}}}
  \end{equation}
  Es claro
  que esto nos mete en problemas solo si \(z \in -\mathbb{N}_0\).
  En el entorno de \(-n\),
  para \(\lvert w \rvert\) chico,
  la función queda representada por:
  \begin{equation*}
    \Gamma(- n + w)
      = \frac{\Gamma(1 + w)}{(-n + w)^{\underline{n + 1}}}
  \end{equation*}
  Como \(z^{\underline{n + 1}}\) tiene un cero simple en \(-n\),
  vemos que \(\Gamma(z)\)
  tiene polos simples en los enteros negativos.
  Es fácil calcular sus residuos:%
    \index{\(\Gamma\)!residuo}
  \begin{equation}
    \label{eq:Gamma-residues}
    \begin{split}
      \res(\Gamma, 0)
	&= \lim_{z \rightarrow 0} z \frac{\Gamma(z + 1)}{z}
	 = 1 \\
      \res(\Gamma, -n)
	&= \lim_{z \rightarrow -n} (z + n)
	     \frac{\Gamma(z + n + 1)}{z^{\underline{n + 1}}}
	 = \frac{1}{(-n)^{\underline{n}}}
	 = \frac{(-1)^n}{n!}
    \end{split}
  \end{equation}

  La función \(\Gamma\) satisface muchas identidades notables.
  Por ejemplo,
  tenemos:
  \begin{theorem}[Fórmula de reflexión de Euler]
    \index{Euler, formula de reflexion para \(\Gamma\)@Euler, fórmula de reflexión para \(\Gamma\)|see{\(\Gamma\)!fórmula de reflexión}}
    \index{\(\Gamma\)!formula de reflexion@fórmula de reflexión}
    Se cumple:
    \begin{equation}
      \label{eq:Gamma-reflection}
      \Gamma(z) \, \Gamma(1 - z)
	= \frac{\pi}{\sin \pi z}
      \end{equation}
  \end{theorem}
  Seguimos la demostración de Stein y~Shakarchi~%
    \cite{stein10:_compl_analy}.
  \begin{proof}
    Primeramente, \(\pi / \sin \pi z\)
    tiene polos en \(\mathbb{Z}\).
    El residuo en \(n \in \mathbb{Z}\) es:
    \begin{equation*}
      \res \left( \frac{\pi}{\sin \pi z}, n \right)
	= \frac{\pi}{\pi \cos \pi n}
	= (-1)^n
    \end{equation*}
    La función \(\Gamma(1 - z)\) tiene polos simples
    en \(z \in - \mathbb{N}\),
    allí \(\Gamma(z)\) es holomorfa.
    Para \(n \in \mathbb{N}\):
    \begin{equation*}
      \res(\Gamma(z) \, \Gamma(1 - z), - n)
	= \res(\Gamma(z), -n) \, \Gamma(n + 1)
	= \frac{(-1)^n}{n!} \, n!
	= (-1)^{-n}
    \end{equation*}
    De la misma forma,
    para \(n \in \mathbb{N}_0\):
    \begin{equation*}
      \res(\Gamma(z) \, \Gamma(1 - z), n)
	= (-1)^n
    \end{equation*}
    O sea,
    los polos y residuos respectivos de ambas funciones coinciden.

    Enseguida,
    \(\Gamma(z) \, \Gamma(1 - z)\)
    y \(\pi / \sin \pi z\) son ambas periódicas,
    con período 1:
    \begin{equation*}
      \Gamma(z + 1) \, \Gamma(1 - (z + 1))
	= z \Gamma(z) \cdot \frac{\Gamma(1 - z)}{z}
	= \Gamma(z) \, \Gamma(1 - z)
    \end{equation*}

    Finalmente,
    demostramos que ambas funciones coinciden en \(0 < s < 1\).
    Podemos escribir:
    \begin{equation*}
      \Gamma(1 - s)
	= \int_0^\infty u^{-s} \mathrm{e}^{-u} \, \mathrm{d} u
	= t \int_0^\infty \mathrm{e}^{-v t} (v t)^{-s}
	      \, \mathrm{d} v
    \end{equation*}
    Acá usamos el cambio de variables \(u = v t\).
    Luego:
    \begin{align*}
      \Gamma(s) \, \Gamma(1 - s)
	&= \int_0^\infty
	       \mathrm{e}^{-t} t^{s - 1} \, \Gamma(1 - s)
	     \, \mathrm{d} t \\
	&= \int_0^\infty
	     \mathrm{e}^{-t} t^{s - 1}
	       \left(
		 t \int_0^\infty \mathrm{e}^{- v t} (v t)^{-s}
		     \, \mathrm{d} v
	       \right) \, \mathrm{d} t \\
	&= \int_0^\infty
	     \int_0^\infty
	       \mathrm{e}^{-t (1 + v)} v^{-s} \,
		 \mathrm{d} v \, \mathrm{d} t \\
	&= \int_0^\infty \frac{v^{-s}}{1 + v} \, \mathrm{d} v
    \end{align*}
    Con el cambio de variables \(v = \mathrm{e}^t\)
    queda la integral que evaluamos
    en~\eqref{eq:integral-Gamma(z)Gamma(1-z)}:
    \begin{equation*}
      \int_{-\infty}^\infty
	\frac{\mathrm{e}^{- s t}}{1 + \mathrm{e}^t} \, \mathrm{d} t
	= \frac{\pi}{\sin s \pi}
    \end{equation*}

    Uniendo todas las piezas,
    ambas funciones deben ser iguales.
  \end{proof}
  De acá resulta directamente
  el valor para argumento no entero más usado:
  \begin{align}
    ( \Gamma( 1/2 ))^2
      &= \Gamma(1/2) \, \Gamma(1 - 1/2)
       = \frac{\pi}{\sin \pi / 2}
       = \pi \notag \\
    \Gamma(1/2)
      &= \sqrt{\pi} \label{eq:Gamma(1/2)}
  \end{align}

  Íntimamente relacionada es la función \(\mathrm{B}\)
  (beta mayúscula),%
    \index{\(\mathrm{B}\)|textbfhy}
  definida para \(\Re x, \Re y > 0\):
  \begin{equation}
    \label{eq:definition-Beta}
    \mathrm{B}(x, y)
      = \int_0^1 t^{x - 1} (1 - t)^{y - 1} \, \mathrm{d} t
  \end{equation}
  Es claro que es simétrica:
  \begin{equation}
    \label{eq:Beta-symmetry}
    \mathrm{B}(x, y)
      = \mathrm{B}(y, x)
  \end{equation}
  También:
  \begin{equation*}
    \Gamma(x) \, \Gamma(y)
      = \int_0^\infty \mathrm{e}^{-u} u^{x - 1} \, \mathrm{d} u
	  \int_0^\infty \mathrm{e}^{-v} v^{y - 1} \, \mathrm{d} v
      = \int_0^\infty \int_0^\infty
	  \mathrm{e}^{-u - v} u^{x - 1} v^{y - 1}
	     \, \mathrm{d} u \, \mathrm{d} v
  \end{equation*}
  El cambio de variables \(u = s t\) y \(v = s (1 - t)\) da:
  \begin{equation*}
    \Gamma(x) \, \Gamma(y)
      = \int_0^\infty \mathrm{e}^{-s} s^{x + y - 1} \, \mathrm{d} s
	  \int_0^1 t^{x - 1} (1 - t)^{y - 1} \, \mathrm{d} t
      = \Gamma(x + y) \mathrm{B}(x, y)
  \end{equation*}
  de donde resulta la identidad básica,
  que sirve para definir \(\mathrm{B}(x, y)\):
  \begin{equation}
    \label{eq:Gamma-Beta}
    \mathrm{B}(x, y)
      = \frac{\Gamma(x) \, \Gamma(y)}{\Gamma(x + y)}
  \end{equation}

  Por la fórmula de reducción~\eqref{eq:Gamma-reduction-n}%
    \index{\(\Gamma\)!formula de reduccion@fórmula de reducción}
  tenemos de~\eqref{eq:coeficiente-binomial}:%
    \index{coeficiente binomial}
  \begin{equation*}
    \binom{\alpha}{n}
      = \frac{\alpha^{\underline{n}}}{n!}
      = \frac{\Gamma(\alpha + 1)}{\Gamma(\alpha - n + 1) \, n!}
  \end{equation*}
  Con~\eqref{eq:Gamma-factorial}
  esto se parece a~\eqref{eq:coeficiente-binomial-factorial},
  que hace sentido adoptar como definición:
  \begin{equation}
    \label{eq:complex-binomial-coefficient}
    \binom{m + n}{m}
      = \frac{(m + n)!}{m! \, n!}
      = \frac{\Gamma(m + n + 1)}{\Gamma(m + 1) \, \Gamma(n + 1)}
  \end{equation}

%%% Local Variables:
%%% mode: latex
%%% TeX-master: "clases"
%%% End:


%%% Local Variables:
%%% mode: latex
%%% TeX-master: "clases"
%%% End:

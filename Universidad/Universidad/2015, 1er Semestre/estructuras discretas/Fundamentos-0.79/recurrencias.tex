% recurrencias.tex
%
% Copyright (c) 2013-2015 Horst H. von Brand
% Derechos reservados. Vea COPYRIGHT para detalles

\chapter{Recurrencias}
\label{sec:recurrencias}
\index{recurrencia|textbfhy}

  Es común encontrarse con situaciones
  en las cuales debemos resolver alguna \emph{recurrencia},
  vale decir,
  tenemos una ecuación que relaciona valores de una secuencia
  (generalmente adicionando algunos valores iniciales).
  Esto aparece tanto en la solución de problemas combinatorios
  como en el análisis de diversos algoritmos.
  Algunas de las técnicas que discutiremos
  se desarrollaron inicialmente
  para su aplicación en campos diversos como la economía
  o el control de procesos.
  Quienes hayan profundizado en el estudio
  de la solución de ecuaciones diferenciales
  hallarán paralelos sorprendentes
  (y divergencias importantes)
  con esa área.
  No disponemos de espacio
  para estudiar ese fenómeno en más detalle
  (ni es de nuestro interés inmediato).

\section{Definición del problema}
\label{sec:definicion-problema-recurrencia}

  La situación general puede describirse:
  \begin{equation}
    \label{eq:recurrence-general}
    f(a_n, a_{n + 1}, \dotsc, a_{n + k}, n)
      = 0
  \end{equation}
  Si en~\eqref{eq:recurrence-general} aparecen \(a_n\) y \(a_{n + k}\),
  se habla de una \emph{recurrencia de orden \(k\)}.%
    \index{recurrencia!orden|textbfhy}
  En general harán falta \(k\) valores para determinar la secuencia,
  típicamente dados como \(a_0\) hasta \(a_{k - 1}\),
  de~\eqref{eq:recurrence-general} podremos obtener \(a_k\),
  con \(a_1\) a \(a_k\) tenemos \(a_{k + 1}\),
  y así sucesivamente.
  Es claro que hallar una expresión cerrada
  para los términos de tales secuencias
  será posible solo en situaciones especiales.

\section{Recurrencias lineales}
\label{sec:recurrencias-lineales-teoria}
 \index{recurrencia!lineal|textbfhy}

  Una recurrencia se dice \emph{lineal} si puede escribirse de la forma:
  \begin{equation}
    \label{eq:recurrence-linear}
    u_k(n) a_{n + k}
	+ u_{k - 1} (n) a_{n + k - 1}
	+ \dotsb
	+ u_0(n) a_n
      = f(n)
  \end{equation}
  donde \(u_i(n)\) y \(f(n)\) son funciones conocidas.
  Si tanto \(u_k\) como \(u_0\) son diferentes de cero,
  es una \emph{recurrencia de orden \(k\)}.
  Si \(f(n) = 0\),
  se dice que la recurrencia es \emph{homogénea},
    \index{recurrencia!lineal!homogenea@homogénea|textbfhy}
  en caso contrario \emph{no homogénea}.
  El estudio de las recurrencias lineales
  hace uso del álgebra lineal,
  para mayores detalles véanse por ejempo Strang~%
    \cite{strang09:_intr_linear_algebra}
  o Treil~%
    \cite{treil14:_linear_algeb_done_wrong}.

  Es claro que si las secuencias \(\langle x_n \rangle\)
  y \(\langle y_n \rangle\) satisfacen una recurrencia lineal homogénea,
  la combinación lineal \(\langle \alpha x_n + \beta y_n \rangle\)
  también la satisface.
  Esta es la razón del nombre.
  Si expresamos la recurrencia homogénea como:
  \begin{equation}
    \label{eq:recurrence-linear-homogeneous}
    a_{n + k}
      = u_{k - 1}(n) a_{n + k - 1}
	 + u_{k - 2}(n) a_{n + k - 2}
	 + \dotsb
	 + u_0(n) a_n
  \end{equation}
  con los vectores%
    \index{vector}
    \(\boldsymbol{a}_n = (a_{n + k - 1}, a_{n + k - 2}, \dotsc, a_n)\)
  podemos expresar la recurrencia~\eqref{eq:recurrence-linear-homogeneous}
  como:%
    \index{matriz}
  \begin{equation}
    \label{eq:recurrence-linear-homogeneous-matrix-form}
    \boldsymbol{a}_{n + 1}
      = \boldsymbol{U}_n \cdot \boldsymbol{a}_n
  \end{equation}
  donde:
  \begin{equation}
    \label{eq:recurrence-linear-homogeneous-matrix}
    \boldsymbol{U}_n
      = \begin{pmatrix}
	  u_{k - 1}(n) & u_{k - 2}(n) & \cdots & u_1(n) & u_0(n) \\
	  1	       & 0	      & \cdots & 0	&   0	 \\
	  0	       & 1	      & \cdots & 0	&   0	 \\
	  \vdots       & \vdots	      & \ddots & \vdots & \vdots \\
	  0	       & 0	      & \cdots & 1	&   0
	\end{pmatrix}
  \end{equation}
  lo que nos permite expresar:
  \begin{equation}
    \label{eq:recurrence-linear-homogeneous-matrix-soln}
    \boldsymbol{a}_n
      = \boldsymbol{U}_{n - 1}
	  \cdot \boldsymbol{U}_{n - 2}
	  \cdot \dotsb
	  \cdot \boldsymbol{U}_0
	  \cdot \boldsymbol{a}_0
  \end{equation}
  Esto dice que la solución de la ecuación lineal homogénea
  es la combinación lineal
  de \(k\) soluciones linealmente independientes:%
    \index{recurrencia!lineal!solucion@solución|textbfhy}
  Podemos elegir
  los \(k\) componentes de \(\boldsymbol{a}_0\) independientemente,
  si ninguna de las matrices \(\boldsymbol{U}_n\) es singular
  vectores iniciales linealmente independientes
  darán vectores finales linealmente independientes.
  La solución general de la recurrencia lineal no homogénea
  puede expresarse como una solución particular%
    \index{recurrencia!lineal!solucion particular@solución particular}
  y la combinación lineal de \(k\) soluciones linealmente independientes
  de la recurrencia homogénea.

  En caso que \(\boldsymbol{U}_n\) sea una matriz constante \(\boldsymbol{U}\)
  (la recurrencia tiene coeficientes constantes),
  la ecuación~\eqref{eq:recurrence-linear-homogeneous-matrix-soln}
  se reduce a:
  \begin{equation}
    \label{eq:recurrence-linear-homogeneous-matrix-const-soln}
    \boldsymbol{a}_n
      = \boldsymbol{U}^n \cdot \boldsymbol{a}_0
  \end{equation}
  Técnicas eficientes para el cálculo de potencias
  (ver la sección~\ref{sec:potencias})%
    \index{potencia!calculo@cálculo}
  permiten obtener el vector \(\boldsymbol{a}_n\)
  rápidamente de~\eqref{eq:recurrence-linear-homogeneous-matrix-const-soln}.
  Esto ofrece una alternativa a las técnicas expuestas
  en el capítulo~\ref{cha:funciones-generatrices}.%
    \index{generatriz}

  Una matriz \(\boldsymbol{A}\) se dice \emph{diagonalizable}%
    \index{matriz!diagonalizable|textbfhy}
  si hay una matriz diagonal \(\boldsymbol{D}\)%
    \index{matriz!diagonal}
  y una matriz invertible \(\boldsymbol{P}\)%
    \index{matriz!invertible}
  tales que
    \(\boldsymbol{A} = \boldsymbol{P}^{-1} \boldsymbol{D} \boldsymbol{P}\).
  Calcular potencias de una matriz expresada de esta forma
  es particularmente simple,
  ya que
    \(\boldsymbol{A}^n = \boldsymbol{P}^{-1} \boldsymbol{D}^n \boldsymbol{P}\),
  y la potencia de la matriz diagonal
  es simplemente las potencias de sus elementos.
  Los elementos de la matriz \(\boldsymbol{D}\)
  resultan ser los valores propios de \(\boldsymbol{A}\),%
    \index{matriz!valor propio}
  las soluciones de la ecuación
    \(\det(\boldsymbol{A} - \lambda \boldsymbol{I}) = 0\).
  Esto da una forma alternativa
  de expresar la solución de la recurrencia~\eqref{eq:recurrence-linear}
  para el caso de coeficientes constantes.

\section{Recurrencias lineales de primer orden}
\label{sec:rec-lineal-1er}

  Un caso de particular interés práctico son las recurrencias lineales
  de primer orden,%
    \index{recurrencia!lineal!primer orden}
  que podemos escribir:
  \begin{equation}
    \label{eq:recurrencia-lineal-1}
    a_{n + 1}
      = u_n a_n + f_n
  \end{equation}
  Vemos que si dividimos~\eqref{eq:recurrencia-lineal-1}
  por el \emph{factor sumador}:%
    \index{recurrencia!lineal!factor sumador}
  \begin{equation}
    \label{eq:recurrencia-lineal-2}
    s_n
      = \prod_{0 \le k \le n} u_k
  \end{equation}
  (lo que presupone que \(u_k \ne 0\) en el rango de interés)
  queda:
  \begin{equation*}
    \frac{a_{n + 1}}{s_n} - \frac{a_n}{s_{n - 1}}
      = \frac{f_n}{s_n}
  \end{equation*}
  Sumando ambos lados obtenemos la solución.

  La fórmula general es bastante engorrosa,
  ilustraremos la técnica mediante un ejemplo.
  % http://math.stackexchange.com 135803
  Sea:
  \begin{equation}
    \label{eq:recurrencia-ejemplo-1}
    a_{n + 1}
      = \frac{2 (n + 1) a_n + 5 (n + 1)!}{3}
      \qquad a_0 = 5
  \end{equation}
  Reordenando un poco:
  \begin{equation*}
    a_{n + 1} - \frac{2 (n + 1)}{3} \, a_n
      = \frac{5 (n + 1)!}{3}
  \end{equation*}
  Vemos que el factor sumador es:
  \begin{align*}
    s_n
      &= \prod_{0 \le k \le n} \frac{2 (n + 1)}{3} \\
      &= \left( \frac{2}{3} \right)^{n + 1} \, (n + 1)!
  \end{align*}
  Dividiendo la recurrencia por esto y sumando para \(0 \le k \le n - 1\)
  resulta:
  \begin{align*}
    \frac{a_n}{(2/3)^n n!} - \frac{a_0}{s_{-1}}
      &= \frac{5}{3} \, \sum_{0 \le k \le n - 1}
			  \left( \frac{3}{2} \right)^{k + 1} \\
    \frac{a_n}{(2/3)^n n!} - \frac{5}{1}
      &= \frac{5}{3} \cdot \frac{3}{2} \cdot \frac{(3/2)^n - 1}{3/2 - 1} \\
    \frac{a_n}{(2/3)^n n!}
      &= 5 + 5 \cdot \left( (3/2)^n - 1 \right) \\
      &= 5 \cdot (3/2)^n \\
    a_n
      &= 5 n!
  \end{align*}

  A veces esto sirve para simplificar sumas.
  Siguiendo esencialmente la estrategia de Rockett~%
    \cite{rockett81:_sums_inver_binom_coeff}
  calcularemos:
  \begin{equation*}
    \sum_{0 \le k \le n} \binom{n}{k}^{-1}
      = \frac{1}{n!} \sum_{0 \le k \le n} k! (n - k)!
  \end{equation*}
  Nos concentramos en la suma:
  \begin{align}
    S_n
      &= \sum_{0 \le k \le n} k! (n - k)!
	    \label{eq:sum-k!*(n-k)!-def} \\
      &= \sum_{0 \le k \le n - 1} k! (n - k)! + n! \notag \\
      &= \sum_{0 \le k \le n - 1} k! (n - 1 - k)! ((n + 1) - (k + 1))
	   + n! \notag \\
      &= (n + 1) \sum_{0 \le k \le n - 1} k! (n - 1 - k)!
	   - \sum_{0 \le k \le n - 1} k! (n - 1 - k)! (k + 1)
	   + n! \notag \\
      &= (n + 1) S_{n - 1}
	   - \sum_{0 \le k \le n - 1} (k + 1)! (n - (k + 1))!
	   + n! \notag \\
      &= (n + 1) S_{n - 1}
	   - \sum_{1 \le k \le n} k! (n - k)!
	   + n! \notag \\
      &= (n + 1) S_{n - 1}
	   - (S_n - n!)
	   + n! \notag \\
   2 S_n
      &= (n + 1) S_{n - 1} + 2 n!
	    \label{eq:sum-k!*(n-k)!-rec}
  \end{align}
  Esta es una recurrencia lineal de primer orden.%
    \index{recurrencia!lineal!primer orden}
  De la definición~\eqref{eq:sum-k!*(n-k)!-def} es \(S_0 = 1\).

  El factor sumador de~\eqref{eq:sum-k!*(n-k)!-rec} es \(2^{-n} (n + 1)!\):%
    \index{factor sumador|see{recurrencia!lineal!primer orden}}
  \begin{align*}
    \frac{2^{n + 1}}{(n + 1)!} S_n
      &= \frac{2^n}{n!} S_{n - 1} + \frac{2^{n + 1}}{n + 1} \\
    \frac{2^{n + 1}}{(n + 1)!} S_n - \frac{2}{1!} S_0
      &= \sum_{1 \le k \le n} \frac{2^{k + 1}}{k + 1} \\
  \intertext{Casualmente coincide con el término para \(k = 0\):}
    \frac{2^{n + 1}}{(n + 1)!} S_n
      &= \sum_{0 \le k \le n} \frac{2^{k + 1}}{k + 1}
  \end{align*}
  O sea:
  \begin{equation}
    \label{eq:sum-k!*(n-k)!-value}
    S_n
      = \frac{(n + 1)!}{2^{n + 1}}
	  \sum_{0 \le k \le n} \frac{2^{k + 1}}{k + 1}
  \end{equation}
  Con esto nuestra suma original es:
  \begin{equation}
    \label{eq:reciprocal-binomial-sum}
    \sum_{0 \le k \le n} \binom{n}{k}^{-1}
      = \frac{n + 1}{2^{n + 1}} \sum_{0 \le k \le n} \frac{2^{k + 1}}{k + 1}
  \end{equation}

\section{Recurrencias lineales de coeficientes constantes}
\label{sec:recurrencias-lineales}
\index{recurrencia!lineal!coeficientes constantes}

  Un caso particularmente importante
  es el de relaciones de recurrencias de la forma:
  \begin{equation*}
    a_k u_{n + k} + a_{k - 1} u_{n + k - 1} + \dotsb + a_0 u_n = f(n)
  \end{equation*}
  donde los \(a_i\) son constantes y \(f(n)\) es una función cualquiera.
  Esto se llama
  una \emph{relación de recurrencia lineal de coeficientes constantes}
  (de \emph{orden k}, si \(a_k \ne 0\)).
  Si \(f(n) = 0\),
  se dice \emph{homogénea}.
  Se requieren \(k\) condiciones adicionales para fijar la solución,
  que generalmente toman la forma de \emph{condiciones iniciales}
  dando los valores de \(u_0\) hasta \(u_{k - 1}\).
  Esto completa una \emph{recurrencia lineal}.
  La recurrencia de Fibonacci que resolvimos antes
  es una recurrencia de segundo orden,
  lineal,
  de coeficientes constantes,
  homogénea.
  La recurrencia a la que nos llevó
  la Competencia de Ensayos de la Universidad de Miskatonic
  (sección~\ref{sec:conjetura->teorema})
  es de primer orden,
  lineal de coeficientes constantes,
  no homogénea.

  Tratar el caso general es bastante engorroso,
  mostraremos el procedimiento mediante un ejemplo.
  De forma similar a la aplicación de funciones generatrices ordinarias
  presentada acá
  pueden aplicarse funciones generatrices exponenciales
  como lo hicimos en la sección~\ref{sec:Fibonacci-exponenciales}
  para los números de Fibonacci.
  Cuál se usa en un caso particular
  dependerá de lo que resulte más simple.

  Consideremos la recurrencia:
  \begin{equation*}
    u_{n + 2} + 4 u_n = 5 n^2 \qquad \text{\(u_0 = 1\), \(u_1 = 3\)}
  \end{equation*}
  Los primeros valores son:
  \begin{equation*}
    \left\langle
      1, 3, -4, -7, 36, 73, -64, -167, 436, 913, \dotsc
    \right\rangle
  \end{equation*}
  Aplicando nuestra estrategia general,
  definimos:
    \index{recurrencia!generatriz}
  \begin{equation*}
    U(z) = \sum_{n \ge 0} u_n z^n
  \end{equation*}
  Siguiendo la receta,
  y aplicando las propiedades de funciones generatrices ordinarias queda:
  \begin{equation*}
    \frac{U(z) - 3 z - 1}{z^2} + 4 \, U(z)
      = 5 (z \mathrm{D})^2 \left(\frac{1}{1 - z}\right) \\
  \end{equation*}
  Despejando y expresando en fracciones parciales:
  \begin{equation*}
    U(z)
      = \frac{82 z + 37}{25 (1 + 4 z^2)}
	  + \frac{2}{(1 - z)^3}
	  - \frac{19}{5 (1 - z)^2}
	  + \frac{33}{25 (1 - z)}
  \end{equation*}
  Luego basta ``leer'' los coeficientes en esto.
  Puntos interesantes ponen los términos:
  \begin{align*}
    \frac{1}{1 + 4 z^2}
      &= \sum_{n \ge 0} (-4)^n z^{2 n} \\
    \frac{z}{1 + 4 z^2}
      &= \sum_{n \ge 0} (-4)^n z^{2 n + 1} \\
    \frac{1}{(1 - z)^3}
      &= \sum_{n \ge 0} \binom{-3}{n} \, (-1)^n z^n \\
      &= \sum_{n \ge 0} \frac{(n + 2) (n + 1)}{2} \, z^n \\
    \frac{1}{(1 - z)^2}
      &= \sum_{n  \ge 0} \binom{-2}{n} \, (-1)^n z^n \\
      &= \sum_{n \ge 0} (n + 1) z^n
  \end{align*}
  Podemos entonces expresar la solución como:
  \begin{align*}
    u_{2 k}
      &= \frac{37}{25} \cdot (-4)^k
	   + 2 \cdot \frac{(2 k + 2) (2 k + 1)}{2}
	   - \frac{19}{5} \cdot (2 k + 1)
	   + \frac{33}{25} \\
      &= \frac{37}{25} \cdot (-4)^k
	   + 4 k^2 - \frac{8}{5} k - \frac{12}{25} \\
    u_{2 k + 1}
      &= \frac{82}{25} \cdot (-4)^k
	   + 2 \cdot \frac{(2 k + 3) (2 k + 2)}{2}
	   - \frac{19}{5} \cdot (2 k + 2)
	   + \frac{33}{25} \\
      &= \frac{82}{25} \cdot (-4)^k
	   + 4 k^2
	   + \frac{12}{5} k
	   - \frac{7}{25}
  \end{align*}
  Esta separación en términos pares e impares es incómoda.
  Usando números complejos podemos factorizar más:%
    \index{C (numeros complejos)@\(\mathbb{C}\) (números complejos)}
  \begin{align*}
    1 + 4 z^2
      &= (1 - 2 \mathrm{i} z) (1 + 2 \mathrm{i} z) \\
    \frac{1}{1 + 4 z^2}
      &= \frac{1}{2} \, \left(
			  \frac{1}{1 + 2 \mathrm{i} z}
				     + \frac{1}{1 - 2 \mathrm{i} z}
			\right) \\
    \frac{z}{1 + 4 z^2}
      &= \frac{\mathrm{i}}{4} \,
	   \left(
	     \frac{1}{1 + 2 \mathrm{i} z}
	       - \frac{1}{1 - 2 \mathrm{i} z}
	   \right)
  \end{align*}
  y podemos entonces expresar:
  \begin{align*}
    u_n
      &= \frac{82}{25} \cdot
	   \frac{\mathrm{i}}{4} \,
	     \bigl(
	       (-2 \mathrm{i})^n - (2 \mathrm{i})^n
	     \bigr)
	   + \frac{37}{25} \cdot
		\frac{1}{2} \,
		  \bigl(
		    (-2 \mathrm{i})^n + (2 \mathrm{i})^n
		  \bigr) \\
      &\hspace{3em}
	   + 2 \cdot \frac{(n + 2) (n + 1)}{2}
	   - \frac{19}{5} \cdot (n + 1)
	   + \frac{33}{25} \\
      &= \frac{\left(
		 (37 - 41 \mathrm{i}) \mathrm{i}^n
		    + (37 + 41 \mathrm{i}) (-\mathrm{i})^n
	       \right) 2^n
	      }{50}
	   + n^2
	   - \frac{4 n}{5}
	   - \frac{12}{25}
  \end{align*}
  Es bien poco probable que hubiéramos adivinado esta solución\ldots

  Alternativamente,
  para el término:
  \begin{equation*}
      \frac{87 z + 37}{25 (1 + 4 z^2)}
      = \frac{37 - 41 \mathrm{i}}{2} \cdot \frac{1}{1 - 2 \mathrm{i} z}
	  + \frac{37 + 41 \mathrm{i}}{2} \cdot \frac{1}{1 + 2 \mathrm{i} z}
  \end{equation*}
  Podemos expresar el aporte mediante exponenciales complejas%
    \index{C (numeros complejos)@\(\mathbb{C}\) (números complejos)!exponencial}
  (conjugar el exponente corresponde a conjugar):
  \begin{equation*}
    \exp(u) \cdot \exp(n v)
      + \exp(\overline{u}) \cdot \exp(n \overline{v})
  \end{equation*}
  Tenemos:
  \begin{align*}
    \frac{37 + 41 \mathrm{i}}{2}
      &= \exp(u) \\
    2 \mathrm{i}
      &= \exp (v)
  \end{align*}
  Tomando en cuenta que:%
    \index{C (numeros complejos)@\(\mathbb{C}\) (números complejos)!funciones trigonometricas@funciones trigonométricas}
  \begin{align*}
    \cos v
      &= \frac{\mathrm{e}^{\mathrm{i} v}
		  + \mathrm{e}^{ - \mathrm{i} v}}
	      {2} \\
    \sin v
      &= - \mathrm{i} \, \frac{\mathrm{e}^{\mathrm{i} v}
				 - \mathrm{e}^{ - \mathrm{i} v}}
			      {2}
  \end{align*}
  si escribimos \(u = a + b \mathrm{i}\) y \(v = c + d \mathrm{i}\)
  después de simplificar la solución:
  \begin{align*}
    \exp(u) \cdot \exp(n v)
      + \exp(\overline{u}) \cdot \exp(n \overline{v})
      &= \exp(a + b \mathrm{i} + n (c + d \mathrm{i}))
	   + \exp(a - b \mathrm{i} + n (c - d \mathrm{i})) \\
      &= \exp(a + n c) (\exp((b + n d) \mathrm{i})
			  + \exp(- (b + n d) \mathrm{i})) \\
      &= 2 \exp(a + n c) \cos (b + n d)
  \end{align*}
  En detalle,
  para este caso tenemos:
  \begin{align*}
    e^a
      &= \frac{\sqrt{37^2 + 41^2}}{2}
       = \frac{5 \sqrt{122}}{2} \\
    \cos b
      &= \frac{37}{\sqrt{37^2 + 41^2}}
       = \frac{37 \sqrt{122}}{610} \\
    e^c
      &= 2 \\
    \cos d
      &= \frac{0}{\sqrt{0^2 + 2^2}}
       = 0
      \qquad d = \frac{\pi}{2}
  \end{align*}
  y la solución puede expresarse:
  \begin{equation*}
    u_n
      = 5 \sqrt{122} \cdot 2^{n - 1}
	  \cdot \cos \left( b + \frac{n \pi}{2} \right)
	   + n^2
	   - \frac{4 n}{5}
	   - \frac{12}{25}
\end{equation*}

% ricatti.tex
%
% Copyright (c) 2013-2014 Horst H. von Brand
% Derechos reservados. Vea COPYRIGHT para detalles

\section{Recurrencia de Ricatti}
\label{sec:Ricatti}
\index{recurrencia!Ricatti|see{Ricatti, recurrencia de}}
\index{Ricatti, recurrencia de}

  Una recurrencia de la forma:
  \begin{equation}
    \label{eq:Ricatti}
    w_{n + 1}
      = \frac{a w_n + b}{c w_n + d}
  \end{equation}
  donde \(c \ne 0\)
  y \(a d - b c \ne 0\) se llama \emph{recurrencia de Ricatti}
  (si \(c = 0\) es simplemente una recurrencia lineal de primer orden;
   si \(a d = b c\) se reduce a \(w_{n + 1} = \text{constante}\)).
  Incidentalmente,
  la recurrencia~\eqref{eq:Fibonacci-search-recurrence-r}
  para~\(r\)
  que hallamos al analizar la búsqueda de Fibonacci
  en la sección~\ref{sec:busqueda-Fibonacci}%
    \index{Fibonacci, busqueda de@Fibonacci, búsqueda de!analisis@análisis}
  es una recurrencia de Ricatti.

  Para resolver este tipo de recurrencias hay varias opciones,
  que exploraremos en lo que sigue.

\subsection{Vía recurrencia de segundo orden}
\label{sec:Ricatti-2nd}

  Seguimos el esquema de Brand~\cite{brand55:_seq_def_difference_eq}.
  Si en~\eqref{eq:Ricatti} substituimos \(y_n \mapsto c w_n + d\),
  queda una recurrencia de la forma:
  \begin{equation}
    \label{eq:Ricatti-2nd-aux-1}
    y_n
      = \alpha - \frac{\beta}{y_{n - 1}}
  \end{equation}
  donde:
  \begin{align*}
    \alpha
      &= a + d \\
    \beta
      &= a d - b c
  \end{align*}
  Claramente eso solo vale si \(a d - b c \ne 0\).
  Substituyendo ahora:
  \begin{equation}
    \label{eq:Ricatti-2nd-y}
    y_n
      = \frac{x_{n + 1}}{x_n}
  \end{equation}
  resulta:
  \begin{equation}
    \label{eq:Ricatti-2nd-aux}
    x_{n + 2} - \alpha x_{n + 1} + \beta x_n
      = 0
  \end{equation}
  Esta es una recurrencia lineal de segundo orden,
  de coeficientes constantes y homogénea.
  Necesitamos dos valores iniciales para resolverla,
  podemos elegir bastante arbitrariamente \(x_0 = 1\),
  dando \(x_1 = y_0\),
  que a su vez podemos obtener de la condición inicial original.

  Para un ejemplo,
  tomemos \(w_0 = 3\) y:
  \begin{equation}
    \label{eq:Ricatti-ex}
    w_{n + 1}
      = \frac{5 w_n + 2}{3 w_n + 4}
  \end{equation}
  Siguiendo los pasos indicados:
  \begin{align*}
    3 w_{n + 1} + 4
      &= 3 \cdot \frac{5 w_n + 2}{3 w_n + 4} + 4 \\
      &= 9 - \frac{14}{3 w_n + 4}
  \end{align*}
  Substituyendo:
  \begin{equation*}
    3 w_n + 4
      = \frac{x_{n + 1}}{x_n}
  \end{equation*}
  y reordenando resulta:
  \begin{equation}
    \label{eq:Ricatti-ex-2nd}
    x_{n + 2} - 9 x_{n + 1} + 14 x_n
      = 0
  \end{equation}
  Con las condiciones iniciales \(x_0 = 1\), \(x_1 = 3 w_0 + 4 = 13\)
  la solución de~\eqref{eq:Ricatti-ex-2nd} es:
  \begin{equation*}
    x_n
      = \frac{11 \cdot 7^n - 6 \cdot 2^n}{5}
  \end{equation*}
  y finalmente:
  \begin{equation}
    \label{eq:Ricatti-ex-2nd-sol}
    w_n
      = \frac{11 \cdot 7^n + 2^{n + 2}}{11 \cdot 7^n - 3 \cdot 2^{n + 1}}
  \end{equation}

\subsection{Reducción a una recurrencia de primer orden}
\label{sec:Ricatti-1}

  Siguiendo a Mitchell~%
    \cite{mitchell00:_riccati_solution}
  definamos la secuencia auxiliar:
  \begin{equation}
    \label{eq:Ricatti-1st-x}
    x_n
      = \frac{1}{1 + \eta w_n}
  \end{equation}
  Expresamos la recurrencia~\eqref{eq:Ricatti}
  en términos de \(x_n\),
  y despejamos \(x_{n + 1}\):
  \begin{equation*}
    x_{n + 1}
      = \frac{(d \eta - c) x_n + c}
	     {(b \eta^2 - (a - d) \eta - c) x_n + a \eta + c}
  \end{equation*}
  Buscamos que esta recurrencia sea lineal,
  o sea:
  \begin{equation*}
    b \eta^2 - (a - d) \eta - c
      = 0
  \end{equation*}
  Arbitrariamente elegimos el signo positivo:
  \begin{equation}
    \label{eq:Ricatti-1st-aux-eta}
    \eta
      = \frac{a - d + \sqrt{(a - d)^2 + 4 b c}}{2 b}
  \end{equation}
  Esto lleva a la ecuación auxiliar:
  \begin{equation}
    \label{eq:Ricatti-1st-aux}
    x_{n + 1}
      = \frac{(d \eta - c) x_n + c}{a \eta + c}
  \end{equation}
  Esta es simple de resolver.

  Aplicado al mismo ejemplo anterior,
  tenemos:
  \begin{align*}
    \eta
      &= \frac{5 - 4 + \sqrt{(5 - 4)^2 + 4 \cdot 2 \cdot 3}}{2 \cdot 2}
       = \frac{3}{2}
  \end{align*}
  La recurrencia auxiliar es:
  \begin{align}
    x_{n + 1}
      &= \frac{(4 \cdot \frac{3}{2} - 3) x_n + 3}
	      {5 \cdot \frac{3}{2} + 3} \notag \\
      &= \frac{2 x_n + 2}{7}
	       \label{eq:Ricatti-ex-1st-aux}
  \end{align}
  De la condición inicial tenemos:
  \begin{equation}
    \label{eq:Ricatti-ex-1st-initial}
    x_0
      = \frac{1}{1 + \eta w_0}
      = \frac{2}{11}
  \end{equation}
  La tradicional danza para resolver recurrencias entrega:
  \begin{equation}
    \label{eq:Ricatti-ex-1st-aux-sol}
    x_n
      = \frac{2}{5} - \frac{84}{385} \cdot \left( \frac{2}{7} \right)^n
  \end{equation}
  De acá con~\eqref{eq:Ricatti-1st-x} resulta:
  \begin{equation}
    \label{eq:Ricatti-ex-1st-sol}
    w_n
      = \frac{11 \cdot 7^n + 2^{n + 2}}{11 \cdot 7^n - 3 \cdot 2^{n + 1}}
  \end{equation}
  Tal como antes.

\subsection{Transformación de Möbius}
\label{sec:Ricatti-Moebius}

  A una transformación de la forma:
  \begin{equation}
    \label{eq:Moebius-tranform}
    w = \frac{a_{1 1} z + a_{1 2}}{a_{2 1} z + a_{2 2}}
  \end{equation}
  donde \(a_{11} a_{22} - a_{12} a_{21} \ne 0\)
  (de lo contrario,
   la expresión se reduce a una constante)
  se le llama \emph{transformación de Möbius}.%
    \index{Mobius, transformacion de@Möbius, transformación de}%
    \index{recurrencia!Ricatti!transformacion de Möbius@transformación de Möbius}
  Una de sus características interesantes es que forman un grupo
  con la composición de funciones,%
    \index{grupo}
  como es fácil demostrar.
  A nosotros puntualmente nos interesa lo siguiente:
  Sean \(A(z)\), \(B(z)\) transformaciones de Möbius.
  \begin{align}
    A(z)
     &= \frac{a_{1 1} z + a_{1 2}}{a_{2 1} z + a_{2 2}}
	  \label{eq:Moebius-A} \\
    B(z)
     &= \frac{b_{1 1} z + b_{1 2}}{b_{2 1} z + b_{2 2}}
	  \label{eq:Moebius-B}
  \end{align}
  La composición es:
  \begin{equation}
    \label{eq:Moebius-AB}
    A(B(z))
      = \frac{(a_{11} b_{11} + a_{12} b_{21}) z
		 + (a_{11} b_{12} + a_{12} b_{22})}
	     {(a_{21} b_{11} + a_{22} b_{21}) z
		 + (a_{21} b_{12} + a_{22} b_{22})}
  \end{equation}
  Si representamos las transformaciones
  por las respectivas matrices de coeficientes,
  vemos que la composición corresponde al producto de las matrices.
  Lo que hace la recurrencia~\eqref{eq:Ricatti}
  es aplicar la transformación de Möbius repetidas veces:
  \begin{equation}
    \label{eq:Ricatti-Moebius}
    w_n
      = A^n(w_0)
  \end{equation}
  a lo que naturalmente corresponde
  el calcular la potencia de la matriz respectiva.
  Para cálculo numérico puede usarse
  entonces una técnica eficiente para calcular potencias,
  como las dadas en la sección~\ref{sec:potencias}.%
    \index{potencia!calculo eficiente@cálculo eficiente}

  En nuestro caso de matrices de \(2 \times 2\)
  los valores propios son simples de obtener:%
    \index{matriz!valor propio}
  \begin{equation}
    \label{eq:diagonalizable-lambda-eqn}
    (a_{11} - \lambda) (a_{22} - \lambda) - a_{12} a_{21}
      = 0
  \end{equation}
  La fórmula cuadrática da:
  \begin{equation}
    \label{eq:diagonalizable-lambda}
    \lambda
      = \frac{a_{11} + a_{22}
		\pm \sqrt{(a_{11} + a_{22})^2
			     - 4 (a_{11} a_{22} - a_{12} a{21})}}
	     {2}
  \end{equation}
  Las columnas de la matriz \(\boldsymbol{P}\) a su vez
  son los vectores propios correspondientes:
  \begin{equation}
    \label{eq:diagonalizable-p-eqn}
    \begin{pmatrix}
      a_{11} - \lambda_i & a_{12} \\
      a_{21}		 & a_{22} - \lambda_i
    \end{pmatrix}
    \cdot
    \begin{pmatrix}
      p_{1 i} \\
      p_{2 i}
    \end{pmatrix}
      =
      \begin{pmatrix}
	0 \\
	0
      \end{pmatrix}
  \end{equation}
  Por construcción,
  la matriz del sistema~\eqref{eq:diagonalizable-p-eqn}
  es singular,
  y solo da una relación entre \(p_{1 i}\) y \(p_{2 i}\).
  Podemos imponer cualquier condición que resulte cómoda.
  Conociendo \(\boldsymbol{D}\) y \(\boldsymbol{P}\)
  podemos calcular las potencias
  y así obtener la solución buscada.

  Volviendo a nuestro manoseado ejemplo,
  tenemos:
  \begin{equation}
    \label{eq:Ricatti-ex-Moebius-A}
    \boldsymbol{A} =
    \begin{pmatrix}
      5 & 2 \\
      3 & 4
    \end{pmatrix}
  \end{equation}
  Los valores propios son \(\lambda_1 = 7\) y \(\lambda_2 = 2\),
  eligiendo valores \(p_{1 i} = 1\):
  \begin{equation}
    \label{eq:eq:Ricatti-ex-Moebius-P}
    \boldsymbol{D} =
    \begin{pmatrix}
	7  & 0 \\
	0  & 2
    \end{pmatrix} \qquad
    \boldsymbol{P} =
    \begin{pmatrix}
	1  & 1 \\
	1  & - \sfrac{3}{2}
    \end{pmatrix} \qquad
    \boldsymbol{P}^{-1} =
    \begin{pmatrix}
      \sfrac{3}{5} & \sfrac{2}{5} \\
      \sfrac{2}{5} & -\sfrac{2}{5}
    \end{pmatrix}
  \end{equation}
  Podemos entonces calcular:
  \begin{align}
    \boldsymbol{A}^n
     &= \boldsymbol{P}^{-1}
	  \cdot \boldsymbol{D}^n
	  \cdot \boldsymbol{P} \notag \\
     &=
     \begin{pmatrix}
       \displaystyle \frac{3 \cdot 7^n + 2^{n + 1}}{5} &
	 \displaystyle \frac{3 \cdot 7^n - 3 \cdot 2^n}{5} \\
       \\
       \displaystyle \frac{2 \cdot 7^n - 2^{n + 1}}{5} &
	 \displaystyle \frac{2 \cdot 7^n + 3 \cdot 2^n}{5}
     \end{pmatrix}
	\label{eq:Ricatti-ex-Moebius-An}
  \end{align}
  Con esto es:
  \begin{equation}
    \label{eq:Ricatti-ex-Moebius-sol-gral}
    w_n
      = \frac{(3 \cdot 7^n + 2^{n + 1}) w_0
		+ (2 \cdot 7^n - 2^{n + 1})}
	     {(2 \cdot 7^n - 2^{n + 1}) w_0
		+ (2 \cdot 7^n + 3 \cdot 2^n)}
  \end{equation}
  Nótese que esta solución muestra explícitamente la dependencia
  de la condición inicial.
  Con nuestra condición inicial \(w_0 = 3\) resulta nuevamente:
  \begin{equation}
    \label{eq:Ricatti-ex-Moebius-sol}
    w_n
      = \frac{11 \cdot 7^n + 2^{n + 2}}{11 \cdot 7^n - 3 \cdot 2^{n + 1}}
  \end{equation}

%%% Local Variables:
%%% mode: latex
%%% TeX-master: "clases"
%%% End:


%%% Local Variables:
%%% mode: latex
%%% TeX-master: "clases"
%%% End:

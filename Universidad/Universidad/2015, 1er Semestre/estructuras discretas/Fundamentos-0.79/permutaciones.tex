% permutaciones.tex
%
% Copyright (c) 2009-2014 Horst H. von Brand
% Derechos reservados. Vea COPYRIGHT para detalles

\chapter{Permutaciones}
\label{cha:permutaciones}

  Informalmente,
  una permutación es un reordenamiento de un conjunto de objetos.
  Las permutaciones aparecen,
  en forma más o menos prominente,
  en casi todos los ámbitos de las matemáticas.
  Al tratar con conjuntos finitos es común estar interesados
  en las distintas formas de ordenarlos,
  posiblemente simplemente para considerar equivalentes
  los distintos órdenes.
  En el análisis de muchos algoritmos,
  particularmente los de ordenamiento,
  son características de las permutaciones
  lo que determina su rendimiento.

\section{Definiciones básicas}
\label{sec:permutaciones-def}

% Fixme: Revisar el material este en Shoup, GKP, TAoCP, Biggs.
% ¡Citarlos!
% Fixme: Aplicar teoría a análisis de algoritmos de ordenamiento
% simples

  Al reordenar elementos de un conjunto estamos definiendo
  una biyección entre la posición original y la nueva.
  Nos interesa estudiar estas biyecciones.
  \begin{definition}
    \index{permutacion@permutación|textbfhy}
    Una \emph{permutación} de un conjunto finito \(\mathcal{X}\)
    es una biyección de \(\mathcal{X}\) a \(\mathcal{X}\).
  \end{definition}
  Comúnmente usaremos \(\mathcal{X} = \{1, 2, \dotsc, n\}\)
  para concretar.
  Por ejemplo,
  una permutación típica de \(\{1, 2, \dotsc, 5\}\) es
  la función \(\alpha\) definida por la tabla:

  \begin{center}
    \begin{tabular}[c]{>{\(}l<{\)}@{\hspace{0.7em}}
		       *{4}{>{\(}c<{\)}@{\hspace{0.5em}}}>{\(}c<{\)}}
      & 1 & 2 & 3 & 4 & 5 \\
      \alpha & \downarrow & \downarrow & \downarrow & \downarrow
	     & \downarrow \\
      & 2 & 4 & 5 & 1 & 3
    \end{tabular}
  \end{center}
  Para abreviar,
  anotaremos una permutación dando
  el elemento al que va el de la posición indicada,
  o sea en este caso particular:
  \begin{equation*}
    \alpha
      = (2\;4\;5\;1\;3)
  \end{equation*}
  Viene a ser simplemente la última línea en lo anterior.

  Hay \(n!\)~permutaciones
  de un conjunto de \(n\) elementos%
    \index{permutacion@permutación!numero@número}
  (para crear la permutación \(\pi\)
   el valor de \(\pi(1)\) puede elegirse de \(n\) maneras,
   una vez elegido este
   quedan solo \(n - 1\) opciones para \(\pi(2)\),
   y así sucesivamente,
   y finalmente queda solo una opción para \(\pi(n)\)).

  Tomemos la anterior permutación \(\alpha\)
  y la permutación \(\beta = (3\;5\;1\;4\;2)\).
  La composición \(\beta \alpha\) se obtiene de aplicar \(\alpha\),
  luego \(\beta\),
  y resulta:
  \begin{equation*}
    \beta \alpha
      = (5\;4\;2\;3\;1)
  \end{equation*}
  En general,
  la composición no es conmutativa,
  en nuestro caso:
  \begin{equation*}
    \alpha \beta
      = (5\;3\;2\;1\;4)
  \end{equation*}
  y claramente \(\alpha \beta \ne \beta \alpha\).
  Cuidado,
  la convención general
  al operar con permutaciones es que las operaciones
  se efectúan de izquierda a derecha,
  no de derecha a izquierda como correspondería
  dado que la operación es composición de funciones.
  O sea,
  \(\alpha \beta \gamma\)
  se interpreta como \((\alpha \beta) \gamma\).
  En todo caso,
  dado que esto es un grupo,
  la operación es asociativa.%
    \index{operacion@operación!asociativa}

  \begin{theorem}
    \label{theo:permutaciones-grupo}
    \index{grupo}
    \index{permutacion@permutación!grupo}
    Las permutaciones cumplen las siguientes:
    \begin{enumerate}[label = (\roman*), ref = (\roman*)]
    \item
      \label{en:pg-1}
      Si \(\pi\) y \(\sigma\)
      son permutaciones de un conjunto de \(n\) elementos,
      lo es también \(\pi \sigma\).
    \item
      \label{en:pg-2}
      Para todas permutaciones \(\pi\), \(\sigma\), \(\tau\)
      de un conjunto,
      se cumple:
      \begin{equation*}
	(\pi \sigma) \tau = \pi (\sigma \tau)
      \end{equation*}
    \item
      \label{en:pg-3}
      La función identidad,
      que llamaremos \(\iota\),
      definida por \(\iota(r) = r\),
      es una permutación,
      y tenemos para toda permutación \(\sigma\):
      \begin{equation*}
	\iota \sigma = \sigma \iota = \sigma
      \end{equation*}
    \item
      \label{en:pg-4}
      Para toda permutación \(\pi\) de un conjunto dado
      hay una permutación inversa
      \(\pi^{-1}\) tal que:
      \begin{equation*}
	\pi \pi^{-1} = \pi^{-1} \pi = \iota
      \end{equation*}
    \end{enumerate}
  \end{theorem}
  \begin{proof}
    La propiedad~\ref{en:pg-1}
    sigue directamente de que la composición de biyecciones
    es una biyección,~%
    \ref{en:pg-2}
    es una propiedad básica de la composición de funciones,~%
    \ref{en:pg-3} es obvio,
    y~\ref{en:pg-4}
    es simplemente que toda biyección tiene una inversa.
  \end{proof}

  El teorema~\ref{theo:permutaciones-grupo}
  equivale a decir que las permutaciones forman un grupo
  con la operación de composición.
  Al grupo de permutaciones de \(n\) elementos
  se le llama el \emph{grupo simétrico de orden \(n\)},%
    \index{grupo!simetrico@simétrico|textbfhy}
  que se anota \(\mathtt{S}_n\).
  Note eso sí que el orden de \(\mathtt{S}_n\) es \(n!\),%
    \index{grupo!orden}
  a diferencia de lo que parece indicar su nombre.

  Es conveniente una notación compacta
  para permutaciones individuales.
  Consideremos nuevamente la permutación \(\alpha\).
  Es \(\alpha(1) = 2\), \(\alpha(2) = 4\), \(\alpha(4) = 1\).
  Esto forma un \emph{ciclo}
  (\(1 \rightarrow 2 \rightarrow 4 \rightarrow 1\))
  de largo \(3\).
  Similarmente,
  \(3\) y \(5\) forman un ciclo de largo \(2\),
  y podemos escribir \(\alpha\) en \emph{notación de ciclos}
  como \(\alpha = (1\;2\;4) (3\;5)\).

  Toda permutación \(\pi\) se puede escribir
  en notación de ciclos mediante el siguiente algoritmo:%
    \index{permutacion@permutación!notacion ciclo@notación ciclo|textbfhy}
  \begin{itemize}
  \item
    Comience con algún símbolo
    (digamos \(1\)),
    y trace el efecto de \(\pi\) sobre él y sus sucesores
    hasta que nuevamente encontremos \(1\),
    así tenemos un ciclo.
  \item
    Tome un símbolo cualquiera que no haya sido considerado aún,
    y construya el ciclo que lo contiene de la misma forma.
  \item
    Continúe de la misma forma
    hasta haber dado cuenta de todos los símbolos.
  \end{itemize}
  Podemos elegir cualquiera de los elementos
  de cada ciclo como primero --
  por ejemplo,
  \((7\;8\;2\;1\;3)\) es lo mismo que \((1\;3\;7\;8\;2)\).
  Por otro lado,
  podemos reordenar los ciclos --
  por ejemplo,
  \((1\;2\;4)(3\;5)\) y \((3\;5) (1\;2\;4)\)
  corresponden a la misma permutación,
  ya que al operar los ciclos en esta representación
  sobre elementos diferentes los ciclos conmutan.
  Lo importante son el número de ciclos,
  sus largos,
  y la disposición de sus elementos.
  Se adopta la convención
  de comenzar cada ciclo con su mínimo elemento,
  y luego ordenar los ciclos en orden de elemento mínimo.

  Para ver que la representación es única,
  debemos mostrar una biyección entre permutaciones y ciclos.
  Si anotamos los ciclos siempre con el máximo elemento al comienzo,
  y listamos los ciclos en orden de máximo elemento creciente,
  tomamos una permutación cualquiera
  podemos interpretarla como escrita en notación de ciclos:
  Los comienzos de cada ciclo son los máximos hasta ese punto.
  Esto constituye una biyección,%
    \index{biyeccion@biyección}
  y toda permutación puede escribirse en ciclos de una única forma.

  Por ejemplo,
  para la permutación \(\beta\) dada anteriormente,
  la notación de ciclos es \(\beta = (1\;3) (2\;5) (4)\).
  Acá 4 forma un ciclo ``degenerado'' por sí mismo,
  ya que \(\beta(4) = 4\).
  Simplemente omitiremos estos ciclos de largo \(1\),
  ya que corresponden a símbolos
  que no son afectados por la permutación.
  En todo caso,
  es mejor no omitirlos
  hasta que uno se haya familiarizado con la notación.
  Puede considerarse la notación de ciclos
  como el producto de las permutaciones con esos ciclos
  (los demás elementos permanecen fijos).
  O sea, en nuestro caso:
  \begin{equation*}
    \beta
      = (1\;3) (2) (4) (5) \cdot (1) (2\;5) (3) (4)
  \end{equation*}
  Esto es consistente
  con la convención de omitir ciclos de largo uno.

  \begin{example}
    Se tienen cartas numeradas \(1\) a \(12\),
    que se reparten en filas
    como muestra la figura~\ref{subfig:cartas-1}.
    \begin{figure}[htbp]
      \centering
      \subfloat[Original]{
	\begin{tabular}[c]{*{3}{>{\(}r<{\)}}}
	   1 &	2 &  3 \\
	   4 &	5 &  6 \\
	   7 &	8 &  9 \\
	  10 & 11 & 12
	\end{tabular}
	\label{subfig:cartas-1}
      }%
      \qquad%
      \subfloat[Columnas]{
	\begin{tabular}[c]{*{3}{>{\(}r<{\)}}}
	   1 &	4 &  7 \\
	  10 &	2 &  5 \\
	   8 & 11 &  3 \\
	   6 &	9 & 12
	\end{tabular}
	\label{subfig:cartas-2}
      }
      \caption{Ordenamiento de cartas}
      \label{fig:cartas}
    \end{figure}
    Luego se toman las cartas por filas
    y se distribuyen en columnas,
    quedando como lo muestra la figura~\ref{subfig:cartas-2}.
    ¿Cuántas veces hay que repetir esta operación
    hasta obtener nuevamente el orden original?

    Sea \(\pi\) la permutación que corresponde a esta operación,
    lo que buscamos es el orden de \(\pi\) en \(\mathtt{S}_{12}\).
    Esta manera de verlo
    nos dice que el problema tiene solución,
    cosa que no es obvia:
    Perfectamente podría ocurrir
    que saliendo de la configuración inicial
    ya no haya manera de volver a ella.
    En notación de ciclos
    tenemos \(\pi = (1) (2\;4\;10\;6\;5) (3\;7\;8\;11\;9) (12)\).
    Las cartas \(1\) y \(12\) no cambian de posición,
    las demás forman dos ciclos de largo \(5\).
    Al repetir esta operación \(5\) veces
    todas las cartas quedan en sus posiciones originales.
  \end{example}

  \begin{example}
    En la Prisión de Dunwich hay \(100\)~prisioneros.
    En su aburrimiento al alcaide se le ocurre un jueguito:
    Toma \(100\)~cajas,
    las rotula con los números de los prisioneros
    y en cada una coloca un número de prisionero al azar.
    Luego ofrece a los prisioneros lo siguiente:
    Cada uno puede elegir \(50\)~de las cajas,
    si alguno no encuentra su número
    los llevará a todos a un paseo a las colinas de Vermont.

    Si se considera que cada prisionero abre \(50\)~cajas al azar,
    es claro que hallará su número la mitad de las veces,
    y por tanto la probabilidad que todos encuentren los suyos
    es \(2^{-100}\),
    verdaderamente microscópico.

    Sin embargo,
    hay un algoritmo que asegura una razonable probabilidad
    de evitar una suerte horrorosa:
    Cada cual abre la caja con su número,
    luego la caja con el número que halló en la primera,
    y así sucesivamente hasta agotar las~\(50\) o hallar el suyo.
    Lo que está haciendo cada prisionero
    es trazar un ciclo de la permutación,
    interesa entonces saber
    cuántas permutaciones de \(100\)~elementos
    tienen ciclos de largo mayor a~\(50\).
    Como son \(100\)~prisioneros,
    una permutación
    puede tener a lo más un ciclo de los largos de interés,
    por lo que contar el número de permutaciones con estos ciclos
    es simplemente contar el número total de tales ciclos
    en permutaciones de \(100\) elementos.
    Si consideramos un ciclo de \(r\) elementos,
    estos podemos elegirlos
    de \(\binom{100}{r}\)~maneras entre los 100,
    y podemos ordenarlos de \((r - 1)!\)~formas en un ciclo.
    Los \(100 - r\)~elementos restantes
    se pueden organizar de \((100 - r)!\)~maneras,
    con lo que el número de permutaciones
    con un ciclo de largo \(r\) son:
    \begin{equation*}
      \binom{100}{r} (r - 1)! (100 - r)!
	= \frac{100!}{r}
    \end{equation*}
    Vale decir,
    exactamente \(1\) en \(r\)
    permutaciones tienen un ciclo de largo \(r\)
    cuando \(r > 50\).
    Nos interesa entonces la proporción:
    \begin{equation*}
      \sum_{51 \le r \le 100} \frac{1}{r}
	= H_{100} - H_{50}
	\approx 0,6882
    \end{equation*}
    Tienen un poco más de \(31\)\% de probabilidades
    de salvarse.
  \end{example}

  Si un ciclo se compone consigo mismo,
  el primer elemento del ciclo termina en el tercer lugar,
  y así sucesivamente.
  Para que el primer elemento de un ciclo de largo \(k\)
  vuelva a su posición original
  debe elevarse a la potencia \(k\).
  Para que todos los elementos de una permutación
  de tipo \([1^{\alpha_1} 2^{\alpha_2} \dotso n^{\alpha_n}]\)
  vuelvan por primera vez a sus posiciones originales
  (lo que determina el orden de la permutación)%
    \index{permutacion@permutación!orden}
  deben volver a sus posiciones originales
  todos los elementos de todos los ciclos.
  Esto es el mínimo común múltiplo de los largos de los ciclos.
  No importa cuántos ciclos de cada largo hay,
  solo si de los largos respectivos hay o no ciclos.

  \begin{definition}
    \index{permutacion@permutación!involucion@involución}
    \index{involucion@involución|see{permutación!involución}}
    Una permutación \(\tau\)
    que es su propio inverso se llama \emph{involución}.
    \index{involucion@involución|textbfhy}
  \end{definition}

  Las involuciones son aquellas permutaciones que solo tienen ciclos
  de largo \(1\) y \(2\),
  quedan representadas
  por \(\MSet(\Cyc_{\le 2}(\mathcal{Z}))\).%
    \index{metodo simbolico@método simbólico}
  Como la función generatriz exponencial de un ciclo de largo \(r\)
  es simplemente \(z^r / r\),
  la función generatriz exponencial \(\widehat{I}(z)\)
  para el número de involuciones de \(n\) elementos es:
  \begin{equation}
    \label{eq:involution-egf}
    \index{involucion@involución!funcion generatriz@función generatriz}
    \widehat{I}(z)
      = \exp\left(
	      z + z^2 / 2
	    \right)
  \end{equation}
  Prácticamente indoloro.

  Un desarreglo%
    \index{desarreglo}
  (ver sección~\ref{sec:desarreglos}
   y también el capítulo~\ref{cha:pie})
  es simplemente una permutación
  que no tiene ciclos de largo uno.
  La expresión simbólica correspondiente%
    \index{metodo simbolico@método simbólico}
  es \(\MSet(\Cyc_{\ge 2}(\mathcal{Z}))\),
  que podemos expresar
    \(\MSet(\Cyc(\mathcal{Z}) - \Cyc_{= 1}(\mathcal{Z}))\),
  la función generatriz exponencial que resulta es:%
    \index{generatriz!exponencial}
  \begin{equation*}
    \widehat{D}(z)
      = \exp ( -\ln (1 - z) - z )
      = \frac{\mathrm{e}^{-z}}{1 - z}
  \end{equation*}
  De acá nuevamente obtenemos
  el número de desarreglos de \(n\) elementos
  como \(D_n = n! \exp\lvert_n (-1)\).

  En el ejemplo de la prisión de Dunwich
  buscábamos el número de permutaciones
  con ciclos de largo mayor a \(50\).
  Este tipo de problemas puede atacarse
  marcando las subestructuras de interés.%
    \index{metodo simbolico@método simbólico}
  Usamos la clase \(\mathcal{U}\) para marcar,
  con un único elemento \(\upsilon\) de tamaño uno,
  y usamos la letra \(u\) en funciones generatrices.
  Si queremos saber cuántos ciclos de largo \(r\) hay en total
  en permutaciones de \(n\) elementos,
  consideraremos:
  \begin{equation*}
    \MSet(\Cyc_{\ne r}(\mathcal{Z})
	      + \mathcal{U} \times \Cyc_{= r}(\mathcal{Z}))
  \end{equation*}
  Abusando de la notación,
  escribimos:
  \begin{equation*}
    \MSet(\Cyc_{\ne r}(\mathcal{Z})
	      + u \Cyc_{= r}(\mathcal{Z}))
      = \MSet(\Cyc(\mathcal{Z}) + (u - 1) \Cyc_{= r}(\mathcal{Z}))
  \end{equation*}
  Tenemos directamente:
  \begin{equation*}
    C_r(z, u)
      = \exp \left( - \ln (1 - z) + (u - 1) \, \frac{z^r}{r}\right)
      = \frac{\mathrm{e}^{(u - 1) z^r / r}}{1 - z}
  \end{equation*}
  El número de ciclos de largo \(r\) es el exponente de \(u\),
  que podemos extraer derivando y haciendo \(u = 1\),
  lo que da la función generatriz exponencial
  para el número total de ciclos de largo \(r\)
  en permutaciones de \(n\) elementos:
  \begin{equation*}
    C_r(z)
      = \left.
	  \frac{\partial}{\partial u} \, C_r(z, u)
	\right\rvert_{u = 1}
      = \frac{z^r}{r} \, \frac{1}{1 - z}
  \end{equation*}
  De acá podemos obtener el número promedio de ciclos de largo \(r\)
  en las permutaciones de \(n\)~elementos,
  al haber \(n!\)~permutaciones
  y ser una función generatriz exponencial
  es directamente el coeficiente de \(z^n\):
  \begin{equation*}
    [ z^n ] \frac{z^r}{r} \, \frac{1}{1 - z}
      = \frac{1}{r}
  \end{equation*}
  Esto ya lo habíamos obtenido en el ejemplo,
  y al discutir desarreglos en el capítulo~\ref{cha:pie}%
    \index{desarreglo}
  vimos que para \(r = 1\) es \(1\),
  pero esta técnica es mucho más general.
% Fixme: ¿Análisis p.ej.de bubblesort, selección?

  A cada permutación \(\pi\) en \(\mathtt{S}_n\)
  corresponde una partición de los \(n\) elementos
  en los elementos de sus ciclos.
  Al número de ciclos de cada largo de la permutación
  le llamaremos su \emph{tipo}.%
    \index{permutacion@permutación!tipo|textbfhy}
  O sea,
  si \(\pi\) tiene \(\alpha_i\) ciclos de largo \(i\)
  (\(1 \le i \le n\)),
  el tipo de \(\pi\) lo anotamos
  \([1^{\alpha_1} 2^{\alpha_2} \dotso n^{\alpha_n}]\).
  Generalmente omitiremos los factores con \(\alpha_i = 0\)
  (largos para los cuales no hay ciclos).

  \begin{definition}
    \label{def:permutacion-conjugada}
    \index{permutacion@permutación!conjugada|textbfhy}
    Si hay una permutación \(\sigma\)
    tal que \(\sigma \alpha \sigma^{-1} = \beta\)
    se dice que \(\beta\) es \emph{conjugada} de \(\alpha\).
  \end{definition}
  El teorema siguiente es básico
  en la teoría algebraica de permutaciones.
  \begin{theorem}
    Dos permutaciones \(\alpha\) y \(\beta\) son conjugadas
    si y solo si tienen el mismo tipo.
  \end{theorem}
  \begin{proof}
    Demostramos implicancias en ambas direcciones.

    Si \(\alpha\) y \(\beta\) son permutaciones conjugadas,
    hay \(\sigma\)
    tal que podemos escribir \(\sigma \alpha \sigma^{-1} = \beta\).
    Tomemos un ciclo \((x_1, x_2, \dotsc, x_r)\) de \(\alpha\),
    vale decir,
    \(\alpha(x_1) = x_2\),
    \(\alpha(x_2) = x_3\), \ldots,
    \(\alpha(x_{r - 1}) = x_r\), \(\alpha(x_r) = x_1\).
    Definamos \(y_i = \sigma(x_i)\) para \(1 \le i \le r\).
    Tenemos,
    tomando los índices módulo \(r\):
    \begin{equation*}
      \beta(y_i)
	= \sigma \alpha \sigma^{-1}(\sigma(x_i))
	= \sigma \alpha (x_i)
	= \sigma(x_{i + 1})
	= y_{i + 1}
    \end{equation*}
    Esto es una biyección entre el ciclo de \(\alpha\)
    que contiene a \(x_1\)
    y el ciclo de \(\beta\) que contiene a \(y_1\),
    y habrán biyecciones similares
    entre los demás ciclos de \(\alpha\) y \(\beta\),
    que así tienen el mismo tipo.

    Por otro lado,
    supongamos que \(\alpha\) y \(\beta\) tienen el mismo tipo.
    Como tienen el mismo número de ciclos de cada uno de los largos,
    podemos construir una biyección
    entre ciclos del mismo largo de \(\alpha\) y \(\beta\),
    digamos \((x_1, x_2, \dotsc, x_r)\) en \(\alpha\)
    corresponde a \((y_1, y_2, \dotsc, y_r)\) en \(\beta\).
    Definiendo \(\sigma(x_i) = y_i\) para este ciclo,
    y de forma similar para los demás ciclos,
    tenemos \(\sigma \alpha \sigma^{-1} = \beta\) ya que
    (considerando los índices módulo el largo del ciclo)
    tenemos:
    \begin{equation*}
      \sigma \alpha \sigma^{-1}(y_i)
	= \sigma \alpha (x_i)
	= \sigma (x_{i + 1})
	= y_{i + 1}
	= \beta(y_i)
      \qedhere
    \end{equation*}
  \end{proof}

  Una manera fundamental de clasificar permutaciones%
    \index{permutacion@permutación!clasificacion@clasificación}
  viene de considerarlas como composición de \emph{transposiciones},
  que son permutaciones que solo intercambian dos elementos.%
    \index{permutacion@permutación!transposicion@transposición|textbfhy}
    \index{transposicion@transposición|see{permutación!transposición}}%
    \glossary{Transposición}
	     {Una permutación que solo intercambia dos elementos}

  Una permutación es una transposición
  si es del tipo \([1^{n - 2} 2^1]\)
  (tiene \(n - 2\) ciclos de largo \(1\) y un ciclo de largo \(2\)).
  Ahora bien,
  el ciclo \((x_1 \, x_2 \dotso x_r)\),
  transforma \((x_1 \, x_2 \dotso x_r)\)
  en \((x_2 \, x_3 \dotso x_r \, x_1)\),
  y este mismo efecto se logra intercambiando \(x_1\) con \(x_2\),
  luego \(x_2\)
  (que ahora está en la posición \(x_1\))
  con \(x_3\),
  y así sucesivamente.
  Así podemos escribir
  \((x_1 \, x_2 \dotso x_r)
      = (x_1 \, x_r) \dotso (x_1 \, x_3) (x_1 \, x_2)\).
  Como toda permutación se puede descomponer en ciclos,
  toda permutación puede expresarse en términos de transposiciones.
  Por ejemplo,
  aplicando la idea indicada arriba a cada uno de los ciclos:
  \begin{equation*}
    (1\;3\;6)(2\;4\;5\;7) = (1\;6) (1\;3) (2\;7) (2\;5) (2\;4)
  \end{equation*}
  Las transposiciones se pueden traslapar,
  un elemento puede moverse más de una vez.
  Esta representación no es única,
  además de reordenar las transposiciones
  podemos usar un conjunto completamente diferente de estas,
  como:
  \begin{equation*}
    (1\;3\;6)(2\;4\;5\;7)
      = (1\;5) (3\;5) (3\;6) (5\;7) (1\;4) (2\;7) (1\;2)
  \end{equation*}
  Sin embargo,
  hay una característica común entre estas descomposiciones.
  Sea \(c(\pi)\) al número total de ciclos de \(\pi\).
  Si \(\pi\)
  es de tipo \([1^{\alpha_1} 2^{\alpha_2} \dotso n^{\alpha_n}]\),
  entonces \(c(\pi) = \alpha_1 + \alpha_2 + \dotsb + \alpha_n\).
  Si combinamos \(\pi\) con una transposición \(\tau\),
  dando \(\tau \pi\),
  interesa determinar
  la relación entre \(c(\pi)\) y \(c(\tau \pi)\).
  Si \(\tau\) intercambia \(a\) con  \(b\),
  tenemos \(\tau(a) = b\), \(\tau(b) = a\)
  y \(\tau(k) = k\) si \(k \ne a, b\).
  Cuando \(a\) y \(b\) están en el mismo ciclo de \(\pi\),
  podemos escribir:
  \begin{equation*}
    \pi = (a \, x \dotso y \, b \dotso z) \text{\ y otros ciclos}
  \end{equation*}
  Veamos \(\tau \pi\).
  Como \(\tau\) intercambia \(a\) con \(b\),
  los ciclos de \(\pi\) que no los incluyen no cambian.
  Sean \(\pi(a) = x\), \(\pi(y) = b\), \(\pi(z) = a\);
  con lo que
  \(\tau \pi(a) = \tau(x) = x\) y \(\tau \pi(y) = \tau(b) = a\).
  De la misma forma \(\tau \pi(b) = \pi(b)\), \ldots,
		\(\tau \pi(z) = \tau(a) = b\).
  Se corta el ciclo y resulta:
  \begin{equation*}
    \tau \pi
      = (a \, x \dotso y) (b \dotso z) \text{\ y los otros ciclos}
  \end{equation*}
  con lo que \(c(\tau \pi) = c(\pi) + 1\).
  Por otro lado,
  si \(a\) y \(b\) pertenecen a ciclos distintos,
  o sea podemos escribir:
  \begin{equation*}
    \pi
      = (a \, x \dotso y) (b \dotso z) \text{\ y otros ciclos}
  \end{equation*}
  vemos que
    \(\tau \pi(y) = \tau(a) = b\) y \(\tau \pi(z) = \tau(b) = a\),
  los ciclos se funden:
  \begin{equation*}
    \tau \pi
      = (a \, x \dotso y \, b \dotso z) \text{\ y los otros ciclos}
  \end{equation*}
  con lo que \(c(\tau \pi) = c(\pi) - 1\).
  En resumen,
  seguir una permutación por una transposición
  cambia el número de ciclos en \(1\),
  y tenemos:
  \begin{theorem}
    \label{theo:permutations-transposiciones}
    Supóngase que la permutación \(\pi\) de \(\mathtt{S}_n\)
    puede escribirse como la composición de \(r\) transposiciones
    y también como la composición de \(r'\) transposiciones.
    Entonces \(r\) y \(r'\) son ambos pares o ambos impares.
  \end{theorem}
  \begin{proof}
    Sea \(\pi = \tau_r \tau_{r - 1} \dotso \tau_1\),
    con \(\tau_i\) (\(1 \le i \le r\)) una transposición.
    Como \(\tau_1\) es un ciclo de largo \(2\)
    y \(n - 2\) ciclos de largo \(1\),
    es \(c(\tau_1) = 1 + (n - 2) = n - 1\).
    Combinando \(\tau_1\)
    con \(\tau_2\), \(\tau_3\), \ldots, \(\tau_r\)
    finalmente se obtiene \(\pi\).
    En cada paso \(c\) aumenta o disminuye en \(1\).
    Sea \(g\) el número de veces que aumenta
    y \(h\) el número de veces que disminuye,
    El número final de ciclos será \(c(\pi) = (n - 1) + g - h\).
    Pero tenemos \(g + h = r - 1\),
    con lo que:
    \begin{align*}
      r &= 1 + g + h \\
	&= 1 + g + (n - 1 + g - c(\pi)) \\
	&= n - c(\pi) + 2 g
    \end{align*}
    De la misma forma,
    para cualquier otra manera de expresar \(\pi\)
    como producto de \(r'\) transposiciones
    hay un entero \(g'\) tal que \(r' = n - c(\pi) + 2 g'\),
    y \(r - r' = 2 (g - g')\),
    que es par.
  \end{proof}
  En vista del teorema~\ref{theo:permutations-transposiciones}
  podemos clasificar las permutaciones
  en \emph{pares} o \emph{impares}%
    \index{permutacion@permutación!par}%
    \index{permutacion@permutación!impar}
  dependiendo del número de transposiciones que las conforman.
  Definimos el \emph{signo} de la permutación \(\rho\),%
    \index{permutacion@permutación!signo}
  escrito \(\sgn \rho\),
  como	\(+1\) si \(\rho\) es par,
  y \(-1\) si es impar:
  \begin{equation*}
    \sgn \rho = (-1)^r
  \end{equation*}
  donde \(\rho\) es la composición de \(r\) transposiciones.
  En particular,
  \(\sgn \iota = (-1)^0 = +1\).
  Claramente,
  si \(\rho\) se puede descomponer en \(r\) transposiciones
  y \(\sigma\) se puede descomponer en \(s\) transposiciones,
  entonces \(\rho \sigma\)
  se puede descomponer en \(r + s\) transposiciones:
  \begin{align*}
    \sgn (\rho \sigma)
       &= (-1)^{r + s} \\
       &= (-1)^r \cdot (-1)^s \\
       &= \sgn \rho \cdot \sgn \sigma
  \end{align*}
  Como \(\rho \rho^{-1} = \iota\),
  resulta \(\sgn \rho = \sgn \rho^{-1}\).

  Recordando nuestra técnica para escribir
  un ciclo de largo \(k\) como \(k - 1\) transposiciones,
  tenemos un algoritmo simple
  para determinar el signo de una permutación:
  Descompóngala en ciclos,
  la paridad de la permutación
  es simplemente la del número de ciclos de largo par.%
    \index{permutacion@permutación!signo!determinar}

  \begin{theorem}
    Para todo entero \(n \ge 2\)
    exactamente la mitad
    de las permutaciones de \(\mathtt{S}_n\) son pares
    y la otra mitad impares.
  \end{theorem}
  \begin{proof}
    Sea \(\pi_1\), \(\pi_2\), \ldots, \(\pi_k\)
    la lista de permutaciones pares
    en \(\mathtt{S}_n\).
    Esta lista no es vacía,
    ya que \(\iota\) es par.

    Sea \(\tau\) una transposición cualquiera en \(S_n\),
    por ejemplo \(\tau = (1\;2)\).
    Entonces \(\tau \pi_1\), \(\tau \pi_2\), \ldots, \(\tau \pi_k\)
    son todas distintas,
    ya que si \(\tau \pi_i = \tau \pi_j\)
    por asociatividad e inverso en el grupo:
    \begin{equation*}
      \pi_i = (\tau^{-1} \tau) \pi_i
	    = \tau^{-1} (\tau \pi_i)
	    = \tau^{-1} (\tau \pi_j)
	    = (\tau^{-1} \tau) \pi_j
	    = \pi_j
    \end{equation*}
    Aún más,
    todas estas permutaciones son impares,
    ya que:
    \begin{equation*}
      \sgn (\tau \pi_i)
	 = \sgn \tau \cdot \sgn \pi_i
	 = (-1) \cdot (+1)
	 = -1
    \end{equation*}
    Finalmente,
    toda permutación impar \(\rho\)
    es de una de las \(\tau \pi_i\) (\(1 \le i \le k\)),
    ya que:
    \begin{equation*}
      \sgn (\tau^{-1} \rho)
	 = \sgn \tau^{-1} \cdot \sgn \rho
	 = (-1) \cdot (-1)
	 = +1
    \end{equation*}
    con lo que \(\tau^{-1} \rho\) es par,
    y debe ser \(\pi_i\)
    para algún \(i\) en el rango \(1 \le i \le k\),
    y así \(\rho = \tau \pi_i\).
    Con esto tenemos una biyección
    entre las permutaciones pares e impares.
  \end{proof}
  El conjunto de permutaciones pares
  es un subgrupo de \(\mathtt{S}_n\)
  (\(\mathtt{S}_n\) es finito,
   y este subconjunto es cerrado respecto de la composición),
  se le llama el \emph{grupo alternante}
  y se anota \(\mathtt{A}_n\).%
    \index{grupo!alternante|textbfhy}

  Esto tiene aplicación en muchas áreas,
  por ejemplo en ``matemáticas recreativas''.
  \begin{example}
    Ocho bloques marcados \(1\) a \(8\) se disponen
    en un marco cuadrado
    como se indica en la figura~\ref{subfig:bloques-1}.
    \begin{figure}[htbp]
      \centering
      \subfloat[Inicial]{
	\setlength{\tabcolsep}{3pt}
	\begin{tabular}{|*{3}{>{\(}c<{\)}|}}
	  \hline
	    \rule[-0.7ex]{0pt}{3ex}%
	  1 & 2 & 3 \\
	  \hline
	    \rule[-0.7ex]{0pt}{3ex}%
	  4 & 5 & 6 \\
	  \hline
	    \rule[-0.7ex]{0pt}{3ex}%
	  7 & 8 &   \\
	  \hline
	\end{tabular}
	\label{subfig:bloques-1}
      }%
      \qquad%
      \subfloat[Final]{
	\setlength{\tabcolsep}{3pt}
	\begin{tabular}{|*{3}{>{\(}c<{\)}|}}
	  \hline
	    \rule[-0.7ex]{0pt}{3ex}%
	  8 & 5 & 2 \\
	  \hline
	    \rule[-0.7ex]{0pt}{3ex}%
	  7 & 4 & 1 \\
	  \hline
	    \rule[-0.7ex]{0pt}{3ex}%
	  6 & 3 &   \\
	  \hline
	\end{tabular}
	\label{subfig:bloques-2}
      }
      \caption{¿Puede hacerse?}
      \label{fig:bloques}
    \end{figure}
    Una movida legal
    corresponde a deslizar un bloque al espacio vacío.
    Determine si es posible
    lograr la configuración de~\ref{subfig:bloques-2}.

    Si anotamos \(\openbox\) para el espacio,
    la configuración inicial
    es \(1\;2\;3\;4\;5\;6\;7\;8\;\openbox\),
    y la final solicitada es \(8\;5\;2\;7\;4\;1\;6\;3\;\openbox\).
    Claramente toda movida legal es una transposición.
    Como el espacio solo puede moverse en vertical u horizontal,
    para regresar a su posición original
    debe hacer un número par de movidas en cada dirección,
    y el número total de movidas es par.
    Luego la permutación que lleva de la configuración inicial
    a otra configuración posible bajo las reglas
    en la que \(\openbox\)
    está nuevamente en la esquina inferior derecha
    está conformada por un número par de transposiciones,
    o sea es par.
    La permutación entre la configuración final solicitada
    y la inicial es \((1\;8\;3\;2\;5\;4\;7\;6)(\openbox)\);
    el ciclo de largo \(8\) que aparece acá es impar,
    por lo que esto no puede hacerse.

    Nótese que esto sirve
    para demostrar lo que \emph{no} puede hacerse,
    no significa que todas las permutaciones pares puedan lograrse
    con estas restricciones.
  \end{example}

  Es simple contabilizar las permutaciones de un tipo particular.%
    \index{permutacion@permutación!tipo!numero@número}
  Supongamos,
  por ejemplo,
  que nos interesa saber cuántos elementos de \(\mathtt{S}_{14}\)
  tienen tipo \([2^2 3^2 4^1]\).
  Esto corresponde a poner
  los símbolos \(1\), \(2\), \ldots, \(14\) en el patrón:
  \begin{equation*}
    (.\;.) (.\;.) (.\;.\;.) (.\;.\;.) (.\;.\;.\;.)
  \end{equation*}
  y hay \(14!\)~formas
  de distribuir los \(14\) símbolos en las \(14\) posiciones.
  Sin embargo,
  muchas de estas dan la misma permutación.
  Considerando cada ciclo,
  podemos elegir el primer elemento de él
  y el resto quedará determinado por la permutación.
  Así hay \(2\)~maneras de obtener cada \(2\)\nobreakdash-ciclo,
  \(3\)~maneras de obtener cada \(3\)\nobreakdash-ciclo,
  y \(4\)~maneras de obtener cada \(4\)\nobreakdash-ciclo.
  El ordenamiento interno de cada ciclo se puede llevar a cabo
  de \(2^2 \cdot 3^2 \cdot 4\) formas en este caso.
  En general,
  si el tipo es \([1^{\alpha_1} 2^{\alpha_2} \dotso n^{\alpha_n}]\),
  habrán \(1^{\alpha_1} \cdot 2^{\alpha_2}
	     \cdot \dotsm \cdot n^{\alpha_n}\)~%
  ordenamientos internos equivalentes.
  Por otro lado,
  el orden de los ciclos del mismo largo es arbitrario,
  por lo que el número de reordenamientos en el caso general será~%
  \(\alpha_1 ! \, \alpha_2 ! \dotso \alpha_n !\),
  y el número de permutaciones
  de tipo \([1^{\alpha_1} 2^{\alpha_2} \dotso n^{\alpha_n}]\) es:
  \begin{equation*}
    \frac{n!}{1^{\alpha_1} 2^{\alpha_2} \dotso n^{\alpha_n} \,
	      \alpha_1 ! \, \alpha_2 ! \dotso \alpha_n !}
  \end{equation*}
  Esto se ve seductoramente similar a un término multinomial,
  pero debe recordarse que acá la condición es:
  \begin{equation*}
    \sum_k k \alpha_k = n
  \end{equation*}

% Fixme: Paulina Silva <pasilva@alumnos.inf.utfsm.cl>
%	 pide un ejemplo más aplicado

% Fixme: Análisis somero de bubblesort/inserción

%%% Local Variables:
%%% mode: latex
%%% TeX-master: "clases"
%%% End:

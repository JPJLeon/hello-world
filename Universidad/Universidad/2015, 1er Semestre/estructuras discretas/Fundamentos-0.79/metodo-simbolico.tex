% metodo-simbolico.tex
%
% Copyright (c) 2011-2014 Horst H. von Brand
% Derechos reservados. Vea COPYRIGHT para detalles

\chapter{El método simbólico}
\label{cha:metodo-simbolico}
\index{metodo simbolico@método simbólico|textbfhy}

  Vimos antes
  (capítulo~\ref{cha:funciones-generatrices})
  que las operaciones aritméticas entre funciones generatrices
  dan funciones generatrices
  que corresponden a combinaciones
  de los objetos que éstas representan.%
    \index{generatriz!combinatoria}
  Expandiendo esta observación,
  veremos un marco en el cual derivar ecuaciones
  para las funciones generatrices de interés
  en problemas combinatorios
  es casi automático,
  como sistematizado por Flajolet y Sedgewick~%
    \cite{flajolet09:_analy_combin}
  y aplicado a análisis de algoritmos por Sedgewick y Flajolet~%
    \cite{sedgewick13:_introd_anal_algor}.
  Otra visión en la misma línea presenta Wilf~%
    \cite{wilf06:_gfology}.
  Claro está que igual queda la tarea
  de extraer la información buscada
  de la ecuación resultante.
  Más adelante discutiremos herramientas para esta segunda tarea.

\section{Un primer ejemplo}
\label{sec:MS-ejemplo}

  Consideremos primero el derivar una ecuación generatriz
  para el número de árboles binarios con \(n\) nodos,
  definidos diciendo que un árbol binario es una de las siguientes:
  \begin{itemize}
  \item
    Es \emph{vacío}.
  \item
    Consta de un \emph{nodo raíz} y dos subárboles binarios
    (izquierdo y derecho).
  \end{itemize}
  Una manera de modelar esto es usar la variable \(z\)
  para marcar los nodos,
  vía usar \(\beta\) para representar un árbol binario
  y \(\lvert \beta \rvert\) para su número de nodos,
  definir la función generatriz:
  \begin{equation*}
    B(z)
      = \sum_{\beta} z^{\lvert \beta \rvert}
  \end{equation*}
  El coeficiente de \(z^n\)
  en \(B(z)\) es el número que nos interesa.

  Directamente de la definición de árbol binario%
    \index{arbol binario@árbol binario}
  sabemos que hay un árbol binario vacío,
  que al no tener nodos aporta \(1\) a \(B(z)\).
  Los demás se pueden dividir en un nodo raíz
  y dos subárboles binarios,
  vale decir:
  \begin{align*}
    B(z)
      &= 1 + \sum_{\beta_1, \beta_2}
	       z^{1 + \lvert \beta_1 \rvert
		    + \lvert \beta_2 \rvert} \\
      &= 1 + z B^2 (z)
  \end{align*}
  Hay una íntima relación entre la definición recursiva
  y nuestra ecuación para la función generatriz.
  Lo que buscamos es sistematizar y extender esta observación.

  Estamos interesados
  en colecciones de objetos.
  Formalmente:
  \begin{definition}
    \index{metodo simbolico@método simbólico!clase|textbfhy}
    Una \emph{clase} \(\mathcal{A}\)
    es un conjunto numerable
    de \emph{objetos} \(\alpha \in \mathcal{A}\).
    A cada objeto \(\alpha\) se le asocia un \emph{tamaño},
    \(\lvert \alpha \rvert \in \mathbb{N}_0\).
    El conjunto de objetos con un tamaño dado es finito.
  \end{definition}
  Usaremos consistentemente letra caligráfica,
  como \(\mathcal{A}\),
  para clases,
  y la misma letra para identificar nociones relacionadas.
  Así,
  para la clase \(\mathcal{A}\)
  generalmente usaremos \(\alpha\) para un elemento de la clase
  y llamaremos \(a_n\)
  al número de objetos de la clase de tamaño \(n\).
  Usaremos \(\mathcal{A}_n\)
  para referirnos al conjunto de objetos de la clase \(\mathcal{A}\)
  de tamaño \(n\),
  con lo que \(a_n = \lvert \mathcal{A}_n \rvert\).
  A las funciones generatrices ordinaria y exponencial
  correspondientes
  les llamaremos \(A(z)\) y \(\widehat{A}(z)\),
  respectivamente:
  \begin{align*}
    A(z)
      &= \sum_{\alpha \in \mathcal{A}} z^{\lvert \alpha \rvert}
       = \sum_{n \ge 0} a_n z^n \\
    \widehat{A}(z)
      &= \sum_{\alpha \in \mathcal{A}}
	   \frac{z^{\lvert \alpha \rvert}}{\lvert \alpha \rvert !}
       = \sum_{n \ge 0} a_n \, \frac{z^n}{n!}
  \end{align*}

  Nuestro siguiente objetivo es construir nuevas clases
  a partir de las que ya tenemos.
  Debe tenerse presente que como lo que nos interesa
  es contar el número de objetos de cada tamaño,
  basta construir objetos con distribución de tamaños adecuada
  (o sea,
   relacionados con lo que deseamos contar por una biyección).
  Comúnmente el tamaño de los objetos es el número de alguna clase de átomos
  que lo componen.

  Las clases más elementales son \(\varnothing\),
  la clase que no contiene objetos;
  \(\mathcal{E} = \{\epsilon\}\),
  la clase que contiene únicamente el objeto vacío \(\epsilon\)
  (de tamaño nulo);
  y la clase que comúnmente llamaremos \(\mathcal{Z}\),
  conteniendo un único objeto de tamaño uno
  (que llamaremos \(\zeta\) por consistencia).
  Luego definimos operaciones que combinan
  las clases \(\mathcal{A}\) y \(\mathcal{B}\)
  mediante \emph{unión combinatoria} \(\mathcal{A} + \mathcal{B}\),
  en que aparecen los \(\alpha\) y los \(\beta\) con sus tamaños
  (los objetos individuales se ``decoran'' con su proveniencia,
   de forma que \(\mathcal{A}\) y  \(\mathcal{B}\)
   no necesitan ser disjuntos;
   pero generalmente nos preocuparemos
   que \(\mathcal{A}\) y \(\mathcal{B}\)
   sean disjuntos,
   o podemos usar el principio de inclusión y exclusión
   para contar los conjuntos de interés).
  Ocasionalmente restaremos objetos de una clase,
  lo que debe interpretarse sin decoraciones
  (estamos dejando fuera ciertos elementos,
   simplemente).
  Usaremos \emph{producto cartesiano}
  \(\mathcal{A} \times \mathcal{B}\),%
    \index{producto cartesiano}
  cuyos elementos son pares \((\alpha, \beta)\)
  y el tamaño del par
  es \(\lvert \alpha \rvert + \lvert \beta \rvert\).
  Otras operaciones son formar \emph{secuencias}%
    \index{secuencia}
  de elementos de \(\mathcal{A}\)
  (se anota \(\Seq(\mathcal{A})\)),
  formar \emph{conjuntos}%
    \index{conjunto}
  \(\Set(\mathcal{A})\)
  y \emph{multiconjuntos}
    \index{multiconjunto}
  \(\MSet(\mathcal{A})\)
  de elementos de \(\mathcal{A}\).
  Consideramos también la operación \(\Cyc(\mathcal{A})\),
  que consiste en ordenar elementos de \(\mathcal{A}\) en un círculo
  (una secuencia conectando inicio y fin).%
    \index{ciclo}
  Usaremos también la operación de \emph{composición},
  que anotaremos \(\mathcal{A} \circ \mathbf{B}\),
  definida mediante para cada objeto \(\alpha \in \mathcal{A}\)
  construir un nuevo objeto substituyendo
  \(\lvert \alpha \rvert\) elementos de \(\mathcal{B}\) por sus átomos.
  Otra operación útil es \emph{marcar} uno de los átomos de cada objeto,
  cosa que anotaremos \(\mathcal{A}^\bullet\).
  El tamaño de un objeto compuesto
  es simplemente la suma de los tamaños de los componentes.
  De incluir objetos de tamaño cero
  en estas construcciones pueden crearse infinitos objetos de un tamaño dado,
  lo que no es una clase según nuestra definición.
  Por ello estas construcciones son aplicables
  sólo si \(\mathcal{A}_0 = \varnothing\).

  Otro juego popular de notaciones para estas operaciones es
  \(\mathfrak{S}(\mathcal{A})\) para secuencia,
  \(\mathfrak{P}(\mathcal{A})\) para conjunto
  (de \emph{\foreignlanguage{english}{powerset}}),
  \(\mathfrak{M}(\mathcal{A})\) para multiconjunto
  \(\mathfrak{C}(\mathcal{A})\) para ciclo,
  y \(\theta \mathcal{A}\) para marcar un átomo.

  Es importante recalcar las relaciones y diferencias
  entre las estructuras.
  En una secuencia es central
  el orden de las piezas que la componen.
  Ejemplo son las palabras,
  interesa el orden exacto de las letras
  (y estas pueden repetirse).
  En un conjunto solo interesa si el elemento está presente o no,
  no hay orden.
  En un conjunto un elemento en particular está o no presente,
  a un multiconjunto puede pertenecer varias veces.

  En lo que sigue haremos distinción
  entre objetos rotulados y sin rotular.
  Para algunos ejemplos de la distinción véase la sección~%
    \ref{sec:combinatorial-applications}.%
    \index{combinatoria!objetos rotulados}
  Generalmente los rótulos se refieren a algún orden externo
  u otra marca que distingue a los elementos.
  Si un objeto se considera creado de átomos idénticos
  (intercambiables)
  corresponde considerarlos no rotulados;
  si un objeto está compuesto de átomos diferenciables
  podemos considerarlos rotulados secuencialmente,
  y estamos frente a objetos rotulados.
  Un punto que produce particular confusión
  es que tiene perfecto sentido
  hablar de secuencias de elementos sin rotular.
  La secuencia impone un orden,
  pero elementos iguales se consideran indistinguibles
  (en una palabra interesa el orden de las letras,
   pero al intercambiar dos letras iguales
   la palabra sigue siendo la misma).
  Recuerde la discusión de la sección~\ref{sec:tao-bookkeeper}.

\section{Objetos sin rotular}
\label{sec:sin-rotular}
\index{metodo simbolico@método simbólico!objetos no rotulados}

  Nuestro primer teorema
  relaciona las funciones generatrices ordinarias
  respectivas para algunas de las operaciones entre clases
  definidas antes.
  Las funciones generatrices de las clases \(\varnothing\),
  \(\mathcal{E}\) y \(\mathcal{Z}\)
  son,
  respectivamente,
  \(0\), \(1\) y \(z\).
  En las derivaciones
  de las transferencias de ecuaciones simbólicas
  a ecuaciones para las funciones generatrices
  lo que nos interesa es contar los objetos entre manos,
  recurriremos a biyecciones para ello en algunos de los casos.
  \begin{theorem}[Método simbólico, OGF]
    \index{metodo simbolico@método simbólico!teorema de transferencia!objetos no rotulados|textbfhy}
    \label{theo:ms-OGF}
    Sean \(\mathcal{A}\) y \(\mathcal{B}\) clases de objetos,
    con funciones generatrices ordinarias
    respectivamente \(A(z)\) y \(B(z)\).
    Entonces tenemos
    las siguientes funciones generatrices ordinarias:
    \begin{enumerate}
    \item
      Para enumerar \(\mathcal{A}^\bullet\):
      \begin{equation*}
	z \mathrm{D} A(z)
      \end{equation*}
    \item
      Para enumerar \(\mathcal{A} + \mathcal{B}\):
      \begin{equation*}
	A(z) + B(z)
      \end{equation*}
    \item
      Para enumerar \(\mathcal{A} \times \mathcal{B}\):
      \begin{equation*}
	A(z) \cdot B(z)
      \end{equation*}
    \item
      Para enumerar \(\Seq(\mathcal{A})\):
      \begin{equation*}
	\frac{1}{1 - A(z)}
      \end{equation*}
    \item
      Para enumerar \(\mathcal{A} \circ \mathcal{B}\):
      \begin{equation*}
	A(B(z))
      \end{equation*}
    \item
      Para enumerar \(\Set(\mathcal{A})\):
      \begin{equation*}
	\prod_{\alpha \in \mathcal{A}}
	   \left( 1 + z^{\lvert \alpha \rvert} \right)
	  = \prod_{n \ge 1} (1 + z^n)^{a_n}
	  = \exp \left(
		   \sum_{k \ge 1} \frac{(-1)^{k + 1}}{k} \, A(z^k)
		 \right)
      \end{equation*}
    \item
      Para enumerar \(\MSet(\mathcal{A})\):
      \begin{equation*}
	\prod_{\alpha \in \mathcal{A}}
	   \frac{1}{1 - z^{\lvert \alpha \rvert}}
	  = \prod_{n \ge 1} \frac{1}{(1 - z^n)^{a_n}}
	  = \exp\left(
		   \sum_{k \ge 1} \frac{A(z^k)}{k}
		\right)
      \end{equation*}
    \item
      Para enumerar \(\Cyc(\mathcal{A})\):
      \begin{equation*}
	\sum_{n \ge 1} \frac{\phi(n)}{n} \, \ln \frac{1}{1 - A(z^n)}
      \end{equation*}
    \end{enumerate}
  \end{theorem}
  \begin{proof}
    Usamos libremente resultados sobre funciones generatrices,
    capítulo~\ref{cha:funciones-generatrices}%
      \index{generatriz},
    en las demostraciones de cada caso.
    Usaremos casos ya demostrados en las demostraciones sucesivas.
    \begin{enumerate}
    \item % mark A
      El objeto \(\alpha \in \mathcal{A}\)
      da lugar a \(\lvert \alpha \rvert\) objetos
      al marcar cada uno de sus átomos,
      lo que da la función generatriz:
      \begin{equation*}
	\sum_{\alpha \in \mathcal{A}}
	  \lvert \alpha \rvert z^{\lvert \alpha \rvert}
      \end{equation*}
      Esto es lo indicado.
    \item % A + B
      Si hay \(a_n\) elementos de \(\mathcal{A}\) de tamaño \(n\)
      y \(b_n\) elementos de \(\mathcal{B}\) de tamaño \(n\),
      habrán \(a_n + b_n\) elementos
      de \(\mathcal{A} + \mathcal{B}\)
      de tamaño \(n\).

      Alternativamente,
      usando la notación de Iverson
      (ver la sección~\ref{sec:sumatorias-productorias}):%
	\index{Iverson, convencion de@Iverson, convención de}
      \begin{equation*}
	\sum_{\mathclap{\gamma \in \mathcal{A} \cup \mathcal{B}}}
	  z^{\lvert \gamma \rvert}
	  = \; \sum_{\mathclap{\gamma \in \mathcal{A}
					    \cup \mathcal{B}}}
		 \left(
		   [\gamma \in \mathcal{A}]
		     z^{\lvert \gamma \rvert}
		      + [\gamma \in \mathcal{B}]
			  z^{\lvert \gamma \rvert}
		 \right)
	  = \sum_{\alpha \in \mathcal{A}} z^{\lvert \alpha \rvert}
	      + \sum_{\beta \in \mathcal{B}}
		  z^{\lvert \beta \rvert}
	  = A(z) + B(z)
      \end{equation*}
    \item % A x B
      Hay:
      \begin{equation*}
	\sum_{0 \le k \le n} a_k b_{n - k}
      \end{equation*}
      maneras de combinar elementos de \(\mathcal{A}\)
      con elementos de \(\mathcal{B}\) cuyos tamaños sumen \(n\),
      y este es precisamente
      el coeficiente de \(z^n\) en \(A(z) \cdot B(z)\).

      Alternativamente:
      \begin{align*}
	\sum_{\mathclap{\gamma \in \mathcal{A} \times \mathcal{B}}}
	     z^{\lvert \gamma \rvert}
	  = \sum_{\mathclap{\substack{
			      \alpha \in \mathcal{A} \\
			      \beta \in \mathcal{B}
		 }}} z^{\lvert \alpha \rvert + \lvert \beta \rvert}
	  = \left(
	       \sum_{\alpha \in \mathcal{A}}
		 z^{\lvert \alpha \rvert}
	     \right)
	       \cdot \left(
			\sum_{\beta \in \mathcal{B}}
			  z^{\lvert \beta \rvert}
		     \right)
	  = A(z) \cdot B(z)
      \end{align*}
    \item % Seq(A)
      Hay una manera de obtener la secuencia de largo 0
      (aporta el objeto vacío \(\epsilon\)),
      las secuencias de largo \(1\)
      son simplemente los elementos de \(\mathcal{A}\),
      las secuencias de largo \(2\)
      son elementos de \(\mathcal{A} \times \mathcal{A}\),
      y así sucesivamente.
      O sea,
      las secuencias se representan mediante:
      \begin{equation*}
	\mathcal{E}
	  + \mathcal{A}
	  + \mathcal{A} \times \mathcal{A}
	  + \mathcal{A} \times \mathcal{A} \times \mathcal{A}
	  + \dotsb
      \end{equation*}
      Por la segunda parte
      y la serie geométrica~\eqref{eq:serie-geometrica},
      la función generatriz correspondiente es:
      \begin{equation*}
	1 + A(z) + A^2(z) + A^3(z) + \dotsb
	  = \frac{1}{1 - A(z)}
      \end{equation*}
    \item % A o B
      Si elegimos \(\alpha \in \mathcal{A}\),
      vemos que debemos reemplazar sus \(\lvert \alpha \rvert\)~átomos
      por elementos de \(\mathcal{B}\)
      en el orden indicado por la estructura de \(\alpha\).
      Pero tal secuencia es enumerada por \(B(z)^{\lvert \alpha \rvert}\),
      con lo que al sumar:
      \begin{equation*}
	\sum_{\alpha \in \mathcal{A}} B(z)^{\lvert \alpha \rvert}
	  = A(B(z)
      \end{equation*}
    \item % Set(A)
      La clase de los subconjuntos finitos de \(\mathcal{A}\)
      queda representada por el producto simbólico:
      \begin{equation*}
	\prod_{\alpha \in \mathcal{A}} (\mathcal{E} + \{\alpha\})
      \end{equation*}
      ya que
      al distribuir los productos de todas las formas posibles
      aparecen todos los subconjuntos de \(\mathcal{A}\).
      Directamente obtenemos entonces:
      \begin{equation*}
	\prod_{\alpha \in \mathcal{A}}
	    \left( 1 + z^{\lvert \alpha \rvert} \right)
	  = \prod_{n \ge 0} (1 + z^n)^{a_n}
      \end{equation*}
      Otra forma de verlo es que cada elemento de tamaño \(n\)
      aporta un factor \(1 + z^n\),
      si hay \(a_n\) de estos
      el aporte total es \((1 + z^n)^{a_n}\).
      Esta es la primera parte de lo aseverado.
      Aplicando logaritmo:
      \begin{align*}
	\sum_{\alpha \in \mathcal{A}}
	    \ln \left(1 + z^{\lvert \alpha \rvert} \right)
	  &= -\sum_{\alpha \in \mathcal{A}}
		\sum_{k \ge 1}
		  \frac{(-1)^k z^{\lvert \alpha \rvert k}}{k}  \\
	  &= -\sum_{k \ge 1} \frac{(-1)^k}{k} \,
		\sum_{\alpha \in \mathcal{A}}
		  z^{\lvert \alpha \rvert k} \\
	  &= \sum_{k \ge 1} \frac{(-1)^{k + 1} \, A(z^k)}{k}
      \end{align*}
      Exponenciando lo último
      resulta equivalente a la segunda parte.
    \item % A o B
      Para cada objeto \(\alpha \in \mathcal{A}\)
      esto se traduce
      en una secuencia de \(\lvert \alpha \rvert\) elementos de \(\mathcal{B}\)
      a ser reemplazados por los átomos de \(\alpha\),
      con lo que la función generatriz respectiva es:
      \begin{equation*}
	\sum_{\alpha \in \mathcal{A}} B(z)^{\lvert \alpha \rvert}
      \end{equation*}
      que es lo prometido.
    \item % MSet(A)
      Podemos considerar un multiconjunto finito
      como la combinación de una secuencia
      para cada tipo de elemento:
      \begin{equation*}
	\prod_{\alpha \in \mathcal{A}} \Seq(\{ \alpha \})
      \end{equation*}
      La función generatriz buscada es:
      \begin{equation*}
	\prod_{\alpha \in \mathcal{A}}
	  \frac{1}{1 - z^{\lvert \alpha \rvert}}
	  = \prod_{n \ge 0} \frac{1}{(1 - z^n)^{a_n}}
      \end{equation*}
      Esto provee la primera parte.
      Nuevamente aplicamos logaritmo para simplificar:
      \begin{align*}
	\ln \prod_{\alpha \in \mathcal{A}}
	       \frac{1}{1 - z^{\lvert \alpha \rvert}}
	  &= - \sum_{\alpha \in \mathcal{A}}
		 \ln \left( 1 - z^{\lvert \alpha \rvert} \right) \\
	  &= \sum_{\alpha \in \mathcal{A}}
	       \sum_{k \ge 1}
		 \frac{z^{k \lvert \alpha \rvert}}{k} \\
	  &= \sum_{k \ge 1}
	       \frac{1}{k} \,
		 \sum_{\alpha \in \mathcal{A}}
		   z^{k \lvert \alpha \rvert} \\
	  &= \sum_{k \ge 1}
	       \frac{A(z^k)}{k}
      \end{align*}
    \item % Cyc(A)
      Esta situación es más compleja de tratar,
      la discutiremos en la sección~\ref{sec:ogf-ciclo} más abajo.
      \qedhere
    \end{enumerate}
  \end{proof}
  La utilidad del teorema~\ref{theo:ms-OGF}
  es que de cómo construir
  la clase de objetos que nos interesa
  da directamente una ecuación satisfecha por la función generatriz.
  Claro que igual resta extraer los coeficientes,
  tarea en la cual la fórmula de inversión de Lagrange
  (teorema~\ref{theo:LIF})%
    \index{Lagrange, inversion de@Lagrange, inversión de}
  es invaluable.

\subsection{Algunas aplicaciones}
\label{sec:ms-ogf-aplicaciones}

  La clase de los árboles binarios \(\mathcal{B}\)
  es por definición es la unión disjunta del árbol vacío
  y la clase de tuplas de un nodo (la raíz)
  y dos árboles binarios.%
    \index{arbol binario@árbol binario}
  O sea:
  \begin{equation*}
    \mathcal{B}
      = \mathcal{E}
	  + \mathcal{Z} \times \mathcal{B} \times \mathcal{B}
  \end{equation*}
  de donde directamente
  igual que antes obtenemos:
  \begin{equation*}
    B(z)
      = 1 + z B^2(z)
  \end{equation*}
  Con el cambio de variable \(u(z) = B(z) - 1\) queda:
  \begin{equation*}
    u(z)
      = z (1 + u(z))^2
  \end{equation*}
  Es aplicable la fórmula de inversión de Lagrange,%
    \index{Lagrange, inversion de@Lagrange, inversión de}
  teorema~\ref{theo:LIF},
  con \(\phi(u) = (u + 1)^2\) y \(f(u) = u\):
  \begin{align*}
    \left[ z^n \right] u(z)
      &= \frac{1}{n} \, \left[ u^{n - 1} \right] \, \phi(u)^n \\
      &= \frac{1}{n} \, \left[ u^{n - 1} \right] \, (u + 1)^{2 n} \\
      &= \frac{1}{n} \, \left[ u^{n - 1} \right] \,
	   \sum_{k \ge 0} \binom{2 n}{k} u^k \\
      &= \frac{1}{n} \, \binom{2 n}{n - 1}
  \end{align*}
  Tenemos,
  como \(u(z) = B(z) - 1\) y sabemos que \(b_0 = 1\):
  \begin{equation*}
    b_n =
    \begin{cases}
      \displaystyle
	\frac{1}{n}\binom{2 n}{n - 1}
	  = \frac{1}{n + 1} \, \binom{2 n}{n}
	       & \text{si \(n \ge 1\)} \\
      1
	       & \text{si \(n = 0\)}
    \end{cases}
  \end{equation*}
  Casualmente la expresión simplificada para \(n \ge 1\)
  da el valor correcto \(b_0 = 1\).
  A estos números ya los habíamos mencionado
  en~\eqref{eq:Catalan-numbers},%
    \index{Catalan, numeros de@Catalan, números de}
  son los números de Catalan.
  Es \(b_n = C_n\).

  Sea ahora \(\mathcal{A}\)
  la clase de \emph{árboles con raíz ordenados},%
    \index{arbol con raiz@árbol con raiz!ordenado}
  formados por un nodo raíz
  conectado a las raíces de una secuencia de árboles ordenados.
  La idea es que la raíz tiene hijos en un cierto orden.
  Simbólicamente:
  \begin{equation*}
    \mathcal{A}
      = \mathcal{Z} \times \Seq(\mathcal{A})
  \end{equation*}
  El método simbólico entrega directamente la ecuación:
  \begin{equation*}
    A(z)
      = \frac{z}{1 - A(z)}
  \end{equation*}
  Nuevamente es aplicable la fórmula de inversión de Lagrange,%
    \index{Lagrange, inversion de@Lagrange, inversión de}
  con \(\phi(A) = (1 - A)^{-1}\) y \(f(A) = A\):
  \begin{align*}
    \left[ z^n \right] A(z)
      &= \frac{1}{n} \, \left[ A^{n - 1} \right] \, \phi(A)^n \\
      &= \frac{1}{n} \, \left[ A^{n - 1} \right] \, (1 - A)^{-n} \\
      &= \frac{1}{n} \, \binom{2 n - 2}{n - 1} \\
      &= C_{n - 1}
  \end{align*}
  Otra vez números de Catalan.%
    \index{Catalan, numeros de@Catalan, números de}
  Un combinatorista de verdad considerará esto
  como el desafío de encontrar una biyección entre árboles binarios
  y árboles ordenados,
  nosotros nos contentaremos con consignar el resultado.

  La manera obvia de representar \(\mathbb{N}_0\)
  es mediante secuencias de marcas,
  como \(||||\) para 4;
  simbólicamente \(\mathbb{N}_0 = \Seq(\mathcal{Z})\).
  Para calcular el número de multiconjuntos de \(k\) elementos
  tomados entre \(n\),%
    \index{multiconjunto!numero@número}
  un multiconjunto queda representado
  por las cuentas de los \(n\) elementos de que se compone,
  y eso corresponde a:
  \begin{equation*}
    \mathbb{N}_0 \times \dotsb \times \mathbb{N}_0
      = (\Seq(\mathcal{Z}))^n
  \end{equation*}
  Para obtener el número que nos interesa:
  \begin{equation*}
    \multiset{n}{k}
      = \left[ z^k \right] (1 - z)^{-n}
      = (-1)^n \binom{-n}{k}
      = \binom{n + k - 1}{n}
  \end{equation*}
  Este resultado ya lo dedujimos
  en el capítulo~\ref{cha:combinatoria-elemental}.

  Consideremos \emph{árboles~\(2\)-\(3\)},%
     \index{arbol 2-3@árbol \(2\)-\(3\)}
  constando de un único nodo,
  o de un nodo conectado a \(2\) o \(3\) árboles~\(2\)-\(3\).
  Estos son de interés como estructuras de datos,
  dado que es fácil mantenerlos balanceados,
  de forma que todas las hojas estén a la misma distancia de la raíz.
  Vemos que de un árbol~\(2\)-\(3\) balanceado obtenemos uno mayor
  reemplazando simultáneamente todas las hojas por dos o tres nodos.
  Si consideramos el tamaño del árbol \(2\)-\(3\) como el número de sus hojas,
  descritas por la clase \(\mathcal{D}\),
  esto lleva a la ecuación simbólica:
  \begin{equation*}
    \mathcal{D}
      = \mathcal{Z}
	  + \mathcal{D}
	      \circ \left( \mathcal{Z}^2 + \mathcal{Z}^3 \right)
  \end{equation*}
  que nos da la ecuación funcional:
  \begin{equation}
    \label{eq:balanced-2-3-trees}
    D(z)
      = z + D(z^2 + z^3)
  \end{equation}
  Es claro que substituir \(s\) veces
  partiendo de la estimación inicial \(D(z) = z\)
  nos entrega hasta el coeficiente de \(z^{2 s}\).
  Resulta:
  \begin{equation*}
    D(z)
      = z + z^2 + z^3 + z^4 + 2 z^5 + 2 z^6 + 3 z^7 + 4 z^8 + 5 z^9
	 + 8 z^{10} + 14 z^{11} + 23 z^{12} + 32 z^{13} + 43 z^{14}
	 + \dotsb
  \end{equation*}
  El comportamiento de los coeficientes
  de soluciones de ecuaciones como~\eqref{eq:balanced-2-3-trees}
  es bastante extraño,
  Odlyzko~\cite{odlyzko82:_period_oscil_coeff_power_series}
  demuestra que oscilan y da rangos.

  Los \emph{árboles con raíz} constan de un nodo raíz
  conectado a una colección de árboles con raíz.%
    \index{arbol con raiz@árbol con raiz}
  La clase \(\mathcal{R}\) correspondiente cumple:
  \begin{equation*}
    \mathcal{R}
      = \mathcal{Z} \times \MSet(\mathcal{R})
  \end{equation*}
  Para la función generatriz queda:
  \begin{equation}
    \label{eq:rooted-tree-fe}
    R(z)
      = z \, \exp \left( \sum_{k \ge 1} \frac{R(z^k)}{k} \right)
  \end{equation}
  Ciertamente es una ecuación harto fea,
  pero puede usarse para obtener sucesivamente los \(r_n\).

  Una manera general
  de atacar ecuaciones como~\eqref{eq:rooted-tree-fe}
  es ver que da \(R\) en términos de \(R\),
  y comenzar con alguna aproximación
  para ir refinándola.
  De partida,
  vemos que \(r_0 = 0\),
  con lo que tenemos una aproximación inicial \(R^{(0)}(z) = 0\).
  Substituyendo en~\eqref{eq:rooted-tree-fe}
  obtenemos \(R^{(1)}(z) = z\).
  Como al substituir \(R^{(1)}(z)\) en~\eqref{eq:rooted-tree-fe}
  ya no aparecerán nuevos términos en \(z\),
  sabemos que \(r_1 = 1\).
  De la misma forma,
  cuando substituyamos \(R^{(n)}(z)\) en~\eqref{eq:rooted-tree-fe}
  ya no aparecerán nuevas contribuciones al coeficiente de \(z^n\),
  y este proceso converge
  según definimos en el capítulo~\ref{cha:series-formales}.
  No tiene sentido retener más términos al calcular \(R^{(n)}(z)\),
  los demás no influyen sobre el coeficiente de \(z^n\).
  Armado con un paquete de álgebra simbólica%
    \index{algebra simbolica@álgebra simbólica}
  o mucha paciencia
  se pueden calcular términos adicionales:
  \begin{equation}
    \label{eq:rooted-tree-gf}
    R(z)
      = z + z^2 + 2 z^3 + 4 z^4 + 9 z^5 + 20 z^6 + 48 z^7
	  + 115 z^8 + \dotsb
  \end{equation}
  Se pueden extraer estimaciones asintóticas
  de~\eqref{eq:rooted-tree-fe},
  ver por ejemplo a Flajolet y Sedgewick~%
    \cite{flajolet09:_analy_combin}
  o a Knuth~\cite{knuth11:_combin_alg_1},
  pero las técnicas a emplear escapan con mucho a nuestro ámbito.
  Ya Pólya~%
    \cite{polya87:_combin_enumer_group_graph_chemic_compoun}
  encontró:
  \begin{equation*}
    r_n
     \sim 0,4399 \cdot 2,9558^n \cdot n^{- 3 / 2}
  \end{equation*}

  Una \emph{combinación} de \(n\) es expresarlo como una suma.%
    \index{numero natural@número natural!combinacion@combinación}
  Por ejemplo,
  hay \(8\) combinaciones de \(4\):
  \begin{equation*}
      4
	= 3 + 1
	= 2 + 2
	= 2 + 1 + 1
	= 1 + 3
	= 1 + 2 + 1
	= 1 + 1 + 2
	= 1 + 1 + 1 + 1
  \end{equation*}
  Llamemos \(c(n)\) al número de combinaciones de \(n\).

  A la clase de los naturales
  podemos representarla como de secuencias de marcas.
  Por ejemplo, \(5\) es \(|||||\).
  Son secuencias no vacías,
  el primer natural es \(1\).
  Así:
  \begin{equation*}
    \mathbb{N}
      = \mathcal{Z} \times \Seq(\mathcal{Z})
  \end{equation*}
  Otra forma de representar al natural \(n\)
  es mediante una bolsa de \(n\) piedritas,
  que sugiere:
  \begin{equation*}
    \mathbb{N}
      = \mathcal{Z} \times \MSet(\mathcal{Z})
  \end{equation*}
  Las reglas de transferencia del teorema~\ref{theo:ms-OGF}
  en este caso particular
  dan la misma función generatriz para ambas:
  \begin{equation*}
    N(z)
      = \frac{z}{1 - z}
  \end{equation*}

  A su vez,
  una combinación no es más que una secuencia de naturales:
  \begin{equation*}
    \mathcal{C}
      = \Seq(\mathbb{N})
  \end{equation*}
  Directamente resulta:
  \begin{align*}
    C(z)
      &= \sum_{n \ge 0} c(n) z^n \\
      &= \frac{1}{1 - N(z)} \\
      &= \frac{1}{2} + \frac{1}{2} \cdot \frac{1}{1 - 2 z} \\
    c(n)
      &= \frac{1}{2} \, [n = 0] + \frac{1}{2} \cdot 2^n \\
      &= \frac{1}{2} \, [n = 0] + 2^{n - 1}
  \end{align*}
  Esto es consistente con \(c(4) = 8\) obtenido arriba.

% palabras-sin-patrones.tex
%
% Copyright (c) 2013-2014 Horst H. von Brand
% Derechos reservados. Vea COPYRIGHT para detalles

\subsection{Palabras que no contienen un patrón dado}
\label{sec:strings-excluding-pattern}
\index{palabra!numero@número}

  Nos interesan secuencias que no contengan un patrón dado.%
    \index{palabra!patron@patrón}
  Un ejemplo simple es secuencias binarias sin ceros seguidos.
  Llamemos \(\mathcal{B}_{00}\) a esta clase.
  Un elemento de \(\mathcal{B}_{00}\) puede ser vacío o \(0\),
  o es \(1\) o \(01\)
  seguido por un elemento de \(\mathcal{B}_{00}\).
  O sea:
  \begin{equation*}
    \mathcal{B}_{00}
      = \mathcal{E} + \{0\} + \{1, 01\} \times \mathcal{B}_{00}
  \end{equation*}
  Si \(z\) marca cada símbolo,
  para la respectiva función generatriz ordinaria \(B_{00}(z)\):%
    \index{generatriz}
  \begin{align*}
    B_{00}(z)
      &= 1 + z + (z + z^2) B_{00}(z) \\
  \intertext{Despejando:}
    B_{00}(z)
      &= \frac{1 + z}{1 - z - z^2}
  \end{align*}
  Resulta ser
    \(\left[ z^n \right] B_{00}(z) = F_n + F_{n + 1} = F_{n + 2}\),
  un número de Fibonacci.%
    \index{Fibonacci, numeros de@Fibonacci, números de}

  Si ahora buscamos que no contenga \(k\) ceros seguidos,
  podemos expresar:
  \begin{equation*}
    \mathcal{B}_{0^k}
      = \mathcal{P}_{< k}
	  + \mathcal{P}_{< k}
	      \times \{ 1 \} \times \mathcal{B}_{0^k}
  \end{equation*}
  Acá \(\mathcal{P}_{< k}\)
  es la clase de secuencias de menos de \(k\) ceros:
  \begin{equation*}
    \mathcal{P}_{< k}
      = \mathcal{E} + \{0\} + \{0\}^2 + \dotsb + \{0\}^{k - 1}
  \end{equation*}
  Las respectivas funciones generatrices ordinarias cumplen:
  \begin{align}
    P_{< k} (z)
      &= 1 + z + z^2 + \dotsb + z^{k - 1} \notag \\
      &= \frac{1 - z^k}{1 - z} \notag \\
    B_{0^k} (z)
      &= \frac{1 - z^k}{1 - z} (1 + z B_{0^k} (z)) \notag \\
  \intertext{Despejando:}
    B_{0^k} (z)
      &= \frac{1 - z^k}{1 - 2 z + z^{k + 1}}
	   \label{eq:B0^k}
  \end{align}
  Podemos extraer información adicional de acá.
  Los coeficientes de \(B_{0^k}(z)\)
  son el número de \foreignlanguage{english}{strings}
  que no contienen \(0^k\),
  el coeficiente de \(z^n\)
  dividido por \(2^n\) es la proporción del total:
  \begin{align*}
    B_{0^k}(z)
      &= \sum_{n \ge 0}
	   \text{\{\# de largo \(n\) sin \(0^k\)\}} z^n \\
    B_{0^k}(z / 2)
      &= \sum_{n \ge 0}
	   \text{\{\# de largo \(n\) sin \(0^k\)\}} / 2^n z^n \\
    B_{0^k}(1 / 2)
      &= \sum_{n \ge 0}
	   \text{\{\# de largo \(n\) sin \(0^k\)\}} / 2^n \\
      &= \sum_{n \ge 0}
	   \Pr( \text{No hay \(0^k\) en los primeros \(n\)}) \\
      &= \sum_{n \ge 0}
	   \Pr( \text{Primer \(0^k\)
		      termina después de \({} > n\)} ) \\
      &= \text{Posición esperada
	       del fin de los primeros \(k\) ceros}
  \end{align*}
  A esto se le llama \emph{tiempo de espera}%
    \index{palabra!tiempo de espera|textbfhy}
  (en inglés,
   \emph{\foreignlanguage{english}{waiting time}}).%
     \index{palabra!waiting time@\emph{\foreignlanguage{english}{waiting time}}|see{palabra!tiempo de espera}}
  Resulta:
  \begin{theorem}
    \label{theo:waiting-time-2^k}
    El tiempo de espera para los primeros \(k\) ceros
    en un \foreignlanguage{english}{string} binario al azar es:
    \begin{equation}
      \label{eq:waiting-time-2^k}
      B_{0^k}(1/2)
	= 2^{k + 1} - 2
    \end{equation}
  \end{theorem}
  O sea,
  en promedio hay que esperar \(30\)~bits hasta hallar \(0000\).
  La pregunta obvia es si esto vale también para otros patrones
  de largo cuatro,
  por ejemplo \(0001\).
  Resulta que no es así.
  Consideremos la primera vez que aparece \(000\).
  Es igualmente probable que continúe \(0000\) o \(0001\).
  Si \(0000\) no calza es \(0001\),
  y para \(0000\) debemos esperar al menos \(4\)~bits más.
  Si \(0001\) no calza,
  es porque es \(0000\) y el próximo bit puede completar \(0001\).

  Consideremos un patrón \(p\) de largo \(k\) arbitrario
  tomados entre \(s\) símbolos entonces.
  Sea \(\mathcal{B}_p\)
  la clase de \foreignlanguage{english}{strings}
  que no contienen \(p\),
  y sea \(\mathcal{T}_p\)
  la clase de \foreignlanguage{english}{strings}
  que terminan en \(p\),
  pero en los cuales \(p\) no aparece nunca antes del final.
  Es claro que \(\mathcal{B}_p\) y \(\mathcal{T}_p\) son disjuntos.
  Si agregamos un símbolo a un \foreignlanguage{english}{string}
  en \(\mathcal{B}_p\),
  el resultado es un \foreignlanguage{english}{string} no vacío
  en \(\mathcal{B}_p\) o en \(\mathcal{T}_p\).
  O sea:%
    \index{metodo simbolico@método simbólico}
  \begin{equation*}
    \mathcal{B}_p + \mathcal{T}_p
      = \mathcal{E} + \mathcal{B}_p \times \{ 0, 1, \dotsc, s - 1 \}
  \end{equation*}
  Esto nos da la ecuación funcional
  para las respectivas funciones generatrices ordinarias:%
    \index{generatriz}
  \begin{equation}
    \label{eq:B+T-fe}
    B_p(z) + T_p(z)
      = 1 + s z B_p(z)
  \end{equation}
  Hace falta determinar \(\mathcal{T}_p\).
  Es similar a \(\mathcal{B}_p \times \{ p \}\),
  pero debemos considerar que un elemento de \(\mathcal{B}_p\)
  puede terminar en ``casi'' \(p\),
  con lo que solo le falta una cola.

  El desarrollo que sigue
  es básicamente de Odlyzko~\cite{odlyzko85:_enum_strings}.
  Escribiremos \(\lvert x \rvert\)
  para el largo del \foreignlanguage{english}{string} \(x\)
  (el número de símbolos que lo componen).
  Para describir la manera
  en que dos \foreignlanguage{english}{strings} se traslapan
  definimos la \emph{correlación}%
    \index{palabra!correlacion@correlación|textbfhy}
  entre los \foreignlanguage{english}{string} \(x\) e \(y\)
  (posiblemente de distinto largo)
  como el polinomio \(c_{x y}(t)\) de grado \(\lvert x \rvert - 1\)
  tal que el coeficiente de \(t^k\)
  se determina ubicando \(y\) bajo \(x\)
  de manera que el primer caracter de \(y\)
  cae bajo el \(k\)\nobreakdash-ésimo caracter de \(x\).
  El coeficiente es \(1\) si ambos son iguales donde traslapan,
  \(0\) en caso contrario.
  Por ejemplo,
  si \(x = \mathtt{c a b c a b c}\)
  e \(y = \mathtt{a b c a b c d e}\),
  resulta \(c_{x y}(t) = t^4 + t\),
  como muestra el cuadro~\ref{tab:cxy}.
  \begin{table}[ht]
    \centering
    \begin{tabular}{>{\(}l<{\)}*{16}{>{\(\mathtt}c<{\)}}>{\(}r<{\)}}
      x: & c & a & b & c & a & b & c &	 &   &	 &   &	 &   &	 &   \\
      y: & a & b & c & a & b & c & d & e &   &	 &   &	 &   &	 & 0 \\
	 &   & a & b & c & a & b & c & d & e &	 &   &	 &   &	 & 1 \\
	 &   &	 & a & b & c & a & b & c & d & e &   &	 &   &	 & 0 \\
	 &   &	 &   & a & b & c & a & b & c & d & e &	 &   &	 & 0 \\
	 &   &	 &   &	 & a & b & c & a & b & c & d & e &   &	 & 1 \\
	 &   &	 &   &	 &   & a & b & c & a & b & c & d & e &	 & 0 \\
	 &   &	 &   &	 &   &	 & a & b & c & a & b & c & d & e & 0
    \end{tabular}
    \caption{Cálculo de $c_{x y} (t) = t^4 + t$}
    \label{tab:cxy}
  \end{table}
  Nótese que en general \(c_{x y}(t) \ne c_{y x}(t)\)
  (en el ejemplo es \(c_{y x}(t) = 0\)).
  De particular interés
  es la \emph{autocorrelación} \(c_x(t) = c_{x x}(t)\),
  la correlación de un \foreignlanguage{english}{string}
  consigo mismo.%
    \index{palabra!autocorrelacion@autocorrelación|textbfhy}
  En el ejemplo,
  \(c_x (t) = t^6 + t^3 + 1\).

  Fijemos un patrón \(p\) de largo \(k\),
  y escribamos:
  \begin{align*}
    B_p(z)
      &= \sum_{n \ge 0} b_n z^n \\
    T_p(z)
      &= \sum_{n \ge 0} t_n z^n
  \end{align*}
  Consideremos uno de los \(b_n\)
  \foreignlanguage{english}{strings} de largo \(n\)
  que no terminan en \(p\),
  y adosemos \(p\) al final.
  Sea \(n + r\) la posición
  en la cual por primera vez termina \(p\) en el resultado,
  donde \(0 < r \le k\).
  Como \(p\) también aparece al final,
  deben coincidir el prefijo de largo \(k - r\) de \(p\)
  y el sufijo de largo \(k - r\) de \(p\),
  o sea,
  \(\left[ t^{k - r} \right] c_p(t) = 1\).

  Para un ejemplo,
  sea el patrón \(p = \mathtt{a a b a}\)
  y el \foreignlanguage{english}{string}
    \(x = \mathtt{a b a b b a a b} \in \mathcal{B}_p\).
  Es \(k = \lvert p \rvert = 4\)
  y \(n = \lvert x \rvert = 8\).
  Vemos que
    \(x p = \mathtt{a b a b b
		    \textcolor{red}{a a b a}
		    \textcolor{blue}{a b a}}\),
  o sea,
  \(r = 1\)
  (hay un traslapo de \(k - r = 4 - 1 = 3\)
   entre el principio del patrón
   y el final del \foreignlanguage{english}{string}).
  Tenemos un miembro de \(\mathcal{T}_p\) de largo \(n + r = 9\)
  y una cola de largo \(k - r = 3\),
  determinados en forma única por \(x\) y \(p\).
  Esta descomposición solo es posible cuando
  \(\left[ t^{k - r} \right] c_p (t) = 1\).

  Nos interesa contar estos \foreignlanguage{english}{string}.
  Hay \(t_{n + r}\) de ellos,
  la descomposición descrita es una biyección.
  Vale decir,
  como los coeficientes de \(c_p\) son cero o uno:
  \begin{equation}
    \label{eq:bn=tnc}
    b_n
      = \sum_{0 < r \le k}
	  t_{n + r} \left[ t^{k - r} \right] c_p (t)
  \end{equation}
  Multiplicando~\eqref{eq:bn=tnc}
  por \(z^{n + k}\) y sumando para \(n \ge 0\)
  (recordar que \(k = \lvert p \rvert \ge \deg(c_p(t)) + 1\))
  da:
  \begin{align}
    B_p(z) z^k
      &= \sum_{n \ge 0}
	   z^{n + k}
	   \sum_{0 < r \le k}
	     t_{n + r}
	     \left[ t^{k - r} \right] c_p (t)
		 \notag \\
  \intertext{Esta es la convolución de \(T_p(z)\) con \(c_p(z)\):}
    B_p(z) z^k
      &= T_p(z) c_p(z) \label{eq:T-fe}
  \end{align}

  Uniendo las piezas anteriores
  tenemos un resultado de Solov'ev~%
    \cite{solovev66:_combin_ident_its_applic_probl}:
  \begin{theorem}
    \label{theo:Bp-gf}
    Sea \(p\) un patrón de largo \(k\) formado por \(s\) símbolos,
    con autocorrelación \(c_p(z)\).
    Entonces
    el número de \foreignlanguage{english}{strings} de largo \(n\)
    que no contienen el patrón \(p\)
    tiene función generatriz ordinaria:
    \begin{equation}
      \label{eq:Bp-gf}
      B_p(z)
	= \frac{c_p(z)}{(1 - s z) c_p(z) + z^k}
    \end{equation}
    El tiempo de espera para el patrón \(p\)%
      \index{palabra!tiempo de espera}
    está dado por:
    \begin{equation}
      \label{eq:WT-p}
      W_p
	= s^k c_p (1 / s)
    \qedhere
    \end{equation}
  \end{theorem}
  \begin{proof}
    La ecuación~\eqref{eq:Bp-gf}
    es la solución del sistema de ecuaciones~\eqref{eq:B+T-fe}
    y~\eqref{eq:T-fe}.
    El tiempo de espera
    es como se discutió antes para el patrón \(0^k\),
    la expresión dada
    resulta de substituir \(z = 1 / s\) en~\eqref{eq:Bp-gf}.
  \end{proof}
  Incidentalmente,
  al ser \(c_p\)
  un polinomio de coeficientes enteros de grado menor a \(k\),
  por~\eqref{eq:WT-p} el tiempo de espera siempre es un entero.

  Completando la discusión previa,
  tenemos \(c_{0000} (z) = 1 + z + z^2 + z^3\)
  y \(c_{0001} (z) = 1\).
  El tiempo de espera para el patrón \(p\)
  está dado por \(2^4 c_p(1/2)\).
  Para nuestros dos patrones,
  por las fórmulas desarrolladas antes:
  \begin{align}
    B_{0000}(z)
      &= \frac{1 - z^4}{1 - 2 z + z^5}
	      \label{B0000-fg} \\
    W_{0000}
      &= 30   \label{B0000-wt} \\
    B_{0001}(z)
      &= \frac{1}{1 - 2 z + z^4}
	      \label{B0001-fg} \\
    W_{0001}
      &= 16   \label{B0001-wt}
  \end{align}
  Por técnicas similares se pueden manejar conjuntos de patrones.

%%% Local Variables:
%%% mode: latex
%%% TeX-master: "clases"
%%% End:


\subsection{Construcción ciclo}
\label{sec:ogf-ciclo}

  El tratamiento de \(\Cyc(\mathcal{A})\)
  en el teorema~\ref{theo:ms-OGF}
  requiere considerar simetrías en la secuencia subyacente.%
    \index{ciclo!simetria@simetría}
  Por ejemplo,
  el ciclo \((a b c d)\) resulta de las cuatro secuencias
  \(a b c d\), \(b c d a\), \(c d a b\) y \(d a b c\);
  pero el ciclo \((a b a b)\) resulta solo de las dos
  \(a b a b\) y \(b a b a\).
  Por ahora
  hablaremos de secuencias y ciclos de símbolos,
  para luego aplicar lo aprendido a clases
  y sus funciones generatrices.
  Usaremos conceptos de funciones aritméticas
  y el anillo de Dirichlet%
    \index{Dirichlet, anillo de}
  en lo sucesivo,
  véase la sección~\ref{sec:estructura-Un}.

  Nuestro desarrollo sigue a Flajolet y Soria~%
    \cite{flajolet91:_cycle_constr}.
  Llamemos \emph{secuencia primitiva} a una secuencia
  que no es la repetición de una secuencia más corta.%
    \index{secuencia!primitiva|textbfhy}
  O sea,
  \(a b a a b\) es primitiva,
  \(a b a b = (a b)^2\) no lo es.
  La secuencia más corta
  tal que la secuencia dada
  se puede escribir como una repetición de ella
  la llamamos la \emph{raíz} de la secuencia.%
    \index{secuencia!raiz@raíz|textbfhy}
  En el ejemplo,
  la raíz de \(a b a a b\) es \(a b a a b\),
  ya que es primitiva;
  la raíz de \(a b a b\) es \(a b\).
  Sea \(s\) el número de símbolos en el alfabeto
  sobre el cual se consideran las secuencias.
  Si llamamos \(p_n\)
  al número de secuencias primitivas de largo \(n\),
  como toda secuencia es la repetición de una secuencia primitiva,
  debe ser:
  \begin{equation*}
    s^n
      = \sum_{d \mid n} p_d
  \end{equation*}
  El lado izquierdo es el número total de secuencias de largo \(n\),
  el lado derecho cuenta este mismo número
  como secuencias primitivas de largo \(d\)
  que se repiten para dar el largo \(n\).
  Inversión de Möbius,
  teorema~\ref{theo:Moebius-inversion},
  entrega:
  \begin{equation*}
    p_n
      = \sum_{d \mid n} \mu(d) s^{n / d}
  \end{equation*}
  Necesitamos extender este resultado
  a las funciones generatrices respectivas.
  \begin{lemma}[Inversión de Möbius]
    \index{Mobius, inversion de@Möbius, inversión de|textbfhy}
    \label{lem:GF-Moebius-inversion}
    Sean secuencias \(\langle u_n \rangle_{n \ge 0}\)
    y \(\langle v_n \rangle_{n \ge 0}\)
    tales que para \(n \ge 1\):
    \begin{equation*}
      u_n
	= \sum_{d \mid n} v_d
    \end{equation*}
    Entonces las funciones generatrices ordinarias
    respectivas \(U(z)\) y \(V(z)\)
    cumplen:
    \begin{align*}
      U(z)
	&= \sum_{n \ge 1} V(z^n) \\
      V(z)
	&= \sum_{n \ge 1} \mu(n) U(z^n)
    \end{align*}
  \end{lemma}
  \begin{proof}
    Primero,
    usando la convención de Iverson:%
      \index{Iverson, convencion de@Iverson, convención de}
    \begin{align*}
      U(z)
	&= \sum_{n \ge 1} u_n z^n \\
	&= \sum_{n \ge 1}
	     \sum_{a \ge 1} \sum_{b \ge 1} [a b = n] v_a z^n \\
	&= \sum_{a \ge 1} \sum_{b \ge 1} v_a z^{a b}
	     \sum_{n \ge 1} [a b = n] \\
	&= \sum_{b \ge 1} \sum_{a \ge 1} v_a (z^b)^a \\
	&= \sum_{b \ge 1} V(z^b)
     \end{align*}
     De la misma forma:
     \begin{align*}
       V(z)
	 &= \sum_{n \ge 1} \sum_{a \ge 1} \sum_{b \ge 1} [a b = n]
	      \mu(a) u_b z^n \\
	 &= \sum_{a \ge 1} \mu(a) \sum_{b \ge 1} u_b z^{a b} \\
	 &= \sum_{a \ge 1} \mu(a) U(z^a)
       \qedhere
     \end{align*}
  \end{proof}
  Queda claro que
  si \(\alpha(n)\) tiene inversa de Dirichlet \(\alpha^{-1}(n)\) y:
  \begin{align}
    u_n
      &= \sum_{d \mid n} \alpha(n / d) v_d \notag \\
    v_n
      &= \sum_{d \mid n} \alpha^{-1}(n / d) u_d \notag \\
    \intertext{entonces:}
    U(z)
      &= \sum_{n \ge 1} \alpha(n) V(z^n)
	     \label{eq:Dirichlet-sum-gf} \\
    V(z)
      &= \sum_{n \ge 1} \alpha^{-1}(n) U(z^n)
	     \label{eq:Dirichlet-inversion-gf}
  \end{align}

  Igual que en el caso de secuencias
  podemos hablar de \emph{ciclos primitivos}%
    \index{ciclo!primitivo}
  y sus \emph{raíces},
  nos interesa la relación entre secuencias y ciclos.
  \begin{figure}[ht]
    \centering
    \subfloat[Raíz \(a\)]{\pgfimage{images/6cycle-1}}%
    \hspace{3em}%
    \subfloat[Raíz \(a b\)]{\pgfimage{images/6cycle-2}}%
    \hspace{3em}%
    \subfloat[Raíz \(a a b\)]{\pgfimage{images/6cycle-3}}%
    \hspace{3em}%
    \subfloat[Primitivo]{\pgfimage{images/6cycle-6}}
    \caption{Ciclos de largo seis}
    \label{fig:6cycles}
  \end{figure}
  La figura~\ref{fig:6cycles} muestra algunos ciclos de largo seis
  con sus raíces.

  Sea \(\omega \) una secuencia primitiva de largo \(l\),
  y consideremos el ciclo formado por \(r\) copias de \(\omega\),
  \(\omega^r\).
  Si lo rotamos en un múltiplo de \(l\) posiciones
  obtenemos el original.
  Si lo rotamos en menos de \(l\) posiciones,
  el efecto es dividir \(\omega = \alpha \beta\)
  y trasladar \(\alpha\) al final,
  queda \(\beta (\alpha \beta)^{r - 1} \alpha = (\beta \alpha)^r\),
  nuevamente un ciclo con raíz de largo \(l\).
  Esto ocurre al rotar en cualquier número de posiciones
  que no es un múltiplo de \(l\).
  Si la secuencia es primitiva,
  todas las rotaciones de la misma también lo son.
  Vale decir,
  hay un mapa \(1\) a \(l\) de secuencias primitivas de largo \(l\)
  a ciclos primitivos de largo \(l\).
  A su vez,
  todo ciclo es la repetición de un ciclo primitivo
  (su raíz).

  Traduzcamos lo anterior a funciones generatrices ahora.
  Consideremos primeramente
  secuencias de al menos un \(\mathcal{A}\).
  Si además del tamaño total
  nos interesa el número de \(\mathcal{A}\) componentes,
  usando una clase auxiliar \(\mathcal{U}\)
  con un único elemento de tamaño uno
  podemos representar la clase \(\mathcal{S}\)
  que nos interesa como:
  \begin{equation*}
    \mathcal{S}
      = \Seq_{\ge 1} (\mathcal{U} \times \mathcal{A})
  \end{equation*}
  De esto,
  usando \(u\)
  para contar el número de \(\mathcal{U}\) participantes
  (vale decir,
   el número de \(\mathcal{A}\) que componen nuestra secuencia)
  el método simbólico da:
  \begin{equation*}
    S(z, u)
      = \frac{u A(z)}{1 - u A(z)}
  \end{equation*}
  La función generatriz \(S_p(z, u)\)
  del número de secuencias primitivas formadas por \(\mathcal{A}\)
  queda determinado por la ecuación implícita:
  \begin{equation*}
    S(z, u)
      = \sum_{n \ge 1} S_p(z^n, u^n)
  \end{equation*}
  de donde por el lema~\ref{lem:GF-Moebius-inversion}:
  \begin{equation}
    \label{eq:ms-primitive-sequences}
    S_p(z, u)
      = \sum_{n \ge 1} \mu(n) \, \frac{u^n A(z^n)}{1 - u^n A(z^n)}
  \end{equation}
  Pero nos interesa la función generatriz de los ciclos primitivos,
  \(C_p(z, u)\).
  Por lo discutido antes,
  esto se obtiene de \(S_p(z, u)\)
  haciendo el reemplazo \(u^l \mapsto u^l / l\),
  y esto a su vez integrando término a término se obtiene como:
  \begin{align}
    C_p(z, u)
      &= \int_0^u S_p(z, v) \, \frac{\mathrm{d} v}{v} \notag \\
      &= \sum_{n \ge 1}
	   \frac{\mu(n)}{n} \, \ln \frac{1}{1 - u^n A(z^n)}
	    \label{eq:gf-primitive-cycle}
  \end{align}
  Construimos ciclos completos repitiendo ciclos primitivos,
  lo que corresponde a la inversa de Dirichlet
  de~\eqref{eq:gf-primitive-cycle}:
  \begin{equation}
    \label{eq:Cycle-binary}
    C(z, u)
      = \sum_{n \ge 1}
	  \frac{\phi(n)}{n} \, \ln \frac{1}{1 - u^n A(z^n)}
  \end{equation}
  Substituyendo \(u = 1\) obtenemos la ecuación prometida.

  Resta demostrar que en el anillo de Dirichlet:%
    \index{Dirichlet, anillo de}
  \begin{equation*}
    \left( \frac{\mu(n)}{n} \right)^{-1}
      = \frac{\phi(n)}{n}
  \end{equation*}
  La identidad de Gauß%
    \index{Gauss, identidad de@Gauß, identidad de}
  (teorema~\ref{theo:Gauss-identity}) dice:
  \begin{equation*}
    n = \sum_{d \mid n} \phi(d)
  \end{equation*}
  Por inversión de Möbius:%
    \index{Mobius, inversion de@Möbius, inversión de}
  \begin{equation*}
    \phi(n)
      = \sum_{d \mid n} \mu(d) \, \frac{n}{d}
  \end{equation*}
  que es equivalente a lo que buscábamos demostrar.

% irreducible-polynomials.tex
%
% Copyright (c) 2013-2014 Horst H. von Brand
% Derechos reservados. Vea COPYRIGHT para detalles

\subsection{Polinomios irreductibles en \(\mathbb{F}_q\)}
\label{sec:count-irreductible-polynomials}

  Recordamos del capítulo~\ref{cha:campos-finitos}
  que los polinomios
  \(\mathbb{F}_q[x]\) para \(q\) la potencia de un primo
  son un dominio euclidiano,
  por lo que
  por la teoría de la sección~\ref{sec:dominios-euclidianos},
  en particular el teorema~\ref{theo:PID=>UFD},
  nos asegura que hay factorización única
  (salvo unidades)
  en \(\mathbb{F}_q[x]\).
  Para obviar las unidades,
  consideremos polinomios mónicos.

  Si para el polinomio \(\alpha(x) \in \mathbb{F}_q[x]\)
  consideramos su grado como tamaño,
  vemos que multiplicar polinomios es simplemente sumar sus tamaños.
  Podemos entonces considerar
  la clase \(\mathcal{P}\)
  de polinomios mónicos en \(\mathbb{F}_q[x]\)
  con \(\lvert \alpha(x) \rvert = \deg(\alpha)\),
  y combinar polinomios corresponde a multiplicarlos.
  Es claro que hay \(q^n\) polinomios mónicos de grado \(n\),
  o sea la función generatriz ordinaria
  que cuenta polinomios mónicos es:
  \begin{equation*}
    P(z)
      = \sum_{n \ge 0} q^n z^n
      = \frac{1}{1 - q z}
  \end{equation*}

  Consideremos la clase \(\mathcal{I}\)
  de polinomios mónicos irreductibles,
  contados por la función generatriz ordinaria:
  \begin{equation*}
    I(z)
      = \sum_{n \ge 0} N_n z^n
  \end{equation*}
  Factorización única significa que todo polinomio
  corresponde a un multiconjunto de polinomios irreductibles:
  \begin{equation}
    \label{eq:polynomials=MSet(irreducibles)}
    \mathcal{P}
      = \MSet(\mathcal{I})
  \end{equation}
  Resulta interesante
  contar con una forma de resolver ecuaciones implícitas
  como~\eqref{eq:polynomials=MSet(irreducibles)}.
  \begin{theorem}
    \label{theo:A=MSet(B)}
    Sean \(\mathcal{A}\) y \(\mathcal{B}\)
    clases de objetos no rotulados
    relacionadas mediante:
    \begin{equation*}
      \mathcal{A}
	= \MSet(\mathcal{B})
    \end{equation*}
    Entonces las funciones generatrices ordinarias
    respectivas cumplen:
    \begin{equation}
      \label{eq:A=MSet(B)-->B}
      B(z)
	= \sum_{k \ge 1} \frac{\mu(k)}{k} \, \ln A(z^k)
    \end{equation}
  \end{theorem}
  \begin{proof}
    El método simbólico da:
    \begin{equation*}
      A(z)
	= \exp \left( \sum_{k \ge 1} \frac{B(z^k)}{k} \right)
    \end{equation*}
    Tomando logaritmos:
    \begin{align*}
      \ln A(z)
	&= \sum_{r \ge 1} \frac{B(z^r)}{r} \\
	&= \sum_{r \ge 1} \frac{1}{r} \, \sum_{s \ge 1} b_s z^{r s}
    \end{align*}
    Extraemos coeficientes:
    \begin{align*}
      n \left[ z^n \right] \ln A(z) \\
	&= \sum_{r \ge 1} \frac{n}{r} \, \sum_{s \ge 1} b_s z^{r s} \\
	&= \sum_{r s = n} s b_s
    \end{align*}
    Este es exactamente el caso
    del lema~\ref{lem:GF-Moebius-inversion},
    lo que entrega lo enunciado.
  \end{proof}

  Con el teorema~\ref{theo:A=MSet(B)}
  queda de~\eqref{eq:polynomials=MSet(irreducibles)}:
  \begin{align*}
    I(z)
      = \sum_{k \ge 1} \frac{\mu(k)}{k} \, \frac{1}{1 - q z^k}
  \end{align*}
  Tenemos nuevamente el resultado
  del teorema~\ref{theo:number-irreducible-polynomials}:
  \begin{equation*}
    N_n
      = \frac{1}{n} \, \sum_{d \mid n} \mu(n / d)  \, q^d
  \end{equation*}

%%% Local Variables:
%%% mode: latex
%%% TeX-master: "clases"
%%% End:


\section{Objetos rotulados}
\label{sec:rotulados}
\index{metodo simbolico@método simbólico!objetos rotulados}

  En la discusión previa solo interesaba el tamaño de los objetos,
  no su disposición particular.
  Consideraremos ahora objetos rotulados,
  donde importa cómo se compone el objeto de sus partes
  (los átomos están numerados,
   o se ubican en orden).

  El objeto más simple con partes rotuladas son las permutaciones
  (biyecciones \(\sigma \colon [n] \rightarrow [n]\),
   podemos considerarlas secuencias de átomos numerados).
  Para la función generatriz exponencial tenemos,
  ya que hay \(n!\) permutaciones de \(n\) elementos:
  \begin{equation*}
    \sum_{\sigma}
	\frac{z^{\lvert \sigma \rvert}}{\lvert \sigma \rvert !}
      = \sum_{n \ge 0} n! \, \frac{z^n}{n!}
      = \frac{1}{1 - z}
  \end{equation*}

  Lo siguiente más simple de considerar
  es colecciones de ciclos rotulados.
  Por ejemplo,
  escribimos \((1\;3\;2)\) para el objeto
  en que viene \(3\) luego de \(1\),
  \(2\) sigue a \(3\),
  y a su vez \(1\) sigue a \(2\).
  Así \((2\;1\;3)\) es solo otra forma de anotar el ciclo anterior,
  que no es lo mismo que \((3\;1\;2)\).
  Interesa definir formas consistentes
  de combinar objetos rotulados.
  Por ejemplo,
  al combinar el ciclo \((1\;2)\) con el ciclo \((1\;3\;2)\)
  resultará un objeto con \(5\) rótulos,
  y debemos ver cómo los distribuimos entre las partes.
  El cuadro~\ref{tab:ciclo+ciclo}
  reseña las posibilidades al respetar
  el orden de los elementos asignados a cada parte.
  \begin{table}[htbp]
    \centering
    \begin{tabular}{*{4}{>{\(}l<{\)}}}
      (1\;2) (3\;5\;4) & (2\;3) (1\;5\;4) & (3\;4) (1\;5\;2) &
	  (4\;5) (1\;3\;2) \\
      (1\;3) (2\;5\;4) & (2\;4) (1\;3\;5) & (3\;5) (1\;4\;2) \\
      (1\;4) (2\;5\;3) & (2\;5) (1\;3\;4) \\
      (1\;5) (2\;4\;3)
    \end{tabular}
    \caption{Combinando los ciclos $(1\;2)$ y $(1\;3\;2)$}
    \label{tab:ciclo+ciclo}
  \end{table}
  Es claro que lo que estamos haciendo es elegir
  un subconjunto de \(2\) rótulos
  de entre los \(5\) para asignárselos al primer ciclo.
  El combinar
  dos clases de objetos \(\mathcal{A}\) y \(\mathcal{B}\)
  de esta forma lo anotaremos \(\mathcal{A} \star \mathcal{B}\).
  Otra operación común es la \emph{composición},
  anotada \(\mathcal{A} \circ \mathcal{B}\).
  La idea es elegir un elemento \(\alpha \in \mathcal{A}\),
  luego elegir \(\lvert \alpha \rvert\) elementos
  de \(\mathcal{B}\),
  y reemplazar los \(\mathcal{B}\) por las partes de \(\alpha\),
  en el orden que están rotuladas;
  para finalmente asignar rótulos a los átomos
  que conforman la.estructura completa
  respetando el orden de los rótulos
  al interior de los \(\mathcal{B}\).
  Ocasionalmente es útil \emph{marcar}
  uno de los componentes del objeto,
  operación que anotaremos \(\mathcal{A}^\bullet\).
  Otra notación común para esta operación es \(\Theta \mathcal{A}\).
  Usaremos también la construcción \(\MSet(\mathcal{A})\),
  que podemos considerar como una secuencia de elementos numerados
  obviando el orden.
  Cuidado,
  muchos textos le llaman \(\Set()\) a esta operación.

  Tenemos el siguiente teorema:
  \begin{theorem}[Método simbólico, EGF]
    \index{metodo simbolico@método simbólico!teorema de transferencia!objetos rotulados|textbfhy}
    \label{theo:ms-EGF}
    Sean \(\mathcal{A}\) y \(\mathcal{B}\) clases de objetos,
    con funciones generatrices exponenciales
    \(\widehat{A}(z)\) y \(\widehat{B}(z)\),
    respectivamente.
    Entonces tenemos
    las siguientes funciones generatrices exponenciales:
    \begin{enumerate}
    \item
      Para enumerar \(\mathcal{A}^\bullet\):
      \begin{equation*}
	z \mathrm{D} \widehat{A}(z)
      \end{equation*}
    \item
      Para enumerar \(\mathcal{A} + \mathcal{B}\):
      \begin{equation*}
	\widehat{A}(z) + \widehat{B}(z)
      \end{equation*}
    \item
      Para enumerar \(\mathcal{A} \star \mathcal{B}\):
      \begin{equation*}
	\widehat{A}(z) \cdot \widehat{B}(z)
      \end{equation*}
    \item
      Para enumerar \(\mathcal{A}^\bullet\):
      \begin{equation*}
	z \widehat{A}'(z)
      \end{equation*}
    \item
      Para enumerar \(\mathcal{A} \circ \mathcal{B}\):
      \begin{equation*}
	\widehat{A}(\widehat{B}(z))
      \end{equation*}
    \item
      Para enumerar \(\Seq(\mathcal{A})\):
      \begin{equation*}
	\frac{1}{1 - \widehat{A}(z)}
      \end{equation*}
    \item
      Para enumerar \(\MSet(\mathcal{A})\):
      \begin{equation*}
	\mathrm{e}^{\widehat{A}(z)}
      \end{equation*}
    \item
      Para enumerar \(\Cyc(\mathcal{A})\):
      \begin{equation*}
	-\ln(1 - \widehat{A}(z))
      \end{equation*}
    \end{enumerate}
  \end{theorem}
  \begin{proof}
    Usaremos casos ya demostrados en las demostraciones sucesivas.
    \begin{enumerate}
    \item % mark A
      El objeto \(\alpha \in \mathcal{A}\)
      da lugar a \(\lvert \alpha \rvert\) objetos
      al marcar cada uno de sus átomos,
      lo que da la función generatriz exponencial:
      \begin{equation*}
	\sum_{\alpha \in \mathcal{A}}
	  \lvert \alpha \rvert
	    \frac{z^{\lvert \alpha \rvert}}{\lvert \alpha \rvert !}
      \end{equation*}
      Esto es lo indicado.
    \item % A + B
      Nuevamente trivial.
    \item % A x B
      El número de objetos \(\gamma\) que se obtienen
      al combinar \(\alpha \in \mathcal{A}\)
      con \(\beta \in \mathcal{B}\) es:
      \begin{equation*}
	\binom{\lvert \alpha \rvert + \lvert \beta \rvert}
	      {\lvert \alpha \rvert}
      \end{equation*}
      y tenemos la función generatriz exponencial:
      \begin{equation*}
	\sum_{\gamma \in \mathcal{A} \star \mathcal{B}}
	    \frac{z^{\lvert \gamma \rvert}}{\lvert \gamma \rvert !}
	  = \sum_{\substack{
		     \alpha \in \mathcal{A} \\
		     \beta \in \mathcal{B}
		 }}
	       \binom{\lvert \alpha \rvert + \lvert \beta \rvert}
		     {\lvert \alpha \rvert}
		  \frac{z^{\lvert \alpha \rvert
			    + \lvert \beta \rvert}}
		       {(\lvert \alpha \rvert
			    + \lvert \beta \rvert)!}
	  = \left(
	      \sum_{\alpha \in \mathcal{A}}
		\frac{z^{\lvert \alpha \rvert}}
		     {\lvert \alpha \rvert !}
	    \right)
	      \cdot \left(
		\sum_{\beta \in \mathcal{B}}
		  \frac{z^{\lvert \beta \rvert}}
		       {\lvert \beta \rvert !}
		    \right)
	  = \widehat{A}(z) \cdot \widehat{B}(z)
      \end{equation*}
    \item % Mark(A)
      Si tomamos un objeto \(\alpha \in \mathcal{A}\)
      de tamaño \(\lvert \alpha \rvert\),
      estamos creando \(\lvert \alpha \rvert\) nuevos objetos
      al marcar cada uno de sus componentes.
      La función generatriz resultante es:
      \begin{equation*}
	\sum_{\alpha \in \mathcal{A}}
	  \lvert \alpha \rvert \frac{z^{\lvert \alpha \rvert}}
				    {\lvert \alpha \rvert !}
	  = z \widehat{A}'(z)
      \end{equation*}
    \item % A circ B
      Tomemos \(\alpha \in \mathcal{A}\),
      de tamaño \(n = \lvert \alpha \rvert\),
      y \(n\) elementos de \(\mathcal{B}\) en orden
      a ser reemplazados por las partes de \(\alpha\).
      Esa secuencia de \(\mathcal{B}\) es representada por:
      \begin{equation*}
	\mathcal{B}
	  \star \mathcal{B}
	  \star \dotsb
	  \star \mathcal{B}
      \end{equation*}
      con función generatriz exponencial:
      \begin{equation*}
	\widehat{B}^n (z)
      \end{equation*}
      Sumando sobre las contribuciones:
      \begin{equation*}
	\sum_{\alpha \in \mathcal{A}}
	   \frac{\widehat{B}^{\lvert \alpha \rvert}(z)}
		{\lvert \alpha \rvert \, !}
      \end{equation*}
      Esto es lo prometido.
    \item % Seq(A)
      Primeramente,
      para \(\Seq(\mathcal{Z})\),
      como hay \(n!\) secuencias de largo \(n\):
      \begin{equation*}
	\sum_{n \ge 0} n! \frac{z^n}{n!}
	  = \frac{1}{1 - z}
      \end{equation*}
      Aplicando composición se obtiene lo indicado.
    \item % MSet(A)
      Hay un único multiconjunto de \(n\) elementos rotulados
      (se rotulan simplemente de 1 a  \(n\)),
      con lo que \(\MSet(\mathcal{Z})\) corresponde a:
      \begin{equation*}
	\sum_{n \ge 0} \frac{z^n}{n!}
	  = \exp(z)
      \end{equation*}
      Al aplicar composición resulta lo anunciado.

      Otra demostración es considerar el multiconjunto de \(\mathcal{A}\),
      descrito por \(\mathcal{M} = \MSet(\mathcal{A})\).
      Si marcamos uno de los átomos de \(\mathcal{M}\)
      estamos marcando uno de los \(\mathcal{A}\),
      el resto sigue formando un multiconjunto de \(\mathcal{A}\):
      \begin{equation*}
	\mathcal{M}^\bullet
	  = \mathcal{A}^\bullet \star \mathcal{M}
      \end{equation*}
      Por lo anterior:
      \begin{equation*}
	z M'(z)
	  = z A'(z) M(z)
      \end{equation*}
      Hay un único multiconjunto de tamaño \(0\),
      o sea \(M(0) = 1\);
      y hemos impuesto la condición
      que no hay objetos de tamaño \(0\) en \(\mathcal{A}\),
      vale decir,
      \(A(0) = 0\).
      Así la solución a la ecuación diferencial es:
      \begin{equation*}
	M(z)
	  = \exp(A(z))
      \end{equation*}
    \item % Cyc(A)
      Consideremos un ciclo de \(\mathcal{A}\),
      o sea \(\mathcal{C} = \Cyc(\mathcal{A})\).
      Si marcamos los \(\mathcal{C}\),
      estamos marcando uno de los \(\mathcal{A}\),
      y el resto es una secuencia:
      \begin{equation*}
	\mathcal{C}^\bullet
	  = \mathcal{A}^\bullet \star \Seq(\mathcal{A})
      \end{equation*}
      Esto se traduce en la ecuación diferencial:
      \begin{equation*}
	z \widehat{C}'(z)
	  = z A'(z) \frac{1}{1 - A(z)}
      \end{equation*}
      Integrando bajo el entendido \(C(0) = 0\) con \(A(0) = 0\)
      se obtiene lo indicado.
      \qedhere
    \end{enumerate}
  \end{proof}

\subsection{Rotulado o no rotulado, esa es la cuestión\ldots}
\label{sec:rotulado-o-no}

  Después de las exposiciones anteriores
  el amable lector estará comprensiblemente confundido
  respecto de cuándo considerar rotulados los objetos entre manos.
  Como regla general,
  se deben considerar no rotulados los objetos
  en los cuales piezas iguales son intercambiables.
  Al considerar un canasto de frutas,
  como en el primer ejemplo
  de la sección~\ref{sec:combinatorial-applications},
  consideramos que solo es relevante el número
  de las frutas de los distintos tipos.
  En los términos presentes,
  son multiconjuntos de objetos no rotulados,
  y la clase canasto
  (número par de manzanas,
   número de plátanos divisible por cinco,
   a lo más cuatro naranjas,
   opcionalmente una sandía)
  se representa simbólicamente como:
  \begin{equation*}
    \mathcal{C}
      = \MSet(\mathcal{Z} \times \mathcal{Z})
	  \times \MSet(\mathcal{Z}
			 \times \mathcal{Z}
			 \times \mathcal{Z}
			 \times \mathcal{Z}
			 \times \mathcal{Z})
	  \times \MSet_{\le 4}(\mathcal{Z})
	  \times \MSet_{\le 1}(\mathcal{Z})
  \end{equation*}
  Aplicando las reglas de transferencia
  del teorema~\ref{theo:ms-OGF}
  resulta la función generatriz ordinaria:
  \begin{align*}
    C(z)
      &= \frac{1}{1 - z^2}
	   \cdot \frac{1}{1 - z^5}
	   \cdot (1 + z + z^2 + z^3 + z^4)
	   \cdot (1 + z) \\
      &= \frac{1}{(1 - z)^2}
  \end{align*}
  y en consecuencia el número de canastos con \(n\) frutas es:
  \begin{equation*}
    \left[ z^n \right] C(z)
      = n + 1
  \end{equation*}
  como habíamos deducido antes.

  Las permutaciones son secuencias de elementos distinguibles,%
    \index{permutacion@permutación!generatriz}
  por lo que se consideran objetos rotulados.
  La clase de permutaciones
  queda entonces representada por la expresión simbólica:
  \begin{equation*}
    \mathcal{P}
      = \Seq(\mathcal{Z})
  \end{equation*}
  Las reglas del teorema~\ref{theo:ms-EGF}
  dan la función generatriz exponencial:
  \begin{equation*}
    \widehat{P}(z)
      = \frac{1}{1 - z}
  \end{equation*}
  de donde el número de permutaciones de \(n\) elementos es:
  \begin{equation*}
    n! \left[ z^n \right] \widehat{P}(z)
      = n!
  \end{equation*}
  como ya sabíamos.

  Hay situaciones en las cuales los objetos en sí
  son indistinguibles,
  pero los consideramos rotulados por sus posiciones.
  Un ejemplo popular considera un buque que hace señales
  mediante \(12\)~banderas de colores blanco, rojo, azul y negro.
  Se restringen las señales
  a tener un número par de banderas blancas
  e impar de rojas.
  Se pregunta cuántas señales diferentes puede dar,
  suponiendo que tiene banderas suficientes de cada color.
  Esto puede responderse por las técnicas
  de la sección~\ref{sec:tao-bookkeeper},
  pero resulta engorroso.
  Si consideramos cada color de bandera
  como el multiconjunto de esa bandera rotuladas por su posición,
  la regla de distribución de rótulos de la operación \(\star\)
  exactamente corresponde
  a barajar las banderas de los distintos colores.
  Considerando entonces objetos rotulados,
  la clase de señales queda descrita simbólicamente por:
  \begin{equation*}
    \mathcal{S}
      = \MSet_{\text{even}} (\mathcal{Z})
	  \star \MSet_{\text{odd}} (\mathcal{Z})
	  \star \MSet(\mathcal{Z})
	  \star \MSet(\mathcal{Z})
  \end{equation*}
  El teorema~\ref{theo:ms-EGF} da la función generatriz exponencial
  (el multiconjunto de número par de elementos da los términos pares
   de la serie para la exponencial,
   multiconjuntos con número impar
   de elementos da los términos impares;
   y esto a su vez da coseno y seno hiperbólicos):
  \begin{align*}
    \widehat{S}(z)
      &= \cosh z \, \sinh z \, \mathrm{e}^{2 z} \\
      &= \frac{\mathrm{e}^{4 z} - 1}{4}
  \end{align*}
  y el número de señales que pueden formarse con \(12\)~banderas es:
  \begin{equation*}
    12! \left[ z^{12} \right] \widehat{S}(z)
      = 12! \, \frac{1}{4} \frac{4^{12}}{12!}
      = 4^{11}
  \end{equation*}

\subsection{Algunas aplicaciones de objetos rotulados}
\label{sec:ms-egf-aplicaciones}

  Un primer ejemplo simple
  es determinar
  el número de organizaciones circulares de \(n\) elementos.
  Al ser diferentes,
  podemos considerarlos rotulados.
  Quedan representadas simbólicamente por \(\Cyc(\mathcal{Z})\),
  con función generatriz exponencial:
  \begin{equation*}
    \ln \frac{1}{1 - z}
  \end{equation*}
  Como es una función generatriz exponencial,
  interesa:
  \begin{equation*}
    n! \left[ z^n \right] \ln \frac{1}{1 - z}
      = n! \left[ z^n \right] \sum_{n \ge 1} \frac{z^n}{n}
      = (n - 1)!
  \end{equation*}

  Las permutaciones
  podemos representarlas como secuencias rotuladas,
  \(\Seq(\mathcal{Z})\),
  con función generatriz exponencial:
  \begin{equation*}
    \frac{1}{1 - z}
  \end{equation*}
  La función generatriz exponencial
  para colecciones de ciclos \(\MSet(\Cyc(\mathcal{Z}))\) es:
  \begin{equation*}
    \exp( -\ln(1 - z))
      = \frac{1}{1 - z}
  \end{equation*}
  Vale decir,
  hay tantas maneras de distribuir \(n\) elementos en ciclos
  como hay permutaciones de esos \(n\) elementos.
  Volveremos a esto en el capítulo~\ref{cha:permutaciones}.

  Sea \(\mathcal{D}\) la clase de los desarreglos.%
    \index{metodo simbolico@método simbólico!desarreglos}
  Como las permutaciones son elementos que se mantienen fijos
  (podemos representarlos como su conjunto)
  y elementos que no están en sus posiciones
  (desarreglos),
  podemos expresar:
  \begin{equation*}
    \Seq(\mathcal{Z})
      = \mathcal{D} \star \Set(\mathcal{Z})
  \end{equation*}
  O sea:
  \begin{equation*}
    \frac{1}{1 - z}
      = \widehat{D}(z) \cdot \mathrm{e}^z
  \end{equation*}
  Como antes.

  Podemos modificar los operadores,
  por ejemplo anotar \(\MSet_{\ge 1}(\mathcal{A})\)
  para conjuntos de al menos un \(\mathcal{A}\),
  con ajustes a sus expansiones
  sugeridas por la demostración del teorema del caso.
  Hay identidades evidentes,
  como
  \(\Seq_{\ge 1}(\mathcal{A})
      = \mathcal{A} \star \Seq(\mathcal{A})\)
  que pueden simplificar los desarrollos.

  Una permutación consta de sus puntos fijos
  y el desarreglo de los restantes.
  Si tiene exactamente \(k\) puntos fijos:
  \begin{equation*}
    \mathcal{D} \star \MSet_k(\mathcal{Z})
  \end{equation*}
  Representando un multiconjunto de \(k\) elementos
  como secuencia obviando el orden,
  esto da la función generatriz exponencial:%
    \index{generatriz!exponencial}
  \begin{equation*}
    \frac{\mathrm{e}^{-z}}{1 - z} \cdot \frac{z^k}{k!}
  \end{equation*}
  Extraemos coeficientes:
  \begin{align*}
    n! \left[ z^n \right] \frac{z^k \mathrm{e}^{-z}}{k! (1 - z)}
      &= \frac{n!}{k!} \left[ z^{n - k} \right]
			 \frac{\mathrm{e}^{-z}}{1 - z} \\
      &= \frac{n!}{k!} \exp \rvert_{n - k} (-1)
  \end{align*}

  Para contar todas las maneras de particionar un conjunto
  tenemos la expresión simbólica:
  \begin{equation*}
    \MSet(\MSet_{\ge 1}(\mathcal{Z}))
  \end{equation*}
  que se traduce directamente en la función generatriz exponencial
  de los \emph{números de Bell}~%
    \cite{bell34:_expon_number}:%
    \index{Bell, numeros de@Bell, números de}
  \begin{equation}
    \label{eq:Bell-egf}
    \widehat{B}(z)
      = \mathrm{e}^{\mathrm{e}^z - 1}
  \end{equation}

  Para obtener una fórmula explícita para \(B_n\)
  salimos del espacio estricto de las series formales.
  Las manipulaciones se justifican ya que las series involucradas
  convergen uniformemente para todo \(z\).
  \begin{align*}
    \widehat{B}(z)
      &= \frac{\mathrm{e}^{\mathrm{e}^z}}{e} \\
      &= \frac{1}{e}
	   \, \sum_{r \ge 0} \frac{\mathrm{e}^{r z}}{r!} \\
      &= \frac{1}{e} \, \sum_{r \ge 0} \frac{1}{r!}
			  \sum_{s \ge 0} \frac{(r z)^s}{s!} \\
      &= \frac{1}{e} \, \sum_{s \ge 0} \frac{z^s}{s!}
			  \sum_{r \ge 0} \frac{r^s}{r!}
  \end{align*}
  El número de Bell \(B_n\)
  es el coeficiente de \(z^n / n!\) en esto:
  \begin{equation}
    \label{eq:Dobinski}
    B_n
      = \frac{1}{e} \, \sum_{r \ge 0} \frac{r^n}{r!}
  \end{equation}
  El resultado~\eqref{eq:Dobinski}
  se conoce como \emph{ecuación de Dobiński}~%
    \cite{dobinski77:_summir_reihe}.%
    \index{Dobinski, ecuacion de@Dobiński, ecuación de}%
    \index{Bell, numeros de@Bell, números de!formula@fórmula|see{Dobiński, ecuación de}}

  Derivando~\eqref{eq:Bell-egf}
  obtenemos la ecuación diferencial,
  cuyo valor inicial resulta directamente de la función generatriz:
  \begin{equation}
    \label{eq:Bell-egf-ode}
    \widehat{B}'(z)
      = \mathrm{e}^z \widehat{B}(z)
      \qquad \widehat{B}(0) = 1
  \end{equation}
  El lado izquierdo es un desplazamiento,
  el derecho corresponde a una convolución binomial:
  \begin{equation}
    \label{eq:Bell-recurrence}
    \index{Bell, numeros de@Bell, números de!recurrencia}
    B_{n + 1}
      = \sum_{0 \le k \le n} \binom{n}{k} B_k
      \qquad B_0 = 1
  \end{equation}
  El lector interesado verificará
  que el truco \(z \mathrm{D} \log\)
  aplicado a~\eqref{eq:Bell-egf}%
    \index{derivada logaritmica@derivada logarítmica}
  lleva a la misma recurrencia~\eqref{eq:Bell-recurrence}.

  Un ejemplo clásico es considerar árboles rotulados,
  formados por un nodo raíz conectados a un conjunto de árboles.%
    \index{arbol con raiz@árbol con raiz!rotulado}
  Esto lleva directamente a:
  \begin{equation*}
    \mathcal{T}
      = \mathcal{Z} \star \MSet(\mathcal{T})
  \end{equation*}
  que se traduce en la ecuación
  para la función generatriz \(\widehat{T}(z)\):
  \begin{equation*}
    \widehat{T}(z)
      = z \mathrm{e}^{\widehat{T}(z)}
  \end{equation*}
  Inversión de Lagrange%
    \index{Lagrange, inversion de@Lagrange, inversión de}
  da directamente la afamada fórmula de Cayley:%
    \index{Cayley, formula de@Cayley, fórmula de}
  \begin{align*}
    \frac{t_n}{n!}
      &= \frac{1}{n} \left[ u^{n - 1} \right] \mathrm{e}^{n u} \\
      &= \frac{1}{n} \cdot \frac{n^{n - 1}}{(n - 1)!} \\
     t_n
       &= n^{n - 1}
  \end{align*}

  \begin{figure}[ht]
    \centering
    \pgfimage{images/function-graph}
    \caption{Una función de \([21]\) a \([21]\)}
    \label{fig:function-graph}
  \end{figure}
  Consideremos una función de \([n] \rightarrow [n]\),
  como por ejemplo
  la graficada en la figura~\ref{fig:function-graph}
  vía indicar por flechas el valor de la función.
  Vemos que los valores se organizan en árboles,
  y a su vez estos en ciclos.
  Esto se describe mediante las ecuaciones simbólicas:
  \begin{align*}
    \mathcal{T}
      &= \mathcal{Z} \star \MSet(\mathcal{T}) \\
    \mathcal{F}
      &= \MSet(\Cyc(\mathcal{T}))
  \end{align*}
  Esto lleva a las ecuaciones funcionales:
  \begin{align*}
    \widehat{T}(z)
      &= z \mathrm{e}^{\widehat{T}(z)} \\
    \widehat{F}(z)
      &= \exp(- \ln (1 - \widehat{T}(z))) \\
      &= \frac{1}{1 - \widehat{T}(z)}
  \end{align*}
  Podemos aplicar inversión de Lagrange,
  teorema~\ref{theo:LIF},
  con \(\phi(u) = \mathrm{e}^u\) y \(f(u) = (1 - u)^{-1}\):%
    \index{Lagrange, inversion de@Lagrange, inversión de}
  \begin{align*}
    \frac{f_n}{n!}
      &= \frac{1}{n}
	   \left[ u^{n - 1} \right]
	      \left( (1 - u)^{-2} \mathrm{e}^{n u} \right) \\
      &= \frac{1}{n}
	   \left[ u^{n - 1} \right]
	     \sum_{k \ge 0} (k + 1) u^k \mathrm{e}^{n u} \\
      &= \frac{1}{n}
	   \sum_{k \ge 0} (k + 1)
	     \left[ u^{n - k - 1} \right] \mathrm{e}^{n u} \\
      &= \frac{1}{n}
	   \sum_{k \ge 0} (k + 1)
	     \frac{n^{n - k - 1}}{(n - k - 1)!} \\
    f_n
      &= (n - 1)! \, \sum_{k \ge 0} (k + 1)
	   \frac{n^{n - k - 1}}{(n - k - 1)!}
  \end{align*}
  No es precisamente una fórmula bonita,
  pero no fue difícil de deducir.

\subsection{Operaciones adicionales}
\label{sec:ms-egf-operaciones-extra}
\index{metodo simbolico@método simbólico!operaciones adicionales}

  Hay operaciones adicionales que son de interés ocasional.
  La definición de la siguiente operación es un tanto bizarra,
  pero pronto la aplicaremos.
  Sean \(\mathcal{A}\) y \(\mathcal{B}\) clases de objetos,
  con \(\alpha \in \mathcal{A}\) y \(\beta \in \mathcal{B}\).
  Definimos el \emph{producto cajonado}%
    \index{metodo simbolico@método simbólico!producto cajonado}
  (en el inglés original
     \emph{\foreignlanguage{english}{boxed product}},
   término bastante poco descriptivo)
  entre \(\alpha\) y \(\beta\),
  que se anota \(\alpha^{\square} \star \beta\),
  combinando \(\alpha\) y \(\beta\)
  y rotulando el resultado de forma que el mínimo rótulo
  se asigna a la parte \(\alpha\).
  Por ejemplo:
  \begin{equation*}
    (2, 1, 3)^\square \star (2, 1)
      = \{(2, 1, 3, 5, 4),
	  (2, 1, 4, 5, 3),
	  (2, 1, 5, 4, 3),
	  (3, 1, 4, 5, 2),
	  (3, 1, 5, 4, 2),
	  (4, 1, 5, 3, 2)
	\}
  \end{equation*}
  Lo que estamos haciendo
  es elegir \(\lvert \alpha \rvert - 1\) rótulos
  de entre \(\lvert \alpha \rvert + \lvert \beta \rvert - 1\).
  Extendemos esta operación a las clases respectivas
  uniendo los conjuntos resultantes.
  Si llamamos \(n = \lvert \alpha \rvert + \lvert \beta \rvert\),
  \(k = \lvert \alpha \rvert\)
  (y por tanto \(n - k = \lvert \beta \rvert\))
  el número de nuevos objetos de tamaño \(n\) creados así es:
  \begin{equation*}
    \sum_{1 \le k \le n}
      \binom{n - 1}{k - 1} a_k b_{n - k}
      = \sum_{0 \le k \le n - 1}
	  \binom{n - 1}{k} a_{k + 1} b_{n - 1 - k}
  \end{equation*}
  Esta es la convolución binomial
  de las secuencias \(\langle a_{n + 1} \rangle_{n \ge 0}\)
  y \(\langle b_n \rangle_{n \ge 0}\),
  pero desplazada en una posición a la derecha.
  Desplazamiento a la derecha es derivar,
  con lo que desplazar a la izquierda es integrar:
  \begin{equation*}
    \mathcal{A}^\square \star \mathcal{B}
      \egf \int_0^z \mathrm{D} \widehat{A}(u)
		  \cdot \widehat{B}(u) \, \mathrm{d} u
  \end{equation*}

  Pongamos en uso esta operación.
  Una \emph{permutación alternante}%
    \index{permutacion@permutación!alternante}
  es tal que:
  \begin{equation*}
    a_1 < a_2 > a_3 < \dotsb
  \end{equation*}
  Consideremos primero las que tienen un número impar de elementos,
  clase \(\mathcal{T}\).
  Si nos fijamos en su máximo,
  vemos que divide la permutación en forma única
  en una permutación alternante de largo impar,
  el máximo,
  y una permutación alternante de largo impar.

  La operación \(a_i \mapsto n + 1 - a_i\)
  es claramente una biyección entre permutaciones,
  y hace que el máximo pase a ser el mínimo.
  Podemos además interpretar la combinación
  \([\square,
       (\square, \dotsc, \square),
       (\square, \dotsc, \square)]\)
  como ubicando el primer elemento entre las secuencias.
  Con estas biyecciones en mente,
  podemos contar la clase \(\mathcal{T}\)
  de permutaciones alternantes de largo impar mediante:
  \begin{equation}
    \label{eq:class-alternating-odd}
    \mathcal{T}
      = \mathcal{Z}
	  + \mathcal{Z}^\square
	      \star (\mathcal{T} \star \mathcal{T})
  \end{equation}
  lo que lleva a la ecuación:
  \begin{equation*}
    \widehat{T}(z)
      = z + \int_0^z u' \cdot \widehat{T}^2(u) \, \mathrm{d} u
  \end{equation*}
  y a la ecuación diferencial:
  \begin{equation*}
    \widehat{T}'(z)
      = 1 + \widehat{T}^2(z)
    \quad \widehat{T}(0) = 0
  \end{equation*}
  de donde:
  \begin{equation}
    \label{eq:egf-alternating-odd}
    \widehat{T}(z)
      = \tan z
  \end{equation}

  Si consideramos las permutaciones alternantes de largo par
  (terminan en una subida),
  clase \(\mathcal{S}\),
  vemos que su máximo las divide en una permutación alternante de largo impar
  que termina en una bajada
  (descrita por \(\mathcal{T}\)),
  el máximo,
  y una permutación alternante de largo par
  que termina en subida
  (descrita por \(\mathcal{S}\)):
  \begin{equation}
    \label{eq:class-alternating-even}
    \mathcal{S}
      = \mathcal{E}
	  + \mathcal{Z}^\square
	      \star (\mathcal{T} \star \mathcal{S})
  \end{equation}
  Esto da lugar a la ecuación:
  \begin{equation*}
    \widehat{S}(z)
      = 1 + \int_0^z u' \cdot \widehat{S}(u) \widehat{T}(u)
	      \, \mathrm{d} u
  \end{equation*}
  con solución:
  \begin{equation}
    \label{eq:egf-alternating-even}
    \widehat{S}(z)
      = \sec z
  \end{equation}
  Obtenemos el curioso resultado de André~%
    \cite{andre81:_sur_permut_alter}
  (ver también Stanley~\cite{stanley09:_survey_alter_permut})
  para los números \(E_n\) que cuentan permutaciones alternantes:
  \begin{equation}
    \label{eq:egf-Euler-numbers}
    \widehat{E}(z)
      = \sec z + \tan z
  \end{equation}
  Los coeficientes son los \emph{números de Euler},%
    \index{Euler, numeros de@Euler, números de}
  quien había llegado a los \(E_{2 n + 1}\) por otro camino.

  Una manera de obtener una recurrencia es observar:
  \begin{align*}
    \widehat{E}'(z)
      &= \sec^2 z + \tan z \sec z \\
    \widehat{E}^2(z)
      &= \tan^2 z + 2 \tan z \sec z + \sec^2 z \\
      &= 2 \tan z \sec z + 2 \sec^2 z - 1 \\
      &= 2 \widehat{E}'(z) - 1
  \end{align*}
  Usando las propiedades de funciones generatrices exponenciales%
    \index{generatriz!exponencial}
  vemos que cumplen la recurrencia:
  \begin{equation}
    \label{eq:recurrence-Euler}
    2 E_{n + 1}
      = [n = 1] + \sum_{0 \le k \le n} \binom{n}{k} E_k E_{n - k}
    \quad E_0 = 1
  \end{equation}

%%% Local Variables:
%%% mode: latex
%%% TeX-master: "clases"
%%% End:

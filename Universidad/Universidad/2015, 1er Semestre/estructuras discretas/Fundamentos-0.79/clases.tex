% clases.tex
%
% Copyright (c) 2008-2014 Horst H. von Brand
% Derechos reservados. Vea COPYRIGHT para detalles

% Documento maestro de las clases de Fundamentos de Informática
% No cambiar formato.

\documentclass[czech, english, german, french, latin, spanish, fleqn]
	      {memoir}
% Fonts
\usepackage{lmodern, fourier}
\usepackage[mathscr]{euscript}
\usepackage[utf8]{inputenc}
\usepackage{icomma}
% Languages
\usepackage{babel}
\usepackage{babelbib, cite}
\setbtxfallbacklanguage{english}
% General LaTeX settings
\usepackage{fixltx2e}
\usepackage{pgf}
\usepackage{subfig}
\usepackage{enumitem}
\usepackage{array}
\usepackage{dcolumn}
\usepackage{multirow}
\usepackage{calc}
\usepackage[autostyle]{csquotes}
\usepackage[punct-after = false]{fnpct}
   % See http://es.wikipedia.org/wiki/Wikipedia_discusi%C3%B3n:Temas_recurrentes/Posici%C3%B3n_de_las_referencias
% Index, glossary
\usepackage{makeidx}
\newcommand*{\textbfhy}[1]{\textbf{\hyperpage{#1}}}
% Math stuff
\usepackage{amsmath, amsthm, amssymb}
\usepackage[all]{xy}
\usepackage{xfrac}
\usepackage{centernot}
\usepackage{scalerel}
% General Computer Science
\usepackage[algochapter, ruled, noline]{algorithm2e}
\usepackage{listings}
\usepackage[basic]{complexity}
\newcommand\cplusplus{C\nolinebreak[4]\hspace{-.03em}%
		      \raisebox{.3ex}{\relsize{-0.75}{\textbf{++}}}}
% microtype should come after fonts and languages, and most packages
\usepackage[babel=true, stretch=10]{microtype}
% Hyperref (PDF output settings), should be last
\usepackage[ocgcolorlinks, urlcolor=blue]{hyperref}

%%%
%%% From Herbert Voß' Mathmode
%%%

\def\mathllap{\mathpalette\mathllapinternal}
\def\mathllapinternal#1#2{%
	\llap{$\mathsurround=0pt#1{#2}$}% $
}
\def\clap#1{\hbox to 0pt{\hss#1\hss}}
\def\mathclap{\mathpalette\mathclapinternal}
\def\mathclapinternal#1#2{%
	\clap{$\mathsurround=0pt#1{#2}$}%
}
\def\mathrlap{\mathpalette\mathrlapinternal}
\def\mathrlapinternal#1#2{%
	\rlap{$\mathsurround=0pt#1{#2}$}% $
}

%%%
%%% PDF information settings
%%%

\hypersetup{pdftitle = {Fundamentos de Informática},
	    pdfauthor ={Horst H. von Brand},
	    unicode, pdfdisplaydoctitle
	   }

%%%
%%% For memoir
%%%

\setsecnumdepth{subsubsection}
\settocdepth{subsection}

\cftsetindents{section}{1.5em}{3.0em}
\cftsetindents{subsection}{4.5em}{3.3em}

\cftsetindents{figure}{0pt}{3.0em}
\cftsetindents{table}{0pt}{3.0em}

% Page layout

\semiisopage
\checkandfixthelayout

% Chapter style

\chapterstyle{southall}


%%%
%%% For listings
%%%

\lstloadlanguages{C, sh}

%%%
%%% AMS-LaTeX theorem stuff
%%%

\theoremstyle{plain}
\newtheorem{theorem}{Teorema}[chapter]
\newtheorem{lemma}[theorem]{Lema}
\newtheorem{corollary}[theorem]{Corolario}
\newtheorem{proposition}{Proposición}[chapter]
\newtheorem{conjecture}{Conjetura}[chapter]
\newtheorem*{axiom}{Axioma}

\theoremstyle{definition}
\newtheorem{definition}{Definición}[chapter]
\newtheorem{example}{Ejemplo}[chapter]

\theoremstyle{remark}
\newtheorem{remark}{Nota}

\newenvironment{solution}[1][Solución]{\begin{trivlist}
\item[\hskip \labelsep {\bfseries #1}]}{\end{trivlist}}

\DeclareMathOperator{\chr}{chr}
\DeclareMathOperator{\lcm}{lcm}
\DeclareMathOperator{\ord}{ord}
\DeclareMathOperator{\sgn}{sgn}
\DeclareMathOperator{\val}{val}

\DeclareMathOperator{\var}{var}

\DeclareMathOperator{\Arg}{Arg}
\DeclareMathOperator{\Log}{Log}
\DeclareMathOperator{\res}{res}
\DeclareMathOperator{\pp}{pp}

\DeclareMathOperator{\inflow}{inflow}
\DeclareMathOperator{\outflow}{outflow}

\DeclareMathOperator{\Seq}{\textsc{Seq}}
\DeclareMathOperator{\Cyc}{\textsc{Cyc}}
\DeclareMathOperator{\Set}{\textsc{Set}}
\DeclareMathOperator{\MSet}{\textsc{MSet}}

\DeclareMathOperator{\ogf}{\stackrel{\text{ogf}}{\longleftrightarrow}}
\DeclareMathOperator{\egf}{\stackrel{\text{egf}}{\longleftrightarrow}}

\newcommand{\cycle}[2]{\genfrac{[}{]}{0pt}{}{#1}{#2}}	 % Stirling 1a especie
\newcommand{\classes}[2]{\genfrac{\{}{\}}{0pt}{}{#1}{#2}} % Stirling 2a especie
\newcommand{\lah}[2]{\genfrac{\lfloor}{\rfloor}{0pt}{}{#1}{#2}}	 % Lah

\newcommand{\multiset}[2]{\left( \!\! \binom{#1}{#2} \!\! \right)}

\newcommand{\Circle}{\raisebox{-1pt}{\scalerel*{\bullet}{\bigodot}}}

%%%
%%% Index, glossary
%%%

\makeindex
\makeglossary

%%%
%%% End of preamble
%%%

\begin{document}
\selectlanguage{spanish}
\bibliographystyle{babplain-fl}

%%%
%%% For listings
%%%

\lstset{language=[ANSI]C,
	basicstyle=\sffamily, commentstyle=\slshape,
	extendedchars, frame=lines, numbers=none,
	float, floatplacement=ht, captionpos=b,
	xleftmargin=1em, xrightmargin=1em
       }
\lstset{literate={á}{{\'a}}1
		 {é}{{\'e}}1
		 {í}{{\'\i}}1
		 {ó}{{\'o}}1
		 {ú}{{\'u}}1
		 {ü}{{\"u}}1
		 {Á}{{\'A}}1
		 {ñ}{{\~n}}1
		 {É}{{\'E}}1
		 {Í}{{\'I}}1
		 {Ó}{{\'O}}1
		 {Ú}{{\'U}}1
		 {Ü}{{\"U}}1
		 {Ñ}{{\~N}}1
		 {¿}{{?`}}1
		 {¡}{{!`}}1
	}

\renewcommand{\lstlistingname}{Listado}
\renewcommand{\lstlistlistingname}{Índice de listados}
\newlistof{lstlisting}{lol}{\lstlistlistingname}
\newlistentry[chapter]{lstlisting}{lol}{0}
\setlength{\cftlstlistingnumwidth}{3em}

%%%
%%% For algorithm2e
%%%

\SetAlgorithmName{Algoritmo}{Algoritmo}{Índice de algoritmos}
\newlistof{listofalgorithms}{loa}{\listalgorithmcfname}
\makeatletter
\renewcommand{\l@algocf}{\@dottedtocline{1}{1em}{3.5em}}
\makeatother

\SetAlCapSty{mdseries}
\SetKw{KwFunction}{function}
\SetKw{KwProcedure}{procedure}
\SetKw{KwVariables}{variables}
\SetKw{KwDownto}{downto}
\SetKwBlock{Loop}{loop}{end}
\SetKw{KwContinue}{continue}
\SetKw{KwBreak}{break}

\frontmatter

%% Bastard title page

\thispagestyle{empty}
\vspace*{\fill}
{\fontsize{24}{26}\sffamily\bfseries
  Fundamentos de Informática
}
\vspace*{\fill}

\cleardoublepage

%% Title page

\thispagestyle{empty}
\vspace*{0.5cm}
\begin{raggedleft}
  {\fontsize{45}{47}\sffamily\bfseries
    Fundamentos\\
    de\\
    Informática
    \par
  }
\end{raggedleft}
\vfill
\vfill
\noindent
{\huge\sffamily
  \href{mailto:vonbrand@inf.utfsm.cl}{Horst H. von Brand}
}
\vfill
\begin{center}
  \pgfimage[width=0.12\textwidth]{images/necronomicon}\\[0.3\baselineskip]

  \today

  {\large\sffamily
   Departamento de Informática\\
   Universidad Técnica Federico Santa María
  }
\end{center}

\clearpage

%% Copyright page
\input{class-version}

\thispagestyle{empty}
\vspace*{2ex}
\begin{small}
  \noindent
  \textcopyright\,2008-\the\year\ Horst H. von Brand \\
  Todos los derechos reservados.\\[1.2em]
  Compuesto por el autor en \LaTeXe{} con Utopia para textos
  y Fourier-GUTenberg para matemáticas.\\[2em]
  Versión \classversion \\[0.7em]
  Se autoriza el uso de esta versión preliminar
  para cualquier fin educacional
  en una institución de enseñanza superior,
  en cuyo caso solo se permite
  el cobro de una tarifa razonable de reproducción.
  Se prohibe todo uso comercial.
\end{small}

\cleardoublepage

%% Dedication page

\thispagestyle{empty}
\vspace*{\fill}
\vspace*{\fill}
\noindent
Agradezco a mi familia,
a quienes he descuidado demasiado
durante el desarrollo del presente texto.

\vspace*{\baselineskip}
\noindent
El Departamento de Informática
de la Universidad Técnica Federico Santa María
provee el ambiente ideal de trabajo.

\vspace*{\baselineskip}
\noindent
Dedico este texto a mis estudiantes,
que sufrieron versiones preliminares del mismo.
Sus sugerencias y preguntas
ayudaron inmensamente a mejorarlo.
\vspace*{\fill}

\cleardoublepage

\tableofcontents

\clearpage

\listoffigures

\clearpage

\listoftables

\clearpage

\lstlistoflistings

\clearpage

\listofalgorithms

\cleardoublepage

\thispagestyle{plain}
% prefacio.tex
%
% Copyright (c) 2012-2014 Horst H. von Brand
% Derechos reservados. Vea COPYRIGHT para detalles

\chapter*{Prefacio}
\label{cha:prefacio}

  Este documento presenta
  (y extiende substancialmente)
  la materia de los ramos \emph{Fundamentos de Informática I},
  \emph{Fundamentos de Informática II}
  y \emph{Estructuras Discretas}
  como dictados durante los años 2009 a~2014
  en la Casa Central de Universidad Técnica Federico Santa María
  por el autor.
  El tratamiento de algunos temas es definitivamente no tradicional,
  y en algunas áreas el autor sigue líneas de razonamiento
  que le parecen interesantes,
  aún si no son directamente parte del curso.
  Así hay material adicional
  a la materia oficial del curso en estos apuntes,
  que no se vio en clase.
  Se notan resultados que están fuera del temario
  donde ayudan a iluminar los temas tratados.
  Se ha hecho el intento de juntar todo al material relevante,
  en forma accesible para el no especialista.
  Intentamos también seguir la exhortación de Knuth de no perderse
  en abstracción excesiva.

  Veremos una colección de temas
  que en conjunto se conocen
  bajo el nombre de \emph{matemáticas discretas}.
  Trataremos de razonamiento matemático,
  y nos ocuparemos más que nada de fenómenos discretos,
  en contraposición de lo continuo que es el ámbito del cálculo.
  Siendo un área menos conocida,
  encontraremos en ella resultados sorprendentes
  y técnicas ingeniosas.
  La importancia en la informática
  es que en computación no se tratan fenómenos continuos.
  El aprender a razonar en el ámbito de objetos discretos,
  y las técnicas que veremos durante el curso de estos ramos,
  serán útiles a la hora de diseñar sistemas
  y evaluar su desempeño.
  El análisis complejo ofrece herramientas poderosas,
  particularmente para derivar estimaciones asintóticas
  de muchas de las cantidades de interés.
  Al no ser materia cubierta
  en el currículum tradicional de las carreras de ingeniería,
  se incluye una breve reseña de los resultados requeridos.

  La razón de fondo de preocuparse de cómo razonar,
  en particular en el ámbito de la informática,
  es que cada día dependemos más de sistemas informáticos,
  que han dejado de ser accesorios
  para transformarse en parte indispensable
  de nuestra vida diaria.
  Fallas en tales sistemas pueden tener consecuencias desastrosas.
  Ejemplos de estas situaciones abundan,
  lamentablemente.
  Particularmente preocupantes son cuando errores lógicos
  son los causantes.
  En este sentido,
  parte del objetivo
  de los presentes ramos en el currículum de informática
  es entrenar en el arte
  de enfrentar problemas en forma estructurada,
  y de reconocer cuándo se tiene una solución correcta.

%%% Local Variables:
%%% mode: latex
%%% TeX-master: t
%%% End:


\mainmatter

% preliminares.tex
%
% Copyright (c) 2009-2014 Horst H. von Brand
% Derechos reservados. Vea COPYRIGHT para detalles

\chapter{Preliminares}
\label{cha:preliminares}

  El capítulo
  resume algunas nociones y notaciones
  que usaremos en el resto del texto.
  No debiera presentar material realmente nuevo para el lector.

  El rango de nociones manejados en matemáticas es muy amplio,
  y la notación bastante variada.
  La notación
  (y la nomenclatura)
  usada por distintos autores no es uniforme,
  por tanto servirá también
  para definir la notación que usaremos.
  Donde hay diversas notaciones en uso más o menos común,
  se anotarán las alternativas.

  Repasaremos lógica matemática,
  conjuntos
  y notaciones para sumatorias y productorias.
  Introducimos potencias factoriales,
  las funciones piso
  (\emph{\foreignlanguage{english}{floor}})
  y techo
  (\emph{\foreignlanguage{english}{ceil}}),
  que se usan frecuentemente en matemáticas discretas.
  Definiremos notaciones asintóticas,
  que usaremos más adelante al discutir algunos algoritmos.

\section{Notación de lógica matemática}
\label{sec:notacion-logica}
\index{logica matematica@lógica matemática!notacion@notación|textbfhy}

  Usaremos la notación de la lógica matemática
  con frecuencia en lo que sigue,
  la introduciremos informalmente acá.

\subsection{Proposiciones}
\label{sec:proposiciones}

  \begin{definition}
    Una \emph{proposición} es una aseveración
    que puede ser verdadera o falsa.
    \index{proposicion@proposición|textbfhy}
    \glossary{Proposición}
	     {En lógica, una aseveración que puede ser verdadera o falsa}
  \end{definition}
  Algunos ejemplos son:
  \begin{proposition}
    \label{prop:llueve}
    Está lloviendo en Valparaíso.
  \end{proposition}
  \begin{proposition}
    \label{prop:Fermat-5}
    El número \(2^{32} + 1\) es primo
  \end{proposition}
  \begin{proposition}
    \label{prop:zeta(3)}
    El número real
    \begin{equation*}
      \zeta(5)
	= \sum_{k \ge 1} k^{-5}
    \end{equation*}
    es irracional.
  \end{proposition}
  La verdad de~\ref{prop:llueve} depende del momento,
  \ref{prop:Fermat-5} es falsa
  (es \(2^{32} + 1 = 641 \cdot 6\,700\,417\)),
  y nadie sabe si~\ref{prop:zeta(3)} es cierta o no.

  Gran parte de nuestro lenguaje no son proposiciones,
  con lo que la lógica no es capaz de representarlo todo.
  Por ejemplo,
  a una pregunta,
  a una orden
  o a una interjección no se le puede asignar verdad o falsedad.

\subsection{Conectivas lógicas}
\label{sec:conectivas-logicas}
\index{logica matematica@lógica matemática!conectivas|textbfhy}
\index{conectivas logicas@conectivas lógicas|see{lógica matemática!conectivas}}

  Comúnmente combinamos proposiciones como
  ``Si llueve, uso paraguas'',
  ``Se puede viajar a Concepción en bus o en avión'',
  ``No traje mis documentos'',
  ``Apruebo el ramo solo si obtengo al menos 35 en la prueba''
  o ``El cartel es rojo y azul''.
  Para precisar el significado de estas combinaciones
  usamos \emph{tablas de verdad}.
    \index{tabla de verdad (logica)@tabla de verdad (lógica)}
  \begin{table}[htbp]
    \centering
    \subfloat[Negación]{
      \begin{tabular}[t]{|c|c|}
	\firsthline
	  \rule[-0.9ex]{0pt}{3ex}%
	\(\pmb{P}\) & \(\boldsymbol{\neg} \pmb{P}\) \\
	\hline
	F & V \\
	V & F \\
	\hline
	\multicolumn{2}{c}{} \\
	\multicolumn{2}{c}{}
      \end{tabular}
    }
    \hspace*{4em}
    \subfloat[Conectivas]{
      \begin{tabular}[t]{|*{6}{c|}}
	\firsthline
	  \rule[-0.9ex]{0pt}{3ex}%
	\(\pmb{P}\) & \(\pmb{Q}\) & \(\pmb{P} \boldsymbol{\vee} \pmb{Q}\)
		    & \(\pmb{P} \boldsymbol{\wedge} \pmb{Q}\)
		    & \(\pmb{P} \boldsymbol{\implies} \pmb{Q}\)
		    & \(\pmb{P} \boldsymbol{\iff} \pmb{Q}\) \\
	\hline
	F & F & F & F & V & V \\
	F & V & V & F & V & F \\
	V & F & V & F & F & F \\
	V & V & V & V & V & V \\
	\lasthline
      \end{tabular}
    }
    \caption{Tablas de verdad para conectivas básicas}
    \label{tab:conectivas-logicas}
    \index{conectivas logicas@conectivas lógicas!tablas de verdad}
  \end{table}
  Al combinar proposiciones \(P\) y \(Q\)
  mediante las operaciones indicadas
  obtenemos el cuadro~\ref{tab:conectivas-logicas},
  donde falso se indica mediante \(F\)
  y verdadero mediante \(V\).
  Se anota \(\vee\) para \emph{o}
  (disyunción),
  usamos \(\wedge\) para \emph{y}
  (conjunción),
  para \emph{si \ldots entonces} escribimos \(\implies\)
  (implicancia),%
    \index{implicancia (logica)@implicancia (lógica)}
  y para expresar \emph{si y solo si} usamos \(\iff\)
  (equivalencia).%
    \index{equivalencia (logica)@equivalencia (lógica)}
  Estas operaciones no son todas necesarias,
  por ejemplo \(P \implies Q\)
  es equivalente a \(\neg P \vee Q\),
  y \(P \iff Q\) no es más que
  \((P \implies Q) \wedge (Q \implies P)\).
  Nótese que la tabla de verdad para \(P \implies Q\)
  coincide con \(\neg Q \implies \neg P\),
  su \emph{contrapositivo}.%
     \index{contrapositivo}

  En castellano
  (como en la mayoría de los lenguajes modernos)
  hay ambigüedad,
  en que ``A o B'' puede entenderse como \emph{A o B, o ambos}
  (inclusivo),%
    \index{o inclusivo}
  o también como \emph{A o B, pero no ambos}
  (exclusivo).%
    \index{o exclusivo}
  Un mito~%
    \cite{sep-disjunction}
  bastante extendido entre los lógicos
  es que en el latín hay dos disyunciones;
  \emph{\foreignlanguage{latin}{vel}},
  que expresa el sentido inclusivo,
  y \emph{\foreignlanguage{latin}{aut}},
  que expresa el exclusivo.
  En realidad,
  ambas son ambiguas como en los lenguajes modernos.
  La noción empleada en lógica matemática es inclusiva,
  y nuestra notación
  sugiere el latín \emph{\foreignlanguage{latin}{vel}}.

  Suele decirse ``\(A\) es suficiente para \(B\)''%
    \index{implicancia (logica)@implicancia (lógica)}
  si \(A \implies B\)
  (si la implicancia es cierta,
   saber que \(A\) es cierto asegura que \(B\) es cierto,
   si \(A\) es falso \(B\) puede ser cierto como no serlo),
  y que ``\(A\) es necesario para \(B\)''%
    \index{implicancia (logica)@implicancia (lógica)}
  o ``\(A\) solo si \(B\)''%
    \index{solo si (logica)@solo si (lógica)|see{implicancia (lógica)}}
  si \(B \implies A\)
  (saber que \(A\) es cierto permite concluir que debe serlo \(B\)).
  Así suele decirse ``\(A\) es necesario y suficiente para \(B\)''
  o ``\(A\) si y solo si \(B\)''%
    \index{si y solo si (logica)@si y solo si (lógica)}
  para expresar \(A \iff B\).
  Note que hay diversas formas de expresar lo mismo,
  y es fácil confundirse.

  Al decir ``Hoy almorzaré bife a lo pobre o pescado frito con ensalada''
  se entiende que es uno o el otro, no ambos;
  en ``Leo novelas policiales o históricas''
  se subentiende que son ambas;
  mientras ``Sus mascotas son perros o gatos''
  no queda claro si es solo uno o el otro,
  o posiblemente ambas.
  De la misma forma,
  ``Si llueve, llevo paraguas''
  puede entenderse
  como que llevo paraguas exclusivamente cuando llueve,
  nuestra conectiva implica incluye la posibilidad de llevarlo
  incluso si no llueve.
  La aseveración sobre hábitos de lectura
  se interpretaría como que no leo nada más
  que novelas policiales o históricas,
  nuestra formalización no es así de excluyente.
  Hay cosas que se subentienden u omiten
  en el lenguaje cotidiano,
  si se quiere lograr precisión eso no es aceptable.

  \begin{table}[ht]
    \centering
    \begin{tabular}{|*{2}{>{\(}c<{\)}}|l|}
      \hline
      \multicolumn{2}{|c|}{\rule[-0.7ex]{0pt}{3ex}\textbf{Leyes}} &
	\multicolumn{1}{c|}{\textbf{Nombre(s)}} \\
      \hline
      \multicolumn{2}{|>{\(}c<{\)}|}{\rule[-0.7ex]{0pt}{3ex}%
	\neg \neg P \equiv P
      } &
	Doble negación \\
      P \vee \neg P \equiv V &
	P \wedge \neg P \equiv F &
	Medio excluido/contradicción \\
      P \vee F \equiv P &
	P \wedge V \equiv P &
	Identidad \\
      P \vee V \equiv V &
	P \wedge F \equiv F &
	Dominación \\
      P \vee P \equiv P &
	P \wedge P \equiv P &
	Idempotencia \\
      \neg (P \vee Q) \equiv \neg P \wedge \neg Q   &
	\neg (P \wedge Q) \equiv \neg P \vee \neg Q &
	de Morgan \\
      P \vee Q \equiv Q \vee P &
	P \wedge Q \equiv Q \wedge P &
	Conmutatividad \\
      (P \vee Q) \vee R \equiv P \vee (Q \vee R) &
	(P \wedge Q) \wedge R \equiv P \wedge (Q \wedge R) &
	Asociatividad \\
      P \vee (Q \wedge R) \equiv (P \vee Q) \wedge (P \vee R) &
	P \wedge (Q \vee R) \equiv (P \wedge Q) \vee (P \wedge R) &
	Distributividad \\
      P \vee (P \wedge Q) \equiv P &
	P \wedge (P \vee Q) \equiv P &
	Absorción \\
      \hline
    \end{tabular}
    \caption{Identidades adicionales}
    \label{tab:logical-identities}
    \index{logica@lógica!identidad}
  \end{table}
  Identidades útiles
  reseña el cuadro~\ref{tab:logical-identities},
  propiedades importantes
  da el cuadro~\ref{tab:propiedades-importantes-logica}.
  Nótese que la segunda columna
  del cuadro~\ref{tab:logical-identities}
  se obtiene de la primera
  intercambiando \(\wedge\) con \(\vee\) y \(V\) con \(F\).
  Esto es lo que se conoce como \emph{dualidad},%
    \index{dualidad (logica)@dualidad (lógica)}
  obtenemos dos equivalencias por el precio de una.
  \begin{table}[htbp]
    \centering
    \begin{tabular}{|l|>{\(}l<{\)}|}
      \hline
      \multicolumn{1}{|c|}{\rule[-0.7ex]{0pt}{3ex}\textbf{Nombre}} &
	\multicolumn{1}{c|}{\textbf{Propiedad}} \\
      \hline\rule[-0.7ex]{0pt}{3ex}%
      Definición de implicancia
	& P \implies Q \equiv \neg P \vee Q \\
      Definición de si y solo si
	& P \iff Q
	      \equiv (P \implies Q) \wedge (Q \implies P) \\
      Contrapositivo
	& P \implies Q \equiv \neg Q \implies \neg P \\
      Reducción al absurdo
	& P \implies Q \equiv P \wedge \neg Q \implies F \\
      Transitividad de implicancia
	& (P \implies Q) \wedge (Q \implies R)
	      \implies (P \implies R) \\
      \hline
    \end{tabular}
    \caption{Propiedades importantes de las operaciones lógicas}
    \label{tab:propiedades-importantes-logica}
    \index{logica@lógica!identidad}
  \end{table}
  A \(Q \implies P\)
  se le llama el \emph{recíproco}%
    \index{reciproco (logica)@recíproco (lógica)}
  de \(P \implies Q\)
  (en inglés,
   \emph{\foreignlanguage{english}{converse}}).%
     \index{converse@\emph{\foreignlanguage{english}{converse}}|see{recíproco}}
  No hay relación
  entre los valores de verdad de una implicancia y su recíproca,
  como es fácil demostrar.
  Las identidades de los cuadros~\ref{tab:logical-identities}
  y~\ref{tab:propiedades-importantes-logica} debieran memorizarse.

\subsection{Lógica de predicados}
\label{sec:predicados}
\index{logica de predicados@lógica de predicados|textbfhy}

  Queremos razonar sobre individuos de un conjunto,
  no solo de proposiciones que son verdaderas o falsas.
  Usaremos la convención
  que letras mayúsculas representan variables,
  y letras minúsculas constantes.

  \begin{definition}
    Un \emph{predicado}
    es una función cuyo valor es verdadero o falso.
    \index{predicado (logica)@predicado (lógica)|textbfhy}%
    \index{logica@lógica!predicado}
  \end{definition}

  Usaremos la convención que letras mayúsculas denotan constantes,
  y letras minúsculas variables.%
    \index{logica@lógica!convencion variables@convención variables}
  Por ejemplo,
  tenemos el predicado \(\text{prime}(x)\)
  que es verdadero exactamente cuando \(x\) es un número primo.
  Así,
  tanto \(\text{prime}(2)\)
  como \(\text{prime}(2^{16} + 1)\) son verdaderos,
  mientras \(\text{prime}(2\,743)\) es falso
  (\(2\,743 = 13 \cdot 211\)).

  Podemos considerar la expresión:
  \begin{equation*}
    a = 3 b
  \end{equation*}
  como un predicado de dos argumentos
  (\(a\) y \(b\))
  que es verdadero exactamente cuando \(a = 3 b\).
  Asimismo,
  \begin{equation*}
    17 = 3 x
  \end{equation*}
  es un predicado que es falso para todos los naturales \(x\)
  y es cierto para el racional \(17 / 3\).

  Usamos las conectivas lógicas para combinar predicados,%
    \index{logica@lógica!conectiva}
  construyendo predicados más complejos.
  Por ejemplo,
  \begin{equation*}
    a x^2 + b x + c = 0
       \implies x = \frac{-b \pm \sqrt{b^2 - 4 a c}}{2a}
  \end{equation*}
  resulta ser cierto
  si consideramos \(a\), \(b\), \(c\) y \(x\)
  como números complejos.
  Está claro que es importante indicar
  de qué conjunto toman valores las variables indicados.

\subsection{Cuantificadores}
\label{sec:cuantificadores}
\index{cuantificadores (logica)@cuantificadores (lógica)|textbfhy}
\index{logica@lógica!cuantificador}

  Desearemos expresar que un predicado es cierto
  para todos los posibles valores de las variables involucradas,
  o que es cierto al menos para uno de ellos.
  Esto se expresa mediante \emph{cuantificadores}.
  Los que usaremos son \(\forall\) (para todo)
  y \(\exists\) (existe).
  Una proposición expresada concisamente es
  \begin{equation*}
    \forall n \colon \text{prime}(n^2 + n + 41)
  \end{equation*}
  donde \(\text{prime}(x)\) es el predicado discutido antes.
  La variable usada queda atada al cuantificador
  y no tiene significado fuera.
  Explicitando el conjunto al que pertenecen las variables,
  escribiremos por ejemplo:
  \begin{equation}
    \label{eq:primos?}
    \forall n \in \mathbb{N} \colon \text{prime}(n^2 + n + 41)
  \end{equation}
  Veamos diferentes valores de \(n\),
  como resume el cuadro~\ref{tab:polinomio-primo}.
  \begin{table}[htbp]
    \centering
    \begin{tabular}{|>{\(}r<{\)}|>{\(}r<{\)}|c|}
      \hline
      \multicolumn{1}{|c|}{\rule[-0.7ex]{0pt}{3ex}\(\boldsymbol{n}\)} &
	\multicolumn{1}{c|}{\textbf{Valor}} &
	\multicolumn{1}{c|}{\textbf{¿Primo?}} \\
      \hline
	\rule[-0.7ex]{0pt}{3ex}%
       0	 &     41 & Si \\
       1	 &     43 & Si \\
       2	 &     47 & Si \\
       \multicolumn{1}{|c|}{\phantom{0}\vdots} &
	 \multicolumn{1}{c|}{\phantom{00\;}\vdots} &
	 \multicolumn{1}{c|}{\vdots} \\
      39	 & 1\,601 & Si \\
      \hline
    \end{tabular}
    \caption{Valores de $n^2 + n + 41$ para $1 \le n \le 39$}
    \label{tab:polinomio-primo}
  \end{table}
  Hasta \(n = 39\) vamos bien.
  Pero resulta \(40^2 + 40 + 41 = 1\,681 = 41 \cdot 41\),
  con lo que la proposición~\ref{eq:primos?} es falsa.
  Si nos preguntamos que valores de \(n\) dan un valor compuesto,
  caemos en cuenta que para \(n = 41\)
  todos los términos son divisibles por \(41\).
  En efecto,
  se reduce a \(41^2\),
  que no es primo.
  Al polinomio~\eqref{eq:primos?}
  se le conoce como polinomio de Euler,%
    \index{Euler, polinomio de}
  hay una variedad de polinomios
  que dan primos para sus primeros valores.

\section{Conjuntos}
\label{sec:conjuntos}
\index{conjunto|textbfhy}

  Una de las nociones más importantes en las matemáticas actuales
  es la de conjunto.
  En términos simples,
  un conjunto es una colección de elementos
  bien definida,
  vale decir,
  para cada elemento se puede determinar claramente
  si pertenece o no al conjunto.
  Para indicar que el elemento \(a\)
  pertenece al conjunto \(\mathcal{A}\),
  se escribe \(a \in \mathcal{A}\),%
    \index{\(\in\) (pertenece)|textbfhy}
  para indicar que no pertenece se anota \(a \notin \mathcal{A}\).%
    \index{\(\not\in\) (no pertenece)|textbfhy}
  A veces resulta más cómodo escribir estas relaciones al revés,
  o sea anotar \(\mathcal{A} \ni a\)%
    \index{\(\ni\) (contiene)|textbfhy}
  o \(\mathcal{A} \centernot\ni a\),
    \index{\(\centernot\ni\) (no contiene)|textbfhy}
  respectivamente.

  Ciertos conjuntos tienen notación especial por su frecuente uso.
  El conjunto especial que no tiene elementos
  se llama el \emph{conjunto vacío} y se anota \(\varnothing\).%
    \index{\(\varnothing\) (conjunto vacio)@\(\varnothing\) (conjunto vacío)|textbfhy}%
    \index{conjunto!vacio@vacío}
  Otros son el conjunto de los \emph{números naturales},
  \(\mathbb{N} = \{1, 2, 3, \dotsc\}\);%
    \index{N (números naturales)@\(\mathbb{N}\) (números naturales)|textbfhy}
  el de los \emph{números enteros},
  \(\mathbb{Z} = \{\dotsc, - 3, -2, -1, 0, 1, 2, 3, \dotsc\}\);%
    \index{Z (numeros enteros)@\(\mathbb{Z}\) (números enteros)|textbfhy}
  los \emph{números racionales},
  \(\mathbb{Q}\);%
    \index{Q (números racionales)@\(\mathbb{Q}\) (números racionales)|textbfhy}
  los \emph{reales},
  \(\mathbb{R}\);%
    \index{R (números reales)@\(\mathbb{R}\) (números reales)|textbfhy}
  y los \emph{complejos},
  \(\mathbb{C}\).
    \index{N (números naturales)@\(\mathbb{N}\) (números naturales)|textbfhy}
  El conjunto de los naturales más el cero lo anotaremos
  \(\mathbb{N}_0\).%
    \index{N 0 (números naturales y cero)@\(\mathbb{N}_0\) (números naturales y cero)|textbfhy}
  Anotaremos \(\mathbb{Q}^+\)
  para los números racionales mayores a cero,%
    \index{Q + números racionales positivos)@\(\mathbb{Q}^+\) (números racionales positivos)|textbfhy}
  y similarmente \(\mathbb{R}^+\) para los reales.%
    \index{R + números reales positivos)@\(\mathbb{R}^+\) (números reales positivos)|textbfhy}

  Una manera de describir un conjunto es por \emph{extensión},%
    \index{conjunto!descripcion por extension@descripción por extensión}
  nombrando cada uno de sus elementos:
  \begin{equation*}
    \mathcal{A}
      = \{1, 2, 3, 4, 5\}
  \end{equation*}
  Esto resulta incómodo para conjuntos grandes,
  por lo que suele usarse alguna notación como la siguiente
  para no dar explícitamente todos los elementos:
  \begin{equation*}
    \mathcal{B}
      = \{1, 2, \dotsc, 128\}
  \end{equation*}
  Esto también se usa si estamos frente a conjuntos infinitos:
  \begin{equation*}
    \mathcal{C}
      = \{1, 2, 4, 8, \dotsc\}
  \end{equation*}
  El problema es que no queda claro exactamente cuáles
  son los elementos a incluir.
  Una forma alternativa
  de describir conjuntos es por \emph{intención},%
    \index{conjunto!descripcion por intencion@descripción por intención}
  describiendo de alguna forma los elementos que lo componen.
  Haciendo referencia a los conjuntos definidos antes:
  \begin{align*}
    \mathcal{B}
      &= \{x \colon 1 \le x \le 128\}
	 \qquad \text{(aunque tal vez es
			\(\mathcal{B}
			    =\{2^k \colon 0 \le k < 8\}\))} \\
    \mathcal{C}
      &= \{2^k \colon k \ge 0\}
  \end{align*}

  Es importante tener presente
  que un elemento pertenece o no al conjunto,
  no puede pertenecer más de una vez a él.
  Otro detalle es que no hay ningún orden
  entre los elementos de un conjunto.
  El conjunto \(\{1, 2, 3, 4, 5\}\) es exactamente el mismo que
  \(\{4, 2, 1, 5, 3\}\),
  o que \(\{2, 1, 2, 5, 4, 2, 3\}\).

  Operaciones comunes entre conjuntos%
    \index{conjunto!operaciones|textbfhy}
  se describen mediante diagramas de Venn
  en la figura~\ref{fig:Venn}.%
    \index{Venn, diagrama de|textbfhy}
  \begin{figure}[htbp]
    \centering
    \subfloat[Unión]{\pgfimage{images/union}}
    \qquad
    \subfloat[Intersección]{\pgfimage{images/interseccion}}

    \subfloat[Diferencia]{\pgfimage{images/diferencia}}
    \qquad
    \subfloat[Diferencia simétrica]{\pgfimage{images/diferencia-simetrica}}
    \\
    \subfloat[Complemento]{\pgfimage{images/complemento}}
    \caption{Diagramas de Venn para operaciones entre conjuntos}
    \label{fig:Venn}
  \end{figure}
  \begin{description}
  \item[Unión:]
    \index{conjunto!union@unión|textbfhy}
    Los elementos que pertenecen a \(\mathcal{A}\)
    o a \(\mathcal{B}\) o a ambos
    se anota \(\mathcal{A} \cup \mathcal{B}\).
    Es fácil ver
      que \(\mathcal{A} \cup \mathcal{B}
	      = \mathcal{B} \cup \mathcal{A}\).
    Como \((\mathcal{A} \cup \mathcal{B}) \cup \mathcal{C}
	      = \mathcal{A} \cup (\mathcal{B} \cup \mathcal{C})\),
    se suele omitir el paréntesis en tales expresiones.%
      \index{operacion@operación!asociativa}
  \item[Intersección:]
    \index{conjunto!interseccion@intersección|textbfhy}
    Aquellos elementos
    que pertenecen a \(\mathcal{A}\) y a \(\mathcal{B}\)
    se anotan \(\mathcal{A} \cap \mathcal{B}\).
    Es \(\mathcal{A} \cap \mathcal{B}
	   = \mathcal{B} \cap \mathcal{A}\).
    Nuevamente,
    \((\mathcal{A} \cap \mathcal{B}) \cap \mathcal{C}
	 = \mathcal{A} \cap (\mathcal{B} \cap \mathcal{C})\),
    y convencionalmente se omiten los paréntesis.%
      \index{operacion@operación!asociativa}
    Si tienen intersección vacía
    (o sea,
    \(\mathcal{A} \cap \mathcal{B} = \varnothing\)),
    se dice que son \emph{disjuntos}.%
      \index{conjunto!disjunto|textbfhy}
  \item[Resta:]
    \index{conjunto!resta|textbfhy}
    Los elementos de \(\mathcal{A}\)
    que no pertenecen a \(\mathcal{B}\)
    se anotan \(\mathcal{A} \smallsetminus \mathcal{B}\).
  \item[Diferencia simétrica:]
    \index{conjunto!diferencia simetrica@diferencia simétrica|textbfhy}
    Los que pertenecen a \(\mathcal{A}\) o a \(\mathcal{B}\),
    pero no a ambos,
    se escriben \(\mathcal{A} \vartriangle \mathcal{B}\).
    También tenemos
      \(\mathcal{A} \vartriangle \mathcal{B}
	  = \mathcal{B} \vartriangle \mathcal{A}\).
    Considerando las diferentes áreas del diagrama de Venn
    para \((\mathcal{A} \vartriangle \mathcal{B}) \vartriangle \mathcal{C}\)
    vemos que incluye los elementos de exactamente uno o tres de los conjuntos,
    con lo que
    \((\mathcal{A} \vartriangle \mathcal{B}) \vartriangle \mathcal{C}
         = \mathcal{A} \vartriangle (\mathcal{B}) \vartriangle \mathcal{C})\).%
      \index{operacion@operación!asociativa}

  \item[Complemento:]
    \index{conjunto!complemento|textbfhy}
      Si estamos considerando
      un conjunto de elementos como ámbito de discusión,
      lo tomamos como \emph{universo}%
        \index{conjunto!universo}
      (comúnmente anotado \(\mathcal{U}\)),
      y tenemos el \emph{complemento} del conjunto \(\mathcal{A}\)
      como aquellos elementos de \(\mathcal{U}\)
      que no pertenecen a \(\mathcal{A}\).
      Esto lo anotaremos \(\overline{\mathcal{A}}\).
  \end{description}

  Resulta importante comparar conjuntos.
  \begin{description}
  \item[Igualdad:]
    \index{conjunto!igualdad|textbfhy}
    Dos conjuntos son \emph{iguales}
    cuando tienen los mismos elementos.
    Esto se anota \(\mathcal{A} = \mathcal{B}\).
  \item[Subconjunto:]
    \index{conjunto!subconjunto|textbfhy}
    Si todos los elementos de \(\mathcal{A}\)
    pertenecen a \(\mathcal{B}\)
    se anota \(\mathcal{A} \subseteq \mathcal{B}\),
    y se dice que \(A\) es \emph{subconjunto} de \(\mathcal{B}\).
    Si queremos excluir
    la posibilidad \(\mathcal{A} = \mathcal{B}\),
    escribimos \(\mathcal{A} \subset \mathcal{B}\)
    (a veces se le llama \emph{subconjunto propio}).
    Para indicar
    que \(\mathcal{A}\) no es subconjunto de \(\mathcal{B}\)
    se anota \(\mathcal{A} \centernot\subseteq \mathcal{B}\).
    Nótese
    que algunos usan la notación \(\mathcal{A} \subset \mathcal{B}\)
    para lo que anotamos \(\mathcal{A} \subseteq \mathcal{B}\),
    y usan \(\mathcal{A} \subsetneq \mathcal{B}\)
    para lo que llamamos \(\mathcal{A} \subset \mathcal{B}\).
  \item[Superconjunto:]
    \index{conjunto!superconjunto|textbfhy}
    Si \(\mathcal{A} \subseteq \mathcal{B}\),
    también anotamos \(\mathcal{B} \supseteq \mathcal{A}\),
    y similarmente si \(\mathcal{A} \subset \mathcal{B}\)
    anotamos también \(\mathcal{B} \supset \mathcal{A}\).
    Decimos que \(\mathcal{B}\)
    es \emph{superconjunto} de \(\mathcal{A}\).
  \end{description}

  Las operaciones entre conjuntos
  pueden expresarse usando notación lógica:%
    \index{conjunto!operaciones!notacion logica@notación lógica|textbfhy}
  \begin{align}
    \mathcal{A} \cup \mathcal{B}
      &= \{x \colon x \in \mathcal{A} \vee x \in \mathcal{B}\}
	  \label{eq:set-union} \\
    \mathcal{A} \cap \mathcal{B}
      &= \{x \colon x \in \mathcal{A} \wedge x \in \mathcal{B}\}
	  \label{eq:set-intersection} \\
    \mathcal{A} \subseteq \mathcal{B}
      &\equiv x \in \mathcal{A} \implies x \in \mathcal{B}
	  \label{eq:set-subset} \\
    \mathcal{A} = \mathcal{B}
      &\equiv x \in \mathcal{B} \iff x \in \mathcal{A}
	  \label{eq:set-equals}
  \end{align}
  La forma de la unión y la intersección
  sugieren la forma de la conectiva lógica
  en~\eqref{eq:set-union} y en~\eqref{eq:set-intersection}.
  La dirección de la implicancia en~\eqref{eq:set-subset}
  puede recordarse como ambas apuntando en la misma dirección.

  Podemos definir diferencia y diferencia simétrica
  en términos de las operaciones tradicionales:
  \begin{align*}
    \mathcal{A} \smallsetminus \mathcal{B}
      &= \mathcal{A} \cap \overline{\mathcal{B}} \\
    \mathcal{A} \vartriangle \mathcal{B}
      &= (\mathcal{A} \cup \mathcal{B})
	    \smallsetminus (\mathcal{A} \cap \mathcal{B}) \\
      &= (\mathcal{A} \cap \overline{\mathcal{B}})
	   \cup (\overline{\mathcal{A}} \cap \mathcal{B})
  \end{align*}
  Algunas propiedades simples de las anteriores
  son las dadas en el cuadro~\ref{tab:properties-set-operations},
  donde \(\mathcal{A}\), \(\mathcal{B}\) y \(\mathcal{C}\)
  representan conjuntos cualquiera.
  Compárese con el cuadro~\ref{tab:logical-identities}.
  \begin{table}[ht]
    \centering
    \begin{tabular}{|*{2}{>{\(}c<{\)}}|l|}
      \hline
      \multicolumn{2}{|c|}{\rule[-0.7ex]{0pt}{3ex}\textbf{Leyes}} &
	\multicolumn{1}{c|}{\textbf{Nombre(s)}} \\
      \hline
	\multicolumn{2}{|>{\(}c<{\)}|}{\rule[-0.7ex]{0pt}{4ex}%
	  \overline{\overline{\mathcal{A}}} = \mathcal{A}
	} &
	  Doble complemento \\
	\mathcal{A} \cup \overline{\mathcal{A}} = \mathcal{U} &
	  \mathcal{A} \cap \overline{\mathcal{A}} = \varnothing &
	  Complemento \\
	\mathcal{A} \cup \varnothing = \mathcal{A} &
	  \mathcal{A} \cap \mathcal{U} = \mathcal{A} &
	  Identidad \\
	\mathcal{A} \cup \varnothing = \mathcal{A} &
	  \mathcal{A} \cap \mathcal{U} = \mathcal{A} &
	  Dominación \\
	\overline{\mathcal{A} \cup \mathcal{B}}
	     = \overline{\mathcal{A}} \cap \overline{\mathcal{B}} &
	  \overline{\mathcal{A} \cap \mathcal{B}}
	       = \overline{\mathcal{A}} \cup \overline{\mathcal{B}} &
	  de Morgan \\
	\mathcal{A} \cup \mathcal{B} = \mathcal{B} \cup \mathcal{A} &
	  \mathcal{A} \cap \mathcal{B} = \mathcal{B} \cap \mathcal{A} &
	  Conmutatividad \\
	(\mathcal{A} \cup \mathcal{B}) \cup \mathcal{C} &
	  (\mathcal{A} \cap \mathcal{B}) \cap \mathcal{C} &
	  Asociatividad \\
	\mathcal{A} \cup (\mathcal{B} \cap \mathcal{C})
	     = (\mathcal{A} \cup \mathcal{B})
		 \cap (\mathcal{A} \cup \mathcal{C}) &
	  \mathcal{A} \cap (\mathcal{B} \cup \mathcal{C})
	       = (\mathcal{A} \cap \mathcal{B})
		   \cup (\mathcal{A} \cap \mathcal{C}) &
	  Distributividad \\
       \mathcal{A} \cup (\mathcal{A} \cap \mathcal{B}) = \mathcal{A} &
	  \mathcal{A} \cap (\mathcal{A} \cup \mathcal{B}) = \mathcal{A} &
	  Absorción \\
       \hline
    \end{tabular}
    \caption{Propiedades de las operaciones entre conjuntos}
    \label{tab:properties-set-operations}
    \index{conjunto!operaciones!propiedades}
  \end{table}
  También tenemos que si \(\mathcal{A} \subset \mathcal{B}\)
  y \(\mathcal{B} \subset \mathcal{C}\)
  entonces \(\mathcal{A} \subset \mathcal{C}\).
  Una manera de demostrar igualdad entre conjuntos%
    \index{conjunto!igualdad!demostrar}
  es usar el hecho que si \(\mathcal{A} \subseteq \mathcal{B}\)
  y \(\mathcal{B} \subseteq \mathcal{A}\),
  entonces \(\mathcal{A} = \mathcal{B}\).

  Otra noción importante es el número de elementos del conjunto,
  su \emph{cardinalidad}.%
    \index{conjunto!cardinalidad|textbfhy}
  La cardinalidad del conjunto \(\mathcal{A}\)
  se anotará \(\lvert \mathcal{A} \rvert\).
  Para conjuntos finitos,
  es simplemente el número de elementos del conjunto.
  Si \(\mathcal{A} = \{1, 2, 4, 8, 16, 32\}\),
  tenemos \(\lvert \mathcal{A} \rvert = 6\).
  Sólo para \(\varnothing\)
  se cumple \(\lvert \varnothing \rvert = 0\).
  También tenemos que si \(\mathcal{A} \subseteq \mathcal{B}\)
  entonces
    \(\lvert \mathcal{A} \rvert \le \lvert \mathcal{B} \rvert\)\@.
  Más adelante
  (capítulo~\ref{cha:numerabilidad})
  consideraremos conjuntos infinitos también.

  Es común referirse a rangos de elementos de algún conjunto ordenado,
  típicamente \(\mathbb{R}\) y ocasionalmente \(\mathbb{N}\).%
    \index{conjunto!rangos (notacion)@rangos (notación)|textbfhy}
  Para ello usaremos las notaciones siguientes.
  \begin{align*}
    (a, b)
      &= \{x \colon a < x < b\} \\
    [a, b)
      &= \{x \colon a \le x < b\} \\
    (a, b]
      &= \{x \colon a < x \le b\} \\
    [a, b]
      &= \{x \colon a \le x \le b\}
  \end{align*}
  Si un extremo es abierto
  (no incluye el elemento del caso)
  usamos paréntesis,
  en caso que el extremo es cerrado
  (el elemento indicado está incluido)
  usamos corchetes
  (paréntesis cuadrados).
  En el caso especial de un rango de los primeros naturales
  anotaremos:
  \begin{equation*}
    [1, n] = [n]
  \end{equation*}

  Al conjunto de todos los subconjuntos
  de algún conjunto \(\mathcal{A}\)
  (el \emph{conjunto potencia} de \(\mathcal{A}\))%
    \index{conjunto!conjunto potencia|textbfhy}
  lo anotaremos \(2^{\mathcal{A}}\).
  También es común notación como \(\mathfrak{P}(\mathcal{A})\).
  Por ejemplo:
  \begin{equation*}
    2^{\{1, 2, 3\}}
      = \{\varnothing,
	  \{1\}, \{2\}, \{3\},
	  \{1, 2\}, \{1, 3\}, \{2, 3\},
	  \{1, 2, 3\}\}
  \end{equation*}

\section{Tuplas}
\label{sec:tuplas}
\index{tupla|textbfhy}

  A objetos que combinan varios elementos
  se les llama \emph{tuplas}.
  Escribimos las componentes de la tupla separados por comas
  y la tupla completa encerrada entre paréntesis.
  El caso más común es el de pares de elementos,
  como \((1, 2)\),
  pero también podemos tener tríos como \((x, y, z)\),
  y así sucesivamente.
  Es lamentable que esta notación para pares
  pueda confundirse con rangos abiertos,
  como descritos antes.
  Deberá quedar claro del contexto lo que se está discutiendo.

  El orden importa,
  el par \((1, 2)\) no es lo mismo que \((2, 1)\).
  Pueden repetirse elementos,
  \((5, 2, 1, 5)\) es una tupla válida.
  Nada dice que los elementos deban ser tomados del mismo conjunto,
  podemos tener pares formados por un polinomio
  y un número complejo,
  como \((3x^2 - 5x + 2, 3 + 5 \mathrm{i})\).

  Una manera de construir tuplas
  es mediante producto cartesiano entre conjuntos:%
    \index{conjunto!producto cartesiano|textbfhy}
  \begin{equation*}
    \mathcal{A} \times \mathcal{B}
      = \{(a, b) \colon a \in \mathcal{A} \wedge b \in \mathcal{B}\}
  \end{equation*}
  Podemos construir tríos
  vía \(\mathcal{A} \times \mathcal{B} \times \mathcal{C}\)
  y así sucesivamente,
  donde interpretamos
    \(\mathcal{A} \times \mathcal{B} \times \mathcal{C}\)
  como \((\mathcal{A} \times \mathcal{B}) \times \mathcal{C}\),
  y la tupla \((a, b, c)\)
  como una abreviatura del par \(((a, b), c)\),
  y de forma similar para largos mayores
  (llamaremos \emph{largo} al número de elementos de la tupla).%
     \index{tupla!largo|textbfhy}
  Usaremos potencias para indicar tuplas de un largo dado
  tomando elementos de un conjunto.
  Formalmente,
  para cualquier conjunto \(\mathcal{A}\) definimos:
  \begin{equation}
    \label{eq:A^n}
    \begin{split}
      \mathcal{A}^1
	&= \mathcal{A} \\
      \mathcal{A}^{n + 1}
	&= \mathcal{A}^n \times \mathcal{A}
	     \quad \text{si \(n \ge 1\)}
    \end{split}
  \end{equation}

  Tuplas de elementos de un mismo conjunto,
  particularmente si son de largos posiblemente diferentes,
  se llaman \emph{secuencias}.%
    \index{secuencia|textbfhy}
  En una secuencia interesa el largo
  (que para la secuencia \(\sigma\)
   anotaremos \(\lvert \sigma \rvert\))%
    \index{secuencia!largo (notacion)@largo (notación)|textbfhy}
  y el elemento en cada posición.
  Hay una única secuencia de largo 0,
  que generalmente llamaremos \(\epsilon\).%
    \index{\(\epsilon\) (secuencia vacia)@\(\epsilon\) (secuencia vacía)|textbfhy}
  Si las secuencias son finitas y de elementos atómicos
  (\emph{símbolos})
  se les suele llamar \emph{palabras}
  o \emph{\foreignlanguage{english}{strings}}
    \index{string@\emph{\foreignlanguage{english}{string}}|see{secuencia}}
  (por el término en inglés).
  Más adelante trataremos extensamente con secuencias infinitas.

\section{Multiconjuntos}
\label{sec:preliminares-multiconjuntos}
\index{multiconjunto|textbfhy}

  Un elemento puede pertenecer varias veces
  a un \emph{multiconjunto}.
  Anotamos las repeticiones mediante exponentes.
  Un ejemplo
  es \(\{a, b, a, c, b, a\}
	 = \{a, a, a, b, b, c\}
	 = \{a^3, b^2, c^1\}\).

  Las operaciones entre multiconjuntos
  son afines a las de conjuntos:%
    \index{multiconjunto!operaciones|seealso{conjunto!operaciones}}
  La unión es juntar todos los elementos
  de los multiconjuntos que se unen,
  la intersección es lo que tienen en común.
  Por ejemplo:
  \begin{align*}
    \{a^2, b^5, c^2\} \cup \{b^2, c^3, d^1\}
      &= \{a^2, b^7, c^5, d^1\} \\
    \{a^2, b^5, c^2\} \cap \{b^2, c^3, d^1\}
      &= \{b^2, c^2\}
  \end{align*}
  En forma similar a las demás operaciones entre conjuntos
  definimos las operaciones respectivas para multiconjuntos.
  Para multiconjuntos tiene sentido hablar del universo
  del que se toman los elementos,
  pero no la idea de complemento.
  Por ejemplo:
  \begin{align*}
    \{a^2, b^5, c^2\} \smallsetminus \{b^2, c^3, d^1\}
      &= \{a^2, b^3\} \\
    \{a^2, b^5, c^2\} \vartriangle \{b^2, c^3, d^1\}
      &= \{a^2, b^3, c^1, d^1\}
  \end{align*}
  Un multiconjunto es subconjunto de otro
  si todos sus elementos
  (con las repeticiones respectivas)
  aparecen en el segundo:
  \begin{align*}
    \{a^2, b^5, c^2\}
      &\subseteq \{a^3, b^5, c^6, d^1\} \\
    \{a^4, b^5, c^2\}
      &\centernot\subseteq \{a^3, b^5, d^3\}
  \end{align*}
  Una manera de asimilar las operaciones con las de conjuntos
  es que la unión considera sumar los elementos de ambos,
  la intersección es el mínimo.

\section{Sumatorias, productorias y yerbas afines}
\label{sec:sumatorias-productorias}
\index{sumatoria|textbfhy}
\index{productoria|textbfhy}

  Es común referirse a sumas de términos parecidos,
  como:
  \begin{equation*}
    a + a^2 + \dotsb + a^{16}
  \end{equation*}
  La notación indicada es sugestiva,
  pero no queda realmente claro cuáles son los términos a incluir.
  Esto lo anotamos mediante sumatoria:
  \begin{equation*}
    \sum_{1 \le k \le 16} a^k
  \end{equation*}
  Aunque podríamos también referirnos a
    \begin{equation*}
    \sum_{0 \le k \le 4} a^{2^k}
  \end{equation*}
  Nótese que la variable índice es irrelevante:
    \index{sumatoria!indice@índice}
  \begin{equation*}
    \sum_{1 \le k \le 16} a^k
      = \sum_{1 \le r \le 16} a^r
  \end{equation*}
  Además,
  esa variable queda atada a la suma,
  de forma que no tiene significado fuera:
  \begin{equation*}
    \sum_{1 \le r \le 10} a^r + \sum_{1 \le s \le 10} b^s
      = \sum_{1 \le k \le 10} a^k + \sum_{1 \le k \le 10} b^k
      = \sum_{1 \le i \le 10} \left( a^i + b^i \right)
  \end{equation*}
  Un caso típico es cuando tenemos un conjunto \(\mathcal{A}\),
  y queremos sumar sobre los \(k \in \mathcal{A}\),
  como en:
  \begin{equation*}
    \sum_{k \in \mathcal{A}} a^k
  \end{equation*}
  Incluso podemos tener varias condiciones,
  en cuyo caso se entiende
  que estamos sumando sobre el valor del índice
  que cumple con todas ellas:
  \begin{equation*}
    \sum_{\mathclap{\substack{
		      0 \le k \le 10 \\
		      k \text{\ múltiplo de 3}
	 }}} \; a^k
      = a^0 + a^3 + a^6 + a^9
  \end{equation*}
  Ocasionalmente usaremos un único signo de suma
  para indicar suma sobre varias variables,
  como por ejemplo:
  \begin{align*}
    \sum_{\mathclap{\substack{
		      0 \le r \le 2 \\
		      1 \le s \le 3
	 }}} \; a^{2 r + s}
      &= a^1 + a^2 + a^3 + a^3 + a^4 + a^5 + a^4 + a^5 + a^6 \\
      &= a + a^2 + 2 a^3 + 2 a^4 + 2 a^5 + a^6
  \end{align*}
  Un problema con esta notación es que no explicita
  las variables sobre las que se suma.
  Esto deberá quedar claro del contexto.

  Con esta convención,
  expresamos sumas infinitas como por ejemplo
  la del problema de Basilea:%
    \index{Basilea, problema de}
  \begin{equation*}
    \sum_{k \ge 1} \frac{1}{k^2}
      = \frac{\pi^2}{6}
  \end{equation*}
  Una notación útil es la convención de Iverson~%
    \index{Iverson, Kenneth E.}%
    \index{Iverson, convencion de@Iverson, convención de|textbfhy}%
    \index{APL (lenguaje de programacion)@APL (lenguaje de programación)}%
    \cite{iverson62:_APL},
  uno de cuyos campeones es Knuth~%
    \index{Knuth, Donald E.}%
    \cite{graham94:_concr_mathem, knuth92:_two_notes_notat}.
  Se anota una condición entre corchetes
  para indicar el valor \(0\) si la condición es falsa,
  y \(1\) si es verdadera.
  Sirve,
  entre otras cosas,
  para eliminar condiciones de las sumas,
  transformándolas en sumas infinitas
  y llevando las condiciones
  a una posición más visible y manipulable.
  Una aplicación simple es:
  \begin{equation*}
    \sum_{k \in \mathcal{A}} f(k) + \sum_{k \in \mathcal{B}} f(k)
      = \sum_k f(k) [ k \in \mathcal{A} ]
	  + \sum_k f(k) [ k \in \mathcal{B} ]
      = \sum_k f(k) ([ k \in \mathcal{A} ] + [ k \in \mathcal{B} ])
  \end{equation*}
  Como estamos incluyendo dos veces
  los elementos en la intersección:
  \begin{equation*}
    [ k \in \mathcal{A} ] + [ k \in \mathcal{B} ]
      = [ k \in \mathcal{A} \cup \mathcal{B} ]
	  + [ k \in \mathcal{A} \cap \mathcal{B} ]
  \end{equation*}
  O sea:
  \begin{equation*}
    \sum_k f(k) ([ k \in \mathcal{A} ] + [ k \in \mathcal{B} ])
      = \sum_k f(k) [ k \in \mathcal{A} \cup \mathcal{B} ]
	  + \sum_k f(k)[ k \in \mathcal{A} \cap \mathcal{B} ]
  \end{equation*}
  Nuestra suma original en notación más tradicional es:
  \begin{equation}
    \label{eq:suma-union-interseccion}
    \sum_{k \in \mathcal{A}} f(k) + \sum_{k \in \mathcal{B}} f(k)
      = \sum_{k \in \mathcal{A} \cup \mathcal{B}} f(k)
	  + \sum_{k \in \mathcal{A} \cap \mathcal{B}} f(k)
  \end{equation}
  También ayuda al intercambiar órdenes de sumas.
  El multiplicar símbolos de Iverson%
    \index{Iverson, convencion de@Iverson, convención de!operaciones}
  corresponde a que ambas condiciones sean ciertas:
  \begin{align}
    \sum_{1 \le j \le n} \sum_{1 \le k \le j} f(j, k)
      &= \sum_{j, k} f(j, k) [1 \le j \le n] [1 \le k \le j]
	    \notag \\
      &= \sum_{j, k} f(j, k) [1 \le k \le j \le n]
	    \notag \\
      &= \sum_{j, k} f(j, k) [1 \le k \le n] [ k \le j \le n]
	    \notag \\
      &= \sum_{1 \le k \le n} \sum_{k \le j \le n} f(j, k)
	    \label{eq:suma-intercambio}
  \end{align}
  Otro ejemplo interesante es:
  \begin{align}
    \sum_{2 \le k \le n} \sum_{1 \le j \le k - 1} \frac{1}{k - j}
      &= \sum_k [2 \le k \le n] \sum_j [1 \le j \le k - 1]
	   \cdot \frac{1}{k - j}
	       \notag \\
      &= \sum_{k, j}  [2 \le k \le n] \cdot [1 \le j \le k - 1]
	   \cdot \frac{1}{k - j}
	       \label{eq:M.SE}
  \end{align}
  Podemos reorganizar:
  \begin{equation}
    \label{eq:M.SE:rangos}
    [2 \le k \le n] \cdot [1 \le j \le k - 1]
      = [1 \le j < k \le n]
      = [1 \le j \le n - 1] \cdot [j + 1 \le k \le n]
  \end{equation}
  Introduciendo la nueva variable \(m = k - j\),
  tenemos:
  \begin{align}
    \label{eq:M.SE:rango-m}
    [j + 1 \le k \le n]
      = [j + 1 \le m + j \le n]
      = [1 \le m \le n - j]
  \end{align}
  Usando~\eqref{eq:M.SE:rangos} con~\eqref{eq:M.SE:rango-m}
  en~\eqref{eq:M.SE}
  permite cambiar el orden de las sumas:
  \begin{align}
    \sum_{2 \le k \le n} \sum_{1 \le j \le k - 1} \frac{1}{k - j}
      &= \sum_{k, j}  [1 \le j \le n - 1] \cdot [j + 1 \le k \le n]
	   \cdot \frac{1}{k - j} \notag \\
      &= \sum_{1 \le j \le n - 1} \sum_{1 \le m \le n - j}
	   \frac{1}{m} \\
      &= \sum_{1 \le j \le n - 1} H_{n - j} \label{eq:M.SE:2}
  \end{align}
  Acá usamos la definición de los \emph{números harmónicos}:%
    \index{numeros harmonicos@números harmónicos|textbfhy}
  \begin{equation}
    \label{eq:definicion-numeros-harmonicos}
    H_n
      = \sum_{1 \le k \le n} \frac{1}{k}
  \end{equation}
  La suma~\eqref{eq:M.SE:2}
  es sobre \(n - j\) entre 1 y \(n - 1\) en reversa,
  que es lo mismo que sumar sobre \(j\) de 1 a \(n - 1\):
  \begin{equation}
    \label{eq:M.SE:3}
    \sum_{2 \le k \le n} \sum_{1 \le j \le k - 1} \frac{1}{k - j}
      = \sum_{1 \le j \le n - 1} H_j
  \end{equation}

  De forma similar a las sumas podemos expresar productos,
  por ejemplo factoriales:
  \begin{align*}
    \prod_{\mathclap{1 \le k \le n}} \; k
      = n!
  \end{align*}
  También productos infinitos,
  como el producto de Wallis%
    \index{Wallis, producto de}
  (lo demostraremos en el teorema~\ref{theo:producto-Wallis}):
  \begin{equation*}
    \prod_{k \ge 1} \frac{2 k}{2 k - 1} \cdot \frac{2 k}{2 k + 1}
      = \frac{2}{1} \cdot \frac{2}{3}
	  \cdot \frac{4}{3} \cdot \frac{4}{5}
	  \cdot \frac{6}{5} \cdot \frac{6}{7}
	  \cdot \frac{8}{7} \cdot \frac{8}{9} \cdots
      = \frac{\pi}{2}
  \end{equation*}
  Esta idea es aplicable a toda operación asociativa y conmutativa,
  ya que implícitamente estamos obviando completamente el orden
  en que se efectúan las operaciones
  entre los elementos considerados.
  Claro que tal cosa solo es válida en sumas
  (o productos, etc)
  finitas
  o sumas infinitas que convergen en forma incondicional.
  Podemos expresar
  uniones e intersecciones:
  \begin{equation*}
    \bigcup_{\mathclap{1 \le k \le 10}} \; \mathcal{A}_k
    \hspace{4em}
    \bigcap_{\mathclap{1 \le k \le 10}} \; \mathcal{A}_k
  \end{equation*}

  Hace falta definir qué se entenderá por una suma sin términos,
  un producto sin factores,
  etc.%
  \index{sumatoria!vacia@vacía}%
  \index{productoria!vacia@vacía}
  La convención general es que el resultado es el valor neutro
  para la operación indicada.
  Vale decir,
  con \(\mathcal{U}\) el conjunto universo:
  \begin{alignat*}{2}
    \sum_{k \in \varnothing}  t_k
      &= 0
	& \quad & \text{porque \(t + 0 = t\)} \\
    \prod_{k \in \varnothing} t_k
      &= 1
	&& \text{porque \(t \cdot 1 = t\)} \\
    \bigcup_{k \in \varnothing} t_k
      &= \varnothing
	&& \text{porque \(t \cup \varnothing = t\)} \\
    \bigcap_{k \in \varnothing} t_k
      &= \mathcal{U}
	&& \text{porque \(t \cap \mathcal{U} = t\)}
  \end{alignat*}
  Podemos partir con esto y extender la suma
  (o el producto, etc)
  un elemento a la vez.
  La definición completa de la sumatoria es:
  \begin{equation*}
    \sum_{k \in \varnothing} t_k
      = 0
    \hspace{4em}
    \sum_{\mathclap{k \in \mathcal{A} \cup \{n\}}} \;\; t_k
      = \sum_{k \in \mathcal{A}} t_k + t_n
	   \quad \text{si \(n \notin \mathcal{A}\)}
  \end{equation*}
  El caso más común,
  claro está,
  es extender un rango.
  Expresamos \(k \in \varnothing\)
  mediante el rango \(0 \le k \le -1\):
  \begin{equation*}
    \sum_{\mathclap{0 \le k \le -1}} \; t_k
      = 0
   \hspace{4em}
   \sum_{\mathclap{0 \le k \le n + 1}} \;\; t_k
      = \sum_{0 \le k \le n} t_k + t_{n + 1}
  \end{equation*}

\section{Potencias factoriales}
\label{sec:preliminares-potencias-factoriales}
\index{potencia factorial|textbfhy}

  Siguiendo a Knuth%
    \index{Knuth, Donald E.}
  (ver por ejemplo~\cite{graham94:_concr_mathem})
  usamos las siguientes notaciones para \(x\) cualquiera
  y entero no negativo \(n\):
  \begin{align}
    \label{eq:def-factorial-desciende}
    x^{\underline{n}}
      &= \prod_{0 \le k < n} (x - k) \\
      &= x \cdot (x - 1) \dotsm (x - (n - 1)) \notag \\
      &= x \cdot (x - 1) \dotsm (x - n + 1) \notag \\
    \label{eq:def-factorial-asciende}
    x^{\overline{n}}
      &= \prod_{0 \le k < n} (x + k) \\
      &= x \cdot (x + 1) \dotsm (x + n - 1) \notag
  \end{align}
  Nótese que
  son exactamente \(n\) factores,
  como en las potencias convencionales.
  Siguiendo la convención de que productos vacíos son \(1\):
  \begin{equation}
    \label{eq:potencias-factoriales-0}
    x^{\underline{0}} = x^{\overline{0}} = 1
  \end{equation}
  Les llamamos \emph{potencias factoriales en bajada}
  y \emph{en subida}
  (en inglés
    \emph{\foreignlanguage{english}{falling factorial power}}
   y \emph{\foreignlanguage{english}{rising factorial power}}),
  respectivamente.
  Hay una variedad de otras notaciones en uso para esto;
  particularmente común es \((x)_k\)
  (\emph{símbolo de Pochhammer})%
    \index{Pochhammer, simbolo de@Pochhammer, símbolo de}
    \glossary{\((x)_k\)}
	     {Símbolo de Pochhammer,
	      potencia factorial en bajada \(x^{\underline{k}}\)}
  para la potencia en bajada
  (aunque ocasionalmente
   se usa para potencias factoriales en subida).
  Se usa también \(x^{(n)}\) para la potencia factorial en subida.
  Algunos autores usan \((x)^+_n\) y \((x)^-_n\)
  para potencias factoriales en subida y bajada,
  respectivamente.

  Una de las razones que hacen útil esta notación es lo siguiente:
  \begin{equation*}
    \frac{\mathrm{d}^n}{\mathrm{d} u^n} u^x
      = x (x - 1) (x - 2) \dotsm (x - n + 1) u^{x - n}
      = x^{\underline{n}} u^{x - n}
  \end{equation*}
  Esto vale para \(x \in \mathbb{C}\),
  y con la convención
  que la \(0\)\nobreakdash-ésima derivada es no hacer nada,
  vale para todo \(n \in \mathbb{N}_0\).

  Nótese que si \(m\), \(n\) y \(k\) son enteros no negativos:
  \begin{align}
    m^{\underline{n + k}}
      &= m^{\underline{n}} \cdot (m - n)^{\underline{k}}
	  \label{eq:potencia-factorial-bajada-suma} \\
    m^{\overline{n + k}}
      &= m^{\overline{n}} \cdot (m + n)^{\overline{k}}
	  \label{eq:potencia-factorial-subida-suma}
  \end{align}
  En particular:
  \begin{equation}
    \label{eq:factorial-potencia-factorial}
    (m + n)^{\underline{n}} \cdot m!
      = (m + 1)^{\overline{n}} \cdot m!
      = (m + n)!
  \end{equation}
  Esto porque:
  \begin{alignat*}{2}
    (m + n)^{\underline{n}} \cdot m!
      &= (m + n) (m + n - 1) \dotsm (m + 1) \cdot m!
      &&= (m + n)! \\
    (m + 1)^{\overline{n}} \cdot m!
      &= (m + 1) (m + 2) \dotsm (m + n) \cdot m!
      &&= (m + n)!
  \end{alignat*}
  Tenemos también:
  \begin{equation}
    \label{eq:factorial=potencia-factorial}
    n^{\underline{n}} = 1^{\overline{n}} = n!
  \end{equation}

  Otra relación notable es la siguiente:
  \begin{equation}
    \label{eq:potencia-factorial-negativo}
    x^{\underline{k}}
      = (-1)^k (-x)^{\overline{k}}
  \end{equation}

\section{Cálculo de diferencias finitas}
\label{sec:finite-differences}
\index{calculo de diferencias finitas@cálculo de diferencias finitas|textbfhy}

  Si definimos el operador \(\Delta\):%
    \index{\(\Delta\)}
  \begin{equation}
    \label{eq:define-Delta}
    \Delta f(n)
      = f(n + 1) - f(n)
  \end{equation}
  El operador es claramente lineal:
  \begin{equation}
    \label{eq:Delta-lineal}
    \Delta ( \alpha f(n) + \beta g(n) )
      = \alpha \Delta f(n) + \beta \Delta g(n)
  \end{equation}
  Tenemos el operador inverso,
  que también es lineal:%
    \index{\(\Sigma\)}
  \begin{equation}
    \label{eq:define-Sigma}
    \Sigma f(n)
      = \sum_{0 \le k < n} f(k) + c
  \end{equation}
  El límite inferior de la suma es arbitrario,
  pero hay que fijar alguno.
  Como \(\Delta c = 0\) vale para cualquier constante \(c\):
  \begin{equation}
    \label{eq:Sigma-Delta-inverses}
    \begin{split}
      \Delta \Sigma f(n) &= f(n) \\
      \Sigma \Delta f(n) &= f(n) + c
    \end{split}
  \end{equation}
  para alguna constante arbitraria \(c\).
  Tenemos:
  \begin{align}
    \Delta n^{\underline{k}}
      &= (n + 1)^{\underline{k}} - n^{\underline{k}} \notag \\
      &= (n + 1) n^{\underline{k - 1}}
	    - n^{\underline{k - 1}} \cdot (n - k + 1) \notag \\
      &= (n + 1 - (n - k + 1)) n^{\underline{k - 1}} \notag \\
      &= k n^{\underline{k - 1}}
	   \label{eq:Delta-falling-factorial}
  \end{align}
  De~\eqref{eq:Sigma-Delta-inverses}
  y~\eqref{eq:Delta-falling-factorial}
  tenemos también:
  \begin{equation}
    \label{eq:Sigma-falling-factorial}
    \Sigma n^{\underline{k}}
      = \frac{n^{\underline{k + 1}}}{k + 1} + c
  \end{equation}
  Otras relaciones de interés son:
  \begin{align}
    \Delta c
      &= 0 \qquad \text{(\(c\) es una constante)}
	 \label{eq:Delta-constant} \\
    \Delta c^n
      &= (c - 1) c^n
	 \label{eq:Delta-c^n} \\
 \intertext{En particular:}
    \Delta 2^n
      &= 2^n
	 \label{eq:Delta-2^n}
  \end{align}
  Demostrar relaciones similares
  a~\eqref{eq:Delta-falling-factorial} y~\eqref{eq:Sigma-falling-factorial}
  para potencias factoriales en subida
  queda de ejercicio.

  Un resultado interesante es la \emph{suma por partes},%
    \index{suma por partes}
  afín a la integración por partes.
  Comenzamos con:
  \begin{align}
    \Delta x_k y_k
      &= x_{k + 1} y_{k + 1} - x_k y_k \notag \\
      &= x_{k + 1} y_{k + 1}
	  - x_k y_{k + 1}
	  + x_k y_{k + 1}
	  - x_k y_k
	   \notag \\
      &= y_{k + 1} \Delta x_k + x_k \Delta y_k
	   \label{eq:Delta_xy}
  \end{align}
  De~\eqref{eq:Delta_xy} reorganizando y sumando tenemos:
  \begin{equation}
    \label{eq:sum-parts}
    \sum_{0 \le k \le n} x_k \Delta y_k
      = x_{n + 1} y_{n + 1} - x_0 y_0
	  - \sum_{0 \le k \le n} y_{k + 1} \Delta x_k
  \end{equation}
  La relación~\eqref{eq:sum-parts} permite
  completar la suma~\eqref{eq:M.SE}.
  En~\eqref{eq:M.SE:3} habíamos llegado a:
  \begin{equation*}
    \sum_{2 \le k \le n} \sum_{1 \le j \le k - 1} \frac{1}{k - j}
      = \sum_{1 \le j \le n - 1} H_j
  \end{equation*}
  Podemos considerar:
  \begin{equation*}
    x_j \Delta y_j
      = H_j \cdot 1
    \qquad
    \Delta x_j
      = H_{j + 1} - H_j = \frac{1}{j + 1}
    \qquad
    y_j
      = j
  \end{equation*}
  Substituyendo en~\eqref{eq:sum-parts},
  con ajustes de índices:
  \begin{equation*}
    \sum_{1 \le j \le n - 1} H_j
      = n H_n - 1 \cdot H_1
	  - \sum_{1 \le j \le n - 1} (j + 1) \, \frac{1}{j + 1}
      = n H_n - n
  \end{equation*}
  y finalmente:
  \begin{equation}
    \label{eq:M.SE:4}
    \sum_{2 \le k \le n} \sum_{1 \le j \le k - 1} \frac{1}{k - j}
      = n H_n - n
  \end{equation}

  Las relaciones~\eqref{eq:Delta-lineal},
  \eqref{eq:Sigma-Delta-inverses},
  \eqref{eq:Sigma-falling-factorial}
  y \eqref{eq:sum-parts}
  recuerdan a las del cálculo infinitesimal,
  y son base del \emph{cálculo de diferencias finitas},%
    \index{calculo de diferencias finitas@cálculo de diferencias finitas}
  con aplicaciones
  en la aproximación de funciones mediante polinomios
  y la interpolación.
  Referencia obligada es Milne-Thomson~%
    \cite{milne-thomson33:_calculus_finite_differences},
  una visión más moderna ofrecen Graham, Knuth y~Patashnik~%
    \cite{graham94:_concr_mathem}.
  Extensiones de las paralelas indicadas
  dan pie al \emph{cálculo umbral},
  formalizado por Roman y Rota~%
    \cite{roman78:_umbral_calculus},
  una visión general dan Bucchianico y Loeb~%
    \cite{bucchianico00:_survey_umbral_calculus}.

\section{Funciones \emph{floor} y \emph{ceil}}
\label{sec:floor-ceil}
\index{funcion piso@función piso|textbfhy}
\index{funcion techo@función techo|textbfhy}

  Siguiendo nuevamente a Knuth,%
      \index{Knuth, Donald E.}
  usaremos la notación \(\lfloor x \rfloor\)
    \glossary{\(\lfloor x \rfloor\)}
	     {Piso, entero inmediatamente inferior a \(x\)}
  para el entero inmediatamente inferior a \(x\)
  (en inglés le llaman \emph{\foreignlanguage{english}{floor}},
   piso)
  y \(\lceil x \rceil\)
    \glossary{\(\lceil x \rceil\)}
	     {Techo, entero inmediatamente superior a \(x\)}
  para el entero inmediatamente superior
  (le llaman \emph{\foreignlanguage{english}{ceil}},
   por \emph{\foreignlanguage{english}{ceiling}},
   techo en inglés).
  La notación es mnemónica en que indica el entero inferior
  a través de marcar un ``piso'',
  y el entero superior mediante un ``techo''.
  Fue Gauß%
    \index{Gauss, Carl Friedrich@Gauß, Carl Friedrich}
  quien introdujo
  la noción de \emph{\foreignlanguage{english}{floor}}
  con la notación \([x]\),
  que se mantuvo como estándar indiscutible hasta que Iverson~%
    \cite{iverson62:_APL}
  introdujera las usadas acá en 1962.
  La notación de Gauß sigue siendo común en teoría de números.
  Algunos usan \(]x[\) para \(\lceil x \rceil\),
  y hay quienes interpretan \(\lfloor x \rfloor\)
  como la ``parte entera'',
  el entero más cercano en dirección al cero.
  Nuestra definición es más regular,
  no depende del signo.
  Por ejemplo:
  \begin{align*}
    \left\lfloor -\frac{2}{3} \right\rfloor
      &= -1 \\
    \lfloor \pi \rfloor
      &= 3 \\
    \lceil \pi \rceil
      &= 4 \\
    \lfloor -3 \rfloor
      &= \lceil -3 \rceil
      = -3
  \end{align*}

  Una función relacionada es la \emph{parte fraccional},%
    \index{parte fraccional|textbfhy}
  o \emph{función diente de sierra}%
    \index{diente de sierra|see{parte fraccional}}
  (en inglés \emph{\foreignlanguage{english}{sawtooth}}):
  \begin{equation}
    \label{eq:def-diente-sierra}
    \{x\}
      = x - \lfloor x \rfloor
  \end{equation}

  Se cumplen las siguientes relaciones básicas%
    \index{parte fraccional!identidades}%
    \index{funcion piso@función piso!identidades}%
    \index{funcion techo@función techo!identidades}
  para estas funciones:
  \begin{align}
    \label{eq:cotas-floor}
    \lfloor x \rfloor	&\le x	   <   \lfloor x \rfloor + 1 \\
    \label{eq:cotas-ceil}
    \lceil x \rceil - 1 &<   x	   \le \lceil x \rceil
  \end{align}
  Expresado de otra forma:
  \begin{align}
    \label{eq:cotas-floor-2}
    x - 1 &<   \lfloor x \rfloor \le x	   \\
    \label{eq:cotas-ceil-2}
    x	  &\le \lceil x \rceil	 <   x + 1 \\
    \label{eq:cotas-diente-sierra}
    0	  &\le \{x\}		 < 1
  \end{align}
  Es claro que:
  \begin{align}
    \label{eq:ceil-negativo}
    \lfloor x \rfloor
      &= -\lceil -x \rceil \\
    \label{eq:floor-negativo}
    \lceil x \rceil
      &= -\lfloor -x \rfloor
  \end{align}
  De la definición,
  ecuación~\eqref{eq:def-diente-sierra},
  tenemos directamente:
  \begin{equation}
    \label{eq:floor-diente-sierra}
    \lfloor x \rfloor
      = x - \{ x \}
  \end{equation}
  Usando las relaciones para cambio de signo
  tenemos:
  \begin{align}
    x
      &= \lceil x \rceil + (x - \lceil x \rceil)
       = \lceil x \rceil - (- x + \lfloor -x \rfloor )	\notag \\
  \intertext{y resulta}
    \lceil x \rceil
      &= x + \{ -x \}  \label{eq:ceil-diente-sierra}
  \end{align}

  Si \(n\) es un entero y \(x\) un real cualquiera,
  se cumplen:
  \begin{align}
    \label{eq:floor-real+int}
    \lfloor x + n \rfloor
      &= \lfloor x \rfloor + n \\
    \label{eq:ceil-real+int}
    \lceil x + n \rceil
      &= \lceil x \rceil + n \\
    \{ x + n \}
      &= \{ x \}
  \end{align}
  En cambio,
  para \(x\) e \(y\) reales cualquiera tenemos:
  \begin{align}
    \label{eq:floor-sum}
    \lfloor x \rfloor + \lfloor y \rfloor
      &\le \lfloor x + y \rfloor
       \le \lfloor x \rfloor + \lfloor y \rfloor + 1 \\
    \label{eq:ceil-sum}
    \lceil x \rceil + \lceil y \rceil -1
      &\le \lceil x + y \rceil
       \le \lceil x \rceil + \lceil y \rceil
  \end{align}

  Algunas identidades más interesantes
  con \(x\) real,
  \(m\) entero y \(n\) natural son~%
    \cite{graham94:_concr_mathem}:
  \begin{align}
    \label{eq:floor-real-fraccion}
    \left\lfloor \frac{x + m}{n} \right\rfloor
      &= \left\lfloor
	   \frac{\lfloor x \rfloor + m}{n}
	 \right\rfloor \\
    \label{eq:ceil-real-fraccion}
    \left\lceil \frac{x + m}{n} \right\rceil
      &= \left\lceil \frac{\lceil x \rceil + m}{n} \right\rceil
  \end{align}
  La primera servirá de ejemplo de demostración con estas nociones:
  \begin{equation*}
    \left\lfloor \frac{x + m}{n} \right\rfloor
      = \left\lfloor
	  \frac{\lfloor x \rfloor + \{x\} + m}{n}
	\right\rfloor
      = \left\lfloor
	  \frac{\lfloor x \rfloor + m}{n} + \frac{\{x\}}{n}
	\right\rfloor
  \end{equation*}
  Como \(\{x\} < 1\),
  el término final no influye.

  Con \(m\) positivo:
  \begin{align}
    \label{eq:n-suma-fracciones-floor}
    n &= \left\lfloor \frac{n}{m} \right\rfloor
	   + \left\lfloor \frac{n + 1}{m} \right\rfloor
	   + \left\lfloor \frac{n + 2}{m} \right\rfloor
	   + \dotsb
	   + \left\lfloor \frac{n + m - 1}{m} \right\rfloor \\
    \label{eq:n-suma-fracciones-ceil}
    n &= \left\lceil \frac{n}{m} \right\rceil
	   + \left\lceil \frac{n - 1}{m} \right\rceil
	   + \left\lceil \frac{n - 2}{m} \right\rceil
	   + \dotsb
	   + \left\lceil \frac{n - m + 1}{m} \right\rceil
  \end{align}
  Para demostrar la primera de éstas,
  llamemos \(n = q \cdot m + r\),
  con \(0 \le r < m\).
  Escribamos:
  \begin{equation*}
    \sum_{0 \le k < m} \left\lfloor \frac{n + k}{m} \right\rfloor
      = \sum_{0 \le k < m}
	  \left\lfloor q + \frac{r + k}{m} \right\rfloor
      = m \cdot q +
	 \sum_{0 \le k < m}
	   \left\lfloor \frac{r + k}{m} \right\rfloor
  \end{equation*}
  Los términos de la última suma son \(0\) si \(r + k < m\)
  y \(1\) si \(r + k \ge m\),
  que es el rango \(m - r \le k < m\)
  con exactamente \(r\) elementos,
  y resulta lo indicado.

  Más generalmente,
  con \(x \in \mathbb{R}\):
  \begin{align}
    \label{eq:floor-mx-suma}
    \lfloor m x \rfloor
      &= \left\lfloor x \right\rfloor
	   + \left\lfloor x + \frac{1}{m} \right\rfloor
	   + \left\lfloor x + \frac{2}{m} \right\rfloor
	   + \dotsb
	   + \left\lfloor x + \frac{m - 1}{m} \right\rfloor \\
    \label{eq:ceil-mx-suma}
    \lceil m x \rceil
      &= \left\lceil x \right\rceil
	   + \left\lceil x - \frac{1}{m} \right\rceil
	   + \left\lceil x - \frac{2}{m} \right\rceil
	   + \dotsb
	   + \left\lceil x - \frac{m - 1}{m} \right\rceil
  \end{align}
  A la relación~\eqref{eq:floor-mx-suma}
  se le conoce como \emph{identidad de Hermite}.%
    \index{Hermite, identidad de}
  La siguiente demostración es debida a Matsuoka~%
    \cite{matsuoka64:_Hermite_identity}.
  Sea:
  \begin{equation*}
    f(x)
      = \lfloor m x \rfloor
	  - \lfloor x \rfloor
	  - \left\lfloor x + \frac{1}{m} \right\rfloor
	  - \left\lfloor x + \frac{2}{m} \right\rfloor
	  - \dotsb
	  - \left\lfloor x + \frac{m - 1}{m} \right\rfloor
  \end{equation*}
  Entonces,
  como para \(\alpha\), \(\beta\) reales siempre es
  \(\lfloor \alpha + 1 \rfloor - \lfloor \beta + 1 \rfloor
      = \lfloor \alpha \rfloor - \lfloor \beta \rfloor\)
  tenemos:
  \begin{align*}
    f\left( x + \frac{1}{m} \right)
      &= \lfloor m x + 1 \rfloor
	   - \left\lfloor x + \frac{1}{m}\right\rfloor
	   - \left\lfloor x + \frac{2}{m} \right\rfloor
	   - \dotsb
	   - \left\lfloor x + \frac{m - 1}{m} \right\rfloor
	   - \lfloor x + 1 \rfloor \\
      &= \lfloor m x \rfloor
	   - \lfloor x \rfloor
	   - \left\lfloor x + \frac{1}{m} \right\rfloor
	   - \left\lfloor x + \frac{2}{m} \right\rfloor
	   - \dotsb
	   - \left\lfloor x + \frac{m - 1}{m} \right\rfloor \\
      &= f(x)
  \end{align*}
  Por el otro lado,
  para \(0 \le x < 1 / m\) tenemos \(f(x) = 0\),
  con lo que \(f(x) = 0\) para todo \(x\),
  que es lo que queríamos probar.
  La relación~\eqref{eq:ceil-mx-suma} se demuestra de forma similar.

  Alternativamente,
  para la identidad de Hermite
  por~\eqref{eq:floor-real-fraccion} podemos escribir:
  \begin{equation*}
    \left\lfloor \frac{\lfloor m x \rfloor + k}{m} \right\rfloor
      = \left\lfloor \frac{m x + k}{m} \right\rfloor
      = \left\lfloor x + \frac{k}{m} \right\rfloor
  \end{equation*}
  Por~\eqref{eq:n-suma-fracciones-floor} resulta lo indicado.

  Para \(m\) positivo,
  las siguientes permiten transformar pisos en techos y viceversa:
  \begin{align}
    \label{eq:floor-fraccion}
    \left\lfloor \frac{n}{m} \right\rfloor
      &= \left\lceil \frac{n - m + 1}{m} \right\rceil
       = \left\lceil \frac{n + 1}{m} \right\rceil - 1 \\
    \label{eq:ceil-fraccion}
    \left\lceil \frac{n}{m} \right\rceil
      &= \left\lfloor \frac{n + m - 1}{m} \right\rfloor
       = \left\lfloor \frac{n - 1}{m} \right\rfloor + 1
  \end{align}
  Para demostrar~\eqref{eq:floor-fraccion},
  sea \(n = q m + r\) con \(0 \le r < m\).
  Entonces:
  \begin{align*}
    n - m + 1
      &= (q - 1) m + r + 1 \\
    n + 1
      &= q m + r + 1
  \end{align*}
  Como \(1 \le r + 1 \le m\),
  resultan las relaciones indicadas en~\eqref{eq:floor-fraccion}.

  A veces resultan útiles:
  \begin{align}
    \label{eq:frac-floor-floor}
    \left\lfloor \frac{\lfloor x / m \rfloor}{n} \right\rfloor
      &= \left\lfloor \frac{x}{m n} \right\rfloor \\
    \label{eq:frac-ceil-ceil}
    \left\lceil \frac{\lceil x / m \rceil}{n} \right\rceil
      &= \left\lceil \frac{x}{m n} \right\rceil
  \end{align}
  Estas son reflejo de lo siguiente:
  Sea \(f(x)\) una función continua monótona creciente
  con la propiedad especial
  que si \(f(x)\) es entero entonces \(x\) es entero.
  Entonces:
  \begin{align*}
    \lfloor f(\lfloor x \rfloor)\rfloor
      &= \lfloor f(x) \rfloor \\
    \lceil f(\lceil x \rceil)\rceil
      &= \lceil f(x) \rceil
  \end{align*}
  Veremos el primer caso,
  el otro es análogo.
  Cuando \(x = \lfloor x \rfloor\) no hay nada que demostrar.
  Supongamos entonces que \(\lfloor x \rfloor < x\),
  con lo que \(f(\lfloor x \rfloor) < f(x)\)
  ya que \(f\) es monótona,
  y \(\lfloor f(\lfloor x \rfloor) \rfloor
	\le \lfloor f(x) \rfloor\).
  Si fuera \(\lfloor f(\lfloor x \rfloor) \rfloor
	       < \lfloor f(x) \rfloor\),
  habría un número \(y\) tal que \(\lfloor x \rfloor < y \le x\)
  con \(f(y) = \lfloor f(x) \rfloor\) dado que \(f\) es continua.
  Pero entonces \(f(y)\) es entero,
  luego por la propiedad especial \(y\) es entero también.
  Como no hay enteros \(y\)
  que cumplen \(\lfloor x \rfloor < y \le x\),
  debe ser
    \(\lfloor f(\lfloor x \rfloor) \rfloor
	= \lfloor f(x) \rfloor\).

  Como ejemplo,
  esto da:
  \begin{equation*}
    \lfloor \sqrt{\lfloor x \rfloor} \rfloor
      = \lfloor \sqrt{x} \rfloor
  \end{equation*}

\section{Otros resultados de interés}
\label{sec:resultados-interesantes}

  En esta sección recogeremos la demostración de algunos resultados
  que usaremos más adelante.

  \begin{theorem}[Desigualdad de Cauchy-Schwarz]
    \index{Cauchy-Schwarz, desigualdad de}
    \index{desigualdad de Cauchy-Schwarz|see{Cauchy-Schwarz, desigualdad de}}
    \label{theo:Cauchy-Schwarz}
    Sean vectores de números reales
    \mbox{\(\boldsymbol{a} = (a_1, a_2, \dotsc, a_n)\)}
    y \mbox{\(\boldsymbol{b} =(b_1, b_2, \dotsc, b_n)\)}.
    Definamos:
    \begin{alignat*}{4}
      p &= \sum_{1 \le k \le n} a_k^2
	&\qquad&
      q &= \sum_{1 \le k \le n} a_k b_k
	&\qquad&
      r &= \sum_{1 \le k \le n} b_k^2
    \end{alignat*}
    Entonces \(q^2 \le p r\).
  \end{theorem}
  \begin{proof}
    Sea \(x \in \mathbb{R}\),
    y consideremos la suma
    \begin{equation*}
      s(x)
	= \sum_{1 \le k \le n} (a_k x + b_k)^2
    \end{equation*}
    Siendo \(s(x)\) una suma de cuadrados de números reales,
    \(s(x) \ge 0\).
    Expandiendo los cuadrados y agrupando términos
    la suma se expresa:
    \begin{equation*}
      s(x)
	= p x^2 + 2 q x + r
    \end{equation*}
    Este polinomio tiene a lo más un cero real ya que nunca es negativo,
    por lo que su discriminante no es positivo:
    \begin{equation*}
      4 q^2 - 4 p r
	\le 0
    \end{equation*}
    Esto equivale a lo anunciado.
  \end{proof}

  \begin{theorem}[Desigualdad triangular]
    \index{desigualdad triangular}
    \label{theo:desigualdad-triangular}
    Sean \(a\) y \(b\) números complejos.
    Entonces
      \mbox{\(\lvert a + b \rvert
		\le \lvert a \rvert + \lvert b \rvert\)}.
  \end{theorem}
  Se le llama ``desigualdad triangular'' porque
  si se consideran \(a\) y \(b\) como los lados de un triángulo,
  \(a + b\) es el tercer lado
  (vea la figura~\ref{fig:triangle}),
  \begin{figure}[htbp]
    \centering
    \pgfimage{images/triangle}
    \caption{Desigualdad triangular}
    \label{fig:triangle}
  \end{figure}
  y el teorema dice
  que la suma de los largos de dos lados del triángulo
  es mayor que el largo del tercero
  (salvo cuando el triángulo es en realidad una única línea,
   en cuyo caso la suma de los largos de dos de los lados
   es igual al largo del tercero).
   \begin{proof}
     Sean \(a = u + \mathrm{i} v\) y \(b = x + \mathrm{i} y\).
     En estos términos,
     anotando \(\overline{a}\) para el conjugado de \(a\):
     \begin{align*}
       \lvert a \rvert^2
	 &= u^2 + v^2
	  = a \cdot \overline{a} \\
       \lvert a + b \rvert^2
	 &= (a + b) \cdot \overline{(a + b)} \\
	 &= (a + b) \cdot (\overline{a} + \overline{b}) \\
	 &= a \cdot \overline{a}
	     + b \cdot \overline{b}
	     + a \cdot \overline{b}
	     + \overline{a} \cdot b \\
	 &= \lvert a \rvert^2 + \lvert b \rvert^2
	     + a \cdot \overline{b}
	     + \overline{a} \cdot b
     \end{align*}
     Por otro lado:
     \begin{equation*}
       a \cdot \overline{b} + \overline{a} \cdot b
	 = (u + \mathrm{i} v) \cdot (x - \mathrm{i} y)
	     + (u - \mathrm{i} v) \cdot (x + \mathrm{i} y)
	 = 2 u x + 2 v y
     \end{equation*}
     Por la desigualdad de Cauchy-Schwarz es
     \((u x + v y)^2 \le (u^2 + v^2) \cdot (x^2 + y^2)\),
     así que \(u x + v y \le \lvert a \rvert \cdot \lvert b \rvert\)
     y resulta:
     \begin{equation*}
       \lvert a + b \rvert^2
	 = \lvert a \rvert^2 + \lvert b \rvert^2
	    + a \cdot \overline{b}
	    + \overline{a} \cdot b
	 \le \lvert a \rvert^2
	    + 2 \lvert a \rvert \cdot \lvert b \rvert
	    + \lvert b \rvert^2
	 = \left(\lvert a \rvert + \lvert b \rvert\right)^2
     \end{equation*}
     Tomando raíces cuadradas obtenemos lo prometido.
   \end{proof}
   Es obvio que esto puede extenderse a sumas finitas:
   \begin{equation}
     \label{eq:triangle-inequality-sum}
     \left\lvert \sum_{1 \le k \le n} a_k \right\rvert
       \le \sum_{1 \le k \le n} \lvert a_k \rvert
   \end{equation}

  Una observación simple
  es lo que llaman el \emph{principio del palomar}%
    \index{principio del palomar}
  (en inglés
    \emph{\foreignlanguage{english}{pigeonhole principle}}):%
    \index{pigeonhole principle@\emph{\foreignlanguage{english}{pigeonhole principle}}|see{principio del palomar}}
  Si hay \(n\) palomares,
  y hay más de \(n\) palomas,
  hay al menos un palomar con más de una paloma.
  Si hay infinitas palomas,
  al menos uno de los palomares tiene infinitos huéspedes.
  \begin{theorem}[Principio extendido del palomar]
    \label{theo:pigeonhole}
    Si hay \(n\) palomares,
    y \(r\) palomas,
    hay un palomar con al menos \(\lceil r / n \rceil\) palomas.
  \end{theorem}
  \begin{proof}
    La demostración es por contradicción.
    Supongamos que todos los palomares
    tienen menos de \(\lceil r / n \rceil\) palomas.
    Entonces el número total de palomas cumple:
    \begin{equation*}
      r
	\le n \cdot \left(
		      \left\lceil \frac{r}{n} \right\rceil - 1
		    \right)
	< n \cdot \frac{r}{n}
	= r
    \end{equation*}
    Concluimos \(r < r\),
    lo que es absurdo.
  \end{proof}

  Para un ejemplo,
  considere los números \(1, 2, \dotsc, 2 n\),
  y elija \(n + 1\) de ellos formando el conjunto \(\mathcal{A}\).
  Entonces hay dos números relativamente primos en \(\mathcal{A}\),
  porque hay dos números que difieren en \(1\)
  (considere los \(n\) ``palomares''
   formados por pares adyacentes
   \((1, 2)\), \((3, 4)\), \ldots, \((2 n - 1, 2 n)\),
   al menos uno contiene dos números).

  Considere nuevamente la situación anterior.
  Entonces entre los números de \(\mathcal{A}\)
  hay uno que divide a otro.
  Acá podemos escribir cada uno de los \(2 n\) elementos
  como \(2^k m\),
  con \(m\) impar en el rango \(1 \le m \le 2 n - 1\).
  Como hay \(n + 1\) números,
  pero solo \(n\) posibles partes impares,
  alguna debe repetirse,
  y en consecuencia
  tenemos un par que solo difiere en la potencia de 2.

\section{Notación asintótica}
\label{sec:notacion-asintotica}
\index{notacion asintotica@notación asintótica|seealso{Bachmann-Landau, notaciones de}}

  \begin{table}[htbp]
    \centering
    \begin{tabular}[c]{|l|l|}
      \hline
      \multicolumn{1}{|c|}{\rule[-0.7ex]{0pt}{3ex}\textbf{Notación}} &
	 \multicolumn{1}{c|}{\textbf{Definición}} \\
      \hline
	\rule[-0.7ex]{0pt}{3ex}%
      \vphantom{\(\biggl(\biggr)\)}
      \(f(n) \underset{n \rightarrow \infty}{=} O(g(n))\) &
	 \(\exists k > 0, n_0 \forall n > n_0 \colon
	      \lvert f(n) \rvert \le k \cdot g(n)\) \\
      \vphantom{\(\biggl(\biggr)\)}
      \(f(n) \underset{n \rightarrow \infty}{=} \Omega(g(n))\) &
	 \(\exists k > 0, n_0 \forall n > n_0 \colon
	       \lvert f(n) \rvert \ge k \cdot g(n)\) \\
      \vphantom{\(\biggl(\biggr)\)}
      \(f(n) \underset{n \rightarrow \infty}{=} \Theta(g(n))\) &
	  \(\exists k_1 > 0, k_2 > 0, n_0 \forall n > n_0 \colon
	       k_1 \cdot g(n) \le \lvert f(n) \rvert
			      \le k_2 \cdot g(n)\) \\
      \vphantom{\(\biggl(\biggr)\)}
      \(f(n) \underset{n \rightarrow \infty}{=} o(g(n))\) &
	  \(\forall \epsilon > 0 \exists n_0 \forall n > n_0 \colon
	       \lvert f(n) \rvert \le \epsilon \cdot g(n)\) \\
      \vphantom{\(\biggl(\biggr)\)}
      \(f(n) \underset{n \rightarrow \infty}{=} \omega(g(n))\) &
	  \(\forall \epsilon > 0 \exists n_0 \forall n > n_0 \colon
	       \lvert f(n) \rvert \ge \epsilon \cdot g(n)\) \\
      \vphantom{\(\biggl(\biggr)\)}
      \(f(n) \underset{n \rightarrow \infty}{\sim} g(n)\) &
	       \(\displaystyle \lim_{n \rightarrow \infty}
				 f(n) / g(n) = 1\)
		     \rule[-2ex]{0pt}{3ex} \\
      \hline
    \end{tabular}
    \caption{Notaciones de Bachmann-Landau}
    \label{tab:bachmann-landau}
    \index{Bachmann-Landau, notaciones de|textbfhy}
  \end{table}
  A la familia del cuadro~\ref{tab:bachmann-landau}
  se le llama \emph{notaciones de Bachmann-Landau},
  aunque en realidad fue Knuth~%
    \cite{knuth76:_big_omicr_omega_theta}%
    \index{Knuth, Donald E.}
  quien las definió como las damos acá.
  Por convención
  se escriben las cosas de la forma más simple posible,
  vale decir no se escribe \(O(\pi)\) sino \(O(1)\),
  tampoco \(O(3 n^2 + 17)\) sino simplemente \(O(n^2)\).
  Aunque acá esto se ha anotado en términos de la variable \(n\),
  lo que hace subentender \(n \in \mathbb{N}\),
  la notación tiene perfecto sentido para \(n \in \mathbb{R}\).

  Intuitivamente:
  \begin{description}
  \item[\boldmath
	  \(f(n) \underset{n \rightarrow \infty}{=} O(g(n))\):%
	\unboldmath]
    Indica que \(f\) crece a lo más como (una constante por) \(g\)
    cuando \(n \rightarrow \infty\),
    que \(g\) acota a \(f\) por arriba.
    Se aplica
    cuando \(\lim_{n \rightarrow \infty} f(n) = \infty\).
  \item[\boldmath
	  \(f(n) \underset{n \rightarrow \infty}{=} \Omega(g(n))\):%
	\unboldmath]
    Significa que \(f\)
    crece a lo menos como (una constante por) \(g\)
    cuando \(n \rightarrow \infty\),
    que \(g\) acota a \(f\) por abajo.
    Al igual que el caso anterior,
    se aplica más que nada
    cuando \(\lim_{n \rightarrow \infty} f(n) = \infty\).
  \item[\boldmath
	  \(f(n) \underset{n \rightarrow \infty}{=} \Theta(g(n))\):%
	\unboldmath]
    En este caso \(f\) está acotada por arriba y abajo por \(g\),
    ambas crecen a la misma tasa
    (dentro de un factor constante)
    cuando  \(n \rightarrow \infty\).
    Resume el caso en que
      \(f(n) \underset{n \rightarrow \infty}{=} O(g(n))\)
    y también
      \(f(n) \underset{n \rightarrow \infty}{=} \Omega(g(n))\).
  \item[\boldmath
	  \(f(n) \underset{n \rightarrow \infty}{=} o(g(n))\):%
	\unboldmath]
    Acá \(f\) es dominada asintóticamente por \(g\),
    \(f\) disminuye más rápidamente que \(g\)
    cuando  \(n \rightarrow \infty\).
    Esto se usa más que nada para comparar funciones
    tales que \(\lim_{n \rightarrow \infty} f(n) = 0\).
  \item[\boldmath
	  \(f(n) \underset{n \rightarrow \infty}{=} \omega(g(n))\):%
	\unboldmath]
    La situación acá es la inversa de la anterior,
    \(f\) domina asintóticamente a \(g\),
    \(g\) disminuye más rápidamente que \(f\)
    cuando  \(n \rightarrow \infty\).
    Útil principalmente cuando
      \(\lim_{n \rightarrow \infty} f(n) = 0\).
  \item[\boldmath
	  \(f(n) \underset{n \rightarrow \infty}{\sim} g(n)\):%
	\unboldmath]
    Asintóticamente,
    ambas funciones son iguales.
  \end{description}
  Más que nada usaremos las primeras tres,
  las otras se mencionan por completitud.

  Es común el caso de considerar el caso en que \(x \rightarrow 0\)
  (o alguna otra constante),
  no \(n \rightarrow \infty\),
  y por ejemplo la definición de \(O()\) se debe leer como:
  \begin{equation*}
    f(x)
      \underset{x \rightarrow a}{=} O(g(x))
	 \text{\ si\ }
	   \exists k
	     \forall \epsilon > 0 \; \forall x \colon
	       \lvert x - a \rvert \le \epsilon \implies
		 \lvert f(n) \rvert \le k \cdot g(n)
  \end{equation*}
  Para las demás notaciones se definen en forma afín.

  Nótense las similitudes
  con las definiciones de los límites respectivos:%
    \index{Bachmann-Landau, notaciones de!relacion con limites@relación con límites}
  \begin{align*}
    \lim_{n \rightarrow \infty} f(n) = a
       &\text{\ cuando\ }
	 \forall \epsilon > 0 \exists N \colon
	   n \ge N \implies \lvert f(n) - a \rvert \le \epsilon \\
    \lim_{x \rightarrow x_0} f(x) = a
       &\text{\ cuando\ }
	 \forall \epsilon > 0 \; \exists \delta \colon
	   0 < \lvert x - x_0 \rvert \le \delta
	      \implies \lvert f(n) - a \rvert \le \epsilon
  \end{align*}
  Lejos las más usadas son \(n \rightarrow \infty\),
  y se suele solo indicar explícitamente en otros casos.
  Si la variable es real
  normalmente se estará frente a la situación \(x \rightarrow 0\).

  Podemos deducir relaciones entre las distintas situaciones,
  suponiendo que \(f(n)\) y \(g(n)\) no se anulan.
  Las definiciones
  inmediatamente indican que \(f(n) = \Theta(g(n))\)
  si \(f(n) = \Omega(g(n))\) y también \(f(n) = O(g(n))\).
  Es fácil ver que si \(f(n) = O(g(n))\)
  entonces \(g(n) = \Omega(f(n))\),
  y viceversa.
  Similarmente,
  \(f(n) = o(g(n))\) si y solo si \(g(n) = \omega(f(n))\).
  Vemos también que si \(f(n) = o(g(n))\)
  no puede ser simultáneamente \(f(n) = \Omega(g(n))\).
  Si \(f(n) \sim g(n)\),
  es claro que \(f(n) = \Theta(g(n))\),
  pero por ejemplo \(3 n = \Theta(n)\)
  y \(3 n \not\sim n\).
  \begin{figure}[ht]
    \centering
    \pgfimage{images/Bachmann-Landau}
    \caption{Relación entre $f(n)$ y $g(n)$
	     en notaciones de Bachmann-Landau}
    \label{fig:Bachmann-Landau}
  \end{figure}
  La figura~\ref{fig:Bachmann-Landau}
  resume las relaciones indicadas.

  Las notaciones dadas
  son abusivas.
  Por ejemplo,
  \(n^{\sfrac{1}{2}} = O(n^2)\) y \(3 n^2 - 17 = O(n^2)\)
  definitivamente no permiten concluir
  que \(n^{\sfrac{1}{2}} = 3 n^2 - 17\).
  Una notación más ajustada
  sería considerar \(O(g(n))\) como el \emph{conjunto} de funciones
  que cumplen lo indicado,
  y decir \(f(n) \in O(g(n))\).
  La forma más simple de evitar problemas
  es considerar siempre que el lado derecho de igualdades
  que usan estas notaciones
  es una versión menos precisa del lado izquierdo.

  También se usa notación como:
  \begin{equation*}
    f(n) = 3 n^3 + 2 n^2 + O(n)
  \end{equation*}
  con la intención de indicar que hay una función \(g(n)\)
  que da:
  \begin{equation*}
    f(n) = 3 n^3 + 2 n^2 + g(n)
  \end{equation*}
  donde \(g(n) = O(n)\).
  Este uso hace preferible el no considerar las notaciones
  como conjuntos de funciones.

  Recordemos la definición:
  \begin{equation*}
    \lim_{n \rightarrow \infty} f(n) = a
  \end{equation*}
  si para todo \(\epsilon > 0\) existe \(n_0\)
  tal que para todo \(n \ge n_0\)
  se cumple \(\lvert f(n) - a \rvert \le \epsilon\).
  Considerar los límites suele ser útil
  para determinar relaciones asintóticas.
  Incluso hay quienes usan resultados como el teorema siguiente
  para definirlas.
  \begin{theorem}
    \label{theo:limite-O-Omega}
    Dadas funciones \(f(n)\) y \(g(n)\),
    si \(\lim_{n \rightarrow \infty} f(n) / g(n)\) es finito,
    entonces \(f(n) = O(g(n))\).
    Si además el límite es positivo,
    entonces \(f(n) = \Theta(g(n))\)
    (en particular,
     tenemos \(f(n) = \Omega(g(n))\)).
     Si \(\lim_{n \rightarrow \infty} f(n) / g(n) = \infty\),
     entonces \(f(n) = \Omega(g(n))\).
  \end{theorem}
  \begin{proof}
    Por definición
    \begin{equation*}
      \lim_{n \rightarrow \infty} \frac{f(n)}{g(n)} = a
    \end{equation*}
    con \(a\) finito
    cuando para todo \(\epsilon > 0\)
    hay \(n_0\) tal que siempre que \(n \ge n_0\) tenemos
    \(\lvert f(n) / g(n) - a \rvert \le \epsilon\).
    Esto puede expresarse como:
    \begin{equation*}
      (a - \epsilon) g(n) \le f(n) \le	(a + \epsilon) g(n)
	\text{\ cuando\ } n \ge n_0
    \end{equation*}
    La segunda desigualdad corresponde a la definición
    de \(f(n) = O(g(n))\).

    Si el límite \(a > 0\),
    podemos elegir \(\epsilon < a\),
    lo que da constantes positivas en ambas desigualdades,
    y corresponde a \(f(n) = \Theta(g(n))\).
    La definición de \(f(n) = \Omega(g(n))\)
    es simplemente una de las desigualdades del caso anterior.

    Por definición,
    \begin{equation*}
      \lim_{n \rightarrow \infty} \frac{f(n)}{g(n)}
	= \infty
    \end{equation*}
    significa que para todo \(c > 0\) hay \(n_0\)
    tal que si \(n \ge n_0\):
    \begin{equation*}
      \frac{f(n)}{g(n)}
	\ge c
      \qquad
      f(n)
	\ge c g(n)
    \end{equation*}
    En esto basta elegir \(c\) con su correspondiente \(n_0\)
    para satisfacer la definición de \(f(n) = \Omega(g(n))\).
  \end{proof}

  Nótese que hay situaciones en las cuales esto no es aplicable.
  Por ejemplo,
  sea
  \begin{equation*}
    f(n)
      = 3 n^2 \sin^2 n + n / 2
  \end{equation*}
  Sabemos que \(0 \le \sin^2 n \le 1\),
  con lo que para todo \(n > 1\):
  \begin{align*}
    f(n)
      &\ge \frac{1}{2} \, n \\
    f(n)
      &\le 3 n^2 + \frac{1}{2} \, n
       \le \frac{7}{2} \, n^2
  \end{align*}
  Estas corresponden directamente a \(f(n) = \Omega(n)\)
  y \(f(n) = O(n^2)\).
  Pero los límites
  a los que hace referencia
  el teorema~\ref{theo:limite-O-Omega} no ayudan:
  \(\lim_{n \rightarrow \infty} f(n) / n^2\) no existe
  (la razón oscila entre 0 y 3)
  y tampoco existe \(\lim_{n \rightarrow \infty} f(n) / n\)
  (la razón oscila entre \(1 / 2\) y \(3 n\)).
  No hay \(\alpha\) tal que \(f(n) = \Theta(n^\alpha)\).

  Sabemos del teorema de Taylor%
    \index{Taylor, teorema de}
  (para condiciones sobre la función
   y otros véase un texto de cálculo,
  por ejemplo el de Stein y Barcellos~%
   \cite{stein92:_calculus_anal_geom})
  que si \(a \le x\) y \(x\) está dentro del radio de convergencia
  podemos escribir:
  \begin{equation*}
    f(x)
      = \sum_{0 \le r \le k}
	  \frac{f^{(r)}(a)}{r!} \, (x - a)^r + R_k(x)
  \end{equation*}
  La forma de Lagrange del residuo es:
  \begin{equation*}
    R_k(x) = \frac{f^{(k + 1)}(\xi)}{(k + 1)!} \, (x - a)^{k + 1}
  \end{equation*}
  donde \(a \le \xi \le x\).
  Si \(f^{(k + 1)}(x)\) es acotado en el rango de interés,
  podemos decir:
  \begin{equation*}
    f(x)
      = \sum_{0 \le r \le k} \frac{f^{(i)}(a)}{r!} \, (x - a)^r
	  + O((x - a)^{k + 1})
  \end{equation*}
  Por ejemplo:
  \begin{equation*}
    \frac{n}{n - 1}
      = \frac{1}{1 - 1 / n}
  \end{equation*}
  Aplicando el teorema de Maclaurin:%
    \index{Maclaurin, teorema de}
  \begin{equation*}
    \frac{1}{1 - x}
      = 1 + x + O(x^2)
  \end{equation*}
  porque
  \begin{equation*}
    \frac{\mathrm{d}^2}{\mathrm{d} x^2} \, \frac{1}{1 - x}
      = \frac{2}{(1 - x)^3}
  \end{equation*}
  En el rango de interés
  (digamos \(n \ge 2\),
   que se traduce en \(0 < 1 / n \le 1 / 2\),
   donde lo único que realmente interesa es que \(1 / n < 1\))
  la derivada está acotada.
  Resulta:
  \begin{equation*}
    \frac{n}{n - 1}
      = \frac{1}{1 - 1 / n}
      = 1 + \frac{1}{n} + O(n^{-2})
  \end{equation*}

  Suponiendo que \(a_r \ne 0\),
  tenemos que:
  \begin{equation*}
    a_r n^r + a_{r - 1} n^{r - 1} + \dotsb + a_0 = \Theta(n^r)
  \end{equation*}
  Esto porque:
  \begin{equation*}
    \lim_{n \rightarrow \infty}
	 \frac{a_r n^r + a_{r - 1} n^{r - 1} + \dotsb + a_0}{n^r}
      = a_r
  \end{equation*}
  También:
  \begin{equation*}
    a_r n^r + a_{r - 1} n^{r - 1} + \dotsb + a_0 \sim a_r n^r
  \end{equation*}
  Para demostrar esto debemos calcular:
  \begin{equation*}
    \lim_{n \rightarrow \infty}
	 \frac{a_r n^r + a_{r - 1} n^{r - 1} + \dotsb + a_0}
	      {a_r n^r}
      = \lim_{n \rightarrow \infty}
	     \left(
	 1 + \frac{a_{r - 1} n^{r - 1} + \dotsb + a_0}{a_r n^r}
	     \right)
      = 1
  \end{equation*}

  Es fácil demostrar que para valores reales de \(a < b\):
  \begin{equation*}
    n^a
      = O(n^b)
    \hspace{3em}
    n^b
      = \Omega(n^a)
  \end{equation*}
  Porque tenemos:
  \begin{equation*}
    \lim_{n \rightarrow \infty} \frac{n^b}{n^a}
      = \lim_{n \rightarrow \infty} n^{b - a}
      = \infty
  \end{equation*}
  Por el teorema~\ref{theo:limite-O-Omega} se cumple lo indicado.
  Lo otro se demuestra de forma similar.

  También resulta para \(a \ge 0\) y \(b > 1\):
  \begin{equation*}
    n^a = O(b^n)
  \end{equation*}
  Si \(a = 0\),
  el resultado es inmediato.
  Si \(a > 0\),
  hacemos:
  \begin{equation*}
    \lim_{n \rightarrow \infty} \frac{n^a}{b^n}
      = \exp\left(
	 \lim_{n \rightarrow \infty}
	     (a \ln n - n \ln b)
	     \right)
      = 0
  \end{equation*}
  El teorema~\ref{theo:limite-O-Omega} entrega lo aseverado.
  Nótese que esto significa,
  por ejemplo,
  que \(n^{100} = O(1,01^n)\),
  aunque las constantes involucradas son gigantescas.

  Hay que tener cuidado de no interpretar \(f(n) = O(g(n))\)
  en el sentido que \(g(n)\) es de alguna forma ``mejor posible''.
  Por ejemplo,
  \(\sqrt{5 n^2 + 2} = O(n^3)\),
  como es fácil demostrar,
  y eso está lejos de ser ajustado.
  Como:
  \begin{equation*}
    \lim_{n \rightarrow \infty} \frac{\sqrt{5 n^2 + 2}}{n}
       = \sqrt{5}
  \end{equation*}
  sabemos del teorema~\ref{theo:limite-O-Omega}
  que \(\sqrt{5 n^2 + 2} = \Theta(n)\).
  Pero también,
  dado que para \(x \ge 0\) es \(1 + x \le (1 + x)^2\),
  o \((1 + x)^{1/2} \le 1 + x\),
  lo que es lo mismo que \((1 + x)^{1/2} = 1 + O(x)\):
  \begin{equation*}
    \sqrt{5 n^2 + 2}
       = n \sqrt{5} \cdot \sqrt{1 + \frac{2}{5 n^2}}
       = n \sqrt{5} \cdot \left( 1 + O(1 / n^2) \right)
       = n \sqrt{5} + O(1 / n)
  \end{equation*}
  Acá usamos que \(f(n) \cdot O(g(n)) = O(f(n) \cdot g(n))\),
  como se demuestra fácilmente.
    \index{Bachmann-Landau, notaciones de!operaciones}
  Este tipo de ideas pueden extenderse bastante,
  hasta obtener reglas para manipular y simplificar
  toda variedad de expresiones asintóticas.
  Para mayores detalles refiérase por ejemplo a
  Graham, Knuth y Patashnik~%
    \cite{graham94:_concr_mathem}
  o a Sedgewick y Flajolet~%
    \cite[capítulo 4]{sedgewick13:_introd_anal_algor}.
  En resumen,
  se aplican las reglas básicas del álgebra,
  con algún cuidado:
  Hay que recordar que las notaciones de Bachmann-Landau
  representan cotas,
  no valores exactos.
  Debemos siempre ponernos en el peor caso,
  en particular no podemos contar con cancelaciones en sumas.
  Así:
  \begin{align*}
    O(f(n)) \pm O(g(n))
      &= O(f(n) + g(n)) \\
    O(f(n)) \cdot O(g(n))
      &= O(f(n) \cdot g(n))
  \end{align*}

  Como un ejemplo obtengamos una aproximación
  para \(\mathrm{e}^{1 / n} \sqrt{n^2 - 2 n}\).
  Primeramente,
  por el teorema de Maclaurin:%
    \index{Maclaurin, teorema de}
  \begin{align*}
    \mathrm{e}^{1 / n}
      &= 1 + \frac{1}{n} + O\left(n^{-2}\right) \\
    \sqrt{n^2 - 2 n}
      &= n \sqrt{1 - 2 / n} \\
      &= n \left(
	      1 - \frac{1}{n}
		- \frac{1}{2 n^2}
		+ O\left(n^{-3}\right)
	    \right)
  \end{align*}
  Multiplicando directamente tenemos:
  \begin{equation*}
     \mathrm{e}^{1 / n} \sqrt{n^2 - 2 n}
      = \left(
	  1 + \frac{1}{n} + O\left(n^{-2}\right)
	\right)
	  \cdot n
	     \left(
	1 - \frac{1}{n} - \frac{1}{2 n^2} + O\left(n^{-3}\right)
	     \right)
      = n - \frac{1}{n} + O(n^{-2})
  \end{equation*}
  Suele ocurrir
  que la cota para el resultado
  es mucho peor que las cotas que entran.
  Si la cota \(O(n^{-2})\) que resulta no fuera suficiente,
  deberemos volver atrás
  y obtener mejores aproximaciones de partida.
  Normalmente deben entrar cotas de la misma precisión
  para no desperdiciarla en el proceso.

\section{Notación asintótica en algoritmos}
\label{sec:asintotica-algoritmos}
\index{notacion asintotica@notación asintótica!algoritmos}

  Comúnmente se usa notación asintótica
  para expresar tiempos de ejecución
  de algoritmos.
  En este tipo de aplicación
  el que \(O(g(n))\) ``oculte'' factores constantes
  es cómodo,
  así los resultados no dependen de detalles de la máquina
  ni de cómo se programó el algoritmo.
  Para obtener esta clase de estimaciones
  basta fijarse en alguna operación clave,
  tal que el costo de las demás operaciones sean proporcionales
  (o menos)
  que las operaciones claves.
  Por ejemplo,
  en el listado~\ref{lst:insercion}
  aparece una rutina de ordenamiento por inserción%
    \index{ordenamiento!insercion@inserción}
  codificada en C~%
    \cite{kernighan88:_c_progr_lang}.%
    \index{C (lenguaje de programacion)@C (lenguaje de programación)}
  \lstinputlisting[language=C,
		   xleftmargin=3em, numbers=left,
		   caption={Ordenamiento por inserción},
		   label=lst:insercion]
		   {code/insertion.c}
  Acá las líneas 6, 7 y~10 se ejecutan \(n - 1\) veces,
  mientras las líneas 8 y~9
  se ejecutan entre \(n - 1\) y \(n (n - 1) / 2\) veces,
  respectivamente cuando el arreglo ya está ordenado
  y si viene exactamente en orden inverso.%
    \index{analisis de algoritmos@análisis de algoritmos!ordenamiento!insercion@inserción}
  Si simplemente contamos ``líneas de C ejecutadas''
  (lo que es válido
   bajo el supuesto que cada línea toma un tiempo máximo,
   independiente de los valores de las variables involucradas)
  para un valor dado de \(n\),
  el tiempo de ejecución será alguna expresión de la forma
  \begin{align*}
    T_{\text{min}}(n)
      &= 3 (n - 1) + 2 (n - 1) \\
      &= 5 n - 5 \\
    T_{\text{max}}(n)
      &= 3 (n - 1) + 2 \, \frac{n (n - 1)}{2} \\
      &= n^2 + 2 n - 3
  \end{align*}
  Podemos decir
  que para el tiempo de ejecución \(T(n)\) del programa
  tenemos una cota superior dado que los ciclos anidados
  de las líneas~6 a~11
  dan que la línea~9 se ejecuta a lo más \(n^2\) veces,
  mientras que en el mejor caso se ejecuta solo \(n\) veces,
  lo que da:
  \begin{align*}
    T(n)
      &= O(n^2) \\
    T(n)
      &= \Omega(n)
  \end{align*}
  Esto coincide con lo obtenido antes.
  Puede verse que esta clase de estimaciones son simples de obtener,
  y son más sencillas de usar que funciones detalladas.
  Además tienen la ventaja de obviar posibles diferencias
  en tiempos de ejecución entre instrucciones.
  Cabe notar que las anteriores cotas son las mejores posibles,
  y en este caso no se puede obtener
  la misma función como cota inferior y superior
  (\(\Theta\)).
  De todas formas,
  es útil contar con valores asintóticos del tiempo de ejecución~%
    \cite{sedgewick13:_introd_anal_algor},
  el que un \(O(\cdot)\)
  o incluso \(\Theta(\cdot)\) oculte constantes
  hace fácil obtener la cota,
  pero poder decir que el tiempo de ejecución
  (o el número de ciertas operaciones)
  cumple \(T(n) \sim a g(n)\) es mucho más valioso.

  Está claro que se puede hacer un análisis mucho más detallado,
  contabilizando cada tipo de operación que ejecuta el programa,
  y considerando el tiempo que demanda
  en una implementación particular.%
    \index{analisis de algoritmos@análisis de algoritmos}
  El ejemplo clásico es el monumental trabajo de Knuth~%
    \cite{knuth97:_fundam_algor,
	  knuth97:_semin_algor,
	  knuth98:_sortin_searc,
	  knuth11:_combin_alg_1}
    \index{Knuth, Donald E.}
  en análisis de algoritmos.
  Esto es mucho más trabajo,
  y en caso de cambiar de plataforma
  (diferente compilador,
   cambian opciones de compilación,
   otro lenguaje o nuevo computador)
  gran parte del análisis hay que repetirlo.
  Para la mayor parte de los efectos
  basta con nuestro análisis somero.
  Si tenemos que elegir entre un algoritmo
  con tiempo de ejecución \(O(n^2)\)
  y otro con tiempo de ejecución \(O(n^3)\),
  para valores suficientemente grandes de \(n\) ganará el primero.
  Sin embargo,
  puede ocurrir que los valores de \(n\) de interés práctico
  sean más importantes
  los factores constantes ocultados por la notación.
  Mucho más detalle de cómo lograr esta clase de estimaciones
  se encuentra en textos sobre algoritmos o estructuras de datos,
  por ejemplo en~%
    \cite{aho74:_design_anal_comp_algor,
	  cormen09:_introd_algor,
	  sedgewick13:_introd_anal_algor,
	  skiena08:_algor_desig_manual}.
  Otros algoritmos son mucho más complejos de manejar,
  el desarrollo de mejores algoritmos y el análisis de su desempeño
  son áreas de investigación activa.
  En~%
     \cite{bentley82:_writing_efficent_programs,
	   bentley00:_progr_pearl,
	   kernighan99:_practice_progr}
  se discute cómo llevar algoritmos a buenos programas.

  Los algoritmos efectúan operaciones discretas
  sobre estructuras de datos discretas,
  evaluar su rendimiento es estudiar esas estructuras
  y contar las operaciones efectuadas sobre ellas.
  Esta es una de las razones que hacen que las matemáticas discretas
  sean fundamentales en la informática.

%%% Local Variables:
%%% mode: latex
%%% TeX-master: "clases"
%%% End:


% relaciones-funciones.tex
%
% Copyright (c) 2009-2014 Horst H. von Brand
% Derechos reservados. Vea COPYRIGHT para detalles

\chapter{Relaciones y funciones}
\label{cha:relaciones-funciones}

  Conceptos básicos de todas las matemáticas
  son los de \emph{relación} y \emph{función}.
  A pesar de su simplicidad,
  ofrecen aspectos de bastante interés.
  Nuevamente,
  la mayor parte del material presentado
  debe considerarse como sistematización de conocimientos previos.
  Una buena sistematización del área es el texto de Düntsch y Gediga~%
    \cite{duentsch00:_sets_relat_funct}.

\section{Relaciones}
\label{sec:relaciones}
\index{relacion@relación|textbfhy}

  \begin{definition}
    Sean \(\mathcal{A}\) y \(\mathcal{B}\) conjuntos.
    Una \emph{relación} \(R\) entre \(\mathcal{A}\) y \(\mathcal{B}\)
    es un subconjunto de \(\mathcal{A} \times \mathcal{B}\).

    Si \((a, b) \in R\),
    se anota \(a \mathrel{R} b\).
    Similarmente,
    para \((a, b) \notin R\)
    se anota \(a \mathrel{\centernot R} b\).
  \end{definition}

  Esto en rigor solo describe \emph{relaciones binarias},
  es perfectamente posible considerar relaciones
  de un solo elemento, de dos, tres o más elementos.
  Las relaciones binarias son lejos las más importantes,
  así que nos restringiremos a ellas acá.

  Según esta definición son relaciones ``menor a''
  entre números naturales,
  ``pololea con'' entre personas,
  ``precio de'' entre libros y sus precios en las librerías de la ciudad.
  Nótese que perfectamente pueden haber varios elementos relacionados,
  como por ejemplo \(1 < 2\), \(1 < 17\), \(1 < 31\).
  De la misma forma,
  el mismo libro puede tener precios diferentes
  en distintas librerías,
  pueden haber varias ediciones,
  o una librería tiene copias usadas más o menos deterioradas.

  Relacionado a lo anterior están los siguientes conceptos:
  \begin{definition}
    \index{relacion@relación!transpuesta|textbfhy}
    Sea \(R\) una relación entre \(\mathcal{A}\) y \(\mathcal{B}\).
    A la relación:
    \begin{equation*}
      R^{-1}
	= \{(y, x) \colon x \mathrel{R} y\}
    \end{equation*}
    se le llama la \emph{transpuesta} de \(R\).
  \end{definition}
  \begin{definition}
    \index{relacion@relación!composicion@composición|textbfhy}
    Sea \(R_1\) una relación entre \(\mathcal{A}\) y \(\mathcal{B}\),
    y \(R_2\) una relación entre \(\mathcal{B}\) y \(\mathcal{C}\).
    A la relación \(R\) definida mediante:
    \begin{equation*}
      R = \{(x, z) \colon \exists y \in \mathcal{B} \colon
			    x \mathrel{R_1} y \wedge y \mathrel{R_2} z\}
    \end{equation*}
    se le llama la \emph{composición} de \(R_1\) y \(R_2\),
    y se anota \(R = R_2 \circ R_1\)
  \end{definition}
  Si consideramos que \(x \mathrel{R_1} y\)
  como que la relación \(R_1\) lleva de \(x\) a \(y\),
  y de la misma forma \(y \mathrel{R_2} z\)
  que \(R_2\) lleva de \(y\) a \(z\),
  entonces \(R_2 \circ R_1\)
  es una relación que lleva de \(x\) a \(z\).

  El caso más común de relación es el en que ambos conjuntos son el mismo.
  Si \(R \subseteq \mathcal{U} \times \mathcal{U}\)
  se habla de una relación sobre \(\mathcal{U}\).
  Algunas propiedades de relaciones entre elementos del mismo conjunto
  tienen nombres especiales:
  \begin{definition}
    Sea \(R\) una relación sobre \(\mathcal{U}\).
    Entonces:
    \begin{itemize}
    \item
      \index{relacion@relación!reflexiva|textbfhy}
      Si para todo \(a \in \mathcal{U}\) se cumple \(a \mathrel{R} a\),
      la relación se llama \emph{reflexiva}.
    \item
      \index{relacion@relación!irreflexiva|textbfhy}
      Si para ningún \(a \in \mathcal{U}\) se cumple \(a \mathrel{R} a\),
      la relación se llama \emph{irreflexiva}.
    \item
      \index{relacion@relación!transitiva|textbfhy}
      Si para todo \(a, b, c \in \mathcal{U}\)
      se cumple que si \(a \mathrel{R} b\)
      y \(b \mathrel{R} c\) entonces \(a \mathrel{R} c\)
      la relación se dice \emph{transitiva}.
    \item
      \index{relacion@relación!simetrica@simétrica|textbfhy}
      Si para todo \(a, b \in \mathcal{U}\)
      siempre que \(a \mathrel{R} b\) se tiene que \(b \mathrel{R} a\),
      a la relación se le llama \emph{simétrica}.
    \item
      \index{relacion@relación!antisimetrica@antisimétrica|textbfhy}
      Si para todo \(a, b \in \mathcal{U}\),
      siempre que \(a \mathrel{R} b\)
      y \(b \mathrel{R} a\) entonces \(a = b\)
      se dice que la relación es \emph{antisimétrica}.
    \item
      \index{relacion@relación!total|textbfhy}
      Si para todo \(a, b \in \mathcal{U}\),
      se cumple \(a \mathrel{R} b\) o \(b \mathrel{R} a\),
      se le llama relación \emph{total}.
    \end{itemize}
  \end{definition}
  Ejemplos de relaciones reflexivas son \(=\) y \(\le\) sobre~\(\mathbb{Z}\).
  La relación \(\ne\) no es reflexiva ni transitiva ni antisimétrica,
  pero es simétrica,
  la relación \(<\) en \(\mathbb{R}\) es irreflexiva,
  es transitiva,
  no es simétrica y es antisimétrica
  (en forma vacía,
   ya que no hay \(a, b \in \mathbb{R}\) con \(a < b\) y \(b < a\)).
  La relación ``pololea con'' es simétrica,
  y definitivamente no es antisimétrica.
  La relación \(\ge\) en \(\mathbb{Z}\) es antisimétrica,
  y no es simétrica.
  Tanto \(\le\) como \(\ge\) sobre \(\mathbb{R}\) son totales
  (para cada par \(a\), \(b\)
   se cumple una de \(a \le b\) o \(b \le a\),
   incluso cuando \(a = b\)).
  La relación ``conoce a'' entre personas no es total
  (hay gente que no se conoce entre sí).

% Fixme: Mostrar ejemplos de c/propiedad por separado (o ejercicios...)
% Fixme: Diagramas de Hasse para relaciones, funciones, ...

  Nótese que una relación total
  sobre un conjunto no vacío necesariamente es reflexiva,
  ya que la definición exige que para cualquier par \(a\), \(b\)
  (incluyendo el caso \(a = b\))
  debe darse una de \(a \mathrel{R} b\) o \(b \mathrel{R} a\).

  Algunas combinaciones de las propiedades se repiten frecuentemente
  y llevan a propiedades interesantes de la relación,
  con lo que merecen nombres especiales.
  \begin{definition}
    \index{relacion@relación!equivalencia|textbfhy}
    Sea \(R\) una relación sobre \(\mathcal{U}\).
    Si \(R\) es reflexiva, simétrica y transitiva
    se le llama \emph{relación de equivalencia}.
  \end{definition}
  El caso clásico de relación de equivalencia es la igualdad.
  Otros ejemplos son la congruencia y semejanza geométricas.
  Veremos
  (y usaremos)
  muchas más en lo que sigue.

  La característica más importante de las relaciones de equivalencia
  está dada por el siguiente teorema.
  \begin{theorem}
    \label{theo:clases-equivalencia}
    \index{relacion@relación!equivalencia!clases|see{clases de equivalencia}}
    \index{clases de equivalencia|textbfhy}
    Sea \(R \subseteq \mathcal{U}^2\) una relación de equivalencia.
    Entonces los conjuntos definidos por:
    \begin{equation*}
      [a]_R
	= \{x \in \mathcal{U} \colon a \mathrel{R} x\}
    \end{equation*}
    son disjuntos y su unión es todo \(\mathcal{U}\).
  \end{theorem}
  \begin{proof}
    Hay dos cosas que demostrar acá:
    \begin{enumerate}
    \item
      \(\displaystyle \bigcup_{a \in \mathcal{U}} [a]_R = \mathcal{U}\)
    \item
      \(\displaystyle [a]_R \cap [b]_R = \varnothing\) o \([a]_R = [b]_R\)
    \end{enumerate}

    Para el primer punto,
    por reflexividad
    \(x \in [x]_R\),
    con lo que todo elemento \(x \in \mathcal{U}\)
    aparece al menos en la clase \([x]_R\),
    y la unión de todas las clases es \(\mathcal{U}\).

    Para el segundo punto
    consideremos dos elementos distintos \(a, b \in \mathcal{U}\),
    y veamos las clases \([a]_R\) y \([b]_R\).
    Si estos conjuntos son disjuntos,
    no hay nada que demostrar.
    Si no son disjuntos,
    habrá \(x \in \mathcal{U}\)
    en la intersección,
    o sea \(x \in [a]_R\) y \(x \in [b]_R\).
    Tenemos:
    \begin{alignat}{2}
      x &\mathrel{R} a
	&\qquad
	& \text{por suposición}		\label{eq:EC-1} \\
      x &\mathrel{R} b
	&&\text{por suposición}		\label{eq:EC-2} \\
      a &\mathrel{R} x
	&& \text{por simetría de \(R\) y~\eqref{eq:EC-2}} \label{eq:EC-3} \\
      a &\mathrel{R} b
	&& \text{por transitividad de \(R\) con~\eqref{eq:EC-2}
		 y~\eqref{eq:EC-3}} \label{eq:EC-4} \\
    \intertext{
      Ahora bien,
      si elegimos \(y \in [a]_R\):
    }
      y &\in [a]_R
	&& \text{por suposición} \label{eq:EC-5} \\
      y &\mathrel{R} a
	&& \text{por definición de \([a]_R\)} \label{eq:EC-6} \\
      y &\mathrel{R} b
	&& \text{por transitividad,
		 de~\eqref{eq:EC-6} con~\eqref{eq:EC-4}} \label{eq:EC-7} \\
      y &\in [b]_R
	&& \text{por definición de \([b]_R\)} \label{eq:EC-8}
    \end{alignat}
    Vale decir \([a]_R \subseteq [b]_R\).
    Por simetría,
    también \([b]_R \subseteq [a]_R\),
    y así \([a]_R = [b]_R\).
  \end{proof}
  \begin{definition}
    A los conjuntos \([a]_R\) les llamamos
    \emph{clases de equivalencia de \(R\)},
    al conjunto \([a]_R\) le llamamos
    la \emph{clase de equivalencia de \(a\) (en \(R\))}.
  \end{definition}
  Omitiremos la relación en la notación de clases de equivalencia
  cuando se subentienda cuál es la relación considerada.

  A la situación del teorema~\ref{theo:clases-equivalencia}
  se le dice que las clases \emph{particionan} el conjunto \(\mathcal{U}\).%
    \index{conjunto!particiones|textbfhy}
  Las clases corresponden precisamente
  a los conjuntos de elementos que la relación considera ``equivalentes''.
  \begin{example}
    Una relación \(R\) tal que si \(a \mathrel{R} b\) y \(a \mathrel{R} c\)
    entonces \(b \mathrel{R} c\) se llama \emph{euclidiana}.
    Demuestre que toda relación euclidiana reflexiva
    es una relación de equivalencia.

    Para demostrar que \(R\) es relación de equivalencia,
    debemos demostrar que es reflexiva,
    simétrica y transitiva.
    La relación dada es reflexiva por hipótesis,
    falta demostrar las otras dos propiedades.
    \begin{description}
    \item[Simetría:]
      Supongamos que \(a \mathrel{R} b\).
      Por reflexividad de \(R\),
      sabemos que \(a \mathrel{R} a\);
      y de \(a \mathrel{R} b\) y \(a \mathrel{R} a\)
      deducimos \(b \mathrel{R} a\).
    \item[Transitividad:]
      Supongamos \(a \mathrel{R} b\) y \(b \mathrel{R} c\).
      Por simetría,
      es \(b \mathrel{R} a\);
      y de \(b \mathrel{R} a\) y \(b \mathrel{R} c\)
      concluimos \(a \mathrel{R} c\).
    \end{description}
  \end{example}
  \begin{example}
    Consideremos las siguientes relaciones.
    Nótese que para demostrar que una de las propiedades \emph{no} vale,
    basta encontrar un único caso en que falla;
    para demostrar que \emph{si} vale hay que cubrir todos los casos.
    \begin{description}
    \item[\boldmath \(a \mathrel{R_1} b\) si \(a b = 100\),
	  sobre \(\mathbb{N}\):\unboldmath]
      Analizamos las distintas propiedades en turno:
      \begin{itemize}
      \item
	No es reflexiva,
	ya que por ejemplo \(5\cdot 5 \ne 100\).
      \item
	Es simétrica
	(si \(a b = 100\), entonces \(b a = 100\)).
      \item
	No es transitiva,
	ya que \(a b = 100\) y \(b c = 100\) no significa \(a c = 100\).
	Por ejemplo, tenemos \(5 \cdot 20 = 100\) y \(20 \cdot 5 = 100\),
	pero \(5 \cdot 5 \ne 100\).
      \item
	No es antisimétrica,
	ya que por ejemplo \(5 \mathrel{R_1} 20\)
	y \(20 \mathrel{R_1} 5\),
	pero \(5 \ne 20\).
      \item
	No es total,
	ya que por ejemplo el natural \(3\)
	no tiene ningún natural relacionado.
      \end{itemize}
    \item[\boldmath \(a \mathrel{R_2} b\) si \(a + b\) es par,
	  sobre \(\mathbb{N}\):\unboldmath]
      Esta es una relación de equivalencia.
      Las clases son los números pares y los impares.
      No es total.
    \item[\boldmath \(x \mathrel{R_3} y\) siempre que \(x - y\) es racional,
	  sobre \(\mathbb{R}\):\unboldmath]
      Es relación de equivalencia.
      En detalle:
      \begin{description}
      \item[Reflexiva:]
	\(x \mathrel{R_3} x\) corresponde a \(x - x\) racional,
	y \(0\) es racional.
      \item[Transitiva:]
	Si \(x \mathrel{R_3} y\) y también \(y \mathrel{R_3} z\),
	quiere decir que \(x -y\) e \(y - z\) son racionales,
	con lo que \(x - z = (x - y) + (y - z)\)
	también es racional.
      \item[Simétrica:]
	Si \(x \mathrel{R_3} y\),
	entonces \(x - y\) es racional,
	y lo es \(-(x - y) = y - x\),
	o sea,
	\(y \mathrel{R_3} x\).
      \end{description}
      Sabemos que hay clases de equivalencia de \(R_3\),
      aunque no son fáciles de describir.
    \item[\boldmath \((x_1, y_1) \mathrel{R_4} (x_2, y_2)\)
	  cuando \(x_1^2 + y_1^2 = x_2^2 + y_2^2\),
	  sobre \(\mathbb{R}^2\):\unboldmath]
      Equivalencia en \(\mathbb{R}^2\).
      Las clases de equivalencia son circunferencias de radio \(r\)
      alrededor del origen,
      definidas por \(x^2 + y^2 = r^2\).
    \end{description}
    Considere una relación \(R\) y su transpuesta \(R^{-1}\)
    sobre algún universo \(\mathcal{U}\).
    Veamos qué podemos decir acerca de \(R^{-1}\)
    si sabemos que:
    \begin{description}
    \item[\boldmath \(R\) es simétrica:\unboldmath]
      \index{relacion@relación!simetrica@simétrica!transpuesta}
      Que \(R\) sea simétrica significa
      que siempre que \(a \mathrel{R} b\) también \(b \mathrel{R} a\).
      Expresando esto en términos de \(R^{-1}\),
      es que \(b \mathrel{R^{-1}} a\) siempre que \(a \mathrel{R^{-1}} b\),
      y la transpuesta también lo es.
    \item[\boldmath \(R\) es antisimétrica:\unboldmath]
      \index{relacion@relación!antisimetrica@antisimétrica!transpuesta}
      Similar al caso anterior,
      se ve que en tal caso \(R^{-1}\) también es antisimétrica.
    \item[\boldmath \(R\) es transitiva:\unboldmath]
      \index{relacion@relación!transitiva!transpuesta}
      Si \(R\) es transitiva,
      \(R^{-1}\) también lo es
      (``caminando en la dirección contraria'').
    \item[\boldmath \(R\) es reflexiva:\unboldmath]
      \index{relacion@relación!reflexiva!transpuesta}
      Si \(a \mathrel{R} a\),
      entonces \(a \mathrel{R^{-1}} a\),
      y \(R^{-1}\) también es reflexiva.
    \item[\boldmath \(R\) es total:\unboldmath]
      \index{relacion@relación!total!transpuesta}
      Si \(a \mathrel{R} b\) o \(b \mathrel{R} a\)
      entonces también \(a \mathrel{R^{-1}} b\) o \(b \mathrel{R^{-1}} a\),
      y \(R^{-1}\) también es total.
    \end{description}
    Se recomienda al lector analizar en detalle estas observaciones,
    algunas no son tan simples como parecen.
  \end{example}
  \begin{definition}
    \index{relacion@relación!orden|textbfhy}
    Sea \(R\) una relación sobre un conjunto \(\mathcal{U}\).
    Si \(R\) es reflexiva, transitiva y antisimétrica
    se le llama \emph{relación de orden}.
    A una relación de orden que es total
    se le llama \emph{relación de orden total},
    en caso contrario es \emph{parcial}.
  \end{definition}

  Ejemplos clásicos de relaciones de orden son \(\le\) y \(\ge\).
  En \(\mathbb{Z}\) ambas son totales.
  Otra relación de orden es \(\subseteq\) entre conjuntos.
  No es total,
  ya que dos conjuntos
  no necesariamente se relacionan uno como subconjunto del otro.

  Otro ejemplo es la relación ``divide a'',
  definida sobre \(\mathbb{N}\) mediante:
  \begin{equation*}
    a \mid b
       \text{\ si y solo si existe \(c \in \mathbb{N}\) tal que\ }
	  b = a \cdot c
  \end{equation*}
  Veamos en detalle esto último:
  \begin{description}
  \item[Reflexividad:]
    \(a \mid a\) ya que \(a = a \cdot 1\)
    y \(1 \in \mathbb{N}\).
  \item[Transitividad:]
    \(a \mid b\) y \(b \mid c\)
    significa que existen \(m, n \in \mathbb{N}\)
    tales que:
    \begin{equation*}
      b = a \cdot m \qquad
      c = b \cdot n
    \end{equation*}
    con lo que:
    \begin{equation*}
      c
	= b \cdot n
	= (a \cdot m) \cdot n
	= a \cdot (m \cdot n)
    \end{equation*}
    que es decir \(a \mid c\).
  \item[Antisimetría:]
    Supongamos \(a \mid b\) y \(b \mid a\).
    Entonces existen \(m, n \in \mathbb{N}\) tales que:
    \begin{equation*}
      b = a \cdot m \qquad
      a = b \cdot n
    \end{equation*}
    Esto lleva a:
    \begin{align*}
      a
	&= (a \cdot m) \cdot n \\
      a \cdot 1
	&= a \cdot (m \cdot n) \\
      1
	&= m \cdot n
    \end{align*}
    Si ahora demostramos \(m = 1\),
    tenemos \(b = a \cdot m = a \cdot 1 = a\).
    Esto lo haremos por contradicción.
    Sabemos que \(1 \le m\);
    supongamos entonces que \(1 < m\),
    vale decir que para algún \(c \in \mathbb{N}\):
    \begin{align*}
      1 + c
	&= m \\
      1 \cdot n + c \cdot n
	&= m \cdot n \\
      n + c \cdot n
	&= 1
    \end{align*}
    Esto significa que \(n < 1\),
    y tal \(n\) no existe.
  \end{description}
  Este no es un orden total,
  ya que por ejemplo no se da ni \(6 \mid 15\) ni \(15 \mid 6\).

  Otros ejemplos dan las notaciones asintóticas de Bachmann-Landau%
    \index{Bachmann-Landau, notaciones de}
  definidas en la sección~\ref{sec:notacion-asintotica}.
  Consideremos la relación entre funciones \(f\) y \(g\)
  dada cuando \(f(n) = \Theta(g(n))\).
  Si \(f(n) = \Theta(g(n))\),
  hay \(n_0\) y constantes positivas \(c_1\) y \(c_2\)
  tales que para todo \(n \ge n_0\)
  se cumple \(c_1 g(n) \le f(n) \le c_2 g(n)\).
  Esta relación es reflexiva
  (podemos tomar \(n_0 = 1\), \(c_1 = 1/2\) y \(c_2 = 2\)),
  con lo que \(f(n) = \Theta(f(n))\).
  Es simétrica,
  ya que para \(n \ge n_0\) tenemos:
  \begin{equation*}
    \frac{1}{c_2} \, f(n) \le g(n) \le \frac{1}{c_1} \, f(n)
  \end{equation*}
  vale decir,
  \(g(n) = \Theta(f(n))\).
  También es transitiva,
  ya que si \(f(n) = \Theta(g(n))\)
  y además \(g(n) = \Theta(h(n))\),
  existen constantes \(n_0'\) y \(c_1'\) y \(c_2'\)
  con \(c_1' h(n) \le g(n) \le c_2' h(n)\) cuando \(n \ge n_0'\).
  Al tomar \(n \ge \max(n_0, n_0')\),
  combinando resulta
  \(c_1 c_1' h(n) \le f(n) \le c_2 c_2' h(n)\).
  Esto corresponde a la definición de \(f(n) = \Theta(g(n))\).

  Si consideramos \(f(n) = \Theta(g(n))\)
  como una especie de ``igualdad'' entre funciones,
  un desarrollo similar hará considerar \(f(n) = O(g(n))\)
  como un ``menor o igual que'',
  y similarmente \(f(n) = \Omega(g(n))\) como ``mayor o igual a''.
  Nótese eso sí que estas relaciones \emph{no} son totales.
  Tómense por ejemplo las funciones:
  \begin{align*}
    f(n)
      &= \begin{cases}
	   1 & \text{si \(n\) es par} \\
	   n & \text{caso contrario}
	 \end{cases}\\
    g(n)
      &= \begin{cases}
	   n & \text{si \(n\) es par} \\
	   1 & \text{caso contrario}
	 \end{cases}
  \end{align*}
  No se cumple \(f(n) = \Omega(g(n))\)
  ni \(f(n) = O(g(n))\),
  y tampoco \(g(n) = \Omega(f(n))\)
  ni \(g(n) = O(f(n))\).
  Estas funciones resultan ser no comparables.

\section{Funciones}
\label{sec:funciones}
\index{funcion@función|textbfhy}

  Históricamente,
  en los inicios del análisis se hablaba de ``curvas''
  (ver por ejemplo incluso el título del texto de l'Hôpital~%
    \cite{lHopital96:_analy_infin_petit_lignes_courb}),
  el que Euler hablara de ``funciones''
  (que inicialmente definiera esencialmente como expresiones algebraicas,
   para más adelante acercarse al concepto actual)
  fue un importante avance.
  El concepto se refinó,
  llegando a hacerse central en matemática
  en su forma actual,
  en que la función \(f\) asigna exactamente un valor \(f(x)\)
  a cada \(x\).

  Una \emph{función} es simplemente un tipo especial de relación.
  Para ser más precisos:
  \begin{definition}
    Sean \(\mathcal{D}\) y \(\mathcal{R}\) conjuntos.
    Decimos que una relación \(f\)
    es una \emph{función de \(\mathcal{D}\) a \(\mathcal{R}\)}
    si a cada \(x \in \mathcal{D}\)
    le relaciona exactamente un elemento \(z\) de \(\mathcal{R}\).
    Se anota \(f \colon \mathcal{D} \rightarrow \mathcal{R}\),
    llamamos \emph{dominio} a \(\mathcal{D}\)%
      \index{funcion@función!dominio|textbfhy}
    y \emph{codominio}%
      \index{funcion@función!codominio|textbfhy}
    o \emph{recorrido}%
      \index{funcion@función!recorrido|textbfhy}
    a \(\mathcal{R}\).
    Si \((x, z) \in f\) llamamos a \(z\) el \emph{valor de \(f\) en \(x\)},
    o \emph{imagen} de \(x\),%
      \index{funcion@función!imagen|textbfhy}
    y anotamos \(f(x)\).
    Al conjunto de todos los valores de la función
    se le llama su \emph{rango}.%
      \index{funcion@función!rango|textbfhy}
    Se le llama \emph{preimagen de \(z\)}%
      \index{funcion@función!preimagen|textbfhy}
    a cualquier \(x\) tal que \(f(x) = z\).
  \end{definition}

  Los puntos centrales de la definición son:
  \begin{enumerate}
  \item
    \(f(x)\) está definido para todos los \(x \in \mathcal{D}\).
  \item
    A cada \(x \in \mathcal{D}\)
    la función le asigna exactamente un valor en \(\mathcal{R}\).
  \end{enumerate}
  En vez de escribir \(f(x) = x^2 + 2\)
  anotaremos también \(f \colon x \mapsto x^2 + 2\).

  El caso más común de funciones en matemática elemental
  tiene dominio y rango conjuntos de números,
  por ejemplo \(\mathbb{N}\).
  Si rango y dominio son conjuntos de números,
  el método más simple de especificar la función
  es mediante una fórmula,
  como:
  \begin{equation*}
    f(n) = n^2 + n + 41
  \end{equation*}
  No siempre es posible llegar a una fórmula cerrada.
  En el caso particular en que el dominio es \(\mathbb{N}\)
  una alternativa es usar una definición recursiva.
  Un ejemplo importante es:
  \begin{equation*}
    f(1)
      = 1, \quad
	f(2)
	  = 1, \quad
	f(n + 2)
	  = f(n + 1) + f(n)\;\,(n \ge 1)
  \end{equation*}
  Esta función es la secuencia de los números de Fibonacci,
    \index{Fibonacci, numeros de@Fibonacci, números de}
  que comienza:
  \begin{equation*}
    \left\langle 1, 1, 2, 3, 5, 8, 13, 21, 34, \dotsc \right\rangle\
  \end{equation*}

  Nuevamente,
  hay clasificaciones:
  \begin{definition}
    Sea \(f \colon \mathcal{D} \rightarrow \mathcal{R}\) una función.
    Entonces:
    \begin{itemize}
    \item
      \index{funcion@función!inyectiva|textbfhy}%
      \index{funcion@función!uno a uno}
      Si para todo \(x, y \in \mathcal{D}\)
      con \(x \ne y\), \(f(x) \ne f(y)\),
      se dice \emph{inyectiva}
      (o \emph{uno a uno}).
    \item
      \index{funcion@función!sobreyectiva|textbfhy}
      Si para todo \(y \in \mathcal{R}\)
      hay \(x \in \mathcal{D}\) tal que \(f(x) = y\)
      se le llama \emph{sobreyectiva}
      (o simplemente \emph{sobre}).
    \item
      \index{funcion@función!biyectiva|textbfhy}%
      \index{biyeccion@biyección|see{función!biyectiva}}
      Si la función es inyectiva y sobreyectiva se dice \emph{biyectiva},
      también se le llama \emph{biyección}.
    \end{itemize}
  \end{definition}

  Para demostrar
  que una función \(f \colon \mathcal{X} \rightarrow \mathcal{Y}\)
  es inyectiva,
  debemos demostrar que si \(a \ne b\) entonces \(f(a) \ne f(b)\).
  La manera más sencilla de hacer esto suele ser demostrar el contrapositivo:
  Si \(f(a) = f(b)\),
  entonces \(a = b\).
  Para demostrar
  que una función \(f \colon \mathcal{X} \rightarrow \mathcal{Y}\)
  es sobreyectiva,
  hay que demostrar que para cada \(y \in \mathcal{Y}\)
  hay al menos un \(x \in \mathcal{X}\)
  tal que \(f(x) = y\).
  Para demostrar que una función es biyectiva
  hay que demostrar las dos anteriores.

  Como un ejemplo,
  anotamos para \(a, b \in \mathbb{R}\) con \(a < b\)
  el intervalo abierto \((a, b) = \{x \colon a < x < b\}\),
  y \(\mathbb{R}^+\) para los reales positivos.
  Definimos la función \(f \colon (a, b) \rightarrow \mathbb{R}^+\) mediante:
  \begin{equation*}
    f(t)
      = \frac{t - a}{b - t}
  \end{equation*}
  Primeramente,
  \(f\) es una función,
  ya que a cada \(t \in (a, b)\)
  le asigna un único valor en \(\mathbb{R}^+\).
  También es inyectiva,
  ya que:
  \begin{align*}
    f(t)
      &= z \\
      &= \frac{t - a}{b - t} \\
    t
      &= \frac{a + z b}{z + 1}
  \end{align*}
  Así,
  a un valor dado de \(z > 0\) le corresponde a un único valor de \(t\).
  Además es sobreyectiva,
  ya que si \(z > 0\)
  la última expresión siempre está definida.
  Como hay una biyección entre \(\mathbb{R}\) y un rango,
  podemos concluir que hay tantos números reales en un rango cualquiera
  como el total de los reales.
  Volveremos sobre este punto en el capítulo~\ref{cha:numerabilidad}.

  Podemos construir nuevas funciones partiendo de funciones dadas,
  dado que son simplemente relaciones:
  \begin{definition}
    \index{funcion@función!composicion@composición|textbfhy}
    Sean \(f \colon \mathcal{A} \rightarrow \mathcal{B}\)
    y \(g \colon \mathcal{B} \rightarrow \mathcal{C}\) funciones.
    La \emph{composición} de \(f\) y \(g\) está definida
    ya que son relaciones.
    La \emph{función inversa de \(f\)},
    \(f^{-1} \colon \mathcal{B} \rightarrow \mathcal{A}\)%
      \index{funcion@función!inversa|textbfhy}
    se define como la transpuesta de la relación,
    \(f^{-1}(z) = x\) siempre que \(f(x) = z\).
  \end{definition}
  La función inversa solo puede existir
  si el rango de \(f\) es su recorrido \(\mathcal{B}\)
  (\(f\) es sobreyectiva),
  y además un elemento de \(\mathcal{B}\) tiene una única preimagen
  (\(f\) es inyectiva).
  Combinando ambas,
  \(f\) es biyectiva.
  Además es fácil ver que en tal caso \((f^{-1})^{-1} = f\).

  Si \(f\), \(g\), \(h\) son funciones,
  es simple demostrar que
  \((f \circ g) \circ h = f \circ (g \circ h)\)
  (acá los paréntesis indican en qué orden se componen las funciones).
  \begin{theorem}
    \label{theo:gof}
    Sean \(f \colon \mathcal{A} \rightarrow \mathcal{B}\)
    y \(g \colon \mathcal{B} \rightarrow \mathcal{C}\) funciones.
    Entonces:
    \begin{enumerate}
    \item
      \label{theo:gof:inyectivas}
      Si \(f\) y \(g\) son inyectivas,
      lo es \(g \circ f\).
    \item
      \label{theo:gof:sobreyectivas}
      Si \(f\) y \(g\) son sobreyectivas,
      lo es \(g \circ f\).
    \item
      \label{theo:gof:biyectivas}
      Si \(f\) y \(g\) son biyectivas,
      lo es \(g \circ f\).
      La función inversa de la composición
      es \((g \circ f)^{-1} = f^{-1} \circ g^{-1}\).
    \item
      \label{theo:gof:gof-inyectiva}
      Si \(g \circ f\) es inyectiva,
      \(f\) es inyectiva
      (\(g\) puede no serlo).
    \item
      \label{theo:gof:gof-sobreyectiva}
      Si \(g \circ f\) es sobreyectiva,
      \(g\) es sobreyectiva
      (\(f\) puede no serlo).
    \end{enumerate}
  \end{theorem}
  \begin{proof}
    Demostramos cada punto por turno.
    \begin{enumerate}
    \item
      Supongamos \((g \circ f)(x) = (g \circ f)(y)\).
      Esto es \(g(f(x)) = g(f(y))\),
      y como supusimos \(g\) inyectiva,
      quiere decir que \(f(x) = f(y)\),
      y esto a su vez que \(x = y\) ya que \(f\) es inyectiva.
    \item
      Si \(f\) y \(g\) son sobreyectivas,
      quiere decir que para cada \(c \in \mathcal{C}\)
      hay algún \(b \in \mathcal{B}\) tal que \(g(b) = c\),
      y también que para cada \(b \in \mathcal{B}\) hay \(a \in \mathcal{A}\)
      tal que \(f(a) = b\).
      Combinando estas,
      para cada \(c \in \mathcal{C}\) hay algún \(a \in \mathcal{A}\)
      tal que \(g(f(a)) = c\),
      que es decir \((g \circ f)(a) = c\),
      y \(g \circ f\) es sobreyectiva también.
    \item
      Esto se obtiene combinando las partes~(\ref{theo:gof:inyectivas})
      y~(\ref{theo:gof:sobreyectivas}).
      La función \(g \circ f\) lleva
      (vía \(f\))
      de \(\mathcal{A}\) a \(\mathcal{B}\),
      y luego
      (vía \(g\))
      de \(\mathcal{B}\) a \(\mathcal{C}\).

      Para la inversa de \(g \circ f\)
      usamos asociatividad y la definición de inversa:
      \begin{equation*}
	(f^{-1} \circ g^{-1}) \circ (g \circ f)
	  = f^{-1} \circ (g^{-1} \circ g) \circ f
	  = f^{-1} \circ f
	  = \iota
      \end{equation*}
      Terminamos con la función identidad,
      y componer por la derecha se trata de forma análoga
      resultando la identidad también.
      Concluimos que \(f^{-1} \circ g^{-1}\) es la inversa prometida.
    \item
      Sean \(x, y \in \mathcal{A}\) tales que \(f(x) = f(y)\).
      Entonces	\(g(f(x)) = g(f(y))\),
      y como suponemos \(g \circ f\) inyectiva,
      \(x = y\),
      con lo que \(f\) es inyectiva.
      Mostraremos más adelante que \(g\) no tiene porqué ser inyectiva.
    \item
      Como \(g \circ f\) es sobreyectiva,
      para cada \(c \in \mathcal{C}\) hay \(a \in \mathcal{A}\)
      para el cual \((g \circ f)(a) = g(f(a)) = c\),
      y podemos elegir \(b = f(a)\)
      para demostrar que para todo \(c \in \mathcal{C}\)
      hay \(b \in \mathcal{B}\)
      tal que \(g(b) = c\).
      Mostraremos más adelante que \(f\) no tiene porqué ser sobreyectiva.
      \qedhere
    \end{enumerate}
  \end{proof}
  Nótese que en la demostración de la parte~(\ref{theo:gof:gof-inyectiva})
  nada podemos concluir sobre \(g\),
  puede ser que en lo anterior no ``usamos'' valores de \(g\)
  que hacen fallar la inyección.
  Ver figura~\ref{fig:gof-is:inyectiva}.
  De forma similar,
  en la parte~(\ref{theo:gof:gof-sobreyectiva})
  nada podemos decir sobre \(f\),
  ver figura~\ref{fig:gof-is:sobreyectiva}.
  \begin{figure}[htbp]
    \centering
    \subfloat[Inyectiva]
	     {\pgfimage[height=2in]{images/gof-inyectiva}
		\label{fig:gof-is:inyectiva}}
    \hspace{1in}
    \subfloat[Sobreyectiva]
	     {\pgfimage[height=2in]{images/gof-sobreyectiva}
		\label{fig:gof-is:sobreyectiva}}
    \caption{Funciones compuestas inyectivas y sobreyectivas}
    \label{fig:gof-is}
  \end{figure}

  En el caso especial en que dominio y codominio son iguales,
  podemos hacer algunas cosas adicionales:
  \begin{itemize}
  \item
    La \emph{función identidad},%
      \index{funcion@función!identidad|textbfhy}
    \(\iota \colon \mathcal{A} \rightarrow \mathcal{A}\),
    se define mediante \(\iota(x) = x\) para todo \(x \in \mathcal{A}\).
    Claramente cumple con \(\iota \circ f = f \circ \iota = f\)
    para toda función \(f\).
    Además,
    \(\iota^{-1} = \iota\).
  \item
    Es obvio de la definición que \(f \circ f^{-1} = f^{-1} \circ f = \iota\)
    si \(f\) es biyectiva.
  \end{itemize}

  En el caso de una función \(f \colon \mathcal{X} \rightarrow \mathcal{Z}\),
  con \(\mathcal{A} \subseteq \mathcal{X}\)
  se usa la notación
    \(f(\mathcal{A}) = \{f(x) \colon x \in \mathcal{A}\}\)
  para la imagen de un subconjunto del dominio.%
    \index{funcion@función!imagen de un conjunto|textbfhy}
  Igualmente,
  para \(\mathcal{B} \subseteq \mathcal{Z}\)
  se anota \(f^{-1} (\mathcal{B}) = \{x \colon f(x) \in \mathcal{B}\}\)
  para la preimagen de un conjunto.%
    \index{funcion@función!preimagen de un conjunto|textbfhy}
  En el último caso la función inversa no tiene porqué existir
  para que la notación tenga sentido.

\section{Operaciones}
\label{sec:operaciones}
\index{operacion@operación|textbfhy}

  Un caso particular importante de funciones son las \emph{operaciones}
  sobre un conjunto \(\mathcal{A}\),
  que formalmente no son más
  que funciones
    \(\operatorname{op} \colon \mathcal{A}^n \rightarrow \mathcal{A}\).
  Nótese que esta definición trae implícito
  que la operación entrega un valor en \(\mathcal{A}\)
  para todos sus posibles argumentos,
  cosa que suele indicarse diciendo que la operación es \emph{cerrada}.%
    \index{operacion@operación!cerrada|textbfhy}
  Se suelen distinguir operaciones \emph{unarias}%
    \index{operacion@operación!unaria|textbfhy}
  si tienen un único argumento,
  y \emph{binarias} si tienen dos.%
    \index{operacion@operación!binaria|textbfhy}
  Pueden considerarse operaciones de más de dos argumentos,
  pero son raras en la práctica.
  Nótese también que por ejemplo la división entre números reales
  no es una operación según nuestra definición
  (\(a / b\) no está definido si \(b = 0\)).
  Tampoco lo son las relaciones,
  ya que por ejemplo \(a < b\) toma dos reales y entrega verdadero o falso.

  Las operaciones de uso común suelen anotarse en forma especial,
  por ejemplo para la operación binaria
  de suma de \(a\) y \(b\) anotamos \(a + b\).
  Esto se conoce como \emph{notación infijo}.%
    \index{operacion@operación!notacion infijo@notación infijo|textbfhy}
  Para operaciones unarias es posible la notación \emph{prefijo}%
    \index{operacion@operación!notacion prefijo@notación prefijo|textbfhy}
  (como en \(- a\) o \(\tan \alpha\))
  o \emph{postfijo}%
    \index{operacion@operación!notacion postfijo@notación postfijo|textbfhy}
  (es el caso del factorial,
   como en \(n!\)).

  En rigor,
  una expresión como \(a + b + c\) no tiene sentido,
  debiera indicarse el orden en que se efectúan las operaciones
  mediante paréntesis.
  Para ahorrar notación,
  se suelen adoptar convenciones:
  Si \(a \circ b \circ c\) ha de interpretarse como
  \((a \circ b) \circ c\)
  se dice que la operación \emph{asocia hacia la izquierda}%
    \index{operacion@operación!asociativa izquierda|textbfhy}
  (o que es \emph{asociativa izquierda}),
  si en cambio \(a \circ b \circ c\)
  significa \(a \circ (b \circ c)\)
  se dice que \emph{asocia hacia la derecha}%
    \index{operacion@operación!asociativa derecha|textbfhy}
  o es \emph{asociativa derecha}.
  En caso que \(a \circ b \circ c\) no se le dé sentido,
  se le llama \emph{no asociativa}.%
    \index{operacion@operación!no asociativa|textbfhy}
  Las operaciones comunes se consideran todas asociativas izquierdas;
  salvo las potencias,
  que son asociativas derechas
  (o sea,
   \(2^{3^4} = 2^{(3^4)}\)).

  Si tenemos dos operaciones \(\odot\) y \(\oplus\),
  y se interpretan
  \(a \odot b \oplus c\) como \((a \odot b) \oplus c\)
  y \(a \oplus b \odot c\) como \(a \oplus (b \odot c)\)
  (siempre se efectúa \(\odot\) antes de \(\oplus\))
  se dice que \(\odot\) tiene \emph{mayor precedencia} que \(\oplus\).
    \index{operacion@operación!precedencia|textbfhy}
  En caso que ambas operaciones sean asociativas izquierdas,
  y \(a \odot b \oplus c\) se interpreta como \((a \odot b) \oplus c\)
  y \(a \oplus b \odot c\) como \((a \oplus b) \odot c\)
  (siempre se efectúan las operaciones de izquierda a derecha),
  se dice que tienen la \emph{misma precedencia}.

  Hay ciertas propiedades de las funciones mismas que son de interés también.
  \begin{itemize}
  \item
    \index{operacion@operación!conmutativa|textbfhy}
    Si \(a \circ b = b \circ a\) la operación se dice \emph{conmutativa}.
  \item
    \index{operacion@operación!asociativa|textbfhy}
    Si \((a \circ b) \circ c = a \circ (b \circ c)\),
    la operación se llama \emph{asociativa}.
    Esto no debe confundirse
    con las convenciones de notación mencionadas antes.
  \item
    \index{operacion@operación!elemento neutro|textbfhy}
    De existir un elemento \(e\) tal que
    \(a \circ e = e \circ a = a\) para todos los \(a\),
    a \(e\) se le llama \emph{elemento neutro} de la operación.
  \item
    \index{operacion@operación!distributiva|textbfhy}
    Si siempre se cumple
    \((a \oplus b) \odot c = (a \odot c) \oplus (b \odot c)\),
    se dice que \(\odot\) \emph{distribuye sobre} \(\oplus\)
    \emph{por la derecha},
    en forma similar
    si \(a \odot (b \oplus c) = (a \odot b) \oplus (a \odot c)\)
    \emph{por la izquierda}.
    Si \(\odot\) distribuye sobre \(\oplus\) tanto por la derecha
    como por la izquierda,
    se dice simplemente que distribuye sobre ella.
  \end{itemize}

%%% Local Variables:
%%% mode: latex
%%% TeX-master: "clases"
%%% End:


% demostraciones.tex
%
% Copyright (c) 2009-2014 Horst H. von Brand
% Derechos reservados. Vea COPYRIGHT para detalles

\chapter{Demostraciones}
\label{cha:demostraciones}
\index{demostracion@demostración|textbfhy}

  La forma general de funcionar de las matemáticas
  es deducir nuevos resultados partiendo de cosas ya demostradas.
  Se busca tener una cadena sólida de deducciones,
  no se aceptan ``cosas obvias'' ni razonamientos por analogía.
  Lo que se entiende como prueba
  es mucho más riguroso que lo aceptado en otras áreas,
  lo que se busca es que no quede ningún espacio posible de duda.
  En este capítulo comentaremos del razonamiento matemático
  y discutiremos técnicas de demostración comunes,
  mostrando algunas aplicaciones interesantes de las mismas.

\section{Razonamiento matemático}
\label{sec:razonamiento-matematico}
\index{razonamiento matematico@razonamiento matemático}

  Aun insistiendo sobre demostraciones rigurosas que no dejan espacio a dudas,
  las matemáticas son una actividad humana,
  repleta de errores y paradojas,
  como relatan Kleiner y Movshovitz-Hadar~%
    \cite{kleiner94:_role_paradox_evol_math}.
  Incluso Barbeau~\cite{barbeau00:_math_fallacies_flaws_flimflam}
  se dedicó durante más de una década a recopilar errores,
  desde desarrollos completamente incorrectos
  que llevan a la solución correcta
  a razonamientos cuyo rango de validez no es simple de determinar,
  y los discute en detalle.

  Entre otros,
  Wigner ha dicho que la efectividad de las matemáticas
  en las ciencias naturales
  no es para nada razonable~%
    \cite{wigner60:_unreas_effec_math_nat_sci}.
  Hamming considera que simplicidad de las matemáticas
  y la aplicabilidad de los mismos conceptos
  en áreas totalmente diferentes
  no tiene explicación racional~%
    \cite{hamming80:_unreas_effectiveness_math}.
  Renz expone que el papel de una demostración en las matemáticas
  ha ido cambiando,
  junto con lo que se considera una demostración válida~%
    \cite{renz81:_math_proof}.
  Thurston~\cite{thurston94:_proof_progr_math}
  da su visión como matemático profesional.
  Precisamente el siglo~XX vio profundas controversias
  sobre el significado de las matemáticas
  y las demostraciones en particular
  (ver Kleiner~%
     \cite{kleiner91:_rigor_proof_math}
   para una perspectiva histórica;
   de~Millo~%
     \cite{demillo79:_social_proces_proof_theor_progr}
   arguye que debe considerarse como una actividad social,
   en las líneas de la definición de ``ciencia'' de Kuhn~%
     \cite{kuhn70:_structure_science_revolutions}).

  Para construir una cadena sólida de conclusiones
  debemos comenzar con alguna base,
  que asumimos verdadera sin demostración.
  Esto
  (que fue el aporte más importante de Euclides)%
    \index{Euclides}
  es lo que se conoce como el \emph{método axiomático},%
    \index{metodo axiomatico@método axiomático|see{axioma}}%
    \index{axioma}%
    \glossary{Axioma}
	     {Aseveración que se asume cierta.
	      Punto de partida del razonamiento matemático.}
  y los puntos de partida son los \emph{axiomas}.
  Claro que Euclides%
    \index{Euclides}
  y sus sucesores hasta la época de Gauß
  (1777--1855)%
    \index{Gauss, Carl Friedrich@Gauß, Carl Friedrich}
  consideraban un axioma como una verdad simple,
  indiscutible.
  La visión actual
  (desde alrededor de 1900,
   de manos particularmente de Hilbert)%
     \index{Hilbert, David}
  es que un axioma
  simplemente describe la relación entre los términos no definidos
  de la teoría entre manos,
  y se usan como punto de partida para deducir nuevas relaciones.
  Cuando los axiomas se cumplen,
  estamos resolviendo problemas en áreas diversas de una sola vez.

  Nótese que en matemática
  (como en las demás ciencias)
  ``teoría'' es un cuerpo organizado de conocimiento%
    \index{teoria@teoría}%
    \glossary{Teoría}
	     {En ciencias, cuerpo organizado de conocimiento
	      con técnicas y reglas asociadas.}
  aplicable a un rango relativamente amplio de situaciones.
  No se refiere,
  como suele usarse el término coloquialmente,
  a una sospecha que puede o no ser cierta.
  Por ejemplo,
  al hablar de grupos%
    \index{grupo}
  (cosa que haremos en mayor detalle
   en la sección~\ref{sec:aritmetica-Zm})
  partimos con un conjunto de elementos y una operación
  que cumplen ciertos axiomas.
  Estas propiedades simples se cumplen
  en una gran variedad de situaciones,
  y la teoría es de amplia aplicación.

  Algunos términos para indicar el papel que una proposición
  tiene en un cuerpo mayor son los siguientes:
  \begin{itemize}
  \item
    \index{teorema|textbfhy}
    Un \emph{teorema} es una proposición importante,
    un resultado central.
    \glossary{Teorema}{Resultado importante, independiente.}
  \item
    \index{corolario|textbfhy}
    Un \emph{corolario} es un resultado
    (teorema)
    que se obtiene casi inmediatamente
    de alguna proposición anterior.
    \glossary{Corolario}{Resultado casi inmediato de una proposición anterior.}
  \item
    \index{lema|textbfhy}
    Un \emph{lema} es un resultado preliminar,
    un paso para demostrar proposiciones más adelante.
    \glossary{Lema}{Resultado auxiliar, usado para demostrar un teorema.}
  \end{itemize}
  Esto no es para nada una división precisa,
  hay lemas que resultaron mucho más importantes que los teoremas
  que se demostraron usándolos.
  Un poco en broma se dice que el sueño de todo matemático
  no es ser conocido por algún teorema,
  sino por un lema.
  No es uniforme el uso de esta nomenclatura,
  hay autores que llaman ``proposición''
  a todos los resultados que demuestran,%
    \index{proposicion@proposición}%
    \glossary{Proposición}
	     {Hay quienes llaman ``proposición'' a todos sus resultados.
	      Reservamos el término para resultados sin mayor importancia
	      en sí mismos.}
  independiente de su importancia
  o de cuán fácil resultan de demostrar
  de proposiciones anteriores.
  Usaremos la división tradicional,
  y nombraremos simplemente ``proposición''
  a un resultado independiente,
  sin mayor relevancia posterior.

  Los teoremas y lemas suelen nombrarse
  por quienes los demostraron por primera vez,
  aunque hay bastantes excepciones a esta regla.
  Por ejemplo,
  el importante resultado conocido como \emph{lema de Burnside}%
    \index{Burnside, lema de}%
    \index{Burnside, William}
  el mismo Burnside se lo atribuye a Frobenius,%
    \index{Frobenius, Ferdinand Georg}
  aunque mucho antes lo había demostrado Cauchy.%
    \index{Cauchy, Augustin-Louis}
  Burnside demostró su utilidad en una influyente publicación~%
    \cite{burnside97:_theor_group_finit_order}.
  Algunos lo llaman ``el lema que no es de Burnside''
  por esta enrevesada historia.
  El \emph{teorema de Borges}%
    \index{Borges, teorema de}
  es llamado así en honor al cuento \emph{La biblioteca de Babel}~%
    \cite{borges41:_biblioteca_babel},
  del ilustre escritor argentino%
    \index{Borges, Jorge Luis}
  por Flajolet y Sedgewick~\cite{flajolet09:_analy_combin}.

  La conocida \emph{regla de l'Hôpital} para calcular límites%
    \index{Hopital, regla de@l'Hôpital, regla de}
  en realidad se debe a Johann Bernoulli,%
    \index{Bernoulli, Johann}
  a quien el marqués de l'Hôpital%
    \index{Hopital, Guillaume Marquis@l'Hôpital, Guillaume Marquis de}
  había contratado como tutor en matemáticas,
  con un contrato que decía en parte
  \emph{``darle sus resultados,
  para usarlos a gusto''.}
  El marqués publicó el primer texto impreso
  de cálculo diferencial~%
    \cite{lHopital96:_analy_infin_petit_lignes_courb},
  basado en gran medida en el trabajo de Bernoulli.
  En el prefacio de su libro l'Hôpital indica
  que usó libremente resultados de otros
  y que felizmente daría el crédito
  a quienes los reclamaran como suyos.
  Igual Bernoulli se quejó amargamente
  que sus aportes no eran reconocidos como debían.
  Y la regla quedó con el nombre del marqués.

  Otro caso interesante lo provee el \emph{postulado de Bertrand},%
    \index{Bertrand, postulado de}%
    \index{Bertrand, Joseph Louis Francois@Bertrand, Joseph Louis François}
  que dice que entre los naturales \(n\) y \(2 n\)
  siempre hay un número primo,
  cosa que notó Bertrand en~1845
  y verificó hasta \(3\,000\,000\)~%
    \cite{bertrand45:_memoir};
  pero fue demostrado por primera vez por Chebyshev~%
    \cite{chebyshev54:_Bertrand}.%
    \index{Chebyshev, Pafnuty Lvovich}
  Luego Ramanujan~%
    \cite{ramanujan19:_proof_postul}%
    \index{Ramanujan, Srinivasa}
  dio una demostración mucho más sencilla,
  que a su vez fue mejorada por Erdős~%
    \cite{erdos30:_beweis_satz_tschebyschef}%
    \index{Erdos, Paul@Erdős, Paul}
  en su primera publicación
  (tenía 19~años).
  Y este resultado se conoce como ``postulado'',
  no como teorema.
  También se conoce
  por los nombres de \emph{teorema de Bertrand-Chebyshev}%
    \index{Bertrand-Chebyshev, teorema de|see{Bertrand, postulado de}}
  o \emph{teorema de Chebyshev}.%
    \index{Chebyshev, teorema de|see{Bertrand, postulado de}}

  Tal vez el caso más famoso
  es el del \emph{último} (o gran) \emph{teorema de Fermat},%
    \index{Fermat, ultimo teorema@Fermat, último teorema}%
    \index{Fermat, Pierre de}
  quien en 1637 anotó en el margen de un libro
  que tenía una maravillosa demostración de que \(x^n + y^n = z^n\)
  no tiene soluciones
  en números naturales \(x\), \(y\), \(z\) si \(n > 2\),
  pero que la demostración no cabía en ese margen.
  Este resultado se demostró recién en 1995~%
    \cite{wiles95:_modul_ellip_curves_Fermat},
  usando técnicas muy nuevas.
  Se le llamó ``último teorema''
  porque de muchos resultados anunciados sin demostración por Fermat
  fue el último en ser resuelto.

  Es común que resultados importantes
  tengan muchas demostraciones diferentes.
  Se ha dicho que el teorema de Pitágoras%
    \index{Pitagoras, teorema de@Pitágoras, teorema de}%
    \index{Pitagoras@Pitágoras}
  es el que más demostraciones tiene,
  Loomis~\cite{loomis68:_pythag_propos}
  lista 367~demostraciones distintas.
  Hay que considerar que un teorema (u otro resultado)
  tiene interés como herramienta a ser aplicada,
  pero su demostración
  también sirve para iluminar relaciones entre distintos resultados.
  Seleccionar la demostración más simple de entender
  es vital a la hora de elegir cómo enseñar a nuevas generaciones,
  muchas de las demostraciones que veremos
  son radicalmente diferentes
  a las demostraciones originales de los mismos resultados
  (a veces incluso cubren el tema en forma mucho más amplia).
  Contar con varias demostraciones independientes
  de resultados importantes
  además ayuda a aumentar la confianza en ellos.

  Hay varios esquemas de demostración
  con las que uno debe familiarizarse.
  En el resto del texto usaremos estos esquemas con frecuencia.
  Algunas pistas adicionales sobre cómo estructurar una demostración
  da Cusick~%
    \cite{cusick:_how_write_proofs}.
  Una discusión mucho más detallada de técnicas de demostración
  que la que puede darse en este exiguo espacio ofrece Hammack~%
    \cite{hammack13:_book_proof},
  incluyendo un amplio rango de ejercicios.
  Zeitz~\cite{zeitz07:_art_craft_probl_solving}
  distingue entre \emph{ejercicios},
  en los cuales el plan de ataque está claro
  (aunque llevarlo adelante puede incluir desarrollos complejos)
  y \emph{problemas},
  en los cuales no está claro de antemano cuál es el camino más adecuado,
  y tal vez siquiera si hay una solución.
  Nos interesa entrenar en resolución de problemas
  más que en solución de ejercicios.
  Taylor~\cite{taylor07:_introd_proof}
  da las siguientes recomendaciones:
  \begin{itemize}
  \item \textbf{Conozca y entienda las definiciones} --
    Razonamiento preciso requiere saber sobre qué estamos razonando.
    Si un término no es familiar,
    busque su definición.
  \item \textbf{Desarrolle ejemplos} --
    Asegúrese que lo que intenta demostrar
    tiene alguna posibilidad de ser cierto.
    Son pocas las instancias en que exhibir un ejemplo
    es demostración suficiente
    (salvo que queramos demostrar que algo existe).
    Igualmente,
    un par de ejemplos ayudan a familiarizarse con el terreno.
    Es un buen momento para verificar
    que aplica correctamente las definiciones.
    Por lo demás,
    en el desarrollo de ejemplos puede tropezar con una idea
    o relación útil
    para demostrar el caso general.
    La inspiración nace en los lugares más extraños.
  \item \textbf{Busque contraejemplos} --
    Si sospecha que lo que intenta demostrar es falso,
    busque un contraejemplo.
    Incluso si es cierto,
    buscar contraejemplos y analizar porqué la búsqueda falla
    puede indicar métodos de ataque.
  \item \textbf{Intente usar las técnicas estándar de demostración} --
    Las técnicas que discutiremos más abajo
    han sido probadas y refinadas
    por generaciones.
    Hay situaciones en que ninguna de ellas es aplicable,
    pero son muy raras.
    No se encasille en una técnica,
    intente variantes.
  \item \textbf{Parta con un esqueleto} --
    Escriba lo que quiere demostrar,
    y un esbozo a llenar para la demostración.
    Vaya completando detalles.
    Puede ser útil trabajar desde ambos extremos
    (desde las hipótesis hacia la conclusión,
     y desde la conclusión hacia las hipótesis),
    en la esperanza que se encuentren al medio.
  \item \textbf{Sea persistente} --
    No se desanime si el primer intento no funciona,
    pruebe otro camino.
  \item \textbf{Navaja de Ockham} --
    Si todo lo demás es igual,
    la solución más simple es mejor.
  \end{itemize}

\section{Desenrollar definiciones}
\label{sec:desenrollar-definciones}

  Una estrategia básica,
  aplicable siempre que no se conozcan relaciones directas
  que ayuden,
  es reducir términos a sus definiciones.

  \begin{definition}
    Un número se dice \emph{algebraico}%
      \index{numero@número!algebraico|textbfhy}
    si es un cero de un polinomio con coeficientes enteros.
  \end{definition}

  \begin{proposition}
    \label{prop:racional+algebraico}
    La suma de un número racional y uno algebraico
    es algebraico.
  \end{proposition}
  No tenemos nada que relacione números racionales y algebraicos,
  así que partimos de las definiciones.
  \begin{proof}
    Sea \(\alpha\) un número algebraico,
    y \(\rho\) un número racional.
    Por definición de número algebraico,
    hay un polinomio
      \(p(x) = a_n x^n + a_{n - 1} x^{n - 1} + \dotsb + a_0\)
    tal que \(p(\alpha) = 0\).
    Por la definición de número racional,%
      \index{numero@número!racional}
    \(\rho = a / b\),
    con \(a\) y \(b\) enteros
    y \(b \ne 0\).
    Vemos que \(b^n (x - \rho)^k = b^{n - k} (b x - a)^k\),
    si \(k \le n\)
    esto último es un polinomio con coeficientes enteros.
    Así \(q(x) = b^n \cdot p(x - \rho)\) es un polinomio
    de coeficientes enteros.
    Pero \(q(\alpha + \rho) = b^n \cdot p(\alpha) = 0\),
    con lo que \(\alpha + \rho\) es un cero de un polinomio
    de coeficientes enteros,
    y \(\alpha + \rho\) es algebraico.
  \end{proof}
  Esta técnica la usaremos con mucha frecuencia en lo que sigue.

  Se suele marcar el comienzo de la demostración
  mediante algo como la palabra ``Demostración'',
  y el fin de la misma mediante algo como \(\Box\)
  o Q.E.D.\
  (abreviatura del latín
   \emph{``\foreignlanguage{latin}{quod erat demostrandum}'',}
   que es decir ``lo que se quería demostrar'').

\section{Implicancias}
\label{sec:implicancias}
\index{demostracion@demostración!implicancia}
\index{implicancia (logica)@implicancia (lógica)}

  Interesan proposiciones de la forma ``Si \(P\), entonces \(Q\)''.
  También se expresan como ``\(P\) implica \(Q\)'',
  diciendo ``\(Q\) es necesario para \(P\)'',%
    \index{necesario|see{implicancia (lógica)}}
  mediante ``\(P\) solo si \(Q\)''
  o también ``\(P\) es suficiente para \(Q\)''.%
    \index{suficiente (logica)@suficiente (lógica)|see{implicancia (lógica)}}
  Esta nomenclatura se aclara si se revisa la tabla de verdad
  de nuestra relación ``implica''.
  Si \(P \implies Q\) es verdadero,
  y \(Q\) es falso,
  definitivamente es falso \(P\);
  por lo que \(P\) verdadero es posible solo si \(Q\) es verdadero.
  Por otro lado,
  si \(P\) es verdadero,
  siempre es verdadero \(Q\);
  pero \(Q\) puede ser verdadero siendo \(P\) falso.

  Proposiciones relacionadas a \(P \implies Q\)
  son su \emph{recíproco} \(Q \implies P\),%
    \index{reciproco@recíproco}
  su \emph{inverso} \(\neg P \implies \neg Q\)%
    \index{inverso}
  y su \emph{contrapositivo} \(\neg Q \implies \neg P\).%
    \index{contrapositivo}
  Es importante distinguirlas;
  la implicancia y su contrapositivo son equivalentes,
  y el recíproco y el inverso son equivalentes
  (el inverso es el contrapositivo del recíproco).
  Note que en inglés el recíproco
  se llama \emph{\foreignlanguage{english}{converse}}.

  El recíproco no siempre se cumple.
  Por ejemplo,
  los primos de la forma \(a^n - 1\)
  deben tener \(n = 1\) o \(a = 2\)
  (por la factorización
     \(a^n - 1 = (a - 1) (a^{n - 1} + \dotsc + 1)\)).
  Además,
  con \(a = 2\) debe ser primo \(n\),
  porque si fuera \(n = u v\) entonces:
  \begin{equation*}
    2^{u v} - 1
      = (2^u - 1) (2^{(v - 1) u} + 2^{(v - 2) u} + \dotsc + 1)
  \end{equation*}
  A estos primos se les llama \emph{primos de Mersenne},
    \index{Mersenne, primo de}%
    \index{Mersenne, Marin}
  en honor a quien los estudió por primera vez.
  Sin embargo,
  el recíproco
  (\(2^p - 1\) es primo si \(p\) es primo)
  no se cumple,
  el primer contraejemplo es \(2^{11} - 1 = 2047 = 23 \cdot 89\).

\subsection{Primer método -- Demostración directa}
\label{sec:implicancias-1}

  Para demostrar que \(P\) implica \(Q\):
  \begin{enumerate}
  \item
    Escriba ``Supongamos \(P\)''
  \item
    Demuestre que \(Q\) es una consecuencia lógica.
  \end{enumerate}

  Por ejemplo:
  \begin{proposition}
    Si \(0 \le x \le 2\),
    entonces \(-x^3 + 4 x + 1 > 0\)
  \end{proposition}
  Antes de demostrar esto,
  haremos algún trabajo de borrador para ver porqué es cierto.
  La desigualdad es cierta si \(x = 0\),
  el lado izquierdo es 1 y \(1 > 0\).
  Al aumentar \(x\),
  el término \(4 x\)
  (que es positivo)
  inicialmente es de mayor magnitud que \(-x^3\)
  (que es negativo).
  Por ejemplo,
  para \(x = 1\) tenemos \(4 x = 4\),
  mientras \(-x^3 = -1\).
  Considerando estos dos términos,
  da la impresión que \(-x^3\)
  recién comienza a dominar cuando \(x > 2\).
  O sea,
  \(-x^3 + 4 x\) no será negativo para todo \(x\) entre 0 y 2,
  lo que significaría que \(-x^3 + 4 x + 1\) es positivo.

  Hasta acá vamos bien.
  Necesitamos reemplazar los ``da la impresión'' y ``parece''
  en lo anterior por pasos lógicos sólidos.
  Una manera de enfrentar el término \(-x^3 + 4 x\) es factorizando:
  \begin{equation}
    \label{eq:ejemplo-demostracion-implica}
    -x^3 + 4 x = x (2 - x) (2 + x)
  \end{equation}
  ¡Bien!
  Para \(0 \le x \le 2\),
  ninguno de los factores
  del lado izquierdo de~\eqref{eq:ejemplo-demostracion-implica}
  es negativo,
  y por tanto el producto no es negativo.

  Formalmente,
  esto queda expresado en la siguiente demostración.
  Probablemente alguien que se encuentre con ella quedará convencido
  que el resultado es correcto,
  pero igual se preguntará de dónde salió este razonamiento.

  \begin{proof}
    Por hipótesis \(0 \le x \le 2\).
    Entonces ninguno de \(x\), \(2 - x\) o \(2 + x\) es negativo,
    y su producto no es negativo.
    Sumando 1 al producto de estos tres factores
    da un resultado positivo:
    \begin{equation*}
      x (2 - x) (2 + x) + 1 = -x^3 + 4 x + 1 > 0
      \qedhere
    \end{equation*}
  \end{proof}

  Un par de puntos acá que son aplicables a toda demostración.
  \begin{itemize}
  \item
    Frecuentemente habrá que hacer trabajo en borrador
    mientras se construye la demostración.
    El trabajo en borrador
    puede ser todo lo desorganizado que se quiera,
    lleno de ideas que no resultaron,
    diagramas extraños,
    palabrotas,
    lo que sea.
  \item
    La versión definitiva
    de la demostración debe ser concisa y clara.
    Las matemáticas tienen sus propias reglas de estética,
    y una demostración elegante es altamente apreciada.
    Un ejemplo es la colección de demostraciones hermosas
    dadas por Aigner y Ziegler~%
      \cite{aigner14:_proof_the_book}.
    Dunham ha comparado
    algunos de los grandes teoremas de las matemáticas
    con las máximas obras de arte~%
      \cite{dunham90:_journey_genius,dunham97:_mathem_univer},
    que debieran apreciarse como tales.
  \item
    Organizar una demostración compleja tiene mucho en común
    con escribir un programa:
    Hay que dividirla en trozos digeribles,
    particularmente en lemas fáciles de usar%
      \index{lema}
    (y reusables).
    Si alguna parte de la demostración es repetitiva,
    tal vez vale la pena abstraerla,
    y explicarla una vez solamente.
  \end{itemize}

\subsection{Segundo método -- Demostrar el contrapositivo}
\label{sec:implicancias-2}
\index{contrapositivo}

  Una implicación ``\(P\) implica \(Q\)'',
  es lógicamente equivalente a su contrapositiva,
  ``No \(Q\) implica no \(P\)'',
  como puede verse de las tablas de verdad correspondientes.
  Demostrar una es tan bueno como demostrar la otra,
  y puede ser mucho más fácil demostrar el contrapositivo.
  De ser así,
  el esquema es:
  \begin{enumerate}
  \item
    Escriba ``Demostraremos el contrapositivo'',
    luego enuncie éste.
  \item
    Aplique alguna de las otras técnicas.
  \end{enumerate}

  \begin{proposition}
    Si \(r\) es irracional,
    entonces \(\sqrt{r}\) también es irracional.
  \end{proposition}

  Recuerde que un número es racional%
    \index{numero@número!racional}
  si es la razón entre números enteros,
  e irracional en caso contrario.%
    \index{numero@número!irracional}
  \begin{proof}
    Demostraremos el contrapositivo:
    Si \(\sqrt{r}\) es racional,
    entonces \(r\) es racional.

    Supongamos que \(\sqrt{r}\) es racional.
    Esto significa que existen enteros \(a\) y \(b\) tales que:
    \begin{equation}
      \label{eq:sqrt(r)-racional}
      \sqrt{r} = \frac{a}{b}
    \end{equation}
    Entonces,
    elevando~\eqref{eq:sqrt(r)-racional} al cuadrado:
    \begin{equation}
      \label{eq:r-racional}
      r = \frac{a^2}{b^2}
    \end{equation}
    Como en~\eqref{eq:r-racional} \(a^2\) y \(b^2\) son enteros,
    \(r\) es racional.
  \end{proof}

\section{Demostrando un ``Si y solo si''}
\label{sec:ssi}
\index{si y solo si (logica)@si y solo si (lógica)}
\index{demostracion@demostración!si y solo si}

  Muchos teoremas aseguran
  que dos proposiciones son lógicamente equivalentes;
  vale decir,
  una vale si y solo si vale la otra.
  A esto también se le llama ``necesario y suficiente'',
    \index{necesario y suficiente (logica)@necesario y suficiente (lógica)|see{si y solo si}}
  como habíamos indicado antes;
  a veces lo expresaremos mediante ``exactamente cuando''.
  También se dice que ``\(P\) implica \(Q\) y a la inversa''.
  La frase ``si y solo si'' aparece tan comúnmente
  que se suele abreviar \emph{ssi}
    \index{ssi (si y solo si)|see{si y solo si}}
  (en inglés,
   ``\emph{\foreignlanguage{english}{if and only if}}''
  se abrevia \emph{\foreignlanguage{english}{iff}}).%
    \index{iff@\emph{\foreignlanguage{english}{iff}}|see{si y solo si}}
  Nos abstendremos de usar estas abreviaturas,
  es demasiado fácil omitir una letra
  (o no verla al leer).

\subsection{Primer método -- Cada una implica la otra}
\label{sec:equivalencias-1}

  La proposición ``\(P\) si y solo si \(Q\)''
  equivale a la conjunción de las dos proposiciones
  ``\(P\) implica \(Q\)''
  y ``\(Q\) implica \(P\)''.
  Así demostramos dos implicancias:
  \begin{enumerate}
  \item
    Escriba ``Demostraremos que \(P\) implica \(Q\), y viceversa''.
  \item
    Escriba ``Primero demostraremos \(P\) implica \(Q\)''.
    Hágalo usando alguna de las técnicas
    para demostrar implicancias.
  \item
    Escriba ``Ahora demostraremos que \(Q\) implica \(P\)''.
    Nuevamente,
    aplique una de las técnicas para demostrar implicancias.
  \end{enumerate}

  Una variante de esto se da
  cuando se quiere demostrar
  que una colección de proposiciones son equivalentes entre sí.
  Por ejemplo,
  para demostrar que \(P\), \(Q\) y \(R\) son equivalentes,
  podemos demostrar que \(P\) implica \(Q\),
  que \(Q\) implica \(R\),
  y finalmente que \(R\) implica \(P\).

  Una situación a primera vista sorprendente
  se da cuando se demuestra el contrapositivo de la conversa.
  O sea,
  demostramos \(P \implies Q\)
  y \(\neg P \implies \neg Q\).

\subsection{Segundo método -- Cadena de equivalencias}
\label{sec:equivalencias-2}

  Otra alternativa es demostrar una cadena de equivalencias.
  Para demostrar que \(P\) si y solo si \(Q\):
  \begin{enumerate}
  \item
    Escriba ``Construimos una cadena de si y solo si''.
  \item
    Demuestre que \(P\) es equivalente a una segunda proposición,
    que es equivalente a una tercera,
    y así sucesivamente hasta llegar a \(Q\).
  \end{enumerate}
  Esto muchas veces requiere más ingenio que el primer método,
  pero suele dar una demostración simple y clara.

  La desviación estándar \(\sigma\) de un conjunto de valores reales
  \(x_1, x_2, \dotsc, x_n\) se define como:
  \begin{equation}
    \label{eq:def-sigma}
    \index{desviacion estandar@desviación estándar|textbfhy}
    \sigma =
      \sqrt{\frac{(x_1 - \mu)^2
		    + (x_2 - \mu)^2
		    + \dotsb
		    + (x_n - \mu)^2}
		 {n}}
  \end{equation}
  donde la media \(\mu\) está dada por:
  \begin{equation}
    \label{eq:def-mu}
    \index{media|textbfhy}
    \mu = \frac{x_1 + x_2 + \dotsb + x_n}{n}
  \end{equation}

  \begin{proposition}
    La desviación estándar
    de un conjunto de valores \(x_1, x_2, \dotsc, x_n\)
    es cero si y solo si todos los valores son iguales.
  \end{proposition}
  \begin{proof}
    Construiremos una cadena de ``si y solo si'',
    comenzando con la proposición
    de que la desviación estándar es cero:
    \begin{equation}
      \label{eq:sigma=0}
      \sqrt{\frac{(x_1 - \mu)^2
		    + (x_2 - \mu)^2
		    + \dotsb
		    + (x_n - \mu)^2}
		 {n}}
	  = 0
    \end{equation}
    Como el único número cuya raíz cuadrada es cero es cero,
    la ecuación~(\ref{eq:sigma=0}) vale si y solo si:
    \begin{equation}
      \label{eq:sigma2=0}
      (x_1 - \mu)^2 + (x_2 - \mu)^2 + \dotsb + (x_n - \mu)^2
	  = 0
    \end{equation}
    Como el cuadrado de un número real
    nunca es negativo,
    cada término del lado izquierdo
    de~(\ref{eq:sigma2=0}) es no\nobreakdash-negativo.
    Luego,
    el lado izquierdo de~(\ref{eq:sigma2=0}) es cero si y solo si
    cada término es cero.
    Pero el término \((x_i - \mu)^2\) es cero si y solo si
    \(x_i = \mu\),
    o sea,
    todos son iguales a la media.
  \end{proof}

  Requeriremos lo que viene,
  que es un resultado sin particular importancia por sí mismo,
  en la demostración de la proposición siguiente.
  Excusa perfecta para un lema.%
    \index{lema}
  Usaremos demostración por inducción,
  que se discute
  en mayor detalle en la sección~\ref{sec:induccion}.%
    \index{demostracion@demostración!induccion@inducción}
  \begin{lemma}
    \label{lem:9|10^k-1}
    El entero \(10^k - 1\)
    es divisible por \(9\) para todo \(k \ge 0\).
  \end{lemma}
  \begin{proof}
    Por inducción sobre \(k\).
    \begin{description}
    \item[Base:]
      Para \(k = 0\),
      dice que \(0\) es divisible por \(9\),
      lo que es cierto.
    \item[Inducción:]
      Suponemos que \(10^k - 1 = 9 c\) para \(c \in \mathbb{N}_0\).
      Entonces:
      \begin{align*}
	10^{k + 1} - 1
	  &= ((10^k - 1) + 1) \cdot 10 - 1 \\
	  &= (9 c + 1) \cdot 10 - 1 \\
	  &= 90 c + 9
      \end{align*}
      Como ambos términos son divisibles por \(9\),
      lo es la suma.
    \end{description}
    Por inducción,
    vale para \(k \in \mathbb{N}_0\).
  \end{proof}
  \begin{proposition}
    \index{proposicion@proposición}
    El entero \(m\) es divisible por \(9\) si y solo si
    la suma de sus dígitos es divisible por \(9\).
  \end{proposition}
  \begin{proof}
    Sea \(s\)
    la suma de los dígitos de \(m\),
    o sea si \(\left\langle d_k \right\rangle_{k \ge 0}\)
    son los dígitos de \(m\),
    con lo que \(0 \le d_k < 10\),
    tenemos:
    \begin{align}
      m
	&= d_n \cdot 10^n + d_{n - 1} \cdot 10^{n - 1}
	    + \dotsb + d_1 \cdot 10 + d_0
	\label{eq:valor-m-digitos-decimales} \\
      s
	&= d_n + d_{n - 1} + \dotsb + d_1 + d_0
	\label{eq:valor-suma-digitos-decimales} \\
      m - s
	&= d_n \cdot \left(10^n - 1\right)
	   + d_{n - 1} \cdot \left(10^{n - 1} - 1\right)
	   + \dotsb
	   + d_1 \cdot \left(10 - 1\right) + d_0 \cdot (1 - 1)
	\label{eq:valor-m-menos-suma-digitos}
    \end{align}
    Cada uno de los factores \(10^k - 1\)
    en~\eqref{eq:valor-m-menos-suma-digitos} es divisible por \(9\)
    por el lema~\ref{lem:9|10^k-1},
    con lo que \(m - s\) es siempre divisible por \(9\);
    y \(m\) es divisible por \(9\) si y solo si \(s\) lo es.
  \end{proof}

\section{Demostración por casos}
\label{sec:casos}
\index{demostracion@demostración!por casos}

  Acá la idea es dividir una demostración complicada
  en casos más simples,
  y luego demostrar cada caso en turno.
  Es importante asegurarse que los casos resulten exhaustivos,
  vale decir,
  que no queden cabos sueltos.

  Muchas veces los casos aparecen en el problema mismo,
  típicamente en forma de una disyunción.
  Si alguno de los conceptos involucrados está definido por casos
  puede ser necesario considerarlos por separado.
  \begin{proposition}
    Para \(x \in \mathbb{R}\),
    se cumple \(x \le \lvert x \rvert\).
  \end{proposition}
  Simplemente seguimos la definición:
  \begin{equation*}
    \lvert x \rvert
      = \begin{cases}
	   x & \text{si \(x \ge 0\)} \\
	  -x & \text{si \(x < 0\)}
	\end{cases}
  \end{equation*}
  \begin{proof}
    Consideramos por separado los casos \(x \ge 0\)
    y \(x < 0\).
    \begin{description}
    \item[\boldmath\(x \ge 0\)\unboldmath:]
      En este caso es \(\lvert x \rvert = x\),
      y lo anunciado se cumple.
    \item[\boldmath\(x < 0\)\unboldmath:]
      En este caso \(\lvert x \rvert = - x > 0\),
      y \(x < 0 < -x = \lvert x \rvert\).
      También se cumple.
    \end{description}
    Estas son todas las posibilidades.
  \end{proof}
  En otras ocasiones conviene introducir casos
  de manera de tener más con qué trabajar.
  \begin{proposition}
    Si \(n\) es un entero,
    entonces \(n^2 + n\) es par.
  \end{proposition}
  \begin{proof}
    Conviene separar la demostración en dos casos:
    \begin{description}
    \item[\boldmath\(n\)\unboldmath\ es par:]
      En este caso,
      \(n^2\) y \(n\) son ambos pares,
      y su suma es par.
    \item[\boldmath\(n\)\unboldmath\ es impar:]
      Acá
      \(n^2\) y \(n\) son ambos impares,
      y su suma es par.
    \end{description}
    Como estos casos cubren todas las posibilidades de \(n\),
    vale para todo entero.
  \end{proof}
  Como en este ejemplo,
  suele ser útil considerar casos de enteros pares e impares.
  También es común separar en menor, igual o mayor a cero.

  Dadas dos personas,
  estas se han encontrado o no.
  A un conjunto de personas en el que cada par de personas
  se han encontrado lo llamaremos \emph{club},
  a un conjunto de personas en el que ningún par se ha encontrado
  lo llamaremos \emph{extraños}.
  \begin{theorem}
    Toda colección de \(6\) personas
    incluye un club de \(3\) personas
    o un grupo de \(3\) extraños.
  \end{theorem}
  Claramente si esto se cumple con \(6\) personas,
  se cumplirá con todo grupo mayor también.
  Al decir que hay un club de \(3\) personas,
  estamos indicando
  que hay un club de \emph{al menos} tres personas
  (podemos tomar tres cualquiera de ellas como un club de tres).
  \begin{proof}
    La demostración es por casos.
    Sea \(x\) una de las \(6\) personas.
    Hay dos posibilidades:
    \begin{enumerate}
    \item
      Entre las \(5\) personas restantes,
      al menos \(3\) se han encontrado con \(x\).
    \item
      Entre las \(5\) personas restantes,
      al menos \(3\) no se han encontrado con \(x\).
    \end{enumerate}
    Tenemos que asegurarnos que debe darse uno de los dos casos.
    Pero esto es fácil:
    Hemos dividido el grupo de \(5\) en dos,
    los que se han encontrado con \(x\) y los que no;
    uno de los dos grupos debe tener al menos \(3\) miembros.
    Ahora consideraremos cada caso por turno:
    \begin{enumerate}
    \item
      Consideremos el caso en que hay al menos \(3\) personas
      se han encontrado con \(x\),
      y tomemos \(3\) de ellas.
      Tenemos dos casos,
      que nuevamente son exhaustivos:
      \begin{enumerate}
      \item
	Ningún par de estas \(3\) personas se han encontrado.
	Tenemos un grupo de \(3\) extraños,
	con lo que el teorema vale en este caso.
      \item
	Algún par de estas \(3\) personas se han encontrado,
	con lo que este par con \(x\) forman un club,
	y el teorema vale en este caso.
      \end{enumerate}
      O sea,
      el teorema vale
      si hay \(3\) personas que se han encontrado con \(x\).
    \item
      Supongamos que a lo menos \(3\) personas
      nunca se han encontrado con \(x\),
      y consideremos \(3\) de ellas.
      Este caso también se divide en dos:
      \begin{enumerate}
      \item
	Cada par de estas personas se han encontrado entre sí.
	Entonces forman un club,
	y el teorema vale en este caso.
      \item
	Algún par de estas \(3\) personas
	no se han encontrado nunca,
	con lo que forman un grupo de \(3\) extraños con \(x\),
	y en este caso el teorema vale.
      \end{enumerate}
      Así el teorema también vale
      en caso que hayan \(3\) que no se han encontrado con \(x\).
    \end{enumerate}
    Hemos cubierto todas las distintas alternativas,
    y en todas ellas hemos demostrado que el teorema se cumple.
  \end{proof}
  En general no seremos tan detallistas en nuestras demostraciones.
  En particular,
  se ve que los distintos casos son todos muy similares,
  y en la práctica bastaría detallar uno
  e indicar que los demás se manejan de forma afín.
  De todas maneras nuestras demostraciones
  serán más detalladas
  que las usuales entre matemáticos profesionales.
  Es común usar la frase ``sin pérdida de generalidad''%
    \index{sin perdida de generalidad@sin pérdida de generalidad}
  (en inglés
     \emph{\foreignlanguage{english}{without loss of generality}},
   comúnmente abreviado \emph{wlog})
  para indicar
  que solo se tratará uno de varios casos muy similares.
  \begin{proposition}
    Si dos enteros son de paridad opuesta,
    su suma es impar.
  \end{proposition}
  \begin{proof}
    Sean \(m\) y \(n\) enteros de paridad opuesta.
    Sin pérdida de generalidad,
    sea \(m\) par y \(n\) impar.
    Vale decir \(m = 2 a\) y \(n = 2 b + 1\)
    con \(a\) y \(b\) enteros.
    Entonces \(m + n = 2 a + 2 b + 1 = 2 (a + b) + 1\),
    que es impar.
  \end{proof}

  Las demostraciones por casos se consideran poco elegantes,
  particularmente si los casos a considerar son muchos.
  Pero por ejemplo para el famoso problema de los cuatro colores%
    \index{cuatro colores, teorema de}
  (bastan cuatro colores para colorear un mapa
   de manera que no hayan áreas vecinas del mismo color)
  la demostración más simple a la fecha~%
    \cite{robertson97:_four_color_theor}
  involucra analizar alrededor de \(630\)~casos.
  Para este famoso problema
  hay una demostración completamente formal
  escrita en Coq~\cite{coq8.4:_coq_proof_assis}
  por Gonthier~\cite{gonthier08:_formal_proof}.
  Lo interesante es que un problema tan simple
  requiera una solución tan compleja.

\section{Demostración por contradicción}
\label{sec:contradiccion}
\index{demostracion@demostración!contradiccion@contradicción}
\index{reduccion al absurdo@reducción al absurdo|see{demostración!contradicción}}
\index{indirecta!demostracion@demostración|see{demostración!contradicción}}

  A veces una demostración toma la forma
  a la que alude la famosa cita de Sherlock Holmes:%
    \index{Holmes, Sherlock}
  \hybridblockquote{english}
      [Conan Doyle~\cite{conan90:_sign_four}]{%
    How often have I said to you
    that when you have eliminated the impossible,
    whatever remains,
    however improbable,
    must be the truth?%
  }.%
    \index{Conan Doyle, Sir Arthur}
  A esta importante técnica
  también se le llama ``demostración indirecta''
  o ``por reducción al absurdo''
  (en inglés se suele usar
   la frase del latín
     \emph{\foreignlanguage{latin}{reductio ad absurdum}}).
  La idea básica
  es partir con lo contrario de lo que se quiere demostrar,
  y llegar a una contradicción.
  Suele ser una manera fácil de demostrar algo;
  pero la descripción como ``indirecta'' es bastante adecuada,
  puede llevar a demostraciones complejas,
  difíciles de entender.
  Por esta razón Knuth, Larrabee y Roberts~%
     \cite{knuth89:_math_writing}
  recomiendan evitarla en lo posible.

  Partimos de la negación de lo que se quiere demostrar,
  y deducimos algo que se sabe es falso
  (una contradicción).%
    \index{contradiccion@contradicción}
  Sabemos que si \(P\) implica falso,
  la única manera en que esta proposición puede ser verdadera
  es que \(P\) sea falso.
  Debe tenerse cuidado de no usar los resultados intermedios
  de tales demostraciones fuera de ellas,
  como de ellos se deduce algo falso son falsos.

  La forma de construir una demostración según este esquema
  es como sigue:
  \begin{enumerate}
  \item
    Escriba ``La demostración es por contradicción''.
    Enuncie la negación de lo que se desea demostrar,
    indicando que se supone que esto es cierto.
  \item
    Deduzca de lo anterior algo que se sabe es falso.
    Concluya ``Esta contradicción demuestra el teorema''.
  \end{enumerate}

  Ya vimos un ejemplo de esta técnica
  al demostrar el principio extendido del palomar,
  teorema~\ref{theo:pigeonhole}.
  El ejemplo clásico es la demostración
  de que \(\sqrt{2}\) es irracional.
  Dice la leyenda que Pitágoras%
    \index{Pitagoras@Pitágoras}
  se enojó tanto
  con la demostración de la existencia de irracionales
  (que echaba por tierra su filosofía
   de ``todo se expresa en números'',
   donde ``números'' hay que entenderlo
   como números naturales y sus razones)
  que hizo ahogar a quien descubrió esto.
  En realidad,
  este resultado produjo un escándalo mayúsculo en la matemática:
  Gran parte de la teoría de semejanza de figuras
  se basaba
  en que ``obviamente'' todas las proporciones podían expresarse
  como razones entre enteros.
  Fue el genio de Eudoxo de Cnido%
    \index{Eudoxo de Cnido}
  quien resolvió el tema mediante su teoría de proporciones,
    \index{proporciones, teoria de@proporciones, teoría de}
  precursor de la actual definición de límite.

  \begin{theorem}[Hipasso de Metaponto]
    \index{Hipasso de Metaponto}
    \index{numero@número!irracional!\(\sqrt{2}\)}
    \label{theo:sqrt2-irracional}
    \(\sqrt{2}\) es irracional
  \end{theorem}
  La demostración original se perdió,
  pero todo indica
  que debe haber sido a lo largo de las líneas de esta.
  \begin{proof}
    La demostración es por contradicción.%
      \index{demostracion@demostración!contradiccion@contradicción}
    Consideremos el triángulo \(A B C\)
    (véase la figura~\ref{fig:sqrt2}),
    \begin{figure}[htbp]
      \centering
      \pgfimage{images/sqrt2}
      \caption{Diagrama para demostrar que $\sqrt{2}$ es irracional}
      \label{fig:sqrt2}
    \end{figure}
    tal que \(A C = a\) y \(A B = B C = b\),
    con el ángulo \(A B C\) recto.
    Por el teorema de Pitágoras,
    \(a^2 = 2 b^2\),
    o sea \(\sqrt{2} = a / b\).
    Claramente \(b < a < 2 b\)
    (la desigualdad triangular).%
      \index{desigualdad triangular}
    Supongamos ahora que \(a\) y \(b\) son múltiplos enteros
    de una unidad,
    elegida tal que \(b\)
    es el mínimo de los enteros que dan la relación dada.
    Extendemos la recta \(A B\),
    y con centro en \(A\) dibujamos los arcos \(B D\) y \(C E\),
    y dibujamos la recta \(E D\) que corta \(B C\) en \(F\).
    Por construcción
    los triángulos \(A B C\) y \(A D E\) son congruentes
    (tienen el ángulo \(A B C\) en común,
     y respectivamente iguales
     los lados \(A D = A B\) y \(A C = A E\)).
    En particular,
    el ángulo \(A D E\) es recto,
    con lo que es recto \(C D E\).
    El triángulo \(C D F\) es semejante a \(A B C\)%
      \index{triangulo@triángulo!semejanza}
    (comparten el ángulo \(A C B\),
     y ambos son triángulos rectos),
    así que \(D C = D F\).
    De la misma forma,
    \(E B F\) es semejante a \(A B C\),
    y como \(B E = D C\),
    \(C D F\) es congruente a \(E B F\).%
      \index{triangulo@triángulo!congruencia}
    Ahora bien,
    \(B F = B E = a - b\),
    y \(C F = B C - B F = b - (a - b) = 2 b - a\).
    Si \(a\) y \(b\) son enteros,
    también lo son \(a - b\) y \(2 b - a\),
    que son menores que \(a\) y \(b\)
    y están en la misma relación por similitud.
    Esto es absurdo,
    habíamos supuesto que \(b\) era el mínimo entero
    que sirve de denominador en esta razón.
  \end{proof}

  Las demostraciones corrientes actualmente son algebraicas,
  hay que recordar que en tiempos de Pitágoras%
    \index{Pitagoras@Pitágoras}
  no había nada parecido a nuestra álgebra.
  Precisamente el problema que planteaban los irracionales
  hizo que desde los griegos se usara casi exclusivamente
  la geometría como lenguaje matemático
  para expresar cantidades continuas,
  recién por la época de Newton%
    \index{Newton, Isaac}
  se retomó la idea de expresiones numéricas.
  \begin{proof}
    La demostración es por contradicción.%
      \index{demostracion@demostración!contradiccion@contradicción}
    Supongamos que \(\sqrt{2}\) fuera racional.
    Entonces existen números naturales \(a\) y \(b\) tales que:
    \begin{equation}
      \label{eq:sqrt2=racional}
      \sqrt{2} = \frac{a}{b}
    \end{equation}
    En~\eqref{eq:sqrt2=racional} podemos suponer
    que la fracción está expresada en mínimos términos,
    vale decir,
    \(a\) y \(b\) no tienen factores en común.
    En particular,
    a lo más uno de los dos es par.
    Ahora bien,
    elevando~\eqref{eq:sqrt2=racional} al cuadrado tenemos:
    \begin{equation}
      \label{eq:2=a2/b2}
      a^2 = 2 b^2
    \end{equation}
    De~\eqref{eq:2=a2/b2} sabemos que \(a^2\) es par,
    por lo que \(a\) debe ser par,
    digamos \(a = 2 c\).
    Pero esto lleva a:
    \begin{align*}
      4 c^2
	&= 2 b^2 \\
      2 c^2
	&= b^2
    \end{align*}
    con lo que también \(b\) es par.
    Esta contradicción
    de números de los cuales a lo más uno puede ser par
    pero que resultan ser ambos pares
    demuestra que tales \(a\) y \(b\) no pueden existir,
    y \(\sqrt{2}\) es irracional.
  \end{proof}

  Una demostración alternativa es la siguiente:

  \begin{proof}
    La demostración es por contradicción.%
      \index{demostracion@demostración!contradiccion@contradicción}
    Supongamos que \(\sqrt{2}\) es racional,
    y sea \(q\) el menor natural
    tal que \(q' = (\sqrt{2} - 1) q\) es entero.
    Entonces \(0 < q' < q\),
    pero \((\sqrt{2} - 1) q' = q - 2 q' > 0\) es entero.
    Esto contradice la anterior elección de \(q\).
  \end{proof}

  Una demostración
  que no hace uso de divisibilidad se debe a Leo Moser~%
    \cite{moser04:_introd_theor_number}.%
    \index{Moser, Leo}
  \begin{proof}
    Suponga \(\sqrt{2} = a / b\),
    con \(b\) lo más pequeño posible,
    con lo que \(0 < b < a\) ya que \(\sqrt{2} > 1\).
    Entonces:
    \begin{align}
      \frac{2 a b}{a b}
	&= 2 \label{eq:sqrt(2)-Moser-1} \\
      \frac{a^2}{b^2}
	&= 2 \label{eq:sqrt(2)-Moser-2}
    \end{align}
    De~\eqref{eq:sqrt(2)-Moser-1} y~\eqref{eq:sqrt(2)-Moser-2}
    por propiedades de las proporciones,%
      \index{proporcion, propiedades@proporción, propiedades}
    como \(b < a < 2 b\):
    \begin{equation}
      \label{eq:sqrt(2)-Moser-3}
      2
	= \frac{2 a b - a^2}{a b - b^2}
	= \frac{a (2 b - a)}{b (a - b)}
    \end{equation}
    Simplificando~\eqref{eq:sqrt(2)-Moser-3}
    usando la definición de \(a\) y \(b\) resulta:
    \begin{equation*}
      \sqrt{2}
	= \frac{2 b - a}{a - b}
    \end{equation*}
    Como \(b < a < 2 b\) tenemos \(0 < a - b < b\),
    obtenemos una fracción con un denominador menor que el mínimo,
    lo que es imposible.
  \end{proof}

  Vimos en la sección~\ref{sec:cuantificadores}
  el polinomio de Euler,%
    \index{Euler, polinomio de}%
    \index{Euler, Leonhard}
  que da números primos como valores para \(0 \le n \le 39\).
  \begin{proposition}
    Ningún polinomio no constante con coeficientes enteros
    da sólo números primos en los enteros no negativos.
  \end{proposition}
  \begin{proof}
    La demostración es por contradicción.%
      \index{demostracion@demostración!contradiccion@contradicción}
    Sea \(p(x)\) un polinomio no constante
    cuyo valor es primo para todo \(n \ge 0\).
    En particular,
    \(q = p(0)\) es primo.
    Si consideramos \(p(a q)\) para \(a \in \mathbb{N}\),
    vemos que todos sus términos,
    incluyendo el término constante,
    son divisibles por \(q\).
    Como el primo \(q\) divide al primo \(p(a q)\),
    es \(p(a q) = q\).
    Pero entonces \(p(x) = q\) para infinitos valores de \(x\),
    y \(p\) es constante,
    contradiciendo la hipótesis.
  \end{proof}

  Muchas veces la forma más cómoda de demostrar una desigualdad%
    \index{demostracion@demostración!desigualdad}
  es por contradicción.%
      \index{demostracion@demostración!contradiccion@contradicción}
  \begin{proposition}
    Sean \(x\), \(y\) reales mayores a cero.
    Si \(y (y + 1) \le (x + 1)^2\)
    entonces \(y (y - 1) \le x^2\).
  \end{proposition}
  \begin{proof}
    La demostración es por contradicción.
    Sean \(y (y + 1) \le (x + 1)^2\) y \(y (y - 1)> x^2\).
    En tal caso claramente \(y > 1\).

    La primera condición se traduce en:
    \begin{align*}
      y^2 + y
	&\le x^2 + 2 x + 1 \\
    \intertext{Con la suposición \(x^2 < y (y - 1)\):}
      y^2 + y
	&< y^2 - y + (2 x + 1) \\
      y &< x + \frac{1}{2} \\
      y - 1
	&< x - \frac{1}{2} \\
    \intertext{Como \(y > 1\),
	       el lado izquierdo es positivo,
	       y podemos multiplicar las últimas dos desigualdades:}
      y (y - 1)
	&< x^2 - \frac{1}{4}
    \end{align*}
    Esto último contradice a \(x^2 < y (y - 1)\).
  \end{proof}

  Otro buen ejemplo de demostración por contradicción
  es debido a Fourier%
    \index{Fourier, Joseph}
  de un resultado que
  Euler originalmente demostró en~1737.%
    \index{Euler, Leonhard}
  \begin{theorem}
    \label{theo:e-irrational}
    \index{numero@número!irracional!e@\(\mathrm{e}\)}
    El número \(\mathrm{e}\) es irracional.
  \end{theorem}
  \begin{proof}
    La demostración es por contradicción.%
      \index{demostracion@demostración!contradiccion@contradicción}
    Supongamos que \(\mathrm{e}\) es racional.
    Entonces existen enteros \(a\) y \(b\)
    tales que:
    \begin{equation}
      \label{eq:e-rational}
      \mathrm{e} = \frac{a}{b}
    \end{equation}
    También sabemos:
    \begin{equation}
      \label{eq:e-valor}
      \mathrm{e} = \sum_{k \ge 0} \frac{1}{k!}
    \end{equation}
    Consideremos la siguiente expresión,
    que debiera ser entera bajo nuestra suposición:
    \begin{align}
      b! \mathrm{e}
	&= \sum_{k \ge 0} \frac{b!}{k!} \notag \\
	&= \sum_{0 \le k \le b} \frac{b!}{k!}
	     + \sum_{k > b} \frac{b!}{k!}
	\label{eq:b!e}
    \end{align}
    La primera suma en~\eqref{eq:b!e} es un entero,
    nos concentraremos en acotar la segunda,
    a la que llamaremos \(S\),
    para demostrar que no es un entero.
    Más precisamente,
    mostraremos que \(0 < S < 1\):
    \begin{align}
      S &= \sum_{k > b} \frac{b!}{k!} \notag \\
	&= \sum_{k > b} \frac{1}{(b + 1) (b + 2) \dotsm k} \notag \\
	&= \sum_{r \ge 1} \frac{1}{(b + 1) (b + 2) \dotsm (b + r)}
	\label{eq:e-S-simplificado}
    \end{align}
    Cada término de la suma~\eqref{eq:e-S-simplificado} es positivo,
    y es una fracción
    cuyo denominador son \(r\) factores,
    cada uno mayor o igual a \(b + 1\),
    con lo que:
    \begin{equation}
      \label{eq:e-S-cota-termino}
      \frac{1}{(b + 1) (b + 2) \dotsm (b + r)}
	\le \frac{1}{(b + 1)^r}
    \end{equation}
    En consecuencia,
    como los términos después del primero
    son menores que \((b + 1)^{-r}\):
    \begin{align}
      0 < S &<	 \sum_{r \ge 1} (b + 1)^{-r} \notag \\
	    &=	 (b + 1)^{-1} \sum_{r \ge 0} (b + 1)^{-r} \notag \\
	    &=	 \frac{1}{b + 1}
		   \cdot \frac{1}{1 - 1/(b + 1)} \notag \\
	    &=	 \frac{1}{b} \notag \\
	    &\le 1
	    \label{eq:e-S-acotado}
    \end{align}
    Resulta por~\eqref{eq:e-S-acotado}
    que \(b! \mathrm{e}\) no es entero,
    cuando por nuestra suposición debiera serlo.
    Esta contradicción completa la demostración
    que \(\mathrm{e}\) es irracional.
  \end{proof}
  Una interesante demostración alternativa es la siguiente:
  \begin{proof}
    Si \(\mathrm{e}^{-1}\) es irracional,
    también lo es \(\mathrm{e}\).
    Demostraremos por contradicción
    que \(\mathrm{e}^{-1}\) es irracional.%
      \index{demostracion@demostración!contradiccion@contradicción}

    Sabemos que:
    \begin{equation}
      \label{eq:e-1-valor}
      \mathrm{e}^{-1}
	= \sum_{n \ge 0} \frac{(-1)^n}{n!}
    \end{equation}
    Al ser~\eqref{eq:e-1-valor} una serie de términos no nulos,
    que disminuyen en valor absoluto con signos alternantes,
    las sumas parciales
    son alternativamente mayores y menores que el límite.
    O sea,
    para \(m\) cualquiera es:
    \begin{equation}
      \label{eq:e-1-cotas}
      \sum_{0 \le n \le {2 m - 1}} \frac{(-1)^n}{n!}
	< \mathrm{e}^{-1}
	< \sum_{0 \le n \le {2 m}} \frac{(-1)^n}{n!}
    \end{equation}
    Pero la diferencia entre las dos sumas de~\eqref{eq:e-1-cotas}
    es el último término,
    que es \(1 / (2 m)!\),
    y así \(\mathrm{e}^{-1}\)
    no puede ser expresado como una fracción
    con \((2 m)!\) de denominador,
    con lo que su denominador no puede ser divisor de esto.
    Pero al ser \(m\) arbitrariamente grande,
    \(\mathrm{e}^{-1}\) no puede ser expresado como fracción,
    y \(\mathrm{e}\) debe ser irracional.
  \end{proof}

  Otro bonito ejemplo es la demostración de Niven~%
    \cite{niven47:_simple_proof_pi_irrat}:
  \begin{theorem}
    \index{numero@número!irracional!\(\pi\)}
    \label{theo:pi-irrational}
    El número \(\pi\) es irracional.
  \end{theorem}
  \begin{proof}
    Por contradicción.%
      \index{demostracion@demostración!contradiccion@contradicción}
    Supongamos \(\pi = a / b\),
    con \(a\) y \(b\) enteros positivos.
    Definamos los polinomios:
    \begin{align*}
      f(x)
	&= \frac{x^n (a - b x)^n}{n!} \\
      F(x)
	&= f(x)
	     - f^{(2)}(x)
	     + f^{(4)}(x)
	     - \dotsb
	     + (-1)^n f^{(2 n)} (x)
    \end{align*}
    El entero \(n\) lo fijaremos más adelante.

    El coeficiente de \(x^k\) en \(f(x)\) es \(f^{(k)}(0) / k!\)
    (teorema de Maclaurin).%
      \index{Maclaurin, teorema de}
    Por otro lado,
    como \(f(x) = x^n (a - b x)^n / n!\)
    el coeficiente de \(x^k\)
    puede escribirse \(c_k / n!\) para un entero \(c_k\).
    O sea:
    \begin{equation*}
      f^{(k)}(0)
	= \frac{k!}{n!} c_k
    \end{equation*}
    Para \(0 \le k < n\),
    el coeficiente \(c_k = 0\);
    para \(k \ge n\) es entero \(k! / n!\).
    En resumen,
    las derivadas \(f^{(k)}(0)\) son todas enteras.
    Como \(f(a / b - x) = f(x)\),
    lo mismo vale para \(x = \pi = a / b\).

    Por cálculo elemental:
    \begin{align*}
      \frac{\mathrm{d}}{\mathrm{d} x} \,
	\left( F'(x) \sin x - F(x) \cos x \right)
	= F''(x) \sin x + F(x) \sin x
	= f(x) \sin x
    \end{align*}
    por lo que:
    \begin{equation}
      \label{eq:pi-irrational-integral}
      \int_0^\pi f(x) \sin x \, \mathrm{d} x
	= \left( F'(x) \sin x - F(x) \cos x \right)
	    \big\rvert_0^\pi
	= F(\pi) + F(0)
    \end{equation}
    Ahora \(F(\pi) + F(0)\) es un \emph{entero},
    dado que las derivadas de \(f\) en \(0\) y \(\pi\) lo son.
    Pero para \(0 < x < \pi\)
    claramente ambos factores son positivos,
    y tenemos la cruda cota superior
    resultante de tomar el valor máximo de cada factor de \(f(x)\):
    \begin{equation*}
      0 < f(x) \sin x < \frac{\pi^n a^n}{n!}
    \end{equation*}
    con lo que la integral de~\eqref{eq:pi-irrational-integral}
    es positiva,
    pero arbitrariamente pequeña para \(n\) suficientemente grande.
    Esto es absurdo,
    no hay enteros positivos arbitrariamente pequeños.
  \end{proof}

\section{Inducción}
\label{sec:induccion}
\index{induccion@inducción|see{demostración!inducción}}
\index{demostracion@demostración!induccion@inducción}

  Una técnica importante para demostrar un número infinito de casos
  es \emph{inducción}.
  Algunos puntos de lo que sigue se tomaron del resumen de Tuffley~%
    \cite{tuffley09:_induction}.
  Esta importante técnica de demostración se ha usado implícitamente
  desde épocas antiguas.
  Lambrou~%
   \cite{lambrou05:_math_induction_i, lambrou06:_math_induction_ii}
  da una breve reseña de su historia,
  y plantea muchos ejemplos.

  Supongamos que queremos demostrar
  que una proposición es válida para todo \(n \in \mathbb{N}\).
  La manera de hacer esto es:
  \begin{enumerate}
  \item
    Escribir ``Usamos inducción''.
  \item
    Escribir: ``Caso base:''
    Plantear la proposición para \(n = 1\),
    y demostrarla.
  \item
    Escribir ``Inducción:''
    Asumiendo que la proposición es verdadera para \(n = k\),
    demostrar que vale para \(n = k + 1\).
  \end{enumerate}
  La validez de esto se deduce
  de los axiomas de los números naturales,
  tema sobre el que volveremos
  en el capítulo~\ref{cha:numeros-reales}\@.
  Hay algunas variantes que vale la pena distinguir.

\subsection{El caso más común}
\label{sec:induccion-comun}

  A una secuencia \(\left\langle r^n \right\rangle_{r \ge 0}\)
  se le llama \emph{geométrica}
    \index{secuencia!geometrica@geométrica}
  (con razón \(r\)),
  nuestro resultado es la suma de la serie geométrica.
  \begin{theorem}
    \index{suma!geometrica@geométrica}
    \label{theo:suma-geometrica}
    \begin{equation*}
      \sum_{0 \le k \le n} r^k
	= 1 + r + r^2 + \dotsb + r^n
	= \frac{1 - r^{n + 1}}{1 - r}
    \end{equation*}
  \end{theorem}
  \begin{proof}
    Usamos inducción.
    \begin{description}
    \item[Caso base:]
      Cuando \(n = 1\),
      tenemos:
      \begin{equation*}
	\sum_{0 \le k \le n} r^k
	= 1 + r
	= \frac{(1 - r) (1 + r)}{1 - r}
	= \frac{1 - r^2}{1 - r}
      \end{equation*}
      Esto demuestra el caso base.
    \item[Inducción:]
      Suponemos que para \(n = m\) vale:
      \begin{equation*}
	\sum_{0 \le k \le m} r^k
	  = \frac{1 - r^{m + 1}}{1 - r}
      \end{equation*}
      Entonces:
      \begin{align*}
	\sum_{0 \le k \le m + 1} r^k
	  &= \sum_{0 \le k \le m} r^k + r^{m + 1} \\
	  &= \frac{1 - r^{m + 1}}{1 - r} + r^{m + 1} \\
	  &= \frac{1 - r^{m + 1} + r^{m + 1} - r^{m + 2}}{1 - r} \\
	  &= \frac{1 - r^{m + 2}}{1 - r}
      \end{align*}
      que es precisamente la proposición para \(n = m + 1\).
    \end{description}
    Por inducción,
    lo aseverado es verdadero para todo \(n\) natural.

    En realidad,
    el resultado también se cumple para \(n = 0\),
    y por tanto vale para todo \(n \in \mathbb{N}_0\).
  \end{proof}

  Hay que tener cuidado en la demostración del paso de inducción.
  La tentación de trabajar ``a ambos lados'' suele ser fuerte,
  pero conlleva el riesgo de terminar en una identidad
  que no demuestra lo que se desea obtener.
  La forma más simple de evitar problemas
  es partir del lado izquierdo y derivar el lado derecho.
  El desarrollo debe ser estrictamente ``hacia adelante'',
  lo que debemos demostrar
  es la implicancia \(P(n) \implies P(n + 1)\).

  Del teorema~\ref{theo:suma-geometrica}
  sigue que si \(\lvert r \rvert < 1\) entonces:
  \begin{equation}
    \index{suma!geometrica@geométrica!infinita|textbfhy}
    \label{eq:limite-suma-geometrica}
    \lim_{n \rightarrow \infty} \sum_{0 \le k \le n} r^k
      = \lim_{n \rightarrow \infty} \frac{1 - r^{n + 1}}{1 - r}
      = \frac{1}{1 - r}
  \end{equation}
  Esto lo usamos antes
  en nuestra primera demostración que \(e\) es irracional
  (teorema~\ref{theo:e-irrational}).

  Otro caso importante es el siguiente:
  \begin{theorem}
    \label{theo:sum-factorial-powers}
    Para \(m \in \mathbb{N}_0\) tenemos:
    \begin{equation*}
      \sum_{1 \le k \le n} k^{\overline{m}}
	= \frac{n^{\overline{m + 1}}}{m + 1}
    \end{equation*}
  \end{theorem}
  \begin{proof}
    La demostración es por inducción sobre \(n\).
    \begin{description}
    \item[Base:]
      Cuando \(n = 1\)
      el lado izquierdo de lo planteado se reduce a:
      \begin{equation*}
	1^{\overline{m}}
	 = m!
      \end{equation*}
      mientras el lado derecho da:
      \begin{equation*}
	\frac{1^{\overline{m + 1}}}{m + 1}
	  = \frac{(m + 1)!}{m + 1}
	  = m!
      \end{equation*}
      Esto coincide,
      incluso en el caso \(m = 0\).
    \item[Inducción:]
      Suponiendo que vale para \(n\),
      demostramos que vale para \(n + 1\).
      Tenemos:
      \begin{align*}
	\sum_{1 \le k \le n + 1} k^{\overline{m}}
	  &= \sum_{1 \le k \le n}
	       k^{\overline{m}} + (n + 1)^{\overline{m}} \\
      \intertext{Por la hipótesis:}
	\sum_{1 \le k \le n + 1} k^{\overline{m}}
	  &= \frac{n^{\overline{m + 1}}}{m + 1}
	       + (n + 1)^{\overline{m}} \\
	  &= \frac{n \cdot (n + 1)^{\overline{m}}
		     + (m + 1) \cdot (n + 1)^{\overline{m}}}
		  {m + 1} \\
	  &= \frac{(n + m + 1) \cdot (n + 1)^{\overline{m}}}
		  {m + 1} \\
	  &= \frac{(n + 1)^{\overline{m + 1}}}{m + 1}
      \end{align*}
    \end{description}
    Por inducción vale para todo \(n \in \mathbb{N}\).
    Pero vale también para \(n = 0\),
    como es simple verificar.
  \end{proof}
  Una relación similar
  se cumple para potencias factoriales en bajada,
  cuya demostración quedará de ejercicio.
  Es interesante el paralelo
  entre el teorema~\ref{theo:sum-factorial-powers}
  y la integral de \(x^m\).

\subsection{Otro punto de partida}
\label{sec:induccion-partida-diferente}

  Es común querer demostrar que un predicado vale no desde \(1\),
  sino desde algún otro valor \(m\).
  En tal caso formalmente tenemos dos maneras de proceder:
  \begin{itemize}
  \item
    Definir un nuevo predicado:
    \begin{equation*}
      P'(n)
	=
	\begin{cases}
	  \text{Verdadero} & \text{\ si\ \(n < m\)} \\
	  P(n)		   & \text{\ caso contrario}
	\end{cases}
    \end{equation*}
  \item
    Definir un nuevo predicado:
    \begin{equation*}
      P''(n)
	= P(n - m + 1)
    \end{equation*}
  \end{itemize}
  y aplicar inducción a \(P'(n)\) o \(P''(n)\),
  según corresponda.
  En la práctica,
  esto se reduce a indicar que la inducción comienza con otra base.
  Podemos también trabajar en \(\mathbb{N}_0\),
  iniciando la inducción con \(0\).
  Incluso puede comenzar la inducción en un valor negativo.

\subsection{Paso diferente}
\label{sec:induccion-paso-diferente}

  Otra variante se da cuando no es claro obtener el caso \(n + 1\)
  directamente del caso \(n\),
  pero sí podemos obtener el caso \(n + 2\) de \(n\)
  (y \(n + 3\) de \(n + 1\)).
  Más en general,
  hay algún \(k\) tal que \(P(n)\) permite concluir \(P(n + k)\).
  En tal caso debemos tener \(k\) puntos de partida,
  y razonar las distintas secuencias entrelazadas.

  Un ejemplo es el siguiente:
  \begin{proposition}
    Para ningún \(n\) natural es \(n^2 + n + 1\)
    divisible por \(5\).
  \end{proposition}
  \begin{proof}
    La demostración es por inducción.
    \begin{description}
    \item[Casos base:]
      Para los casos base:
      \begin{align*}
	0^2 + 0 + 1 &= \phantom{0}1 \\
	1^2 + 1 + 1 &= \phantom{0}3 \\
	2^2 + 2 + 1 &= \phantom{0}7 \\
	3^2 + 3 + 1 &=		 13 \\
	4^2 + 4 + 1 &=		 21
      \end{align*}
      Ninguno es divisible por \(5\).
    \item[Inducción:]
      Demostramos que si \(n^2 + n + 1\) no es divisible por \(5\),
      tampoco lo es \((n + 5)^2 + (n + 5) + 1\).
      Vemos que:
      \begin{align}
	(n + 5)^2 + (n + 5) + 1
	  &= n^2 + 10 n + 25 + n + 5 + 1 \notag \\
	  &= (n^2 + n + 1) + (10 n + 30)
	       \label{eq:polinomio-modulo-5}
      \end{align}
      En~\eqref{eq:polinomio-modulo-5}
      el segundo término es divisible por~\(5\);
      como el primer término no lo es,
      la suma no es divisible por~\(5\).
    \qedhere
    \end{description}
  \end{proof}
  El lector podrá entretenerse demostrando
  que si contamos con estampillas de \(5\) y \(8\) centavos
  podemos franquear cualquier cantidad mayor a \(27\) centavos.

\subsection{Ida y vuelta}
\label{sec:induccion-ida-y-vuelta}

  La \emph{media aritmética} de \(a_1, a_2, \dotsc, a_n\)%
    \index{media!aritmetica@aritmética}
  (suponemos todos positivos)
  es:
  \begin{equation*}
    \frac{a_1 + a_2 + \dotsb + a_n}{n}
  \end{equation*}
  La \emph{media geométrica} de estos mismos valores%
    \index{media!geometrica@geométrica}
  es:
  \begin{equation*}
    \sqrt[n]{a_1 a_2 \dotsb a_n}
  \end{equation*}

  Un caso especial de inducción
  se da en la demostración de  Cauchy~%
    \cite{cauchy21:_cours_analyse-1}%
    \index{Cauchy, Augustin-Louis}
  de la relación entre las medias aritmética y geométrica.%
    \index{media!aritmetica y geometrica@aritmética y geométrica}
    \index{AGM@\emph{AGM}|see{media!aritmética y geométrica}}
  Resulta simple demostrar
  el caso \(2^{k + 1}\) a partir del caso \(2^k\),
  y usamos inducción en reversa
  (de \(n\) concluimos \(n - 1\))
  para rellenar los espacios.
  \begin{theorem}[Desigualdad entre las medias geométrica
		  y aritmética]
    \label{theo:AM-GM-inequality}
    Para números reales no negativos
    \(a_1, a_2, \dotsc, a_n\)
    se cumple:
    \begin{equation*}
      (a_1 a_2 \dotsm a_n)^{1/n}
	\le \frac{a_1 + a_2 + \dotsb + a_n}{n}
    \end{equation*}
  \end{theorem}
  \begin{proof}
    Usamos una forma especial de inducción.
    Primeramente demostramos por inducción sobre \(k\)
    que si la desigualdad se cumple para \(2^k\)
    vale para \(2^{k + 1}\);
    y luego completamos los casos faltantes
    a través de demostrar que si vale para \(n\)
    también vale para \(n - 1\).

    Primero para potencias de dos.
    Es claro que la desigualdad se cumple para \(2^0\).
    \begin{description}
    \item[Base:]
      Para el caso \(2^1 = 2\),
      consideremos \(a\) y \(b\) positivos:
      \begin{align*}
	(a - b)^2
	  &\ge 0 \\
	a^2 + b^2
	  &\ge 2 a b
      \end{align*}
      Si substituimos \(a \mapsto \sqrt{a_1}\),
      \(b \mapsto \sqrt{a_2}\),
      resulta:
      \begin{equation*}
	\frac{a_1 + a_2}{2}
	  \ge \sqrt{a_1 a_2}
      \end{equation*}
      que es el caso \(n = 2\).
    \item[Inducción:]
      Del caso \(2^k\) concluimos el caso \(2^{k + 1}\).
      Para ello dividimos los \(2^{k + 1}\) valores
      en dos grupos de \(2^k\),
      y combinamos.
      De la hipótesis de inducción:
      \begin{align*}
	\frac{a_1 + \dotsb + a_{2^k}}{2^k}
	  &\ge (a_1 \dotsm a_{2^k})^{1 / 2^k} \\
	\frac{a_{2^k + 1} + \dotsb + a_{2^{k + 1}}}{2^k}
	  &\ge (a_{2^k + 1} \dotsm a_{2^{k + 1}})^{1 / 2^k}
      \end{align*}
      Aplicando el caso \(n = 2\) a las anteriores:
      \begin{align*}
	\frac{\frac{a_1 + \dotsb + a_{2^k}}{2^k}
		+ \frac{a_{2^k + 1} + \dotsb + a_{2^{k + 1}}}{2^k}}
	     {2}
	  &\ge \left(
		 \frac{a_1 + \dotsb + a_{2^k}}{2^k}
		   \cdot \frac{a_{2^k + 1} + \dotsb + a_{2^{k + 1}}}
			      {2^k}
	       \right)^{1 / 2} \\
	\frac{a_1 + \dotsb + a_{2^{k + 1}}}{2^{k + 1}}
	  &\ge \left(
		 \left(
		   a_1 \dotsm a_{2^k}
		 \right)^{1 / {2^k}}
		   \cdot \left(
			   a_{2^k + 1} \dotsm a_{2^{k + 1}}
			 \right)^{1 / {2^k}}
	       \right)^{1 / 2} \\
	  &=   \left(
		   a_1 \dotsm a_{2^{k + 1}}
	       \right)^{1 / 2^{k + 1}}
      \end{align*}
    \end{description}
    Por inducción
    vale para todo \(n = 2^k\) con \(k \in \mathbb{N}_0\).

    Resta demostrar que si vale para \(n\),
    vale para \(n - 1\),
    inducción en reversa.
    Supongamos que vale para \(n\):
    \begin{equation*}
      \frac{a_1 + a_2 + \dotsb +a_n}{n}
	\ge (a_1 a_2 \dotsm a_n)^{1 / n}
    \end{equation*}
    y consideremos:
    \begin{equation*}
      \alpha
	= \frac{a_1 + a_2 + \dotsb + a_{n - 1}}{n - 1}
    \end{equation*}
    Entonces:
    \begin{equation*}
      \frac{a_1 + a_2 + \dotsb + a_{n - 1} + \alpha}{n}
	= \frac{(n - 1) \alpha + \alpha}{n}
	= \alpha
    \end{equation*}
    Por el otro lado,
    usando la hipótesis de inducción:
    \begin{align*}
      \frac{a_1 + a_2 + \dotsb + a_{n - 1} + \alpha}{n}
	&\ge (a_1 a_2 \dotsm a_{n - 1} \alpha)^{1 / n} \\
      \intertext{con lo que}
      \alpha^n
	&\ge a_1 a_2 \dotsm a_{n - 1} \alpha \\
      \alpha^{n - 1}
	&\ge a_1 a_2 \dotsm a_{n - 1}
    \end{align*}
    esto último equivale a lo anunciado.

    Uniendo ambos casos,
    tenemos que la desigualdad vale para todo \(n \in \mathbb{N}\).
  \end{proof}
  La tercera media que reconocía Pitágoras%
    \index{Pitagoras@Pitágoras}
  es la \emph{media harmónica}:%
    \index{media!harmonica@harmónica}
  \begin{equation}
    \label{eq:harmonic-mean}
    \frac{n}{\frac{1}{a_1} + \frac{1}{a_2} + \dotsb + \frac{1}{a_n}}
  \end{equation}
  Tenemos:
  \begin{theorem}[Desigualdad entre las medias geométrica
		  y harmónica]
    \label{theo:GM-HM-inequality}
    Para números reales positivos
    \(a_1, a_2, \dotsc, a_n\)
    se cumple:
    \begin{equation*}
      \frac{n}{\frac{1}{a_1}
	  + \frac{1}{a_2} + \dotsb + \frac{1}{a_n}}
	\le (a_1 a_2 \dotsm a_n)^{1/n}
    \end{equation*}
  \end{theorem}
  \begin{proof}
    Usemos el teorema~\ref{theo:AM-GM-inequality}
    con los recíprocos:
    \begin{align*}
      \frac{\frac{1}{a_1}
	  + \frac{1}{a_2} + \dotsb + \frac{1}{a_n}}{n}
	&\ge \left(
		\frac{1}{a_1} \cdot \frac{1}{a_2}
		  \cdot \dotsb \cdot \frac{1}{a_n}
	     \right)^{1/n} \\
	&= (a_1 a_2 \dotsb a_n)^{-1/n} \\
      \frac{n}{\frac{1}{a_1}
		 + \frac{1}{a_2}
		 + \dotsb
		 + \frac{1}{a_n}}
	&\le (a_1 a_2 \dotsb a_n)^{1/n}
    \end{align*}
    Hay igualdad si y solo si los \(a_i\) son todos iguales.
  \end{proof}

\subsection{Múltiples variables}
\label{sec:induccion-multiple}
\index{demostracion@demostración!induccion@inducción!multivariable|textbfhy}

  Se da la situación en la cual hay varias variables involucradas.
  La mayoría de las veces
  puede resolverse vía fijar algunas de las variables
  y aplicar inducción sobre otra,
  pero en raras ocasiones
  realmente se requiere inducción sobre más de una variable.
  Suponiendo variables \(m\) y \(n\),
  una opción es intentar inducción sobre alguna combinación
  como \(m + n\).
  Sin embargo,
  hay situaciones en las que esto no funciona.

  Un ejemplo de inducción sobre múltiples variables
  es demostrar que:
  \begin{equation}
    \label{eq:def-Cmn}
    C(m, n)
      = \frac{(2 m)! (2 n)!}{n! m! (m + n)!}
  \end{equation}
  siempre es un entero para \(m, n \ge 0\).
  Nuestra estrategia es demostrar
  (por inducción)
  que \(C(m, 0)\) es siempre un entero,
  y luego derivar una relación entre \(C(m, n)\), \(C(m + 1, n)\)
  y \(C(m, n + 1)\) que demuestra
  que si los primeros dos son enteros
  lo es el tercero.
  Esto lo usamos para demostrar por inducción
  que \(C(m, n)\) es siempre entero.

  El siguiente resultado es más general que lo que necesitamos,
  pero resulta más simple de demostrar.
  \begin{theorem}
    Un producto de \(n\) enteros consecutivos
    siempre es divisible por \(n!\).
  \end{theorem}
  \begin{proof}
    Podemos escribir el producto de \(n\) enteros consecutivos
    como \(m^{\underline{n}}\),
    con lo que queremos demostrar que \(n! \mid m^{\underline{n}}\)
    para todo \(m\) y \(n\).
    Esto lo hacemos por inducción sobre \(n\).
    \begin{description}
    \item[Base:]
      El caso \(n = 0\)
      se reduce a \(0! \mid m^{\underline{0}}\),
      que es \(1 \mid 1\),
      independiente del valor de \(m\).
    \item[Inducción:]
      Suponiendo que \(n! \mid k^{\underline{n}}\) para todo \(k\),
      demostramos que \((n + 1)! \mid m^{\underline{n + 1}}\)
      para todo \(m\).
      Sabemos de~\eqref{eq:Sigma-falling-factorial} que:
      \begin{equation}
	\label{eq:sum_k_falling}
	m^{\underline{n + 1}}
	  = (n + 1) \sum_{1 \le k \le m} k^{\underline{n}}
      \end{equation}
      Como (por inducción) cada término de la suma del lado derecho
      de~\eqref{eq:sum_k_falling} es divisible por \(n!\),
      su lado izquierdo es divisible por \((n + 1) n! = (n + 1)!\).
    \end{description}
    Por inducción,
    vale para \(n \in \mathbb{N}_0\).
  \end{proof}

  Vamos ahora a nuestro interés original.
  \begin{proposition}
    El valor \(C(m, n)\) definido por~\eqref{eq:def-Cmn}
    es un entero.
  \end{proposition}
  Calculamos la suma dada en el paso de inducción
  en esperanza de factores comunes
  que resulten en una expresión simple.
  \begin{proof}
    Por inducción sobre \(n\).
    \begin{description}
    \item[Bases:]
      Primeramente,
      si \(n = 0\),
      se reduce a:
      \begin{equation}
	\label{eq:Cm0}
	C(m, 0)
	  = \frac{(2 m)!}{m! m!}
	  = \frac{(2 m)^{\underline{m}}}{m!}
      \end{equation}
      que por~\eqref{eq:sum_k_falling} es un entero
      para todo \(m \in \mathbb{N}_0\).
    \item[Inducción:]
      Suponemos \(C(m, n)\) entero para todo \(m \in \mathbb{N}_0\)
      Consideremos:
      \begin{align}
	C(m + 1, n) + C(m, n + 1)
	  &= \frac{(2 m + 2)! (2 n)!}{(m + 1)! n! (m + n + 1)!}
	       + \frac{(2 m)! (2 n + 2)!}{m! (n + 1)! (m + n + 1)!}
		\notag \\
	  &= \frac{(2 m)! (2 n)!}{m! n! (m + n + 1)!}
	       \cdot \left(
		       \frac{(2 m + 1) (2 m + 2)}{m + 1}
			 + \frac{(2 n + 1) (2 n + 2)}{n + 1}
		     \right) \notag \\
	  &= \frac{(2 m)! (2 n)!}{m! n! (m + n + 1)!}
	       \cdot 4 \cdot (m + n + 1) \notag \\
	  &= 4 C(m, n) \label{eq:4Cmn}
      \end{align}
      Por inducción
      sabemos que son enteros \(C(m + 1, n)\) y \(C(m, n)\).
      Por lo tanto
      también es entero
      \(C(m, n + 1) = 4 C(m, n) - C(m + 1, n)\).
    \end{description}
    Por inducción es entero \(C(m, n)\)
    para \(m, n \in \mathbb{N}_0\).
  \end{proof}
  A pesar de ser oficialmente inducción sobre \(n\),
  también estamos usando el resultado para varios valores de \(m\).

  Otro ejemplo involucra los \emph{números de Fibonacci},%
      \index{Fibonacci, numeros de@Fibonacci, números de}%
      \index{Fibonacci, Leonardo (Leonardo Pisani Bigollo)}%
  definidos mediante la recurrencia:
  \begin{equation}
    \label{eq:Fibonacci-recurrence}
    F_0
      = 0,
    F_1
      = 1,
    F_{n + 2}
      = F_{n + 1} + F_n
  \end{equation}
  y los relacionados \emph{números de Lucas}:%
    \index{Lucas, numeros de@Lucas, números de}%
    \index{Lucas, Edouard@Lucas, Édouard}
  \begin{equation}
    \label{eq:Lucas-recurrence}
    L_0
      = 2,
    L_1
      = 1,
    L_{n + 2}
      = L_{n + 1} + L_n
  \end{equation}
  Nuestro interés es la sorprendente identidad:
  \begin{proposition}
    Para todos \(m, n \ge 0\) se cumple:
    \begin{equation}
      \label{eq:Fibonacci-Lucas}
      2 F_{m + n}
	= L_m F_n + L_n F_m
    \end{equation}
  \end{proposition}
  Quien descubrió esta relación o era brujo o muy ocioso\ldots
  \begin{proof}
    La demostración es por inducción simultánea sobre \(m\) y \(n\);
    si se cumple para \(m + n\)
    y para \(m + n + 1\) deducimos que se cumple con \(m + n + 2\).
    \begin{description}
    \item[Bases:]
      Para \(m + n = 0\) la única posibilidad es \(m = n = 0\),
      para \(m + n = 1\)
      están las combinaciones \(m = 0\) con \(n = 1\)
      y \(m = 1\) con \(n = 0\).
      Por simetría,
      los últimos dos casos se reducen a uno solo:
      \begin{align*}
	2 F_0
	  &= L_0 F_0 + L_0 F_0
	   = 2 \cdot 0 + 1 \cdot 0
	   = 0 \\
	2 F_1
	  &= L_1 F_0 + L_0 F_1
	   = 1 \cdot 0 + 2 \cdot 1
	   = 2
      \end{align*}
      Las dos se cumplen.
    \item[Inducción:]
      Supongamos que vale para \(m\) y \(n\),
      para \(m\) y \(n + 1\),
      y para \(m + 1\) y \(n\):
      \begin{align}
	2 F_{m + n}
	  &= L_m F_n + L_n F_m
	      \label{eq:Fibonacci-Lucas+00} \\
	2 F_{m + n + 1}
	  &= L_m F_{n + 1} + L_{n + 1} F_m
	      \label{eq:Fibonacci-Lucas+01} \\
	  &= L_{m + 1} F_n + L_n F_{m + 1}
	      \label{eq:Fibonacci-Lucas+10}
      \end{align}
      Al sumar~\eqref{eq:Fibonacci-Lucas+00}
      con~\eqref{eq:Fibonacci-Lucas+01} obtenemos
      para el lado derecho:
      \begin{equation*}
	2 F_{m + n} + 2 F_{m + n + 1}
	  = 2 F_{m + n + 2}
      \end{equation*}
      Al lado izquierdo resulta:
      \begin{equation*}
	L_m (F_n + F_{n + 1}) + (L_n + L_{n + 1}) F_m
	  = L_m F_{n + 2} + L_{n + 2} F_m
      \end{equation*}
      O sea,
      para esta combinación de valores se cumple.

      Para la otra combinación,
      sumar~\eqref{eq:Fibonacci-Lucas+00}
      con~\eqref{eq:Fibonacci-Lucas+10} resulta en:
      \begin{equation*}
	(L_m + L_{m + 1}) F_n + L_n (F_m + F_{m + 1})
	  = L_{m + 2} F_n + L_n F_{m + 2}
      \end{equation*}
      Nuevamente se cumple.
    \end{description}
    Por inducción simultánea sobre \(m\) y \(n\),
    lo anunciado se cumple para todo \(m\) y \(n\).
  \end{proof}

\subsection{Inducción fuerte}
\label{sec:induccion-fuerte}
\index{induccion fuerte@inducción fuerte|see{demostración!inducción!fuerte}}
\index{demostracion@demostración!induccion@inducción!fuerte|textbfhy}

  Una variante de la técnica de inducción
  es lo que se conoce como \emph{inducción fuerte},
  donde suponemos no solo la validez para \(n = k\)
  al demostrar el caso \(n = k + 1\),
  sino la validez en todos los casos \(1 \le k \le n\).
  Nótese que no se requiere usar todos los casos anteriores,
  es común que solo se necesiten algunos de ellos.
  En todo caso,
  es raro que se haga la distinción
  entre inducción fuerte y su variante tradicional
  en la literatura profesional.

  Para justificar esto,
  recurrimos a definir un nuevo predicado \(\widetilde{P}(n)\)
  que es cierto si \(P(k)\) es verdadero para \(1, 2, \dotsc, n\).
  Una definición recursiva
  de \(\widetilde{P}(n)\) en términos de \(P(n)\) es:
  \begin{equation}
    \label{eq:P-tilde-definicion}
    \widetilde{P}(n)
     =
     \begin{cases}
       P(1)		     & \text{si \(n = 1\)} \\
       \widetilde{P}(n - 1) \wedge P(n) & \text{si \(n > 1\)} \\
     \end{cases}
  \end{equation}
  Aplicando la equivalencia
    \(A \implies B \equiv A \implies A \wedge B\),
  a definición~\eqref{eq:P-tilde-definicion} de \(\widetilde{P}(n)\)
  podemos escribir:
  \begin{align}
    (\widetilde{P}(n) \implies P(n + 1))
      &\implies
	 (\widetilde{P}(n)
	    \implies \widetilde{P}(n) \wedge P(n + 1))
      \notag \\
      &\implies
	 (\widetilde{P}(n) \wedge \widetilde{P}(n + 1))
      \label{eq:P-tilde-implica-P-tilde+1}
  \end{align}
  y con~\eqref{eq:P-tilde-implica-P-tilde+1}
  la inducción fuerte sobre \(P(n)\)
  no es más que inducción tradicional sobre \(\widetilde{P}(n)\).

  Un ejemplo simple ofrece la proposición siguiente.
  \begin{proposition}
    \begin{equation*}
      T(n)
	=
	\begin{cases}
	  0				      & \text{si\ } n = 0 \\
	  n + T(0) + T(1) + \dotsb + T(n - 1) & \text{si\ } n > 0
	\end{cases}
    \end{equation*}
    Entonces:
    \begin{equation*}
      T(n)
	= 2^n - 1
    \end{equation*}
  \end{proposition}
  \begin{proof}
    Usamos inducción fuerte sobre \(n\).
    \begin{description}
    \item[Caso base:]
      Para \(n = 0\),
      tenemos \(T(0) = 0 = 2^0 - 1\),
      que sigue de la definición de \(T\).
    \item[Inducción:]
      Suponemos cierto el resultado para \(0 \le k < n\),
      y tenemos:
      \begin{align*}
	T(n)
	  &= n + \sum_{0 \le k < n} T(k) \\
	  &= n + \sum_{0 \le k < n} (2^k - 1) \\
	  &= \sum_{0 \le k < n} 2^k \\
	  &= \frac{2^n - 1}{2 - 1} \\
	  &= 2^n - 1
      \end{align*}
      En esto hemos usado la suma de una serie geométrica
      (teorema~\ref{theo:suma-geometrica}).
    \end{description}
    Por inducción,
    vale para todo \(n \in \mathbb{N}\).
  \end{proof}
  Acá tuvimos que usar todos los casos anteriores.

  Otro ejemplo lo pone el juego de Nim,%
    \index{Nim, juego de}
  en el cual dos jugadores se enfrentan con dos pilas de fósforos.
  Por turno cada jugador
  saca un número de fósforos de una de las pilas.
  Gana quien toma el último fósforo.
  \begin{proposition}
    \label{prop:Nim}
    En el juego de Nim,
    si a la partida ambas pilas tienen el mismo número de fósforos,
    el segundo jugador gana.
  \end{proposition}
  La estrategia para el segundo jugador
  es crear y luego mantener esta situación,
  vale decir si el primer jugador saca \(m\) fósforos de una pila,
  el segundo saca el número adecuado de fósforos de la otra
  para que resulten iguales.
  Esto demuestra que el segundo jugador siempre puede ganar.
  \begin{proof}
    Usamos inducción fuerte
    sobre el número \(n\) de fósforos en las pilas.
    \begin{description}
    \item[Base:]
      Cuando \(n = 1\),
      la única movida posible
      para el primer jugador es sacar un fósforo
      de una pila,
      y el segundo jugador se queda con el último.
    \item[Inducción:]
      Supongamos que esto es válido
      para todos los números de fósforos
      en las pilas entre \(1\) y \(n\),
      demostraremos que vale para pilas con \(n + 1\) fósforos.
      Consideremos el caso de dos pilas de \(n + 1\) fósforos,
      con el turno del primer jugador.
      Si este saca \(m\) fósforos de una de las pilas,
      el segundo puede sacar \(m\) de la otra,
      y quedamos en la situación
      con dos pilas de \(n + 1 - m \le n\)
      fósforos cada una.
      Por hipótesis,
      el segundo jugador gana desde esta posición.
    \end{description}
    Por inducción,
    si al comienzo las dos pilas tienen el mismo número de fósforos,
    gana el segundo jugador.
  \end{proof}
  Recurrimos solo a uno de los casos anteriores
  en esta demostración,
  pero al no estar determinado cuál
  necesitamos debemos suponerlos todos.

  Un ejemplo más tentador es el siguiente:
  Se tiene una barra de chocolate
  que se divide en \(n\) cuadraditos.
  Se pide determinar cuántas veces como mínimo
  se debe partir la barra
  para dividirla en sus cuadraditos individuales.
  \begin{proposition}
    \label{prop:cortar-chocolate}
    Para dividir una barra de chocolate de \(n\) cuadraditos
    en sus cuadraditos individuales
    se requieren exactamente \(n - 1\) cortes.
  \end{proposition}
  \begin{proof}
    Usamos inducción fuerte.
    \begin{description}
    \item[Base:]
      Cuando \(n = 1\),
      claramente se requieren \(n - 1 = 0\) cortes.
    \item[Inducción:]
      Suponiendo que la aseveración es cierta
      para \(1 \le k \le n\),
      demostramos que es cierta para \(n + 1\).
      En el primer paso dividimos la barra en dos partes,
      de \(n_1\) y \(n_2\) cuadraditos respectivamente,
      donde \(n_1 + n_2 = n + 1\).
      Por hipótesis de inducción,
      requeriremos \(n_1 - 1\) y \(n_2 - 1\) cortes para las partes,
      respectivamente;
      en total se requieren
	\((n_1 - 1) + (n_2 - 1) + 1 = (n_1 + n_2) - 1 = n\) cortes.
    \end{description}
    Por inducción queda demostrado que se requieren \(n - 1\) cortes
    para todo \(n \in \mathbb{N}\).
  \end{proof}
  Nótese que esta demostración
  es aplicable a cualquier forma de la barra original,
  no solo a las tradicionales barras rectangulares
  que se dividen en cuadraditos.
  La única restricción
  es que cada corte divida una barra en dos partes.

  En esta demostración usamos dos casos anteriores,
  y
  (como en el ejemplo precedente)
  no podemos determinar de antemano cuáles serían,
  por lo que fue necesario suponerlos todos.

  Es común que no resulte posible demostrar el paso inductivo.
  Nuestro instinto nos dice
  que en tal caso lo que debe hacerse es hacer más débil
  lo que deseamos demostrar;
  pero eso no sirve,
  ya que (de resultar) terminamos demostrando menos de lo pedido.
  La solución es demostrar una proposición más fuerte
  (lo que también da más con que trabajar).

  Un mecenas
  (llamémosle \foreignlanguage{english}{August})
  dona un pavimento para el patio de la Universidad de Miskatonic,%
    \index{pavimentacion@pavimentación}
  que casualmente tiene tamaño \mbox{\(2^n \times 2^n\)}.
  Pone como condición que debe instalarse una estatua suya
  adyacente al centro del patio.
  Archer Harris
  (el arquitecto de la Universidad)
  tiene sus propias ideas,
  todo debe pavimentarse con losas de forma en L
  (ver la figura~\ref{fig:tile}).
  \begin{figure}[htbp]
    \centering
    \pgfimage{images/tile}
    \caption{Forma de una losa}
    \label{fig:tile}
  \end{figure}
  La base de la estatua de \foreignlanguage{english}{August}
  tiene exactamente el tamaño
  de uno de los tres cuadrados que forman la losa.

  Lo que buscamos entonces es demostrar:
  \begin{proposition}
    \label{prop:pavimentar}
    Dadas las condiciones enunciadas,
    es posible pavimentar
    un patio de tamaño \mbox{\(2^n \times 2^n\)}
    dejando un espacio adyacente al centro
    para la estatua de \foreignlanguage{english}{August}.
  \end{proposition}

  \begin{figure}[htbp]
    \centering
    \pgfimage{images/tiling-fail}
    \caption{Intento fallido de inducción}
    \label{fig:tiling-fail}
  \end{figure}
  \begin{proof}[Demostración (Intento fallido)]
    \renewcommand{\qedsymbol}{\textthing}
    Usamos inducción.
    \begin{description}
    \item[Caso base:]
      Claramente se puede lograr lo pedido cuando \(n = 1\),
      con una losa y la estatua ocupando la esquina faltante
      (queda ``lo más cerca del centro posible'' de esta forma).
    \item[Inducción:]
      Suponiendo que se puede ubicar a
      \foreignlanguage{english}{August}
      adyacente al centro para tamaño \(2^n\),
      intentamos ahora demostrar
      que se puede hacer para \(2^{n + 1}\).
      Pero esto lleva a la situación
      de la figura~\ref{fig:tiling-fail},
      que no ayuda en nada.
      \qedhere
    \end{description}
  \end{proof}

  La solución es demostrar una cosa más fuerte,
  debemos buscar la condición que nos permita cerrar el ciclo.
  Es claro que \mbox{\(2^{n + 1} \times 2^{n + 1}\)} es el cuadrado
  en el cual la estatua de \foreignlanguage{english}{August}
  está en la esquina adyacente al centro
  rodeado por tres otros cuadrados similares llenos,
  como en la figura~\ref{fig:tiling-around}.
  \begin{figure}[htbp]
    \centering
    \pgfimage{images/tiling-around}
    \caption{División del patio de $2^{n + 1} \times 2^{n + 1}$
	     en cuatro de $2^n \times 2^n$}
    \label{fig:tiling-around}
  \end{figure}
  Algunos dibujos para el caso \mbox{\(4 \times 4\)}
  muestran que la manera de cubrir
  excluyendo el cuadrado \mbox{\(2 \times 2\)}
  en la esquina inferior izquierda
  es la que da la figura~\ref{subfig:tiling-around-sol}.
  \begin{figure}[htbp]
    \centering
    \subfloat[Alrededores]{
      \pgfimage{images/tiling-around-sol}
      \label{subfig:tiling-around-sol}
    }
    \qquad
    \subfloat[Otra vista]{
      \pgfimage{images/tiling-around-sol-square}
      \label{subfig:tiling-around-sol-square}
    }
    \qquad
    \subfloat[Solución]{
      \pgfimage{images/tiling-sol}
      \label{subfig:tiling-sol}
    }
    \caption{Análisis del patio de $4 \times 4$}
    \label{fig:tiling-around-sol}
  \end{figure}
  Esto puede considerarse
  como tres cuadrados de \mbox{\(2 \times 2\)}
  con cuadraditos faltantes en esquinas que se unen al centro,
  como la figura~\ref{subfig:tiling-around-sol-square}.
  La solución completa de \mbox{\(4 \times 4\)}
  la muestra la figura~\ref{subfig:tiling-sol}.

  Ahora tenemos dos caminos posibles:
  \begin{itemize}
  \item
    Notando que nuestro patio de \mbox{\(4 \times 4\)}
    lo hemos dividido
    como en la figura~\ref{fig:tiling-around}
    en un área de la forma de una losa
    y un cuadrado del doble tamaño,
    podemos construir
    cuatro cuadrados de \mbox{\(2^{n - 1} \times 2^{n - 1}\)}
    con un cuadradito faltante en una esquina cada uno,
    unir tres de las esquinas al centro y cubrirlas con una losa,
    dejando la cuarta adyacente al centro
    como en la figura~\ref{subfig:tiling-sol}.
  \item
    El espacio libre de la figura~\ref{subfig:tiling-around-sol}
    en un cuadrado de \mbox{\(4 \times 4\)}
    podemos cubrirlo de forma de de dejar el espacio
    para la estatua de \foreignlanguage{english}{August}
    en cualquiera de las posiciones en la esquina inferior,
    y por simetría
    en cualquiera de las posiciones en el patio completo.
    Esto hace sospechar
    que se puede ubicar a \foreignlanguage{english}{August}
    en cualquier posición en el patio.
  \end{itemize}
  La segunda estrategia lleva a la siguiente demostración,
  dejamos el detalle de la primera como entretención al lector.
  \begin{figure}[htbp]
    \centering
    \pgfimage{images/tiling-any}
    \caption{Inducción dejando libre cualquier cuadradito}
    \label{fig:tiling-any}
  \end{figure}
  \begin{proof}
    Usamos inducción
    para demostrar
    que la estatua de \foreignlanguage{english}{August}
    puede ubicarse en cualquier posición
    en un patio de \mbox{\(2^n \times 2^n\)}.
    \begin{description}
    \item[Caso base:]
      Para \(n = 1\) es simplemente que la losa
      se puede ubicar en cualquier orientación.
    \item[Inducción:]
      Suponiendo que es posible
      ubicar a \foreignlanguage{english}{August}
      en cualquier lugar en un patio de \mbox{\(2^n \times 2^n\)},
      podemos dividir
      nuestro patio de \mbox{\(2^{n + 1} \times 2^{n + 1}\)}
      en cuatro cuadrados de \mbox{\(2^n \times 2^n\)}.
      La posición designada de \foreignlanguage{english}{August}
      estará en uno de los cuadrantes,
      y por hipótesis podemos cubrir el resto de éste.
      En los otros tres cuadrantes
      ubicamos el espacio
      que \foreignlanguage{english}{August} ocuparía
      en una de las esquinas,
      ver la figura~\ref{fig:tiling-any}.
      También estos pavimentos son posibles,
      por hipótesis.
      Pero entonces podemos cubrir
      las tres esquinas adyacentes al centro
      con una losa,
      y es posible ubicar a \foreignlanguage{english}{August}
      en cualquier posición
      en un patio de \mbox{\(2^{n + 1} \times 2^{n + 1}\)}.
    \end{description}
    Por inducción,
    es posible ubicar a August en cualquier posición
    en un patio de \mbox{\(2^n \times 2^n\)}
    para todo \(n \in \mathbb{N}\).
    En particular,
    es posible ubicarlo adyacente al centro.
  \end{proof}

  El que resulte más fácil demostrar algo más general
  que lo que se busca directamente
  se conoce como la \emph{paradoja del inventor}.%
    \index{inventor, paradoja del|see{paradoja del inventor}}%
    \index{paradoja del inventor}
  Otro ejemplo de este fenómeno es el siguiente:%
    \index{Basilea, problema de}
  \begin{theorem}
    \label{theo:Basilea-converge}
    \begin{equation*}
      1 + \frac{1}{4} + \frac{1}{9} + \dotsb + \frac{1}{n^2}
	< 2
    \end{equation*}
  \end{theorem}
  Esto no se puede demostrar por inducción directamente,
  ya que al sumar algo al lado derecho
  se echa a perder la condición.
  Para poder cerrar el ciclo con un valor menor a \(2\)
  buscamos alguna diferencia \(d_n\),
  lo más simple posible,
  que dé:
  \begin{equation}
    \label{eq:Basilea-dn-condicion}
    2 - d_n + \frac{1}{(n + 1)^2}
     \le 2 - d_{n + 1}
  \end{equation}
  o,
  lo que es lo mismo:
  \begin{equation}
    \label{eq:Basilea-dn-condicion-2}
    d_n - d_{n + 1}
      \ge \frac{1}{(n + 1)^2}
  \end{equation}
  Debe ser \(0 < d_n \le 1\) para no meternos en problemas.
  Sumando~\eqref{eq:Basilea-dn-condicion-2}
  de \(n + 1\) en adelante
  vemos que en realidad queremos que:
  \begin{equation}
    \label{eq:Basilea-dn-condicion-3}
    d_n
      > \sum_{k \ge n + 1} \frac{1}{k^2}
  \end{equation}
  Podemos estimar crudamente la suma mediante una integral:
  \begin{equation}
    \label{eq:Basilea-condicion-aproximacion}
    \sum_{k \ge n + 1} \frac{1}{k^2}
      \approx \int_{n + 1}^\infty \frac{\mathrm{d} x}{x^2}
      = \frac{1}{n + 1}
  \end{equation}
  La relación~\eqref{eq:Basilea-condicion-aproximacion}
  hace sospechar que algo como \(d_n = 1 / n\)
  (la expresión \(1 / n\) es más simple que \(1 / (n + 1)\),
   además que siendo mayor da mayor holgura)
  pueda funcionar,
  lo que lleva a la demostración siguiente.
  \begin{proof}
    Por inducción demostramos un resultado más fuerte:
    \begin{equation}
      \label{eq:Basilea-converge-hipotesis}
      1 + \frac{1}{4} + \frac{1}{9} + \dotsb + \frac{1}{n^2}
	\le 2 - \frac{1}{n}
    \end{equation}
    \begin{description}
    \item[Caso base:]
      Para \(n = 1\) la ecuación~\eqref{eq:Basilea-converge-hipotesis}
      se reduce a \(1 \le 1\),
      que ciertamente es verdad.
    \item[Inducción:]
      Suponiendo que~\eqref{eq:Basilea-converge-hipotesis}
      vale para \(n\),
      demostramos que vale para \mbox{\(n + 1\)}.
      Partiendo de nuestra hipótesis:
      \begin{align}
	1 + \frac{1}{4} + \frac{1}{9} + \dotsb + \frac{1}{n^2}
	  &\le 2 - \frac{1}{n} \notag \\
	1 + \frac{1}{4}
	  + \dotsb + \frac{1}{n^2} + \frac{1}{(n + 1)^2}
	  &\le 2 - \frac{1}{n} + \frac{1}{(n + 1)^2}
	  \label{eq:Basilea-converge-induccion}
      \end{align}
      Consideremos los últimos dos términos
      de~\eqref{eq:Basilea-converge-induccion},
      donde como \(n \ge 1\):
      \begin{align}
	\frac{1}{n} - \frac{1}{(n + 1)^2}
	  &= \frac{n^2 + n + 1}{n (n + 1)^2} \notag \\
	  &> \frac{n^2 + n}{n (n + 1)^2} \notag \\
	  &= \frac{1}{n + 1}
	  \label{eq:Basilea-converge-induccion-cota}
      \end{align}
      Con~\eqref{eq:Basilea-converge-induccion-cota}
      en~\eqref{eq:Basilea-converge-induccion}
      tenemos:
      \begin{align*}
	1 + \frac{1}{4}
	  + \dotsb + \frac{1}{n^2} + \frac{1}{(n + 1)^2}
	  &\le 2 - \frac{1}{n} + \frac{1}{(n + 1)^2} \\
	  &\le 2 - \frac{1}{n + 1}
      \end{align*}
      como se prometió.
    \end{description}
    Por inducción~\eqref{eq:Basilea-converge-hipotesis}
    vale para todo \(n \in \mathbb{N}\).
    Pero:
    \begin{equation*}
      1 + \frac{1}{4} + \frac{1}{9} + \dotsb + \frac{1}{n^2}
	\le 2 - \frac{1}{n} < 2
    \end{equation*}
    que es lo que se quería probar.
  \end{proof}

\subsection{Inducción estructural}
\label{sec:induccion-estructural}
\index{induccion estructural@inducción estructural|see{demostración!inducción!estructural}}
\index{demostracion@demostración!induccion@inducción!estructural|textbfhy}

  Esta es una generalización
  del método de demostración por inducción.
  La idea es aplicable a estructuras definidas recursivamente,
  como árboles binarios,
    \index{arbol binario@árbol binario}
  expresiones aritméticas
  o listas.

  Supongamos un orden parcial bien fundado%
    \index{relacion@relación!bien fundada}
  (una relación \(R\) es \emph{bien fundada}
   si todo subconjunto de \(\mathcal{X}\)
   tiene un \emph{mínimo} respecto de \(R\),
   o sea,
   si todo subconjunto no vacío \(\mathcal{S}\) de \(\mathcal{X}\)
   contiene al menos un elemento \(m\)
   tal que para todo \(s \in \mathcal{S}\)
   es \(s \not\mathrel{R} m\).
  Dicho de otra forma,
  toda secuencia \(x_1 > x_2 > x_3 \dotso\)
  en \(\mathcal{X}\) termina,
  si anotamos \(>\)
  para la transpuesta
  de \(R\).

  Para demostrar que \(P(x)\) es cierto
  para todo \(x \in \mathcal{X}\),
  demostramos (base) que vale para las elementos mínimos,
  y que (inducción) si vale para todo \(y\)
  tal que \(y \mathrel{R} x\),
  entonces vale para \(x\).
  Esto se justifica
  suponiendo que algún elemento \(x \in	 \mathcal{X}\)
  es el mínimo contraejemplo.
  Entonces no puede ser un elemento mínimo de \(\mathcal{X}\),
  ya que por la base para todos ellos se cumple \(P()\);
  y como vale para todos los \(y\) con \(y \mathrel{R} x\)
  ya que \(x\) es el mínimo contraejemplo,
  vale para \(x\),
  una clara contradicción.

  Por ejemplo,
  consideremos listas,
  definidas mediante:
  \begin{itemize}
  \item
    \([]\) es una lista
    (la \emph{lista vacía})
   \item
     Si \(H\) es un elemento
     y \(T\) es una lista,
     \(H : T\) es una lista.
     Llamaremos \emph{cabeza} a \(H\),
     y \emph{cola} a \(T\).
  \end{itemize}
  Esto define un orden parcial entre listas,
  en que \(T \le L\)
  si \(L = H : T\) o \(L = T\).

  Para listas definimos una operación \(\operatorname{length}\)
  mediante:
  \begin{description}
  \item[LEN1]
    \(\operatorname{length}([]) = 0\)
  \item[LEN2]
    \(\operatorname{length}(H : T) = 1 + \operatorname{length}(T)\)
  \end{description}
  Definimos la operación de concatenación de listas,
  anotada \(L_1 \cdot L_2\),
  por:
  \begin{description}
  \item[CAT1]
    \([] \cdot L = L\)
  \item[CAT2]
    \((H : T) \cdot L = H : (T \cdot L)\)
  \end{description}
  Interesa demostrar:
  \begin{theorem}
    \label{theo:largo-concatenacion-listas}
    \begin{equation}
      \label{eq:largo-concatenado}
      \operatorname{length}(L \cdot M)
      = \operatorname{length}(L) + \operatorname{length}(M)
    \end{equation}
  \end{theorem}
  \begin{proof}
    Esto lo demostramos por inducción estructural sobre listas.
    \(P(L)\) asevera que~\eqref{eq:largo-concatenado}
    vale para toda lista \(L\).
    \begin{description}
    \item[Base:]
      Primero demostramos que \(P([])\) es verdadero:
      \begin{align*}
	\operatorname{length}([] \cdot M)
	&= \operatorname{length}(M)
	   && \text{por CAT1} \\
	&= 0 + \operatorname{length}(M) \\
	&= \operatorname{length}([]) + \operatorname{length}(M)
	   && \text{por LEN1}
      \end{align*}
    \item[Inducción:]
      Si \(L\) no es vacía,
      consta de cabeza y cola,
      o sea \(L = H : T\).
      Nuestra hipótesis de inducción
      es que si \(P(T)\) vale para la cola \(T\) de la lista,
      entonces vale para la lista.
      En detalle,
      nuestra hipótesis es:
      \begin{equation*}
	\operatorname{length}(T \cdot M)
	  = \operatorname{length}(T) + \operatorname{length}(M)
      \end{equation*}
      y tenemos:
      \begin{align*}
	\operatorname{length}(L \cdot M)
	  &= \operatorname{length}((H : T) \cdot M) \\
	  &= \operatorname{length}(H : (T \cdot M))
	     && \text{por CAT2} \\
	  &= 1 + \operatorname{length}(T \cdot M)
	     && \text{por LEN2} \\
	  &= 1 + \operatorname{length}(T) + \operatorname{length}(M)
	     && \text{por hipótesis} \\
	  &= \operatorname{length}(H : T) + \operatorname{length}(M)
	     && \text{por LEN2} \\
	  &= \operatorname{length}(L) + \operatorname{length}(M)
      \end{align*}
    \end{description}
    Por inducción estructural,
    vale para todas las listas.
  \end{proof}

  Si definimos expresiones algebraicas como:
  \begin{itemize}
  \item
    Un \emph{átomo}
    (una variable o constante)
  \item
    Una expresión entre paréntesis
  \item
    Dos expresiones unidas por un \emph{operador},
    uno de \(\{+, -, \cdot, /\}\)%
      \index{operador}
  \end{itemize}
  Esto define una relación,
  en la cual todas las cadenas terminan en átomos;
  por tanto es bien fundada.

  \begin{theorem}
    \label{theo:operandos-operadores}
    En toda expresión con \(n\) átomos hay \(n - 1\) operadores.
  \end{theorem}
  \begin{proof}
    Por inducción estructural.
    \begin{description}
    \item[Base:]
      Si la expresión es un único átomo,
      tiene \(1\) átomo y \(0\) operadores.
    \item[Inducción:]
      Si la expresión es una expresión entre paréntesis,
      tiene el mismo número de átomos y operadores que esta última.

      Si la expresión consta de dos expresiones \(E_1\) y \(E_2\)
      unidas por un operador,
      por inducción si \(E_1\) tiene \(n_1\) átomos
      tiene \(n_1 - 1\) operadores,
      y si \(E_2\) tiene \(n_2\) átomos
      tiene \(n_2 - 1\) operadores.
      La expresión completa tiene \(n = n_1 + n_2\) átomos
      y \(n_1 - 1 + n_2 - 1 + 1 = n - 1\) operadores.
    \end{description}
    Por inducción estructural,
    vale para todas las expresiones.
  \end{proof}

  A esta técnica se le conoce como \emph{inducción estructural}
  porque
  (como muestran los ejemplos)
  la aplicación típica es a estructuras definidas recursivamente,
  partiendo de estructuras mínimas y reglas para combinarlas
  creando estructuras más complejas.
  La demostración por inducción estructural en tal caso
  corresponde a demostrar
  que la propiedad vale para las estructuras mínimas,
  y que si la propiedad vale para las estructuras componentes
  también vale para la estructura que las incluye.

  Las demostraciones por inducción
  vistas en la sección~\ref{sec:induccion}
  corresponden a considerar el conjunto \(\mathbb{N}\)
  con la relación sucesor
  que relaciona a \(n\) con \(n + 1\);
  la inducción fuerte
  corresponde a \(\mathbb{N}\) con la relación menor que.

\section{Demostrar existencia}
\label{sec:existencia}
\index{demostracion@demostración!existencia}

  Muchas veces interesa demostrar que algún objeto
  que cumple ciertas condiciones existe.
  La forma más simple de hacer esto es exhibir el objeto del caso
  (para demostrar que hay primos pares,
   basta exhibir el primo \(2\))
  o dar un mecanismo claro que permite construirlo.
  Sin embargo,
  se da también la situación
  que podemos demostrar la existencia del objeto,
  sin poder asirlo de forma más concreta.
  Esto definitivamente es bastante poco satisfactorio,
  pero válido de todas formas.
  Un ejemplo es lo siguiente:
  \begin{proposition}
    Hay números irracionales \(\alpha\) y \(\beta\)%
      \index{numero@número!irracional}
    tales que \(\alpha^\beta\) es racional.
  \end{proposition}
  \begin{proof}
    Sabemos que \(\sqrt{2}\) es irracional.
    Entonces:
    \begin{equation*}
      \left( \sqrt{2}^{\sqrt{2}} \right)^{\sqrt{2}}
	= \sqrt{2}^2
	= 2
    \end{equation*}
    Ahora bien,
    hay dos opciones para \(\sqrt{2}^{\sqrt{2}}\):
    Si es racional,
    entonces podemos tomar \(\alpha = \beta = \sqrt{2}\)
    como ejemplo.
    Si es irracional,
    tomamos \(\alpha = \sqrt{2}^{\sqrt{2}}\) y \(\beta = \sqrt{2}\)
    como ejemplo.
  \end{proof}
  Bonito,
  pero es frustrante en que no da un ejemplo concreto.
  Otro ejemplo clásico es la demostración de Cantor%
    \index{Cantor, Georg}
  (que discutiremos en el capítulo~\ref{cha:numerabilidad})
  que hay infinitos números trascendentes%
    \index{numero@número!trascendente}
  (números irracionales
   que no son ceros de un polinomio con coeficientes enteros)
  sin dar luces de cómo obtener alguno.

  Una demostración alternativa,
  que da ejemplos concretos,
  es:
  \begin{proof}
    Sea \(\alpha = \sqrt{2}\),
    \(\beta = \log_2 9\),
    con lo que \(\alpha^\beta = 3\).

    Sabemos que \(\alpha\) es irracional
    por el teorema~\ref{theo:sqrt2-irracional}.
    Demostramos que \(\beta\) es irracional por contradicción.%
      \index{demostracion@demostración!contradiccion@contradicción}
    Supongamos que \(\beta\) fuera racional,
    digamos que para naturales \(m\) y \(n\)
    tenemos \(\beta = m / n\).
    Claramente es \(\beta = \log_2 9 > 0\),
    con lo que \(m > 0\).
    Entonces sería \(9^n = 2^m\),
    pero \(9^n\) es impar,
    mientras \(2^m\) es par.
    Esto es imposible.
  \end{proof}

  Se ha demostrado
  que el número de Hilbert%
    \index{Hilbert, numero de@Hilbert, número de}%
    \index{Hilbert, David}
  (también conocido como constante de Gelfond-Schneider)%
    \index{Gelfond-Schneider, constante de|see{Hilbert, número de}}
  \(2^{\sqrt{2}}\) es irracional
  (incluso es trascendente),
    \index{numero@número!trascendente}
  y da otro ejemplo concreto al elevar a \(\sqrt{2}\).
  Por lo demás,
  siendo trascendente \(2^{\sqrt{2}}\),
  su raíz
  (el número que consideramos arriba)
  también es trascendente.
  Claro que para discutir estos temas habría que profundizar mucho más\ldots

\section{Refutaciones}
\label{sec:refutaciones}
\index{refutacion@refutación}

  Nos hemos concentrado hasta acá
  en demostrar cosas que sabemos ciertas.
  En la cruda realidad
  de las matemáticas nos encontramos con mayor frecuencia
  con aseveraciones que no sabemos si son ciertas o falsas
  (\emph{conjeturas}).%
    \index{conjetura}
  La tarea se compone,
  entonces,
  de determinar si la conjetura es verdadera o no,
  y luego demostrar que es verdadera
  o que no se cumple.
  Si no somos capaces de demostrar que la conjetura es cierta,
  esto no demuestra que sea falsa:
  Puede ser cierta,
  simplemente no hemos sido capaces de demostrarlo.

  Igual que para demostrar que una aseveración es cierta,
  hay ciertas técnicas para demostrar que es falsa.
  La forma más simple de refutar la aseveración \(P\)
  es demostrar \(\neg P\).
  Esta última a su vez es susceptible
  de cualquiera de las técnicas ya discutidas.
  Igual hay una variedad de situaciones especiales
  que merecen atención,
  como explica Hammack en su texto~%
    \cite{hammack13:_book_proof}.

\subsection{Refutar aseveraciones universales: Contraejemplo}
\label{sec:refutar-universal}
\index{contraejemplo}

  Si debemos demostrar
  que la aseveración \(\forall x \in S. P(x)\) es falsa,
  nuestra indicación general es demostrar su negación:
  \begin{equation*}
    \neg (\forall x \in S. P(x))
      \equiv \exists x \in S. \neg P(x)
  \end{equation*}
  O sea,
  debemos hallar \(x\) tal que \(P(x)\) sea falso.
  Basta exhibir un solo contraejemplo
  para que no valga para todo \(x\).
  Por ejemplo:
  \begin{conjecture}
    Si \(n^2 - n\) es par,
    entonces \(n\) es par.
  \end{conjecture}
  \begin{proof}[Refutación]
    Exhibimos un contraejemplo:
    Para \(n = 1\) tenemos \(n^2 - n = 0\),
    que es par.
    Pero \(n\) es impar.
  \end{proof}

\subsection{Refutar existencia}
\label{sec:refutar-existencia}

  Si queremos demostrar que es falso \(\exists x \in S. P(x)\),
  nuevamente es demostrar la negación:
  \begin{equation*}
    \neg (\exists x \in S. P(x))
      \equiv \forall x \in S. \neg P(x)
  \end{equation*}
  Como esto es una aseveración universal,
  un ejemplo no es suficiente.
  \begin{conjecture}
    Hay números primos \(p\) y \(q\)
    tales que \(p - q = 97\).
  \end{conjecture}
  \begin{proof}[Refutación]
    Demostramos que si \(q\) es primo,
    entonces \(p = q + 97\) no es primo.
    Dividimos la demostración en dos casos:
    \begin{description}
    \item[\boldmath\(q = 2\)\unboldmath:]
      En este caso,
      \(q + 97 = 99 = 3 \cdot 3 \cdot 11\),
      que no es primo.
    \item[\boldmath\(q\)\unboldmath\ es un primo impar:]
      Si \(q\) es impar,
      entonces \(q + 97\) es un par mayor a \(2\),
      y por tanto no es primo.
    \end{description}
  \end{proof}

\subsection{Refutar por contradicción}
\label{sec:refutar-contradiccion}
\index{demostracion@demostración!contradiccion@contradicción}

  Ciertamente es posible aplicar la técnica de reducción al absurdo
  a la tarea de refutar.
  Sólo que en este caso lo que buscamos es obtener una contradicción
  de la aseveración misma,
  no de su negación.
  \begin{conjecture}
    Hay un real \(x\) para el cual \(x^4 < x < x^2\).
  \end{conjecture}
  \begin{proof}[Refutación]
    Por contradicción.%
      \index{refutacion@refutación!contradiccion@contradicción}
    Supongamos que la conjetura es cierta.
    Sea \(x\) un número real para el cual \(x^4 < x < x^2\).
    Como \(x > x^4\),
    \(x\) es positivo.
    Partimos con:
    \begin{equation*}
      x^4 < x < x^2
    \end{equation*}
    Dividiendo por \(x\):
    \begin{align*}
      x^3     &< 1 < x \\
      x^3 - 1 &< 0 < x - 1
    \end{align*}
    Factorizamos \(x^3 - 1\):
    \begin{equation*}
      (x - 1) (x^2 + x + 1)
	< 0
    \end{equation*}
    Como \(x - 1 > 0\), podemos dividir:
    \begin{equation*}
      x^2 + x + 1
	< 0
    \end{equation*}
    Pero esto último es imposible,
    ya que \(x > 0\).
  \end{proof}

\section{Conjetura a teorema}
\label{sec:conjetura->teorema}

  Una pregunta abierta luego de lo anterior
  es cómo se transforma una buena sospecha en un teorema
  (una solución al problema).
  Daremos un ejemplo simple,
  que más adelante servirá
  para ilustrar algunas de las técnicas más poderosas
  que presenta este texto.
  Ejemplos mucho más acabados
  muestran Bruckner, Thomson y Bruckner~%
    \cite{bruckner11:_mathem_discovery}.

  \begin{example}
    En la Competencia de Ensayos de la Universidad de Miskatonic
    los ensayos deben entregarse anónimamente,
    cada uno acompañado por una tarjeta que identifica al autor.
    Las tarjetas codificadas son tarjetas cuadradas idénticas
    de \(n \times n\)\,[mm]
    (\(n\) es un número impar)
    divididas por ambos lados en cuadrados de \(1\)\,[mm].
    En uno de estos cuadrados se perfora un agujero redondo.

    Sea \(b_n\) el número de tarjetas diferentes
    que se pueden producir de esta forma,
    bajo el supuesto que las tarjetas se pueden rotar e invertir.
    Se pide encontrar una fórmula para \(b_n\) en términos de \(n\),
    y usarla para determinar cuánto debe ser \(n\)
    si se esperan \(100\)~participantes en el concurso.
  \end{example}

% Note: El caso par no se divide "decentemente" en uno más chico
%	del mismo tipo

  \begin{solution}
    Atacaremos el problema en etapas.

    \begin{description}
    \item[Paso 1: Experimentar.]
      Como nos dijeron que \(n\) es impar,
      analizaremos los casos \(n = 1, 3, 5\) para comenzar.
      En el caso \(n = 1\),
      hay un único cuadradito,
      y por tanto es posible una única chapa.
      \(b_1 = 1\).

      En el caso \(n = 3\),
      un par de intentos
      muestran que solo hay \(3\) posibilidades diferentes,
      como indica la figura~\ref{fig:chapas-3}.
      \begin{figure}[htbp]
	\centering
	\subfloat{\pgfimage{images/badge-3a}}%
	  \hspace*{2em}%
	\subfloat{\pgfimage{images/badge-3b}}%
	  \hspace*{2em}%
	\subfloat{\pgfimage{images/badge-3c}}
	\caption{Chapas posibles con $n = 3$}
	\label{fig:chapas-3}
      \end{figure}
      Cualquier otra alternativa puede rotarse
      de forma de obtener una de estas.
      O sea,
      \(b_3 = 3\).

      Algo de trabajo adicional lleva a concluir que \(b_5 = 6\),
      véase la figura~\ref{fig:chapas-5}.
      \begin{figure}[htbp]
	\centering
	\subfloat{\pgfimage[width=0.275\textwidth]{images/badge-5a}}%
	  \hspace*{1em}%
	\subfloat{\pgfimage[width=0.275\textwidth]{images/badge-5b}}%
	  \hspace*{1em}%
	\subfloat{\pgfimage[width=0.275\textwidth]{images/badge-5c}}
	  \vspace*{1em}
	\subfloat{\pgfimage[width=0.275\textwidth]{images/badge-5d}}%
	  \hspace*{1em}%
	\subfloat{\pgfimage[width=0.275\textwidth]{images/badge-5e}}%
	  \hspace*{1em}%
	\subfloat{\pgfimage[width=0.275\textwidth]{images/badge-5f}}
	\caption{Chapas posibles con $n = 5$}
	\label{fig:chapas-5}
      \end{figure}
    \item[Paso 2: Adivinar.]
      Hay dos estrategias posibles para adivinar la solución.
      La ``heroica''
      es adivinar una fórmula para \(b_n\) directamente,
      en base a la información obtenida hasta acá
      (posiblemente complementada con valores adicionales).
      De ser así,
      se puede proceder directamente al paso 5.

      La otra es usar la estrategia ``segura'',
      que pasa por encontrar una relación recursiva
      entre los valores buscados.
      Acá la pregunta es cómo pasar de \(n = 1\) a \(n = 3\),
      y de \(n = 3\) a \(n = 5\),
      y así sucesivamente.
      Tenemos hasta acá:
      \begin{align}
	b_1
	  &= 1
	  \label{eq:ensayos-1} \\
	b_3
	  &= b_1 + 2
	  \label{eq:ensayos-3} \\
	b_5
	  &= b_3 + 3
	  \label{eq:ensayos-5}
      \end{align}
      Tal vez \(b_7 = b_5 + 4 = 10\),
      y en general se cumple:
      \begin{equation}
	\label{eq:ensayos-conjetura}
	b_{2 r + 1} = b_{2 r - 1} + r + 1
      \end{equation}
      Con esto tenemos una \emph{conjetura}.
    \item[Paso 3: Entender.]
      Debemos corroborar
      nuestra conjetura~\eqref{eq:ensayos-conjetura},
      con el caso siguiente \(n = 7\).
      Esto puede hacerse verificando \(b_7 = 10\) directamente,
      pero es mucho mejor analizar cómo se relacionan
      \(b_3\) con \(b_5\)
      (deben analizarse
       las figuras~\ref{fig:chapas-3} y~\ref{fig:chapas-5})
      y \(b_7\) con \(b_5\).
      Esto último lo ilustra la figura~\ref{fig:badge-7}.
      \begin{figure}[htbp]
	\centering
	\pgfimage[height=5cm]{images/badge-7}
	\caption{Chapa de $7 \times 7$
		 como chapa de $5 \times 5$ con borde}
	\label{fig:badge-7}
      \end{figure}
      Puede verse que la chapa de \(7 \times 7\) puede considerarse
      como una de \(5 \times 5\) con un borde.
      Si se perfora alguno de los cuadraditos del centro,
      se obtienen \(b_5\) posibilidades.
      Las otras opciones resultan de perforar el borde.
      Un momento de reflexión,
      también comparando el caso \(n = 3\) con el \(n = 5\),
      muestra que solo la mitad del borde superior
      aporta alternativas esencialmente diferentes,
      o sea,
      aporta \(4\)~más.
      Hemos demostrado que \(b_7 = b_5 + 4\).

      Con esto se completa el trabajo pesado.
      El mismo razonamiento confirma
      la conjetura~\eqref{eq:ensayos-conjetura}
      para el caso general.
    \item[Paso 4: La fórmula.]
      Nos piden una fórmula para los \(b_n\),
      y ahora sabemos que quedan descritos
      por la secuencia definida recursivamente por:
      \begin{equation}
	\label{eq:ensayos-recurrencia}
	b_{2 r + 1}
	  =
	  \begin{cases}
	    1			& \text{si \(r = 0\)} \\
	    b_{2 r - 1} + r + 1	& \text{si \(r \ge 1\)}
	  \end{cases}
      \end{equation}
      Ya conocemos los valores
      \(b_1 = 1\), \(b_3 = 3\), \(b_5 = 6\), \(b_7 = 10\),
      y podemos calcular valores adicionales según se requieran.
      Posiblemente ya hayamos dado en cuenta que:
      \begin{equation}
	\label{eq:ensayos-valores}
	 1 = (1 \cdot 2) / 2,\hspace{1ex}
	 3 = (2 \cdot 3) / 2,\hspace{1ex}
	 6 = (3 \cdot 4) / 2,\hspace{1ex}
	10 = (4 \cdot 5) / 2
      \end{equation}
      lo que lleva a sospechar:
      \begin{equation}
	\label{eq:ensayos-valor-explicito}
	b_{2 r + 1}
	  = \frac{1}{2} \, (r + 2) (r + 1)
	       \quad\text{para todo \(r \in \mathbb{N}\)}
      \end{equation}
    \item[Paso 5: Demostración.]
      Resta demostrar
      que la fórmula~\eqref{eq:ensayos-valor-explicito}
      es correcta.
      \begin{proposition}
	\label{prop:Miskatonic-essay-competition}
	Para todo \(r \in \mathbb{N}_0\),
	\(b_{2 r + 1} = \frac{1}{2} \, (r + 2) (r + 1)\)
      \end{proposition}
      \begin{proof}
	\index{demostracion@demostración!induccion@inducción}
	Por inducción sobre \(r\).
	\begin{description}
	\item[Base:]
	  La fórmula~\eqref{eq:ensayos-valor-explicito}
	  vale cuando \(r = 0\),
	  ya que \(b_1 = 1\)
	  y \((0 + 2) (0 + 1) / 2 = 1\).
	\item[Inducción:]
	  Supongamos que~\eqref{eq:ensayos-valor-explicito}
	  se cumple para \(r = k\),
	  vale decir,
	  \(b_{2 k + 1} = \frac{1}{2} \, (k + 2) (k + 1)\).
	  Usando la relación~\eqref{eq:ensayos-conjetura}
	  descubierta en el paso~3:
	  \begin{align*}
	    b_{2 (k + 1) + 1}
	      &= b_{2 k + 1} + k + 2 \\
	      &= \frac{1}{2} \, (k + 2) (k + 1) + k + 2 \\
	      &= \frac{1}{2} \, (k + 2) (k + 3) \\
	      &= \frac{1}{2} \, ((k + 1) + 2) ((k + 1) + 1)
	  \end{align*}
	\end{description}
	La fórmula vale
	para todo \(r \in \mathbb{N}_0\) por inducción.
      \end{proof}
    \item[Paso 6: La respuesta.]
      El duro trabajo de los matemáticos
      se aprecia solo cuando entrega una respuesta definitiva
      a un problema práctico.
      En este caso,
      la ``respuesta definitiva'' es el tamaño de la tarjeta
      necesaria para \(100\)~participantes,
      o sea,
      el valor \(r^*\)
      tal que \(b_{2 r^* + 1} \le 100 < b_{2 r^* + 3}\).
      Planteamos:
      \begin{equation}
	\label{eq:ensayos-desigualdad}
	\frac{1}{2} \, (r + 2) (r + 1)
	  \ge 100
      \end{equation}
      Los ceros del polinomio que corresponde
      a~\eqref{eq:ensayos-desigualdad}
      son \(r = -3,217\) y \(r = 12,65\).
      Como para \(r \ge 0\)
      la expresión
      del lado izquierdo
      de la desigualdad~\eqref{eq:ensayos-desigualdad}
      es creciente,
      aproximamos al entero superior y tenemos \(r^* = 13\).
      Las chapas son de \(27 \times 27\)\,[mm],
      lo que da para \(105\)~participantes.
    \end{description}
  \end{solution}

  Para mayor detalle sobre cómo resolver problemas,
  particularmente en el ámbito matemático,
  dirigimos al lector al clásico de Pólya~%
    \cite{polya45:_how_to_solve_it},%
    \index{Polya, George@Pólya, George}
  quien plantea cuatro pasos esenciales:
  \begin{enumerate}
  \item
    \textbf{Entender el problema.}
    Muchas veces este paso se omite como obvio,
    pero es frecuente empantanarse
    por no entender completamente el problema.
    Como remedio se sugieren preguntas como:
    \begin{itemize}
    \item
      ¿Que se busca encontrar?
    \item
      Plantee el problema en sus propias palabras.
    \item
      ¿Se le ocurre un dibujo o diagrama
      que ayude a entender el problema?
    \item
      ¿Hay suficiente información para resolver el problema?
    \end{itemize}
  \item
    \textbf{Idear un plan.}
    Hay muchas maneras razonables de resolver problemas.
    La habilidad de elegir la estrategia apropiada
    se adquiere resolviendo sistemática y ordenadamente
    problemas diversos.
    Una lista parcial de estrategias a considerar es:
    \begin{itemize}
    \item
      Adivine y verifique.
    \item
      Elabore una lista ordenada.
    \item
      Elimine posibilidades.
    \item
      Use simetría.
    \item
      Considere casos especiales.
    \item
      Razonamiento directo.
    \item
      Busque patrones.
    \item
      Resuelva un problema más simple.
    \item
      Trabaje en reversa.
    \end{itemize}
  \item
    \textbf{Lleve a cabo el plan.}
    Esto suele ser más fácil que idear el plan.
    Generalmente solo requiere cuidado y paciencia,
    si tiene las habilidades requeridas.
    Persista.
    Si no funciona,
    descártelo e intente otro.
  \item
    \textbf{Revise/extienda.}
    Puede ganar mucho tomándose el tiempo de reflexionar,
    mirando atrás para ver qué funcionó y qué no.
    Esto ayuda a elegir la estrategia adecuada en casos futuros
    relacionados con este.
  \end{enumerate}
  Pero tal vez más importante que la habilidad de resolver problemas
  es hallar preguntas interesantes a resolver.
  Brown y Walter~\cite{brown05:_art_problem_posing}
  animan a cultivar el arte de plantear problemas.

%%% Local Variables:
%%% mode: latex
%%% TeX-master: "clases"
%%% End:


% correctitud-programas.tex
%
% Copyright (c) 2012-2014 Horst H. von Brand
% Derechos reservados. Vea COPYRIGHT para detalles

\chapter{Correctitud de programas}
\label{cha:programas}
\index{programa!correctitud}

  Cada día dependemos más de programas computacionales.
  Lamentablemente,
  muchos programas han demostrado ser incorrectos,
  errores de programación han dado lugar a costos inmensos
  (el reporte de Calude, Calude y Marcus~%
     \cite{calude07:_provin_progr}
   da algunos ejemplos notorios,
   pero todos hemos sufrido en menor o mayor medida
   a causa del comportamiento erróneo de programas).
  Dijkstra~\cite{dijkstra89:_cruelty_teach_comput_scien}%
    \index{Dijkstra, Edsger W.}
  observa que la computación
  significa un cambio radical en nuestro mundo,
  al introducir artefactos cuyo funcionamiento no es continuo,
  y más aún dar lugar a fenómenos que abarcan órdenes de magnitud
  absolutamente inimaginables en otras disciplinas.
  Veremos cómo aplicar técnicas de razonamiento matemático
  a la tarea de asegurar que un programa funcione como se espera.

\section{Lógica de Hoare}
\label{sec:logica-de-hoare}
\index{Hoare, logica de@Hoare, lógica de}

% Lógica de Hoare
%   \cite{floyd67:_assig_meanin_progr,
%	  hoare69:_axiom_basis_comput_progr,
%	  gries87:_scien_progr}
%
% Correctitud parcial/total
% Técnicas de razonamiento
% Ejemplos ~~--> Programming Pearls
% Herramientas como splint <http://www.splint.org>
% o sparse <http://www.kernel.org/pub/software/devel/sparse>
% (a la pasada cosas como doxygen, valgrind, ...?)

  Ya a fines de los años 60,
  Floyd~\cite{floyd67:_assig_meanin_progr}%
    \index{Floyd, Robert W.}
  y luego Hoare~\cite{hoare69:_axiom_basis_comput_progr}%
    \index{Hoare, Tony (Sir Charles Anthony Richard)}
  propusieron demostrar formalmente
  que los programas cumplen con sus especificaciones,
  idea que fue luego desarrollada por Dijkstra~%
    \cite{dijkstra76:_discip_progr}
  y de alguna manera completada por Gries~%
    \cite{gries87:_scien_progr}.%
    \index{Gries, David}
  Posiblemente la formalización más completa es la que da Apt~%
    \cite{apt81:_ten_years_hoare_logic,
	  apt84:_ten_years_hoare_logic}.
  Una descripción formal reciente
  (con orientación pedagógica)
  dan Gordon y~Collavizza~%
    \cite{gordon10:_forwar_hoare}.
  Jones~%
    \cite{jones03:_early_searc_tract_ways_reason_about_progr}
  resume la historia desde sus comienzos.
  Una crítica del área da Gutmann en su tesis~%
    \cite[capítulos~4 y~5]{gutmann04:_crypt_sec_arch}.

  Demostrar correctitud de programas aumenta la confianza
    \index{programa!verificacion formal@verificación formal}
  de que funcionen correctamente,
  pero las técnicas formales son complejas y caras
  de aplicar en general.
  Aún con ayuda del computador
  hoy solo es posible verificar completamente
  programas relativamente pequeños.
  Tal vez el ejemplo más notable de verificación
  es la del micronúcleo seL4~%
    \cite{Klein:2010:SFV:1743546.1743574, klein11:_l4},%
    \index{seL4}
  un sistema de unas \(8\,700\)~líneas de C
  de las cuales se verificaron formalmente unas \(7\,500\).
  Para contraste,
  el núcleo de un sistema operativo de uso común,
  como Linux,
    \index{Linux}
  tiene casi doce millones de líneas de código.

  Sigue vigente el famoso dicho:
  \hybridblockquote{english}
      [Donald E. Knuth~%
	 \cite{knuth77:_notes_emde_boas_const_prior_deques}]{%
    Beware of bugs in the above code;
    I have only proved it correct,
    not tried it%
  }.%
    \index{Knuth, Donald E.}
  Al efecto,
  Berry~\cite{berry92:_academ_legit_softw_eng_discipl}
  cita el ejemplo de un programita de \(25\)~líneas de Algol%
    \index{Algol (lenguaje de programacion)@Algol (lenguaje de programación)}
  especificado
  y demostrado correcto informalmente en 1969
  por Naur~\cite{naur69:_progr_action_clust},%
    \index{Naur, Peter}
  publicado sin probarlo.
  El año siguiente,
  revisando la publicación de Naur Leavenworth~%
    \cite{leavenworth70:_19420}
  encontró un error que habría sido evidente
  de ejecutar el programa.
  Aún después,
  London~\cite{london71:_softw_reliab_proving_prog_correct}
  halló tres errores adicionales que habrían sido fáciles de hallar
  probando el programa.
  Ofreció un programa corregido,
  acompañando esta vez una demostración formal de correctitud.
  Igualmente,
  Goodenough y Gerhart~%
    \cite{goodenough75:_towar_theor_test_data_selec}
  hallaron tres errores adicionales en el programa de London,
  que nuevamente habrían sido evidentes al probar el programa.
  A su vez,
  las especificaciones dieron lugar
  a una comedia de errores adicional.
    \index{programa!verificacion formal@verificación formal!comedia}
  El hecho que incluso ejemplos mínimos,
  de las manos de los máximos expertos del área,
  den lugar a tal espectáculo
  es prueba patente de lo complejo del tema.

  Siempre debe tenerse presente el dicho:
  \hybridblockquote{english}
      [C. A. R. Hoare~\cite{hoare81:_old_clothes}]{%
    There are two ways of constructing a software design:
    One way is to make it so simple
    that there are \emph{obviously} no deficiencies,
    and the other way is to make it so complicated
    that there are no \emph{obvious} deficiencies.
    The first method is far more difficult.%
  }.%
    \index{Hoare, Tony (Sir Charles Anthony Richard)}
  En la misma línea nos advierten:
  \hybridblockquote{english}
      [D. E. Knuth~%
	 \cite{knuth74:_struc_progr_goto, knuth89:_errors_TeX}]{%
    Premature optimization is the root of all evil.%
  }.%
    \index{Knuth, Donald E.}%
    \index{premature optimization@\emph{\foreignlanguage{english}{premature optimization}}}

\section{Lógica de Hoare}
\label{sec:logica-Hoare}
\index{Hoare, logica de@Hoare, lógica de|textbfhy}

  La idea tras el trabajo de Hoare y sus sucesores
  es que durante su ejecución un programa pasa por \emph{estados}
  bien definidos,
  y que estos estados cambian conforme se ejecutan instrucciones.
  Mediante \emph{predicados}
  sobre los valores de las variables
  se describen los aspectos de interés de cada estado,
  y se busca demostrar que dadas ciertas condiciones iniciales
  al finalizar el programa se cumple lo solicitado
  (\emph{correctitud parcial})%
    \index{programa!correctitud parcial|textbfhy}
  y,
  generalmente en forma separada de lo anterior,
  que el programa siempre termina con el resultado correcto
  (\emph{correctitud total}).%
    \index{programa!correctitud total|textbfhy}
  Esto involucra relacionar los predicados antes y después
  de las distintas instrucciones del lenguaje empleado.

  La forma general de expresar los resultados es:
  \begin{equation*}
    \{P\} \text{\ programa\ } \{Q\}
  \end{equation*}
  para expresar que si se cumple \(P\) antes del programa,
  cuando éste termine se cumple \(Q\).

\section{Búsqueda binaria}
\label{sec:busqueda-binaria}
\index{busqueda binaria@búsqueda binaria}

  Como un ejemplo,
  desarrollaremos paso a paso una función de búsqueda binaria,
  siguiendo la exposición de Bentley~%
    \cite[capítulo 4]{bentley00:_progr_pearl}.%
    \index{Bentley, Jon}
  Es también de interés histórico,
    \index{busqueda binaria@búsqueda binaria!historia}
  la primera exposición del programa fue publicada en 1946,
  la primera versión correcta recién en 1962.
  Bentley reporta haber usado el programa
  como ejercicio en cursos para programadores profesionales.
  Al cabo de una hora,
  revisaban sus programas durante media hora adicional,
  y consistentemente \(90\)\% de ellos hallaban errores.
  Bentley reconoce
  que no está seguro si el \(10\)\% restante era correcto.

  Nuestro problema es determinar si el arreglo ordenado
  \(x[1..N]\) contiene el valor \(t\).
  Más precisamente,
  sabemos que \(N \ge 0\)
  y que \(x[1] \le x[2] \le \dotso \le x[N]\).
  Los tipos de \(t\)
  y de los elementos de \(x\) son los mismos,
  el pseudocódigo debiera funcionar igualmente bien para enteros,
  reales o \foreignlanguage{english}{strings}.
  La respuesta es el entero \(p\);
  si \(p\) es cero
  \(t\) no está en \(x[1..N]\),
  en caso contrario \(1 \le p \le N\)
  y \(t = x[p]\).

  Búsqueda binaria resuelve este problema
  siguiendo la pista a un rango dentro del arreglo
  en el que \(t\) tiene que estar
  si es que está en el arreglo.
  Inicialmente el rango es el arreglo completo.
  El rango se acorta comparando \(t\)
  con el elemento medio del rango
  y descartando la mitad que no puede contenerlo.
  El proceso continúa hasta que se halle el elemento
  o el rango esté vacío.

\subsection{Escribiendo el programa}
\label{sec:busqueda-binaria-escribir}

  La idea clave es que siempre sabemos que si \(t\)
  está en alguna parte de \(x[1..N]\),
  está en un cierto rango.
  Usaremos \(\mathrm{DebeEstar}(\mathit{rango})\)
  para significar que si \(t\) está en el arreglo,
  debe estar en el \(\mathit{rango}\) indicado.
  Con esto,
  la descripción informal
  se puede transformar
  en el esbozo de programa~\ref{alg:bs-sketch-1}.
  \begin{algorithm}[H]
    \DontPrintSemicolon
    Inicialice \(\mathit{rango}\) para designar \(x[1..N]\) \;
    \Loop{
      \{ Invariante: \(\mathrm{DebeEstar}(\mathit{rango})\) \} \;
      \If{\(\mathit{rango}\) es vacío}{
	Retorne que \(t\) no está en el arreglo \;
      }
      Calcule \(m\), el punto medio de \(\mathit{rango}\) \;
      Use \(m\) como prueba para encoger el rango,
	si halla \(t\) en el proceso retorne su posición \;
    }
    \caption{Esbozo de búsqueda binaria}
    \label{alg:bs-sketch-1}
  \end{algorithm}
  Parte crucial de nuestro esbozo es el \emph{invariante de ciclo},
    \index{programa!invariante de ciclo|textbfhy}
  que encerramos entre llaves.
  Se cumple siempre al comienzo de cada ciclo
  (de allí su nombre),
  y formaliza nuestra idea intuitiva presentada antes.

  Refinamos el esbozo,
  asegurándonos que nuestras acciones respetan el invariante.
  Primeramente debemos decidir
  cómo representar el \(\mathit{rango}\).
  Usaremos dos índices \(i\) (inferior)
  y \(s\) (superior) para el rango \(i..s\).
  (Hay otras opciones,
   como inicio y largo.)
  El siguiente paso es la inicialización:
  ¿Qué valores deben tener \(i\) y \(s\) para que
  \(\mathrm{DebeEstar}(i, s)\) se cumpla inicialmente?
  La respuesta obvia es \(i = 1\) y \(s = N\):
  \(\mathrm{DebeEstar}(1, N)\) dice que si \(t\) está en \(x\),
  está en \(x[1..N]\),
  exactamente lo que sabemos al comenzar el programa.
  El siguiente paso es verificar si el rango es vacío,
  cosa que se da siempre que \(i > s\).
  Hallar el punto medio del rango es:
  \begin{equation*}
    m = \left\lfloor \frac{i + s}{2} \right\rfloor
  \end{equation*}
  Uniendo las piezas
  obtenemos el segundo esbozo~\ref{alg:bs-sketch-2}.
  Para evitar confusiones con la relación de igualdad u otras,
  en nuestros algoritmos
  usaremos \(\leftarrow\) para indicar asignación.
  \begin{algorithm}
    \DontPrintSemicolon

    \(i \leftarrow 1\); \(s \leftarrow N\) \;
    \Loop{
      \{ Invariante: \(\mathrm{DebeEstar}(i, s)\) \} \;
      \If{i > s}{
	\(p \leftarrow 0\); \KwBreak \;
      }
      \(m \leftarrow \lfloor (i + s) / 2 \rfloor\) \;
      Use \(m\) como prueba para encoger el rango \(i .. s\),
	si halla \(t\) en el proceso retorne su posición \;
    }
    \caption{Búsqueda binaria: Segundo esbozo}
    \label{alg:bs-sketch-2}
  \end{algorithm}
  Resta refinar la última acción del ciclo.
  Corresponderá a comparar \(t\) con \(x[m]\)
  y tomar la acción apropiada:

  \RestyleAlgo{plain}
  \begin{algorithm}[H]
    \DontPrintSemicolon

    \lCase{\(t < x[m]\):}{Acción a} \;
    \lCase{\(t = x[m]\):}{Acción b} \;
    \lCase{\(t > x[m]\):}{Acción c} \;
  \end{algorithm}
  \RestyleAlgo{algoruled}

  \noindent
  Sabemos que la acción correcta
  en el segundo caso es asignar \(m\) a \(p\)
  y quebrar el ciclo.
  En el primer caso el rango se reduce a \(i .. m - 1\),
  en el tercero a \(m + 1 .. s\).
  Esto se logra asignando a \(s\) o \(i\),
  respectivamente.
  Expresado en términos de las estructuras de control comunes
  resulta~\ref{alg:bs-sketch-3}.
  \begin{algorithm}
    \DontPrintSemicolon
    \(i \leftarrow 1\); \(s \leftarrow N\) \;
    \Loop{
      \{ Invariante: \(\mathrm{DebeEstar}(i, s)\) \} \;
      \If{i > s}{
	\(p \leftarrow 0\); \KwBreak \;
      }
      \(m \leftarrow \lfloor (i + s) / 2 \rfloor\) \;
      \uIf{\(t < x[m]\)}{
	\(s \leftarrow m - 1\) \;
       }
       \eIf{\(t = x[m]\)}{
	 \(p \leftarrow m\); \KwBreak \;
       }
       {
	 \(i \leftarrow m + 1\) \;
       }
    }
    \caption{Búsqueda binaria: Pseudocódigo final}
    \label{alg:bs-sketch-3}
  \end{algorithm}
  En rigor,
  solo hemos demostrado
  (muy informalmente,
   claro está)
  que si el ciclo termina entrega el resultado correcto.
  Debemos asegurar además que el ciclo siempre termina,
  vale decir,
  que el largo del rango disminuye en cada paso.
  Como el largo es un entero,
  no puede disminuir indefinidamente.

  Igualmente,
  en implementaciones en máquinas reales
  el algoritmo~\ref{alg:bs-sketch-3}
  tiene un error fatal:
  Si la suma de \(i + s\)
  es mayor que el máximo entero representable,
  el cálculo del nuevo valor de \(m\) falla.
  Esto puede sonar a sofisma,
  pero fue un error real en la biblioteca de Java.
  La solución correcta es escribir esa línea como
  \(m \leftarrow i + \lfloor (s - i) / 2 \rfloor\).
  De esa forma,
  las variables y valores intermedios están acotados por \(n\),
  y no hay rebalses.

  Sea \(l\) el largo del rango al comenzar un ciclo.
  Llamemos \(l'\) al largo después del ciclo,
  suponiendo \(t \ne x[m]\).
  Entonces:
  \begin{equation*}
    l'
      = \begin{cases}
	  (m - 1) - i + 1
	    = (\lfloor (i + s) / 2 \rfloor - 1) - i + 1
	    \le (s - i) / 2
	    = (l + 1) / 2 & \text{si \(t < x[m]\)} \\
	  s - (m + 1) - 1
	    = s - (\lfloor (i + s) / 2 \rfloor + 1) - 1
	    \le s - (i + s) / 2 - 1
	    = l / 2 - 3 / 2
	    < l & \text{si \(t > x[m]\)}
	\end{cases}
  \end{equation*}
  En el caso \(t < x[m]\),
  si \(l > 1\) claramente \(l' \le (l + 1) / 2 < l\),
  pero para \(l = 1\) resulta la cota inútil \(l' \le 1\).
  En esta última situación
  es \(i = s\),
  y el algoritmo asigna \(s = m - 1 = i - 1\),
  cumpliendo la condición de término.
  Cada iteración
  nos acerca a la condición que hace terminar el ciclo,
  el programa termina.

  Es sencillo ahora traducir el pseudocódigo~\ref{alg:bs-sketch-3}
  a un lenguaje de programación como C,%
    \index{C (lenguaje de programacion)@C (lenguaje de programación)}
  ver el listado~\ref{lst:busqueda-binaria}.
  Nuestra versión incluye la precaución contra rebalses aritméticos
  mencionada antes.
  \lstinputlisting[language=C,
		   caption={Búsqueda binaria en C},
		   label=lst:busqueda-binaria]
		   {code/binary-search.c}
  Nuestro desarrollo cuidadoso
  da gran confianza de que el programa resultante sea correcto.
  El registrar el invariante del ciclo ayudará a nuestros lectores
  convencerse de ello.

  Para no caer en un programa demostrado correcto
  que falla con el primer caso de prueba
    \index{programa!casos de prueba}
  (como se narra en la introducción del presente capítulo),
  usamos el computador para lo que es mejor:
  Tareas rutinarias.
  Un pequeño programa bombardea esta función con casos de prueba:
  Se crea un arreglo de \(22\) elementos
  que se llena con los valores \(0\) a \(21\),
  luego se buscan los valores de \(1\) a \(N\) en él
  dando el rango de \(1\) a \(N\) para la búsqueda,
  con \(N\) variando de \(1\) a \(20\).
  Después se buscan los valores \(0\) y \(N + 1\)
  (ambos presentes,
   pero fuera del rango indicado),
  y finalmente los valores de \(0,5\) a \(N + 0,5\)
  con paso \(1\).
  Con esto cubrimos valores populares para errores:
  \(1\), \(2\), \(3\),
  algunas potencias de \(2\),
  números que difieren de potencias de \(2\) en \(1\).
  Y se cumplió lo indicado
  respecto de que los errores se cometen
  principalmente en las partes simples
  del programa:
  Algunos de los casos de prueba iniciales fallaron,
  estaban escritos incorrectamente\ldots

\section{Exponenciación}
\label{sec:programa-exponenciar}
\index{programa!exponenciar}

  El calcular una potencia vía multiplicaciones sucesivas
  viene directamente de la definición de potencia.
  Sin embargo,
  hay métodos más eficientes.
  Partiendo de la identidad:
  \begin{equation*}
    n = d \cdot \left\lfloor \frac{n}{d} \right\rfloor + n \bmod d
  \end{equation*}
  donde \(n \bmod d\) es el resto de la división de \(n\) por \(d\),
  podemos expresar potencias:
  \begin{equation}
    \label{eq:exponenciacion-binaria}
    x^n
      = \left( x^{\lfloor n / 2 \rfloor} \right)^2
	  \cdot x^{n \bmod 2}
  \end{equation}
  Tenemos inmediatamente el algoritmo recursivo~%
    \ref{alg:exponenciacion-binaria-recursivo}.
  \begin{algorithm}
    \DontPrintSemicolon
    \SetKwFunction{Power}{power}

    \KwFunction \Power\FuncSty{(}\ArgSty{\(x, \; n\)}\FuncSty{)} \;
    \BlankLine
    \uIf{\(n = 0\)}{\Return{\(1\)}}
    \Else{
      \(s \leftarrow \Power(x, \; \lfloor n / 2 \rfloor)\) \;
      \(r \leftarrow s^2\) \;
      \If{\(n \bmod 2 = 1\)}{\(r \leftarrow r \cdot x\)}
      \Return{\(r\)} \;
    }
    \caption{Exponenciación binaria recursiva}
    \label{alg:exponenciacion-binaria-recursivo}
  \end{algorithm}
  La justificación
  es que suponiendo que la función \(\mathtt{power}(x, k)\)
  hace correctamente su trabajo para \(k < n\),
  por la identidad~\eqref{eq:exponenciacion-binaria}
  calcula correctamente la potencia \(n\).
  Para justificar que termina,
  vemos que cada vez se invoca la función recursivamente
  con un valor menor para el parámetro \(n\),
  por lo que tarde o temprano llega al valor \(0\),
  que no involucra recursión.
  Esto es en esencia una demostración por inducción fuerte.

  Una versión alternativa
  (no recursiva)
  del algoritmo~\ref{alg:exponenciacion-binaria-recursivo}
  es el algoritmo~\ref{alg:exponenciacion-binaria}.
  Usamos operaciones con bits
  al estilo C~\cite{kernighan88:_c_progr_lang}.
  \begin{algorithm}
    \DontPrintSemicolon
    \SetKwFunction{Power}{power}

    \KwFunction \Power\FuncSty{(}\ArgSty{\(x, \; n\)}\FuncSty{)} \;
    \BlankLine
    \If{\(n = 0\)}{
      \Return{\(1\)} \;
    }
    \BlankLine
    \For{\(i \leftarrow 0\);
	 \(n \mathbin{\&} (1 \ll i)\);
	 \(i \leftarrow i + 1\)}{
      \tcc{Nada}
    }
    \tcc{Ahora \((i \ge 0) \wedge (2^i \le n < 2^{i + 1})\)} \;
    \(r \leftarrow x\) \;
    \For{; \(i > 0\); \(i \leftarrow i - 1\)}{
      \tcc{Invariante: \(u = \lfloor n / 2^i \rfloor\),
		       \(v = n \bmod 2^i\),
	    \((r = x^u) \wedge (x^n = r^{2^i} \cdot x^v)\)}
      \(r \leftarrow r^2\) \;
      \If{\(n \mathbin{\&} (1 \ll (i - 1))\)}{
	\(r \leftarrow r \cdot x\) \;
      }
    }
    \tcc{Del invariante con \(i = 0\) resulta \(r = x^n\)}
    \Return{\(r\)} \;
    \caption{Exponenciación binaria no recursiva}
    \label{alg:exponenciacion-binaria}
  \end{algorithm}

  Un ejemplo adicional es el algoritmo~\ref{alg:sqrt-integer}%
    \index{programa!raiz cuadrada entera@raíz cuadrada entera}
  para obtener la raíz cuadrada entera,
  vale decir,
  calcular \(\lfloor \sqrt{n} \rfloor\).
  Todas las variables del algoritmo son enteras.
  Es simple verificar el invariante,
  y con la condición de término del ciclo
  tenemos que \(u^2 \le n < (u + 1)^2\),
  vale decir,
  \(u \le \sqrt{n} < u + 1\),
  o sea \(u = \lfloor \sqrt{n} \rfloor\).
  En cada ciclo aumenta \(u\) o disminuye \(v\),
  por lo que el algoritmo termina.

  \begin{algorithm}
    \DontPrintSemicolon
    \SetKwFunction{ISqrt}{isqrt}

    \KwFunction \ISqrt \FuncSty{(}\ArgSty{\(n\)}\FuncSty{)} \;
    \BlankLine
    \(u \leftarrow 0\) \;
    \(v \leftarrow n + 1\) \;
    \While{ \(u + 1 \ne v\)}{
      \tcc{Invariante: \(u^2 \le n \le v^2\) y \(u + 1 \le v\)}
      \(x \leftarrow \lfloor (u + v) / 2 \rfloor\) \;
      \uIf{ \(x^2 \le n\) }{
	\(u \leftarrow x\) \;
      }
      \Else{
	\(v \leftarrow x\) \;
      }
    }
    \Return{\(u\)} \;
    \caption{Cálculo de $\lfloor \sqrt{n} \rfloor$}
    \label{alg:sqrt-integer}
  \end{algorithm}

\section{Algunos principios}
\label{sec:algunos-principios}

  El ejemplo de búsqueda binaria
  ilustra las fortalezas de la verificación de programas:
  El problema a resolver es importante y requiere código cuidadoso,
  lo desarrollamos guiados por ideas de verificación,
  y el análisis de correctitud usa herramientas generales.
  En todo caso,
  en la práctica el nivel de detalle del desarrollo
  será substancialmente menor.
  Algunos principios generales resultan de la discusión:
  \begin{description}
  \item[Afirmaciones:]\index{programa!afirmaciones}
    Permiten expresar las relaciones entre datos de entrada,
    variables internas y resultados en forma precisa.
  \item[Estructuras de control secuenciales:]%
      \index{programa!estructuras de control}
    La estructura de control más simple es ejecutar una acción,
    seguida por otra.
    Entendemos un programa por afirmaciones entre las instrucciones
    y razonamos de una a la siguiente.
  \item[Estructuras de selección:]
    Durante la ejecución,
    se toma una de varias acciones.
    Demostramos la correctitud de tales estructuras
    razonando de la afirmación antes de la estructura
    junto a la condición especial que nos lleva al presente caso.
    Lo que podemos concluir luego de cada una de las opciones
    se cumple al finalizar la estructura completa.
  \item[Estructuras de iteración:]
    Argüir la correctitud de un ciclo tiene tres fases:
    Inicialización,
    preservación
    y término.
    Primero demostramos que el invariante del ciclo
    se establece antes del ciclo
    (inicialización),
    luego que si el invariante se cumple al comienzo del ciclo
    se cumple cuando el ciclo termina,
    y finalmente demostramos que el ciclo
    se ejecuta un número finito de veces.
    Después del ciclo sabemos que se cumple el invariante
    y la condición de término se cumplió.
  \item[Variables:]\index{programa!variables}
    Muchos programas modifican los valores de variables,
    pero interesa expresar
    que cierta relación se cumple respecto de los valores iniciales.
    Una posibilidad
    es usar la convención que si la variable es \(n\),
    su valor original se representa mediante \(n_0\).
    Otra opción es definir variables fantasmas
    que recogen valores de interés.
    Usamos variables fantasmas
    en el algoritmo~\ref{alg:exponenciacion-binaria}
    para expresar el invariante en forma más sencilla.
  \item[Subrutinas:]\index{programa!subrutinas}
    Para verificar una subrutina,
    explicitamos su propósito mediante dos afirmaciones:
    Su precondición describe el estado antes de ejecutarla,
    la postcondición describe lo que garantiza del estado
    una vez que finaliza.
    Externamente
    usamos estas afirmaciones para razonar sobre sus usos,
    internamente verificamos que dadas las precondiciones
    estamos garantizando las postcondiciones.
    Esto es válido también en el caso de rutinas recursivas:
    Suponemos que las llamadas recursivas
    ``hacen lo correcto'',
    y en base a ello demostramos que la rutina cumple su contrato.
  \item[Recursión:]\index{programa!recursión}
    Verificar la versión recursiva de la exponenciación binaria
    resulta substancialmente más sencillo
    que verificar la versión no recursiva.
    Esta observación es aplicable en general.
  \end{description}
  Hallar afirmaciones simples
  (particularmente invariantes de ciclos)%
    \index{programa!invariante de ciclo}
  no es nada fácil.
  Esta manera de enfrentar el problema de escribir un programa
  formaliza la manera en que entendemos un programa
  (muchas veces la explicación del funcionamiento
   es a través de una traza de la ejecución,
     \index{programa!traza}
   pero está claro
   que el detalle de los valores que toman las variables
   en esa instancia particular
   no es todo lo que se está transmitiendo).
  Por esta razón conviene familiarizarse con este enfoque.

  Las partes difíciles de un programa
  hacen que se recurra a métodos más formales,
  mientras las partes simples
  se desarrollan de la forma tradicional.
  La experiencia común es que las partes difíciles
  luego funcionan correctamente la primera vez,
  las fallas están en las partes simples.

  No se ha logrado el objetivo inicial de estos esfuerzos:
  Contar con algún artilugio que tome un programa
  e indique ``correcto'' o ``incorrecto''.
  Igualmente logramos algo muy valioso:
  Una comprensión mejor del proceso de programación.
  Lenguajes de programación actuales se definen al menos en parte
  cuidando que sea fácil razonar con sus operaciones.
  Técnicas similares a las usadas acá para demostrar correctitud
  emplean los compiladores para ``optimizar'' código
    \index{programa!optimizacion de codigo@optimización de código}
  (en realidad, solo intentan mejorar características de ejecución):
  Dependiendo de restricciones deducidas sobre los valores,
  el código puede especializarse o incluso omitirse
  sin cambiar los resultados.
  Refinamientos del análisis son técnicas de ejecución simbólica,%
    \index{programa!ejecucion simbolica@ejecución simbólica}
  que pueden hallar errores
  o al menos generar automáticamente casos de prueba%
    \index{programa!casos de prueba}
  para la mayoría del código,
  un ejemplo es KLEE~%
    \cite{cadar08:_klee}.%
    \index{KLEE}
  También hay herramientas que ayudan a construir y verificar
  demostraciones de correctitud,
  como Frama-C~%
    \cite{cuoq12:_frama_c_softw_analy_persp}.%
    \index{Frama-C}

%%% Local Variables:
%%% mode: latex
%%% TeX-master: "clases"
%%% End:


% reales.tex
%
% Copyright (c) 2009-2014 Horst H. von Brand
% Derechos reservados. Vea COPYRIGHT para detalles

\chapter{Números reales}
\label{cha:numeros-reales}
\index{numero@número!real|textbfhy}
\index{R (numeros reales)@\(\mathbb{R}\) (números reales)}

  Los números reales son fundamentales en mucha de la matemática
  que usamos diariamente.
  No nos detendremos en un estudio detallado de ellos,
  los exploraremos como ejemplo de campo,
  estructura algebraica que trataremos en algún detalle más adelante
  (mostrando el funcionamiento del método axiomático
   y algunas técnicas de demostración).

\section{Axiomas de los reales}
\label{sec:axiomas-reales}
\index{R (numeros reales)@\(\mathbb{R}\) (números reales)!axioma}

  Daremos una breve introducción a los números reales y sus propiedades,
  siguiendo en lo general a Chen~%
    \cite[capítulo 1]{chen08:_first_year_calculus}.
  Anotamos \(\mathbb{R}\) para el conjunto de números reales,%
    \index{R (números reales)@\(\mathbb{R}\) (números reales)}
  con operaciones \(+\) y \(\cdot\).
  Los números reales con sus operaciones cumplen los siguientes axiomas,%
    \index{axioma!numeros reales@números reales|textbfhy}%
    \index{numero@número!real!axiomas|see{axioma!números reales}}
  que simplemente daremos por hechos.
  Estos axiomas describen lo que se conoce como un \emph{campo}%
    \index{campo (algebra)@campo (álgebra)}
  (en inglés \emph{\foreignlanguage{english}{field}}).%
    \index{field@\emph{\foreignlanguage{english}{field}}|see{campo (álgebra)}}
  En esta lista \(a\), \(b\), \(c\) son reales cualquiera.
  \begin{enumerate}[label=\textbf{R\arabic{*}:}, ref=R\arabic{*}]
  \item\label{Re:suma-asociativa}%
    \index{operacion@operación!asociativa}
    La suma es asociativa:
    \((a + b) + c = a + (b + c)\).
  \item\label{Re:cero}%
    \index{operacion@operación!elemento neutro}
    Hay un \emph{elemento neutro para la suma} \(0 \in \mathbb{R}\)
    tal que \(a + 0 = a\)
  \item\label{Re:inverso-aditivo}%
    \index{operacion@operación!inverso}
    Hay un elemento \(-a \in \mathbb{R}\)
    tal que \(a + (-a) = 0\).
    A \(-a\) se le llama \emph{inverso aditivo} de \(a\).
  \item\label{Re:suma-conmutativa}%
    \index{operacion@operación!conmutativa}
    La suma es conmutativa:
    \(a + b = b + a\).
  \item\label{Re:multiplicacion-asociativa}%
    \index{operacion@operación!asociativa}
    La multiplicación es asociativa:
    \((a \cdot b) \cdot c = a \cdot (b \cdot c)\)
  \item\label{Re:distributiva}%
    \index{operacion@operación!distributiva}
    La multiplicación distribuye sobre la suma:
    \((a + b) \cdot c = (a \cdot c) + (b \cdot c)\).
  \item\label{Re:uno}
    Hay un elemento \emph{neutro para la multiplicación} \(1 \in \mathbb{R}\)
    tal que \(a \cdot 1 = a\).
  \item\label{Re:multiplicacion-conmutativa}
    La multiplicación es conmutativa:
    \(a \cdot b = b \cdot a\).
  \item\label{Re:inverso-multiplicativo}
    Para \(a \ne 0\) hay un elemento \(a^{-1} \in \mathbb{R}\)
    tal que \(a \cdot a^{-1} = 1\).
    A \(a^{-1}\) se le llama \emph{inverso multiplicativo} de \(a\).
  \end{enumerate}

  Entre los reales tenemos además un orden,%
    \index{relacion@relación!orden}
  una relación \(<\) que cumple los siguientes axiomas adicionales,
  donde \(a\), \(b\) y \(c\) nuevamente denotan números reales cualquiera.
  Anotamos \(a > b\) si \(b < a\),
  como es convencional.
  \begin{enumerate}[label=\textbf{O\arabic{*}:}, ref=O\arabic{*}]
  \item\label{Re:tricotomia}%
    \index{relacion@relación!orden!tricotomia@tricotomía}
    Se cumple exactamente uno de \(a < b\), \(a = b\) o \(a > b\).
    Esta propiedad se conoce como \emph{tricotomía}.
  \item\label{Re:transitidad<}%
    \index{relacion@relación!transitiva}
    Si \(a < b\) y \(b < c\) entonces \(a < c\).
  \item\label{Re:>+}
    Si \(a < b\),
    entonces \(a + c < b + c\).
  \item\label{Re:>*}
    Si \(a < b\) y \(c > 0\)
    entonces \(a \cdot c < b \cdot c\).
  \end{enumerate}
  En términos de las propiedades de relaciones,
  diríamos que \(<\) es transitiva e irreflexiva,%
    \index{relacion@relación!irreflexiva}
  y que la relación \(\le\) en \(\mathbb{R}\),
  definida mediante \mbox{\(a \le b \equiv (a < b) \vee (a = b)\)},
  es un orden total.%
    \index{relacion@relación!orden total}

  Un subconjunto de los reales
  es el conjunto de los números naturales,%
    \index{N (números naturales)@\(\mathbb{N}\) (números naturales)}
  \(\mathbb{N} = \{1, 2, 3, \dotsc\}\).
  Los siguientes axiomas dan sus principales características.%
    \index{axioma!numeros naturales@números naturales|textbfhy}
  Primeramente,
  \(\mathbb{N} \subseteq \mathbb{R}\) con las mismas operaciones
  y relación de orden.
  Enseguida:
  \begin{enumerate}[label=\textbf{N\arabic{*}:}, ref=N\arabic{*}]
  \item\label{N:uno}
    \(1 \in \mathbb{N}\)
  \item\label{N:sucesor}
    Si \(n \in \mathbb{N}\),
    entonces \(n + 1 \in \mathbb{N}\).
    A \(n + 1\) se le llama el \emph{sucesor} de \(n\).
  \item\label{N:solo-sucesores}
    Todo \(n \in \mathbb{N}\) tal que \(n \ne 1\)
    es el sucesor de un único número natural.
  \item\label{N:buen-orden}
    Todo subconjunto no vacío de \(\mathbb{N}\) contiene un elemento mínimo.
    Esto se conoce como \emph{principio de buen orden}.%
      \index{buen orden, principio de}
  \end{enumerate}
  Puede demostrarse que el principio de buen orden
  es equivalente al \emph{principio de inducción},
    \index{induccion@inducción!principio de}
  acá demostraremos este último partiendo de buen orden:
  \begin{theorem}[Principio de inducción]
    \label{theo:principio-induccion}
    Sea \(p(\cdot)\) un predicado que cumple:
    \begin{enumerate}[label=(\roman{*})]
    \item
      \(p(1)\) es verdadero
    \item
      \(p(n) \implies p(n + 1)\)
    \end{enumerate}
    Entonces \(p(n)\) es verdadero para todo \(n \in \mathbb{N}\).
  \end{theorem}
  \begin{proof}
    Por contradicción.
    Supongamos un conjunto no vacío
    \(\mathcal{C} \subseteq \mathbb{N}\) para el que \(p(\cdot)\) no vale.
    Por el principio del buen orden,
    \(\mathcal{C}\) contiene su elemento mínimo,
    llamémosle \(m\).
    Sabemos que \(p(1)\) es cierto,
    así que \(m > 1\) y es el sucesor de un natural \(n\).
    Como \(m\) es el mínimo para el que no vale \(p(\cdot)\),
    \(p(n)\) es cierto,
    pero entonces es cierto \(p(n + 1) = p(m)\),
    contradiciendo la elección de \(m\).
  \end{proof}

  El conjunto \(\mathbb{Z}\) de los enteros
  es la extensión de \(\mathbb{N}\)
  para incluir a \(0\)
  y los números de la forma \(-n\) para \(n \in \mathbb{N}\).
  El conjunto \(\mathbb{Q}\) de los racionales
  es el conjunto de números de la forma \(a \cdot b^{-1}\),
  con \(a \in \mathbb{Z}\) y \(b \in \mathbb{N}\).

  Los axiomas de campo y de orden valen para \(\mathbb{Q}\).%
    \index{Q (números racionales)@\(\mathbb{Q}\) (números racionales)}
  Pero vimos que \(\mathbb{Q}\) es incompleto
  (por el teorema~\ref{theo:sqrt2-irracional},
   \(\sqrt{2} \notin \mathbb{Q}\)).
  Veremos una propiedad que distingue a \(\mathbb{R}\) de \(\mathbb{Q}\).
  Se le conoce como \emph{axioma de completitud},%
    \index{completitud, axioma de}
  que nosotros definiremos en términos de cotas.
  \begin{definition}
    \index{numero@número!irracional|textbfhy}
    A un número \(x \in \mathbb{R} \smallsetminus \mathbb{Q}\)
    se le llama \emph{irracional}.%
      \index{numero@número!irracional|textbfhy}
  \end{definition}

  \begin{definition}
    Un conjunto no vacío \(\mathcal{S}\) de números reales
    se dice \emph{acotado por arriba} si hay \(C \in \mathbb{R}\)
    tal que \(x \le C\) para todo \(x \in \mathcal{S}\).
    A \(C\) se le llama \emph{cota superior} de \(\mathcal{S}\).%
      \index{cota!superior}
    Si hay \(c \in \mathbb{R}\)
    tal que \(x \ge c\) para todo \(x \in \mathcal{S}\),
    se dice \emph{acotado por abajo}
    y a \(c\) se le llama \emph{cota inferior} de \(\mathcal{S}\).%
      \index{cota!inferior}
    Si \(\mathcal{S}\) es acotado por abajo y por arriba,
    de dice \emph{acotado}.%
      \index{acotado}%
      \index{cota}
  \end{definition}
  Por ejemplo,
  \(\mathbb{N}\) es acotado por abajo pero no por arriba
  (cosa que demostraremos en el teorema~\ref{theo:arquimedeana}),
  el conjunto \(\mathbb{Q}\) no tiene cota inferior ni superior,
  mientras \(\{x \in \mathbb{R} \colon 1 < x \le 3\}\) es acotado.

  \begin{axiom}[Supremo]
    \label{Re:supremo}
    \index{supremo, axioma de|textbfhy}
    Sea \(\mathcal{S} \subseteq \mathbb{R}\) no vacío,
    acotado por arriba.
    Entonces existe \(M \in \mathbb{R}\) tal que
    \begin{enumerate}[label=(\alph{*})]
    \item
      \(M\) es cota superior de \(\mathcal{S}\).
    \item
      Para todo \(\epsilon > 0\) hay \(s \in \mathcal{S}\)
      tal que \(s > M - \epsilon\).
    \end{enumerate}
  \end{axiom}
  Esto dice que no pueden haber cotas superiores menores que \(M\),
  y asegura que \(M\) es un número real.
  \begin{definition}
    A \(M\) se le llama el \emph{supremo}
    (o mínima cota superior)
    de \(\mathcal{S}\),
    se anota \(M = \sup \mathcal{S}\).
  \end{definition}
  El axioma del supremo puede expresarse en la forma obviamente equivalente:
  \begin{axiom}[Ínfimo]
    \label{Re:infimo}
    \index{infimo, axioma de@ínfimo, axioma de|textbfhy}
    Sea \(\mathcal{S} \subseteq \mathbb{R}\) no vacío,
    acotado por abajo.
    Entonces existe \(m \in \mathbb{R}\) tal que
    \begin{enumerate}[label=(\alph{*})]
    \item
      \(m\) es cota inferior de \(\mathcal{S}\).
    \item
      Para todo \(\epsilon > 0\) hay \(s \in \mathcal{S}\)
      tal que \(s > m + \epsilon\).
    \end{enumerate}
  \end{axiom}
  \begin{definition}
    A este \(m\) se le llama \emph{ínfimo}
    (máxima cota inferior)
    de \(\mathcal{S}\),
    se anota \mbox{\(m = \inf \mathcal{S}\)}.
  \end{definition}

  Como un ejemplo,
  demostraremos que \(\sqrt{2}\) es real,%
    \index{numero@número!irracional!\(\sqrt{2}\)}
  y por tanto irracional.
  \begin{theorem}
    \label{theo:sqrt(2)-real}
    Hay un real positivo \(r\) que cumple \(r^2 = 2\).
  \end{theorem}
  \begin{proof}
    Sea \(\mathcal{S} = \{x \in \mathbb{R} \colon x^2 < 2\}\).
    Entonces \(\mathcal{S}\) no es vacío,
    ya que \(1^2 < 2\);
    y tiene a \(2\) como cota superior,
    ya que si \(x > 2\) entonces \(x^2 > 4 > 2\).
    Por el axioma del supremo,
    hay \(r \in \mathbb{R}\) tal que \(r = \sup \mathcal{S}\).
    Claramente \(r > 0\),
    ya que \(1 \in \mathcal{S}\).
    Demostraremos por contradicción que \(r^2 = 2\).
    Supongamos \(r^2 \ne 2\).
    Por el axioma~\ref{Re:tricotomia},
    debe ser entonces \(r^2 < 2\) o \(r^2 > 2\).
    Para demostrar que estas no se cumplen,
    basta exhibir un \(\epsilon > 0\) para cada caso
    para el cual falla.
    Por turno:
    \begin{description}
    \item[\boldmath\(r^2 < 2\):\unboldmath]
      Sea \(\epsilon > 0\)
      y consideremos:
      \begin{equation}
	\label{eq:r+eps}
	(r + \epsilon)^2
	  = r^2 + 2 r \epsilon + \epsilon^2
	  < r^2 + (2 r + 1) \epsilon
      \end{equation}
      Si ahora \(\epsilon < (2 - r^2) / (2 r + 1)\),
      la última expresión en~\eqref{eq:r+eps} es menor a \(2\),
      lo que contradice la elección de \(r\) como supremo.
    \item[\boldmath\(r^2 > 2\):\unboldmath]
      Para \(\epsilon > 0\)
      calculamos:
      \begin{equation}
	\label{eq:r-eps}
	(r - \epsilon)^2
	  = r^2 - 2 r \epsilon + \epsilon^2
	  > r^2 - 2 r \epsilon
      \end{equation}
      Si elegimos \(\epsilon < (r^2 - 2) / (2 r)\),
      la última expresión de~\eqref{eq:r-eps} es mayor a \(2\),
      nuevamente contradiciendo la elección de \(r\) como supremo.
    \end{description}
    En consecuencia,
    debe ser \(r^2 = 2\).
  \end{proof}

  Algunas consecuencias de la completitud de los reales son las siguientes.
  \begin{theorem}[Propiedad arquimedeana]
    \index{propiedad arquimedeana}
    \label{theo:arquimedeana}
    Para todo \(x \in \mathbb{R}\)
    hay \(n \in \mathbb{N}\)
    tal que \(n > x\).
  \end{theorem}
  \begin{proof}
    La demostración es por contradicción.
    Supongamos que \(x \in \mathbb{R}\),
    y que para todo \(n \in \mathbb{N}\) es \(n \le x\).
    Entonces \(x\) es una cota superior para \(\mathbb{N}\),
    y el conjunto \(\mathbb{N}\) tiene un supremo por completitud,
    sea \(M = \sup \mathbb{N}\).
    Así \(M \ge n\) para \(n \in \mathbb{N}\),
    en particular es \(M \ge n\) para \(n = 2, 3, 4, \dotsc\).
    Pero cada número natural
    (salvo \(1\))
    es el sucesor de un número natural,
    con lo que \(M \ge k + 1\) para \(k = 1, 2, 3, \dotsc\),
    o \(M - 1 \ge k\) para todo \(k \in \mathbb{N}\).
    Pero habíamos elegido \(M\) como el supremo de \(\mathbb{N}\),
    no puede tener cotas superiores menores.
  \end{proof}
  Esta demostración completa nuestra aseveración anterior que \(\mathbb{N}\)
  no tiene cota superior.

  \begin{theorem}
    \label{theo:Q-dense}
    Los racionales e irracionales son densos en \(\mathbb{R}\),
      \index{conjunto!denso|textbfhy}
    o sea entre cada par de reales distintos hay un racional y un irracional.
  \end{theorem}
  \begin{proof}
    Supongamos \(x, y \in \mathbb{R}\),
    con \(x < y\).
    Primero hay \(r \in \mathbb{Q}\) tal que \(x < r < y\).
    Supongamos \(x > 0\) por ahora.
    Por la propiedad arquimedeana,%
      \index{propiedad arquimedeana}
    teorema~\ref{theo:arquimedeana},
    existe \(b \in \mathbb{N}\) tal que \(b > (y - x)^{-1}\),
    de manera que \(b (y - x) > 1\).
    Por la propiedad arquimedeana existe \(n \in \mathbb{N}\)
    tal que \(n > b x\),
    con lo que el conjunto
      \(\mathcal{S} = \{n \in \mathbb{N} \colon n > b x\}\)
    no es vacío,
    y por el axioma~\ref{N:buen-orden} contiene su mínimo,%
      \index{buen orden, principio de}
    llamémosle \(a\).
    Entonces \(a - 1 \le b x\):
    Si fuera \(a = 1\),
    \(a - 1 = 0 < b x\);
    en caso contrario \(a - 1 > b x\)
    contradice la elección de \(a\) como mínimo.
    Se sigue:
    \begin{equation*}
      b x
	< a
	= (a - 1) + 1
	< b x + b (y - x)
	= b y
    \end{equation*}
    de forma que:
    \begin{equation*}
      x < a \cdot b^{-1} < y
    \end{equation*}
    Veamos ahora el caso \(x \le 0\).
    Por la propiedad arquimedeana,
    existe \(k \in \mathbb{N}\) tal que \(k > -x\),
    de manera que \(k + x > 0\).
    Por lo anterior,
    hay \(s \in \mathbb{Q}\) tal que \(x + k < s < y + k\),
    y \(x < s - k < y\)
    donde \(s - k \in \mathbb{Q}\).

    Para hallar un irracional entre \(x\) e \(y\),
    vemos por lo anterior que hay \(r_1 \in \mathbb{Q}\)
    tal que \(x < r_1 < y\);
    de la misma forma hay \(r_2 \in \mathbb{Q}\)
    tal que \(r_1 < r_2 < y\).
    Como \(1 < \sqrt{2} < 2\),
    es:
    \begin{equation*}
      x < r_1 < r_1 + (r_2 - r_1) / \sqrt{2} < r_2 < y
    \end{equation*}
    y \(r_1 + (r_2 - r_1) / \sqrt{2}\) claramente es irracional.
  \end{proof}

%%% Local Variables:
%%% mode: latex
%%% TeX-master: "clases"
%%% End:


% numerabilidad.tex
%
% Copyright (c) 2011-2014 Horst H. von Brand
% Derechos reservados. Vea COPYRIGHT para detalles

\chapter{Numerabilidad}
\label{cha:numerabilidad}
\index{numerabilidad|textbfhy}

  La manera más básica de contar es construir una biyección%
    \index{biyeccion@biyección}
  entre dos conjuntos,
  que de esa forma tienen la misma cardinalidad
  (``número de elementos'').
  Por ejemplo,
  para determinar si en una sala hay tantos asistentes como sillas
  basta solicitar que todos se sienten.
  Si no sobran sillas vacías ni quedan personas de pie,
  hay tantas personas como sillas.

  Nuestro interés está en la existencia
  de diferentes infinitos,
  en particular la demostración de Cantor%
    \index{Cantor, Georg}
  de que hay más números reales que enteros.
  No haremos uso de esto en el texto presente,
  pero los conceptos y las técnicas de demostración usadas
  muestran ser centrales en el estudio de la computabilidad
  (los límites de lo que un algoritmo puede hacer).

\section{Cardinalidad}
\label{sec:cardinalidad}
\index{cardinalidad}

  Ya indicamos que la manera fundamental de asignar un ``tamaño'' a un conjunto
  es hallar una correspondencia con un conjunto prototipo.
  Si los conjuntos son finitos,
  esta operación es conocida;
  buscamos extender la definición de lo que significa a conjuntos infinitos
  de forma de poder razonar sobre ellos.
  \begin{definition}
    La \emph{cardinalidad} del conjunto
    \(\{1, 2, \dotsc, n\}\) con \(n \in \mathbb{N}\)
    es \(n\).
    La cardinalidad de \(\varnothing\) es \(0\).
    Un conjunto cuya cardinalidad es \(n \in \mathbb{N}_0\)
    se dice \emph{finito}.%
      \index{conjunto!finito}
  \end{definition}
  A partir de aquí podemos definir igualdad de cardinalidades
  mediante biyecciones,
  incluso para conjuntos infinitos.%
      \index{conjunto!infinito}
  Anotaremos \(\lvert \mathcal{A} \rvert\)
  para la cardinalidad del conjunto \(\mathcal{A}\).
  Formalmente:
  \begin{definition}
    \index{conjunto!cardinalidad|textbfhy}
    Dos conjuntos \(\mathcal{A}\) y \(\mathcal{B}\)
    tienen la misma cardinalidad,
    lo que se anota
      \(\lvert \mathcal{A} \rvert = \lvert \mathcal{B} \rvert\),
    si hay una biyección
      \(\phi: \mathcal{A} \rightarrow \mathcal{B}\).
    Decimos que
      \(\lvert \mathcal{A} \rvert \le \lvert \mathcal{B} \rvert\)
    si hay una inyección
      \(\gamma: \mathcal{A} \rightarrow \mathcal{B}\).
    Si \(\lvert \mathcal{A} \rvert \le \lvert \mathcal{B} \rvert\)
    pero no existe biyección
    entre \(\mathcal{A}\) y \(\mathcal{B}\),
    decimos
      \(\lvert \mathcal{A} \rvert < \lvert \mathcal{B} \rvert\).
  \end{definition}
  Las notaciones indicadas son sugerentes.
  Para justificarlas debemos demostrar que
  la igualdad de cardinalidades
  es una relación de equivalencia.%
    \index{relacion@relación!equivalencia}
  El que la función identidad es una biyección provee reflexividad,
  como la función inversa de una biyección
  es una biyección da simetría
  y ya que la composición de biyecciones
  es una biyección es transitiva.
  Además debemos verificar
  que \(\lvert \mathcal{A} \rvert \le \lvert \mathcal{B} \rvert\)
  es una relación de orden%
    \index{relacion@relación!orden}
  (es transitiva,
   reflexiva y simétrica
   bajo el entendido de la igualdad de cardinalidades).
  La reflexividad es obvia,
  la función identidad es una inyección.
  La transitividad es simple,
  ya que la composición de inyecciones es una inyección.
  Demostrar simetría es más complejo,
  es el tema de nuestro siguiente teorema.
  Como ya es costumbre,
  fue primeramente demostrado por Dedekind,%
    \index{Dedekind, Richard}
  quien no aparece entre los créditos.
  La demostración que mostramos se debe a Julius Kőnig.%
    \index{Konig, Julius@König, Julius}
  \begin{theorem}[Cantor-Bernstein-Schröder]
    \index{Cantor-Bernstein-Schroder, teorema de@Cantor-Bernstein-Schröder, teorema de}
    \label{theo:Cantor-Bernstein-Schroeder}
    Si hay inyecciones
      \(f \colon \mathcal{A} \rightarrow \mathcal{B}\),
    y \(g \colon \mathcal{B} \rightarrow \mathcal{A}\),
    entonces hay una biyección
    entre \(\mathcal{A}\) y \(\mathcal{B}\).
  \end{theorem}
  \begin{proof}
    Sin pérdida de generalidad podemos suponer que \(\mathcal{A}\)
    y \(\mathcal{B}\) son disjuntos.
    Partiendo de un elemento \(a \in \mathcal{A}\) cualquiera,
    podemos definir
    una secuencia en \(\mathcal{A}\) y \(\mathcal{B}\)
    en ambas direcciones
    aplicando repetidas veces \(f\) y \(g\),
    y \(f^{-1}\) y \(g^{-1}\) donde estén definidas:
    \begin{equation*}
      \cdots \rightarrow f^{-1}(g^{-1}(a))
	     \rightarrow g^{-1}(a)
	     \rightarrow a
	     \rightarrow f(a)
	     \rightarrow g(f(a))
	     \rightarrow \cdots
    \end{equation*}
    Por ser inyectivas
    (no hay preimágenes repetidas),
    esta es una cadena.
    Cada elemento de \(\mathcal{A} \cup \mathcal{B}\)
    pertenece a exactamente una cadena,
    esto define una partición.
    Hay varias posibilidades de cadenas diferentes,
    y juntando biyecciones construidas para cada cadena
    obtenemos una biyección entre \(\mathcal{A}\) y \(\mathcal{B}\).
    \begin{description}
    \item[La cadena es infinita en ambas direcciones:]
      En este caso,
      \(f\) es una biyección.
    \item[Es un ciclo:]
      Nuevamente,
      \(f\) es una biyección.
    \item[\boldmath Termina en \(\mathcal{A}\):\unboldmath]
      También en este caso \(f\) es una biyección
    \item[\boldmath Termina en \(\mathcal{B}\)
	  pero no en \(\mathcal{A}\):\unboldmath]
      En este caso \(g\) define una biyección
    \end{description}
    Tenemos la biyección prometida.
  \end{proof}

  Estudiaremos los conjuntos infinitos en algo más de detalle,
  partiendo por \(\mathbb{N}\).
  \begin{definition}
    \index{conjunto!numerable|textbfhy}
    Un conjunto \(\mathcal{X}\) se dice \emph{infinito numerable}
    si hay una biyección entre \(\mathcal{X}\) y \(\mathbb{N}\).
    Un conjunto se llama \emph{numerable} si es finito
    o es infinito numerable.
  \end{definition}
  Esta definición dice que si \(\mathcal{X}\) es infinito numerable,
  entonces podemos escribir \(\mathcal{X} = \{x_1, x_2, \dotsc\}\),
  bajo el entendido
  que existe la biyección
    \(\phi \colon \mathcal{X} \rightarrow \mathbb{N}\)
  con \(\phi(x_n) = n\) para cada \(n \in \mathbb{N}\).
  \begin{theorem}
    \label{theo:union-numerable}
    La unión numerable de conjuntos numerables es numerable.
  \end{theorem}
  \begin{proof}
    Sea \(\mathcal{I}\) un conjunto índice numerable,
    tal que para cada \(i \in \mathcal{I}\)
    el conjunto \(\mathcal{X}_i\) es numerable.
    Entonces \(\mathcal{I}\) es finito o es infinito numerable.
    Consideraremos solo el segundo caso,
    el primero requiere modificaciones menores.

    Como \(\mathcal{I}\) es infinito numerable,
    hay una biyección entre \(\mathcal{I}\) y \(\mathbb{N}\),
    con lo que podemos adoptar \(\mathbb{N}\) como conjunto índice
    sin pérdida de generalidad.
    Definamos:
    \begin{equation*}
      \mathcal{X}
	= \bigcup_{n \in \mathbb{N}} \mathcal{X}_n
    \end{equation*}
    Si \(\mathcal{X}\) es finito,
    estamos listos.

    Supongamos entonces \(\mathcal{X}\) infinito.
    Como para todo \(n \in \mathbb{N}\)
    el conjunto \(\mathcal{X}_n\) es numerable,
    podemos escribir
      \(\mathcal{X}_n = \{x_{n 1}, x_{n 2}, \dotsc\}\).
    Con la convención que si \(\mathcal{X}_n\) es finito
    la secuencia
       \(\left\langle
	   x_{n 1}, x_{n 2}, x_{n 3}, \dotsc
	 \right\rangle\)
    simplemente repite los elementos de \(\mathcal{X}_n\),
    podemos escribir la matriz doblemente infinita:
    \begin{equation*}
      \begin{array}{*{4}{c}}
	x_{1 1} & x_{1 2} & x_{1 3} & \dotso \\
	x_{2 1} & x_{2 2} & x_{2 3} & \dotso \\
	x_{3 1} & x_{3 2} & x_{3 3} & \dotso \\
	\vdots	& \vdots  & \vdots  & \ddots
      \end{array}
    \end{equation*}
    Esta matriz podemos recorrerla diagonalmente,
    siguiendo elementos \(x_{i j}\) en orden de \(i + j\) creciente:
    Primero \(x_{1 1}\),
    luego \(x_{1 2}\) y \(x_{2 1}\),
    después \(x_{1 3}\), \(x_{2 2}\) y \(x_{3 1}\),
    y así sucesivamente,
    omitiendo elementos ya listados.
    Así construimos una biyección
    entre \(\mathbb{N}\) y \(\mathcal{X}\).
  \end{proof}
  \begin{theorem}
    \label{theo:subconjunto-numerable}
    Todo subconjunto de un conjunto numerable es numerable.
  \end{theorem}
  \begin{proof}
    Sea \(\mathcal{X}\) un conjunto numerable.
    Si \(\mathcal{X}\) es finito,
    la conclusión es inmediata.
    Supongamos entonces \(\mathcal{X}\) infinito numerable,
    e \(\mathcal{Y} \subseteq \mathcal{X}\).
    Si \(\mathcal{Y}\) es finito,
    estamos listos.
    Supongamos entonces que \(\mathcal{Y}\) es infinito.
    Definimos la secuencia
      \(\left\langle n_1, n_2, \dotsc \right\rangle\)
    mediante:
    \begin{align*}
      n_1
	&= \min \{n \in \mathbb{N} \colon x_n \in \mathcal{Y}\} \\
      n_k
	&= \min \{n \in \mathbb{N} \colon
		     n > n_{k - 1} \wedge x_n \in \mathcal{Y}\}
    \end{align*}
    La secuencia \(\left\langle n_k \right\rangle_{k \ge 1}\)
    define una biyección entre \(\mathbb{N}\) e \(\mathcal{Y}\)
    (asocia un índice \(k\) con cada elemento de \(\mathcal{Y}\)).
  \end{proof}

  Llegamos así a los resultados más importantes que veremos acá.
  \begin{theorem}
    \label{theo:Z-numerable}
    El conjunto \(\mathbb{Z}\) es numerable.
  \end{theorem}
  \begin{proof}
    Tenemos la unión de tres conjuntos numerables:
    \begin{equation*}
      \mathbb{Z} = \mathbb{N} \cup \{0\} \cup \{-1, -2, -3, \dotsc\}
      \qedhere
    \end{equation*}
  \end{proof}
  \begin{theorem}
    \label{theo:Q-numerable}
    El conjunto \(\mathbb{Q}\) es numerable.
  \end{theorem}
  \begin{proof}
    Podemos representar \(r \in \mathbb{Q}\) como \(r = a / b\),
    con \(a \in \mathbb{Z}\) y \(b \in \mathbb{N}\).
    El conjunto de fracciones con denominador \(b\) es numerable,
    y la colección de tales conjuntos es numerable.
    Por el teorema~\ref{theo:union-numerable},
    su unión \(\mathbb{Q}\) es numerable.
  \end{proof}
  \begin{theorem}[Cantor]
    \index{Cantor, teorema de}
    \index{Cantor, Georg}
    \index{demostracion@demostración!argumento diagonal}
    \label{theo:R-no-numerable}
    El conjunto \(\mathbb{R}\) no es numerable.
  \end{theorem}
  \begin{proof}
    La demostración es por contradicción.
    En vista del teorema~\ref{theo:subconjunto-numerable},
    basta demostrar que \([0, 1)\) no es numerable.
    Supongamos entonces
    que hay una biyección entre \([0, 1)\) y \(\mathbb{N}\).
    Un número \(x \in [0, 1)\)
    puede expresarse en notación decimal como
    \(0,d_1 d_2 d_3 \dotso\),
    donde \(0 \le d_i \le 9\) son los dígitos correspondientes
    de su expansión decimal.
    Ponemos la condición adicional
    que la expansión no es solo nueves
    a partir de un punto dado,
    para evitar ambigüedades.
    Con la biyección supuesta tendremos una matriz
    en que \(d_{i j}\) corresponde al \(j\)\nobreakdash-ésimo dígito
    del \(i\)\nobreakdash-ésimo número:
    \begin{equation*}
      \begin{array}{r*{10}{c@{\,}}l}
	1: & d_{1, 1} & d_{1, 2} & d_{1, 3} & d_{1, 4} & d_{1, 5} & d_{1, 6} &
	     d_{1, 7} & d_{1, 8} & d_{1, 9} & d_{1, 10} & \cdots \\
	2: & d_{2, 1} & d_{2, 2} & d_{2, 3} & d_{2, 4} & d_{2, 5} & d_{2, 6} &
	     d_{2, 7} & d_{2, 8} & d_{2, 9} & d_{2, 10} & \cdots \\
	\vdots\;\;\,
	   & \multicolumn{10}{c}{\vdots} \\
	n: & d_{n, 1} & d_{n, 2} & d_{n, 3} & d_{n, 4} & d_{n, 5} & d_{n, 6} &
	     d_{n, 7} & d_{n, 8} & d_{n, 9} & d_{n, 10} & \cdots \\
	\vdots\;\;\,
	   & \multicolumn{10}{c}{\vdots}
      \end{array}
    \end{equation*}
    Consideremos el número \(y = 0,v_1 v_2 v_3 \dotso\),
    cuyos dígitos se definen por:
    \begin{equation*}
      v_i
	=
	\begin{cases}
	  2 & \text{\ si \(d_{i i} = 1\)} \\
	  1 & \text{\ si \(d_{i i} \ne 1\)}
	\end{cases}
    \end{equation*}
    Claramente \(y \in [0, 1)\),
    no tiene solo nueves a partir de una posición dada,
    y es diferente
    al menos en el dígito \(i\)\nobreakdash-ésimo
    del \(i\)\nobreakdash-ésimo número de la lista,
    con lo que \(y\) no está en esta lista
    que supuestamente los contiene a todos.
    Esta contradicción completa la demostración.
  \end{proof}
  Alguna variante de este argumento diagonal
  se usa en muchas demostraciones
  relacionadas a conjuntos infinitos.

  Nótese que el conjunto \(\mathbb{R} \smallsetminus \mathbb{Q}\)
  no es numerable,
  con lo que en cierto sentido
  hay más números irracionales que racionales.%
    \index{numero@número!irracional}
  Incluso más:
  Un número se llama \emph{algebraico}%
    \index{numero@número!algebraico}
  si es un cero de un polinomio con coeficientes enteros.%
     \index{polinomio!coeficientes enteros}
  Entonces:
  \begin{theorem}[Cantor]
    \label{theo:algebraicos-numerable}
    El conjunto de números reales algebraicos es numerable.
  \end{theorem}
  \begin{proof}
   Dado un polinomio
   \(a_n x^n + a_{n - 1} x^{n - 1} + \dotsb + a_0\),
   su \emph{altura} es
     \(n + \lvert a_n \rvert
	 + \lvert a_{n - 1} \rvert
	 + \dotsb
	 + \lvert a_0 \rvert
     \).
   Hay un número finito de polinomios de cada altura
   y un polinomio de grado \(n\)
   tiene a lo más \(n\) raíces reales
   (esto lo demostraremos
    en el capítulo~\ref{cha:anillos-polinomios}),
   con lo que hay un número finito de números algebraicos reales
   de cada altura.
   Los números algebraicos son entonces una unión numerable
   de conjuntos numerables,
   y por tanto numerables.
  \end{proof}
  Los números reales no algebraicos se llaman \emph{trascendentes}.%
    \index{numero@número!trascendente}
  Euler ya sospechó su existencia,%
    \index{Euler, Leonhard}
  la demostración anterior indica
  que hay muchos más números trascendentes que algebraicos,
  pero no exhibe ninguno.

  Un importante teorema es el siguiente:
  \begin{theorem}[Cantor]
    \label{theo:A<powerset(A)}
    \(\lvert \mathcal{A} \rvert < \lvert 2^{\mathcal{A}} \rvert\)
  \end{theorem}
  \begin{proof}
    Por contradicción.
    Sea \(f \colon \mathcal{A} \rightarrow 2^{\mathcal{A}}\)
    una función cualquiera,
    construimos un conjunto \(\mathcal{T} \in 2^{\mathcal{A}}\)
    que no es imagen de ningún \(\alpha \in \mathcal{A}\),
    con lo que \(f(\cdot)\) no puede ser una biyección
    por no ser sobre.
    Para todo \(\alpha \in \mathcal{A}\),
    debe ser \(\alpha \in \mathcal{T}\)
    o \(\alpha \notin \mathcal{T}\).
    Sea \(\mathcal{T}
	    = \{\alpha \in \mathcal{A} \colon
		  \alpha \notin f(\alpha)\}\).
    Si \(\alpha \in \mathcal{T}\),
    entonces \(\alpha \notin f(\alpha)\),
    de forma que \(f(\alpha) \ne \mathcal{T}\).
    Por el otro lado,
    si \(\alpha \notin \mathcal{T}\),
    entonces \(\alpha \in f(\alpha)\),
    y nuevamente \(f(\alpha) \ne \mathcal{T}\).
    La función \(f\) no es sobre
    ya que su rango no incluye a \(\mathcal{T}\),
    no puede ser biyección.
  \end{proof}
  Esto es nuevamente el argumento diagonal%
    \index{demostracion@demostración!argumento diagonal}
  que usamos para demostrar el teorema~\ref{theo:R-no-numerable}.
  Este teorema demuestra que hay infinitas cardinalidades
  mayores que la de \(\mathbb{N}\).

  Resulta que el conjunto
  de subconjuntos \emph{finitos} de un conjunto numerable
  es numerable.
  Si el conjunto universo es finito,
  el conjunto de sus subconjuntos es finito,
  y por tanto numerable.
  Si el conjunto universo es infinito,
  sin pérdida de generalidad podemos tomarlo como \(\mathbb{N}\).
  Los conjuntos \(S_n = 2^{[1, n]}\) son todos finitos,
  y la unión de todos ellos
  es la unión numerable de conjuntos numerables.

  Vemos también que rangos abiertos y cerrados de \(\mathbb{R}\)
  tienen la misma cardinalidad:
  Por ejemplo,
  como entre \((0, 1)\) y \([0, 1]\) tenemos las inyecciones
  \(x \mapsto x\) e \(y \mapsto (y + 1) / 3\),
  por el teorema~\ref{theo:Cantor-Bernstein-Schroeder}
  tienen la misma cardinalidad.
  Con la biyección \(x \mapsto (x + a) / (b - a)\)
  entre \([0, 1]\) y \([a, b]\)
  todos los rangos finitos tienen la misma cardinalidad.
  La biyección \(x \mapsto \tanh x\)
  entre \(\mathbb{R}\) y \((-1, 1)\)
  muestra que rangos infinitos comparten la misma cardinalidad
  de rangos finitos.

  Lo que sí resulta sorprendente es que \(\mathbb{R}\)
  y \(\mathbb{C}\)
  tienen la misma cardinalidad,
  e incluso en general
  \(\lvert \mathbb{R} \rvert
      = \lvert \mathbb{R}^n \rvert\) para \(n \ge 1\).
  Para ilustrar la demostración general,
  mostraremos una biyección entre \((0, 1]\)
  y el cuadrado \(0 < x, y \le 1\).
  Representamos \(z \in (0, 1]\)
  mediante su expansión decimal que no termina
  (en vez de \(0,5\) escribimos \(0,4\overline{9}\)).
  Dividimos \(z\) en el par \((x, y)\) en \((0, 1]\)
  por la vía de cortar la expansión de \(z\)
  en grupos de ceros y el dígito que sigue.
  Vale decir,
  si \(z = 0,1003049\dotso\) obtenemos
  \(x = 0,104\dotso\) e \(y = 0,0039\dotso\).
  Nótese que como la expansión decimal de \(z\)
  no termina en una secuencia infinita de ceros,
  siempre tendremos dónde cortar para el siguiente;
  y las expansiones resultantes para \(x\) e \(y\)
  nunca terminan en secuencias infinitas de ceros.
  Esto provee una biyección
  entre un rango finito y un cuadrado finito,
  podemos extender ambos como antes.
  La misma idea puede usarse para \(n\) mayor.

%%% Local Variables:
%%% mode: latex
%%% TeX-master: "clases"
%%% End:


% teoria-numeros.tex
%
% Copyright (c) 2009-2015 Horst H. von Brand
% Derechos reservados. Vea COPYRIGHT para detalles

\chapter{Teoría de números}
\label{cha:teoria-numeros}
\index{teoria de numeros@teoría de números}

  El estudio de los números enteros tiene una larga y distinguida historia,
  algo de la cual describe Ore~\cite{ore69:_invit_number_theo}.
  La importancia de la teoría de números en informática
  es porque mucho de lo que se hace en el computador
  es trabajar con números
  (o cosas que se representan como tales,
   o debemos contar objetos con ciertas características
   para determinar los recursos que se requieren).
  Mucha de la tecnología criptográfica moderna%
    \index{criptografia@criptología}
  se basa en resultados de la teoría de números
  y áreas afines.
  Nuestra presentación sigue la de Richman~%
    \cite{richman71:_number_theor}
  en que explícitamente discute las estructuras algebraicas involucradas.

\section{Algunas herramientas}
\label{sec:algunas-herramientas}

  Para calcular con números grandes
  resulta cómodo el programa \texttt{bc(1)},%
    \index{bc(1)@\texttt{bc(1)}}
  para computaciones más complejas
  son útiles \texttt{PARI/GP}~%
    \cite{PARI:2.7.2}%
    \index{PARI/GP@\texttt{PARI/GP}}
  o \texttt{maxima}~%
    \cite{maxima14b:_computer_algebra}.%
    \index{maxima@\texttt{maxima}}
  El sistema Sage~\cite{stein14:_Sage-6.3}%
    \index{Sage@\texttt{Sage}}
  agrupa varios sistemas para computación numérica y simbólica
  (incluyendo los mencionados)
  de código abierto bajo una interfaz común.%
    \index{codigo abierto@código abierto}
  Para uso en programas se recomiendan las bibliotecas GMP~%
    \cite{granlund14:_gnu_multip_precis_arith_librar},%
    \index{GMP@\texttt{GMP}}
 CLN~%
    \cite{haible14:_CLN_1.3.4}%
    \index{CLN@\texttt{CLN}}
  o NTL~%
    \cite{shoup14:_ntl}.%
    \index{NTL@\texttt{NTL}}
  Detalles de muchos algoritmos en \cplusplus{}%
    \index{C++ (lenguaje de programacion)@\cplusplus{} (lenguaje de programación)}
  da Arndt~%
    \cite{arndt11:_matters_computational}.

\section{Propiedades básicas}
\label{sec:propiedades-basicas}
\index{Z (numeros enteros)@\(\mathbb{Z}\) (números enteros)}

  Ya sabemos
  (ver el capítulo~\ref{cha:relaciones-funciones})
  que la relación ``divide a'' es una relación de orden en \(\mathbb{N}\)%
    \index{relacion@relación!orden}
  (en \(\mathbb{Z}\) no es antisimétrica,
   ya que por ejemplo \(-3 \mid 3\) y \(3 \mid -3\),
   pero \(3 \ne -3\)).
  Algunas propiedades adicionales son:
  \begin{enumerate}
  \item
    Si \(a \mid b\),
    entonces \(a \mid b \cdot c\)
  \item
    Si \(a \mid b\) y \(a \mid c\),
    entonces \(a \mid (s b + t c)\)
    para todo \(s, t \in \mathbb{Z}\)
  \item
    Siempre que \(c \ne 0\),
    \(a \mid b\) si y solo si \(a c \mid b c\)
  \end{enumerate}
  Si \(a \mid b\),
  decimos que \(b\) es múltiplo de \(a\),
  o que \(a\) es un factor de \(b\).

  Una propiedad básica de los enteros es la siguiente:
  \begin{theorem}[Algoritmo de división]
    \index{algoritmo de division@algoritmo de división}
    \label{theo:division}
    Sean \(n, d \in \mathbb{Z}\)
    con \(d > 0\).
    Entonces existen enteros \(q, r\) únicos tales que:
    \begin{equation*}
      n = q \cdot d + r \quad 0 \le r < d
    \end{equation*}
  \end{theorem}
  \begin{proof}
    El conjunto de ``restos'' es:
    \begin{equation}
      \mathcal{R} = \{n - s d \colon s \in \mathbb{Z} \wedge n - s d \ge 0\}
    \end{equation}
    Este conjunto no es vacío,
    siempre será \(n + (\lvert n \rvert + 1) \cdot d > 0\),
    y esto está en \(\mathcal{R}\).
    Siendo \(\mathcal{R}\) un conjunto de enteros acotado por debajo,
    contiene su mínimo,
    llamémosle \(r\),
    que podemos expresar \(r = n - q \cdot d\).
    Debemos demostrar que \(r < d\).
    Esto lo haremos por contradicción.%
      \index{demostracion@demostración!contradiccion@contradicción}
    Si suponemos \(r \ge d\),
    podemos escribir:
    \begin{equation*}
      r - d = n - (q + 1) \cdot d
    \end{equation*}
    donde \(r - d \ge 0\),
    pertenece a \(\mathcal{R}\)
    y es menor que \(r\),
    contradiciendo la elección de \(r\) como el menor elemento
    en \(\mathcal{R}\).

    Falta demostrar que \(q\) y \(r\) son únicos.
    Nuevamente procedemos por contradicción.%
      \index{demostracion@demostración!contradiccion@contradicción}
    Supongamos dos soluciones diferentes \(q_1, r_1\) y \(q_2, r_2\),
    donde podemos tomar \(r_2 \ge r_1\) sin pérdida de generalidad.
    Entonces:
    \begin{align}
      n
	&= q_1 \cdot d + r_1 \notag \\
      n
	&= q_2 \cdot d + r_2 \notag \\
      0
	&= (q_2 - q_1) \cdot d + (r_2 - r_1) \label{eq:(q2-q1)*d+(r2-r1)}
    \end{align}
    Sabemos que \(0 \le r_2 < d\) por lo anterior.
    Como asumimos \(r_1 \le r_2\),
    tenemos que \(0 \le r_2 - r_1\).
    Pero también,
    dado que \(0 \le r_1 < d\):
    \begin{align*}
      r_2
	&< d \\
      r_2 - r_1
	&< d - r_1
	  \le d
    \end{align*}
    En resumen,
    resulta:
    \begin{equation}
      \label{eq:r2-r1}
      0 \le r_2 - r_1 < d
    \end{equation}
    Pero de la ecuación~\eqref{eq:(q2-q1)*d+(r2-r1)} tenemos:
    \begin{equation}
      \label{eq:q1-q2}
      (q_1 - q_2) \cdot d
	= r_2 - r_1
    \end{equation}
    Como el lado izquierdo de~\eqref{eq:q1-q2} es divisible por \(d\)
    también lo es el derecho;
    pero por~\eqref{eq:r2-r1} esto es menor que \(d\),
    y la única posibilidad es que sea cero.
    Concluimos que \(r_1 = r_2\).
    Pero siendo cero el lado derecho de~\eqref{eq:q1-q2},
    y el izquierdo el producto de \(d\)
    y \(q_1 - q_2\),
    concluimos que \(q_1 - q_2 = 0\),
    o sea \(q_1 = q_2\).
    Pero habíamos supuesto que \((q_1, r_1) \ne (q_2, r_2)\),
    contradicción que demuestra que \(q\) y \(r\) son únicos.
  \end{proof}

  Es común que queramos hablar del resto de la división,%
    \index{modulo@módulo|textbfhy}
  para ello introducimos la notación:
  \begin{equation*}
    r = a \bmod b
  \end{equation*}
  El cociente respectivo puede expresarse simplemente como:%
    \index{cociente entero}
  \begin{equation*}
    q = \left\lfloor \frac{a}{b} \right\rfloor
  \end{equation*}
  No se requiere una notación especial para esto,
  se usa con menos frecuencia que el resto.

\section{Máximo común divisor}
\label{sec:GCD}
\index{maximo comun divisor@máximo común divisor|textbfhy}
\index{GCD@\emph{GCD}|see{máximo común divisor}}

  Sean \(a, b \in \mathbb{Z}\),
  y consideremos
  \(\mathcal{I} = \{u a + v b \colon u, v \in \mathbb{Z}\}\).%
    \index{ideal}
  Si \(a = b = 0\),
  entonces \(\mathcal{I} = \{0\}\).
  En caso contrario,
  este conjunto no es vacío,
  y contiene elementos positivos
  (siempre es \(a^2 + b^2 \in \mathcal{I}\)).
  Consideremos el mínimo elemento positivo de \(\mathcal{I}\),
  llamémosle \(m\).
  Entonces hay \(s, t \in \mathbb{Z}\)
  tales que:
  \begin{equation}
    \label{eq:base-Bezout}
    m = s a + t b
  \end{equation}
  Demostraremos que \(m\) divide a todos los elementos de \(\mathcal{I}\).

  Sea \(n \in \mathcal{I}\),
  que significa \(n = s' a + t' b\) para algún \(s'\) y \(t'\).
  Por el algoritmo de división:%
    \index{algoritmo de division@algoritmo de división}
  \begin{equation}
    \label{eq:n=qm+r}
    n = q m + r \quad 0 \le r < m
  \end{equation}
  Pero:
  \begin{align*}
    n
      &= s' a + t' b \\
      &= q (s a + t b) + r	 \\
    r
      &= (s' - q s) a + (t' - q t) b
  \end{align*}
  con lo que \(r \in \mathcal{I}\).
  Como \(m\) es el mínimo elemento positivo de \(\mathcal{I}\),
  por~\eqref{eq:n=qm+r}
  solo puede ser \(r = 0\),
  y \(m \mid n\);
  e \(\mathcal{I}\) es el conjunto de los múltiplos de \(m\).
  Como \(a\) y \(b\) pertenecen a \(\mathcal{I}\),
  \(m\) es un divisor común de ambos.
  Pero por otro lado,
  cualquier divisor de \(a\) y \(b\) debe dividir a \(m = s a + t b\),
  con lo que \(m\) es máximo.
  Este número lo anotaremos \(\gcd(a, b)\)
  (por \emph{\foreignlanguage{english}{greatest common divisor}} en inglés).
  Otra notación común es \((a, b)\).
  Para completar la definición de esta función,
  el máximo común divisor entre \(a \ne 0\) y \(0\) es simplemente \(a\),
  y podemos definir \(\gcd(0, 0) = 0\).
  A la importante relación \(\gcd(a, b) = s a + t b\)
  se le llama \emph{identidad de Bézout}.
    \index{Bezout, identidad de@Bézout, identidad de|textbfhy}%
    \index{Bezout, Etienne@Bézout, Étienne}
  En caso que \(\gcd(a, b) = 1\) se dice que \(a\) y \(b\)
  son \emph{relativamente primos}%
    \index{relativamente primos}%
  o \emph{coprimos}.%
    \index{coprimos|see{relativamente primos}}

  Nótese que si:
  \begin{equation}
    \label{eq:Bezout}
    \gcd(a, b)
      = s a + t b
  \end{equation}
  también son soluciones a \(\gcd(a, b) = s' a + t' b\)
  para todo \(k \in \mathbb{Z}\):
  \begin{equation}
    \label{eq:Bezout-coefficients}
    \begin{split}
      s' &= s + \frac{k b}{\gcd(a, b)} \\
      t' &= t - \frac{k a}{\gcd(a, b)}
    \end{split}
  \end{equation}
  Esto es fácil de ver substituyendo~\eqref{eq:Bezout-coefficients}.

  \begin{lemma}
    \index{maximo comun divisor@máximo común divisor!propiedades}
    \index{operacion@operación!maximo comun divisor@máximo común divisor}
    \label{lem:gcd}
    Tenemos las siguientes propiedades del máximo común divisor:
    \begin{enumerate}
    \item
      \label{lem:gcd:conmutativo}
      \(\gcd(a, b) = \gcd(b, a)\)
    \item
      \label{lem:gcd:signos}
      \(\gcd(a, b) = \gcd(\pm a, \pm b)\)
    \item
      \label{lem:gcd:gcd_maximal}
      Todo divisor común de \(a\) y \(b\) divide a \(\gcd(a, b)\).
    \item
      \label{lem:gcd:factor_comun}
      \(\gcd(k a, k b) = \lvert k  \rvert \cdot \gcd(a, b)\).
    \item
      \label{lem:gcd:dividir_gcd}
      Si \(m = \gcd(a, b)\),
      entonces \(\gcd(a / m, b / m) = 1\).
      Nótese que \(a / m\) y \(b / m\) son enteros acá,
      la división es exacta.
    \item
      \label{lem:gcd:producto}
      Si \(\gcd(a, b) = 1\) y \(\gcd(a, c) = 1\),
      entonces \(\gcd(a, b c) = 1\).
    \item
      \label{lem:gcd:divisor}
      Si \(a \mid b c\) y \(\gcd(a, b) = 1\),
      entonces \(a \mid c\).
    \item
      \label{lem:gcd:producto-coprimos-divide}
      Si \(\gcd(a, b) = 1\),
      y \(a \mid c\) y \(b \mid c\),
      entonces \(a b \mid c\).
    \end{enumerate}
  \end{lemma}
  \begin{proof}
    Demostramos cada parte por turno.
    \begin{enumerate}
    \item
      Los elementos de
      \(\mathcal{I} = \{u a + v b \colon u, v \in \mathbb{Z}\}\)
      e \(\mathcal{I}' = \{u b + v a \colon u, v \in \mathbb{Z}\}\)
      son los mismos,
      y lo son sus mínimos elementos positivos.
    \item
      Nuevamente,
      los elementos del conjunto \(\mathcal{I}\) respectivo
      son los mismos para ambos lados de la ecuación.
    \item
      Esto lo vimos antes.
    \item
      De la identidad de Bézout%
	\index{Bezout, identidad de@Bézout, identidad de}
      sabemos que hay \(s, t \in \mathbb{Z}\)
      tales que:
      \begin{align*}
	\gcd(k a, k b)
	  &= s (\lvert k  \rvert \cdot a)
	       + t (\lvert k  \rvert \cdot b) \\
	  &= \lvert k  \rvert \cdot (s a + t b)
      \end{align*}
      En particular,
      este es el mínimo de todos los valores positivos
      que se pueden obtener eligiendo \(s, t \in \mathbb{Z}\),
      por lo que \(s a + t b\)
      debe también ser el mínimo positivo de esta última expresión,
      \(s a + t b = \gcd(a, b)\).
    \item
      De la identidad de Bézout
      sabemos que hay \(s\) y \(t\) que dan \(m = \gcd(a, b)\) como:
      \begin{align*}
	m &= s a + t b \\
	1 &= s (a / m) + t (b / m)
      \end{align*}
      con lo que \(\gcd(a / m, b / m) = 1\).
      Esto también implica que \(\gcd(s, t) = 1\).
    \item
      Si \(\gcd(a, b) = \gcd(a, c) = 1\)
      existen \(s, t, u, v \in \mathbb{Z}\)
      tales que:
      \begin{align*}
	1 &= s a + t b \\
	1 &= u a + v c
      \end{align*}
      Entonces:
      \begin{align*}
	t b
	  &= 1 - s a \\
	v c
	  &= 1 - u a \\
	t v b c
	  &= 1 - (s + u) a + su a^2 \\
	1
	  &= (s + u - s u a) a + (t v) b c
      \end{align*}
      Esto es el mínimo positivo,
      y por tanto es \(\gcd(a, b c)\).
    \item
      \(a \mid b c\) significa que existe \(k\) tal que \(b c = k a\).
      Tenemos:
      \begin{align*}
	1
	  &= s a + t b \\
	c
	  &= s a c + t b c \\
	  &= (s c + t k) \cdot a
      \end{align*}
      y esto último dice que \(a \mid c\).
    \item
      Existen \(x, y \in \mathbb{Z}\) tales que \(c = a x = b y\).
      Por la identidad de Bézout
      existen \(u, v \in \mathbb{Z}\) con:
      \begin{align*}
	a u + b v
	  &= 1 \\
	a c u + b c v
	  &= c \\
	a b u y + a b v x
	  &= c \\
	a b (u y + v x)
	  &= c
      \end{align*}
      con lo que \(a b \mid c\).
      \qedhere
    \end{enumerate}
  \end{proof}

  Como ejemplo
  demostramos \(\gcd(x^2, y^2) = \left(\gcd(x, y)\right)^2\).
  Si \(\gcd(x, y) = 1\),
  aplicando la propiedad~(\ref{lem:gcd:producto})
  con \(a = x\), \(b = c = y\)
  tenemos \(\gcd(x, y^2) = 1\).
  Repitiendo esto con \(a = y^2\) y \(b = c = x\)
  resulta \(\gcd(x^2, y^2) = 1\).
  En realidad,
  podemos demostrar de la misma forma que \(\gcd(x^m, y^n) = 1\),
  para \(m \ge 1\) y \(n \ge 1\).
  Ahora,
  por~(\ref{lem:gcd:factor_comun}),
  si \(x = k u\) y \(y = k v\),
  donde \(k = \gcd(u, v)\) tendremos \(\gcd(u, v) = 1\),
  y \(\gcd(x^2, y^2) = \gcd(k^2 u^2, k^2 v^2)
		     = k^2 \cdot \gcd(u^2, v^2)
		     = \left(\gcd(x, y)\right)^2\).
  Queda como ejercicio demostrar de forma similar
  que \(\gcd(x^m, y^m) = \left(\gcd(x, y)\right)^m\).

  El máximo común divisor es muy importante,
  interesa obtener una forma de calcularlo eficientemente.
  De partida,
  sabemos que si \(m = \gcd(a, b)\),
  entonces \(m \mid u \cdot a + v \cdot b\)
  para todo par \(u, v\).
  En particular,
  \(m \mid a \bmod b\),
  ya que \(a \bmod b = a - q \cdot b\).
  Al revés,
  cualquier divisor común de \(a \bmod b\) y \(b\)
  divide a \(a = a \bmod b + q b\) y a \(b\),
  y por lo tanto a su máximo común divisor.
  Así \(\gcd(a, b) = \gcd(b, a \bmod b)\).
  Esto lleva directamente al algoritmo de Euclides~\ref{alg:Euclides}%
    \index{maximo comun divisor@máximo común divisor!algoritmo|see{Euclides, algoritmo de}}%
    \index{Euclides, algoritmo de}%
    \index{algoritmo!Euclides|see{Euclides, algoritmo de}}%
    \index{Euclides}
  (los 13 tomos de los \emph{Elementos} de este alejandrino del siglo III~AC%
    \index{Euclides, Elementos}
   incluyen teoría de números en los tomos 7 a 9).
  El algoritmo de Euclides de interés histórico también,
  es el algoritmo más antiguo que involucra ciclos,
  y fue el primer algoritmo cuyo rendimiento se analizó matemáticamente
  (por Gabriel Lamé en 1844).%
    \index{Lame, Gabriel Leon Jean Baptiste@Lamé, Gabriel Léon Jean Baptiste}
  El análisis lo discutiremos en la sección~\ref{sec:gcd}.
  \begin{algorithm}[htbp]
    \DontPrintSemicolon
    \SetKwFunction{Gcd}{gcd}

    \KwFunction \Gcd{\(a,\; b\)} \;
    \BlankLine
    \While{\(b > 0\)}{
      \((a, \; b) \leftarrow (b, \; a \bmod b)\) \;
    }
    \Return \(a\) \;
    \caption{Algoritmo de Euclides para calcular $\gcd(a, b)$}
    \label{alg:Euclides}
  \end{algorithm}

  Una función íntimamente relacionada con el máximo común divisor
  es el mínimo común múltiplo,
  que anotaremos \(\lcm(a, b)\)%
    \index{minimo comun multiplo@mínimo común múltiplo}%
    \index{LCM@\texttt{LCM}|see{mínimo común múltiplo}}
  (por \emph{\foreignlanguage{english}{least common multiple}} en inglés).
  Sea \(m\) un múltiplo común de \(a\) y \(b\),
  vale decir \(m = h a = k b\).
  Sean \(a = a_1 \gcd(a, b)\) y \(b = b_1 \gcd(a, b)\),
  dividiendo la relación para \(m\) por \(\gcd(a, b)\)
  resulta \(h a_1 = k b_1\).
  Como por el lema~\ref{lem:gcd} parte~\ref{lem:gcd:dividir_gcd}
  es \(\gcd(a_1, b_1) = 1\),
  por la parte~\ref{lem:gcd:divisor} debe ser  \(a_1 \mid k\),
  y obtenemos el mínimo cuando \(a_1 = k\),
  vale decir:
  \begin{equation}
    \label{eq:compute-lcm}
    \lcm(a, b) = \frac{\lvert a b  \rvert}{\gcd(a, b)}
  \end{equation}

  Una aplicación simple de lo anterior es el siguiente teorema:
  \begin{theorem}[Criterio de cero racional]
    \index{polinomio!criterio de cero racional}
    \label{theo:polinomial-rational-roots}
    Sea \(p(x) = a_n x^n + \dotsb + a_0\)
    un polinomio de coeficientes enteros.
    Todo cero racional \(r = u / v\) de \(p(x)\)
    expresada en mínimos términos cumple
    \(u \mid a_0\) y \(v \mid a_n\).
  \end{theorem}
  \begin{proof}
    Sin pérdida de generalidad
    podemos suponer que los \(a_i\) no tienen factores en común,
    que \(a_n \ne 0\) y que \(a_0 \ne 0\).
    Substituyendo \(u / v\) en \(p(x) = 0\),
    y multiplicando por \(v^n\),
    resulta:
    \begin{equation*}
      a_n u^n + a_{n - 1} u^{n - 1} v + \dotsb a_1 u v^{n - 1} + a_0 v^n
	= 0
    \end{equation*}
    Observamos que todos los términos,
    salvo posiblemente el primero,
    son divisibles por \(v\).
    En consecuencia,
    como la expresión completa es divisible por \(v\),
    tiene que serlo el primer término,
    o sea \(v \mid a_n\),
    ya que supusimos que \(u\) y \(v\) no tienen factores en común.
    Asimismo,
    todos los términos,
    salvo posiblemente el último,
    son divisibles por \(u\).
    Por el mismo razonamiento anterior,
    la única forma que esto se puede cumplir es que
    \(u \mid a_0\).
  \end{proof}
  El teorema~\ref{theo:polinomial-rational-roots}
  restringe los posibles ceros racionales del polinomio a un número finito.
  En el caso de polinomios mónicos
    \index{polinomio!monico@mónico}
  (cuyo coeficiente del término de mayor grado es uno)
  vemos que los ceros son enteros
  de ser racionales.
  Esto da una manera adicional de demostrar que \(\sqrt{2}\) es irracional:%
   \index{numero@número!irracional!\(\sqrt{2}\)}
  Es cero de \(x^2 - 2\),
  como \(\sqrt{2}\) no es entero,
  es irracional.
  Incluso más,
  toda raíz de un entero o es entera o es irracional,%
    \index{numero@número!irracional}
  por un razonamiento similar.

\subsection{Obtener los coeficientes de Bézout}
\label{sec:coeficientes-Bezout}

  Sabemos por la identidad de Bézout que existen enteros \(s, t\)
  tales que:%
    \index{Bezout, identidad de@Bézout, identidad de}
  \begin{equation*}
    \gcd(a, b) = s a + t b
  \end{equation*}
  y encontrar estos es de interés también.
  Una manera de proceder es ir registrando paso a paso los valores
  en el algoritmo de Euclides,%
    \index{Euclides, algoritmo de}
  y luego ir reemplazando en reversa.
  Por ejemplo,
  para calcular \(\gcd(40\,902, 24\,140)\) hacemos:

  \begin{center}
    \begin{tabular}[htbp]{>{\(}r<{\)}
			  @{${} = {}$}>{\(}r<{\)}
			     @{${} \cdot {}$}>{\(}r<{\)}
			     @{${} + {}$}>{\(}r<{\)}}
      40\,902 & 24\,140 & 1 & 16\,762 \\
      24\,140 & 16\,762 & 1 &  7\,378 \\
      16\,762 &	 7\,378 & 2 &  2\,006 \\
       7\,378 &	 2\,006 & 3 &  1\,360 \\
       2\,006 &	 1\,360 & 1 &	  646 \\
       1\,360 &	    646 & 2 &	   68 \\
	  646 &	     68 & 9 &	   34 \\
	   68 &	     34 & 2 &	    0
    \end{tabular}
  \end{center}
  Sabemos entonces que \(\gcd(40\,902, 24\,140) = 34\).
  De lo anterior también tenemos que:
  \begin{center}
    \begin{tabular}[htbp]{>{\(}r<{\)}
			  @{${} = {}$}>{\(}r<{\)}
			     @{${} - {}$}>{\(}r<{\)}
			     @{${} \cdot {}$}>{\(}r<{\)}}
	    34 &      646 &	 68 & 9 \\
	    68 &   1\,360 &	646 & 2 \\
	   646 &   2\,006 &  1\,360 & 1 \\
	1\,360 &   7\,378 &  2\,006 & 3 \\
	2\,006 &  16\,762 &  7\,378 & 2 \\
	7\,378 &  24\,140 & 16\,762 & 1 \\
       16\,762 &  40\,902 & 24\,140 & 1
    \end{tabular}
  \end{center}
  Substituyendo las expresiones para los restos
  finalmente obtenemos \(34 = 337 \cdot 40\,902 - 571 \cdot 24\,140\).

  Una manera de organizar mejor el trabajo es el algoritmo~\ref{alg:xgcd},
  debido a Blankinship~%
    \cite{blankinship63:_new_version_euclid_algor}.%
    \index{Euclides, algoritmo de!extendido}%
    \index{Blankinship, algoritmo de|see{Euclides, algoritmo de!extendido}}%
    \index{Blankinship, W. A.}
  Usa vectores auxiliares
  \((x_1, x_2, x_3)\), \((y_1, y_2, y_3)\) y \((t_1, t_2, t_3)\),
  que manipula
  de forma que siempre se cumple que:
  \begin{align*}
    x_1 a + x_2 b &= x_3 \\
    y_1 a + y_2 b &= y_3 \\
    t_1 a + t_2 b &= t_3
  \end{align*}
  \begin{algorithm}[htbp]
    \DontPrintSemicolon

    \KwFunction \FuncSty{xgcd(}\ArgSty{\(a, \; b\)}\FuncSty{)} \;
    \BlankLine
    \((x_1, x_2, x_3) \leftarrow (1, 0, a)\) \;
    \((y_1, y_2, y_3) \leftarrow (0, 1, b)\) \;
    \While{\(y_3 \ne 0\)}{
       \(q \leftarrow \lfloor x_3 / y_3 \rfloor\) \;
       \((t_1, t_2, t_3)
	    \leftarrow (x_1, x_2, x_3) - q \cdot (y_1, y_2, y_3)\) \;
       \((x_1, x_2, x_3) \leftarrow (y_1, y_2, y_3)\) \;
       \((y_1, y_2, y_3) \leftarrow (t_1, t_2, t_3)\) \;
    }
    \Return \(x_3 = x_1 \cdot a + x_2 \cdot b\) \;
    \caption{Algoritmo extendido de Euclides}
    \label{alg:xgcd}
  \end{algorithm}
  El algoritmo~\ref{alg:xgcd} es exactamente el mismo
  que el algoritmo~\ref{alg:Euclides}
  respecto de la manipulación de \(x_3\) e \(y_3\),
  con lo que calcula \(\gcd(a, b)\) correctamente;
  por la relación que se mantiene
  entre \(a\), \(b\), \(x_1\), \(x_2\) y \(x_3\)
  obtenemos los coeficientes de Bézout.

  Podemos eliminar buena parte de la computación
  del algoritmo~\ref{alg:xgcd}
  si omitimos \(x_2\), \(y_2\) y \(t_2\),
  y obtenemos \(x_2\) de la relación \(x_1 a + x_2 b = x_3\)
  al final.
  Manejar los casos de \(a\) o \(b\) negativos queda de ejercicio.

  La traza del algoritmo~\ref{alg:xgcd}
  para \(\gcd(40\,902, 24\,140)\)
  da el cuadro~\ref{tab:traza-xgcd}.
  \begin{table}[htbp]
    \centering
    \begin{tabular}{|>{\(}r<{\)}|*{6}{>{\(}r<{\)}|}}
      \hline
      \multicolumn{1}{|c|}{\rule[-0.7ex]{0pt}{3ex}\(\boldsymbol{x_1}\)} &
	\multicolumn{1}{|c|}{\(\boldsymbol{x_2}\)} &
	\multicolumn{1}{c|}{\(\boldsymbol{x_3}\)} &
	\multicolumn{1}{c|}{\(\boldsymbol{y_1}\)} &
	\multicolumn{1}{c|}{\(\boldsymbol{y_2}\)} &
	\multicolumn{1}{c|}{\(\boldsymbol{y_3}\)} &
	\multicolumn{1}{c|}{\(\boldsymbol{q}\)} \\
      \hline\rule[-0.7ex]{0pt}{3ex}%
	1  &	 0  & 40\,902  &     0	&     1	 & 24\,140  &	  1 \\
	0  &	 1  & 24\,140  &     1	&    -1	 & 16\,762  &	  1 \\
	1  &	-1  & 16\,762  &    -1	&     2	 &  7\,378  &	  2 \\
       -1  &	 2  &  7\,378  &     3	&    -5	 &  2\,006  &	  3 \\
	3  &	-5  &  2\,006  &   -10	&    17	 &  1\,360  &	  1 \\
      -10  &	17  &  1\,360  &    13	&   -22	 &     646  &	  2 \\
       13  &   -22  &	  646  &   -36	&    61	 &	68  &	  9 \\
      -36  &	61  &	   68  &   337	&  -571	 &	34  &	  2 \\
      337  &  -571  &	   34  &  -710	&  1203	 &	 0  &	    \\
      \hline
    \end{tabular}
    \caption{Traza del algoritmo extendido de Euclides}
    \label{tab:traza-xgcd}
  \end{table}
  nuevamente:
  \begin{equation*}
    \gcd(40\,902, 24\,140) = 34 = 337 \cdot 40\,902 - 571 \cdot 24\,140
  \end{equation*}

\subsection{Números primos}
\label{sec:primos}

  \begin{definition}
    \index{numero@número!primo|textbfhy}
    \label{def:Z:prime}
    Un entero positivo \(p \ge 2\) se llama \emph{primo}
    si siempre que \(p \mid a b\)
    es \(p \mid a\) o \(p \mid b\).
  \end{definition}
  Esto lo usamos al demostrar antes
  que \(\sqrt{2}\) es irracional
  (teorema~\ref{theo:sqrt2-irracional}):%
    \index{numero@número!irracional!\(\sqrt{2}\)}
  Si \(2 \mid a^2\),
  debe ser \(2 \mid a\) ya que \(2\) es primo.

  La definición más tradicional es que \(p\) es primo
  si sus únicos divisores positivos son \(1\) y \(p\):
  \begin{definition}
    \index{irreductible}
    \label{def:Z:irreducible}
    A un entero \(e\) se le llama \emph{irreductible}
    si siempre que \(e = a b\) es \(a = \pm 1\) o \(b = \pm 1\).
  \end{definition}
  Notar que si \(p\) es primo,
  es irreductible:
  Si escribimos \(p = a b\) por la definición de primo
  \(p\) divide \(a\) o \(b\).
  Sin pérdida de generalidad,
  supongamos que divide a \(b\),
  de forma que \(b = p c\).
  Entonces \(p = a c p\),
  y \(a c = 1\).
  O sea,
  \(a = \pm 1\).

  Y resulta:
  \begin{theorem}
    \label{theo:Z:irreductible=>prime}
    Si \(p\) es irreductible,
    \(p\) es primo.
  \end{theorem}
  \begin{proof}
    Por contradicción.%
      \index{demostracion@demostración!contradiccion@contradicción}
    Supongamos \(p\) primo y reductible,
    con lo que \(p = a b\)
    para \(a \ne \pm 1\) y \(b \ne \pm 1\).
    Así \(p \mid a b\),
    con lo que por definición de primo \(p \mid a\) o \(p \mid b\).
    Si \(p \mid a\),
    es \(a = m p\) para un entero \(m\).
    Pero entonces:
    \begin{equation*}
      p = a b = m p b
    \end{equation*}
    con lo que:
    \begin{equation*}
      p (1 - m b) = 1
    \end{equation*}
    O sea,
    \(p \mid 1\),
    lo que es absurdo.
  \end{proof}

  Todo número natural se puede escribir como un producto de primos.
  Para demostrarlo usamos un truco bastante común:
  Nos fijamos en el mínimo supuesto contraejemplo
  y usamos contradicción.
  \begin{theorem}
    \label{theo:natural=producto-primos}
    Todo número natural se puede expresar como producto de números primos.
  \end{theorem}
  \begin{proof}
    Por convención,
    \(1\) es el producto de cero primos.

    Para números mayores a \(1\),
    la demostración es por contradicción.
    Llamémosle \(m\) al mínimo número que no es un producto de primos.
    Entonces \(m\) no puede ser primo,
    ya que de serlo sería el producto de primos
    (uno solo);
    así podemos escribir \(m = a \cdot b\),
    donde \(1 < a, b < m\)
    y por tanto \(a\) y \(b\) son productos de primos.
    Pero entonces podemos escribir \(m\) como producto de primos,
    contrario a nuestra suposición de que tal cosa no era posible.
  \end{proof}

  Nuestro siguiente objetivo es demostrar
  que la factorización en primos es única.
  Un paso clave en esa dirección es:
  \begin{lemma}
    \label{lem:p-product}
    Sea \(p\) primo y \(x_1, x_2, \dotsc, x_n\) enteros
    tales que \(p \mid x_1 x_2 \dotsm x_n\).
    Entonces \(p \mid x_i\) para algún \(i\).
  \end{lemma}
  \begin{proof}
    Usamos inducción.%
      \index{induccion@inducción}
    \begin{description}
    \item[Base:]
      Cuando \(n = 1\),
      el supuesto se reduce a \(p \mid x_1\),
      y el resultado es inmediato.
      Por la definición de primo,
      para  \(n = 2\)
      si \(p \mid x_1 x_2\),
      divide a uno de los dos.
    \item[Inducción:]
      Supongamos que el resultado es cierto para \(n\),
      y queremos demostrar que es válido para \(n + 1\).
      Tenemos \(p \mid x_1 \dotsm x_n \cdot x_{n + 1}\).
      Por la definición de primo,
      es \(p \mid x_1 \dotsm x_n\)
      o \(p \mid x_{n + 1}\).
      En el primer caso (por inducción) \(p \mid x_i\)
      para algún \(1 \le i \le n\),
      en el segundo \(p \mid x_{n + 1}\).
      En resumen,
      \(p \mid x_i\) para algún \(1 \le i \le n + 1\).
      \qedhere
    \end{description}
  \end{proof}

  \begin{theorem}[Teorema fundamental de la aritmética]
    \index{fundamental de la aritmetica, teorema@fundamental de la aritmética, teorema|textbfhy}
    \label{theo:fundamental-aritmetica}
    Todo entero positivo
    tiene una factorización única en números primos,
    salvo el orden de los factores.
  \end{theorem}
  \begin{proof}
    El número \(1\) es un caso especial,
    se factoriza en \(0\) primos.

    Para los demás procedemos por contradicción,%
      \index{demostracion@demostración!contradiccion@contradicción}
    aplicando el mismo truco anterior.
    Si hay enteros para los que esto no es cierto,
    hay uno mínimo,
    llamémosle \(N\).
    Vale decir,
    podemos escribir \(N = p_1 p_2 \dotsm p_k\)
    y también \(N = q_1 q_2 \dotsm q_l\),
    donde los \(p_i\) son primos
    (no necesariamente distintos),
    y similarmente los \(q_j\).
    Ahora bien,
    como \(p_1 \mid N\),
    sabemos que \(p_1 \mid q_1 q_2 \dotsm q_l\),
    y por tanto \(p_1 \mid q_j\) para algún \(1 \le j \le l\).
    Como \(q_j\) es primo,
    es irreductible y esto significa que \(p_1 = q_j\).
    Pero entonces tenemos
    \begin{equation*}
      N'
	= p_2 p_3 \dotsm p_k
	= q_1 q_2 \dotsm q_{j - 1} q_{j + 1} \dotsm q_l
    \end{equation*}
    y  \(N' < N\) también tendría dos factorizaciones diferentes,
    contrario a la elección de \(N\) como el mínimo natural
    con esa característica.
  \end{proof}

  Otro hecho fundamental fue demostrado primeramente por Euclides.%
    \index{Euclides}
  De este importante resultado se da una variedad de bonitas demostraciones,
  basadas en conceptos totalmente diferentes,
  en Aigner y Ziegler~\cite{aigner14:_proof_the_book}.
  Nuestra variante de la demostración clásica de Euclides
  se debe a Ernst Kummer.%
    \index{Kummer, Ernst}
  \begin{theorem}
    \label{theo:infinitos-primos}
    Hay infinitos números primos.
  \end{theorem}
  \begin{proof}
    Procedemos por contradicción.%
      \index{demostracion@demostración!contradiccion@contradicción}
    Supongamos que hay un número finito de primos,
    \(p_1, p_2, \dotsc p_r\) en orden creciente,
    donde sabemos que \(r > 1\).
    Consideremos:
    \begin{equation*}
      N = p_1 p_2 \dotsm p_r
    \end{equation*}
    El número \(N - 1\) es compuesto,
    ya que es mayor que \(p_r\) y no es primo
    (no aparece en nuestra lista).
    Luego tiene un factor primo \(p\),
    y este factor lo tiene en común con \(N\).
    Entonces \(p\) divide tanto a \(N\) como a \(N - 1\),
    y divide a su diferencia,
    que es decir \(p \mid 1\),
    lo que es absurdo.
  \end{proof}

  Incluso se puede demostrar más.
  El siguiente resultado se debe a Leonhard Euler,%
    \index{Euler, Leonhard}
  la brillante demostración siguiente es de Paul Erdős~%
    \cite{erdos38:_ueber_reihe_1/p}.%
    \index{Erdos, Paul@Erdős, Paul}
  La importancia del mismo radica en que la divergencia de la serie
  da un indicio de la tasa de crecimiento de los números primos.
  Como dijo el mismo Euler,
  ``hay más primos que cuadrados''
  (por el teorema~\ref{theo:Basilea-converge} la serie de
   recíprocos de los cuadrados converge).%
     \index{Basilea, problema de}
  \begin{theorem}
    \label{theo:sum-1/p}
    La serie
    \begin{equation*}
      \sum_p \frac{1}{p}
    \end{equation*}
    (donde la suma es sobre los números primos)
    diverge.
  \end{theorem}
  \begin{proof}
    Por contradicción.%
      \index{demostracion@demostración!contradiccion@contradicción}
    Enumeremos los primos como \(p_1, p_2, \dotsc\) en orden creciente.
    Si la serie converge,
    hay un punto a partir del cual la suma es menor que \(1/2\),
    con lo que podemos escribir:
    \begin{equation*}
      \sum_{i \ge k + 1} \frac{1}{p_i} < \frac{1}{2}
    \end{equation*}
    Llamaremos \emph{primos chicos} a \(p_1, p_2, \dotsc p_k\)
    y \emph{primos grandes} a \(p_{k + 1}, p_{k + 2}, \dotsc\).
    Tomemos algún \(N > p_k\) a determinar más adelante,
    y llamemos \(N_1\) a la cantidad de números hasta \(N\)
    divisibles solo por primos chicos,
    y similarmente \(N_2\) los que tienen divisores grandes.
    Debe ser \(N = N_1 + N_2\),
    pero dada la suposición de arriba
    veremos que podemos elegir \(N\)
    tal que al acotar \(N_1\) y \(N_2\) de forma suficientemente precisa
    concluimos que \(N_1 + N_2 < N\).

    Como \(\lfloor N / p \rfloor\) de los números entre 1 y \(N\)
    son divisibles por el primo \(p\)
    y \(\lfloor x \rfloor \le x\),
    tenemos que:
    \begin{align}
      N_2
	&\le \sum_{i \ge k + 1} \left\lfloor \frac{N}{p_i} \right\rfloor
	   \notag \\
	&\le \sum_{i \ge k + 1} \frac{N}{p_i}
	   \notag \\
	&=    N \sum_{i \ge k + 1} \frac{1}{p_i}
	   \notag \\
	&<  \frac{N}{2}
	   \label{eq:suma-primos-reciprocos-N2}
    \end{align}
    Consideremos ahora los números menores que \(N\)
    que solo tienen factores primos chicos.
    Si tomamos uno cualquiera de ellos y le llamamos \(x\),
    podemos escribir \mbox{\(x = y \cdot z^2\)},
    donde \(y\) no es divisible por el cuadrado de ningún primo.
    Obviamente \(z^2 \le N\),
    con lo que hay a lo más \(\sqrt{N}\) valores posibles de \(z\).
    Como hay \(k\) primos chicos,
    hay a lo más \(2^k\) posibles valores distintos de \(y\).
    Esto da una sobre estimación bastante burda,
    pero suficiente para nuestras necesidades presentes:
    \begin{equation}
      \label{eq:suma-primos-reciprocos-N1}
      N_1 \le 2^k \cdot \sqrt{N}
    \end{equation}
    Ahora queremos elegir \(N\) de forma que \(N_1 < N / 2\),
    o sea por la cota~\eqref{eq:suma-primos-reciprocos-N1}:
    \begin{align}
      2^k \cdot \sqrt{N}
	&< \frac{N}{2} \notag \\
      2^{2 k + 2}
	&< N
      \label{eq:suma-primos-reciprocos-N}
    \end{align}
    Combinando~\eqref{eq:suma-primos-reciprocos-N}
    con nuestra estimación~\eqref{eq:suma-primos-reciprocos-N2},
    para el valor elegido de \(N\) tenemos:
    \begin{equation*}
      N
	= N_1 + N_2
	  < \frac{N}{2} + \frac{N}{2}
	    = N
    \end{equation*}
    Esto es ridículo.
  \end{proof}
  Nuevamente,
  si la serie diverge no puede tener finitos términos
  y hay infinitos números primos.

\section{Congruencias}
\label{sec:congruencias}

  El concepto de \emph{congruencia} está íntimamente relacionado
  con el resto de la división,
  e incluso usan notaciones similares.
  De todas formas es importante distinguirlos.
  \begin{definition}
    \index{congruencia|textbfhy}
    \label{def:Z:congruence}
    Sean \(a, b, m \in \mathbb{Z}\),
    con \(m \ne 0\).
    Definimos:
    \begin{equation*}
      a \equiv b \pmod{m}
    \end{equation*}
    si \(m \mid a - b\).
    Esto se expresa diciendo
    que \emph{\(a\) es congruente con \(b\) módulo \(m\)}.
  \end{definition}
  Esta es una relación de equivalencia:%
    \index{relacion@relación!equivalencia}
  \begin{description}
  \item[Reflexiva:]
    Si \(a = b\),
    la definición se reduce a \(m \mid 0\),
    lo que siempre es cierto.
  \item[Simétrica:]
    \(a \equiv b \pmod{m}\) significa que \(m \mid a - b\),
    pero entonces \(m \mid b - a\),
    que es decir \(b \equiv a \pmod{m}\).
  \item[Transitiva:]
    \(a \equiv b \pmod{m}\) y \(b \equiv c \pmod{m}\)
    significan \(m \mid a - b\) y \(m \mid b - c\),
    que es \(a - b = k_1 m\) y \(b - c = k_2 m\);
    pero entonces \(a - c = (k_1 + k_2) m\),
    o sea \(m \mid a - c\),
    que es decir \(a \equiv c \pmod{m}\).
  \end{description}
  Los siguientes teoremas dan algunas propiedades importantes.

  \begin{theorem}
    \index{congruencia!propiedades}
    \label{theo:+*mod}
    Sea \(m\) un entero positivo
    y sean \(x_1, x_2, y_1, y_2\) enteros tales que:
    \begin{align*}
      x_1
	&\equiv x_2 \pmod{m} \\
      y_1
	&\equiv y_2 \pmod{m}
    \end{align*}
    Entonces:
    \begin{align*}
      x_1 + y_1
	&\equiv x_2 + y_2 \pmod{m} \\
      x_1 \cdot y_1
	&\equiv x_2 \cdot y_2 \pmod{m} \\
    \end{align*}
  \end{theorem}
  \begin{proof}
    Nos dieron que
    \(x_1 - x_2 = m a\),
    \(y_1 - y_2 = m b\),
    para algunos \(x, y \in \mathbb{Z}\).
    Entonces para la suma
    \begin{align*}
      (x_1 + y_1) - (x_2 + y_2)
	&= (x_1 - x_2) + (y_1 - y_2) \\
	&= m a + m b \\
	&= m (a + b)
    \end{align*}
    lo que es decir
    \begin{equation*}
      x_1 + y_1
	\equiv x_2 + y_2 \pmod{m}
    \end{equation*}

    Similarmente,
    para el producto:
    \begin{align*}
      x_1 y_1 - x_2 y_2
	&= x_1 y_1 - x_2 y_1 + x_2 y_1 - x_2 y_2 \\
	&= (x_1 - x_2) y_1 + x_2 (y_1 -y_2) \\
	&= m a y_1 + x_2 m b \\
	&= m (a y_1 + b x_2)
    \end{align*}
    con lo que
    \begin{equation*}
      x_1 \cdot y_1 \equiv x_2 \cdot y_2 \pmod{m}
      \qedhere
    \end{equation*}
  \end{proof}

  \begin{theorem}
    \index{congruencia!propiedades}
    \label{theo:congruencia-mn}
    Sean \(m\), \(n\) enteros positivos.
    Entonces,
    si
    \begin{equation*}
      a \equiv b \pmod{m n}
    \end{equation*}
    tenemos
    \begin{align*}
      a &\equiv b \pmod{m} \\
      a &\equiv b \pmod{n}
    \end{align*}
    Además,
    si
    \begin{align*}
      a &\equiv b \pmod{m} \\
      a &\equiv b \pmod{n}
    \end{align*}
    entonces
    \begin{equation*}
      a \equiv b \pmod{\lcm(m, n)}
    \end{equation*}
    En particular,
    si \(\gcd(m, n) = 1\),
    entonces:
    \begin{equation*}
      a \equiv b \pmod{m n}
    \end{equation*}
  \end{theorem}
  \begin{proof}
    Para la primera aseveración,
    tenemos por definición que \(m n \mid a - b\),
    pero en tal caso \(m \mid a - b\)
    y también \(n \mid a - b\).

    Para la segunda,
    tenemos que:
    \begin{align*}
      m &\mid a - b \\
      n &\mid a - b
    \end{align*}
    Como tanto \(m\) y \(n\) dividen a \(a - b\),
    el mínimo común múltiplo lo divide:
    \begin{equation*}
      \lcm(m, n) \mid a - b
    \end{equation*}
    y obtenemos nuestro resultado.
  \end{proof}

% Fixme: ¿Más propiedades de congruencias? ~~-> TAoCP

\section[Aritmética \texorpdfstring{en $\mathbb{Z}_m$}{con congruencias}]
	{\protect\boldmath
	   Aritmética \texorpdfstring{en $\mathbb{Z}_m$}{con congruencias}%
       \protect\unboldmath}
\label{sec:aritmetica-Zm}

  Para cualquier entero \(x\) y un entero positivo \(m\)
  anotaremos \([x]_m\) para la clase de equivalencia de \(x\)
  en la relación de congruencia módulo \(m\).
  Vale decir,
  \([x]_m\) es el conjunto de todos los enteros \(x'\)
  tales que \(x - x'\) es múltiplo de \(m\):
  \begin{align*}
    [5]_3
      &= \{\dotsc, -4, -1, 2, 5, 8, \dotsc\} \\
    [-3]_7
      &= \{\dotsc, -10, -3, 4, 11, \dotsc\}
  \end{align*}
  Sabemos que las clases de equivalencia particionan \(\mathbb{Z}\).
  Por ejemplo:
  \begin{equation*}
    \mathbb{Z} = [0]_3 \cup [1]_3 \cup [2]_3
  \end{equation*}
  Para cualquier \(m\) dado,
  las clases son \([0]_m\), \([1]_m\), \ldots, \([m - 1]_m\),
  lo que sigue del algoritmo de división,
  ya que cualquier entero \(x\) puede expresarse como:
  \begin{equation*}
    x = q \cdot m + r
  \end{equation*}
  con \(0 \le r < m\),
  y \(x \in [r]_m\) en tal caso.
  Esto motiva:
  \begin{definition}
    \label{def:Zm}
    Sea \(m\) un entero positivo.
    El \emph{conjunto de enteros módulo \(m\)},
    anotado \(\mathbb{Z}_m\),
    es el conjunto de las clases de equivalencia \([x]_m\).
  \end{definition}
  Vale decir,
  \(\mathbb{Z}_m = \{[0]_m, [1]_m, \dotsc, [m - 1]_m\}\).

  Cabe enfatizar que los elementos de \(\mathbb{Z}_m\)
  son subconjuntos de \(\mathbb{Z}\),
  pero muchas veces resulta conveniente
  considerarlos como los enteros \(0, 1, \dotsc, m - 1\)
  (aunque podríamos elegir otro conjunto de representantes
   si resulta conveniente)
  con una estructura aritmética diferente.
  A los elementos de \(\mathbb{Z}_m\) se les suele llamar \emph{residuos}
  (módulo \(m\)).%
    \index{residuo}

  Definimos las operaciones:
  \begin{align*}
    [x]_m \oplus [y]_m
      &= [x + y]_m \\
    [x]_m \odot	 [y]_m
      &= [x \cdot y]_m
  \end{align*}
  Por nuestro teorema~\ref{theo:+*mod}
  estas operaciones están bien definidas.
  \begin{theorem}
    \label{theo:mod-rules}
    Sea \(m\) un entero positivo,
    y sean \(a, b, c \in \mathbb{Z}_m\) cualquiera.
    Para simplificar anotaremos \(0 = [0]_m\), \(1 = [1]_m\).
    Entonces:
    \begin{enumerate}[label=\textbf{G\arabic{*}:}, ref=G\arabic{*}]
    \item\label{G:asociativa}
      \(a \oplus (b \oplus c) = (a \oplus b) \oplus c\)
    \item\label{G:neutro}
      Hay\/ \(0 \in \mathbb{Z}_m\)
      tal que para todo \(a \in \mathbb{Z}_m\)
      se cumple \(a \oplus 0 = a\)
    \item\label{G:inverso}
      Para todo \(a \in \mathbb{Z}_m\) existe \(-a \in \mathbb{Z}_m\)
      tal que \(a \oplus (-a) = 0\)
    \item\label{G:conmutativa}
      \(a \oplus b = b \oplus a\)
    \end{enumerate}
    \begin{enumerate}[label=\textbf{R\arabic{*}:}, ref=R\arabic{*}]
    \item\label{R:asociativa}
      \((a \odot b) \odot c = a \odot (b \odot c)\)
    \item\label{R:distributiva}
      \(a \odot (b \oplus c) = (a \odot b) \oplus (a \odot c)\)
      y
      \((a \oplus b) \odot c = (a \odot c) \oplus (b \odot c)\)
    \item\label{R:neutro}
      Hay\/ \(1 \in \mathbb{Z}_m\)
      tal que para todo \(a \in \mathbb{Z}_m\)
      se cumple \(a \odot 1 = 1 \odot a = a\)
    \item\label{R:conmutativa}
      \(a \odot b = b \odot a\)
    \end{enumerate}
  \end{theorem}
  Hay una curiosa asimetría entre~\ref{G:neutro} y~\ref{R:neutro},
  generalmente se indica que \(a \oplus 0 = 0 \oplus a = a\).
  Demostraremos esto en el teorema~\ref{theo:group-two-sided-identity}.
  Asimismo,
  suele agregarse que \(a \oplus (-a) = (-a) \oplus a = 0\),
  lo que también sigue de las anteriores
  (teorema~\ref{theo:group-mutuality}).
  \begin{proof}
    Demostraremos algunas cosas,
    dejamos el resto a la imaginación del lector.

    Para \ref{G:conmutativa},
    sean \(a = [x]_m\) y \(b = [y]_m\).
    Entonces:
    \begin{equation*}
      \begin{array}{l@{{}={}}ll}
	a \oplus b
	  & [x]_m \oplus [y]_m & \\
	  & [x + y]_m	       & \text{(definición de \(\oplus\))} \\
	  & [y + x]_m	       & \text{(en \(\mathbb{Z}\))} \\
	  & [y]_m \oplus [x]_m & \text{(definición de \(\oplus\))} \\
	  & b \oplus a
      \end{array}
    \end{equation*}

    Para \ref{G:inverso},
    tomamos \(-[x]_m = [-x]_m\),
    que cumple lo solicitado.
  \end{proof}
  Si un conjunto \(G\) con una operación \(\oplus\) cumple~\ref{G:asociativa}
  a~\ref{G:inverso}
  para algún elemento \(0\)
  se le llama \emph{grupo},%
    \index{grupo|textbfhy}%
    \index{axioma!grupo}
  y se anota \(\langle G, \oplus \rangle\).
  Un grupo que cumple~\ref{G:conmutativa}
  es un \emph{grupo conmutativo} o \emph{abeliano}.%
    \index{grupo!abeliano|textbfhy}%
    \index{grupo!conmutativo|see{grupo!abeliano}}%
    \index{axioma!grupo!abeliano}
  Si es un grupo abeliano con \(\oplus\)
  y hay una segunda operación \(\odot\) y un elemento \(1\)
  que cumple con~\ref{R:asociativa} hasta~\ref{R:neutro}
  se le llama \emph{anillo},%
    \index{anillo|textbfhy}%
    \index{axioma!anillo}
  y se anota \(\langle G, \oplus, \odot \rangle\).
  Si además cumple~\ref{R:conmutativa}
  es un \emph{anillo conmutativo}.%
    \index{anillo!conmutativo|textbfhy}%
    \index{axioma!anillo!conmutativo}
  Nuestro teorema afirma entonces que \(\mathbb{Z}_m\)
  con las operaciones \(\oplus\) y \(\odot\) es un anillo conmutativo.
  Cuidado,
  hay quienes omiten el axioma~\ref{R:neutro} al definir anillos,
  y le llaman \emph{anillos con unidad} a lo que nosotros llamamos anillos.%
    \index{anillo!con unidad}
  En inglés algunos usan el nombre \emph{\foreignlanguage{english}{ring}}%
    \index{ring@\emph{\foreignlanguage{english}{ring}}|see{anillo}}
  para nuestros anillos,
  y usan \emph{\foreignlanguage{english}{rng}}
  (por \emph{\foreignlanguage{english}{ring}} sin identidad).
    \index{rng@\emph{\foreignlanguage{english}{rng}}|see{anillo}}
  Asimismo,
  exigiremos que \(0 \ne 1\),
  para evitar los casos especiales
  que produce un anillo con un único elemento.
  Para abreviar,
  generalmente solo se nombra el conjunto,
  omitiendo las operaciones como obvias.
  Así,
  en vez de hablar del grupo \(\langle \mathbb{Z}_m, \oplus \rangle\)
  o de los anillos \(\langle \mathbb{Z}_m, \oplus, \odot \rangle\)
  o \(\langle \mathbb{C}, +, \cdot \rangle\)
  se habla simplemente de \(\mathbb{Z}_m\) o \(\mathbb{C}\),
  subentendiendo las operaciones tradicionales.
  Esto no debiera llevar a confusión.
  Temas como este son el ámbito del álgebra abstracta,%
    \index{algebra abstracta@álgebra abstracta}
  para profundizar en ellos se recomienda el texto de Pinter~%
    \cite{pinter10:_book_abstr_algeb}
  o el de Judson~\cite{judson14:_abstr_algeb}.

  En el caso de grupos abelianos%
    \index{grupo!abeliano}
  por convención se le llama o anota como ``suma'' a la operación,
  y al neutro se le designa con \(0\).
  Al inverso \(-a\)  de la ``suma''
  se le llama \emph{inverso aditivo} en tal caso.%
    \index{operacion@operación!inverso}
  Escribiremos \(a \ominus b = a \oplus (-b)\) para grupos abelianos.
  Por convención se anota como ``multiplicación''
  la operación en grupos generales,
  al neutro se le llama \(1\)
  y el inverso de \(a\) en tal caso se anota \(a^{-1}\).

  Los elementos que llamamos \(0\) y \(1\) arriba
  (axiomas~\ref{G:neutro} y~\ref{R:neutro})%
    \index{operacion@operación!elemento neutro}
  no son necesariamente los enteros cero y uno,
  simplemente los usamos como nombres convenientes.
  Nótese que en el caso de grupos y anillos las operaciones son esenciales,
  debiera anotarse \((R, +)\) y \((R, +, \cdot)\)
  para explicitar las operaciones.
  Los elementos especiales resultan de las operaciones,
  no vale la pena nombrarlos.
  En anillos se anota \(-a\) para el inverso aditivo de \(a\)
  y \(a^{-1}\) para su inverso multiplicativo
  (si existe).

  Generalmente se indica únicamente el conjunto.

  Al número de elementos de un grupo o de un anillo \(R\)
  se le llama su \emph{orden},%
    \index{grupo!orden|textbfhy}%
    \index{anillo!orden|textbfhy}
  anotado \(\lvert R \rvert\).
  Si un subconjunto del grupo es a su vez un grupo con la operación del caso
  se le llama \emph{subgrupo}.%
    \index{subgrupo|textbfhy}
  De la misma forma podemos definir \emph{subanillos},%
    \index{subanillo|textbfhy}
  claro que insistiremos en que incluyan a \(1\)
  para evitar que el subanillo tenga estructura incompatible con el anillo.
  Por ejemplo,
  en \(\mathbb{Z}_6\),
  el subconjunto \(\{0, 3\}\) es cerrado respecto de suma y multiplicación,
  con neutro multiplicativo \(3\).
  Pero en \(\mathbb{Z}_6\) \(3\) ni siquiera tiene inverso multiplicativo.
  .
  Si \(A\) es subgrupo (o subanillo) de \(B\),
  se anota \(A \le B\).

  Un caso es el conjunto de un único elemento \(e\),
  y definimos \(e = e \circ e\).
  Por razones obvias se le llama el \emph{grupo trivial}.%
    \index{grupo!trivial}

  Otro grupo es el de las operaciones en el cubo de Rubik,%
    \index{Rubik, cubo de (grupo)}
  que intercambian las posiciones de las caras de colores de los cubitos.
  Joyner~\cite{joyner08:_adven_group_theory}
  usa este juguete como excusa para introducir la teoría de grupos,
  y muestra cómo aplicarla para resolver este puzzle
  y otros similares.

  Un ejemplo importante lo proveen
  las simetrías de objetos al moverse en el espacio.%
    \index{grupo!simetrias@simetrías}
  Por ejemplo,
  si consideramos un cuadrado
  \begin{figure}[htbp]
    \centering
    \pgfimage{images/cuadrado}
    \caption{Un cuadrado}
    \label{fig:simetria-cuadrado}
  \end{figure}
  (como el de la figura~\ref{fig:simetria-cuadrado}),
  tenemos las siguientes operaciones
  que lo hacen coincidir con su posición original:
  \begin{itemize}
  \item
    No hacer nada
    (identidad, \(\iota\)).
  \item
    Girar en sentido contra reloj en
    \(\pi / 2\), \(\pi\), \(3 \pi / 2\)
    (\(r_1\), \(r_2\) y	 \(r_3\)).
  \item
    Reflejar verticalmente
    (\(f_v\)),
    horizontalmente (\(f_h\)),
    y a través de las diagonales \(1\;3\) y \(2\;4\)
    (\(f_c\) y \(f_d\)).
  \end{itemize}
  Podemos componer estas operaciones vía efectuar una y luego la otra.%
    \index{grupo!simetrias@simetrías}
  En otras palabras,
  \(b \bullet a\) describe aplicar \(a\) y luego \(b\).
  Como una combinación de operaciones
  vuelve el cuadrado a su posición original
  (aunque intercambiando vértices),
  la composición es cerrada.
  Por ejemplo,
  si rotamos el cuadrado en \(3 \pi / 2\)
  y luego lo reflejamos en la horizontal
  el efecto es el mismo que si lo reflejamos en una diagonal:
  \begin{align*}
    f_h \bullet r_3 = f_d
  \end{align*}
  Esto lleva al cuadro~\ref{tab:D8},
  que describe al grupo conocido como \(\mathtt{D}_8\).
  \begin{table}[htbp]
    \centering
    \begin{tabular}{>{\(}c<{\)}|*{8}{>{\(}c<{\)}}}
      \bullet & \iota & r_1 & r_2 & r_3 & f_v & f_h & f_d & f_c \\
      \hline
	\rule[-0.7ex]{0pt}{3ex}%
      \iota & \iota & r_1   & r_2   & r_3   & f_v   & f_h   & f_d   & f_c \\
      r_1   & r_1   & r_2   & r_3   & \iota & f_v   & f_h   & f_d   & f_c \\
      r_2   & r_2   & r_3   & \iota & r_1   & f_c   & f_d   & f_v   & f_h \\
      r_3   & r_3   & \iota & r_1   & r_2   & f_d   & f_c   & f_h   & f_v \\
      f_v   & f_v   & f_d   & f_h   & f_c   & \iota & r_2   & r_1   & r_3 \\
      f_h   & f_h   & f_c   & f_v   & f_d   & r_2   & \iota & r_3   & r_1 \\
      f_d   & f_d   & f_h   & f_c   & f_v   & r_3   & r_1   & \iota & r_2 \\
      f_c   & f_c   & f_v   & f_d   & f_h   & r_1   & r_3   & r_2   & \iota
    \end{tabular}
    \caption{El grupo $\mathtt{D}_8$}
    \label{tab:D8}
  \end{table}
  Es \(\mathtt{D}\) por \emph{diedro},%
    \index{grupo!dihedral|textbfhy}
  el \(8\) por el número de operaciones del grupo.
  En general,
  el polígono regular de \(n\) lados%
    \index{poligono@polígono!regular}
  da lugar al grupo denominado \(\mathtt{D}_{2 n}\),
  hay \(n\) rotaciones y \(n\) reflexiones
  para un total de \(2 n\) operaciones.
  Estos grupos no son conmutativos.

  Un poco más de la teoría elemental de grupos
  provee Chen~%
    \cite[capítulo 4]{chen08:_misc_topics_first_year_math}.
  En particular,
  construye sistemáticamente todos los grupos de orden hasta \(7\).
  Resultados importantes son los siguientes:
  \begin{theorem}
    \label{theo:group-unique-identity}
    El elemento neutro de un grupo es único.
  \end{theorem}
  \begin{proof}
    Supongamos que en el grupo \((G, \odot)\)
    los elementos \(a\) y \(b\) son ambos neutros.
    Entonces:
    \begin{equation*}
      a = a \odot b = b
      \qedhere
    \end{equation*}
  \end{proof}
  \begin{theorem}[Ley de cancelación]
    \index{grupo!cancelacion@cancelación}
    \label{theo:group-cancellation}
    Si \(a, b, c \in G\) son tales que
    \(a \odot c = b \odot c\),
    entonces \(a = b\).
  \end{theorem}
  \begin{proof}
    Tenemos:
    \begin{equation*}
      a
	= a \odot 1
	= a \odot (c \odot c^{-1})
	= (a \odot c) \odot c^{-1}
	= (b \odot c) \odot c^{-1}
	= b \odot (c \odot c^{-1})
	= b
      \qedhere
    \end{equation*}
  \end{proof}
  Puede aplicarse exactamente la misma técnica
  para demostrar cancelación a la izquierda.
  \begin{theorem}[Mutualidad]
    \index{grupo!mutualidad}
    \label{theo:group-mutuality}
    Si en un grupo \(a \odot b = 1\),
    entonces \(b \odot a = 1\).
  \end{theorem}
  \begin{proof}
    Escribamos:
    \begin{equation*}
      1 \odot b
	= b \odot 1
	= b \odot (a \odot b)
	= (b \odot a) \odot b
    \end{equation*}
    Luego aplicamos la ley de cancelación.
  \end{proof}
  Esto nos dice que \((a^{-1})^{-1} = a\).
  \begin{theorem}
    \label{theo:group-two-sided-identity}
    En un grupo,
    \(1 \odot a = a\)
  \end{theorem}
  \begin{proof}
    Podemos escribir:
    \begin{equation*}
      1 \odot a
	= (a \odot a^{-1}) \odot a
	= a \odot (a^{-1} \odot a)
	= a \odot 1
	= a
      \qedhere
    \end{equation*}
  \end{proof}
  \begin{theorem}
    \label{theo:group-inverse-unique}
    El inverso de un elemento es único.
  \end{theorem}
  \begin{proof}
    Sea \(a \in G\) un elemento cualquiera,
    y supongamos que se cumplen \(a \odot b = 1\) y \(a \odot c = 1\).
    Por cancelación por la izquierda,
    \(b = c\).
    Por mutualidad,
    si \(a \odot b = 1\) entonces \(b \odot a = 1\),
    y \(b = a^{-1}\).
  \end{proof}

  En lo que sigue consideraremos un grupo \(G\) con operación \(\odot\)
  (o simplemente se omite).
  El elemento neutro de \(G\) lo denotaremos \(1\).
  Para simplificar notación,
  en un grupo con operación multiplicación
  usaremos la definición para potencias enteras,%
    \index{grupo!potencia}
  donde \(a\) es un elemento cualquiera del grupo:
  \begin{align}
    a^n
      &= \underbrace{a \odot a \odot \dotsm \odot a}_{\text{\(n\) veces}}
	   \notag \\
  \intertext{Formalmente:}
    a^0
      &= 1 \label{eq:definicion-a^0} \\
    a^{k + 1}
      &= a^k \odot a \qquad \text{si \(k \ge 0\)}
	   \label{eq:definicion-a^k}
  \end{align}
  Es fácil ver que si definimos:
  \begin{equation}
    \label{definicion-a^-k}
    a^{-k}
      = \left( a^{-1} \right)^k
  \end{equation}
  se cumplen las propiedades tradicionales de las potencias:
  \begin{align}
    a^{m + n}
      &= a^m \odot a^n \label{eq:propiedad-potencias-+} \\
    (a^m)^n
      &= a^{m n}       \label{eq:propiedad-potencias-*}
  \end{align}
  Si la operación es suma
  usamos la notación \(n \cdot a\)
  (o simplemente \(n a\))
  con \(n \in \mathbb{Z}\);
  si anotamos \(0\) para el neutro aditivo:
  \begin{align*}
    n \cdot a
      &= \underbrace{a \oplus a \oplus \dotsb \oplus a}_{\text{\(n\) veces}}
	   \notag \\
  \intertext{Formalmente:}
    0 \cdot a
      &= 0 \\
    (k + 1) \cdot a
      &= k \cdot a \oplus a \qquad \text{si \(k \ge 0\)} \\
    -k \cdot a
      &= k \cdot (-a)
  \end{align*}

  Sea \(a\) un elemento de un grupo \(G\),
  en el cual usamos notación de multiplicación.
  Los elementos \(a^k\) con \(k \in \mathbb{Z}\)
  forman un subgrupo abeliano de \(G\).%
    \index{subgrupo!generado}
  Si \(G\) es finito,
  el conjunto de los \(a^n\) para \(n \in \mathbb{N}\)
  tiene que contener repeticiones,
  ya que son todos elementos de \(G\).
  Si resulta que \(a^m = a^n\) con \(m > n\),
  tendremos \(a^{m - n} = 1\).
  El mínimo \(n > 0\) tal que \(a^n = 1\)
  (siempre existe si \(G\) es finito,
   aunque incluso hay grupos infinitos
   todos cuyos elementos son de orden finito)
  se llama el \emph{orden de \(a\)},%
    \index{grupo!orden de un elemento|textbfhy}
  que se anota \(\ord_G(a)\)
  (o simplemente \(\ord(a)\),
   si el grupo se subentiende).
  Más aún,
  si \(a^k = 1\),
  entonces \(\ord_G(a) \mid k\).
  Para demostrar esto,
  usamos el algoritmo de división.%
    \index{algoritmo de division@algoritmo de división}
  Con \(n = \ord_G(a)\)
  podemos escribir \(k = q n + r\),
  donde \(0 \le r < n\),
  y:
  \begin{align}
    a^k
      &= a^{q n + r} \notag \\
    1
      &= a^{q n} \odot a^r \notag \\
      &= (a^n)^q \odot a^r \notag \\
      &= a^r  \label{eq:a^k=r}
  \end{align}
  Como \(0 \le r < n\),
  por la definición de orden la única opción en~\eqref{eq:a^k=r} es \(r = 0\),
  y \(n \mid k\).

  \begin{definition}
    \index{grupo!ciclico@cíclico|textbfhy}
    \label{def:grupo-ciclico}
    Sea \(G\) un grupo.
    Si todos los elementos de \(G\) se pueden escribir como
    \(g^k\) para algún elemento \(g \in G\) y \(k \in \mathbb{Z}\)
    a \(G\) se le llama \emph{grupo cíclico},
    y a \(g\) se le llama
    \emph{generador} del grupo.%
      \index{grupo!generador}
  \end{definition}
  Si \(a\) tiene orden finito \(n\),
  los elementos \(1 = a^0, a, \dotsc, a^{n - 1}\)
  forman un grupo cíclico de orden \(n\),
  llamado \emph{el grupo generado por \(a\)},
  que se anota \(\langle a \rangle\).
  Tenemos \(a^{n - 1} = a^{-1}\),
  lo que completa la condición de subgrupo que dimos antes
  en el teorema~\ref{theo:sugroup}.

  Vamos ahora a anillos,
  que son las estructuras que realmente nos interesan acá.
  \begin{definition}
    \index{anillo!unidad|textbfhy}
    \label{def:ring-unit}
    En un anillo \(R\),
    si para \(r\) hay un \(x\) tal que \(r \odot x = x \odot r = 1\),
    se dice que \(r\) es \emph{invertible},
    que \(x\) es su \emph{inverso}
    y escribimos \(x = r^{-1}\).
    Nótese que también es \(x^{-1} = r\).
    A los elementos invertibles del anillo \(R\)
    se les llama \emph{unidades},
    y su conjunto se denota \(R^\times\).
  \end{definition}
  Notaciones alternativas para el grupo de unidades de \(R\) son
  \(U(R)\), \(R^*\) y \(E(R)\).

  \begin{theorem}
    \index{anillo!grupo de unidades}
    El conjunto de unidades de un anillo forma un grupo con la multiplicación.
  \end{theorem}
  \begin{proof}
    La multiplicación es asociativa en \(R\),
    así que lo es en el subconjunto \(R^\times\).
    Toda unidad tiene inverso por definición,
    y \(1\) siempre es una unidad.
    Si \(a\) y \(b\) son invertibles,
    entonces \(a \odot b\) tiene inverso \(b^{-1} \odot a^{-1}\),
    y la multiplicación es cerrada en \(R^\times\).
    Estas son las propiedades que definen a un grupo.
  \end{proof}

  Un caso especial muy importante es:
  \begin{definition}
    \index{campo (algebra)@campo (álgebra)|textbfhy}
    \index{axioma!campo}
    \label{def:field}
    \glossary{Campo}
	     {Anillo conmutativo en el cual
	      todos los elementos distintos de cero
	      son invertibles.}
    Un anillo conmutativo en el que todos los elementos distintos de \(0\)
    son invertibles se llama \emph{campo}.
  \end{definition}
  \noindent
  Ya habíamos encontrado este concepto al discutir \(\mathbb{R}\)
  en el capítulo~\ref{cha:numeros-reales}.

  Para simplificar la notación usaremos \(x\) con \(0 \le x \le m - 1\)
  para denotar al conjunto \([x]_m\),
  y usaremos \(+\) y \(\cdot\)
  (o simplemente omitiremos la multiplicación)
  en vez de \(\oplus\) y \(\odot\).
  Para eliminar paréntesis,
  usaremos la convención común de ``multiplicaciones antes de sumas''.
  El valor de \(m\) se debe indicar o quedar claro por el contexto.

  Nótese que en \(\mathbb{Z}_m\)
  no hay orden ni idea de ``positivo''.
  Como en \(\mathbb{Z}_m\) es
  \begin{equation*}
    0, 1, \dotsc, m - 1, 0, 1, \dotsc
  \end{equation*}
  (al llegar al final comienza nuevamente)
  a veces se le llama ``aritmética de reloj''.%
    \index{aritmetica de reloj@aritmética de reloj|see{congruencias}}

  Algunas de las propiedades conocidas de \(\mathbb{Z}\)
  no siempre valen en \(\mathbb{Z}_m\).
  Por ejemplo,
  en \(\mathbb{Z}\)
  si \(a \cdot c = b \cdot c\) con \(c \ne 0\)
  entonces \(a = b\).
  Sin embargo,
  en \(\mathbb{Z}_{15}\)
  tenemos
  \begin{equation*}
    4 \cdot 9 =	 14 \cdot 9 = 6
  \end{equation*}
  e incluso
  \begin{equation*}
    12 \cdot 10 = 0
  \end{equation*}
  También pueden haber elementos que tienen más de dos raíces cuadradas.
  Por ejemplo:
  \begin{equation*}
    4 = 2 \cdot 2 = 7 \cdot 7 = 8 \cdot 8 = 13 \cdot 13
  \end{equation*}
  Nótese que módulo \(15\) tenemos \(13 \equiv -2\) y \(8 \equiv -7\),
  con lo que las raíces cuadradas de \(4\) en \(\mathbb{Z}_{15}\)
  son \(\pm 2\) y \(\pm 7\),
  mientras en \(\mathbb{Z}\) solo están \(\pm 2\).
  Algunos elementos tienen inverso multiplicativo
  en \(\mathbb{Z}_{15}\),
  como:
  \begin{equation*}
    7 \cdot 13 = 8 \cdot 2 = 1
  \end{equation*}
  Otros no lo tienen,
  por ejemplo \(6\),
  como es fácil verificar revisando todos los productos \(6 \cdot k\)
  con \(0 \le k < 15\) módulo \(15\).
  Hay elementos con una única raíz cuadrada,
  como \(15 = 15 \cdot 15\) en \(\mathbb{Z}_{30}\),
  que incluso es su propia raíz cuadrada.
  \begin{definition}
    \index{anillo!divisor de cero|textbfhy}
    \label{def:ring-zero-divisor}
    Si en un anillo hay elementos \(a\), \(b\)
    tales que \(a \odot b = 0\)
    se les llama \emph{divisor izquierdo de cero} a \(a\)
    y \emph{divisor derecho de cero} a \(b\).
    A ambos se les llama \emph{divisores de cero}.
  \end{definition}
  \noindent
  Nótese que \(0\) siempre es un divisor de cero.

  Casos muy importantes de anillos son:
  \begin{definition}
    \index{dominio integral|textbfhy}
    \glossary{Dominio integral}
	     {Un anillo conmutativo sin divisores propios de cero.}
    \label{def:ID}
    Un anillo conmutativo en el cual sólo \(0\) es divisor de cero
    se llama \emph{dominio integral}.
  \end{definition}
  \noindent
  El ejemplo clásico de dominio integral es \(\mathbb{Z}\).
  En un dominio integral podemos cancelar en productos:
  \begin{theorem}
    \label{theo:ID-*-cancellation}
    En un dominio integral,
    si \(a x = b x\) con \(x \ne 0\)
    entonces \(a = b\).
  \end{theorem}
  \begin{proof}
    Tenemos:
    \begin{align*}
      a x
	&= b x \\
      a x - b x
	&= 0 \\
      (a - b) x
	&= 0
    \end{align*}
    Como \(x \ne 0\) es \(a - b = 0\),
    o sea \(a = b\).
  \end{proof}

  Veamos criterios para identificar subgrupos.
  \begin{theorem}
    \index{grupo!subgrupo}
    \label{theo:sugroup}
    Sea \(G\) un grupo,
    y \(H \subset G\).
    Si para todo \(a, b \in H\) es \(a \odot b^{-1} \in H\),
    \(H\) es un subgrupo de \(G\).
  \end{theorem}
  \begin{proof}
    Que \(H\) sea subgrupo de \(G\) significa simplemente
    que ese subconjunto es cerrado respecto de la operación e inversos
    (las demás propiedades se ``heredan'' del grupo),
    y que contiene al neutro \(e\).

    Si \(a \in H\),
    por hipótesis está \(a \odot a^{-1} = e\).
    Pero así también está \(e \odot a^{-1} = a^{-1}\).
    Con esto,
    si \(a, b \in H\),
    tenemos que \(a \odot (b^{-1})^{-1} = a \odot b \in H\),
    completando la demostración.
  \end{proof}
  Por ejemplo,
  podemos ver que en el grupo \(D_8\) mostrado en el cuadro~\ref{tab:D8}
  los elementos \(\{\iota, r_1, r_2, r_3\}\) forman un subgrupo.
  \begin{lemma}
    \label{lem:intersection-subgroups}
    Sean \(A\) y \(B\) subgrupos de \(G\).
    Entonces \(A \cap B\) es un subgrupo de \(G\).
  \end{lemma}
  \begin{proof}
    Supongamos \(x, y \in A \cap B\).
    Entonces \(y^{-1} \in A\) y también \(y^{-1} \in B\),
    por ser grupos.
    Pero entonces \(y^{-1} \in A \cap B\).
    De forma similar,
    \(x \cdot y^{-1} \in A \cap B\),
    y por el teorema~\ref{theo:sugroup} \(A \cap B\) es un subgrupo de \(G\).
  \end{proof}
  También para anillos:
  \begin{lemma}
    \label{lem:intersection-subrings}
    Sean \(A\) y \(B\) subanillos de \(R\).
    Entonces \(A \cap B\) es un subanillo de \(R\).
  \end{lemma}
  La demostración se omite,
  sigue la misma idea que la del lema~\ref{lem:intersection-subgroups}.

  Los axiomas de anillo tienen algunas consecuencias simples.
  \begin{theorem}
    \label{theo:ring-01}
    En un anillo,
    \(0\) y \(1\) son únicos,
    y para cada \(a\) hay un único \(-a\).
  \end{theorem}
  \begin{proof}
    Para la suma,
    el que \(0\) es único
    no es más que el teorema~\ref{theo:group-unique-identity};
    la misma técnica de la demostración puede aplicarse
    a \(1\) y la multiplicación.

    Que hay un único \(-a\) es el teorema~\ref{theo:group-inverse-unique}
    aplicado a la suma.
  \end{proof}

  Para evitar paréntesis,
  usaremos la convención tradicional
  que las operaciones son asociativas izquierdas,
  y que \(\odot\) tiene mayor precedencia que \(\oplus\).
  A veces anotaremos \(a \ominus b\) para \(a \oplus (-b)\).

  Otras consecuencias simples son:
  \begin{theorem}
    \label{theo:ring-0a}
    En un anillo,
    para todo \(a\) tenemos
    \(0 \odot a = a \odot 0 = 0\)
  \end{theorem}
  \begin{proof}
    Es aplicar los axiomas de anillo.
    Para \(0 \odot a\):
    \begin{align*}
      0 \odot a
	&= (0 \oplus 0) \odot a \\
	&= (0 \odot a) \oplus (0 \odot a)
    \end{align*}
    Sumando \(-(0 \odot a)\) a ambos lados obtenemos la conclusión buscada.
    La otra parte se demuestra de forma similar.
  \end{proof}
  De acá,
  si \(a\) es una unidad del anillo,
  entonces no es un divisor de cero,
  ya que si:
  \begin{align*}
    a \odot b
      &= 0 \\
    a^{-1} \odot (a \odot b)
      &= (a^{-1} \odot a) \odot b \\
      &= 1 \odot b \\
      &= b \\
  \intertext{Por el otro lado:}
    a^{-1} \odot 0
      &= 0
  \end{align*}
  O sea,
  \(b = 0\),
  y \(a\) no es divisor de cero.
  \begin{theorem}
    \label{theo:ring-(a*-b)}
    En un anillo,
    tenemos \((-a) \odot b = a \odot (-b) = -(a \odot b)\).
  \end{theorem}
  \begin{proof}
    Primero demostramos \((-1) \odot a = a \odot (-1) = -a\).
    \begin{align*}
      (-1) \odot a \oplus a
	&= (-1) \odot a \oplus 1 \odot a \\
	&= ((-1) \oplus 1) \odot a \\
	&= 0 \odot a \\
	&= 0
    \end{align*}
    Por conmutatividad de la suma en un anillo
    \(a \oplus ((-1) \odot a) = 0\),
    y en conjunto estas dos corresponden precisamente
    a la definición de \(-a\) en un grupo.
    La otra parte se demuestra de forma afín.

    Usando esto:
    \begin{align*}
      (-a) \odot b
	&= ((-1) \odot a) \odot b \\
	&= (-1) \odot (a \odot b) \\
	&= -(a \odot b)
    \end{align*}
    La otra parte se demuestra de la misma forma.
  \end{proof}
  Tenemos también:
  \begin{theorem}
    \label{theo:finite-ring-units}
    En un anillo finito \(R\),
    los elementos son unidades o divisores de cero.
  \end{theorem}
  \begin{proof}
    Tomemos \(a \ne 0\) del anillo,
    y consideremos el conjunto \(a R = \{a \odot x \colon x \in R\}\).
    Supongamos que hay elementos repetidos en \(a R\),
    digamos \(a \odot x_1 = a \odot x_2\) con \(x_1 \ne x_2\).
    Entonces:
    \begin{align*}
      a \odot x_1
	&= a \odot x_2 \\
      a \odot x_1 \ominus a \odot x_2
	&= 0 \\
      a \odot (x_1 \ominus x_2)
	&= 0
    \end{align*}
    Como \(x_1 \ominus x_2 \ne 0\),
    \(a\) es un divisor de cero.

    Si no hay elementos repetidos en \(a R\),
    como hay \(\lvert R \rvert\) elementos en ese conjunto
    tiene que estar 1,
    digamos \(a b = 1\).
    Entonces:
    \begin{align*}
      a \odot b
	&= 1 \\
      (a \odot b) \odot a
	&= a \\
      a \odot (b \odot a) \ominus a
	&= 0 \\
      a \odot ((b \odot a) \ominus 1)
	&= 0
    \end{align*}
    Sabemos que \(a \odot 0 = 0\),
    como en \(a R\) no hay elementos repetidos es:
    \begin{align*}
      b \odot a \ominus 1
	&= 0 \\
      b \odot a
	&= 1
    \end{align*}
    y \(b\) es el inverso de \(a\),
    \(a\) es un una unidad.
  \end{proof}
  Una consecuencia inmediata del teorema~\ref{theo:finite-ring-units}
  es:
  \begin{corollary}
    \label{cor:finite-ID=field}
    Todo dominio integral finito es un campo.
  \end{corollary}

  Ejemplos adicionales de anillo los ponen los enteros \(\mathbb{Z}\),%
    \index{Z (numeros enteros)@\(\mathbb{Z}\) (números enteros)}%
    \index{anillo}
  los complejos \(\mathbb{C}\)
    \index{C (numeros complejos)@\(\mathbb{C}\) (números complejos)}
  y matrices cuadradas sobre \(\mathbb{R}\).
    \index{anillo!matrices}

% curvas-elipticas.tex
%
% Copyright (c) 2012-2014 Horst H. von Brand
% Derechos reservados. Vea COPYRIGHT para detalles

\subsection{Curvas elípticas}
\label{sec:curvas-elipticas}
\index{curva eliptica@curva elíptica|textbfhy}

  Una \emph{curva elíptica}
  está definida por una ecuación de la forma:
  \begin{equation*}
    \label{eq:elliptic-curve}
    y^2
      = x^3 + a x + b
  \end{equation*}
  \begin{figure}
    \centering
    \subfloat[\(y^2 = x^3 - x\)]{
      \pgfimage{images/curva-eliptica-1}
    }%
    \hspace{1em}%
    \subfloat[\(y^2 = x^3 - x + 1\)]{
      \pgfimage{images/curva-eliptica-2}
    }
    \caption{Curvas elípticas}
    \label{fig:curvas-elipticas}
  \end{figure}
  que no tiene puntos aislados,
  no se intersecta a sí misma y no tiene cuernos.
  Algebraicamente,
  el discriminante \(\Delta = - 16 (4 a^3 + 27 b^2) \ne 0\).%
    \index{discriminante}
  El gráfico de la curva tiene dos componentes si \(\Delta > 0\)
  y uno solo si \(\Delta < 0\),
  ver la figura~\ref{fig:curvas-elipticas}.
  El nombre no tiene relación con la forma de la curva,
  sino con el hecho que se requieren funciones elípticas
  para representarlas paramétricamente.

  \begin{figure}
    \centering
    \subfloat[Suma de \(\mathtt{P}_1 = (x_1, y_1)\)
	      y \(\mathtt{P}_2 = (x_2, y_2)\)]{
      \pgfimage{images/suma-curva-eliptica}
      \label{subfig:suma-curva-eliptica}
    }%
    \hspace{1em}%
    \subfloat[Doble de \(\mathtt{P} = (x, y)\)]{
      \pgfimage{images/doble-curva-eliptica}
      \label{subfig:doble-curva-eliptica}
    }
    \caption{Sumas en curvas elípticas}
    \label{fig:sumas-curva-eliptica}
  \end{figure}
  Dados dos puntos \(\mathtt{P}_1 = (x_1, y_1)\)
  y \(\mathtt{P}_2 = (x_2, y_2)\) sobre una curva elíptica
  podemos definir la suma
  como el punto donde la recta entre los puntos corta la curva
  reflejado en el eje \(x\),
  véase~\ref{subfig:suma-curva-eliptica} para un ejemplo.
  Esto hace que si \(\mathtt{P}_1\),
  \(\mathtt{P}_2\) y \(\mathtt{P}_3\)
  son puntos sobre la curva,
  es \(\mathtt{P}_1 + \mathtt{P}_2 + \mathtt{P}_3 = 0\),
  donde 0 es el punto en el infinito.
  En caso que \(x_1 = x_2\)
  hay dos posibilidades:
  Si \(y_1 = - y_2\),
  (incluyendo el caso en que los puntos coinciden),
  la suma se define como \(0\)
  (el punto en el infinito).
  Tenemos así para \(\mathtt{P} = (x, y)\)
  que \(- \mathtt{P} = (x, -y)\).
  En caso contrario
  definimos \(\mathtt{P}_1 + \mathtt{P}_2 = \mathtt{P}_3\)
  con \(\mathtt{P}_3 = (x_3, y_3)\)
  mediante:
  \begin{align}
    s
      &= \frac{y_2 - y_1}{x_2 - x_1} \notag \\
    x_3
      &= s^2 - x_1 - x_2 \label{eq:suma-curva-eliptica} \\
    y_3
      &= y_1 + s (x_3 - x_1) \notag
  \end{align}
  Para sumar el punto \(\mathtt{P} = (x, y)\) consigo mismo
  corresponde usar la tangente a la curva,
  ver la figura~\ref{subfig:doble-curva-eliptica},
  lo que da \(\mathtt{P}_2 = (x_2, y_2)\):
  \begin{align}
    s
      &= \frac{3 x + a}{2 y} \notag \\
    x_2
      &= s^2 - 2 x \label{eq:doble-curva-eliptica} \\
    y_2
      &= y + s (x_2 - x) \notag
  \end{align}
  Es rutina verificar que esto define un grupo abeliano.%
    \index{curva eliptica@curva elíptica!grupo}%
    \index{grupo!abeliano}

  Lo interesante es que las relaciones
  que definen la suma en curvas elípticas
  valen en cualquier campo,
  por lo que podemos considerar
  el grupo definido por la curva elíptica sobre un campo cualquiera.
  Si la característica del campo \(F\) no es \(2\) ni \(3\)
  (vale decir,
   no es \(2 x = 0\) ni \(3 x = 0\) para todo \(x \in F\);
   la discusión formal deberá esperar
   al capítulo~\ref{cha:campos-finitos}),
  toda curva elíptica puede escribirse en la forma:
  \begin{equation*}
    y^2
      = x^3 - p x - q
  \end{equation*}
  tal que el lado derecho no tiene ceros repetidos.
  Interesan los puntos con coordenadas en \(F\).
  El teorema de Hasse~%
    \cite{hasse36:_EC-I, hasse36:_EC-II, hasse36:_EC-III}
  da las cotas para el número \(N\) de elementos en curvas elípticas
  sobre el campo finito de \(q\) elementos:
  \begin{equation*}
    \lvert N - (q + 1) \rvert
      \le 2 \sqrt{q}
  \end{equation*}

  Las curvas elípticas son importantes en teoría de números,
  y tienen bastantes aplicaciones prácticas,
  particularmente se están haciendo muy populares en criptografía.%
    \index{criptografia@criptografía}%
  El sistema \texttt{PARI/GP}~\cite{PARI:2.7.2}%
    \index{PARI/GP@\texttt{PARI/GP}}
  incluye soporte para operar en los grupos respectivos.
  El paquete GAP~%
    \cite{GAP:4.7.5}%
    \index{GAP@\texttt{GAP}}
  tiene extenso soporte para trabajar con grupos,
  incluyendo grupos de curvas elípticas.

%%% Local Variables:
%%% mode: latex
%%% TeX-master: "clases"
%%% End:


\subsection{Anillos cuadráticos}
\label{sec:anillos-cuadraticos}

  Un ejemplo menos conocido de anillo conmutativo lo pone
  \(\mathbb{Z}[\sqrt{2}]\),%
    \index{anillo!cuadratico@cuadrático}
  definido como el conjunto \(\{a + b \sqrt{2} \colon a, b \in \mathbb{Z}\}\),
  con las operaciones tradicionales de los reales.
  Primeramente,
  las operaciones en \(\mathbb{Z}[\sqrt{2}]\) están bien definidas:
  \begin{align*}
    (a_1 + b_1 \sqrt{2}) + (a_2 + b_2 \sqrt{2})
      &= (a_1 + a_2) + (b_1 + b_2) \sqrt{2} \\
    (a_1 + b_1 \sqrt{2}) \cdot (a_2 + b_2 \sqrt{2})
      &= (a_1 a_2 + 2 b_1 b_2) + (a_1 b_2 + a_2 b_1) \sqrt{2}
  \end{align*}
  Los coeficientes en estas expresiones son todos enteros,
  y al ser \(\sqrt{2}\) irracional no hay maneras alternativas%
    \index{numero@número!irracional!\(\sqrt{2}\)}
  de representar el mismo elemento.
  Como los elementos son simplemente números reales,
  las asociatividades y conmutatividades de las operaciones,
  y la distributividad,
  se ``heredan'' de  \(\mathbb{R}\).%
    \index{R (números reales)@\(\mathbb{R}\) (números reales)}
  Podemos representar:
  \begin{align*}
    0 &= 0 + 0 \cdot \sqrt{2} \\
    1 &= 1 + 0 \cdot \sqrt{2}
  \end{align*}
  Tenemos
  \begin{equation*}
    \left(a + b \sqrt{2}\right) + \left((-a) + (-b) \sqrt{2}\right)
      = 0
  \end{equation*}
  lo que provee de un inverso aditivo.
  Como este es un subanillo de los reales,
  no hay divisores de cero salvo \(0\),
  y es un dominio integral.%
    \index{dominio integral}

  Busquemos el inverso de \(a + b \sqrt{2}\) en \(\mathbb{Z}[\sqrt{2}]\):
  \begin{equation}
    \label{eq:inverso-Z[sqrt(2)]-1}
    \frac{1}{a + b \sqrt{2}}
      = \frac{a - b \sqrt{2}}{a^2 - 2 b^2}
  \end{equation}
  Para que~\eqref{eq:inverso-Z[sqrt(2)]-1} pertenezca a nuestro anillo,
  debe ser:
  \begin{equation}
    \label{eq:inverso-Z[sqrt(2)]-2}
    a^2 - 2 b^2
      = \pm 1
  \end{equation}
  Una solución es \(a = b = 1\),
  con lo que \(1 + \sqrt{2}\) es una unidad.

  A ecuaciones de la forma
  \begin{equation}
    \label{eq:Pell}
    x^2 - d y^2
      = 1
  \end{equation}
  con \(d > 1\) se les llama \emph{ecuaciones de Pell}%
    \footnote{Euler%
		\index{Euler, Leonhard}
	      erróneamente se la atribuyó
	      a John Pell~(1611--1685),%
		\index{Pell, John}
	      probablemente confundiéndolo con
	      William Brounker~(1620--1684)%
		\index{Brounker, William}
	      quien fue el primer europeo en estudiarla.
	      Brahmagupta%
		\index{Brahmagupta}
	      en la India en 628 ya describe cómo resolverla.
	      Los números de Pell%
		\index{Pell, numeros de@Pell, números de}
	      (soluciones para el caso \(d = 2\))
	      se conocen desde Pitágoras.%
		\index{Pitagoras@Pitágoras}},%
    \index{Pell, ecuacion de@Pell, ecuación de}
  discutimos el caso \(d = 2\) más arriba.
  La discusión siguiente sigue esencialmente a Djukić~%
      \cite{djukic07:_Pell_equation}.

  Vemos que si \(d\) es un cuadrado perfecto,
    \index{cuadrado perfecto}
  solo es posible la \emph{solución trivial} \(x = 1\), \(y = 0\).
    \index{Pell, ecuacion de@Pell, ecuación de!solucion trivial@solución trivial}
  Enseguida,
  podemos suponer que \(x\) e \(y\) son no negativos,
  por los cuadrados sus signos no importan.
  Podemos factorizar el lado derecho de la ecuación~\eqref{eq:Pell}
  \begin{equation*}
    x^2 - d y^2
      = (x + y \sqrt{d}) (x - y \sqrt{d})
  \end{equation*}
  lo que nos lleva de vuelta al anillo \(\mathbb{Z}[\sqrt{d}]\).%
    \index{anillo!cuadratico@cuadrático}

  \begin{definition}
    \index{anillo!cuadratico@cuadrático!conjugado}
    \index{anillo!cuadratico@cuadrático!norma}
    En el anillo \(\mathbb{Z}[\sqrt{d}]\)
    el \emph{conjugado} de \(z = a + b \sqrt{d}\)
    es \(\overline{z} = a - b \sqrt{d}\),
    y su \emph{norma} es \(N(z) = z \cdot \overline{z} = a^2 - d b^2\).
    Llamamos \emph{parte entera} a \(a\)
    y \emph{parte irracional} a \(b\).
  \end{definition}
  En estos términos,
  son unidades de \(\mathbb{Z}[\sqrt{d}]\)
  exactamente los elementos de norma~\(\pm 1\).
  El inverso de la unidad \(z\) es \(\pm \overline{z}\),
  dependiendo del signo de \(N(z)\).
  Resulta:
  \begin{theorem}
    En \(\mathbb{Z}[\sqrt{d}]\)
    la norma y el conjugado son multiplicativos,
    o sea \(N(z_1 z_2) = N(z_1) \cdot N(z_2)\)
    y \(\overline{z_1 z_2} = \overline{z_1} \cdot \overline{z_2}\)
  \end{theorem}
  \begin{proof}
    Primeramente,
    con \(z_1 = a_1 + b_1 \sqrt{d}\) y \(z_2 = a_2 + b_2 \sqrt{d}\),
    tenemos:
    \begin{align}
      \overline{z_1} \cdot \overline{z_2}
	&= (a_1 - b_1 \sqrt{d}) \cdot (a_2 - b_2 \sqrt{d}) \notag \\
	&= (a_1 a_2 + d b_1 b_2) - (a_1 b_2 + a_2 b_1) \sqrt{d} \notag \\
	&= \overline{z_1 z_2} \label{eq:conjugado-producto}
    \end{align}
    Con esto:
    \begin{align}
      N(z_1 z_2)
	&= (z_1 z_2) \cdot (\overline{z_1 z_2}) \notag \\
	&= (z_1 \overline{z_1}) \cdot (z_2 \overline{z_2}) \notag \\
	&= N(z_1) \cdot N(z_2) \label{eq:norma-producto}
	\qedhere
    \end{align}
  \end{proof}
  Pero también:
  \begin{theorem}
    Si \(z_0\) es el elemento mínimo de \(\mathbb{Z}[\sqrt{d}]\)
    con \(z_0 > 1\) y \(N(z_0) = 1\),
    todos los elementos \(z \in \mathbb{Z}[\sqrt{d}]\) con \(N(z) = 1\)
    están dados por \(z = \pm z_0^n\) con \(n \in \mathbb{Z}\).
  \end{theorem}
  \begin{proof}
    Suponga que \(N(z) = z \cdot \overline{z} = 1\) para \(z > 1\),
    con lo que \(z^{-1} = \overline{z}\).
    Hay un único \(k \in \mathbb{N}_0\)
    tal que \(z_0^k \le z < z_0^{k + 1}\).
    Así \(z_1 = z z_0^{-k} = z (\overline{z_0})^k\)
    cumple \(1 \le z_1 < z_0\),
    y tenemos \(N(z_1) = N(z) \cdot N(z_0)^{-k} = 1\).
    Por la minimalidad de \(z_0\),
    es \(z_1 = 1\)
    y \(z = z_0^k\).
  \end{proof}
  Al par \((x_0, y_0)\) o a \(z_0 = x_0 + y_0 \sqrt{d}\)
  se le llama \emph{solución fundamental} de la ecuación de Pell.%
    \index{Pell, ecuacion de@Pell, ecuación de!solucion fundamental@solución fundamental}

  Todos los anillos \(\mathbb{Z}[\sqrt{d}]\)
  tienen infinitas unidades.
  \begin{theorem}[Dirichlet]
    \label{theo:Dirichlet}
    Sea \(\alpha\) un número irracional%
      \index{numero@número!irracional}
    y \(n\) un entero positivo.
    Entonces hay \(p \in \mathbb{Z}\) y \(q \in [1, n]\)
    tales que:
    \begin{equation}
      \label{eq:desigualdad-Dirichlet}
      \left\lvert \alpha - \frac{p}{q} \right\rvert < \frac{1}{(n + 1) q}
    \end{equation}
  \end{theorem}
  \begin{proof}
    La desigualdad~\eqref{eq:desigualdad-Dirichlet}
    es equivalente a \(\lvert q \alpha - p \rvert < 1 / (n + 1)\).
    Entre los \(n + 2\) números
      \(0, \{\alpha\}, \{2 \alpha\}, \dotsc, \{n \alpha\}, 1\),
    por el principio del palomar%
      \index{principio del palomar}
    (teorema~\ref{theo:pigeonhole})
    en el segmento \([0, 1]\) hay dos que difieren en menos de \(1 / (n + 1)\)
    (si \(\alpha\) es racional podrían diferir en exactamente \(1 / (n + 1)\)).
    Si éstos son \(\{a \alpha\}\) y \(\{b \alpha\}\)
    basta hacer \(q = \lvert a - b \rvert\);
    si son \(\{a \alpha\}\) y 0 o 1,
    basta hacer \(q = a\).
    En cualquiera de los casos,
    \(p\) es el entero más cercano a \(a \alpha\).
  \end{proof}
  De acá,
  como \(n + 1 > q\) es \(1 / ((n + 1) q) < 1 / q^2\),
  sigue inmediatamente:
  \begin{corollary}
    \label{cor:Dirichlet}
    Si \(\alpha\) es un real arbitrario,
    hay infinitos pares de enteros positivos \((p, q)\) tales que:
    \begin{equation*}
       \left\lvert \alpha - \frac{p}{q} \right\rvert
	 < \frac{1}{q^2}
    \end{equation*}
  \end{corollary}
  Así resulta:
  \begin{theorem}
    \label{theo:Pell-soluble}
    La ecuación de Pell
    \begin{equation*}
      x^2 - d y^2
	= 1
    \end{equation*}
    donde \(d\) no es un cuadrado
    tiene una solución no trivial en los enteros.
  \end{theorem}
  \begin{proof}
    Aplicando el corolario~\ref{cor:Dirichlet} a \(\alpha = \sqrt{d}\),
    vemos que hay infinitos pares \((a, b)\) tales que:
    \begin{equation*}
      \left\lvert a - b \sqrt{d} \right\rvert
	< \frac{1}{b}
    \end{equation*}
    Notamos que por la desigualdad triangular,%
      \index{desigualdad triangular}
    teorema~\ref{theo:desigualdad-triangular}:
    \begin{equation*}
      \left\lvert a + b \sqrt{d} \right\rvert
	\le \left\lvert a - b \sqrt{d} \right\rvert
	      + \left\lvert 2 b \sqrt{d} \right\rvert
	\le \frac{1}{b} + 2 b \sqrt{d}
    \end{equation*}
    Con esto:
    \begin{equation*}
      \left\lvert a^2 - b^2 d \right\rvert
	= \left\lvert a + b \sqrt{d} \right\rvert
	    \cdot \left\lvert a - b \sqrt{d} \right\rvert
	\le \left( \frac{1}{b} + 2 b \sqrt{d} \right) \cdot \frac{1}{b}
	\le 2 \sqrt{d} + 1
    \end{equation*}
    Pero al haber infinitos pares
    con normas
      \(\left\lvert N(a + b \sqrt{d}) \right\rvert \le 2 \sqrt{d} + 1\),
    y siendo enteras las normas
    por el principio del palomar%
      \index{principio del palomar}
    hay infinitos pares
    con una misma norma \(N\).
    Acá \(N \ne 0\),
    ya que solo para \(z = 0\) es \(N(z) = 0\).
    Si ahora consideramos todos los pares de norma \(N\),
    nuevamente por el principio del palomar hay infinitos pares
    \(z_1 = (a_1, b_1)\) y \(z_2 = (a_2, b_2)\)
    tales que \(a_1 \equiv a_2\) y \(b_1 \equiv b_2 \pmod{N}\),
    por lo que debe haber \(z_1 \ne \pm z_2\) entre ellos.
    Consideremos:
    \begin{align*}
      z
	&= a + b \sqrt{d}
	 = \frac{z_1}{z_2}
	 = \frac{z_1 \overline{z_2}}{N(z_2)} \\
      N(z)
	&=\frac{N(z_1)}{N(z_2)}
	 = 1
    \end{align*}
    Como \(z_1 \ne \pm z_2\),
    sabemos que \(z \ne \pm 1\).
    Como \(N(z_2) = N\),
    resultan:
    \begin{align*}
      a
	&= \frac{a_1 a_2 - d b_1 b_2}{N} \\
      b
	&= \frac{a_1 b_2 - a_2 b_1}{N}
    \end{align*}
    Observamos que,
    ya que \(a_1 \equiv a_2 \pmod{N}\) y \(b_1 \equiv b_2 \pmod{N}\):
    \begin{align*}
      a_1 a_2 - d b_1 b_2
	&\equiv a_1 a_1 - d b_1 b_1
	 \equiv 0 \pmod{N} \\
      a_1 b_2 - a_2 b_1
	&\equiv a_1 b_1 - a_1 b_1 \phantom{d}
	 \equiv 0 \pmod{N}
    \end{align*}
    con lo que \(a, b \in \mathbb{Z}\),
    o sea \(z \in \mathbb{Z}[\sqrt{d}]\) con \(N(z) = 1\).
  \end{proof}
  También nos interesa saber
  si hay soluciones a \(x^2 - d y^2 = -1\),
  ya que también son unidades de \(\mathbb{Z}[\sqrt{d}]\).
  \begin{theorem}
    \label{theo:Pell-neg-soluble}
    La ecuación \(x^2 - d y^2 = -1\) tiene solución si y solo si
    existe \(z_1 \in \mathbb{Z}[\sqrt{d}]\)
    tal que \(z_1^2 = z_0\).
  \end{theorem}
  \begin{proof}
    Demostramos implicancias en ambas direcciones.
    Si hay tal \(z_1\)
    es menor que \(z_0\),
    y como \(N(z_0) = N(z_1^2) = N(z_1)^2 = 1\),
    debe ser \(N(z_1) = -1\).

    Si \(z\) es solución de \(N(z) = -1\),
    entonces \(N(z^2) = 1\).
    En particular,
    la mínima solución \(z_1 \in \mathbb{Z}[\sqrt{d}]\)
    de la ecuación \(N(z) = -1\) tal que \(z_1 > 1\)
    da lugar a la mínima solución \(z_0 = z_1^2\) de \(N(z) = 1\).
  \end{proof}
  Hace falta determinar raíces cuadradas en \(\mathbb{Z}[\sqrt{d}]\):
  \begin{equation*}
    (x_1 + y_1 \sqrt{d})^2
      = (x_1^2 + d y_1^2) + 2 x_1 y_1 \sqrt{d}
      = (2 d y_1 - 1) + 2 x_1 y_1 \sqrt{d}
      = x_0 + y_0 \sqrt{d}
  \end{equation*}
  Acá usamos \(x_1^2 + d y_1^2 = 1\).
  O sea:
  \begin{align*}
    y_1
      &= \frac{x_0 + 1}{2 d} \\
    x_1
      &= \frac{y_0}{2 y_1}
       = \frac{d y_0}{x_0 + 1}
  \end{align*}
  En \(\mathbb{Z}[\sqrt{2}]\) la solución fundamental es \((3, 2)\),
  como su raíz cuadrada resulta el par \((1, 1) \in \mathbb{Z}[\sqrt{2}]\),
  con \(N(1 + \sqrt{2}) = -1\),
  con lo que todas las unidades
  son \(\pm (1 + \sqrt{2})^n\) para \(n \in \mathbb{Z}\).

  Lenstra~%
    \cite{lenstra02:_solving_pell_equat}
  da algo de la historia de la ecuación de Pell%
    \index{Pell, ecuacion de@Pell, ecuación de}
  y describe algoritmos para obtener la solución fundamental.
  Mayores detalles de la fascinante teoría relacionada con estos anillos
  da por ejemplo Djukić~%
    \cite{djukic07:_arith_exten_Q}.

\subsection{Cuaterniones}
\label{sec:cuaterniones}
\index{cuaterniones|textbfhy}

  Otro ejemplo interesante de anillo lo ponen los \emph{cuaterniones}~%
    \cite{hamilton44:_quaternions},
  una extensión de los números complejos
  inventada para manejar posiciones en el espacio
  como se pueden manejar puntos en el plano con números complejos.
  Hoy han sido reemplazados casi universalmente por vectores,
    \index{vector}
  más flexibles y generales.
  Considerados una curiosidad histórica por mucho tiempo,
  últimamente han encontrado utilidad en diversas áreas,
  como representación eficiente de movimientos y rotaciones en el espacio%
    \index{cuaterniones!en computacion grafica@en computación gráfica}
  en computación gráfica,
  ver por ejemplo Dorst, Fontijne y Mann~%
    \cite{dorst07:_geometric_algebra_comp_sci}.

  Con \(a, b, c, d \in \mathbb{R}\)
  el cuaternión \(z \in \mathbb{H}\) puede describirse
  \begin{equation}
    \label{eq:definicion-cuaternion}
    z = a + b \mathrm{i} + c \mathrm{j} + d \mathrm{k}
  \end{equation}
  donde
  \begin{equation}
    \label{eq:cuaterniones-unidades}
    \mathrm{i}^2
      = \mathrm{j}^2
      = \mathrm{k}^2
      = \mathrm{i} \mathrm{j} \mathrm{k} = -1
  \end{equation}
  Definimos las operaciones con estos objetos
  como para polinomios en \(\mathrm{i}\), \(\mathrm{j}\) y \(\mathrm{k}\),
  aplicando~\eqref{eq:cuaterniones-unidades} luego.

  Resulta que el anillo de cuaterniones no es conmutativo,
    \index{anillo!cuaterniones|see{cuaterniones}}
  la tabla de multiplicación de los elementos unitarios
  es el cuadro~\ref{tab:cuaterniones},
  \begin{table}[htbp]
    \centering
    \begin{tabular}{>{\(}c<{\)}|*{4}{>{\(}r<{\)}}}
      \cdot & 1 & \mathrm{i} & \mathrm{j} & \mathrm{k} \\
      \hline
      1		 & 1	      & \mathrm{i}	& \mathrm{j}  &
	  \mathrm{k}  \\
      \mathrm{i} & \mathrm{i} & -1		& \mathrm{k}  &
	  -\mathrm{j} \\
      \mathrm{j} & \mathrm{j} & -\mathrm{k} & -1	  &
	  \mathrm{i}  \\
      \mathrm{k} & \mathrm{k} & \mathrm{j}	& -\mathrm{i} &
	  -1
    \end{tabular}
    \caption{Multiplicación de cuaterniones}
    \label{tab:cuaterniones}
  \end{table}
  pero todo elemento diferente de cero tiene inverso multiplicativo.
  En detalle,
  si definimos el \emph{conjugado} del cuaternión%
    \index{cuaterniones!conjugado}
    \(q = a + b \mathrm{i} + c \mathrm{j} + d \mathrm{k}\)
  como \(\overline{q} = a - b \mathrm{i} - c \mathrm{j} - d \mathrm{k}\),
  resulta que \(q \overline{q} = a^2 + b^2 + c^2 + d^2\);
  y la \emph{norma} de \(q\)%
    \index{cuaterniones!norma}
  se define como \(\lVert q \rVert = \sqrt{q \overline{q}}\).
  Nótese que \(\overline{p q} = \overline{q} \cdot \overline{p}\).
  Con esto,
  resulta que la norma es multiplicativa,
  ya que la multiplicación entre un cuaternión cualquiera y un real
  conmuta:
  \begin{align*}
    \lVert p q \rVert^2
      &= (p q) \cdot \overline{(p q)} \\
      &= p \cdot q \overline{q} \cdot \overline{p} \\
      &= \lVert p \rVert^2 \cdot \lVert q \rVert^2
  \end{align*}
  Así resulta el \emph{recíproco}%
    \index{cuaterniones!reciproco@recíproco}
  \begin{equation*}
    q^{-1}
      = \frac{\overline{q}}{\lVert q \rVert}
  \end{equation*}
  que claramente cumple \(q q^{-1} = q^{-1} q = 1\).
  La notación \(p / q\) no tiene sentido en cuaterniones,
  ya que \(p q^{-1} \ne q^{-1} p\) en general.

\section{Los teoremas de Lagrange, Euler y Fermat}
\label{sec:Lagrange-Euler-Fermat}

  Un resultado importante que relaciona grupos y subgrupos es el siguiente:%
    \index{grupo!subgrupo}
  \begin{theorem}[Lagrange]
    \index{Lagrange, teorema de}
    \index{Lagrange, Joseph-Louis}
    \label{theo:Lagrange}
    Sea \(G\) un grupo finito,
    y \(H\) un subgrupo de \(G\).
    Entonces \(\lvert H \rvert\) divide a \(\lvert G \rvert\).
  \end{theorem}
  \begin{proof}
    Sea \(a \in G\).
    Al conjunto \(a H = \{a \odot h \colon h \in H\}\)
    se le llama \emph{coset (izquierdo) de \(H\)}%
      \index{coset|textbfhy}
    (de forma afín el \emph{coset derecho}
     \(H a = \{h \odot a \colon h \in H\}\)).
    Demostraremos que todos los cosets tienen el mismo número de elementos,
    y que particionan \(G\),
    de lo que el resultado es inmediato.

    Primeramente,
    el coset \(a H = \{a \odot h \colon h \in H\}\)
    no tiene elementos repetidos,
    porque supongamos que hay \(h, g \in H\) tales que
    \(a \odot h = a \odot g\),
    por la ley de cancelación es \(h = g\).
    Resulta simplemente \(\lvert a H \rvert = \lvert H \rvert\).

    Definamos la relación \(R\) sobre \(G\) mediante
    \(x \mathrel{R} y\) si y solo si
    hay \(h \in H\) tal que \(x = y \odot h\).
    Entonces \(R\) es una relación de equivalencia:%
      \index{relacion@relación!equivalencia}
    \begin{description}
    \item[Reflexiva:]
      Necesariamente \(1 \in H\),
      con lo que \(x \mathrel{R} x\).
    \item[Simétrica:]
      Esto porque \(x \mathrel{R} y\) corresponde a \(x = y \odot h\)
      para \(h \in H\),
      pero entonces también \(y = x \odot h^{-1}\),
      y como \(h^{-1} \in H\) tenemos \(y \mathrel{R} x\).
    \item[Transitiva:]
      Si \(x \mathrel{R} y\) y \(y \mathrel{R} z\)
      entonces hay \(h_1, h_2 \in H\) tales que
      \(x = y \odot h_1\) y \(y = z \odot h_2\).
      Combinando éstos,
      \(x = z \odot (h_2 \odot h_1)\),
      y \(h_2 \odot h_1 \in H\),
      con lo que \(x \mathrel{R} z\).
    \end{description}
    Las clases de equivalencia de \(R\) son precisamente los cosets de \(H\):
    \(x \in [y]\) siempre que podemos escribir \(x = y \odot h\)
    con \(h \in H\),
     o sea, \(x \in y H\),
     con lo que \([y] = y H\).
    Pero \(\lvert a H \rvert = \lvert H \rvert\)
    como vimos antes,
    y tenemos nuestro resultado.
  \end{proof}

  En particular,
  consideremos el subgrupo generado por el elemento \(a \in G\),%
    \index{grupo!subgrupo generado}%
    \index{grupo!orden}
    \index{grupo!orden de un elemento}
  vale decir,
  si el orden de \(a\) es \(n\)
  los elementos \(a^0\), \(a^1\), \ldots, \(a^{n - 1}\).
  Este subgrupo tiene orden \(n\),
  con lo que \(n\) divide a \(\lvert G \rvert\).

  Nos abocaremos a un estudio más detallado de \(\mathbb{Z}^\times_m\),
  una vez adquiridas algunas herramientas algebraicas adicionales.
  \begin{theorem}
    \label{theo:a-invertible}
    El elemento \(a \in \mathbb{Z}_m\) es invertible
    si y solo si \(a\) y \(m\) son coprimos.
    En particular,
    si \(p\) es primo todos los elementos de \(\mathbb{Z}_p\)
    (salvo \(0\))
    son invertibles,
    y \(\mathbb{Z}_p\) es un campo.%
      \index{campo (algebra)@campo (álgebra)}
  \end{theorem}
  \begin{proof}
    Demostramos implicancia en ambos sentidos.
    Supongamos \(a\) invertible.
    Entonces existen enteros \(b\) y \(k\) tales que \(a b - 1 = k m\),
    que es decir \(a b - k m = 1\).
    Pero si existen tales \(b\) y \(k\) entonces \(\gcd(a, m) = 1\).

    Al revés,
    supongamos \(\gcd(a, m) = 1\).
    Entonces
    (por la identidad de Bézout)%
      \index{Bezout, identidad de@Bézout, identidad de}
    existen \(s\), \(t\) tales que:
    \begin{align*}
      s \cdot a + t \cdot m
	&=	1\\
      s \cdot a
	&\equiv 1 \pmod{m}
    \end{align*}
    y \(s\) es el inverso de \(a\).
  \end{proof}
  Resulta que el número de unidades de \(\mathbb{Z}_m\)
  es una cantidad muy importante.
  Por el teorema~\ref{theo:a-invertible},
  no es más que la cantidad de números en \(1, 2, \dotsc, m\)
  que son relativamente primos a \(m\),
  que se anota \(\phi(m)\)
  (función \(\phi\) de Euler).%
    \index{\(\phi\) de Euler}%
    \index{Euler, Leonhard}

  Un ejemplo lo pone \(\mathbb{Z}_{12}\),
  donde tenemos la tabla de multiplicación~\ref{tab:Z12}.
  \begin{table}[htbp]
    \centering
    \renewcommand{\tabcolsep}{3pt}
    \begin{tabular}{>{\(}r<{\)}|*{12}{>{\(}r<{\)}}}
      \multicolumn{1}{c|}{\(\cdot\)} &
	    \;0 & \phantom{0}1 & \phantom{0}2 & \phantom{0}3 & \phantom{0}4
		& \phantom{0}5 & \phantom{0}6 & \phantom{0}7 & \phantom{0}8
		& \phantom{0}9 & 10 & 11 \\
      \hline
	\rule[-0.7ex]{0pt}{3ex}%
       0 &  0 &	 0 &  0 &  0 &	0 &  0 &  0 &  0 &  0 &	 0 &  0 &  0 \\
       1 &  0 &	 1 &  2 &  3 &	4 &  5 &  6 &  7 &  8 &	 9 & 10 & 11 \\
       2 &  0 &	 2 &  4 &  6 &	8 & 10 &  0 &  2 &  4 &	 6 &  8 & 10 \\
       3 &  0 &	 3 &  6 &  9 &	0 &  3 &  6 &  9 &  0 &	 3 &  6 &  9 \\
       4 &  0 &	 4 &  8 &  0 &	4 &  8 &  0 &  4 &  8 &	 0 &  4 &  8 \\
       5 &  0 &	 5 & 10 &  3 &	8 &  1 &  6 & 11 &  4 &	 9 &  2 &  7 \\
       6 &  0 &	 6 &  0 &  6 &	0 &  6 &  0 &  6 &  0 &	 6 &  0 &  6 \\
       7 &  0 &	 7 &  2 &  9 &	4 & 11 &  6 &  1 &  8 &	 3 & 10 &  5 \\
       8 &  0 &	 8 &  4 &  0 &	8 &  4 &  0 &  8 &  4 &	 0 &  8 &  4 \\
       9 &  0 &	 9 &  6 &  3 &	0 &  9 &  6 &  3 &  0 &	 9 &  6 &  3 \\
      10 &  0 & 10 &  8 &  6 &	4 &  2 &  0 & 10 &  8 &	 6 &  4 &  2 \\
      11 &  0 & 11 & 10 &  9 &	8 &  7 &  6 &  5 &  4 &	 3 &  2 &  1
    \end{tabular}
    \caption{La tabla de multiplicación en $\mathbb{Z}_{12}$}
    \label{tab:Z12}
  \end{table}
  Pueden apreciarse los elementos invertibles
  \(\mathbb{Z}^\times_{12} = \{1, 5, 7, 11\}\),
  con lo que \(\phi(12) = 4\).
  Se ve también que los demás son todos divisores de cero,
  como asegura el teorema~\ref{theo:finite-ring-units}.

  Al considerar el subgrupo de \(\mathbb{Z}^\times_m\) generado por \(a\)
  tenemos del teorema de Lagrange%
    \index{Lagrange, teorema de}
  que el orden de \(a\) divide al orden de \(\mathbb{Z}^\times_m\),
  que es \(\phi(m)\),
  y así:
  \begin{theorem}[Euler]
    \index{Euler, teorema de}
    \index{Euler, Leonhard}
    \label{theo:Euler}
    Si \(a\) y \(m\) son relativamente primos,
    entonces
    \begin{equation*}
      a^{\phi(m)}
	\equiv 1 \pmod{m}
    \end{equation*}
  \end{theorem}
  En el caso de que \(m\) sea primo,
  como \(\phi(p) = p - 1\) para \(p\) primo,
  el teorema de Euler se reduce a:
  \begin{theorem}[Pequeño teorema de Fermat]
    \index{Fermat, pequeno teorema de@Fermat, pequeño teorema de}
    \label{theo:Fermat}
    Si	\(p\) es primo,
    y \(p \centernot\mid a\)
    entonces
    \begin{equation*}
      a^{p - 1}
	\equiv 1 \pmod{p}
    \end{equation*}
  \end{theorem}
  El hecho de que a este se le llame ``pequeño''
  no tiene ninguna relación con su importancia,
  veremos una gran variedad de aplicaciones en lo que sigue.
% Fixme: Un poquito de historia
  El ``gran'' (o ``último'') teorema de Fermat%
    \index{Fermat, ultimo teorema@Fermat, último teorema}
  es uno de los resultados más famosos de las matemáticas.
  Fermat anotó por 1637%
    \index{Fermat, Pierre de}
  en el margen de una copia de la Aritmética de Diofanto%
    \index{Diofanto}
  que había descubierto una demostración verdaderamente maravillosa
  de que \(a^n + b^n = c^n\)
  no tiene soluciones
  para números naturales \(a\), \(b\), \(c\) si \(n > 2\),
  pero que esta no cabía en el margen
  (tenía esta mala costumbre,
   la publicación del libro rayado después de su muerte
   dio trabajo a ejércitos de matemáticos durante bastante tiempo).
  Se le llamó el ``último teorema''
  no por ser la última de sus innumerables conjeturas,
  sino por ser la última importante que quedaba sin resolver
  una vez que Euler terminó de trabajar en ellas.
  Recién en 1995 Andrew Wiles con la ayuda de su estudiante Richard Taylor~%
    \cite{taylor95:_ring_theor_proper_hecke_alg,
	  wiles95:_modul_ellip_curves_Fermat}
  demostró el último teorema de Fermat,
  aunque mediante métodos muy recientes
  (y no es precisamente una demostración ``maravillosa'').
  Generalmente se piensa que Fermat se equivocó
  al creer que tenía una demostración.

% Fixme: Determinar órdenes, etc

%%% Local Variables:
%%% mode: latex
%%% TeX-master: "clases"
%%% End:


% estructura-Zm.tex
%
% Copyright (c) 2012-2014 Horst H. von Brand
% Derechos reservados. Vea COPYRIGHT para detalles

\chapter{\protect\boldmath
	    Estructura \texorpdfstring{de $\mathbb{Z}_m$ y $\mathbb{Z}^\times_m$}
				      {algebraica de clases de congruencia}%
	  \protect\unboldmath}
\label{cha:estructura-Zm}
\index{anillo!residuos|textbfhy}

  En nuestras aplicaciones los anillos de residuos
  son las estructuras algebraicas más importantes.
  Estudiaremos brevemente en sus características principales.
  Suele resultar fructífero descomponer estructuras complejas
  en piezas más simples para ayudar a su análisis.
  Incursionaremos un poco en el área
  de analizar la estructura de grupos abelianos,%
    \index{grupo!abeliano}
  obteniendo algunos resultados de gran interés
  en áreas como la criptología.

% sumas-directas.tex
%
% Copyright (c) 2009-2014 Horst H. von Brand
% Derechos reservados. Vea COPYRIGHT para detalles

\section{Descomposiciones}
\label{sec:descomposiciones}

  Consideremos nuevamente el grupo \(\mathtt{D}_8\),
  que vimos en la sección~\ref{sec:aritmetica-Zm},
  véase el cuadro~\ref{tab:D_8}.
  \begin{table}[htbp]
    \centering
    \renewcommand{\tabcolsep}{2pt}
    \begin{tabular}{>{\(}c<{\)}|*{8}{>{\(}c<{\)}}}
      \bullet & \iota & r_1 & r_2 & r_3 & f_v & f_h & f_d & f_c \\
      \hline
	\rule[-0.7ex]{0pt}{3ex}%
      \iota & \iota & r_1   & r_2   & r_3   & f_v   & f_h   & f_d   & f_c \\
      r_1   & r_1   & r_2   & r_3   & \iota & f_v   & f_h   & f_d   & f_c \\
      r_2   & r_2   & r_3   & \iota & r_1   & f_c   & f_d   & f_v   & f_h \\
      r_3   & r_3   & \iota & r_1   & r_2   & f_d   & f_c   & f_h   & f_v \\
      f_v   & f_v   & f_d   & f_h   & f_c   & \iota & r_2   & r_1   & r_3 \\
      f_h   & f_h   & f_c   & f_v   & f_d   & r_2   & \iota & r_3   & r_1 \\
      f_d   & f_d   & f_h   & f_c   & f_v   & r_3   & r_1   & \iota & r_2 \\
      f_c   & f_c   & f_v   & f_d   & f_h   & r_1   & r_3   & r_2   & \iota
    \end{tabular}
    \caption{El grupo $\mathtt{D}_8$}
    \label{tab:D_8}
  \end{table}
  Si analizamos las operaciones que lo componen,
  vemos que las operaciones \(\{\iota, r_1, r_2, r_3\}\)
  por sí solas también conforman un grupo
  (corresponden a solo girar el cuadrado en el plano,
   sin salir de él),
  o sea forman un subgrupo de \(\mathtt{D}_8\).
  Otros subgrupos están formados por \(\iota\) solo
  (el grupo trivial,
   nuevamente),
  \(\{\iota, r_2\}\),
  \(\{\iota, f_d\}\).

% Fixme: Otros ejemplos de grupo: Z_4, Klein's 4-group,
%	 Z*_9, Z*_{15}, S_3, A_4

  Un ejemplo más simple
  (porque es un grupo abeliano)
  lo da \(\mathbb{Z}_{12}\) con la suma,
  véase el cuadro~\ref{tab:Z12}.
  En \(\mathbb{Z}_{12}\) son subgrupos
  \(\{0\}\),
  \(\{0, 6\}\),
  \(\{0, 4, 8\}\),
  \(\{0, 3, 6, 9\}\),
  \(\{0, 2, 4, 6, 8, 10\}\)
  y \(\{0, 1, 2, 3, 4, 5, 6, 7, 8, 9, 10, 11\}\).

\subsection{Homomorfismos e isomorfismos}
\label{sec:homomorfismos-isomorfismos}
\index{homomorfismo}
\index{isomorfismo}

  Consideremos los grupos \(\mathbb{Z}^\times_8\) y \(\mathbb{Z}^\times_{12}\),
  que casualmente tienen el mismo número de elementos.
  \begin{table}[htbp]
    \centering
    \subfloat[\(\mathbb{Z}^\times_8\)]{
      \renewcommand{\tabcolsep}{3pt}
      \begin{tabular}{>{\(}r<{\)}|*{4}{>{\(}r<{\)}}}
	\multicolumn{1}{c|}{\(\cdot\)}
	       & \phantom{0}1
		      & \phantom{0}3
			    & \phantom{0}5
				  & \phantom{0}7 \\
	\hline
	  \rule[-0.7ex]{0pt}{3ex}%
	      1 &   1 &	  3 &	5 &   7	 \\
	      3 &   3 &	  1 &	7 &   5	 \\
	      5 &   5 &	  7 &	1 &   3	 \\
	      7 &   7 &	  5 &	3 &   1	 \\
      \end{tabular}
      \label{subtab:Z8*}
    }%
    \hspace*{3em}%
    \subfloat[\(\mathbb{Z}^\times_{12}\)]{
      \renewcommand{\tabcolsep}{3pt}
      \begin{tabular}{>{\(}r<{\)}|*{4}{>{\(}r<{\)}}}
	\multicolumn{1}{c|}{\(\cdot\)} &
		1 &   5 &   7 &	 11 \\
	\hline
	  \rule[-0.7ex]{0pt}{3ex}%
		1 &   1 &   5 &	  7 &  11  \\
		5 &   5 &   1 &	 11 &	7  \\
		7 &   7 &  11 &	  1 &	5  \\
	       11 &  11 &   7 &	  5 &	1  \\
      \end{tabular}
      \label{subtab:Z12*}
    }
    \caption{Los grupos $\mathbb{Z}^\times_8$ y $\mathbb{Z}^\times_{12}$}
    \label{tab:Z8*+Z12*}
  \end{table}
  Estas tablas
  (cuadros~\ref{subtab:Z8*} y~\ref{subtab:Z12*})
  son diferentes,
  pero podemos ver que tienen la misma estructura,
  por ejemplo el mapa
  \begin{center}
    \begin{tabular}{>{\(}r<{\)}@{$\leftrightarrow$}>{\(}r<{\)}}
      1 &  1 \\
      3 &  5 \\
      5 &  7 \\
      7 & 11
    \end{tabular}
  \end{center}
  traduce entre ellos.
  Sin embargo,
  hay grupos diferentes con cuatro elementos.
  Por ejemplo,
  tenemos \(\mathbb{Z}^\times_5\)
  (cuadro~\ref{subtab:Z5*})
  y \(\mathbb{Z}_4\)
  (cuadro~\ref{subtab:Z4}).
  \begin{table}[htbp]
    \centering
    \subfloat[\(\mathbb{Z}^\times_5\)]{
      \renewcommand{\tabcolsep}{3pt}
      \begin{tabular}{>{\(}r<{\)}|*{4}{>{\(}r<{\)}}}
	\multicolumn{1}{c|}{\(\cdot\)} &
		1 &   2 &   3 &	  4 \\
	\hline
	  \rule[-0.7ex]{0pt}{3ex}%
	  1 &	1 &   2 &   3 &	  4 \\
	  2 &	2 &   4 &   1 &	  3 \\
	  3 &	3 &   1 &   4 &	  2 \\
	  4 &	4 &   3 &   2 &	  1 \\
      \end{tabular}
      \label{subtab:Z5*}
    }
    \hspace*{3em}
    \subfloat[\(\mathbb{Z}_4\)]{
      \renewcommand{\tabcolsep}{3pt}
      \begin{tabular}{>{\(}r<{\)}|*{4}{>{\(}r<{\)}}}
	\multicolumn{1}{c|}{\(+\)} &
		0 &   1 &   2 &	  3 \\
	\hline
	  \rule[-0.7ex]{0pt}{3ex}%
	  0 &	0 &   1 &   2 &	  3 \\
	  1 &	1 &   2 &   3 &	  0 \\
	  2 &	2 &   3 &   0 &	  1 \\
	  3 &	3 &   0 &   1 &	  2 \\
      \end{tabular}
      \label{subtab:Z4}
    }
    \caption{Los grupos $\mathbb{Z}^\times_5$ y $\mathbb{Z}_4$}
    \label{tab:Z5*+Z4}
  \end{table}
  Nótese que entre \(\mathbb{Z}_4\) y \(\mathbb{Z}^\times_5\)
  también podemos construir una correspondencia,
  a pesar que la operación involucrada es diferente:
  \begin{center}
    \begin{tabular}{>{\(}r<{\)}@{$\leftrightarrow$}>{\(}r<{\)}}
      0 & 1 \\
      1 & 2 \\
      2 & 4 \\
      3 & 3
    \end{tabular}
  \end{center}
  No hay correspondencia posible
  entre \(\mathbb{Z}^\times_5\) y \(\mathbb{Z}^\times_8\):
  En la diagonal
  de la tabla para \(\mathbb{Z}^\times_5\)
  (cuadro~\ref{subtab:Z5*})
  aparecen dos valores diferentes,
  mientras para \(\mathbb{Z}^\times_8\)
  (cuadro~\ref{subtab:Z8*})
  hay uno solo.

  Esta idea de ``misma estructura'' es importante,
  y la capturamos con lo siguiente.
  \begin{definition}
    \label{def:group-isomorphism}
    Sean dos grupos \((G, 1_G, \odot)\) y \((H, 1_H, \otimes)\),
    un \emph{homomorfismo} de \(G\) a \(H\)%
      \index{grupo!homomorfismo|textbfhy}%
      \index{homomorfismo}
    es una función \(h \colon G \rightarrow H\)
    tal que \(h(a \odot b) = h(a) \otimes h(b)\).
    A un homomorfismo que es una biyección se le llama \emph{isomorfismo},%
      \index{grupo!isomorfismo|textbfhy}%
      \index{isomorfismo}
    y se dice en tal caso que los grupos son \emph{isomorfos},
    y se anota \(G \cong H\).
    Un caso importante de isomorfismos son los isomorfismos
    de \(G\) a~\(G\),
    los \emph{automorfismos}.%
      \index{grupo!automorfismo|textbfhy}%
      \index{automorfismo}
  \end{definition}
  Las mismas ideas son aplicables a otras estructuras algebraicas,
  como anillos,
  si la función es homomorfismo
  (o isomorfismo)
  para ambas operaciones.%
    \index{anillo!homomorfismo}%
    \index{anillo!isomorfismo}%
    \index{anillo!automorfismo}

  Un ejemplo conocido de homomorfismo
  es la clasificación de números en pares e impares,
  con las correspondientes reglas de sumas y productos.
  Un isomorfismo útil es el entre \((\mathbb{R}^+, \cdot)\)
  y \((\mathbb{R}, +)\) dado por los logaritmos.

% Fixme: Ejemplos y/o ejercicios varios de grupos, homomorfismos
%	 (Z --> Z_n, ...), isomorfismos (R^+ --> R vía \ln, ...)

  Si \(h \colon G \rightarrow H\) es un homomorfismo,
  y \(1_G\) y \(1_H\) son los elementos neutros de \(G\) y \(H\),
  respectivamente,
  claramente \(h(1_G) = 1_H\),
  y \(h(a^{-1}) = (h(a))^{-1}\).

  Una manera simple de entender un isomorfismo es considerando
  que los dos grupos ``son el mismo'',
  solo cambiando los nombres de los elementos y la operación.
  Es fácil demostrar que los grupos cíclicos finitos de orden \(n\)
  son isomorfos a \(\mathbb{Z}_n\),
  y los infinitos isomorfos a \(\mathbb{Z}\).

  En \(\mathbb{Z}_p\) para \(p\) primo
  hay automorfismos que asocian \(1\) con cada elemento no cero.
  Esto no es más que otra forma de decir que módulo \(p\)
  todos los elementos tienen inverso.

  El isomorfismo entre grupos es una relación de equivalencia:%
    \index{relacion de equivalencia@relación de equivalencia}
  Es reflexiva,
  un grupo es isomorfo a sí mismo;
  es simétrica,
  ya si hay una biyección como la indicada,
  existe la función inversa que cumple las mismas condiciones;
  y es transitiva,
  siendo la composición de los isomorfismos el isomorfismo buscado.
  Es por ser una equivalencia que tiene sentido considerar ``iguales''
  estructuras algebraicas isomorfas.

  Una aplicación es la \emph{prueba de los nueves},
  popular cuando operaciones aritméticas se hacen manualmente.
  Consiste en verificar operaciones aritméticas
  (sumas, restas y multiplicaciones)
  vía calcular el residuo módulo nueve de los operandos,
  operar con los residuos,
  y comparar con el residuo módulo nueve del resultado.
  El punto es que
  (por el teorema~\ref{theo:mod-rules})
  el reducir módulo \(m\) es un homomorfismo
  del anillo \(\mathbb{Z}\) a \(\mathbb{Z}_m\),%
    \index{anillo!homomorfismo}
  por lo que ambos residuos debieran coincidir.
  Calcular el residuo módulo nueve de un número escrito en decimal
  es simplemente sumar sus dígitos
  hasta llegar a un resultado de un único dígito:
  Como \(10 \equiv 1 \pmod{9}\),
  tenemos:
  \begin{equation*}
    \sum_{0 \le k \le n} d_k \cdot 10^k
      \equiv \sum_{0 \le k \le n} d_k \pmod{9}
  \end{equation*}
  Demostramos un resultado similar en el lema~\ref{lem:9|10^k-1}.

\subsection{Sumas directas}
\label{sec:sumas-directas}

  En lo que sigue discutiremos grupos abelianos,
  pero la operación que interesa puntualmente
  es la multiplicación entre enteros.
  Para evitar confusiones,
  usaremos notación de multiplicación y potencias,
  y no sumas como sería por convención general.
  Por lo demás,
  la notación como multiplicación es más compacta.

  Siempre es útil tratar de descomponer estructuras complejas
  en piezas más simples.
  Consideremos
  el grupo de unidades \(\mathbb{Z}^\times_8 = \{1, 3, 5, 7\}\)
  y dos de sus subgrupos,
  \(\{1, 3\}\) y \(\{1, 5\}\).
  Todo elemento de \(\mathbb{Z}^\times_8\)
  puede escribirse como un producto de un elemento de cada uno de estos:
  \begin{alignat*}{2}
    1 &= 1 \cdot 1 &\qquad& 5 = 1 \cdot 5 \\
    3 &= 3 \cdot 1 &&	    7 = 3 \cdot 5
  \end{alignat*}
  Otro ejemplo
  provee \(\mathbb{Z}^\times_{15} = \{1, 2, 4, 7, 8, 11, 13, 14\}\),
  con subgrupos \(\{1, 2, 4, 8\}\) y \(\{1, 11\}\):
  \begin{alignat*}{2}
     1 &= 1 \cdot \phantom{0}1
       &\qquad	8 &= 8 \cdot \phantom{0}1 \\
     2 &= 2 \cdot \phantom{0}1
       &       11 &= 1 \cdot 11 \\
     4 &= 4 \cdot \phantom{0}1
       &       13 &= 8 \cdot 11 \\
     7 &= 2 \cdot 11
       &       14 &= 4 \cdot 11
  \end{alignat*}
  Esto motiva la siguiente:
  \begin{definition}
    \label{def:suma-directa}
    Sean \(A\) y \(B\) subgrupos del grupo abeliano \(G\)
    tales que todo \(g \in G\) puede escribirse de forma única
    como \(g = a \cdot b\),
    con \(a \in A\) y \(b \in B\).
    Entonces escribimos \(G = A B\)
    y decimos que \(G\) es la \emph{suma directa} de \(A\) y \(B\).
  \end{definition}
  La utilidad de esta noción se debe en buena parte
  a que si sabemos qué son \(A\) y \(B\)
  conocemos \(A B\):
  \begin{theorem}
    \label{theo:suma-directa-isomorfos}
    \index{grupo!isomorfismo}
    Si \(G = A B\),
    \(G' = A' B'\)
    y \(A \cong A'\), \(B \cong B'\),
    entonces \(G \cong G'\).
  \end{theorem}
  \begin{proof}
    Supongamos que
    \(f \colon A \rightarrow A'\) y \(h \colon B \rightarrow B'\)
    son isomorfismos,
    construimos un isomorfismo \(k \colon G \rightarrow G'\)
    definiendo:
    \begin{equation*}
      k(g) = f(a) \cdot h(b)
    \end{equation*}
    donde \(a \in A\), \(b \in B\) y \(g = a \cdot b\).
    Primeramente,
    esta definición tiene sentido,
    ya que para \(g \in G\) los elementos \(a\) y \(b\) son únicos,
    con lo que \(k\) es una función.
    Es uno a uno,
    ya que si tomamos \(g_1 \ne g_2\),
    al escribir \(g_1 = a_1 \cdot b_1\) y \(g_2 = a_2 \cdot b_2\),
    necesariamente estos pares son diferentes,
    y como \(f\) y \(h\) son uno a uno,
    tendremos
    \(k(g_1) = f(a_1) \cdot h(b_1) \ne f(a_2) \cdot h(b_2) = k(g_2)\).
    Es sobre ya que si tomamos \(g' \in G'\),
    este puede escribirse de forma única como \(g' = a' \cdot b'\),
    y usando las inversas de \(f\) y \(h\)
    esto lleva al elemento único
    \(g = f^{-1}(a') \cdot h^{-1}(b') \in G\)
    tal que \(k(g) = g'\).
  \end{proof}
  Este enredo oculta algo muy simple:
  Si \(G = A B\),
  se puede expresar \(g \in G\) mediante las ``coordenadas''
  \((a, b)\) con \(g = a \cdot b\),
  y considerar \(A B\) como \(A \times B\) con operación
  \((a, b) \cdot (a', b') = (a \cdot a', b \cdot b')\).
  En estos términos,
  la operación en \(A B\) está completamente determinada
  por las operaciones en \(A\) y \(B\);
  si \(A'\) es una copia de \(A\)
  y \(B'\) es una copia de \(B\),
  entonces \(G' = A' B'\)
  es simplemente una copia de \(G = A B\).

  Analicemos \(\mathbb{Z}^\times_{15}\) y \(\mathbb{Z}^\times_{16}\).
  Ya vimos que \(\mathbb{Z}^\times_{15} = \{1, 2, 4, 8\} \{1, 11\}\);
  mientras \(\mathbb{Z}^\times_{16} = \{1, 3, 5, 7, 9, 11, 13, 15\}\),
  con subgrupos \(\{1, 3, 9, 11\}\) y \(\{1, 7\}\),
  y tenemos \(\mathbb{Z}^\times_{16} = \{1, 3, 9, 11\} \{1, 7\}\).
  Pero \(\{1, 2, 4, 8\}\) y \(\{1, 3, 9, 11\}\)
  son grupos cíclicos de orden \(4\),
  y por tanto isomorfos a \(\mathbb{Z}_4\);
  y por el otro lado \(\{1, 11\}\) y \(\{1, 7\}\) son cíclicos de orden 2,
  isomorfos a \(\mathbb{Z}_2\).
  Entonces \(\mathbb{Z}^\times_{15} \cong \mathbb{Z}^\times_{16}\).

  Para cálculos concretos el siguiente teorema es útil:
  \begin{theorem}
    \label{theo:subgrupo-suma-directa}
    Si \(A\) y \(B\) son subgrupos del grupo abeliano \(G\)
    tales que \(A \cap B = \{1\}\)
    y \(\lvert A \rvert \cdot \lvert B \rvert = \lvert G \rvert\)
    entonces \(G = A B\).
  \end{theorem}
  \begin{proof}
    Consideremos los productos \(a b\)
    con \(a \in A\) y \(b \in B\).
    Demostramos que son diferentes por contradicción.
    Supongamos pares distintos \((a_1, b_1)\) y \((a_2, b_2)\)
    tales que \(a_1 \odot b_1 = a_2 \odot b_2\).
    Entonces \(a_1 \odot a_2^{-1} = b_1^{-1} \odot b_2\).
    Pero \(a_1 \odot a_2^{-1} \in A\)
    y \(b_1 \odot b_2^{-1} \in B\),
    con lo que esto tiene que estar en la intersección entre ambos,
    o sea \(a_1 \odot a_2^{-1} = b_1 \odot b_2^{-1} = 1\),
    con lo que \(a_1 = a_2\) y \(b_1 = b_2\).

    Con esto hay exactamente
      \(\lvert A \rvert \cdot \lvert B \rvert = \lvert G \rvert\)
    productos \(a \odot b\) diferentes,
    que tienen que ser todos los elementos de \(G\).
  \end{proof}
  El grupo \(\mathbb{Z}^\times_{16}\) tiene \(8 = 4 \cdot 2\) elementos,
  con lo que de los subgrupos \(\{1, 3, 9, 11\}\) y \(\{1, 7\}\)
  tenemos \(\mathbb{Z}^\times_{16} = \{1, 3, 9, 11\} \{1, 7\}\),
  ya que \(\{1, 3, 9, 11\} \cap \{1, 7\} = \{1\}\).

  Esto puede extenderse a más de dos subgrupos.
  Por ejemplo,
  \(\mathbb{Z}_{30}\)
  tiene subgrupos  \(\{0, 6, 12, 18, 24\}\) y \(\{0, 5, 10, 15, 20, 25\}\),
  de órdenes \(5\) y \(6\),
  con intersección \(\{0\}\).
  Por el teorema~\ref{theo:subgrupo-suma-directa}
  tenemos la descomposición
  \(\mathbb{Z}_{30} = \{0, 6, 12, 18, 24\} \{0, 5, 10, 15, 20, 25\}\).
  Por su lado,
  \(\{0, 5, 10, 15, 20, 25\}\) tiene subgrupos
  \(\{0, 10, 20\}\) y \(\{0, 15\}\),
  de órdenes \(3\) y \(2\),
  y es
  \(\{0, 5, 10, 15, 20, 25\} = \{0, 10, 20\} \{0, 15\}\).
  Esto sugiere extender la definición~\ref{def:suma-directa} y escribir
  \(\mathbb{Z}_{30} = \{0, 6, 12, 18, 24\}
       \{0, 10, 20\} \{0, 15\}\).
  \begin{definition}
    \label{def:suma-directa-n}
    Sea \(G\) un grupo abeliano,
    y sean \(A_1\), \(A_2\), \ldots, \(A_n\) subgrupos de \(G\)
    tales que todo elemento de \(G\) puede escribirse de forma única
    como \(a_1 \cdot a_2 \dotsm a_n\),
    con \(a_i \in A_i\) para todo \(1 \le i \le n\).
    Entonces \(G\) es la \emph{suma directa}
    de los subgrupos \(A_1\), \(A_2\), \ldots, \(A_n\),
    y anotamos \(G = A_1 A_2 \dotsb A_n\).
  \end{definition}
  Si \(G = A_1 A_2 \dotsb A_n\)
  y \(g = a_1 a_2 \dotso a_n\) con \(a_i \in A_i\)
  decimos que \(a_i\) es el \emph{componente} de \(g\) en \(A_i\).
  Por la definición de suma directa
  el componente de \(g\) en \(A_i\) es único.
  Una relación útil entre el orden del elemento
  y los órdenes de sus componentes es la siguiente:
  \begin{theorem}
    \label{theo:ordenes-suma-directa}
    Si \(G = A_1 A_2 \dotsb A_n\) y \(g \in G\),
    entonces el orden de \(g\)
    es el mínimo común múltiplo de los órdenes de los componentes de \(g\)
  \end{theorem}
  \begin{proof}
    Sea \(g = a_1 \cdot a_2 \dotsm a_n\) con \(a_i \in A_i\).
    Para cualquier entero \(s\)
    tendremos \(g^s = a_1^s \cdot a_2^s \dotsm a_n^s\).
    Como \(a_i^s \in A_i\),
    el componente en \(A_i\) de \(g^s\) es \(a_i^s\).
    Por otro lado,
    el componente de \(1\) en \(A_i\) es \(1\),
    y \(g^s = 1\) solo si \(a_i^s = 1\) para todo \(1 \le i \le n\),
    con lo que \(s\) es un múltiplo del orden de \(a_i\)
    para cada \(1 \le i \le n\),
    y el orden de \(g\) es el menor de todos los posibles múltiplos.
  \end{proof}

  Para ilustrar lo anterior,
  consideremos
  \(\mathbb{Z}^\times_{21} = \{1, 4, 16\} \{1, 8\} \{1, 13\}\).
  Si tomamos \(11 \in \mathbb{Z}^\times_{21}\),
  se descompone en \(11 = 4 \cdot 8 \cdot 1\).
  Las potencias respectivas las da el cuadro~\ref{tab:potencias-Z21*},
  \begin{table}[htbp]
    \centering
    \begin{tabular}{>{\(}l<{\)}*{3}{@{\qquad}>{\(}l<{\)}}}
      11		  & 4		       & 8	     & 1 \\
      11^2 =	       16 & 4^2 =	  16   & 8^2 = 1     &	 \\
      11^3 = \phantom{0}8 & 4^3 = \phantom{0}1 &	     &	 \\
      11^4 = \phantom{0}4 &		       &	     &	 \\
      11^5 = \phantom{0}2 &		       &	     &	 \\
      11^6 = \phantom{0}1 &		       &	     &
    \end{tabular}
    \caption{Potencias en $\mathbb{Z}^\times_{21}$}
    \label{tab:potencias-Z21*}
  \end{table}
  lo que confirma que el orden de \(11\) es \(6 = 3 \cdot 2 \cdot 1\).

\subsection{Sumas directas externas}
\label{sec:sumas-directas-externas}

  Hasta acá hemos descompuesto un grupo en la suma directa de subgrupos.
  La pregunta inversa es si
  dados grupos \(A_1\), \(A_2\), \ldots, \(A_n\),
  podemos construir \(G\) con subgrupos \(H_1\), \(H_2\), \ldots, \(H_n\)
  tales que \(G = H_1 H_2 \dotsb H_n\)
  con \(A_i \cong H_i\) para todo \(1 \le i \le n\).
  La respuesta es afirmativa,
  y la construcción es muy simple.
  Vimos que si \(G = H_1 H_2 \dotsb H_n\),
  entonces \(g \in G\)
  puede escribirse \(g = h_1 h_2 \dotsm h_n\) en forma única,
  con \(h_i \in H_i\).
  Especificar \(g\) es lo mismo que especificar la tupla de coordenadas
  \(h_i\).
  De igual manera,
  dado \(k \in G\)
  podemos escribirlo \(k = k_1 k_2 \dotsm k_n\) en forma única,
  con \(k_i \in H_i\),
  y \(g k = h_1 h_2 \dotso h_n \cdot k_1 k_2 \dotsm k_n
	 = (h_1 k_1) (h_2 k_2) \dotsm (h_n k_n)\),
  donde \(h_i k_i \in H_i\) resulta ser la coordenada de \(g k\).
  Esta situación motiva la definición siguiente:
  \begin{definition}
    Sean \(A_1\), \(A_2\), \ldots, \(A_n\) grupos abelianos.
    La \emph{suma directa (externa)} de \(A_1\), \(A_2\), \ldots, \(A_n\)
    es el conjunto de tuplas \((a_1, a_2, \dotsc, a_n)\) con \(a_i \in A_i\)
    y operación dada por:
    \begin{equation*}
      (a_1, a_2, \dotsc, a_n) \cdot (b_1, b_2, \dotsc, b_n)
	= (a_1 \cdot b_1, a_2 \cdot b_2, \dotsc, a_n \cdot b_n)
    \end{equation*}
    Escribiremos \(G = A_1 \times A_2 \times \dotsb \times A_n\)
    para la suma directa externa de los grupos \(A_1\), \ldots, \(A_n\).
  \end{definition}
  De acá resulta:
  \begin{theorem}
    \label{theo:suma-directa-externa}
    La suma directa (externa)
    de los grupos abelianos \(A_1\), \(A_2\), \ldots, \(A_n\)
    es un grupo abeliano,
    \(G = H_1 H_2 \dotsb H_n\),
    donde \(H_i\) es el conjunto de tuplas
    de la forma \((1, \dotsc, 1, a_i, 1, \dotsc, 1)\)
    con \(a_i \in A_i\)
    (todas las componentes, salvo la \(i\)\nobreakdash-ésima, son 1).
    Además,
    \(H_i \cong A_i\) para todo \(1 \le i \le n\).
  \end{theorem}
  \begin{proof}
    Demostrar que \(G\) es un grupo abeliano es automático;
    hay que verificar
    que la operación es cerrada (inmediato de la definición),
    asociatividad (resulta directamente de la asociatividad en cada \(A_i\)),
    existencia de neutro (resulta ser \((1, 1, \dotsc, 1)\)),
    conmutatividad (directamente de cada \(A_i\))
    e inverso
    (el inverso de \((a_1, a_2, \dotsc, a_n)\)
     es \((a_1^{-1}, a_2^{-1}, \dotsc, a_n^{-1})\)).

    Podemos escribir un elemento \(g \in G\) como:
    \begin{equation*}
      (a_1, a_2, \dotsc, a_n)
	 = (a_1, 1, \dotsc, 1) \cdot
	   (1, a_2, \dotsc, 1) \dotsm
	   (1, 1, \dotsc, a_n)
    \end{equation*}
    Acá \((1, \dotsc, 1, a_i, 1, \dotsc, 1) \in H_i\),
    lo que puede hacerse de una única forma,
    y los \(H_i\) son subgrupos de \(G\).
    Resulta \(G = H_1 H_2 \dotsb H_n\),
    y \(f_i \colon H_i \rightarrow A_i\)
    que mapea \((1,  \dotsc, 1, a_i, 1, \dotsc, 1)\) a \(a_i\)
    es un isomorfismo.
  \end{proof}

  La noción de sumas directas externas
  da una notación conveniente para describir grupos abelianos.
  Por ejemplo,
  vimos \(\mathbb{Z}^\times_{15} = \{1, 2, 4, 8\} \{1, 11\}\);
  pero estos dos son grupos cíclicos de orden 4 y 2, respectivamente,
  con lo que \(\mathbb{Z}^\times_{15} \cong \mathbb{Z}_4 \times \mathbb{Z}_2\)
  dice todo lo que hay que saber sobre \(\mathbb{Z}^\times_{15}\).

\subsection{Comentarios finales}
\label{sec:Zm-comentarios}

  Temas relacionados con grupos,
  anillos y otras estructuras algebraicas
  profundizan bastante textos del área como Connell~%
    \cite{connell04:_elemen_abstr_linear_algeb}
  y Judson~%
    \cite{judson14:_abstr_algeb}.

  Lo que nosotros llamamos \(\mathbb{Z}_m\)
  se conoce formalmente como \(\mathbb{Z} / m \mathbb{Z}\).
  Para justificar esta notación,
  primeramente definimos:
  \begin{definition}
    Sea \(G\) un grupo.
    Un subgrupo \(N\) de \(G\)
    se dice \emph{normal}
    (se anota \(N \lhd G\))
    si para todo \(n \in N\) y \(g \in G\)
    tenemos \(g n g^{-1} \in N\).
  \end{definition}
  Los subgrupos de un grupo abeliano son siempre normales.

  En ciertas situaciones los cosets de un subgrupo
  se pueden dotar con una operación
  heredada del grupo \(G\) para dar un nuevo grupo,
  el \emph{grupo cociente} o \emph{factor}
  \begin{equation*}
    G / N
      = \{g N \colon g \in G\}
  \end{equation*}
  con operación
  \begin{equation*}
    (g N) \bullet (h N)
      = (g h) N
  \end{equation*}
  Esto solo funciona si \(N \lhd G\),
  en cuyo caso el mapa \(g \mapsto g N\)
  es un homomorfismo de \(G\) a \(G / N\).%
    \index{grupo!homomorfismo}

  Ahora bien,
  el coset \(m \mathbb{Z}\) es un subgrupo de \(\mathbb{Z}\),
  y es un subgrupo normal
  ya que todos los subgrupos de un grupo abeliano son normales.
  Vemos que \(a + m \mathbb{Z}\) es precisamente el conjunto
  \(r + m \mathbb{Z}\), donde \(r = a \mod m\),
  y la suma en \(\mathbb{Z} / m \mathbb{Z}\)
  es exactamente como la describimos en~\ref{sec:aritmetica-Zm}.

  Otra notación común es \(\mathbb{Z} / (m)\),
  usando la misma idea anterior
  pero describiendo el conjunto de los múltiplos de \(m\)
  como el ideal generado por \(m\),
  vale decir el conjunto \(\{ r m \colon r \in \mathbb{Z}\}\).
  Estudiaremos este importante concepto
  en la sección~\ref{sec:dominios-euclidianos}.

%%% Local Variables:
%%% mode: latex
%%% TeX-master: "clases"
%%% End:


  Nuestra primera tarea es descomponer \(\mathbb{Z}_m\)
  en forma más sistemática.
  Lo que hemos hecho hasta ahora es tomar elementos que se ven bien
  y considerar los subgrupos que generan,
    \index{grupo!subgrupo}
  tratando de encontrar intersecciones y órdenes adecuados.
  Acá veremos un método general
  para descomponer el grupo \(\mathbb{Z}_m\),
  y plantear el camino para entender mejor
  los grupos \(\mathbb{Z}^\times_m\).

  Por ejemplo,
  \(\mathbb{Z}_{30}
      = \{0, 6, 12, 18, 24\} \{0, 5, 10, 15, 20, 25\}\)
  o sea \(\mathbb{Z}_{30}\)
  es la suma directa de un grupo cíclico de orden \(5\)
  y otro de orden \(6\),
  y por el teorema~\ref{theo:suma-directa-isomorfos}
  esto es
    \(\mathbb{Z}_{30} \cong \mathbb{Z}_5 \times \mathbb{Z}_6\).
  Por otro lado,
  hay una función obvia
  \(f \colon \mathbb{Z}_{30}
     \rightarrow \mathbb{Z}_5 \times \mathbb{Z}_6\):
  Si conocemos un entero \(x\) módulo \(30\),
  sabemos sus residuos módulos \(5\) y \(6\).%
    \index{residuo}
  Por ejemplo,
  si \(x \equiv 13 \pmod{30}\),
  entonces \(x \equiv 13 \equiv 3 \pmod{5}\)
  y \(x \equiv 13 \equiv 1 \pmod{6}\).
  En este caso,
  tendríamos \(f(13) = (3, 1)\).
  Una tabla completa para \(f\) es:
  \begin{equation*}
    \begin{array}{*{5}{l}}
      f(0) = (0, 0) &  f(6) \phantom{0}= (1, 0)
		    & f(12) = (2, 0) & f(18) = (3, 0) &
		      f(24) = (4, 0) \\
      f(1) = (1, 1) &  f(7) \phantom{0}= (2, 1)
		    & f(13) = (3, 1) & f(19) = (4, 1) &
		      f(25) = (0, 1) \\
      f(2) = (2, 2) &  f(8) \phantom{0}= (3, 2)
		    & f(14) = (4, 2) & f(20) = (0, 2) &
		      f(26) = (1, 2) \\
      f(3) = (3, 3) &  f(9) \phantom{0}= (4, 3)
		    & f(15) = (0, 3) & f(21) = (1, 3) &
		      f(27) = (2, 3) \\
      f(4) = (4, 4) & f(10) = (0, 4) & f(16) = (1, 4) &
		      f(22) = (2, 4) &
		      f(28) = (3, 4) \\
      f(5) = (0, 5) & f(11) = (1, 5) & f(17) = (2, 5) &
		      f(23) = (3, 5) &
		      f(29) = (4, 5)
    \end{array}
  \end{equation*}
  El que \(f\) es uno a uno se ve directamente de la tabla.
  Las restantes propiedades son obvias si consideramos
  que cada entrada como
  \(f([x]_{30}) = ([x]_5, [x]_6)\).
  Tenemos el siguiente teorema general:
  \begin{theorem}
    \index{residuo!teorema chino de los!padre del}
    \label{theo:isomorfismo-anillo-Zm}
    Sea \(m = a b\) con \(a, b\) enteros positivos
    tales que \(\gcd(a, b) = 1\).
    Entonces la función:
    \begin{equation*}
      f([x]_m) = ([x]_a, [x]_b) \quad x \in \mathbb{Z}
    \end{equation*}
    es un isomorfismo%
      \index{anillo!isomorfismo}
    entre los anillos \(\mathbb{Z}_m\)
    y \(\mathbb{Z}_a \times \mathbb{Z}_b\).
  \end{theorem}
  \begin{proof}
    Primero debemos demostrar que \(f\) siquiera tiene sentido,
    hay muchas elecciones de \(x\)
    que dan la misma clase \([x]_m\) en \(\mathbb{Z}_m\).
    El teorema~\ref{theo:congruencia-mn}
    asegura que \(f\) es una biyección,
    ya que \(\gcd(a, b) = 1\).

    Además tenemos del teorema~\ref{theo:+*mod} que:
    \begin{align*}
      f([x + y]_m)
	&= ([x + y]_a, [x + y]_b) \\
	&= ([x]_a + [y]_a, [x]_b + [y]_b) \\
	&= ([x]_a, [x]_b)
	     + ([y]_a, [y]_b) \\
	&= f(x) + f(y) \\
      f([x y]_m)
	&= ([x y]_a, [x y]_b) \\
	&= ([x]_a \cdot [y]_a,
	     [x]_b \cdot [y]_b) \\
	&= ([x]_a, [x]_b)
	    \cdot ([y]_a, [y]_b) \\
	&= f(x) \cdot f(y)
    \end{align*}
    y es isomorfismo de anillo.
  \end{proof}
  Hay que tener cuidado en esto,
  la condición de que \(\gcd(a, b) = 1\) es necesaria.%
    \index{relativamente primos}
  Por ejemplo,
  \(\mathbb{Z}_8 \ncong \mathbb{Z}_2 \times \mathbb{Z}_4\),
  ya que:
  \begin{align*}
    2 \equiv 0 \pmod{2}
      &\qquad 2 \equiv 2 \pmod{4} \\
    6 \equiv 0 \pmod{2}
      &\qquad 6 \equiv 2 \pmod{4}
  \end{align*}
  Esta no es una biyección.

  Aplicando el teorema~\ref{theo:isomorfismo-anillo-Zm}
  repetidas veces tenemos:
  \begin{corollary}
    \label{cor:isomorfismo-anillo-Zm}
    Sean \(a_1\), \(a_2\), \ldots, \(a_r\)
    naturales relativamente primos a pares%
      \index{relativamente primos!a pares}
    (o sea,
     \(\gcd(a_i, a_j) = 1\) si \(i \ne j\)),
    y \(m = a_1 a_2 \dotsm a_r\).
    Entonces la función:
    \begin{equation*}
      f([x]_m)
	= ([x]_{a_1}, [x]_{a_2}, \dotsc, [x]_{a_r})
    \end{equation*}
    es un isomorfismo de anillo entre \(\mathbb{Z}_m\) y
    \(\mathbb{Z}_{a_1} \times \mathbb{Z}_{a_2}
	\times \dotsb \times \mathbb{Z}_{a_r}\)
  \end{corollary}

  Una consecuencia inmediata es el siguiente importante teorema:
  \begin{theorem}[Teorema chino de los residuos]
    \index{residuo!teorema chino de los}
    \label{theo:chino-residuos}
    Sean \(a_1\), \(a_2\), \ldots, \(a_r\)
    naturales relativamente primos a pares,
    y \(b_1\), \(b_2\), \ldots, \(b_r\) enteros cualquiera.
    Entonces hay un entero \(x\) tal que:
    \begin{align*}
      x &\equiv b_1 \pmod{a_1} \\
      x &\equiv b_2 \pmod{a_2} \\
	&\;\;\vdots	       \\
      x &\equiv b_r \pmod{a_r}
    \end{align*}
    El entero \(x\) es único módulo \(n = a_1 a_2 \dotsm a_r\).
  \end{theorem}
  \begin{proof}
    Considere el elemento
    \(([b_1]_{a_1}, [b_2]_{a_2}, \dotsc,
       [b_r]_{a_r})\)
    en
    \(\mathbb{Z}_{a_1} \times \dotsb \times \mathbb{Z}_{a_r}\).
    Por el isomorfismo del corolario~\ref{cor:isomorfismo-anillo-Zm}
    hay un único \([x]_m \in \mathbb{Z}_m\)
    tal que \([x]_{a_1} = [b_1]_{a_1}\),
    \([x]_{a_2} = [b_2]_{a_2}\), \ldots,
    \([x]_{a_r} = [b_r]_{a_r}\),
    lo que no es más que otra forma
    de decir que hay un entero \(x\),
    único módulo \(n\),
    que cumple el sistema de ecuaciones indicado.
  \end{proof}
  Al teorema~\ref{theo:isomorfismo-anillo-Zm}
  (o también al corolario~\ref{cor:isomorfismo-anillo-Zm})
  se le debiera llamar
  el \emph{padre del teorema chino de los residuos},%
    \index{residuo!teorema chino de los!padre del}
  son estos resultados los que en realidad más se usan
  bajo ese nombre.
  En inglés se abrevia \emph{CRT},
  por \emph{\foreignlanguage{english}{Chinese Remainder Theorem}}.%
    \index{CRT@\emph{CRT}|see{residuo!teorema chino de los}}

  Usando la notación del teorema chino de los residuos,
  para cálculo concreto
  esto se puede expresar mediante lo siguiente.
  Defina \(s_i\) como
  \(s_i \cdot (n / a_i) \equiv 1 \pmod{a_i}\),
  y defina \(m_i = s_i \cdot (n / a_i)\),
  con lo que \(m_i \equiv [i = j] \pmod{a_j}\).
  Considere:
  \begin{equation*}
    x = \sum_{1 \le i \le r} m_i \cdot b_i
  \end{equation*}
  Entonces:
  \begin{align*}
    x
      &\equiv \sum_{1 \le i \le r} m_i \cdot b_i \pmod{a_k} \\
      &\equiv m_k \cdot b_k			 \pmod{a_k} \\
      &\equiv b_k				 \pmod{a_k}
  \end{align*}
  Con esto el isomorfismo
  del corolario~\ref{cor:isomorfismo-anillo-Zm}
  puede usarse en la práctica.
  La demostración del teorema~\ref{theo:chino-residuos}
  no da luces de cómo obtener el valor \(x\),
  solo asegura que si probamos todas las opciones
  hallaremos exactamente una solución.

  Para clarificar ideas,
  resolveremos un ejemplo.
  Buscamos \(x\) tal que:
  \begin{align*}
    x &\equiv 3 \pmod{5} \\
    x &\equiv 5 \pmod{7} \\
    x &\equiv 1 \pmod{9}
  \end{align*}
  Los módulos son primos entre sí,
  hay una solución única
  módulo \(n = 5 \cdot 7 \cdot 9 = 315\).
  Necesitamos los siguientes inversos:
  \begin{align*}
    s_5
      &= (315 / 5)^{-1} = 2 \text{\ en\ } \mathbb{Z}_5 \\
    s_7
      &= (315 / 7)^{-1} = 5 \text{\ en\ } \mathbb{Z}_7 \\
    s_9
      &= (315 / 9)^{-1} = 8 \text{\ en\ } \mathbb{Z}_9
  \end{align*}
  lo que da los coeficientes,
  ahora módulo \(315\):
  \begin{align*}
    m_5
      &= 2 \cdot (315 / 5) = 126 \\
    m_7
      &= 5 \cdot (315 / 7) = 225 \\
    m_9
      &= 8 \cdot (315 / 9) = 280
  \end{align*}
  Para resolver nuestro problema concreto:
  \begin{align*}
    x &= 126 \cdot 3 + 225 \cdot 5 + 280 \cdot 1 \\
      &= 1\,783 \\
      &\equiv 208 \pmod{315}
  \end{align*}

  Como ejercicio,
  dejamos el problema planteado por Sunzi%
    \index{Sunzi}%
    \index{Sun Tzu|see{Sunzi}}
  en el siglo~IV.
  \begin{verse}
    Hay cierto número de objetos cuyo número es desconocido. \\
    Dividido por \(3\), el resto es \(2\); \\
    por \(5\) el resto es \(3\); \\
    y por \(7\) el resto es \(2\). \\
    ¿Cuántos serán los objetos?
  \end{verse}

  Supongamos ahora que piden:
  \begin{align*}
    x &\equiv 2 \pmod{6} \\
    x &\equiv 1 \pmod{7} \\
    x &\equiv 3 \pmod{9}
  \end{align*}
  Los módulos no son relativamente primos,
  puede no haber solución.
  Tenemos para el par en conflicto:
  \begin{align*}
    x = 6 s + 2	  &= 9 t + 3 \\
	6 s - 9 t &= 1
  \end{align*}
  Esto último es imposible,
  ya que significaría
  que el máximo común divisor entre \(6\) y \(9\)
    \index{maximo comun divisor@máximo común divisor}
  divide a \(1\),
  pero \(\gcd(6, 9) = 3\).
  No hay solución.

  Antes de continuar,
  algunas propiedades adicionales del máximo común divisor
  y el mínimo común múltiplo.
  \begin{lemma}
    \index{maximo comun divisor@máximo común divisor!propiedades}
    \index{minimo comun multiplo@mínimo común múltiplo!propiedades}
    \label{lem:gcd-lcm}
    Se cumplen:
    \begin{align*}
      \gcd(a, \gcd(b, c))
	&= \gcd(\gcd(a, b), c) \\
      \lcm(a, \lcm(b, c))
	&= \lcm(\lcm(a, b), c)
    \end{align*}
    Además:
    \begin{align*}
      \gcd(\lcm(a_1, b), \lcm(a_2, b), \dotsc, \lcm(a_r, b))
	&= \lcm(\gcd(a_1, a_2, \dotsc, a_r), b) \\
      \lcm(\gcd(a_1, b), \gcd(a_2, b), \dotsc, \gcd(a_r, b))
	&= \gcd(\lcm(a_1, a_2, \dotsc, a_r), b)
    \end{align*}
  \end{lemma}
  Básicamente,
  las operaciones son asociativas y cumplen leyes distributivas.
    \index{operacion@operación}
  \begin{proof}
    Por el teorema fundamental de la aritmética,%
      \index{teorema fundamental de la aritmetica@teorema fundamental de la aritmética}
    todo entero puede representarse por el conjunto
    de los divisores que son potencias de números primos.
    En estos términos,
    el máximo común divisor es la intersección de sus argumentos,
    y el mínimo común múltiplo su unión.
    Las identidades indicadas
    son entonces reflejo de la asociatividad
    de la unión e intersección,
    y la distributividad de la unión sobre la intersección
    y viceversa.
  \end{proof}

  Podemos extender el teorema chino de los residuos:
  \begin{theorem}[Teorema chino de los residuos generalizado]
    \index{residuo!teorema chino de los!generalizado}
    \label{theo:chino-residuos-generalizado}
    Para \(a_1\), \(b_1\), \ldots, \(a_r\), \(b_r\) cualquiera
    sean:
    \begin{align*}
      x &\equiv a_1 \pmod{b_1} \\
      x &\equiv a_2 \pmod{b_2} \\
	&\quad \vdots	       \\
      x &\equiv a_r \pmod{b_r}
    \end{align*}
    Estas congruencias tienen solución si y solo si
    \(a_i \equiv a_j \pmod{\gcd(b_i, b_j)}\) para todo \(i\), \(j\).
    Módulo \(\lcm(b_1, b_2, \dotsc, b_r)\)
    la solución es única en tal caso.
  \end{theorem}
  \begin{proof}
    Por inducción sobre \(r\).%
      \index{demostracion@demostración!induccion@inducción}
    Cuando \(r = 1\),
    el resultado es obvio.
    Partiremos con el caso \(r = 2\)
    porque lo usaremos en el paso de inducción.
    \begin{description}
    \item[Base:]
      Tenemos las congruencias:
      \begin{align*}
	x &\equiv a_1 \pmod{b_1} \\
	x &\equiv a_2 \pmod{b_2}
      \end{align*}
      Si \(d \mid b_1\),
      claramente \(x \equiv a_1 \pmod{d}\).
      En particular,
      para \(m_2 = \gcd(b_1, b_2)\)
      debe ser:
      \begin{align*}
	x &\equiv a_1 \pmod{m_2} \\
	x &\equiv a_2 \pmod{m_2}
      \end{align*}
      con lo que no hay solución
      a menos que \(a_1 \equiv a_2 \pmod{m_2}\).
      Esta es la condición sobre los \(b_i\) para el caso \(r = 2\).

      Si \(a_1 \equiv a_2 \pmod{m_2}\),
      por la identidad de Bézout%
	\index{Bezout, identidad de@Bézout, identidad de}
      sabemos que existen enteros \(u_2\) y \(v_2\) tales que:
      \begin{equation*}
	u_2 b_1 + v_2 b_2
	  = \gcd(b_1, b_2)
	  = m_2
      \end{equation*}
      Como \(a_1 \equiv a_2 \pmod{m_2}\),
      hay \(c_2 \in \mathbb{Z}\)
      tal que \(a_1 - a_2 = c_2 \cdot m_2\),
      y \(a_1 - a_2 = c_2 u_2 b_1 + c_2 v_2 b_2\).
      Con esto:
      \begin{equation*}
	s_2
	  = a_1 - c_2 u_2 b_1
	  = a_2 + c_2 v_2 b_2
      \end{equation*}
      cumple ambas congruencias.

      Para demostrar que es única,
      consideremos soluciones \(s\) y \(s'\).
      Vemos que
      \(s \equiv s' \pmod{b_1}\)
      y \(s \equiv s' \pmod{b_2}\),
      y el teorema~\ref{theo:congruencia-mn} asegura que
      \(s \equiv s' \pmod{\lcm(b_1, b_2)}\).
    \item[Inducción:]
      Suponiendo que vale para \(r\) congruencias,
      demostramos que vale para \(r + 1\):
      \begin{align*}
	x &\equiv a_{1\phantom{+1}} \pmod{b_1} \\
	x &\equiv a_{2\phantom{+1}} \pmod{b_2} \\
	  &\quad \vdots		 \\
	x &\equiv a_{r\phantom{+1}} \pmod{b_r} \\
	x &\equiv a_{r + 1} \pmod{b_{r + 1}}
      \end{align*}
      Sea \(s_r\) la solución a las primeras \(r\) congruencias,
      que por inducción existe
      y es única módulo \(\lcm(b_1, b_2, \dotsc, b_r)\).
      Consideremos las congruencias:
      \begin{align*}
	x &\equiv s_r \pmod{\lcm(b_1, b_2, \dotsc, b_r)} \\
	x &\equiv a_{r + 1} \pmod{b_{r + 1}}
      \end{align*}
      Por el caso \(r = 2\) sabemos que hay solución únicamente si:
      \begin{equation*}
	s_{r\phantom{+1}} \equiv a_{r + 1}
	  \pmod{\gcd(\lcm(b_1, \dotsc, b_r), b_{r + 1})}
      \end{equation*}
      Por el lema~\ref{lem:gcd-lcm}:
      \begin{equation*}
	\gcd(\lcm(b_1, \dotsc, b_r), b_{r + 1})
	  = \lcm(\gcd(b_1, b_{r + 1}), \gcd(b_2, b_{r + 1}),
		 \dotsc, \gcd(b_r, b_{r + 1}))
      \end{equation*}
      que es equivalente a:
      \begin{equation*}
	a_{r + 1} \equiv a_i \pmod{\gcd(b_i, b_{r + 1})}
	  \quad\text{para todo \(1 \le i \le r\)}
      \end{equation*}
      Esto extiende la condición sobre los \(a_i\).

      De cumplirse la condición sobre los \(a_i\),
      hay una solución \(s_{r + 1}\) única módulo
      \(\lcm(b_1, \dotsc, b_{r + 1})\),
      que podemos calcular como antes.
      Sean \(u_{r + 1}\) y \(v_{r + 1}\) tales que:
      \begin{equation*}
	u_{r + 1} s_r + v_{r + 1} b_{r + 1}
	  = \gcd(s_r, b_{r + 1})
      \end{equation*}
      Como \(s_r \equiv a_{r + 1} \pmod{b_{r + 1}}\),
      existe \(c_{r + 1} \in \mathbb{Z}\) en
      \(s_r - a_{r + 1} = c_{r + 1} \gcd(s_r, b_{r + 1})\),
      y \(s_{r + 1}\)
      definido como sigue cumple las \(r + 1\) congruencias:
      \begin{equation*}
	s_{r + 1}
	  = s_r - c_{r + 1} u_{r + 1} \lcm(b_1, \dotsc, b_r)
	  = a_{r + 1} + c_{r + 1} v_{r + 1} b_{r + 1}
      \end{equation*}
    \end{description}
    Por inducción lo indicado vale para \(r \in \mathbb{N}\).
  \end{proof}

  La demostración da un algoritmo para obtener la solución.
  Por ejemplo,
  consideremos el sistema:
  \begin{align*}
    x &\equiv 3 \pmod{4} \\
    x &\equiv 5 \pmod{6} \\
    x &\equiv 2 \pmod{9}
  \end{align*}
  Para las primeras dos congruencias tenemos:
  \begin{equation*}
    4 u_2 + 6 v_2
      = \gcd(4, 6)
      = 2
  \end{equation*}
  Obtenemos \(u_2 = -1\), \(v_2 = 1\),
  y tenemos \(a_1 - a_2 = 3 - 5 = -2\) que da \(c_2 = -1\),
  por lo que:
  \begin{equation*}
    s_2
      = 3 - (-1) (-1) 4
      = -1
  \end{equation*}
  Como \(\lcm(4, 6) = 12\),
  para el segundo paso queda el sistema:
  \begin{align*}
    x &\equiv -1	    \pmod{12} \\
    x &\equiv \phantom{-}7  \pmod{\phantom{0}9}
  \end{align*}
  Tenemos:
  \begin{equation*}
    12 u_3 + 9 v_3
      = \gcd(12, 9)
      = 3
  \end{equation*}
  Esto resulta en \(u_3 = 1\) y \(v_3 = -1\),
  para \(s_2 - a_3 = -1 - 2 = -3\) es \(c_3 = -1\),
  y queda:
  \begin{equation*}
    s_3
      = -1 - (-1) \cdot 1 \cdot 12
      = 11
  \end{equation*}
  La solución es única módulo \(\lcm(4, 6, 9) = 36\).

  El algoritmo implícito
  en el teorema~\ref{theo:chino-residuos-generalizado}
  es bastante engorroso.
  Una forma diferente de enfocar el tema es dividir las congruencias
  según los máximos comunes divisores.
  Veamos el ejemplo:
  \begin{align*}
    x &\equiv \phantom{0}9 \pmod{12} \\
    x &\equiv		12 \pmod{21}
  \end{align*}
  Tenemos \(\gcd(12, 21) = 3\),
  con lo que \(12 = 3 \cdot 4\) y \(21 = 3 \cdot 7\).
  La primera congruencia se descompone:
  \begin{align*}
    x &\equiv 9 \equiv 0 \pmod{3} \\
    x &\equiv 9 \equiv 1 \pmod{4}
  \end{align*}
  La segunda da:
  \begin{align*}
    x &\equiv 12 \equiv 0 \pmod{3} \\
    x &\equiv 12 \equiv 5 \pmod{7}
  \end{align*}
  Las congruencias comunes
  (módulo~\(3\)) son consistentes,
  hay solución módulo \(\lcm(12, 21) = 84\).
  El sistema se reduce a:
  \begin{align*}
    x &\equiv 0 \pmod{3} \\
    x &\equiv 1 \pmod{4} \\
    x &\equiv 5 \pmod{7}
  \end{align*}
  El teorema chino de los residuos da:%
    \index{residuo!teorema chino de los}
  \begin{alignat*}{2}
    s_3 &= (4 \cdot 7)^{-1} = 1
	&\quad m_3 &= 1 \cdot 4 \cdot 7 = 28 \\
    s_4 &= (3 \cdot 7)^{-1} = 1
	&\quad m_4 &= 1 \cdot 3 \cdot 7 = 21 \\
    s_7 &= (3 \cdot 4)^{-1} = 3
	&\quad m_7 &= 3 \cdot 3 \cdot 4 = 36
  \end{alignat*}
  y la solución es:
  \begin{align*}
    x &= 0 \cdot 28 + 1 \cdot 21 + 5 \cdot 36 \\
      &\equiv 33 \pmod{84}
  \end{align*}

  Un problema en esta línea planteó Brahmagupta en el siglo~VII.%
    \index{Brahmagupta}
  \begin{verse}
    Una anciana va al mercado,
    y un caballo pisa su canasto y le aplasta los huevos. \\
    \hspace{1em}El jinete ofrece pagar el daño
    y le pregunta cuántos huevos traía. \\
    \hspace{1em}Ella no recuerda el número exacto,
    pero al sacarlos de a dos sobraba un huevo.
    Lo mismo ocurría
    si los sacaba de a tres, cuatro, cinco y seis a la vez,
    pero al sacarlos de a siete no sobró ninguno. \\
    ¿Cuál es el mínimo número de huevos que podría haber tenido?
  \end{verse}

  Una aplicación adicional es la \emph{prueba del once}:
  Vimos antes
  (sección~\ref{sec:descomposiciones})
  que una manera de verificar operaciones aritméticas
  es la prueba de los nueves,
  que es simple de aplicar
  porque calcular el residuo módulo nueve
  de un número escrito en decimal
  es sumar sus dígitos,
  repitiendo el proceso hasta reducir a uno solo.
  Resulta que calcular el residuo módulo once
  es sumar y restar alternativamente los dígitos
  comenzando por el menos significativo:
  Como \(10 \equiv -1 \pmod{11}\),
  tenemos:
  \begin{equation*}
    \sum_{0 \le k \le n} d_k \cdot 10^k
      \equiv \sum_{0 \le k \le n} (-1)^k d_k \pmod{11}
  \end{equation*}
  Si aplicamos la prueba del nueve y la prueba del once,
  como \(\gcd(9, 11) = 1\),
  estamos verificando el resultado módulo \(9 \cdot 11 = 99\).

  El teorema~\ref{theo:chino-residuos}
  (más bien,
   el corolario~\ref{cor:isomorfismo-anillo-Zm})
  ofrece una importante estrategia adicional
  para demostrar teoremas en \(\mathbb{Z}\):
  \begin{enumerate}
  \item
    Primeramente,
    demuestre el resultado para \(p\) primo.
  \item
    Enseguida,
    demuestre que es válido para \(p^\alpha\),
    potencias de primos.
  \item
    Use el (padre del) teorema chino de los residuos
    para combinar los resultados anteriores
    y obtener el caso general.
  \end{enumerate}
  Más adelante aparecerán muchas aplicaciones de esta idea.

\section[Estructura de
	   \texorpdfstring{$\mathbb{Z}^\times_m$}
			  {las unidades de clases de congruencia}]
	{\protect\boldmath
	 Estructura de
	  \texorpdfstring{$\mathbb{Z}^\times_m$}
				{las unidades de clases de congruencia}%
       \protect\unboldmath}
\label{sec:estructura-Un}

  El isomorfismo del teorema~\ref{theo:isomorfismo-anillo-Zm}
  permite demostrar:
  \begin{theorem}
    \label{theo:Z*ab=Z*a+Z*b}
    Si \(a\) y \(b\) son naturales relativamente primos,
    entonces:
    \begin{equation*}
      \mathbb{Z}^\times_{a b}
	\cong \mathbb{Z}^\times_a \times \mathbb{Z}^\times_b
    \end{equation*}
  \end{theorem}
  \begin{proof}
    Sabemos que \(\mathbb{Z}_{a b}\)
    y \(\mathbb{Z}_a \times \mathbb{Z}_b\)
    son anillos isomorfos,
    con lo que \(\mathbb{Z}^\times_{a b}\)
    es isomorfo al grupo de unidades
    de \(\mathbb{Z}_a \times \mathbb{Z}_b\).
    Ahora bien,
    un elemento de \(\mathbb{Z}_a \times \mathbb{Z}_b\)
    es invertible si lo son sus componentes:
    \begin{equation*}
      (x, y) \cdot (x', y')
	= (x x', y y')
	= (1, 1)
    \end{equation*}
    con \(x \in \mathbb{Z}^\times_a\)
    e \(y \in \mathbb{Z}^\times_b\),
    con lo que el grupo de unidades
    de \(\mathbb{Z}_a \times \mathbb{Z}_b\)
    es exactamente
      \(\mathbb{Z}^\times_a \times \mathbb{Z}^\times_b\).
  \end{proof}
  Como corolario,
  tenemos para la función \(\phi\) de Euler:%
    \index{\(\phi\) de Euler}
  \begin{corollary}
    \label{cor:phi-multiplicativa}
    Sea \(\phi\) la función de Euler.
    Si \(a\) y \(b\) son naturales relativamente primos,
    entonces \(\phi(a b) = \phi(a) \cdot \phi(b)\)
  \end{corollary}
  \begin{proof}
    Del teorema~\ref{theo:Z*ab=Z*a+Z*b} sabemos que:
    \begin{equation*}
      \phi(a b)
	= \lvert \mathbb{Z}^\times_{a b} \rvert
	= \lvert \mathbb{Z}^\times_a
		   \times \mathbb{Z}^\times_b \rvert
	= \lvert \mathbb{Z}^\times_a \rvert
	    \cdot \lvert \mathbb{Z}^\times_b \rvert
	= \phi(a) \cdot \phi(b)
	\qedhere
    \end{equation*}
  \end{proof}
  Esta propiedad es importante:
  \begin{definition}
    \index{funcion@función!aritmetica@aritmética|textbfhy}
    \index{funcion@función!aritmetica@aritmética!multiplicativa|textbfhy}
    Una función \(f \colon \mathbb{N} \rightarrow \mathbb{C}\)
    se llama \emph{aritmética}.
    Anotaremos \(\mathscr{A}\)
    para el conjunto de funciones aritméticas.
    Una función aritmética \(f\)
    se llama \emph{multiplicativa}
    si \(f(a \cdot b) = f(a) \cdot f(b)\)
    siempre que \(\gcd(a, b) = 1\).
    A su conjunto lo llamamos \(\mathscr{M}\).
  \end{definition}
  \noindent
  Supongamos que \(f\) es multiplicativa,
  y que para algún \(n \in \mathbb{N}\) es \(f(n) \ne 0\).
  Como \(\gcd(1, n) = 1\):
  \begin{equation*}
    f(n)
      = f(n \cdot 1)
      = f(n) \cdot f(1)
  \end{equation*}
  con lo que \(f(1) = 1\) o \(f(n) = 0\)
  para todo \(n = \mathbb{N}\).

  Por el teorema fundamental de la aritmética%
    \index{teorema fundamental de la aritmetica@teorema fundamental de la aritmética}
  todo entero se puede descomponer
  en un producto de potencias de primos distintos.
  Como potencias de primos diferentes son relativamente primas,
  una función multiplicativa queda determinada por su valor
  para potencias de primos.

  Como acabamos de demostrar que \(\phi\) es multiplicativa,
  tenemos una manera de calcularla:
  \begin{corollary}
    \label{cor:calcular-phi}
    Sea \(n = p_1^{\alpha_1} p_2^{\alpha_2} \dotsm p_r^{\alpha_r}\)
    la factorización completa de \(n\) en primos distintos \(p_i\).
    Entonces:
    \begin{align*}
      \phi(n)
	&= p_1^{\alpha_1 - 1} (p_1 - 1) p_2^{\alpha_2 - 1} (p_2 - 1)
	     \dotsm p_r^{\alpha_r - 1} (p_r - 1) \\
	&= n \cdot \left(1 - \frac{1}{p_1}\right) \cdot
		     \left(1 - \frac{1}{p_2}\right) \dotsm
		     \left(1 - \frac{1}{p_r}\right)
    \end{align*}
  \end{corollary}
  \begin{proof}
    Del corolario~\ref{cor:phi-multiplicativa} sabemos que
      \(\phi(n)
	  = \phi(p_1^{\alpha_1}) \cdot \phi(p_2^{\alpha_2})
	      \dotsm \phi(p_r^{\alpha_r})\).
    Necesitamos el valor de \(\phi(p^\alpha)\),
    para \(p\) primo y \(\alpha\) natural.
    Hay \(p^k\) números entre \(1\) y \(p^k\),
    no son relativamente primos a \(p^\alpha\)
    los \(p^{\alpha - 1}\) múltiplos de \(p\) en este rango:
    \begin{equation*}
      \phi(p^\alpha)
	= p^\alpha - p^{\alpha - 1}
	= p^{\alpha - 1} (p - 1)
	= p^\alpha \cdot \left(1 - \frac{1}{p}\right)
    \end{equation*}
    Multiplicando esto
    sobre las potencias de primos factores de \(n\)
    da lo anunciado.
  \end{proof}
  Algunas funciones aritméticas interesantes adicionales son:
  \begin{description}
  \item[La identidad:]
    \(\iota(n) = n\)
  \item[\boldmath Potencias de \(n\):\unboldmath]
    \(\iota_a(n) = n^a\)
  \item[\boldmath El número de divisores de \(n\):\unboldmath]
    \(\tau(n)
	= \lvert \{ d \in \mathbb{N} \colon d \mid n \} \rvert\)
  \item[\boldmath La suma de los divisores de \(n\):\unboldmath]
    \(\sigma(n)
	= \sum_{d \mid n} d\)
  \item[\boldmath El producto de los divisores de \(n\):\unboldmath]
    \(\pi(n)
	= \prod_{d \mid n} d\)
  \end{description}
  Acá hemos usado nuestra convención general de indicar los índices
  de sumas o productos mediante condiciones,
  en este caso de divisibilidad.

  Un resultado importante para funciones multiplicativas es:
  \begin{theorem}
    \label{theo:sum-multiplicative}
    Sea \(f\) una función aritmética%
      \index{funcion@función!aritmetica@aritmética!multiplicativa}
    y \(S\) definida por:%
      \index{funcion@función!aritmetica@aritmética!funcion suma@función suma|textbfhy}
    \begin{equation*}
      S(n)
	= \sum_{d \mid n} f(d)
    \end{equation*}
    Entonces \(f\) es multiplicativa si y solo si lo es \(S\).
  \end{theorem}
  \begin{proof}
    Demostramos implicancia en ambas direcciones.
    Sean \(x, y \in \mathbb{N}\) relativamente primos,
    y sea \(f\) multiplicativa.
    Sean además \(x_1, x_2, \dotsc, x_r\)
    e \(y_1, y_2, \dotsc, y_s\)
    todos los divisores de \(x\) e \(y\),
    respectivamente.
    Entonces \(\gcd(x_i, y_j) = 1\),
    y \(\{x_i y_j\}_{i, j}\) son todos los divisores de \(x y\):
    \begin{equation*}
      S(x) \cdot S(y)
	= \sum_i f(x_i) \sum_j f(y_j)
	= \sum_{i, j} f(x_i) f(y_j)
	= \sum_{i, j} f(x_i y_j)
	= S(x y)
    \end{equation*}
    y \(S\) es multiplicativa.

    Para el recíproco,
    sea \(S\) multiplicativa.
    Demostramos por inducción fuerte sobre \(n\)%
      \index{demostracion@demostración!induccion@inducción}
    que cuando \(n = x y\) con \(\gcd(x, y) = 1\)
    es \(f(n) = f(x) f(y)\).
    \begin{description}
    \item[Base:]
      El caso \(n = 1\) es trivial:
      \(f(1) = S(1)\).
    \item[Inducción:]
      Para nuestros \(x\) e \(y\) tenemos,
      usando la hipótesis de inducción:
      \begin{align*}
	S(x y)
	  &= \sum_{\substack{
		     u \mid x \\
		     v \mid y
		  }} f(u v)
	   = \sum_{\substack{
		    u \mid x \\
		    v \mid y \\
		    u v < n
		 }} f(u) f(v) + f(x y) \\
      \intertext{Por otro lado,
		 sacando de las sumatorias los términos
		 para \(u = x\) y \(v = y\) queda:}
	S(x) S(y)
	  &= \sum_{u \mid x} f(u) \sum_{v \mid y} f(v)
	   = \sum_{\substack{
		    u \mid x \\
		    v \mid y \\
		    u v < n
		 }} f(u) f(v) + f(x) f(y)
      \end{align*}
      Ambas expresiones son iguales ya que \(S\) es multiplicativa,
      y es \(f(x y) = f(x) f(y)\).
    \end{description}
    Por inducción vale para todo \(n \in \mathbb{N}\).
  \end{proof}
  \begin{corollary}
    \label{cor:formula-sum-function}
    Sea \(f\) una función multiplicativa,%
      \index{funcion@función!aritmetica@aritmética!multiplicativa}
    y sea \(S\) su función suma:%
      \index{funcion@función!aritmetica@aritmética!funcion suma@función suma|textbfhy}
    \begin{equation*}
      S (n)
	= \sum_{d \mid n} f(d)
    \end{equation*}
    Si \(n = p_1^{\alpha_1} p_2^{\alpha_2} \dotsm p_r^{\alpha_r}\)
    es la descomposición de \(n\)
    en factores primos distintos \(p_i\),
    entonces:
    \begin{equation*}
      S (n)
	= \prod_{1 \le i \le r}
	    \left(
	      1 + f(p_i) + f(p_i^2) + \dotsb + f(p_i^{\alpha_i})
	    \right)
    \end{equation*}
  \end{corollary}
  \begin{proof}
    Si \(f\) es multiplicativa,
    lo es \(S\).
    El valor indicado de \(S(n)\)
    corresponde para potencias de primos.
  \end{proof}
  Del teorema~\ref{theo:sum-multiplicative}
  vemos que son multiplicativas:%
    \index{funcion@función!aritmetica@aritmética!multiplicativa}%
    \index{\(\tau\) (numero de divisores)@\(\tau\) (número de divisores)}%
    \index{\(\sigma\) (suma de divisores)}
  \begin{align*}
    \tau(n)
      &= S_1 (n)
       = \sum_{d \mid n} 1 \\
    \sigma(n)
      &= S_{\iota}(n)
       = \sum_{d \mid n} d
  \end{align*}
  Por el corolario~\ref{cor:formula-sum-function}
  en términos de la factorización completa
    \(n = p_1^{\alpha_1} p_2^{\alpha_2} \dotsm p_r^{\alpha_r}\)
  tenemos:
  \begin{align*}
    \tau(n)
      &= \prod_{1 \le i \le r} (\alpha_i + 1) \\
    \sigma(n)
      &= \prod_{1 \le i \le r}
	   \frac{p_i^{\alpha_i + 1} - 1}{p_i - 1}
  \end{align*}
  La función \(\pi(n)\) no es multiplicativa.

  Para los griegos
  la relación de un número con sus divisores propios
  tenía relevancia mística.
  Así reverenciaban especialmente a los \emph{números perfectos},%
    \index{numero@número!perfecto|textbfhy}
  que son la suma de sus divisores propios.
  Conocían los casos \(6 = 1 + 2 + 3\),
  \(28 = 1 + 2 + 4 + 7 + 14\),
  496 y 8128.
  En términos de las funciones definidas antes,
  \(n\) es perfecto cuando \(\sigma(n) = 2 n\)
  (los factores propios de \(n\) suman a \(n\),
   con \(n\) suman \(2 n\)).
  Tenemos también,
  si \(p\) es primo:
  \begin{align*}
    \sigma(p)
      &= p + 1 \\
    \sigma(p^\alpha)
      &= 1 + p + \dotsb + p^\alpha
       = \frac{p^{\alpha + 1} - 1}{p - 1}
  \end{align*}
  Del resultado siguiente cada uno de los participantes
  demostró una implicancia.
  \begin{theorem}[Euclides -- Euler]
    \index{numero@número!perfecto!par}
    \label{theo:even-perfect-numbers}
    Un par \(n\) es perfecto si y solo si
    \(n = 2^{m - 1} \left( 2^m - 1 \right)\) con \(2^m - 1\) primo.
  \end{theorem}
  \begin{proof}
    Demostramos implicancia en ambas direcciones.

    Si \(n = 2^{m - 1} \left( 2^m - 1 \right)\)
    con \(2^m - 1\) primo,
    entonces como \(\sigma\) es multiplicativa:%
      \index{funcion@función!aritmetica@aritmética!multiplicativa}
    \begin{equation*}
      \sigma(n)
	= \sigma \left( 2^{m - 1} \right)
	    \sigma \left( 2^m - 1 \right)
	= \frac{2^m - 1}{2 - 1} \cdot \left( (2^m - 1) + 1 \right)
	= 2^m \left( 2^m - 1 \right)
	= 2 n
    \end{equation*}
    y \(n\) es perfecto.
    Esta parte fue demostrada por Euclides.%
      \index{Euclides}

    Para el recíproco,
    sea \(n = 2^{m - 1} u\) un número perfecto
    con \(m > 1\) y \(u\) impar.
    Entonces:
    \begin{align*}
      2^m u
	&= \sigma \left( 2^{m - 1} u \right)
	 = \left( 2^m - 1 \right) \sigma(u) \\
      \sigma(u)
	&= \frac{2^m u}{2^m - 1}
	 = u + \frac{u}{2^m - 1}
    \end{align*}
    Claramente el último término es un divisor de \(u\).
    Como \(m > 1\),
    \(2^m - 1 > 1\).
    O sea,
    estamos expresando \(\sigma(u)\)%
      \index{\(\sigma\) (suma de divisores)}
    como la suma de dos divisores distintos de \(u\),
    por lo que \(u\) es primo;
    tiene que ser \(u = 2^m - 1\).
    Este es el aporte de Euler.%
      \index{Euler, Leonhard}
  \end{proof}
  Esto resuelve completamente el caso de números perfectos pares.
  Es fácil ver que si \(m\) es compuesto
  lo es \(2^m - 1\),
  por lo que basta considerar \(2^p - 1\) con \(p\) primo.
  A tales primos se les llama \emph{primos de Mersenne},%
    \index{Mersenne, primo de}
  quien los estudió a principios del siglo~XVII.
  Se conocen 48 primos de Mersenne a febrero de 2013,
  incluso el mayor primo conocido a la fecha
  es \(2^{57\,885\,161} - 1\).

  Es primo \(2^2 - 1 = 3\)
  y \(2^{2 - 1} (2^2 - 1) = 6\) es perfecto.
  Asimismo \(2^{3 - 1} (2^3 - 1) = 28\),
  \(2^{5 - 1} (2^5 - 1) = 496\)
  y \(2^{7 - 1} (2^7 - 1) = 8128\)
  son perfectos.

  Por el otro lado,
  determinar si hay números perfectos impares%
    \index{numero@número!perfecto!impar}
  es un problema abierto
  desde antes de Euclides.

  \begin{definition}
    Sean \(f\) y \(g\) funciones aritméticas.
    Su \emph{convolución de Dirichlet} es:%
      \index{Dirichlet, convolucion de@Dirichlet, convolución de|textbfhy}
    \begin{equation}
      \label{eq:Dirichlet-convolution}
      (f * g)(n)
	= \sum_{d \mid n} f(d) g(n / d)
	= \sum_{a b = n} f(a) g(b)
    \end{equation}
  \end{definition}
  Es claro que la operación \(*\) es conmutativa,%
    \index{operacion@operación!conmutativa}
  y es fácil demostrar que es asociativa.%
    \index{operacion@operación!asociativa}
  Con:
  \begin{equation}
    \index{\(\epsilon\) (identidad para convolucion de Dirichlet)@\(\epsilon\) (identidad para convolución de Dirichlet)}
    \label{eq:arithmetic-epsilon}
    \epsilon(n)
      = \begin{cases}
	  1 & \text{si \(n = 1\)} \\
	  0 & \text{si \(n > 1\)}
	\end{cases}
  \end{equation}
  para la función aritmética \(f\)
  tenemos \(f * \epsilon = \epsilon * f = f\),
  lo que da un neutro multiplicativo.%
    \index{operacion@operación!elemento neutro}
  En particular:
  \begin{theorem}
    \label{theo:multiplicative-group-Dirichlet}
    El conjunto \(\mathscr{M}\)
    de funciones aritméticas multiplicativas
    es cerrado respecto de la convolución de Dirichlet.
  \end{theorem}
  \begin{proof}
    Sean \(f, g \in \mathscr{M}\),
    sea \(h = f * g\),
    y \(a, b \in \mathbb{N}\) con \(\gcd(a, b) = 1\).
    Como \(\gcd(a, b) = 1\)
    los factores de \(a b\)
    resultan de todas las combinaciones de factores de \(a\) y \(b\)
    por separado:
    \begin{align*}
      h(a) h(b)
	&= (f * g)(a) \cdot (f * g)(b) \\
	&= \sum_{u_1 v_1 = a}
	     f(u_1) g(v_1) \sum_{u_2 v_2 = b} f(u_2) g(v_2) \\
	&= \sum_{\substack{u_1 v_1 = a \\ u_2 v_2 = b}}
	     f(u_1) g(v_1) f(u_2) g(v_2) \\
	&= \sum_{u_1 u_2 v_1 v_2 = a b}
	     f(u_1) f(u_2) g(v_1) g(v_2) \\
	&= \sum_{u v = a b} f(u) g(v) \\
	&= (f * g)(a b) \\
	&= h(a b)
    \qedhere
    \end{align*}
  \end{proof}
  Incluso podemos calcular inversos.
  \begin{lemma}
    \label{lem:Dirichlet-inverse}
    Toda función aritmética tal que \(f(1) \ne 0\)
    tiene inversa de Dirichlet
    dada por:
    \begin{equation*}
      f^{-1}(n)
	= \begin{cases}
	    \displaystyle
		 \frac{1}{f(1)}
	       & \text{si \(n = 1\)} \\
	    \\
	    \displaystyle
		 - \frac{1}{f(1)}
		     \, \sum_{\substack{a b = n \\ b < n}}
				    f(a) f^{-1}(b)
	       & \text{si \(n > 1\)}
	  \end{cases}
    \end{equation*}
  \end{lemma}
  \begin{proof}
    Corresponde a plantear el sistema de ecuaciones,
    escrito usando la convención de Iverson:
    \begin{align*}
      \epsilon
	&= f * f^{-1} \\
      [n = 1]
	&= \sum_{a b = n} f(a) f^{-1}(b)
    \end{align*}
    La expresión indicada para \(f^{-1}\) satisface este sistema.
  \end{proof}
  Una función importante es:
  \begin{definition}
    \index{Mobius, funcion de@Möbius, función de|textbfhy}
    \index{\(\mu\)|see{Möbius, función de}}
    \index{Mobius, August Ferdinand@Möbius, August Ferdinand}
    La \emph{función de Möbius}
    se define mediante:
    \begin{equation}
      \label{eq:Moebius-mu}
      \mu(n)
	= \begin{cases}
	    1	   &
	       \text{si \(n = 1\)} \\
	    0	   &
	       \text{si \(n\)
		     es divisible por el cuadrado de un primo} \\
	    (-1)^k &
	       \text{si \(n\)
		     es el producto de \(k\) primos diferentes}
	  \end{cases}
    \end{equation}
  \end{definition}
  Un momento de reflexión muestra que \(\mu\) es multiplicativa.
    \index{funcion@función!aritmetica@aritmética!multiplicativa}
  \begin{lemma}
    \label{lem:sum-mu}
    Para \(n \in \mathbb{N}\),
    la función de Möbius satisface:
    \begin{equation*}
      \sum_{d \mid n} \mu(d)
	= \begin{cases}
	    1 & \text{si \(n = 1\)} \\
	    0 & \text{si \(n \ge 2\)}
	  \end{cases}
    \end{equation*}
  \end{lemma}
  \begin{proof}
    Como \(\mu\) es multiplicativa,
    por el teorema~\ref{theo:sum-multiplicative}
    lo es la suma indicada.
    Basta entonces hallar el valor de la suma
    en potencias de un primo \(p\).
    Hay dos casos a considerar:
    \begin{description}
    \item[\boldmath \(p^0 = 1\):\unboldmath]
      En este caso la suma es simplemente \(\mu(1) = 1\).
    \item[\boldmath \(p^\alpha\), con \(\alpha \ge 1\):\unboldmath]
      Acá,
      como \(\mu(p^k) = 0\) si \(k > 1\):
      \begin{equation*}
	\sum_{d \mid p^\alpha} \mu(d)
	  = \mu(1) + \mu(p) + \mu(p^2) + \dotsb + \mu(p^\alpha)
	  = 1 + (-1) + 0 + \dotsb + 0
	  = 0
      \end{equation*}
    \end{description}
    Multiplicando sobre los factores primos de \(n\)
    se obtiene lo prometido.
  \end{proof}
  La curiosa definición de \(\mu\)
  resulta ser simplemente el inverso de la función \(1\),
  \(1 * \mu = \epsilon\),
  cosa que puede verificarse
  usando el lema~\ref{lem:Dirichlet-inverse}.
  En detalle,
  llamando \(\mu = 1^{-1}\),
  tenemos:
  \begin{description}
  \item[\boldmath \(n = 1\):\unboldmath]
    Es \(\mu(1) = 1 / 1(1) = 1\).
  \item[\boldmath \(n > 1\):\unboldmath]
    En general es:
    \begin{equation*}
      \mu(n)
	= - \frac{1}{1} \,
	      \sum_{\substack{a b = n \\ b < n}}
		1 \cdot \mu(b)
	= - \sum_{\substack{d \mid n \\ d < n}} \mu(d)
    \end{equation*}
    O sea,
    sucesivamente por el lema~\ref{lem:Dirichlet-inverse}:
    \begin{align*}
      \mu(2)
	&= - \sum_{\substack{d \mid 2 \\ d < 2}} \mu(d)
	 = - \mu(1)
	 = -1 \\
      \mu(3)
	&= - \sum_{\substack{d \mid 3 \\ d < 3}} \mu(d)
	 = - \mu(1)
	 = -1 \\
      \mu(4)
	&= - \sum_{\substack{d \mid 4 \\ d < 4}} \mu(d)
	 = - (\mu(1) + \mu(2))
	 = 0 \\
      \mu(5)
	&= - \sum_{\substack{d \mid 5 \\ d < 5}} \mu(d)
	 = - \mu(1)
	 = -1 \\
      \mu(6)
	&= - \sum_{\substack{d \mid 6 \\ d < 6}} \mu(d)
	 = - (\mu(1) + \mu(2) + \mu(3))
	 = 1
    \end{align*}
  \end{description}
  Es claro que calcular la inversa
  por esta vía es bastante engorroso.

  Esto hace útil la función de Möbius:
  \begin{theorem}[Inversión de Möbius]
    \index{Mobius, inversion de@Möbius, inversión de|textbfhy}
    \label{theo:Moebius-inversion}
    Sean dos funciones aritméticas
    (no necesariamente multiplicativas)
    tales que para todo \(n \in \mathbb{N}\) se cumple:
    \begin{equation*}
      g(n)
	= \sum_{d \mid n} f(d)
    \end{equation*}
    entonces para todo \(n \in \mathbb{N}\):
    \begin{equation*}
      f(n)
	= \sum_{d \mid n} \mu(d) g(n / d)
    \end{equation*}
  \end{theorem}
  \begin{proof}
    Tenemos:
    \begin{align*}
      g
	&= 1 * f \\
      f
	&= \mu * g
      \qedhere
    \end{align*}
  \end{proof}
  También:
  \begin{lemma}
    \label{lem:Dirichlet-fg=>f}
    Si \(g\) es multiplicativa,
    y lo es \(f * g\),
    entonces es multiplicativa \(f\).
  \end{lemma}
  \begin{proof}
    Si alguna de las funciones es cero,
    el resultado es obvio.
    En caso contrario,
    la demostración es por contradicción.
    Definamos \(h = f * g\) para comodidad.
    Suponemos que \(f\) no es multiplicativa,
    con lo que existen \(m\), \(n\) mínimos con \(\gcd(m, n) = 1\)
    tales que \(f(m n) \ne f(m) f(n)\).
    No puede ser \(m n = 1\),
    ya que \(h(1) = f(1) g(1)\),
    como \(h\) y \(g\) son multiplicativas,
    \(h(1) = g(1) = 1\),
    con lo que \(f(1) = 1\) y \(f(1) = f(1) f(1)\).

    Sabemos entonces que \(m n \ne 1\).
    Calcularemos \(h(m n) = h(m) h(n)\) de dos maneras,
    dejando fuera el término que involucra a \(m n\) con \(f\)
    en ambos casos.
    Comparando ambas llegaremos a una contradicción.
    \begin{align*}
      h(m n)
	&= \sum_{u v = m n} f(u) g(v) \\
	&= \sum_{\substack{u v = m n \\ u < m n}} f(u) g(v)
	     + f(m n) g(1) \\
	&= \sum_{\substack{u v = m n \\ u < m n}} f(u) g(v)
	     + f(m n) \\
      h(m) h(n)
	&= \sum_{u_1 v_1 = m} f(u_1) g(v_1)
	     \sum_{u_2 v_2 = n} f(u_2) g(v_2) \\
	&= \sum_{\substack{u_1 v_1 = m \\ u_2 v_2 = n}}
	     f(u_1) g(v_1) f(u_2) g(v_2) \\
	&= \sum_{u_1 u_2 v_1 v_2 = m n}
	     f(u_1) f(u_2) g(v_1) g(v_2) \\
	&= \sum_{\substack{u_1 u_2 v_1 v_2 = m n \\
			   u_1 < m \\
			   u_2 < n}}
	     f(u_1) f(u_2) g(v_1) g(v_2)
	     + f(m) f(n) g(1) g(1) \\
    \intertext{Pero \(f\) es multiplicativa hasta antes de \(m n\),
	       y \(g\) es multiplicativa:}
	&= \sum_{\substack{u v = m n \\ u < m n}}
	     f(u) g(v)
	     + f(m) f(n) \\
      h(m n) - h(m) h(n)
	&= f(m n) - f(m) f(n) \\
	&\ne 0
    \end{align*}
    Tenemos una contradicción,
    \(f * g\) no es multiplicativa,
    contrario a la hipótesis.
  \end{proof}
  \begin{corollary}
    \label{cor:Dirichlet-inverse-multiplicative}
    La inversa de una función multiplicativa es multiplicativa.
  \end{corollary}
  \begin{proof}
    Si \(f\) es multiplicativa,
    entonces \(f * f^{-1} = \epsilon\)
    cumplen las hipótesis del lema~\ref{lem:Dirichlet-fg=>f}.
  \end{proof}
  Uniendo las piezas:
  \begin{theorem}
    \label{theo:Dirichlet-ring}
    El conjunto de funciones aritméticas%
      \index{funcion@función!aritmetica@aritmética}
    \(\mathscr{A}\) es un anillo conmutativo%
      \index{anillo!conmutativo}
    con suma de funciones
    y convolución de Dirichlet como multiplicación.
    Su grupo de unidades
    es el conjunto de funciones multiplicativas
    que no son cero,
    \(\mathscr{M} \smallsetminus \{0\}\).
  \end{theorem}

  Tenemos también:
  \begin{theorem}[Identidad de Gauß]
    \index{Gauss, identidad de@Gauß, identidad de|textbfhy}
    \index{Gauss, Carl Friedrich@Gauß, Carl Friedrich}
    \label{theo:Gauss-identity}
    Tenemos:
    \begin{equation*}
      \sum_{d \mid n} \phi(d)
	= n
    \end{equation*}
  \end{theorem}
  \begin{proof}
    La suma es multiplicativa,
    basta evaluarla para las potencias de un primo \(p\).
    Pero:
    \begin{align*}
      \sum_{d \mid p^\alpha} \phi(d)
	&= \phi(p^0) + \sum_{1 \le k \le \alpha} \phi(p^k) \\
	&= 1 + \sum_{1 \le k \le \alpha} (p^k - p^{k - 1}) \\
	&= p^\alpha
    \end{align*}
    Resulta la fórmula prometida
    al multiplicar sobre los primos factores de \(n\).
  \end{proof}
  Un resultado útil es el siguiente:
  \begin{theorem}
    \label{theo:sum-mu-f}
    Sea \(f\) una función multiplicativa distinta de cero,
    y sea \(n = p_1^{e_1} p_2^{e_2} \dotsm p_r^{e_r}\)
    con \(p_k\) primos distintos y \(e_k \ge 1\).
    Entonces:
    \begin{equation}
      \label{eq:sum-mu-f}
      \sum_{d \mid n} \mu(d) f(d)
	= \prod_k (1 - f(p_k))
    \end{equation}
  \end{theorem}
  \begin{proof}
    La función \(\mu(n) f(n)\) es multiplicativa,
    con lo que lo es la suma a evaluar.
    Basta evaluar la suma para \(p^e\)
    con \(p\) primo y \(e \ge 1\)
    y luego combinar.
    Como \(f(1) = 1\):
    \begin{align*}
      \sum_{d \mid p^e} \mu(d) f(d)
	&= \sum_{0 \le k \le e} \mu(p^k) f(p^k) \\
	&= \mu(1) f(1) + \mu(p) f(p) \\
	&= 1 - f(p)
    \end{align*}
  \end{proof}

  Usando la convolución de Dirichlet%
    \index{Dirichlet, convolucion de@Dirichlet, convolución de}
  resulta simple demostrar fórmulas
  que de otra forma serían casi imposibles.
  Considere:
  \begin{align*}
    \tau
      &= 1 * 1 \\
    \sigma
      &= \iota * 1 \\
    \phi
      &= \mu * \iota
  \end{align*}
  La última no es más que la identidad de Gauß,%
    \index{Gauss, identidad de@Gauß, identidad de}
  que así podemos reescribir:
  \begin{equation*}
    \sum_{d \mid n} d \mu(d)
      = \phi(n)
  \end{equation*}
  Evaluemos ahora:
  \begin{equation*}
    \sum_{d \mid n} \phi(d) \tau(n / d)
  \end{equation*}
  que es decir:
  \begin{equation*}
    \phi * \tau
      = \mu * \iota * 1 * 1
      = \mu * 1 * \iota * 1
      = \epsilon * \sigma
      = \sigma
  \end{equation*}
  También podemos calcular:
  \begin{equation*}
    \sum_{d \mid n} \mu(d) \tau(n / d)
  \end{equation*}
  que es:
  \begin{equation*}
    \mu * \tau
      = \mu * 1 * 1
      = 1
  \end{equation*}
  Pero también,
  si \(n = p_1^{e_1} p_2^{e_2} \dotsm p_r^{e_r}\)
  con \(p_k\) primos distintos y \(e_k \ge 1\),
  por el teorema~\ref{theo:sum-mu-f}
  al ser \(\tau(p) = 2\):
  \begin{align*}
    \sum_{d \mid n} \mu(d) \tau(d)
      &= \prod_k (1 - \tau(p)) \\
      &= (-1)^r
  \end{align*}
  La suma resulta ser \(1\)
  si \(n\) es divisible por un número par de primos distintos
  y \(-1\) en caso contrario.
  Asimismo:
  \begin{align*}
    \sum_{d \mid n} \mu^2(d)
      &= \prod_k (1 - \mu(p_k)) \\
      &= 2^r
  \end{align*}

  Partiendo de la identidad de Gauß:%
    \index{Gauss, identidad de@Gauß, identidad de}
  \begin{align*}
    \sum_{d \mid n} \phi(d)
      &= n \\
    \phi(n)
      &= \sum_{d \mid n} \mu(d) \frac{n}{d} \\
      &= n \sum_{d \mid n} \frac{\mu(d)}{d} \\
      &= n \prod_k \left( 1 - \frac{1}{p_k} \right)
  \end{align*}
  Otra derivación de la fórmula para \(\phi\).%
    \index{\(\phi\) de Euler}

  Consideremos palabras formadas
  con símbolos de algún alfabeto \(\Sigma\),
  por ejemplo \(\Sigma = \{a, b, c\}\).%
    \index{palabra}
  Una palabra puede ser la repetición de una palabra más corta,
  como \(\mathtt{baba}\).
  A la parte mínima que se repite para formar una palabra
  le llamaremos su \emph{raíz}.%
    \index{palabra!primitiva!raiz@raíz}
  Así,
  la raíz de \(\mathtt{ababab}\) es \(\mathtt{ab}\),
  la raíz de \(\mathtt{acaba}\) es \(\mathtt{acaba}\).
  Nos interesa el número de palabras que son sus propias raíces
  (no son repeticiones de palabras más cortas),
  a las que llamaremos \emph{primitivas}.%
    \index{palabra!primitiva}

  Por ejemplo,
  para \(\Sigma = \{ a, b \}\)
  el número total de palabras de largo \(4\) es \(2^4 = 16\).
  Debemos descontar
  las que se forman repitiendo palabras primitivas.
  Palabras primitivas de largo \(1\)
  son \(\mathtt{a}\) y \(\mathtt{b}\),
  que dan lugar a \(\mathtt{aaaa}\) y \(\mathtt{bbbb}\);
  primitivas de largo \(2\) son \(\mathtt{ab}\) y \(\mathtt{ba}\),
  que dan lugar a \(\mathtt{abab}\) y \(\mathtt{baba}\).
  En total hay \(4\) palabras no primitivas,
  y por lo tanto son \(12\) las primitivas.

  Llamemos \(p(n)\)
  al número de palabras primitivas de largo \(n\)
  (esto claramente depende del número \(s\)
   de símbolos en el alfabeto).
  Como toda palabra es la repetición de alguna palabra
  (cuyo largo divide a \(n\))
  podemos escribir:
  \begin{equation*}
    s^n
      = \sum_{d \mid n} p(d)
  \end{equation*}
  Inversión de Möbius%
    \index{Mobius, inversion de@Möbius, inversión de}
  nos da:
  \begin{equation*}
    p(n)
      = \sum_{d \mid n} \mu(n / d) s^d
  \end{equation*}
    \index{palabra!primitiva!numero@número}

  Para el ejemplo el alfabeto es \(\{ a, b \}\),
  que da \(s = 2\),
  y es \(n = 4\).
  Resulta:
  \begin{align*}
    p(4)
      &= \sum_{d \mid 4} \mu(4 / d) \cdot 2^d \\
      &= \mu(4) \cdot 2^1 + \mu(2) \cdot 2^2 + \mu(1) \cdot 2^4 \\
      &= 0 \cdot 2 - 1 \cdot 4 + 1 \cdot 16 \\
      &= 12
  \end{align*}
  lo que confirma nuestro cálculo anterior.

  Pero la fórmula permite calcular valores mucho mayores
  en forma simple:
  \begin{align*}
    p(12)
      &= \sum_{d \mid 12} \mu(12 / d) \cdot 2^d \\
      &= \mu(12) \cdot 2^1
	   + \mu(6) \cdot 2^2
	   + \mu(4) \cdot 2^3
	   + \mu(2) \cdot 2^6
	   + \mu(1) \cdot 2^{12} \\
      &= 0 \cdot 2
	   + 1 \cdot 4
	   + 0 \cdot 8
	   - 1 \cdot 64
	   + 1 \cdot 4096 \\
      &= 4036
  \end{align*}

  Puede profundizarse bastante
  partiendo de los conceptos anteriores,
  aún sin usar técnicas sofisticadas,
  como muestra magistralmente Moser~%
    \cite{moser04:_introd_theor_number}.

%%% Local Variables:
%%% mode: latex
%%% TeX-master: "clases"
%%% End:


% anillos-polinomios.tex
%
% Copyright (c) 2009-2014 Horst H. von Brand
% Derechos reservados. Vea COPYRIGHT para detalles

\chapter{Anillos de polinomios}
\label{cha:anillos-polinomios}
\index{anillo!polinomios|textbfhy}

  Los polinomios se cuentan entre las funciones más importantes,
  dada su simplicidad.
  Estudiarlos desde un punto de vista algebraico,
  tanto para determinar propiedades de sus ceros
  y su factorización
  como desde el punto de vista más abstracto como ejemplo de anillo,
  es fructífero.
  En particular,
  el estudio de anillos de polinomios lleva naturalmente
  a anillos de series formales,
  que nos ocuparán intensamente más adelante.

\section{Algunas herramientas}
\label{sec:algunas-herramientas-poly}

% Fixme: Agregar ejemplos/ejercicios para aterrizar
% (Iván Lazo <ilazo@alumnos.inf.utfsm.cl> 20130813)

  Un paquete de álgebra simbólica,%
    \index{algebra simbolica@álgebra simbólica}
  como \texttt{maxima}~\cite{maxima14b:_computer_algebra},%
    \index{maxima@\texttt{maxima}}
  ayuda bastante con la operatoria.
  El paquete \texttt{PARI/GP}~\cite{PARI:2.7.2}%
    \index{PARI/GP@\texttt{PARI/GP}}
  incluye extenso soporte para trabajar con polinomios.
  La biblioteca GiNaC~%
    \cite{bauer02:_ginac_fram_symbol_comput, GiNaC:1.6.2}%
    \index{GiNaC@\texttt{GiNaC}}
  permite manipular expresiones simbólicas y numéricas
  directamente en \cplusplus~%
    \cite{ISO11:_C++, stroustrup00:_C++_progr_languag}.%
    \index{C++ (lenguaje de programacion)@\cplusplus{} (lenguaje de programación)}
  \begin{definition}
    \index{polinomio|textbfhy}
    \label{def:ring-polynomials}
    Sea \((R, +, \cdot)\) un anillo,
    y \(x\) un \emph{símbolo}
    (también llamado \emph{indeterminada} o \emph{variable}).
    Definimos \(R[x]\),
    los \emph{polinomios sobre \(R\)},
    como el conjunto de las expresiones:
    \begin{equation*}
      f(x)
	= a_n \cdot x^n + a_{n - 1} \cdot x^{n - 1} + \dotsb + a_0
    \end{equation*}
    El \emph{grado} de \(f\) es la máxima potencia de \(x\)
      \index{polinomio!grado|textbfhy}
    que aparece multiplicada por un coeficiente no cero,
    se anota \(\deg(f)\).
    Al polinomio con todos los coeficientes cero
    (el \emph{polinomio cero})
      \index{polinomio!cero|textbfhy}
    se le asigna el grado \(-\infty\).
    Si solo el término constante (\(a_0\)) es diferente de cero,
    el grado del polinomio es cero,
    y se dice que es un \emph{polinomio constante}.%
      \index{polinomio!constante|textbfhy}
    A polinomios de grado \(1\) se les llama \emph{lineales},%
      \index{polinomio!lineal|textbfhy}
    a los de grado \(2\) \emph{cuadráticos}%
      \index{polinomio!cuadratico@cuadrático|textbfhy}
    y a los de grado \(3\) \emph{cúbicos}.%
      \index{polinomio!cubico@cúbico|textbfhy}
    Se habla del \emph{coeficiente principal}%
      \index{polinomio!coeficiente principal|textbfhy}
    para referirse al coeficiente
    de la máxima potencia de \(x\) en el polinomio.
    Si el coeficiente principal es \(1\),
    el polinomio se llama \emph{mónico}.%
      \index{polinomio!monico@mónico|textbfhy}
  \end{definition}
  Nótese que algunos autores
  simplemente no le asignan grado al polinomio cero.

  No asignamos significado a \(x\) ni a sus potencias.
  Podemos desarrollar toda la teoría
  hablando únicamente de tuplas de coeficientes.
  La notación es sugestiva,
  y más adelante sí consideraremos
  los polinomios como definiendo funciones.

  \begin{definition}
    \index{operacion@operación!polinomios}
    Para polinomios \(f, g \in R[x]\),
    definimos la suma y producto entre ellos
    (bajo el supuesto
     que es una secuencia infinita de coeficientes \(0\)
     a partir de un cierto punto para simplificar)
    como si tratáramos con expresiones en \(R\),
    solo que \(x^k\) conmuta con los elementos de \(R\)
    y se cumple \(x^i x^j = x^{i + j}\):
    \begin{align}
      f(x)
	 &= a_0 + a_1 \cdot x + a_2 \cdot x^2 + \dotsb
	   \label{eq:poly-f} \\
      g(x)
	 &= b_0 + b_1 \cdot x + b_2 \cdot x^2 + \dotsb
	   \label{eq:poly-g}
    \end{align}
    \begin{align}
      f(x) + g(x)
	 &= (a_0 + b_0)
	      + (a_1 + b_1) \cdot x
	      + \dotsb
	      + (a_k + b_k) \cdot x^k
	      + \dotsb \notag \\
	 &= \sum_{k  \ge 0} (a_ k + b_k) x^k \label{eq:poly-f+g}
    \end{align}
    \begin{align}
      f(x) \cdot g(x)
	 &= a_0 \cdot b_0
	      + (a_1 b_0 + a_0 b_1) \cdot x
	      + (a_2 b_0 + a_1 b_1 + a_0 b_2) \cdot x^2
	      + \dotsb \notag \\
	 &= \sum_{\substack{0 \le i \le m \\
			    0 \le j \le n
		 }} a_i b_j x^{i + j} \notag \\
	 &= \sum_{k \ge 0}
	      \left(
		\sum_{0 \le i \le k} a_{k - i} b_i
	      \right) \cdot x^k \label{eq:poly-f*g}
    \end{align}
  \end{definition}
  Resulta que \(R[x]\) con las operaciones
  definidas por~\eqref{eq:poly-f+g} y~\eqref{eq:poly-f*g}
  es un anillo.%
    \index{anillo}
  Es cómodo considerar \(\alpha \in R\)
  como el polinomio constante \(\alpha \in R[x]\).
  Las unidades de \(R[x]\) son los polinomios constantes
  \(\alpha \in R^\times\).
  Es fácil ver que:
  \begin{align}
    \deg(f + g)
      &\le \max\{\deg(f), \deg(g)\} \label{eq:poly-deg-f+g} \\
    \deg(f \cdot g)
      &\le \deg(f) + \deg(g)	    \label{eq:poly-deg-f*g}
  \end{align}
  Por esto resulta útil definir el grado del polinomio \(0\)
  como \(-\infty\),
  evita requerir casos especiales.

  En caso que no hayan divisores de cero
  diferentes de cero en \(R\),
  en~\eqref{eq:poly-deg-f*g} es igualdad.
  Si \(R\) es un dominio integral
    \index{dominio integral}
  (un anillo conmutativo sin divisores de cero distintos de cero),
  \(R[x]\) también es un dominio integral
  (el coeficiente del término de mayor grado
   del producto \(f(x) \cdot g(x)\)
   no es cero si ambos polinomios son diferentes de cero).

  Definimos la \emph{derivada formal}%
    \index{polinomio!derivada formal}
  de un polinomio mediante:
  \begin{align*}
    f(x)
      &= \sum_{0 \le k \le n} a_k x^k \\
    \mathrm{D} f(x)
      &= \sum_{0 \le k \le n - 1} (k + 1 ) a_{k + 1} x^k
  \end{align*}
  Anotaremos alternativamente:
  \begin{equation*}
    f'(x)
      = \mathrm{D} f(x)
  \end{equation*}
  Es fácil verificar que se cumplen las propiedades conocidas
  de las derivadas:
  \begin{theorem}
    \label{theo:polynomial-derivative}
    Sean \(f(x)\) y \(g(x)\) polinomios
    sobre el dominio integral \(R\),
    \(\alpha\) y \(\beta\) elementos de \(R\).
    Entonces:
    \begin{align*}
      \mathrm{D} (\alpha f(x) + \beta g(x))
	&= \alpha f'(x) + \beta g'(x) \\
      \mathrm{D} (f(x) \cdot g(x))
	&= f'(x) g(x) + f(x) g'(x) \\
      \mathrm{D} (f(x)^m)
	&= m f(x)^{m - 1} f'(x)
    \end{align*}
  \end{theorem}
  \begin{proof}
    Definamos:
    \begin{align*}
      f(x)
	&= \sum_{0 \le k \le m} f_k x^k \\
      g(x)
	&= \sum_{0 \le k \le n} g_k x^k
    \end{align*}
    Por comodidad,
    anotaremos sumas infinitas
    bajo el entendido que los términos son todos cero
    desde un índice en adelante.

    Para la primera parte,
    tenemos que:
    \begin{align*}
      \alpha f(x) + \beta g(x)
	&= \sum_{k \ge 0} (\alpha f_k + \beta g_k) x^k \\
      \mathrm{D} (\alpha f(x) + \beta g(x))
	&= \sum_{k \ge 0} (k + 1)
	     (\alpha f_{k + 1} + \beta g_{k + 1}) x^k \\
	&= \alpha \sum_{k \ge 0} (k + 1) f_{k + 1} x^k
	     + \beta \sum_{k \ge 0} (k + 1) g_{k + 1} x^k \\
	&= \alpha f'(x) + \beta g'(x)
    \end{align*}
    Acá usamos el que para \(k \in \mathbb{N}\) y \(a, b \in R\):
    \begin{align*}
      k (a b)
	&= a b + a b + \dotsb + a b \\
	&= a (b + b + \dotsb + b) \\
	&= a (k b)
    \end{align*}

    Para la segunda parte:
    \begin{align*}
      (\mathrm{D} f(x)) g(x) &+ f(x) (\mathrm{D} g'(x)) \\
	&= \sum_{k \ge 0}
	     \left(
	       \sum_{0 \le j \le k} (j + 1) f_{j + 1} g_{k - j}
		  + \sum_{0 \le j \le k}
		      (k + 1 - j) f_j g_{k + 1 - j}
	     \right) x^k \\
	&= \sum_{k \ge 0}
	     \left(
	       \sum_{0 \le j \le k + 1} j f_j g_{k + 1 - j}
		  + \sum_{0 \le j \le k + 1}
		      (k + 1 - j) f_j g_{k + 1 - j}
	     \right) x^k \\
	&= \sum_{k \ge 0}
	      (k + 1)
	      \left(
		\sum_{0 \le j \le k + 1} f_j g_{k + 1 - j}
	      \right) x^k \\
	&= \mathrm{D} (f(x) g(x))
    \end{align*}

    Para la tercera parte,
    usamos inducción sobre \(m\).%
      \index{demostracion@demostración!induccion@inducción}
    \begin{description}
    \item[Base:]
      Cuando \(m = 1\)
      lo aseverado ciertamente se cumple.
    \item[Inducción:]
      Suponiendo que vale para \(m\),
      demostramos que vale para \(m + 1\):
      \begin{align*}
	\mathrm{D} (f(x)^{m + 1})
	  &= \mathrm{D} (f(x)^m f(x)) \\
	  &= m f(x)^{m - 1} f'(x) f(x) + f(x)^m f'(x) \\
	  &= (m + 1) f(x)^m f'(x)
      \end{align*}
      Acá usamos la conmutatividad de \(R[x]\).
    \end{description}
    Por inducción,
    vale para todo \(m \in \mathbb{N}\).
  \end{proof}

  Los anillos de polinomios tienen varias propiedades interesantes,
  por ejemplo un algoritmo de división afín al de los enteros:%
    \index{algoritmo de division@algoritmo de división!polinomios}
  \begin{theorem}
    \label{theo:F[x]:division-algorithm}
    Sean \(a(x)\), \(b(x)\) polinomios sobre un campo \(F\),
    con \(b(x) \ne 0\).
    Entonces existen polinomios únicos \(q(x)\), \(r(x)\)
    tales que:
    \begin{equation*}
      a(x) = b(x) \cdot q(x) + r(x)
    \end{equation*}
    con \(\deg(r) < \deg(b)\)
  \end{theorem}
  \begin{proof}
    Consideremos el conjunto:
    \begin{equation*}
      \mathcal{R}
	= \{a(x) - c(x) \cdot b(x) \colon c(x) \in F[x]\}
    \end{equation*}
    Elijamos un elemento \(r\) de \(\mathcal{R}\) de grado mínimo.
    Entonces \(\deg(r) < \deg(b)\),
    ya que en caso contrario
    podríamos restar un múltiplo de \(b(x)\)
    que anule el término de grado mayor en \(r(x)\)
    y así obtener uno de grado menor.

    Demostramos que son únicos por contradicción.%
      \index{demostracion@demostración!contradiccion@contradicción}
    Supongamos que hay dos pares diferentes,
    o sea:
    \begin{equation*}
      a = b q' + r' \qquad
      a = b q'' + r''
    \end{equation*}
    Sin pérdida de generalidad
    podemos suponer que \(\deg(r') \le \deg(r'')\).
    Como \(F[x]\) es un anillo,
    con \(F\) un campo:
    \begin{align}
      r'' - r'
	&= b (q' - q'') \notag \\
      \deg(r'' - r')
	&= \deg(b (q' - q'')) \notag \\
	&= \deg(b) + \deg(q' - q'') \label{eq:deg(b(q2-q1))} \\
    \intertext{Pero:}
      \deg(r')
	&\le \deg(r'') < \deg(b) \notag \\
      \deg(r'' - r')
	&\le \deg(r'') < \deg(b) \label{eq:deg(r2-r1)}
    \end{align}
    En vista de~\eqref{eq:deg(r2-r1)}
    la única posibilidad en~\eqref{eq:deg(b(q2-q1))}
    es \(\deg(r'' - r') = \deg(q' - q'') = -\infty\),
    vale decir,
    \(q' = q''\) y \(r' = r''\).
    Esto contradice nuestra elección de dos pares diferentes.
  \end{proof}
  Vale la pena comparar esta demostración
  con la del algoritmo de división entre enteros,
  teorema~\ref{theo:division}.

% dominios-euclidianos.tex
%
% Copyright (c) 2012-2014 Horst H. von Brand
% Derechos reservados. Vea COPYRIGHT para detalles

\section{Dominios euclidianos}
\label{sec:dominios-euclidianos}

  A un dominio integral \(D\)
  equipado con una \emph{función euclidiana}
  (a veces llamada \emph{función grado} o simplemente \emph{grado})
  \(g \colon D \smallsetminus \{0\} \rightarrow \mathbb{N}\)
  tal que si \(a, b \in D\),
  con \(b \ne 0\),
  hay \(q, r \in D\)
  tales que \(a = q b + r\)
  con \(r = 0\) o \(g(r) < g(b)\)
  se le llama \emph{dominio euclidiano}.
  Estas estructuras tienen mucho en común con \(\mathbb{Z}\)
  (en particular,
   toman su nombre porque es aplicable el algoritmo de Euclides
   para calcular máximo común divisor,
   y tenemos el equivalente
   de la identidad de \foreignlanguage{french}{Bézout}).
  En el caso de los polinomios,
  el grado sirve como función euclidiana.

  Tenemos algunos resultados simples:
  \begin{theorem}
    \label{theo:ED:g-minimo=>unidad}
    Sea \(D\) un dominio euclidiano con función euclidiana \(g\).
    El valor \(g(a)\) es mínimo si \(a\) es una unidad.
  \end{theorem}
  \begin{proof}
    Tomemos \(a \ne 0\) en \(D\) tal que \(g(a)\) es mínimo.
    Por el algoritmo de división,
    teorema~\ref{theo:F[x]:division-algorithm},
    tenemos \(1 = q a + r\) con \(r = 0\) o \(g(r) < g(a)\).
    Pero \(g(a)\) es mínimo,
    por lo que debe ser \(r = 0\)
    y \(a\) es una unidad.
  \end{proof}
  También tenemos las propiedades
  (ver Rogers~%
    \cite{rogers71:_axiom_euclidean_domain}
   y Samuel~%
    \cite{samuel71:_about_euclidean_rings}):
  \begin{theorem}
    \label{theo:ED:propiedades-f}
    Sea \(D\) un dominio euclidiano
    con función euclidiana \(g\).
    La función definida por:
    \begin{equation*}
      f(a)
	= \min_{x \in D \smallsetminus \{0\}} g(a x)
    \end{equation*}
    es una función euclidiana,
    y cumple:
    \begin{enumerate}[label=(\alph{*})]
    \item
      \label{en:fe:a}
      \(f(a) \le f(a b)\) si \(a b \ne 0\)
    \item
      \label{en:fe:b}
      \(f(a) \le g(a)\) para todo \(a \in D \smallsetminus \{0\}\)
    \item
      \label{en:fe:c}
      \(f(a u) = f(a)\) si y solo si \(u \in D^\times\)
    \end{enumerate}
  \end{theorem}
  \begin{proof}
    Por la definición de \(f\)
    los puntos~\ref{en:fe:a} y~\ref{en:fe:b} son obvios.

    Para demostrar que \(f\) es euclidiana,
    consideremos elementos \(a, b\)
    cualquiera en \(D \smallsetminus \{0\}\).
    Debemos demostrar que si \(b = q a + r\)
    entonces \(r = 0\) o \(f(r) < f(a)\).

    El caso \(r = 0\) es trivial.
    Supongamos entonces \(r \ne 0\).
    Por definición
    es \(f(a) = g(a c)\)
    para algún \(c \in D \smallsetminus \{0\}\).
    De la definición de \(r\) tenemos que \(g(r) < g(a)\)
    por ser \(g\) euclidiana.
    De \(b c = q a c + r c\),
    por ser \(g\) euclidiana es \(g(r c) < g(a c) = f(a)\);
    y por la definición de \(f\) es también \(f(r) \le g(r c)\).
    Uniendo las anteriores queda \(f(r) < f(a)\),
    y \(f\) es euclidiana.

    Para~\ref{en:fe:c} demostramos implicancia en ambas direcciones.
    Primero,
    sea \(u \in D^\times\).
    Por el punto~\ref{en:fe:a}
    es \(f(a) \le f(a u) \le f((a u) u^{-1}) = f(a)\).
    Por otro lado,
    si \(f(a c) = f(a)\),
    escribimos \(a = q a c + r\) con \(f(r) < f(a c) = f(a)\);
    siendo \(r = a (1 - c q)\),
    por la parte~\ref{en:fe:a}
    si \(r \ne 0\) es \(f(r) \ge f(a)\),
    lo que es absurdo.
    Así \(r = 0\) y \(c\) es una unidad.
  \end{proof}
  Supondremos una función euclidiana
  tal que \(f(a) \le f(a b)\) para todo \(a, b \in D\)
  desde ahora,
  ya que simplifica mucha de la discusión que sigue.
  Nótese que en particular el grado de polinomios cumple esto.
  Como la función euclidiana no es única,
  no la incluimos en la definición del dominio.
  \begin{definition}
    Sea \(R\) un dominio integral.
    Si podemos escribir \(m = b c\),
    decimos que \(b\) \emph{divide a} \(m\),
    y anotamos \(b \mid m\).
  \end{definition}
  Esto también se expresa
  diciendo que \(b\) es un \emph{factor} de \(m\),
  o que \(m\) es un \emph{múltiplo} de \(b\).
  \begin{definition}
    Sea \(R\) un dominio integral.
    Dos elementos \(a, b \in R\) se dicen \emph{asociados}
    si \(a = u b\),
    donde \(u\) es una unidad.
    Se anota \(a \sim b\).
  \end{definition}
  Es fácil ver que \(\sim\) es una relación de equivalencia.
  \begin{definition}
    Sea \(R\) un dominio integral.
    Un elemento \(e \in R \smallsetminus R^\times\)
    se llama \emph{irreductible} si siempre que
    \(e = u \cdot v\),
    \(u\) o \(v\) es una unidad.
    En caso contrario,
    decimos que \(e\) es \emph{reductible}.
  \end{definition}
  \begin{definition}
    Sea \(R\) un dominio integral,
    \(p \in R \smallsetminus R^\times\).
    Si \(p \mid a b\) implica que \(p \mid a\) o \(p \mid b\),
    se dice que \(p\) es \emph{primo}.
  \end{definition}
  Vemos que si un elemento es primo,
  es irreductible:
  \begin{lemma}
    \label{lem:prime=>irreducible}
    Sea \(R\) un dominio integral.
    Si \(p \in R\) es primo,
    entonces es irreductible.
  \end{lemma}
  \begin{proof}
    Por contradicción.
    Supongamos \(p\) primo pero no irreductible.
    Así podemos escribir \(p = u v\),
    donde \(u, v \notin R^\times\).
    Pero entonces \(p \mid u v\),
    y por la definición de primo
    es \(p \mid u\) o \(p \mid v\).
    Sin pérdida de generalidad
    podemos suponer \(u = a p\),
    con lo que:
    \begin{align*}
      p
	&= a p v \\
      0
	&= p (1 - a v)
    \end{align*}
    Como no hay divisores de cero distintos de cero en \(R\),
    debe ser \(a v = 1\)
    y \(v\) es una unidad,
    lo que contradice su elección.
  \end{proof}
  El recíproco del lema~\ref{lem:prime=>irreducible}
  no siempre se cumple.
  Consideremos el dominio integral
  \(\mathbb{Z}[\sqrt{-5}]\)
  (ver la sección~\ref{sec:anillos-cuadraticos},
   solo que este es un subanillo de \(\mathbb{C}\)).
  Si \(3 = u \cdot v\),
  debe ser \(N(u) \cdot N(v) = N(3) = 9\),
  con lo que las normas posibles para \(u\) y \(v\)
  son los divisores de 9.
  Si \(N(u) = 1\),
  \(u\) es una unidad.
  Si \(N(u) = 3\),
  con \(u = u_1 + u_2 \sqrt{-5}\)
  es \(u_1^2 + 5 u_2^2 = 3\).
  Esto claramente es imposible con \(u_1\), \(u_2\) enteros.
  Así \(3\) es irreductible en \(\mathbb{Z}[\sqrt{-5}]\).
  Por el otro lado:
  \begin{align*}
    &(2 + \sqrt{-5}) \cdot (2 - \sqrt{-5})
      = 9 \\
  \intertext{con lo que}
    &3
      \mid (2 + \sqrt{-5}) \cdot (2 - \sqrt{-5})
  \end{align*}
  Claramente \(3 \centernot\mid 2 \pm \sqrt{-5}\),
  o sea 3 no es primo en \(\mathbb{Z}[\sqrt{-5}]\).

  \begin{definition}
    Sea \(R\) un dominio integral,
    y sea \(a \in R\) con \(a \ne 0\).
    Entonces se dice
    que \emph{\(a\) tiene factorización única en irreductibles}
    si hay una unidad \(u\)
    e irreductibles \(p_i\) tales que \(a = u p_1 p_2 \dotsm p_r\),
    y además,
    si \(a = v q_1 q_2 \dotsm q_s\) para una unidad \(v\)
    e irreductibles \(q_i\),
    entonces \(r = s\)
    y \(p_i = u_i q_i\)
    para unidades \(u_i\) salvo reordenamiento.
  \end{definition}
  \begin{definition}
    Se dice que \(R\) es un \emph{dominio de factorización única}
    (en inglés
     \emph{\foreignlanguage{english}{Unique Factorization Domain}}.
     abreviado \emph{UFD})
    si todo elemento de \(R\)
    tiene factorización única en irreductibles.
  \end{definition}

  En vista del algoritmo de división en el dominio euclidiano,
  tenemos:
  \begin{theorem}
    \label{theo:ED=>PID}
    Sea \(D\) un dominio euclidiano con función euclidiana \(f\)
    y \(a, b \in D\).
    Entonces el conjunto \(I = \{u a + v b \colon u, v \in D\}\)
    consta de todos los múltiplos de un elemento \(m\).
  \end{theorem}
  La demostración es muy similar
  a la discusión sobre máximo común divisor
  en el capítulo~\ref{cha:teoria-numeros}.
  \begin{proof}
    Si \(a = b = 0\),
    claramente \(I = \{0\}\),
    y lo aseverado se cumple.
    Supongamos entonces que al menos uno
    de \(a\), \(b\) es diferente de 0,
    en cuyo caso \(I\) contiene elementos diferentes de 0.
    Elijamos uno de ellos con \(f\) mínimo,
    llamémosle \(m\).
    Tomemos ahora \(n \in I\) cualquiera.
    Si \(n = 0\),
    se cumple \(m \mid n\),
    y estamos listos.
    Si \(n \ne 0\),
    podemos aplicar el algoritmo de división y escribir:
    \begin{equation*}
      n = q m + r
    \end{equation*}
    donde \(r = 0\) o \(f(r) < f(m)\).
    Dado que hay \(u, v, u', v'\) tales que \(n = u a + v b\)
    y \(m = u' a + v' b\) resulta:
    \begin{align*}
      r
	&= n - q m \\
	&= (u - q u') a + (v - q v') b
    \end{align*}
    con lo que \(r \in I\).
    Pero no puede ser \(f(r) < f(m)\),
    hemos elegido \(m\) precisamente por ser \(f(m)\) mínimo.
    En consecuencia,
    \(r = 0\) y \(m \mid n\).
  \end{proof}
  Conjuntos como \(I\)
  que aparece en la demostración del teorema~\ref{theo:ED=>PID}
  son muy importantes.
  Podemos definir \(m\)
  (o uno de sus asociados,
   que también son parte de \(I\);
   \(m\) no necesariamente es único)
  como un máximo común divisor de \(a\) y \(b\)
  (``máximo'' en el sentido de la función euclidiana \(f\)).
  \begin{definition}
    Sea \(R\) un anillo conmutativo.
    Un \emph{ideal} de \(R\)
    es un conjunto \(I \subseteq R\)
    tal que:
    \begin{enumerate}
    \item
      \((I, +)\) es un subgrupo de \((R, +)\)
    \item
      Para todo \(x \in I\)
      y para todo \(r \in R\)
      se cumple \(r \cdot x \in I\)
    \end{enumerate}
  \end{definition}
  Los ideales son casi subanillos de \(R\)
  (solo falta el elemento \(1\)).
  Hay quienes definen anillos sin \(1\),
  para ellos los ideales son subanillos.
  \begin{definition}
    Sea \(R\) un anillo conmutativo,
    y \(\{x_1, x_2, \dotsc, x_n\} \subseteq R\).
    Al ideal
      \(\{\sum_{1 \le k \le n} u_k x_k \colon u_k \in R\}\)
    se le llama
    el \emph{ideal generado por \(\{x_1, x_2, \dotsc, x_n\}\)},
    que se suele anotar \((x_1, x_2, \dotsc, x_n)\).
    Por la convención que sumas vacías son cero,
    \(\{0\}\) es generado por \(\varnothing\).
    A un ideal generado por un único elemento \(x_1\),
    anotado \((x_1)\),
    se le llama \emph{ideal principal}.
    Si en \(R\) todos los ideales son principales,
    se dice que \(R\) es un \emph{dominio de ideal principal}
    (en inglés
     \emph{\foreignlanguage{english}{Principal Ideal Domain}},
     abreviado \emph{PID}).
  \end{definition}
  En estos términos,
  el teorema~\ref{theo:ED=>PID} asevera que todo dominio euclidiano
  es un dominio de ideal principal.

  \begin{lemma}
    \label{lem:ED:irreductible-coprime}
    Sea \(p\) irreductible en un dominio euclidiano \(D\),
    y \(a\) otro elemento de \(D\).
    Si \(p\) no divide a \(a\),
    entonces \(1\) es un máximo común divisor entre \(a\) y \(p\).
  \end{lemma}
  \begin{proof}
    Sea \(m\) un máximo común divisor de \(a\) y \(p\).
    Por el teorema~\ref{theo:ED=>PID}
    existen \(x, y \in D\) tales que:
    \begin{equation*}
      m = x a + y p
    \end{equation*}
    Como \(m\) divide a \(p\),
    que es irreductible,
    \(m\) es una unidad o \(m \sim p\).
    Si \(m\) es una unidad,
    \(1\) es un máximo común divisor de \(a\) y \(p\)
    y estamos listos.
    En el otro caso,
    por ser \(m\) divisor de \(a\)
    es \(a = c m\) para algún \(c \in D\)
    y como a su vez \(m = u p\) para una unidad \(u\),
    entonces \(a = c u p\) y \(p \mid a\).
  \end{proof}
  Esto nos permite demostrar
  el recíproco del lema~\ref{lem:prime=>irreducible}
  en dominios euclidianos,
  como ya lo hicimos en el teorema~\ref{theo:Z:irreductible=>prime}
  para los enteros:
  \begin{theorem}
    \label{theo:ED:irreducible=>prime}
    En un dominio euclidiano,
    si \(p\) es irreductible
    entonces \(p\) es primo.
  \end{theorem}
  \begin{proof}
    Supongamos que el irreductible \(p\) divide a \(a b\).
    Debemos demostrar que \(p\) divide a \(a\) o a \(b\)
    (o a ambos).
    Si \(p \mid a\),
    estamos listos.
    En caso contrario,
    como \(p \mid a b\),
    hay un \(c \in D\) tal que \(a b = c p\).
    Por el lema~\ref{lem:ED:irreductible-coprime}
    tenemos que \(1\) es un máximo común divisor de \(a\) y \(p\),
    y por el teorema~\ref{theo:ED=>PID} podemos escribir:
    \begin{align*}
      1
	&= u p + v a \\
      b
	&= b u p + v a b \\
	&= (b u + v c) p
    \end{align*}
    con lo que \(p \mid b\).
  \end{proof}
  \begin{lemma}
    \label{lem:ED:primo-divide-producto}
    Si el primo \(p\) divide al producto \(x_1 x_2 \dotsm x_n\),
    entonces \(p \mid x_i\) para algún \(i\).
  \end{lemma}
  \begin{proof}
    Por inducción sobre \(n\).
    Si \(n = 1\),
    no hay nada que demostrar.
    \begin{description}
    \item[Base:]
      Para \(n = 2\),
      por la definición de primo tenemos que si \(p \mid x_1 x_2\),
      entonces \(p \mid x_1\) o \(p \mid x_2\).
    \item[Inducción:]
      Por la hipótesis de inducción,
      si \(p \mid x_1 x_2 \dotsm x_n\)
      entonces \(p \mid x_i\) para \(1 \le i \le n\).
      Si ahora \(p \mid x_1 x_2 \dotsm x_n x_{n + 1}\)
      por el caso \(n = 2\) significa que ya sea
      \(p \mid x_1 x_2 \dotsm x_n\)
      (lo que implica \(p \mid x_i\) para \(1 \le i \le n\))
      o \(p \mid x_{n + 1}\).
      En conjunto,
      \(p \mid x_i\) para \(1 \le i \le n + 1\).
    \end{description}
    Por inducción,
    vale para todo \(n \in \mathbb{N}\).
  \end{proof}
  Así tenemos:
  \begin{theorem}
    \label{theo:PID=>UFD}
    Si \(D\) es un dominio de ideal principal,
    entonces es un dominio de factorización única.
  \end{theorem}
  \begin{proof}
    Por contradicción.
    Consideremos un elemento \(a \in D \smallsetminus D^\times\)
    (distinto de \(0\))
    con \(f(a)\) mínimo
    y que no tiene factorización en irreductibles.
    Entonces \(a\) no es irreductible
    (sería el producto de un irreductible),
    por lo que podemos escribir \(a = b c\),
    con \(b, c \notin D^\times\).
    Por el teorema~\ref{theo:ED:propiedades-f}
    resulta \(f(b) < f(a)\) y \(f(c) < f(a)\).
    Pero entonces \(b\) y \(c\) son producto de irreductibles,
    con lo que lo es \(a\).
    Vale decir,
    tal \(a\) no existe.

    Para demostrar factorización única usamos reducción al absurdo.
    Sea \(a\) un elemento de mínimo \(f\)
    que tiene dos factorizaciones esencialmente diferentes:
    \begin{equation*}
      a
	= u p_1 p_2 \dotsm p_m
	= v q_1 q_2 \dotsm q_n
    \end{equation*}
    donde los \(p_i\) son primos
    (no necesariamente diferentes),
    y similarmente los \(q_i\),
    y \(u\) y \(v\) son unidades.
    Por el lema~\ref{lem:ED:primo-divide-producto},
    esto significa que \(p_1\) divide a \(q_i\) para algún \(i\),
    o sea \(q_i = u_i p_i\) para alguna unidad \(u_i\),
    con lo que:
    \begin{equation*}
      a / p_1
	= u p_2 \dotsm p_m
	= v u_i q_1 \dotsm q_{i -1} q_{i + 1} \dotsm q_n
    \end{equation*}
    tendría dos factorizaciones diferentes,
    pero \(f(a / p_1) < f(a)\),
    lo que contradice la elección de \(a\) como uno de mínimo \(f\)
    con dos factorizaciones.
  \end{proof}
  Esto viene a ser el equivalente
  del teorema fundamental de la aritmética
  (teorema~\ref{theo:fundamental-aritmetica}):
  En un dominio euclidiano
  todo elemento \(a\) es el producto de un número finito de primos.
  Además,
  si tenemos factorizaciones en primos \(p_i\) y \(q_i\):
  \begin{equation*}
    a
      = p_1 p_2 \dotsm p_m
      = q_1 q_2 \dotsm q_n
  \end{equation*}
  entonces cada \(p\) es el asociado de uno de los \(q\).
  En particular,
  \(m = n\).

%%% Local Variables:
%%% mode: latex
%%% TeX-master: "clases"
%%% End:


\section{Factorización de polinomios}
\label{sec:factorizacion-polinomios}
\index{polinomio!factorizacion@factorización}

  Vimos que el conjunto de polinomios \(F[x]\)
  sobre el campo \(F\) es un dominio euclidiano,%
    \index{dominio euclidiano}
  el grado del polinomio sirve de función euclidiana.
  Podemos elegir el polinomio mónico
  como el representante de la clase de asociados,
  con lo que tenemos:%
    \index{polinomio!teorema fundamental de la aritmetica@teorema fundamental de la aritmética}
  \begin{theorem}[Teorema fundamental de la aritmética]
    \label{theo:fundamental-arithmetic-polynomials}
    Todo polinomio en \(F[x]\)
    es una unidad
    o el producto de una unidad y polinomios mónicos irreductibles.
    Esta factorización es única
    (salvo el orden de los factores).
  \end{theorem}
  Podemos caracterizar polinomios con ceros repetidos:
  \begin{lemma}
    \index{polinomio!ceros repetidos}
    \label{lem:repeated-roots}
    Si \(\alpha\) es un cero repetido
    del polinomio \(f(x) \in F[x]\),
    es cero común de \(f(x)\) y \(f'(x)\).
  \end{lemma}
  \begin{proof}
    Consideremos \(f(x) = (x - \alpha)^m g(x)\),
    donde \((x - \alpha) \centernot\mid g(x)\) y \(m > 1\).
    Tenemos:
    \begin{equation*}
      f'(x)
	= m (x - \alpha)^{m - 1} g(x) + (x - \alpha)^m g'(x)
	= (x - \alpha)^{m - 1} (m g(x) + (x - \alpha) g(x))
    \end{equation*}
    Como \(m > 1\),
    esto siempre es divisible por \(x - \alpha\).
  \end{proof}
  Tenemos también:
  \begin{theorem}[Euclides]
    \label{theo:Euclides-polynomials}
    Hay infinitos polinomios irreductibles sobre el campo \(F\).
  \end{theorem}
  \begin{proof}
    Por contradicción.%
      \index{demostracion@demostración!contradiccion@contradicción}
    Supongamos que hay finitos irreductibles
    \(p_1(x)\) a \(p_n(x)\).
    Entonces \(p_1(x) \dotsm p_n(x) + 1\)
    no es divisible por ningún \(p_i(x)\).
  \end{proof}
  Sobre un campo infinito no da nada nuevo
  (los polinomios lineales son todos irreductibles),
  pero sobre un campo finito sí es interesante.

  Consideremos ahora los polinomios como funciones,
  substituyendo elementos del campo por \(x\).

  \begin{corollary}
    \label{cor:polinomio-raiz-factor}
    Sea \(F\) un campo,
    y \(f(x) = a_n x^n + a_{n - 1} x^{n - 1} + \dotsb + a_0\)
    un polinomio de grado \(n\) sobre \(F\),
    y \(\alpha \in F\).
    Entonces \(f(\alpha) = 0\)
    si y solo si \(f(x) = (x - \alpha) g(x)\)
    para algún polinomio \(g \in F[x]\).
  \end{corollary}
  \begin{proof}
    Demostramos implicancia en ambas direcciones.

    Primeramente,
    si \(f(x) = (x - \alpha) g(x)\),
    claramente \(f(\alpha) = 0 \cdot g(\alpha) = 0\).

    Por el algoritmo de división podemos escribir
    \(f(x) = q(x) \cdot (x - \alpha) + r(x)\),
    donde \(\deg(r) < 1\)
    con lo que \(r(x)\) es constante.
    Ahora bien,
    \(f(\alpha)
       = q(\alpha) \cdot (\alpha - \alpha) + r(\alpha) = 0\),
    con lo que al ser constante
      \(r(x)\) es \(r(x) = r(\alpha) = 0\).
    Así \(f(x) = q(x) \cdot (x - \alpha)\),
    y llamando \(g(x) = q(x)\) completa la demostración.
  \end{proof}
  Con esta herramienta básica podemos demostrar
  el resultado siguiente:
  \begin{corollary}
    \index{polinomio!numero de ceros@número de ceros}
    \label{cor:numero-raices-polinomio}
    Si \(f(x) \in F[x]\) es un polinomio de grado \(n \ge 0\),
    entonces \(f(x)\) tiene a lo más \(n\) ceros en \(F\).
  \end{corollary}
  \begin{proof}
    Por inducción.%
      \index{demostracion@demostración!induccion@inducción}
    \begin{description}
    \item[Bases:]
      Para \(n = 0\),
      no hay ceros.

      Para \(n = 1\),
      tenemos:
      \begin{align*}
	&a_1 \cdot x + a_0
	  = 0 \\
	&x
	  = - a_1^{-1} \cdot a_0
      \end{align*}
      que claramente es única.
    \item[Inducción:]
      Suponiendo ahora que todos los polinomios de grado \(n\)
      tienen a lo más \(n\) ceros,
      consideremos un polinomio \(f(x)\) de grado \(n + 1\).
      Si no hay \(\alpha\) tal que \(f(\alpha) = 0\),
      \(f(x)\) tiene 0 ceros y estamos listos.
      Si tiene un cero \(\alpha\),
      por el corolario~\ref{cor:polinomio-raiz-factor}
      podemos escribir \(f(x) = (x - \alpha) \cdot g(x)\)
      con \(\deg(g) = n\).
      Por inducción,
      \(g(x)\) tiene a lo más \(n\) ceros,
      y \(f(x)\) tiene a lo más \(n + 1\) ceros.
    \end{description}
    Por inducción vale lo enunciado.
  \end{proof}

  Suelen ser útiles los términos más fáciles de calcular
  del polinomio
  con ceros \(\alpha_1\), \(\alpha_2\), \ldots, \(\alpha_m\);
  o sea,
  los coeficientes en el producto:%
    \index{Vieta, formulas de@Vieta, fórmulas de}
  \begin{equation}
    \label{eq:poly-coeficientes-ceros}
    (x - \alpha_1) \dotsm (x - \alpha_m)
       = x^m
	 - (\alpha_1 + \alpha_2 + \dotsb + \alpha_m) x^{m - 1}
	 + \dotsb
	 + (-1)^m \alpha_1 \cdot \alpha_2 \dotsm \alpha_m
  \end{equation}
  El coeficiente de \(x^{m - 1}\)
  es el negativo de la suma de los ceros,
  y el término constante es su producto
  con signo que depende de si \(m\) es par o impar.
  Las expresiones~\eqref{eq:poly-coeficientes-ceros}
  son casos particulares de las fórmulas de Vieta:
  Para el polinomio
    \(a_n x^n + a_{n - 1} x^{n - 1} + \dotsb + a_0\),
  los ceros \(\alpha_1, \alpha_2, \dotsc, \alpha_n\) cumplen:
  \begin{equation}
    \label{eq:Vieta-formulas}
    \sum_{1 \le i_1 \le i_2 \le \dotsb \le i_k \le n}
       \alpha_{i_1} \alpha_{i_2} \dotsm \alpha_{i_k}
       = (-1)^k \, \frac{a_{n - k}}{a_n}
  \end{equation}
  La suma en~\eqref{eq:Vieta-formulas} es simplemente
  sobre todos los conjuntos de \(k\) de las ceros.
  Por ejemplo:
  \begin{equation*}
    a x^3 + b x^2 + c x + d
      = a (x - \alpha_1) (x - \alpha_2) (x - \alpha_3)
  \end{equation*}
  resulta en las siguientes tres expresiones:
  \begin{align*}
    \alpha_1 + \alpha_2 + \alpha_3
      &= - \frac{b}{a} \\
    \alpha_1 \alpha_2 + \alpha_1 \alpha_3 + \alpha_2 \alpha_3
      &= \frac{c}{a} \\
    \alpha_1 \alpha_2 \alpha_3
      &= - \frac{d}{a}
  \end{align*}

  \begin{theorem}
    \label{theo:F*-ciclico}
    Sea \(F\) un campo finito,%
      \index{campo (algebra)@campo (álgebra)}
    y \(F^\times\) su grupo de unidades.%
      \index{anillo!grupo de unidades}
    Entonces \(F^\times\) es cíclico.%
      \index{grupo!ciclico@cíclico}
  \end{theorem}
  \begin{proof}
    Sea \(e\) el \emph{exponente} de \(F^\times\),%
      \index{grupo!exponente|textbfhy}
    vale decir,
    el mínimo natural
    tal que \(a^e = 1\) para todo \(a \in F^\times\).
    Esto es el mínimo común múltiplo
    de los órdenes de los elementos de \(F^\times\).
    Por el teorema de Lagrange%
      \index{Lagrange, teorema de}
    todos ellos son factores de \(\lvert F^\times \rvert\),
    con lo que \(e\) es factor de \(\lvert F^\times \rvert\)
    y \(e \le \lvert F^\times \rvert\).

    Por otro lado,
    todos los elementos de \(F^\times\) cumplen:
    \begin{equation*}
      x^e - 1
	= 0
    \end{equation*}
    En el campo \(F\)
    este polinomio puede tener a lo más \(e\) ceros,
    o sea \(e \ge \lvert F^\times \rvert\),
    con lo que \(e = \lvert F^\times \rvert\).
    Siendo \(e\) el mínimo común múltiplo
    de los órdenes en \(F^\times\),
    debe haber un elemento de orden \(e\)
    y \(F^\times\) es cíclico.
  \end{proof}

  \begin{corollary}
    \label{cor:Un-ciclico}
    Si \(p\) es primo,
    \(\mathbb{Z}^\times_p\) es cíclico.
  \end{corollary}
  \begin{proof}
    \(\mathbb{Z}_p\) es un campo finito,
    y \(\mathbb{Z}^\times_p\) es su grupo de unidades.
  \end{proof}

  Esto implica el curioso resultado:
  \begin{theorem}[Wilson]
    \index{Wilson, teorema de}
    \label{theo:Wilson}
    \((n - 1)! \equiv -1 \pmod{n}\) si y solo si \(n\) es primo.
  \end{theorem}
  \begin{proof}
    Demostramos implicancia en ambas direcciones.
    Si \(n\) es primo,
    \(\mathbb{Z}_n\) es un campo,
    y por la demostración del teorema~\ref{theo:F*-ciclico}
    sabemos que:
    \begin{equation*}
      x^{n - 1} - 1
       = \prod_{1 \le k \le n - 1} (x - k)
    \end{equation*}
    La fórmula de Vieta~\eqref{eq:Vieta-formulas}%
      \index{Vieta, formulas de@Vieta, fórmulas de}
    da para el término constante:
    \begin{equation*}
      (-1)^{n - 1} (n - 1)! \equiv -1 \pmod{n}
    \end{equation*}
    Si \(n = 2\),
    es \((-1)^{n - 1} = -1\) y \(1 \equiv -1 \pmod{n}\);
    si \(n\) es un primo impar,
    \((-1)^{n - 1} = 1\).
    De cualquier forma,
    \((n - 1)! \equiv -1 \pmod{n}\).

    Para el recíproco,
    demostramos el contrapositivo:
    Si \(n\) no es primo,
    \((n - 1)! \centernot\equiv -1 \pmod{n}\).
    Hay varios casos:
    Si \(n = 4\)
    es \(3! = 6 \equiv 2 \centernot\equiv -1 \pmod{4}\),
    y se cumple lo enunciado.
    Sea ahora \(n > 4\) compuesto.
    En tal caso podemos escribir \(n = a \cdot b\),
    con \(2 \le a, b < n - 1\).
    De ser \(a \ne b\),
    ambos factores aparecen en \((n - 1)!\),
    y por tanto \(n \mid (n - 1)!\).
    Si es \(n = a^2\),
    será \(a > 2\),
    y entre los factores de \((n - 1)!\)
    estarán \(a\) y \(2 a\),
    con lo que nuevamente \(n \mid (n - 1)!\).
    En estas dos situaciones tenemos \((n - 1)! \equiv 0 \pmod{n}\),
    y obtenemos lo enunciado.
  \end{proof}
  Este resultado no es útil para computación,
  y de uso teórico bastante limitado.

  Una demostración alternativa es considerar un primo impar \(p\),
  y parear cada \(a \in \mathbb{Z}_p\) con su inverso.
  Los únicos elementos
  que no tienen pareja en esto son \(1\) y \(-1\)
  (son sus propios inversos,
   en el campo \(\mathbb{Z}_p\)
   la ecuación \(x^2 - 1 = 0\) puede tener
   a lo más dos raíces),
  por lo que el producto de todos ellos es \(-1\).

\section{Raíces primitivas}
\label{sec:raices-primitivas}
\index{raiz primitiva@raíz primitiva|textbfhy}

  Cuando \(R^\times\) es cíclico,%
    \index{grupo!ciclico@cíclico}
  a un generador de \(R^\times\)
  se le llama \emph{elemento primitivo} de \(R\).%
    \index{anillo!elemento primitivo}
  En el caso de \(\mathbb{Z}_n\)
  se le llama una \emph{raíz primitiva} módulo \(n\)
  (porque todo elemento de \(\mathbb{Z}^\times_n\)
   puede escribirse como potencia del generador).
  Lo anterior demuestra que todo primo \(p\)
  tiene raíces primitivas.
  Una pregunta obvia es qué valores de \(n\)
  tienen raíces primitivas
  (o sea,
   cuándo es cíclico \(\mathbb{Z}^\times_n\)).

  Consideremos \(n = 8\),
  con \(\phi(8) = 4\).
  Por cálculo directo tenemos los órdenes
  de los elementos de \(\mathbb{Z}^\times_8\)
  dados en el cuadro~\ref{tab:ordenes-U8}.
  \begin{table}[htbp]
    \centering
    \begin{tabular}[htbp]{|c|c|}
      \hline
      \multicolumn{1}{|c|}
	  {\rule[-0.7ex]{0pt}{3ex}\(\boldsymbol{k}\)} &
	\multicolumn{1}{c|}{\(\boldsymbol{\ord_8(k)}\)} \\
      \hline\rule[-0.7ex]{0pt}{3ex}%
      1 & 1 \\
      3 & 2 \\
      5 & 2 \\
      7 & 2 \\
      \hline
    \end{tabular}
    \caption{Órdenes de los elementos en $\mathbb{Z}^\times_8$}
    \label{tab:ordenes-U8}
  \end{table}
  Se aprecia que no hay elementos de orden \(4\),
  y \(8\) no tiene raíces primitivas.

  Por otro lado,
  para \(n = 2\) tenemos \(\phi(2) = 1\)
  y claramente 1 es raíz primitiva módulo 2.
  Para \(n = 4\) tenemos \(\phi(4) = 2\),
  y \(3\) es raíz primitiva módulo \(4\).

  Más generalmente,
  tenemos:
  \begin{theorem}
    \label{theo:raices-primitivas-2k}
    No hay raíces primitivas módulo \(2^m\) si \(m \ge 3\).
  \end{theorem}
  \begin{proof}
    Demostraremos esto por inducción.%
      \index{demostracion@demostración!induccion@inducción}
    \begin{description}
    \item[Base:]
      Cuando \(m = 3\),
      vimos antes que \(2^m = 8\) no tiene raíces primitivas.
    \item[Inducción:]
      Suponemos que la aseveración vale para \(m - 1\).
      Como \mbox{\(\phi(2^m) = 2^{m - 1}\)}
      y el único divisor de \(2^{m - 1}\) es \(2\),
      basta demostrar
      que el orden de todo elemento en \(\mathbb{Z}_{2^m}\)
      es divisor de \(2^{m - 2}\).
      Consideremos \(a\) cualquiera relativamente primo a \(2^m\).
      Por inducción
      sabemos que \(a\) no es raíz primitiva de \(2^{m - 1}\),
      que expresado en los términos anteriores es:
      \begin{align}
	a^{2^{m - 3}}
	  &\equiv 1 \pmod{2^{m - 1}} \notag \\
	a^{2^{m - 3}}
	  &= c \cdot 2^{m - 1} + 1   \notag \\
      \intertext{Elevando al cuadrado:}
	a^{2^{m - 2}}
	  &= \left(a^{2^{m - 3}}\right)^2 \notag \\
	  &= c^2 \cdot 2^m + 2 \cdot c \cdot 2^{m - 1} + 1 \notag \\
	  &= (c^2 + c) \cdot 2^m + 1 \notag \\
	  &\equiv 1 \pmod{2^m} \label{eq:a^(2^(m-2))=1}
      \end{align}
    \end{description}
    Por~\eqref{eq:a^(2^(m-2))=1}
    el orden de \(a\) módulo \(2^m\)
    es a lo más \(2^{m - 2} < 2^{m - 1} = \phi(2^m)\),
    y \(a\) no es raíz primitiva.
  \end{proof}
  Esto da un conjunto infinito de enteros sin raíces primitivas.
  Pero aún más:
  \begin{theorem}
    \label{theo:raices-primitivas-mn}
    No hay raíces primitivas módulo \(m n\)
    si \(m\) y \(n\) son enteros impares relativamente primos
    mayores que \(2\).
  \end{theorem}
  \begin{proof}
    Como \(m\) y \(n\) son impares y mayores que \(2\),
    sabemos que \(2\)
    es un factor común entre \(\phi(m)\) y \(\phi(n)\)
    (un primo impar \(p\)
     aporta \(p^{k - 1} (p - 1)\) a \(\phi(\cdot)\)).
    Para cualquier \(a\) tal que \(\gcd(a, m n) = 1\),
    por el teorema de Euler:
    \begin{align}
      a^{\phi(m) \phi(n) / 2}
	&\equiv \left(a^{\phi(m)}\right)^{\phi(n) / 2}
	   \equiv 1^{\phi(n) / 2}
	   \equiv 1 \pmod{m}  \label{eq:a^phiphi/2-m} \\
      a^{\phi(m) \phi(n) / 2}
	&\equiv \left(a^{\phi(n)}\right)^{\phi(m) / 2}
	   \equiv 1^{\phi(m) / 2}
	   \equiv 1 \pmod{n} \label{eq:a^phiphi/2-n}
    \end{align}
    Combinando~\eqref{eq:a^phiphi/2-m} con~\eqref{eq:a^phiphi/2-n}
    mediante el teorema~\ref{theo:congruencia-mn}
    resulta:
    \begin{equation}
      \label{eq:a^phiphi/2-mn}
      a^{\phi(m) \phi(n) / 2} \equiv 1 \pmod{m n}
    \end{equation}
    Por~\eqref{eq:a^phiphi/2-mn}
    el orden de \(a\) divide a \(\phi(m n) / 2\),
    y \(a\) no es raíz primitiva de \(m n\).
    Pero \(a\) es arbitrario,
    con lo que \(m n\) no tiene raíces primitivas.
  \end{proof}

  Lo anterior excluye \(2^k\) para \(k \ge 3\),
  y los números compuestos impares con factores primos distintos.
  Analicemos los restantes.
  \begin{theorem}
    \label{theo:raiz-primitiva-p2}
    Sea \(p\) un primo impar,
    entonces hay una raíz primitiva módulo \(p^2\)
  \end{theorem}
  \begin{proof}
    Sea \(r\) una raíz primitiva de \(p\),
    o sea \(\ord_p (r) = p - 1\).
    Sabemos que si \(n = \ord_{p^2}(r)\)
    entonces \(n \mid \phi(p^2)\),
    vale decir \(n \mid p (p - 1)\).
    Sabemos que si \(r^n \equiv 1 \pmod{p^2}\)
    entonces \(r^n \equiv 1 \pmod{p}\),
    de forma que \(\phi(p) \mid n\).
    Pero si \(p - 1 \mid n\) y \(n \mid p (p - 1)\),
    entonces \(n = p (p - 1)\) o \(n = p - 1\).
    En el primer caso,
    tenemos que \(r\) es raíz primitiva de \(p^2\),
    y estamos listos.

    En el segundo caso,
    consideremos el elemento \(r + p\),
    que sigue siendo raíz primitiva módulo \(p\),
    con lo que su orden es \(p - 1\) o \(p (p - 1)\) módulo \(p^2\).
    Calculamos:
    \begin{equation}
      \label{eq:(r+p)^(p-1)}
      (r + p)^{p - 1}
	= r^{p - 1}
	   + \binom{p - 1}{1} p r^{p - 2}
	   + \sum_{2 \le k \le p - 1}
	       \binom{p - 1}{k} p^k r^{p - 1 - k}
    \end{equation}
    Todos los elementos de la sumatoria en~\eqref{eq:(r+p)^(p-1)}
    son divisibles por \(p^2\).
    Como estamos suponiendo
    que el orden de \(r\) módulo \(p^2\) es \(p - 1\),
    y sabemos que:
    \begin{equation*}
      (p - 1) p r^{p - 2}
	\equiv - p r^{p - 2}
	\centernot\equiv 0 \pmod{p^2}
    \end{equation*}
    (ya que \(\gcd(r, p) = 1\)
     también es \(\gcd(r^{p - 2}, p) = 1\)),
    y queda:
    \begin{equation*}
      (r + p)^{p - 1}
	\equiv 1 - p r^{p - 2}
	\centernot\equiv 1 \pmod{p^2} \\
    \end{equation*}
    y \(r + p\) es raíz primitiva módulo \(p^2\).
  \end{proof}
  Esto parece ser solo un paso al ir de \(p\) a \(p^2\),
  pero resulta ser todo lo que se requiere.
  Antes de seguir,
  un lema técnico.
  \begin{lemma}
    \label{lem:raiz-primitiva-p2-orden-pk}
    Sea \(p\) un primo impar,
    \(r\) una raíz primitiva módulo \(p^2\).
    Entonces para \(m \ge 2\):
    \begin{equation}
      \label{eq:r^(p^(m-2)(p-1))<>1}
      r^{p^{m - 2} (p - 1)}
	\centernot\equiv 1 \pmod{p^m}
    \end{equation}
  \end{lemma}
  \begin{proof}
    La demostración es por inducción desde \(m = 2\).%
      \index{demostracion@demostración!induccion@inducción}
    De partida,
    si \(r\) es raíz primitiva módulo \(p^2\),
    lo es módulo \(p\) y \(\gcd(r, p) = 1\).
    \begin{description}
    \item[Base:]
      Para \(m = 2\)
      el orden de la raíz primitiva \(r\) de \(p^2\)
      es \(\phi(p^2) = p (p - 1)\),
      por lo que:
      \begin{equation}
	\label{eq:r^(p-1)<>1}
	r^{p - 1}
	  \centernot\equiv 1 \pmod{p^2}
      \end{equation}
    \item[Inducción:]
      Supongamos que el resultado vale para \(m - 1\).
      Del teorema de Euler%
	\index{Euler, teorema de}
      tenemos:
      \begin{align}
	r^{\phi(p^{m - 2})}
	  &\equiv 1 \pmod{p^{m - 2}} \notag \\
	r^{p^{m - 3} (p - 1)}
	  &=	  1 + c \cdot p^{m - 2} \label{eq:r^(p^(m-3)(p-1))}
      \end{align}
      Nótese que \(p \centernot\mid c\),
      ya que de lo contrario tendríamos:
      \begin{equation}
	\label{eq:r^(p^(m-3)(p-1))=1}
	r^{p^{m - 3} (p - 1)}
	  \equiv 1 \pmod{p^{m - 1}}
      \end{equation}
      y esto contradice
      nuestra hipótesis de inducción~\eqref{eq:r^(p^(m-2)(p-1))<>1}.
      Elevando~\eqref{eq:r^(p^(m-3)(p-1))=1} a la potencia \(p\),
      queda:
      \begin{align}
	r^{p^{m - 2} (p - 1)}
	  &= \left(1 + c p^{m - 2}\right)^p \notag \\
	  &= 1 + c p^{m - 1}
	      + \sum_{2 \le k \le p} \binom{p}{k} c^k p^{k (m - 2)}
	  \label{eq:r^(p^(m-2)(p-1))}
      \end{align}
      Interesa demostrar
      que cada término de la sumatoria~\eqref{eq:r^(p^(m-2)(p-1))}
      es divisible por \(p^m\).
      En ellos aparece el factor \(p^{k (m - 2) + 1}\)
      (incluyendo \(p\) del coeficiente binomial).%
	\index{coeficiente binomial}
      Interesa acotar el exponente \(k (m - 2) + 1\),
      con \(k \ge 2\) y \(m \ge 3\).
      Es \(m - 2 \ge 1\),
      el mínimo exponente se da para \(k = 2\)
      y tenemos \(k (m - 2) + 1 \ge 2 m - 3 \ge m\).
      En consecuencia:
      \begin{align*}
	r^{p^{m - 2} (p - 1)}
	  &\equiv     1 + c p^{m - 1} \pmod{p^m} \\
	  &\centernot\equiv 1 \pmod{p^m}
      \end{align*}
      como queríamos probar.
      \qedhere
    \end{description}
  \end{proof}
  \begin{theorem}
    \label{theo:raiz-primitiva-pk}
    Toda raíz primitiva módulo \(p^2\) para un primo impar \(p\)
    es raíz primitiva módulo \(p^m\) con \(m \ge 2\)
  \end{theorem}
  \begin{proof}
    Sea \(r\) raíz primitiva módulo \(p^2\),
    con lo que \(\ord_{p^2} (r) = p (p - 1)\)
    ya que \(\phi(p^2) = p (p - 1)\),
    y llamemos \(n = \ord_{p^m} (r)\).
    Como antes,
    por el teorema de Euler sabemos que:
    \begin{equation*}
      r^{\phi(p^m)}
	\equiv r^{p^{m - 1} (p - 1)} \equiv 1 \pmod{p^m}
    \end{equation*}
    con lo que:
    \begin{equation*}
      n \mid p^{m - 1} (p - 1)
    \end{equation*}
    Pero también:
    \begin{equation*}
      r^n
	\equiv 1 \pmod{p^m}
    \end{equation*}
    implica:
    \begin{equation*}
      r^n
	\equiv 1 \pmod{p^2}
    \end{equation*}
    con lo que \(p (p - 1) \mid n\),
    y \(n = p^k (p - 1)\) para algún \(1 \le k \le m - 1\).
    Por el lema~\ref{lem:raiz-primitiva-p2-orden-pk} sabemos que:
    \begin{equation*}
      r^{p^{m - 2} (p - 1)}
	\centernot\equiv 1 \pmod{p^m}
    \end{equation*}
    con lo que \(k = m - 1\),
    y \(r\) es raíz primitiva módulo \(p^m\).
  \end{proof}

  Resta el caso \(2 p^m\),
  donde por el corolario~\ref{cor:isomorfismo-anillo-Zm} resulta
  \(\mathbb{Z}_{2 p^m} \cong \mathbb{Z}_2 \oplus \mathbb{Z}_{p^m}\),
  y \(\mathbb{Z}^\times_{2 p^m}
	\cong \mathbb{Z}^\times_2 \oplus \mathbb{Z}^\times_{p^m}
	\cong \mathbb{Z}^\times_{p^m}\),
  y hay una raíz primitiva.
  En resumen,
  obtenemos el curioso resultado:
  \begin{theorem}
    \label{theo:raices-primitivas}
    Hay raíces primitivas módulo \(n\) si y solo si
    \(n = 2, 4, p^m, 2 p^m\),
    donde \(p\) es un primo impar y \(m \ge 1\).
  \end{theorem}

  Igual resulta interesante hallar el máximo orden módulo \(n\),
  al que llamaremos \(\lambda(n)\).%
    \index{\(\lambda(m)\) (maximo orden modulo \(m\))@\(\lambda(m)\) (máximo orden módulo \(m\))|textbfhy}
  En algunos casos
  lo conocemos por el teorema~\ref{theo:raices-primitivas},
  falta completarlo para los demás valores de \(n\).

  \begin{lemma}
    \label{lem:lambda-2^e}
    El máximo orden en \(\mathbb{Z}_{2^e}\) con \(e \ge 3\)
    es \(\lambda(2^e) = 2^{e - 2}\)
  \end{lemma}
  \begin{proof}
    Consideremos \(a\) impar.
    Entonces uno de \(a \pm 1\) es divisible por \(4\),
    y tenemos para algún \(f \ge 2\) y un \(c\):
    \begin{align*}
      a
	&\equiv 2^f \pm 1 \pmod{2^{f + 1}} \\
	&=	c \cdot 2^{f + 1} + 2^f \pm 1  \\
      a^2
	&=	c^2 \cdot 2^{2 (f + 1)} + 2^{2 f} + 1
		  + 2 \cdot c \cdot 2^{2 f + 1}
		  \pm 2 \cdot c \cdot 2^{f + 1}
		  \pm 2 \cdot 2^f		 \\
	&\equiv 2^{f + 1} + 1 \pmod{2^{f + 2}}
    \end{align*}
    Continuando de la misma forma,
    concluimos que para \(r \ge 1\):
    \begin{equation}
      \label{eq:a^(2^r)}
      a^{2^r}
	\equiv 2^{f + r} + 1 \pmod{2^{f + r + 1}}
    \end{equation}
    Cuando \(f + r + 1 = e\),
    la ecuación~\eqref{eq:a^(2^r)} lleva a:
    \begin{align*}
      a^{2^{e - f - 1}}
	&\equiv		  2^{e - 1} + 1 \pmod{2^e} \\
	&\centernot\equiv 1		\pmod{2^e} \\
      a^{2^{e - f}}
	&\equiv	    2^e + 1 \pmod{2^{e + 1}} \\
	&\equiv	    1	    \pmod{2^e}
    \end{align*}
    O sea,
    \(\ord_{2^e} (a) = 2^{e - f}\),
    donde \(f\) es el valor determinado anteriormente.
    El mínimo valor posible de \(f\) es \(2\),
    y siempre podemos elegir \(a = 4 - 1 = 3\) que da \(f = 2\),
    así \(\lambda(2^e) = 2^{e - 2}\).
  \end{proof}
  Incidentalmente,
  hemos hallado una manera simple de calcular \(\ord_{2^e}(a)\).
  \begin{theorem}
    \label{theo:max-orden-Z*}
    El máximo orden está dado por:
    \begin{align*}
      \lambda(2) = 1, \qquad
	&\lambda(4) = 2,
	  \qquad \lambda(2^e) = 2^{e - 2} \text{\ si\ } e \ge 3 \\
      \lambda(p^e)
	&= p^{e - 1} (p - 1) \text{\ si el primo\ } p > 2 \\
      \lambda(p_1^{e_1} p_2^{e_2} \dotsm p_r^{e_r}&)
	 = \lcm(\lambda(p_1^{e_1}), \lambda(p_2^{e_2}), \dotsc,
		\lambda(p_r^{e_r}))
	   \text{\ si \(p_1, \dotsc, p_r\) son primos}
    \end{align*}
  \end{theorem}
  \begin{proof}
    Los casos \(2\) y \(4\) son obvios.
    El caso \(2^e\) es el tema del lema~\ref{lem:lambda-2^e},
    el caso \(p^e\)
    es inmediato del teorema~\ref{theo:raices-primitivas}
    y \(\phi(p^e) = p^{e - 1} (p - 1)\).

    Ahora,
    para \(\gcd(m, n) = 1\)
    sabemos del teorema chino de los residuos%
      \index{residuo!teorema chino de los}
    (en realidad,
     del corolario~\ref{cor:isomorfismo-anillo-Zm})%
       \index{anillo!isomorfismo}
    que \(\mathbb{Z}^\times_{m n}
	    \cong \mathbb{Z}^\times_m \times \mathbb{Z}^\times_n\).
    El orden de un elemento \(a \in \mathbb{Z}^\times_{m n}\),
    que podemos descomponer en \(u v\)
    con \(u \in \mathbb{Z}^\times_m\)
    y \(v \in \mathbb{Z}^\times_n\),
    es \(\lcm(\ord_m(u), \ord_n(v))\).
    Si elegimos elementos de orden máximo para \(u\) y \(v\),
    \(a\) será de orden máximo,
    su orden es \(\lambda(m n) = \lcm(\lambda(m), \lambda(n))\),
    y el resultado sigue.
  \end{proof}

  Algunos algoritmos
  requieren usar raíces primitivas de un primo grande \(p\),%
    \index{raiz primitiva@raíz primitiva!numero@número}
  y los anteriores no dan muchas luces de cómo encontrar una.
  Por suerte son relativamente numerosas:
  Si tomamos una raíz primitiva \(r\),
  todos los elementos \(r^k\) con \(\gcd(k, p - 1) = 1\)
  también tendrán orden \(p - 1\) y son raíces primitivas.
  Vale decir,
  hay \(\phi(p - 1)\) raíces primitivas del primo \(p\).

%%% Local Variables:
%%% mode: latex
%%% TeX-master: "clases"
%%% End:


% campos-finitos.tex
%
% Copyright (c) 2010, 2012-2014 Horst H. von Brand
% Derechos reservados. Vea COPYRIGHT para detalles

\chapter{Campos finitos}
\label{cha:campos-finitos}
\index{campo (algebra)@campo (álgebra)!finito|textbfhy}

  Gran parte de las matemáticas
  giran alrededor del álgebra abstracta,%
    \index{algebra abstracta@álgebra abstracta}
  que hemos conocido en los grupos%
    \index{grupo}
  y anillos.%
    \index{anillo}
  Pero sin duda las estructuras algebraicas más importantes
  son los campos,%
    \index{campo (algebra)@campo (álgebra)}
  que definimos en el capítulo~\ref{cha:teoria-numeros}
  como anillos conmutativos en los que todos los elementos
  (salvo \(0\))
  tienen inverso multiplicativo.
  Implícitamente usamos el campo de los reales \(\mathbb{R}\),%
    \index{R (números reales)@\(\mathbb{R}\) (números reales)}
  ocasionalmente nos aventuramos a los números complejos \(\mathbb{C}\).%
    \index{C (numeros complejos)@\(\mathbb{C}\) (números complejos)}
  Pero en realidad nuestros cálculos
  casi siempre son con aproximaciones racionales en \(\mathbb{Q}\).%
    \index{Q (números racionales)@\(\mathbb{Q}\) (números racionales)}
  De particular interés son los campos finitos,
  con una bonita teoría y abundantes aplicaciones prácticas.

\section{Propiedades básicas}
\label{sec:propiedades-campo}

  El orden aditivo de \(1\) en el campo \(F\)
  determina en gran medida la estructura del campo,
  y se le llama la \emph{característica del campo}.%
    \index{campo (algebra)@campo (álgebra)!caracteristica@característica}
  Se dice
  \(\chr(F) = n\) si el orden es \(n\)
  y \(\chr(F) = 0\) si es infinito.

  Si \(F\) es un campo,
  y \(K\) es un subcampo de \(F\)%
    \index{campo (algebra)@campo (álgebra)!subcampo}
  (cosa que se anota \(K \le F\))
  se dice que \(F\) es un \emph{campo extensión} de \(K\);%
    \index{campo (algebra)@campo (álgebra)!extension@extensión}
  si \(K \ne F\)
  decimos que \(K\) es un \emph{subcampo propio} de \(F\)%
    \index{campo (algebra)@campo (álgebra)!subcampo!propio}
  (se anota \(K < F\) en este caso).
  Se le llama \emph{subcampo primo} de \(F\)%
    \index{campo (algebra)@campo (álgebra)!subcampo!primo}
  a la intersección entre todos los subcampos de \(F\).
  El subcampo primo no tiene subcampos a su vez.
  Anotamos \(K \cong F\)
  si los campos \(K\) y \(F\) son isomorfos%
    \index{campo (algebra)@campo (álgebra)!isomorfo}
  \begin{theorem}
    \label{theo:prime-subfield}
    Sea \(K\) el subcampo primo de \(F\).
    Entonces:
    \begin{enumerate}[label = (\roman*), ref = (\roman*)]
    \item
      \label{en:prime-subfield-Q}
      Si \(F\) tiene característica \(0\),
      entonces \(K \cong \mathbb{Q}\)
    \item
      \label{en:prime-subfield-Zp}
      Si \(F\) tiene característica \(p\),
      entonces \(p\) es primo y \(K \cong \mathbb{Z}_p\)
    \end{enumerate}
  \end{theorem}
  \begin{proof}
    Llamaremos \(0_K\) y \(1_K\) a los elementos neutros de \(K\).
    \begin{enumerate}[label = (\roman*), ref = (\roman*)]
    \item
      En este caso para \(a \in \mathbb{Z}\)
      tenemos que \(a \cdot 1_K \in K\),
      y como \(K\) es un campo para \(b \in \mathbb{N}\)
      también \((a \cdot 1_K) (b \cdot 1_K)^{-1} \in K\),
      y esto es isomorfo al campo \(\mathbb{Q}\).
      Como \(K\) no tiene subcampos,
      \(K \cong \mathbb{Q}\).
    \item
      Nuevamente para \(a \in \mathbb{Z}\)
      tenemos que \(a \cdot 1_K \in K\),
      pero \(p \cdot 1_K = 0_K\),
      con lo que hay un subcampo de \(K\)
      que es isomorfo a \(\mathbb{Z}_p\),
      y como \(K\) es mínimo es \(K \cong \mathbb{Z}_p\).
      Pero \(\mathbb{Z}_p\) es campo si y solo si \(p\) es primo.
    \qedhere
    \end{enumerate}
  \end{proof}
  Si el campo finito \(F\) no es isomorfo a ningún \(\mathbb{Z}_p\),
  habrá algún elemento que no pertenece a su subcampo primo \(K\),
  llamémosle \(\alpha\).
  Pero en tal caso,
  los elementos \(\{k \alpha \colon k \in K\}\)
  deben ser todos distintos entre sí,
  y salvo \(0\) ninguno pertenece a \(K\).
  Así tenemos elementos
    \(\{k_0 + k_1 \alpha \colon k_0, k_1 \in K\}\).
  Debemos además incluir las potencias de \(\alpha\).
  Si la primera potencia de \(\alpha\)
  que pertenece a \(K\) es \(\alpha^m\),
  tendremos elementos
    \(k_0 + k_1 \alpha + \dotsb + k_{m - 1} \alpha^{m - 1}\),
  todos diferentes.
  Cualquier elemento \(\beta\) aún no considerado
  dará lugar a una construcción similar sobre los anteriores.
  Repitiendo este proceso,
  vemos que hay
  una colección de \(n\) elementos
    \(\alpha_i \in F \smallsetminus K\)
  (elementos como \(\alpha\) y \(\beta\) mencionados arriba,
   sus potencias,
   y productos de ellas)
  tales que eligiendo adecuadamente los \(k_i \in K\)
  podemos representar cualquier elemento \(f \in F\)
  mediante la expresión:
  \begin{equation*}
    f
      = \sum_{1 \le i \le n} k_i \alpha_i
  \end{equation*}
  Por la construcción anterior,
  cada elección de los \(k_i\)
  da lugar a un elemento diferente de \(F\),
  con lo que concluimos que si la característica del campo es \(p\),
  y el campo es finito,
  su orden debe ser \(p^n\) para \(n \in \mathbb{N}\).
  Para discutir este fenómeno se requieren conceptos adicionales,
  para mayores detalles véase por ejemplo el texto de Strang~%
    \cite{strang09:_intr_linear_algebra}.

% espacios-vectoriales.tex
%
% Copyright (c) 2012-2014 Horst H. von Brand
% Derechos reservados. Vea COPYRIGHT para detalles

\section{Espacios vectoriales}
\label{sec:espacios-vectoriales}
\index{espacio vectorial}

  Una estructura algebraica común es el espacio vectorial.
  Es aplicable a una gran variedad de situaciones,
  algunas bastante inesperadas.
  \begin{definition}
    Sea \(F\) un campo
    (sus elementos los llamaremos \emph{escalares})%
      \index{espacio vectorial!escalar|textbfhy}
    y \(V\) un conjunto
    (los \emph{vectores},%
      \index{espacio vectorial!vector|textbfhy}
     que por convención anotaremos en negrita).
    Hay operaciones \emph{suma de vectores}%
      \index{espacio vectorial!operaciones}
    (anotada \(+\))
    y \emph{producto escalar}
    entre un escalar y un vector
    (anotada \(\cdot\)).
    Se dice que \(V\) es un \emph{espacio vectorial sobre \(F\)}
    si cumple con los siguientes axiomas,%
      \index{espacio vectorial!axiomas}%
      \index{axioma!espacio vectorial}
    donde \(\alpha, \beta, \dotsc \in F\),
    y \(\boldsymbol{v}_1, \boldsymbol{v}_2, \dotsc \in V\).
    \begin{enumerate}[label=\textbf{V\arabic{*}:}, ref=V\arabic{*}]
    \item\label{ax:V:associative}
      \((\boldsymbol{v}_1 + \boldsymbol{v}_2) + \boldsymbol{v}_3
	   = \boldsymbol{v}_1
	       + (\boldsymbol{v}_2
	       + \boldsymbol{v}_3)\)
    \item\label{ax:V:neutral}
      Hay un elemento \(\boldsymbol{0} \in V\)
      tal que para todo \(\boldsymbol{v} \in V\)
      se cumple
      \(\boldsymbol{v} + \boldsymbol{0} = \boldsymbol{v}\)
    \item\label{ax:V:inverse}
      Para cada \(\boldsymbol{v} \in V\)
      hay \(- \boldsymbol{v} \in V\)
      tal que \(\boldsymbol{v} + (- \boldsymbol{v})
		  = \boldsymbol{0}\)
    \item\label{ax:V:commutative}
       \(\boldsymbol{v}_1 + \boldsymbol{v}_2
	   = \boldsymbol{v}_2 + \boldsymbol{v}_1\)
    \item\label{ax:V:scalar(vectorsum)}
      \(\alpha \cdot ( \boldsymbol{v}_1 + \boldsymbol{v}_2 )
	  = \alpha \cdot \boldsymbol{v}_1
	      + \alpha \cdot \boldsymbol{v}_2\)
    \item\label{ax:V:(scalarsum)vector}
      \((\alpha + \beta) \cdot \boldsymbol{v}
	  = \alpha \cdot \boldsymbol{v}
	      + \beta \cdot \boldsymbol{v}\)
    \item\label{ax:V:scalar-scalar-vector}
      \(\alpha \cdot (\beta \cdot \boldsymbol{v})
	  = (\alpha \beta) \cdot \boldsymbol{v}\)
     \item\label{ax:V:1-vector}
       Si \(1\) es el neutro multiplicativo de \(F\),
       \(1 \cdot \boldsymbol{v} = \boldsymbol{v}\)
    \end{enumerate}
  \end{definition}
  \noindent
  En resumen,
  \((V, +)\) es un grupo abeliano%
    \index{grupo!abeliano}
  (axiomas~\ref{ax:V:associative} a~\ref{ax:V:commutative}),
  junto con el campo \(F\) y multiplicación escalar que cumple
  los axiomas adicionales~\ref{ax:V:scalar(vectorsum)}
  a~\ref{ax:V:1-vector}.
  Normalmente indicaremos la multiplicación escalar
  por simple yuxtaposición.
  Dejamos de ejercicio
  demostrar que \(0 \cdot \boldsymbol{v} = \boldsymbol{0}\)
  y que \((- \alpha) \cdot \boldsymbol{v}
	    = - ( \alpha \cdot \boldsymbol{v})\).
  \begin{definition}
    Sea \(V\) un espacio vectorial sobre el campo \(F\).
    Si para el conjunto de vectores \(B\)
    es:
    \begin{equation*}
      \sum_{\boldsymbol{b} \in B}
	\alpha_{\boldsymbol{b}} \boldsymbol{b}
	= \boldsymbol{0}
    \end{equation*}
    solo si \(\alpha_{\boldsymbol{b}} = 0\)
    para todo \(\boldsymbol{b} \in B\)
    se dice que esos vectores
    son \emph{linealmente independientes}.%
      \index{espacio vectorial!independencia lineal}
  \end{definition}
  Si un conjunto de vectores no es linealmente independiente
  se dice que son \emph{linealmente dependientes}.
  Nótese que \(\boldsymbol{0}\)
  nunca pertenece
  a un conjunto de vectores linealmente independientes,
  ya que al multiplicarlo
  por cualquier escalar obtenemos \(\boldsymbol{0}\).
  \begin{definition}
    Sea \(V\) un espacio vectorial sobre \(F\),
    y \(B \subseteq V\) un conjunto de vectores.
    El \emph{espacio vectorial generado por \(B\)}
    es el conjunto:
    \begin{equation*}
      \langle B \rangle
	= \left\{
	    \sum_{\boldsymbol{b} \in B}
	      \alpha_{\boldsymbol{b}} \boldsymbol{b}
	       \colon \alpha_{\boldsymbol{b}} \in F
	  \right\}
    \end{equation*}
    Si \(V = \langle B \rangle\),
    se dice que \(B\) \emph{abarca} \(V\).
  \end{definition}
  En particular:
  \begin{definition}
    \index{espacio vectorial!base|textbfhy}
    Una \emph{base} del espacio vectorial \(V\)
    es un conjunto linealmente independiente de vectores \(B\)
    que abarca \(V\).
  \end{definition}
  La representación de \(\boldsymbol{v} \in V\)
  en términos de la base \(B\)
  es única,
  ya que si hubieran dos representaciones diferentes
  darían una dependencia lineal en \(B\).
  Para el vector:
  \begin{equation*}
    \boldsymbol{v}
      = \sum_{\boldsymbol{b} \in B}
	  \alpha_{\boldsymbol{b}} \boldsymbol{b}
  \end{equation*}
  a los coeficientes \(\alpha_{\boldsymbol{b}}\)
  se les llama \emph{componentes} de \(\boldsymbol{v}\)%
    \index{espacio vectorial!componentes (de un vector)|textbfhy}
  (en la base \(B\)).
  \begin{definition}
    \index{espacio vectorial!dimension@dimensión|textbfhy}
    Al número de vectores en una base de \(V\)
    se le llama la \emph{dimensión} de \(V\),
    anotada \(\dim V\).
    Un espacio vectorial abarcado por un conjunto finito de vectores
    se dice de \emph{dimensión finita},
    en caso contrario es de \emph{dimensión infinita}.
    Al espacio vectorial \(\{\boldsymbol{0}\}\)
    se le asigna dimensión cero.
    Se anota \([V : F]\)
    para la dimensión de \(V\) sobre el campo \(F\).
  \end{definition}
  En el caso de espacios vectoriales de dimensión finita
  es simple demostrar que todas las bases
  tienen la misma cardinalidad,
  con lo que nuestra definición de dimensión tiene sentido.
  \begin{theorem}
    \label{theo:espacio-vectorial-li}
    Si \(V\) es un espacio vectorial
    con base
      \(B = \{\boldsymbol{b}_1, \boldsymbol{b}_2,
	       \dotsc, \boldsymbol{b}_n\}\),
    y \(A = \{\boldsymbol{a}_1, \boldsymbol{a}_2,
	       \dotsc, \boldsymbol{a}_r\}\)
    es un conjunto linealmente independiente de vectores en \(V\),
    entonces \(r \le n\).
  \end{theorem}
  \begin{proof}
    Como \(B\) abarca \(V\),
    \(B \cup \{\boldsymbol{a}_1\}\) también abarca \(V\).
    Como \(\boldsymbol{a}_1 \ne \boldsymbol{0}\)
    (\(A\) es linealmente independiente),
    podemos expresar \(\boldsymbol{a}_1\)
    como combinación lineal de los \(B\),
    y en ella algún \(\boldsymbol{b}_t\)
    tendrá coeficiente diferente de 0.
    Ese \(\boldsymbol{b}_t\) puede expresarse en términos
    de \(B_1 = \{\boldsymbol{a}_1, \boldsymbol{b}_1,
		   \boldsymbol{b}_2,
		   \dotsc, \boldsymbol{b}_{t - 1},
		   \boldsymbol{b}_{t + 1},
		 \dotsc, \boldsymbol{b}_n\}\).
    Como todo \(\boldsymbol{v} \in V\)
    puede escribirse como combinación lineal
    de los \(B\),
    también puede escribirse como combinación lineal de los \(B_1\)
    (substituyendo la combinación
     de \(B_1\) que da \(\boldsymbol{b}_t\)
     en la combinación lineal para \(\boldsymbol{v}\)
     se obtiene una nueva combinación lineal).
    Este proceso puede repetirse
    intercambiando un \(A\) por uno de los \(B\),
    manteniendo siempre \(B_k\) como base,
    finalmente llegando
    a \(B_r = \{\boldsymbol{a}_1, \dotsc, \boldsymbol{a}_r,
		\boldsymbol{b}_{m_1}, \boldsymbol{b}_{m_2},
		\dotsc, \boldsymbol{b}_{m_s}\}\)
    (posiblemente no queden \(\boldsymbol{b}_{m_k}\) en \(B_r\)).
    No pueden quedar \(A\) si se acaban los \(B\),
    ya que si fuera así un \(\boldsymbol{a}_i\) sobrante
    no podría representarse como combinación lineal
    de los \(B\),
    y \(B\) no sería una base.
    Tenemos \(A \subseteq B_r\),
    y claramente
      \(\lvert A \rvert \le \lvert B_r \rvert = \lvert B \rvert\).
  \end{proof}
  Esto justifica la definición de la dimensión
  en el caso de espacios vectoriales de dimensión finita:
  \begin{corollary}[Teorema de dimensión de espacios vectoriales]
    \label{cor:espacio-vectorial-dimension}
    Si \(A\) y \(B\) son bases
    de un espacio vectorial de dimensión finita,
    entonces \(\lvert A \rvert = \lvert B \rvert\).
  \end{corollary}
  \begin{proof}
    La base \(A\) es linealmente independiente,
    con lo que por el teorema~\ref{theo:espacio-vectorial-li}
    es \(\lvert A \rvert \le \lvert B \rvert\).
    Por el mismo argumento,
    intercambiando los roles de \(A\) y \(B\),
    \(\lvert B \rvert \le \lvert A \rvert\),
    con lo que \(\lvert A \rvert = \lvert B \rvert\).
  \end{proof}
  Esto nos lleva a:
  \begin{theorem}
    \label{theo:espacio-vectorial-isomorfos}
    Todos los espacios vectoriales de la misma dimensión finita
    sobre \(F\) son isomorfos.
  \end{theorem}
  \begin{proof}
    Sean \(U\) y \(V\) espacios vectoriales
    de la misma dimensión finita,
    con bases \(\{\boldsymbol{a}_k\}_{1 \le k \le n}\)
    y \(\{\boldsymbol{b}_k\}_{1 \le k \le n}\),
    respectivamente.
    Podemos representar todos los vectores \(\boldsymbol{u} \in U\)
    y \(\boldsymbol{v} \in V\)
    mediante:
    \begin{equation*}
      \boldsymbol{u}
	= \sum_{1 \le k \le n} a_k \boldsymbol{a}_k \hspace{2em}
      \boldsymbol{v}
	= \sum_{1 \le k \le n} b_k \boldsymbol{b}_k
    \end{equation*}
    Definimos la biyección \(f \colon U \rightarrow V\)
    mediante:
    \begin{equation*}
      f \colon \sum_{1 \le k \le n} a_k \boldsymbol{a}_k
	\mapsto \sum_{1 \le k \le n} a_k \boldsymbol{b}_k
    \end{equation*}
    Demostrar que la suma vectorial
    y el producto escalar se preservan
    es rutinario.
  \end{proof}
  \noindent
  En vista de la demostración
  del teorema~\ref{theo:espacio-vectorial-isomorfos},
  en un espacio vectorial de dimensión finita
  basta elegir una base,
  cada vector puede representarse
  mediante la secuencia de los coeficientes en \(F\).
  La suma vectorial es sumar componente a componente,
  el producto escalar es multiplicar cada componente por el escalar.
  Es por esta representación que a secuencias de largo fijo
  les llaman vectores.

  Lo anterior solo cubre una peueña parte
  de la extensa teoría relacionada con operaciones lineales.
  Para profundizar en ella recomendamos el texto de Treil~%
    \cite{treil14:_linear_algeb_done_wrong}.

%%% Local Variables:
%%% mode: latex
%%% TeX-master: "clases"
%%% End:


\section{Estructura de los campos finitos}
\label{sec:estructura-campos-finitos}
\index{campo (algebra)@campo (álgebra)!finito!estructura|textbfhy}

  Profundizaremos nuestro estudio de los campos finitos,
  apoyados ahora en lo que sabemos de espacios vectoriales.%
    \index{espacio vectorial}
  \begin{lemma}
    \label{lem:unique-irreductible-root}
    Sea \(K\) el subcampo primo de \(F\),%
      \index{campo (algebra)@campo (álgebra)!subcampo!primo}
    y sea \(\alpha \in F\) el cero de un polinomio en \(K[x]\).
    Entonces hay un polinomio mónico%
      \index{polinomio!monico@mónico}
    único de grado mínimo en \(K[x]\)
    con \(\alpha\) de cero.
  \end{lemma}
  \begin{proof}
    Es simple
    demostrar que \(I = \{f \in K[x] \colon f(\alpha) = 0\}\)
    es un ideal de \(K[x]\).
    Como \(K[x]\) es un dominio de ideal principal,%
      \index{dominio de ideal principal}
    \(I = (h)\) para algún \(h \in K[x]\),
    donde \(h\) es mónico
    y de mínimo grado entre los elementos de \(I\),
    y es único con estas características.
  \end{proof}
  Al polinomio \(h\)
  de la demostración del lema~\ref{lem:unique-irreductible-root}
  se le llama el \emph{polinomio mínimo de \(\alpha\) sobre \(K\)}.%
    \index{polinomio!minimo (de un elemento)@mínimo (de un elemento)}
  Un elemento \(\alpha\) que es cero de un polinomio en \(K[x]\)
  se dice \emph{algebraico sobre \(K\)}.
    \index{anillo!elemento algebraico}
  \begin{theorem}
    \label{theo:minimal-polynomial-divides}
    Sea \(\alpha \in F\) el cero de un polinomio en \(K[x]\)
    y sea \(g\) el polinomio mínimo de \(\alpha\).
    Entonces
    \begin{enumerate}[label = (\roman*), ref = (\roman*)]
    \item
      \(g\) es irreductible en \(K[x]\)%
	\index{anillo!elemento irreductible}
    \item
      \(f(\alpha) = 0\)
      si y solo si \(g \mid f\)
    \end{enumerate}
  \end{theorem}
  \begin{proof}
    Por turno.
    \begin{enumerate}[label = (\roman*), ref = (\roman*)]
    \item
      Como \(g\) tiene un cero en \(F\),
      \(\deg(g) \ge 1\).
      Demostramos que \(g\) es irreductible por contradicción.
      Supongamos que podemos expresar \(g = h_1 h_2\) en \(K[x]\)
      con \(1 \le \deg(h_i) < \deg(g)\)
      para \(i = 1, 2\).
      Entonces \(g(\alpha) = h_1(\alpha) h_2(\alpha) = 0\),
      por lo que \(h_1(\alpha) = 0\) o \(h_2(\alpha) = 0\);
      o sea uno de los polinomios está en el ideal \(I\)%
	\index{anillo!ideal}
      de la demostración
	del lema~\ref{lem:unique-irreductible-root}.
      Al ser \(K[x]\) un dominio de ideal principal,%
	\index{dominio de ideal principal}
      ese ideal es el conjunto de los múltiplos de \(g\),
      con lo que \(g \mid h_1\) o \(g \mid h_2\),
      lo que es imposible porque sus grados son menores al de \(g\).
    \item
      Esto sigue de la definición de \(g\)
      como generador del ideal \(I\) del mencionado lema.
    \qedhere
    \end{enumerate}
  \end{proof}
  Antes de continuar,
  demostraremos que todos los campos finitos de orden \(q\)
  son isomorfos
  (ya sabemos que \(q = p^n\) para un primo \(p\)).
  De partida:
  \begin{theorem}[Polinomio universal]
    \index{polinomio!universal|textbfhy}
    \label{theo:universal-polynomial}
    Sea \(F\) un campo finito de orden \(q\).
    Entonces todos los elementos \(a \in F\)
    cumplen la ecuación:
    \begin{equation*}
      x^q - x = 0
    \end{equation*}
  \end{theorem}
  \begin{proof}
    Por el teorema de Lagrange,%
      \index{Lagrange, teorema de}
    si \(a \ne 0\) el orden multiplicativo de \(a\)
    divide a \(q - 1\);
    en particular:
    \begin{equation*}
      a^{q - 1} - 1
	= 0
    \end{equation*}
    Si multiplicamos esta ecuación por \(a\),
    resulta que para todo \(a \in F\):
    \begin{equation*}
      a^q - a
	= 0
      \qedhere
    \end{equation*}
  \end{proof}
  Nótese que el teorema~\ref{theo:universal-polynomial}
  dice que los elementos del campo finito \(F\)
  de orden \(q\)
  son todas los ceros del \emph{polinomio universal} \(x^q - x\).
  En particular,
  los polinomios mínimos de los elementos de \(F\)
  dividen a \(x^q - x\).
  \begin{theorem}
    \label{theo:finite-field-unique}
    Sean \(F\) y \(F'\) campos finitos de orden \(q\).
    Entonces \(F \cong F'\).
  \end{theorem}
  \begin{proof}
    Sabemos que si \(\lvert F \rvert = \lvert F' \rvert = p^n\),
    la característica de ambos campos es \(p\),
    en particular,
    el campo primo de ambos es isomorfo a \(\mathbb{Z}_p\).

    Sabemos que \(F^\times\) es cíclico
    (teorema~\ref{theo:F*-ciclico}),
    elijamos un generador \(\pi\) de \(F^\times\),
    y sea \(m(x)\) el polinomio mínimo de \(\pi\),
    que por la observación anterior
    (teorema~\ref{theo:minimal-polynomial-divides})
    con \(q = p^n\) en \(F\) cumple:
    \begin{equation*}
      m(x) \mid x^{q - 1} - 1
    \end{equation*}
    Consideremos el polinomio \(m(x)\) en \(F'\) ahora,
    donde también divide a \(x^q - x\)
    (los coeficientes y las operaciones al dividir
     son estrictamente en el subcampo primo,
     serán las mismas en \(F\) y \(F'\)).
    Acá podemos escribir:
    \begin{equation*}
      x^{q - 1} - 1
	= \prod_{a' \in F'^\times} (x - a')
    \end{equation*}
    por lo que \(m(x)\) se factoriza completamente en \(F'\):
    \begin{equation*}
      m(x)
	= (x - a_1') (x - a_2') \dotsm (x - a_d')
    \end{equation*}
    Elijamos un cero cualquiera de \(m(x)\) en \(F'\),
    digamos \(\pi' = a_1'\).
    Observamos primeramente que \(\pi'\) genera \(F'^\times\),
    ya que si su orden fuera \(d < q - 1\),
    cumpliría:
    \begin{equation*}
      x^d - 1
	= 0
    \end{equation*}
    Pero como \(m(x)\) es un polinomio irreductible,%
      \index{polinomio!irreductible}
    debe ser su polinomio mínimo,%
      \index{polinomio!minimo (de un elemento)@mínimo (de un elemento)}
    y en \(F'\):
    \begin{equation*}
      m(x) \mid x^d - 1
    \end{equation*}
    Volviendo a \(F\),
    esto significa que \(\pi\) también satisface esta ecuación,
    y tiene orden \(d < q - 1\)
    (o sea,
     no sería generador de \(F^\times\)).

    Hay un isomorfismo de grupo obvio%
      \index{grupo!isomorfo}
    entre \((F^\times, \cdot)\) y \((F'^\times, \cdot)\):
    \begin{equation*}
      \Theta(\pi^k)
	= \pi'^k
    \end{equation*}
    Podemos extenderlo a una biyección entre \(F\) y \(F'\)
    definiendo:
    \begin{equation*}
      \Theta(0)
	= 0
    \end{equation*}
    Resta demostrar que \(\Theta\) es un isomorfismo para la suma.
    Sean \(a, b \in F\),
    debemos mostrar que:
    \begin{equation*}
      \Theta(a + b)
	= \Theta(a) + \Theta(b)
    \end{equation*}
    Si \(a = 0\) o \(b = 0\),
    el resultado es inmediato,
    así que en lo que sigue \(a \ne 0\) y \(b \ne 0\).
    Debemos considerar los dos casos \(a + b = 0\)
    y \(a + b \ne 0\).
    Veamos primero el segundo,
    más general.
    Sean:
    \begin{equation*}
      a = \pi^i \quad b = \pi^j \quad a + b = \pi^k
    \end{equation*}
    Entonces en \(F\):
    \begin{equation*}
      \pi^i + \pi^j
	= \pi^k
    \end{equation*}
    Vale decir,
    \(\pi\) satisface la ecuación:
    \begin{equation*}
      x^i + x^j - x^k
	= 0
    \end{equation*}
    Por el teorema~\ref{theo:minimal-polynomial-divides}
    en \(F\):
    \begin{equation*}
      m(x) \mid x^i + x^j - x^k
    \end{equation*}
    Pero en tal caso esto también se cumple en \(F'\):
    \begin{equation*}
      \pi'^i + \pi'^j
	= \pi'^k
    \end{equation*}
    Esto es precisamente:
    \begin{equation*}
      \Theta(a + b)
	= \Theta(a) + \Theta(b)
    \end{equation*}

    Resta el caso \(a + b = 0\).
    Si la característica de los campos es \(2\),
    esto significa \(a = b\),
    y en consecuencia como \(\Theta(a) = \Theta(b)\)
    es:
    \begin{equation*}
      \Theta(a + b)
	= \Theta(0)
	= 0
	= \Theta(a) + \Theta(b)
    \end{equation*}
    Si la característica de \(F\) es impar,
    notamos que \(-1\) es el único elemento de orden \(2\),
    ya que:
    \begin{equation*}
      x^2 - 1
	= (x - 1) (x + 1)
    \end{equation*}
    no puede tener más de dos ceros.
    En efecto,
    como \(F\) tiene \(q\) elementos
    (y así \(F^\times\) tiene \(q - 1\) elementos),
    debe ser:
    \begin{equation*}
      -1 = \pi^{\frac{q - 1}{2}}
    \end{equation*}
    dado que el elemento al lado derecho tiene el orden correcto.
    Escribamos \(a = \pi^i\) y \(b = \pi^j\),
    donde podemos suponer sin pérdida de generalidad que \(i > j\),
    con lo que:
    \begin{align*}
      &\pi^i + \pi^j
	= 0 \\
      &\pi^i
	= - \pi^j \\
      &\pi^{i - j}
	= -1 \\
      &i - j
	= \frac{q - 1}{2} \\
      &\pi'^{(i - j)}
	= -1 \\
    \intertext{De acá, aplicando lo anterior en reversa en \(F'\):}
      &\pi'^i + \pi'^j
	= 0
    \end{align*}
    Vale decir:
    \begin{equation*}
      \Theta(a + b)
	= \Theta(a) + \Theta(b)
    \end{equation*}
    En resumen,
    la biyección \(\Theta\) preserva suma y multiplicación,
    es un isomorfismo entre los campos.
  \end{proof}
  Al campo finito de orden \(q\) se le anota \(\mathbb{F}_q\)
  (en la literatura más antigua se suele encontrar la notación
   \(\mathrm{GF}(q)\),
   por la abreviatura de \emph{campo de Galois}%
     \index{Galois, campo de|see{campo (álgebra)!finito}}%
     \index{Galois, Evariste}
   en honor a quien comenzó su estudio).
  Resulta curioso que todo polinomio irreductible de grado \(n\)
  da el mismo campo.
  Resta demostrar que tales campos existen para todo primo \(p\)
  y todo \(n\).

  Vimos
  (teorema~\ref{theo:a-invertible})
  que \(\mathbb{Z}_m = \mathbb{Z} / (m)\)
  es un campo solo cuando \(m\) es primo.
  Hay notables similitudes entre el anillo \(\mathbb{Z}\)
  y los anillos de polinomios \(K[x]\) sobre un campo \(K\)%
    \index{anillo!polinomios}
  -- particularmente si \(K\) es finito.
  Ya vimos un ejemplo de esto:
  En los números enteros
  hay infinitos primos,
  y pueden expresarse en forma esencialmente única
  como producto de primos
  (y un signo,
   vale decir multiplicar por una unidad),
  el teorema fundamental de la aritmética.%
    \index{teorema fundamental de la aritmetica@teorema fundamental de la aritmética}
  Para polinomios
  tenemos el teorema~\ref{theo:fundamental-arithmetic-polynomials}.%
    \index{polinomio!teorema fundamental de la aritmetica@teorema fundamental de la aritmética}
  Los primos en \(\mathbb{Z}\)
  corresponden a los polinomios mónicos irreductibles en \(K[x]\).
  Igual que la relación de congruencia entre enteros,%
    \index{polinomio!congruencia}
  podemos definirla para polinomios
  \(f(x), g(x), m(x) \in K[x]\):
  \begin{equation*}
    f(x)
      \equiv g(x) \pmod{m(x)}
  \end{equation*}
  siempre que para algún \(q(x) \in K[x]\):
  \begin{equation*}
    g(x) - f(x)
      = m(x) q(x)
  \end{equation*}
  Es rutinario verificar que es equivalencia en \(K[x]\),
  y que las clases de equivalencia
  forman un anillo \(K[x] / (m(x))\),
  el \emph{anillo de polinomios sobre \(K\) módulo \(m(x)\)}.
  Este anillo contiene el campo \(K\)
  como los polinomios constantes.
  \begin{theorem}
    \label{theo:polynomial-ring-modulo-m=field}
    El anillo cociente \(K[x] / (m(x))\)
    es un campo si y solo si \(m(x)\) es irreductible.
  \end{theorem}
  \begin{proof}
    Demostramos implicancia en ambas direcciones.
    Para el directo,
    sea \(f(x) \in K[x]\) tal que \(m(x) \centernot\mid f(x)\).
    Sabemos
    (sección~\ref{sec:dominios-euclidianos})%
      \index{dominio euclidiano}
    que los polinomios sobre un campo son un dominio euclidiano,
    es aplicable la identidad de Bézout
    y tenemos un inverso de \(f\).%
      \index{Bezout, identidad de@Bézout, identidad de}

    Para el recíproco,
    usamos contradicción.%
      \index{demostracion@demostración!contradiccion@contradicción}
    Supongamos que \(m(x)\) no es irreductible,
    vale decir:
    \begin{equation*}
      m(x)
	= a(x) \cdot b(x)
    \end{equation*}
    donde \(a(x)\) y \(b(x)\) no son constantes.
    Las clases de equivalencia correspondientes no son cero,
    pero:
    \begin{equation*}
      [a(z)] \cdot [b(x)]
	= [a(x) \cdot b(x)]
	= [m(x)]
	= 0
    \end{equation*}
    Al haber divisores de cero,
    no es campo.
  \end{proof}
  Considerando \(K[x] / (m(x))\) como espacio vectorial,%
    \index{espacio vectorial}
  es una extensión de \(K\):%
    \index{campo (algebra)@campo (álgebra)!extension@extensión}
  \begin{definition}
    \label{def:extension-degree}
    A la dimensión de \(F\) como espacio vectorial sobre \(K\)
    la anotamos \([F : K]\),
    y la llamamos el \emph{grado de la extensión}.%
      \index{campo (algebra)@campo (álgebra)!extension@extensión!grado|textbfhy}
  \end{definition}
  \begin{lemma}
    \label{lem:irreducible-equivalence-classes}
    Si \(m(x)\) es un polinomio irreductible de grado \(d\),
    entonces las clases de equivalencia:
    \begin{equation*}
      [1], [x], \dotsc, [x^{d - 1}]
    \end{equation*}
    forman una base para \(K[x] / (m(x))\).
    En particular:
    \begin{equation*}
      \dim_K K[x] / (m(x)) = \deg m(x)
    \end{equation*}
  \end{lemma}
  \begin{proof}
    Demostramos por contradicción
    que las clases son linealmente independientes.%
      \index{espacio vectorial!independencia lineal}
    Supongamos:
    \begin{equation*}
      c_0 [1] + c_1 [x] + \dotsb + c_{d - 1} [x^{d - 1}]
	= 0
    \end{equation*}
    Esto significa:
    \begin{equation*}
      m(x) \mid c_0 + c_1 x + \dotsb + c_{d - 1} x^{d - 1}
    \end{equation*}
    Esto solo es posible si todos los \(c_i = 0\),
    ya que el grado de \(m\) es \(d\).

    Para demostrar que las clases de equivalencia
    abarcan \(K[x] / (m(x))\),
    elijamos \(f(x) \in K[x]\) cualquiera.
    Podemos dividir:%
      \index{polinomio!algoritmo de division@algoritmo de división}
    \begin{equation*}
      f(x) = m(x) \cdot q(x) + r(x)
      \qquad
      \deg r(x) < \deg m(x)
    \end{equation*}
    Así:
    \begin{equation*}
      f(x) \equiv r(x) \pmod{m(x)}
    \end{equation*}
    O sea,
    si:
    \begin{equation*}
      r(x)
	= r_0 + r_1 x + \dotsb + r_{d - 1} x^{d - 1}
    \end{equation*}
    entonces:
    \begin{equation*}
      [f(x)]
	= [r(x)]
	= r_0 [1] + r_1 [x] + \dotsb + r_{d - 1} [x^{d - 1}]
      \qedhere
    \end{equation*}
  \end{proof}
  \begin{corollary}
    \label{cor:fielp-p^n}
    Si \(m(x) \in \mathbb{F}_q [x]\) es irreductible
    de grado \(n\),
    entonces \(\mathbb{F}_q[x] / (m(x))\) es de orden \(q^n\).
  \end{corollary}
  \begin{proof}
    Inmediato del teorema~\ref{theo:polynomial-ring-modulo-m=field}:
    \(\mathbb{F}_q[x] / (m(x))\)
    es un espacio vectorial de dimensión \(n\)
    sobre \(\mathbb{F}_q\),
    con lo que contiene \(q^n\) elementos.
  \end{proof}
  En lo anterior construimos un campo \(K[x] / (m(x))\)
  partiendo de un campo \(K\) y un polinomio irreductible sobre él.
  Vimos también que todos los campos finitos del mismo orden
  son isomorfos.
  Ahora la construcción inversa,
  buscando la relación entre un campo \(F\) y sus subcampos.
  \begin{definition}
    \label{def:extension-field}
    Sean \(\alpha_1, \alpha_2, \dotsc, \alpha_n \in F\),
    y sea \(K\) un subcampo de \(F\).
    Al mínimo subcampo de \(F\)
    que contiene \(\alpha_1, \alpha_2, \dotsc, \alpha_n\) y \(K\)
    se anota \(K(\alpha_1, \alpha_2, \dotsc, \alpha_n)\).
    A tales campos se les llama \emph{extensiones} de \(K\).%
      \index{campo (algebra)@campo (álgebra)!extension@extensión}
    Si \(F = K(\alpha)\),
    se dice que \(F\) es una \emph{extensión simple} de \(K\).%
      \index{campo (algebra)@campo (álgebra)!extension@extensión!simple}
  \end{definition}
  Nótese la similitud entre la definición~\ref{def:extension-field}
  y la noción de anillos cuadráticos%
    \index{anillo!cuadratico@cuadrático}
  vistos en la sección~\ref{sec:anillos-cuadraticos}.
  Allá usamos la notación \(\mathbb{Z}[\sqrt{2}]\) para el anillo,
  acá hablamos del campo \(\mathbb{Q}(\sqrt{2})\).

  Por la discusión anterior \(K(\alpha_1, \dotsc, \alpha_n)\)
  es un espacio vectorial sobre \(K\).%
    \index{espacio vectorial}
  \begin{theorem}
    \label{theo:extension-isomorphic-minimal-polynomial}
    Sea \(F\) una extensión del campo \(K\),
    y sea \(\alpha \in F\) el cero de un polinomio en \(K[x]\),
    con polinomio mínimo \(g\).%
      \index{anillo!polinomio minimo@polinomio mínimo}
    Entonces:
    \begin{enumerate}[label = (\roman*), ref = (\roman*)]
    \item
      \label{en:K(alpha)-g-1}
      \(K(\alpha)\) es isomorfo a \(K[x] / (g(x))\)
    \item
      \label{en:K(alpha)-g-2}
      \([K(\alpha) : K] = \deg(g)\)
      y \(\{1, \alpha, \alpha^2, \dotsc, \alpha^{n - 1}\}\)
      es una base de \(K(\alpha)\) sobre \(K\)
    \item
      \label{en:K(alpha)-g-3}
      Si \(\beta \in K(\alpha)\)
      es el cero de un polinomio en \(K[x]\),
      el grado del polinomio mínimo de \(\beta\)
      divide al grado de \(g\)
    \end{enumerate}
  \end{theorem}
  \begin{proof}
    Para el punto~\ref{en:K(alpha)-g-1},
    por el lema~\ref{lem:irreducible-equivalence-classes}
    la clase \([x]\) de \(K[x] / (g(x))\)
    satisface la ecuación:
    \begin{equation*}
      g(x) = 0
    \end{equation*}
    En consecuencia,
    \(K(\alpha) \cong K[x] / (g(x))\)
    ya que son campos finitos del mismo orden.
    El punto~\ref{en:K(alpha)-g-2} es inmediato de lo anterior.

    Para~\ref{en:K(alpha)-g-3},
    que \(K(\alpha)\) es un espacio vectorial sobre \(K(\beta)\),
    con lo que el grado
    del polinomio mínimo de \(\alpha\) sobre \(K(\beta)\)
    da la condición de divisibilidad prometida.
  \end{proof}
  De lo anterior tenemos directamente:
  \begin{corollary}
    \label{cor:[H:F]=[H:G][G:F]}
    Sean \(F \le G \le H\) campos finitos.
    Entonces:
    \begin{equation*}
      [H : F]
	= [H : G] \cdot [G : F]
    \end{equation*}
  \end{corollary}

  Lo anterior muestra un cero de cada polinomio irreductible,
  pero nos interesan todas los ceros.
  Al efecto,
  definimos:
  \begin{definition}
    Sea \(f \in K[x]\) un polinomio de grado positivo,
    y \(F\) una extensión de \(K\).
    Decimos que \(f\) \emph{se divide} en \(F\)
    si hay \(a \in K\) y \(\alpha_i \in F\) para \(1 \le i \le n\)
    tales que podemos escribir:
    \begin{equation*}
      f(x)
	= a (x - \alpha_1) (x - \alpha_2) \dotsm (x - \alpha_n)
    \end{equation*}
    El campo \(F\) se llama \emph{campo divisor}
    de \(f\) sobre \(K\)
    si \(f\) se divide en \(F\)
    y además es \(F = K(\alpha_1, \alpha_2, \dotsc, \alpha_n)\).
  \end{definition}

  El siguiente resultado es una pieza angular
  de la teoría de campos.
  \begin{theorem}[Kroneker]
    \index{Kronecker, teorema de}
    \index{Kronecker, Leopold}
    \label{theo:Kroneker}
    Para todo polinomio irreductible \(f(x)\) sobre el campo \(F\)
    hay una extensión en la cual \(f(x)\) tiene un cero.
  \end{theorem}
  \begin{proof}
    Sea \(f(x) = a_0 + a_1 x + \dotsb + a_n x^n \in F[x]\)
      irreductible.
    Consideremos el elemento \([x]\) en el campo \(F[x] / (f(x))\).
    Entonces en \(F[x] / (f(x))\):
    \begin{align*}
      f([x])
	&= [a_0] + [a_1] [x] + \dotsb + [a_n] [x^n] \\
	&= [a_0 + a_1 x + \dotsb + a_n x^n] \\
	&= 0
    \end{align*}
    Así \(F[x] / (f(x))\) es campo divisor,
    y en el \([x]\) es cero.
  \end{proof}
  \noindent
  Esto parece ser solo jugar con la notación,
  pero es más profundo:
  Hay que distinguir entre \(x\)
  (el símbolo usado para describir polinomios formales)
  y \([x]\)
  (la clase de equivalencia del polinomio \(x\) módulo \(f(x)\)
   sobre \(F\)).
  Además usamos las definiciones y propiedades
  de las operaciones entre clases de congruencia.
  También vemos
  que al aplicar repetidas veces el teorema de Kroneker
  obtenemos finalmente el campo divisor de cualquier polinomio.

  \begin{theorem}
    \index{polinomio!campo divisor}
    \label{theo:splitting-field-roots}
    Sea \(F\) un campo,
    \(f\) un polinomio irreductible sobre el campo \(K\)
    con ceros \(\alpha, \beta \in F\).
    Entonces \(K(\alpha) \cong K(\beta)\),
    con un isomorfismo que mantiene fijos los elementos de \(K\)%
      \index{campo (algebra)@campo (álgebra)!isomorfo}
    e intercambia los ceros \(\alpha\) y \(\beta\).
  \end{theorem}
  \begin{proof}
    Por el teorema de Kroneker,
    ambos son isomorfos a \(K[x] / (f(x))\),
    dado que el irreductible \(f\)
    es el polinomio mínimo de \(\alpha\) y \(\beta\).
    El isomorfismo claramente mantiene fijos los elementos de \(K\),
    \(\beta\) se expresa
    como una combinación lineal en \(K(\alpha)\)
    y similarmente \(\alpha\) en \(K(\beta)\).
  \end{proof}

  Lo siguiente básicamente recoge resultados previos.
  \begin{theorem}[Existencia y unicidad del campo divisor]
    \label{theo:E!-splitting-field}
    Todo polinomio tiene campo divisor único:
    \begin{enumerate}[label = (\roman*), ref = (\roman*)]
    \item
      Si \(K\) es un campo
      y \(f\) un polinomio de grado positivo en \(K[x]\),
      entonces existe un campo divisor de \(f\) sobre \(K\).
    \item
      Cualquier par de campos divisores de \(f\) sobre \(K\)
      son isomorfos bajo un isomorfismo
      que mantiene fijos los elementos de \(K\)
      y permuta ceros de \(f\).
    \end{enumerate}
  \end{theorem}
  Así podemos hablar
  de \emph{el} campo divisor de \(f\) sobre \(K\),
  que se obtiene
  adjuntando un número finito de elementos algebraicos a \(K\),
  y es una extensión finita de \(K\).

  \begin{theorem}[Existencia y unicidad de campos finitos]
    \label{theo:E!-FF}
    Para cada primo \(p\) y natural \(n\)
    hay un campo finito de orden \(p^n\).
    Todo campo finito de orden \(q = p^n\)
    es isomorfo al campo divisor
    de \(x^q - x\) sobre \(\mathbb{Z}_p\).
  \end{theorem}
  \begin{proof}
    Sea \(F\) el campo divisor de \(f(x) = x^q - x\)
    en \(\mathbb{Z}_p[x]\).
% Fixme: Traducción de "Splitting field"
    Como \(f'(x) = q x^{q - 1} - 1 = -1\) sobre \(\mathbb{Z}_p\),
    por el lema~\ref{lem:repeated-roots}
    \(f(x)\) no tiene factores repetidos,
    con lo que \(f(x)\) tiene \(q\) ceros en \(F\),
    exactamente los \(q\) elementos de \(F\).

    Por el teorema~%
     \ref{theo:extension-isomorphic-minimal-polynomial},
    el campo divisor es único.
  \end{proof}
  El \emph{polinomio universal} \(U_n (x) = x^{p^n} - x\)%
    \index{anillo!polinomio universal}
  tiene como ceros todos los elementos de \(\mathbb{F}_{p^n}\).
  Resulta que \(U_m (x) \mid U_n (x)\) si y solo si \(m \mid n\),
  pero curiosamente es más fácil demostrar
  algo bastante más general:
  \begin{theorem}
    \label{theo:U-gcd}
    Sobre\/ \(\mathbb{F}_p\):
    \begin{equation*}
      \gcd(U_m (x), U_n (x))
	= U_{\gcd(m, n)} (x)
    \end{equation*}
  \end{theorem}
  \begin{proof}
    Si \(m = n\) no hay nada que demostrar.
    Usamos inducción fuerte sobre \(n\) para \(m < n\).%
      \index{demostracion@demostración!induccion@inducción}
    \begin{description}
    \item[Base:]
      Cuando \(n = 1\),
      no hay nada que demostrar.
    \item[Inducción:]
      Sea \(r = n - m\)
      y consideremos:
      \begin{equation*}
	\left( U_m (x) \right)^p
	  = x^{p^{m + 1}} - x^p
      \end{equation*}
      ya que al aplicar el teorema del binomio%
	\index{binomio, teorema del}
      los términos intermedios
      se anulan por ser divisibles por \(p\).
      Aplicando lo anterior \(r\) veces resulta:
      \begin{align*}
	\left( U_m (x) \right)^{p^r}
	  &= x^{p^{m + n - m}} - x^{p^r} \\
	  & = x^{p^n} - x^{p^r} \\
	  &= U_n (x) - U_r (x)
      \end{align*}
      de lo que obtenemos:
      \begin{equation*}
	\gcd(U_m (x), U_n (x))
	  = \gcd(U_r (x), U_m (x))
      \end{equation*}
      Por inducción:
      \begin{align*}
	\gcd(U_r (x), U_m (x))
	  &= U_{\gcd(r, m)} (x) \\
	  &= U_{\gcd(m, n)} (x)
      \end{align*}
      ya que \(\gcd(r, m) = \gcd(n - m, m) = \gcd(m, n)\).
    \end{description}
    Por inducción
    lo prometido vale para todo \(m, n \in \mathbb{N}\).
  \end{proof}
  Así:
  \begin{corollary}
    \label{cor:U-divides}
    Sobre\/ \(\mathbb{F}_p\) es
    \(U_m (x) \mid U_n (x)\) si y solo si \(m \mid n\).
  \end{corollary}
  \begin{proof}
    Recurrimos a una cadena de equivalencias.
    Es claro que \(U_m (x) \mid U_n (x)\)
    si y solo si \(\gcd(U_m (x), U_n (x)) = U_m (x)\).
    Por el teorema~\ref{theo:U-gcd}
    esto es si y solo si \(\gcd(m, n) = m\),
    que es si y solo si \(m \mid n\).
  \end{proof}

  \begin{theorem}
    \label{theo:finite-field-extension=simple-extension}
    Sea \(F_q\) un campo finito
    y \(F_r\) una extensión finita de \(F_q\).
    Entonces
    \begin{enumerate}[label = (\roman*), ref = (\roman*)]
    \item
      \label{en:tffes-1}
      \(F_r\) es una extensión simple de \(F_q\),
      vale decir,
      hay \(\beta \in F_r\) tal que \(F_r = F_q(\beta)\)
    \item
      \label{en:tffes-2}
      Cualquier elemento primitivo de \(F_r\)
      sirve como elemento definidor \(\beta\)
    \end{enumerate}
  \end{theorem}
  \begin{proof}
    Para~\ref{en:tffes-1},
    sea \(\alpha\) un elemento primitivo de \(F_r\),
    con lo que \(F_q(\alpha) \subseteq F_r\).
    Por otro lado,
    \(F_q(\alpha)\) contiene a \(0\)
    y todas las potencias de \(\alpha\),
    que son los elementos de \(F_r^\times\);
    con lo que \(F_r \subseteq F_q(\alpha)\).
    En consecuencia \(F_r = F_q(\alpha)\).
    El punto~\ref{en:tffes-2} es inmediato de lo anterior.
  \end{proof}
  Así tenemos
  \begin{corollary}
    \label{cor:E-irreducible-degrees}
    Para cada primo \(p\)
    hay polinomios irreductibles
    de todo grado \(n \ge 1\) sobre \(\mathbb{Z}_p\).
  \end{corollary}
  \begin{proof}
    Por el teorema~%
      \ref{theo:extension-isomorphic-minimal-polynomial}
    toda extensión de \(\mathbb{Z}_p\) es isomorfa
    a algún \(\mathbb{Z}_p(\alpha) \cong \mathbb{Z}_p[x] / (g(x))\)
    donde \(g(x)\) es el polinomio mínimo de \(\alpha\),
    por el teorema~\ref{theo:minimal-polynomial-divides}
    el polinomio mínimo es irreductible.
    Por el teorema~\ref{theo:E!-FF}
    hay campos finitos de \(p^n\) elementos
    para todo primo \(p\) y natural \(n\).
    En consecuencia,
    hay polinomios irreductibles
    de todos los grados sobre \(\mathbb{Z}_p\).
  \end{proof}
  Podemos hacer más:
  \begin{theorem}
    \index{polinomio!irreductible!numero@número}
    \label{theo:number-irreducible-polynomials}
    Sea \(N_n\) el número de polinomios irreductibles de grado \(n\)
    sobre\/ \(\mathbb{F}_q\).
    Entonces:
    \begin{equation}
      \label{eq:N-irreducible-polynomials-degree-n}
      N_n
	= \frac{1}{n} \, \sum_{d \mid n} \mu(n / d)  \, q^d
    \end{equation}
  \end{theorem}
  \begin{proof}
    En \(\mathbb{F}_{q^n}\)
    cada elemento es cero de su polinomio mínimo.
    Tal polinomio mínimo de grado \(d\)
    es irreductible sobre \(\mathbb{F}_q\)
    y tiene \(d\) ceros distintos
    en \(\mathbb{F}_{q^n}\).
    Contabilizando los elementos de \(\mathbb{F}_{q^n}\)
    como los ceros de sus polinomios mínimos:
    \begin{equation*}
      \sum_{d \mid n} d N_d
	= q^n
    \end{equation*}
    Aplicando inversión de Möbius%
      \index{Mobius, inversion de@Möbius, inversión de}
    (teorema~\ref{theo:Moebius-inversion})
    obtenemos lo anunciado.
  \end{proof}
  Esto da otra demostración
  de que hay polinomios irreductibles de grado \(n\)
  sobre \(\mathbb{Z}_q\) para todo \(n \in \mathbb{N}\):
  Para \(n = 1\),
  todos los polinomios son irreductibles.
  Si \(n \ge 2\),
  en la suma~\eqref{eq:N-irreducible-polynomials-degree-n}
  el término \(q^n\) es mayor que la suma de los demás,
  ya que como \(\lvert \mu(x) \rvert \le 1\) podemos acotar:
  \begin{equation*}
    \left\lvert
      \sum_{\substack{
	      d \mid n \\
	      d < n
	   }} \mu(n / d) q^d
    \right\rvert
      \le \sum_{\substack{
		  d \mid n \\
		  d < n
	       }} q^d
      \le \sum_{0 \le d \le n - 1} q^d
      = \frac{q^n - 1}{q - 1}
      < q^n
  \end{equation*}
  Así la suma en~\eqref{eq:N-irreducible-polynomials-degree-n}
  nunca se anula si \(n \ge 2\).
  Uniendo este resultado con el caso \(n = 1\),
  \(N_n > 0\) para todo \(n \in \mathbb{N}\).
  En vista del corolario~\ref{cor:U-divides},
  podemos obtener todos los polinomios irreductibles
  sobre \(\mathbb{F}_p\)
  de grado hasta \(n\) como factores de \(U_n(x)\).

% codigo-deteccion-errores.tex
%
% Copyright (c) 2012-2014 Horst H. von Brand
% Derechos reservados. Vea COPYRIGHT para detalles

\section{Códigos de detección y corrección de errores}
\label{sec:codigos-errores}
\index{codigo@código}

  Consideremos \emph{mensajes} de \(m\)~bits de largo%
    \index{mensaje}
  que se transmiten por algún medio
  (podría ser simplemente que se almacenan y se recuperan luego).
  En este proceso pueden ocurrir errores,
  que interesa detectar o corregir.
  El tema fue estudiado inicialmente por Hamming~%
    \cite{hamming50:_error_detec_correc_codes}.%
    \index{Hamming, Richard}
  Para ello usamos \emph{palabras de código}
  de \(n\) bits de largo,%
    \index{codigo@código!palabra}
  donde obviamente \(n \ge m\),
  usando los bits adicionales para detectar o corregir errores,
  usando solo \(2^m\) de las \(2^n\) palabras posibles.
  Si se recibe una palabra errada
  (que no corresponde al código),
  una estrategia obvia es suponer que el código correcto
  es el más cercano al recibido,
  vale decir,
  el que difiere en menos bits del recibido.
  Al número de bits en que difieren dos palabras se les llama
  la \emph{distancia de Hamming}%
    \index{Hamming, distancia de|textbfhy}
  entre ellas.
  Por ejemplo,
  la distancia de Hamming entre \(10111011\) y \(10010100\)
  es \(5\).
  A la distancia de Hamming mínima entre dos palabras de un código
  se le conoce como \emph{distancia de Hamming del código}.%
    \index{codigo@código!Hamming, distancia de|textbfhy}
  Lo que interesa entonces es hallar códigos
  de distancia de Hamming máxima
  en forma uniforme
  (nos interesa que a cada código correcto
   le corresponda un número similar
   de palabras erróneas)
  y por el otro lado hallar formas eficientes
  de determinar si la palabra es correcta
  (solo detectar errores)%
     \index{codigo@código!deteccion de errores@detección de errores}
  o la más cercana a la recibida
  (para corregirlos).%
    \index{codigo@código!correccion de errores@corrección de errores}
  Para detectar \(d\) errores
  se requiere que la distancia de Hamming
  del código sea mayor a \(d + 1\)
  (así la palabra errada nunca coincide con una correcta,
   está al menos a un bit de distancia),
  para corregir \(d\) errores la distancia debe ser \(2 d + 1\)
  (la palabra errada estará a distancia a lo más \(d\)
   de la correcta,
   la siguiente más cercana estará
   a la distancia al menos \(d + 1\)).

\subsection{Códigos de Hamming}
\label{sec:codigos-Hamming}
\index{codigo@código!Hamming}

  Hamming~%
    \cite{hamming50:_error_detec_correc_codes}
  halló una manera de construir códigos de detección
  y corrección de errores de distancia~\(3\)
  (capaces de detectar \(2\) errores y corregir \(1\)).
  Suponiendo \(n = 2^w\)
  y contando los bits de \(1\) a \(2^w\),
  se usan los bits en las posiciones \(k = 1, 2, \dotsc, 2^{w - 1}\)
  como bits de paridad%
    \index{paridad, bits de}
  y los demás como bits de datos.
  La idea es calcular el bit en la posición \(2^k\)
  de forma que los bits en las posiciones
  escritas en binario que tienen \(1\) en la posición~\(k\)
  tengan un número par
  de unos,
  como muestra el cuadro~\ref{tab:paridad-Hamming}.
  \begin{table}[ht]
    \centering
    \begin{tabular}{|l@{\hspace{0.5em}}>{\(}r<{\)}|r*{7}{@{, }r}|}
      \hline
      \multicolumn{2}{|c|}{\textbf{Paridad\rule[-0.7ex]{0pt}{3ex}}} &
	\multicolumn{8}{c|}{\textbf{Posiciones}} \\
      \hline\rule[-0.7ex]{0pt}{3.2ex}%
      0 & 2^0 = \phantom{0}1 &
	 \phantom{0}1 &	 \phantom{0}3 &
	 5 &  7 &  9 & 11 & 13 & 15 \\
      1 & 2^1 = \phantom{0}2 &
	 2 &  3 &  6 &	7 & 10 & 11 & 14 & 15 \\
      2 & 2^2 = \phantom{0}4 &
	 4 &  5 &  6 &	7 & 12 & 13 & 14 & 15 \\
      3 & 2^3 = \phantom{0}8 &
	 8 &  9 & 10 & 11 & 12 & 13 & 14 & 15 \\
      \hline
    \end{tabular}
    \caption{Paridades para el código de Hamming $(15, 4)$}
    \label{tab:paridad-Hamming}
  \end{table}
  Para determinar el bit errado,
  lo que se hace es calcular los bits de paridad correspondientes
  a los datos recibidos,
  si son iguales a lo calculado,
  no se detectan errores;
  en caso de haber diferencias
  considerar los bits de paridad
  como un número binario da el bit errado.
  Por ejemplo,
  si \(w = 15\),
  el código tendrá \(4\)~bits de paridad
  (posiciones \(1\), \(2\), \(4\) y~\(8\))
  y \(11\)~bits de mensaje
  (en las posiciones \(3\), \(5\) a~\(7\) y \(9\) a~\(15\)).
  Se recibe \lstinline[language=C]!0x6A6A!,
  en binario \(0110\,1010\,0110\,1010\),
  revisamos los bits respectivos,
  lo que da \(3\) para \(0\),
  \(6\) para \(1\),
  \(6\) para \(2\)
  y \(4\) para \(3\).
  Esto corresponde a \(0001\),
  que significa que hay un error en la posición \(1\).
  El código correcto es \(0110\,1010\,0110\,1011\)
  o \lstinline[language=C]!0x6A6B!,
  y los bits de mensaje son \(110\,1010\,1100\),
  o \lstinline[language=C]!0x6AC!.
  El código de Hamming es óptimo,
  en el sentido que tiene la máxima distancia de Hamming
  para el número de bits dado.

\subsection{Verificación de redundancia cíclica}
\label{sec:CRC}
\index{codigo@código!redundancia ciclica@redundancia cíclica}
\index{CRC@\emph{CRC}|ver{código!redundancia cíclica}}

  Una manera de construir códigos de detección de errores
  simples de analizar matemáticamente
  fue descubierta por Peterson~%
    \cite{peterson61:_CRC}.
  La técnica tiene además la ventaja
  de poder implementarse en circuitos sencillos y rápidos,
  como veremos luego.
  Se les llama \emph{verificación de redundancia cíclica}
  (en inglés,
   \emph{\foreignlanguage{english}{Cyclic Redundancy Check}},
   o CRC)
  dado que se agregan bits de verificación
  (\emph{\foreignlanguage{english}{check}} en inglés)
  que son redundantes
  (no aportan información)
  según un código cíclico.

  Consideremos un mensaje binario de \(m\) bits,
  \(M = M_{m - 1} M_{m - 2} \dotso M_0\).
  Podemos considerarlo como un polinomio sobre \(\mathbb{Z}_2\):%
    \index{polinomio!campo finito}
  \begin{equation*}
    M(x)
      = M_{m - 1} x^{m - 1}
	 + M_{m - 2} x^{m - 2}
	 + \dotsb
	 + M_1 x
	 + M_0
  \end{equation*}
  Sea además un polinomio \(G(x)\) de grado \(n - 1\).
  Si calculamos:
  \begin{align*}
    r(x)
      &= M(x) x^n \bmod G(x) \\
    T(x)
      &= M(x) x^n - r(x)
  \end{align*}
  Es claro que \(T(x)\) es divisible por \(G(x)\).
  La estrategia es entonces tomar el mensaje,%
    \index{mensaje}
  añadirle \(n - 1\) bits cero al final,
  calcular el resto de esto al dividir por \(G(x)\)
  (un polinomio de grado \(n - 1\))
  y substituir los ceros añadidos por el resto
  (en \(\mathbb{Z}_2[x]\) suma y resta son la misma operación).
  A estos bits agregados los llamaremos \emph{bits de paridad}.
  El resultado es el polinomio \(T(x)\),
  que se trasmite.
  Al recibirlo,
  se calcula el resto de la división con \(G(x)\);
  si el resto es cero,
  el dato recibido es correcto.
  Al polinomio \(G(x)\) se le llama \emph{generador} del código.%
    \index{codigo@código!polinomio generador}%
    \index{polinomio!generador de un codigo@generador de un código|see{código!polinomio generador}}
  Esto es similar a la prueba de los nueves
  que discutimos en el capítulo~\ref{cha:estructura-Zm}.

  Si se transmite \(T\) y se recibe \(R\),
  el error
  (las posiciones de bit erradas)
  es simplemente la diferencia entre los polinomios respectivos:%
    \index{error}
  \begin{equation*}
    E(x)
      = T(x) - R(x)
  \end{equation*}
  Para que nuestra técnica detecte el error,
  debe ser que \(G(x)\) no divida a \(E(x)\).
  Nos interesa entonces estudiar bajo qué condiciones \(G(x)\)
  divide a \(E(x)\) en \(\mathbb{Z}_2[x]\),
  de manera de obtener criterios
  que den buenos polinomios generadores
  (capaces de detectar clases de errores de interés).

  Claramente no todo \(G(x)\) sirve,
  usaremos la teoría desarrollada antes
  para poner algunas condiciones.
  De partida,
  el término constante de \(G(x)\)
  no debe ser cero,
  de otra forma se desperdician bits de paridad:
  \begin{equation*}
    \left( x^n M(x) \right) \bmod \left( x^k p(x) \right)
      = x^{n - k} \left( M(x) \bmod p(x) \right)
  \end{equation*}

  Si consideramos cómo se multiplican polinomios
  en \(\mathbb{Z}_2[x]\),
  vemos que si \(G(x)\) tiene un número par de coeficientes uno
  lo mismo ocurrirá con el producto \(p(x) G(x)\),
  con lo que si \(G(x)\) tiene un número par de coeficientes uno
  detectará todos los errores que cambian un número impar de bits.

  Por otro lado,
  si \(G(x)\) es de grado \(n - 1\)
  no puede dividir a polinomios de grado menor.
  Vale decir,
  será capaz de detectar todos los errores
  que cambian bits en un rango contiguo de menos de \(n\) bits.

  Una \emph{ráfaga}%
    \index{error!rafaga@ráfaga}
  es un bloque de bits cambiados,
  con lo que \(E = 0 0 \dotsm 0 1 1 \dotsm 1 1 0 \dotsm 0\).
  Si se cambian \(r\) bits,
  esto significa que para algún \(k\):
  \begin{align*}
    E(x)
      &= (x^{r - 1} + x^{r - 2} + \dotsb + 1) x^k \\
      &= \frac{(x^r - 1) x^k}{x - 1}
  \end{align*}
  Esto es divisible por \(G(x)\) si lo es \(x^r - 1\).
  De la teoría precedente sobre campos finitos
  sabemos que si \(G(x)\)
  es el polinomio mínimo de un generador de \(\mathbb{F}_{2^n}\)
  (lo que llaman un \emph{polinomio primitivo})
  dividirá a \(x^r - 1\) solo si \(r \ge 2^n - 1\).
  Lamentablemente
  en \(\mathbb{Z}_2[x]\) el polinomio \(x + 1\) es primitivo
  y divide a todos los polinomios con un número par de términos
  (porque \(x - 1\) siempre divide a \(x^k - 1\)).

  Analicemos ahora cómo armar circuitos
  que calculen el resto de la división
  de polinomios en \(\mathbb{Z}_2[x]\).
  Requeriremos memorias de un bit,
  que al pulso de una línea de reloj
  (que no se muestra)
  aceptan un nuevo bit y entregan el anterior.
  La suma en \(\mathbb{Z}_2\)
  es la operación lógica \emph{o exclusivo},
  comúnmente anotada \(\oplus\).
  \begin{figure}[ht]
    \centering
    \subfloat[Registro]{\pgfimage{images/register}
       \label{subfig:register}}
    \qquad
    \subfloat[O exclusivo]{\pgfimage{images/xor}
       \label{subfig:xor}}
    \caption{Elementos de circuitos lógicos}
    \label{fig:circuitos-logicos}
  \end{figure}
  Los elementos de circuito que emplearemos
  se ilustran en la figura~\ref{fig:circuitos-logicos}.
  El amable lector verificará
  (por ejemplo dividiendo \(x^8 + x^5 + x^4 + x^2 + 1\)
   por \(x^4 + x + 1\))
  que el proceso para obtener el resto
  puede describirse de la siguiente forma:
  Si el primer bit del dividendo actual es \(1\),
  sume los términos de menor exponente
  a partir del segundo término del dividendo;
  en caso que el primer bit del dividendo sea \(0\),
  no haga nada.
  Luego descarte el primer bit del dividendo.
  Esto es lo mismo que sumar el primer bit del dividendo actual
  en ciertas posiciones,
  y luego correr todo en una posición.
  En términos de nuestros elementos,
  para el polinomio primitivo \(x^8 + x^4 + x^3 + x^2 + 1\)
  resulta el circuito de la figura~\ref{fig:LFSR-11d}.
  \begin{figure}[ht]
    \centering
    \pgfimage{images/LFSR-11d}
    \caption{Circuito para $x^8 + x^4 + x^3 + x^2 + 1$}
    \label{fig:LFSR-11d}
  \end{figure}
  La operación es la siguiente:
  Inicialmente se cargan ceros en los registros,
  luego se van ingresando los bits del dividendo
  (partiendo por el más significativo)
  al circuito.
  El resto queda en los registros.
  Es claro que interesan polinomios primitivos
  con el mínimo número de términos
  (ya que esto minimiza la circuitería requerida).

  Otro uso interesante resulta de inicializar los registros
  con un valor diferente de cero,
  y luego alimentar el circuito con una corriente de ceros
  (lo que puede lograrse simplemente obviando
   la primera operación a la izquierda en la figura)
  Como hay un número finito de posibilidades
  para los valores de los registros,
  en algún momento se repetirán.
  Si el valor inicial es \(p(x)\),
  lo que estamos haciendo
  es calcular sucesivamente \(x^k p(x) \bmod G(x)\).
  Si \(G(x)\) es primitivo,
  la repetición ocurrirá cuando \(k = 2^n\),
  por lo que los valores en los registros
  habrán recorrido todas las combinaciones de \(n\) bits
  (salvo solo ceros).
  El resultado es un contador simple
  (si solo interesa obtener valores diferentes,
   no necesariamente en orden),
  y la salida del circuito
  (que en nuestra aplicación anterior descartamos)
  es una corriente de números aleatorios
  si se toman de a \(n\) bits.
  Si \(G(x)\) no es primitivo,
  la teoría precedente indica que habrá un \(k\) menor a \(n\)
  que hace que \(x^k p(x) \equiv p(x) \pmod{G(x)}\),
  y nuevamente hay ciclos.
  El lector interesado determinará los posibles ciclos
  para algún polinomio no primitivo,
  como \((x^4 + x + 1) (x + 1) = x^5 + x^4 + x^2 +1\).

%%% Local Variables:
%%% mode: latex
%%% TeX-master: "clases"
%%% End:


%%% Local Variables:
%%% mode: latex
%%% TeX-master: "clases"
%%% End:


% algoritmos.tex
%
% Copyright (c) 2009-2014 Horst H. von Brand
% Derechos reservados. Vea COPYRIGHT para detalles

\chapter{Algoritmos aritméticos}
\label{cha:algoritmos}
\index{algoritmos aritmeticos@algoritmos aritméticos|textbfhy}

  Hay aplicaciones en las cuales se requieren cálculos
  con números de muchos miles de bits de largo.
  Tales algoritmos son particularmente relevantes
  en las técnicas criptográficas modernas,
  basadas en teoría de números y áreas afines.
  Algunos de los algoritmos que discutiremos son relevantes
  incluso para números pequeños.
  También ofrecen ejemplos de técnicas de análisis de algoritmos
  que tienen interés independiente.

\section{Referencias detalladas}
\label{sec:referencias-detalladas}

  Esta es un área muy amplia,
  desarrollada explosivamente desde la aparicición de criptografía
  basada en teoría de números.
  En este reducido espacio es imposible hacerle justicia.
  Una discusión exhaustiva de algoritmos aritméticos y afines,
  con análisis muy detallado de su rendimiento,
  es de Knuth~\cite{knuth97:_semin_algor}.
  Detalles sobre algoritmos numéricos adicionales,
  incluyendo resultados recientes en el área
  y con énfasis en algoritmos aplicables para criptología,
  dan Brent y Zimmermann~%
    \cite{brent10:_moder_comput_arith}.
  Una discusión de algoritmos en \cplusplus{}%
    \index{C++ (lenguaje de programacion)@\cplusplus{} (lenguaje de programación)}
  da Arndt~%
    \cite{arndt11:_matters_computational}.
  Una implementación libre de algoritmos aritméticos
  de buen rendimiento es la biblioteca GMP~%
     \cite{granlund14:_gnu_multip_precis_arith_librar}.%
     \index{GMP@\texttt{GMP}}
  Hay varias otras opciones,
  como NTL~%
     \cite{shoup14:_ntl},%
     \index{NTL@\texttt{NTL}}
  que ofrece una interfaz más cómoda de usar,
  y CLN~%
    \cite{haible14:_CLN_1.3.4}%
    \index{CLN@\texttt{CLN}}
  para uso desde \cplusplus.
  Para cómputo en otras estructuras algebraicas
  (por ejemplo,
   grupos elípticos o campos finitos)
  se recomienda GAP~%
    \cite{GAP:4.7.5}.%
    \index{GAP@\texttt{GAP}}

\section{Máximo común divisor}
\label{sec:gcd}
\index{maximo comun divisor@máximo comun divisor}

  Para el máximo común divisor
  (que ya discutimos en la sección~\ref{sec:GCD}),
  notar que para \(q\) arbitrario debe ser:
  \begin{equation*}
    \gcd(a, b) = \gcd(b, a - q b)
  \end{equation*}
  Esto porque cualquier divisor común de \(b\) y \(a - q b\)
  necesariamente divide a \(a\) también.
  Interesa disminuir lo más posible los valores en cada iteración,
  cosa que se logra si elegimos \(q = \lfloor a / b \rfloor\).
  Como los valores son enteros no negativos
  y disminuyen en cada paso,
  el proceso no puede continuar indefinidamente.
  Esta observación lleva al algoritmo de Euclides,
  algoritmo~\ref{alg:gcd},
  para calcular el máximo común divisor.
  \begin{algorithm}[htbp]
    \DontPrintSemicolon
    \SetKwFunction{Gcd}{gcd}

    \KwFunction \Gcd{\(a, b\)} \;
    \BlankLine
    \While{\(b > 0\)}{
      \((a, b) \leftarrow (b, a \bmod b)\) \;
    }
    \Return \(a\) \;
    \caption{Algoritmo de Euclides para calcular $\gcd(a, b)$}
    \label{alg:gcd}
  \end{algorithm}
  Es interesante considerar el número de iteraciones del algoritmo.
  Sean \(r_i\) los restos en cada paso del algoritmo,
  con el entendido que \(r_0 = a\) y \(r_1 = b\),
  y que \(a > b\)
  (en caso contrario,
   lo único que hace la primera iteración es intercambiarlos).
  Estamos calculando:
  \begin{equation*}
    r_{i + 2} = r_i \bmod r_{i + 1}
  \end{equation*}
  O sea,
  para una secuencia de \(q_i\)
  tenemos las relaciones:
  \begin{equation*}
    r_{i + 2} = r_i - q_i r_{i + 1}
  \end{equation*}
  donde \(r_k \ne 0\) y \(r_{k + 1} = 0\)
  si hay \(k\) iteraciones.
  El peor caso se da cuando \(q_i = 1\) siempre,
  ya que en tal caso
  los \(r_i\) disminuyen lo más lentamente posible.
  Además,
  el caso en que \(\gcd(a, b) = 1\)
  es el en el cual más terreno se debe recorrer.
  Podemos dar vuelta esto,
  y preguntarnos qué tan lejos del final estamos,
  y calcular desde allí:
  \begin{equation}
    \label{eq:Fibonacci}
    F_{k + 2} = F_{k + 1} + F_k \qquad F_0 = 0, F_1 = 1
  \end{equation}
  Esto define la famosa secuencia de Fibonacci:%
    \index{Fibonacci, numeros de@Fibonacci, números de}

  \noindent
    \hspace{2.3em} \(0\), \(1\), \(1\), \(2\), \(3\), \(5\),
		   \(8\), \(13\), \(21\), \ldots

  \noindent
  El resultado es entonces
  que si el algoritmo efectúa \(k\) iteraciones
  entonces \(b \ge F_k\).
  Esto fue demostrado por Lamé en 1844~%
    \cite{lame44:_gcd},%
    \index{Lame, Gabriel Leon Jean Baptiste@Lamé, Gabriel Léon Jean Baptiste}
  lo que inauguró el área de análisis de algoritmos.%
    \index{Euclides, algoritmo de!analisis@análisis}
    \index{analisis de algoritmos@análisis de algoritmos}
  Fue también el primer uso serio de los números de Fibonacci.

  Para completar el análisis interesa saber cómo crece \(F_k\).
  Más adelante
  (capítulo~\ref{cha:aplicaciones})
  veremos cómo tratar esta clase de situaciones,
  por ahora nos contentamos con una cota.
  Si calculamos las razones \(F_{k + 1} / F_k\),
  vemos que parecen converger a una constante cerca de \(1,6\).
  Llamemos:
  \begin{equation*}
    \tau
      = \lim_{k \rightarrow \infty} \frac{F_{k + 1}}{F_k}
  \end{equation*}
  Podemos expresar:
  \begin{align*}
    \frac{F_{k + 2}}{F_k}
      &= \frac{F_{k + 1}}{F_k} + 1 \\
    \frac{F_{k + 2}}{F_{k + 1}} \cdot \frac{F_{k + 1}}{F_k}
      &= \frac{F_{k + 1}}{F_k} + 1 \\
  \end{align*}
  Si ahora hacemos \(k \rightarrow \infty\),
  queda:
  \begin{equation}
    \label{eq:tau}
    \tau^2
      = \tau + 1
  \end{equation}
  La ecuación~\eqref{eq:tau} tiene dos raíces,
  interesa la positiva
  (de mayor magnitud):
  \begin{equation*}
    \tau
      = \frac{1 + \sqrt{5}}{2}
      \approx 1,618
  \end{equation*}
  Se cumplen las siguientes relaciones para \(k \ge 1\):
  \begin{equation}
    \label{eq:cotas-Fk}
    \tau^{k - 2} \le F_k \le \tau^{k - 1}
  \end{equation}
  Requerimos dos valores de partida,
  dado que a la recurrencia~\eqref{eq:Fibonacci}
  entran dos valores.%
    \index{recurrencia}
  Estas se demuestran por inducción.%
    \index{demostracion@demostración!induccion@inducción}
  \begin{description}
  \item[Base:]
    Cuando \(k = 1\) y \(k = 2\) tenemos:
    \begin{align*}
      \tau^{-1}
	&\le F_1 \le 1 \\
      1
	&\le F_2 \le \tau
    \end{align*}
    Ambas son ciertas.
  \item[Inducción:]
    Suponiendo que la aseveración es cierta hasta \(k + 1\),
    planteamos las cotas~\eqref{eq:cotas-Fk} para \(k\) y \(k + 1\):
    \begin{align*}
      \tau^{k - 2} &\le F_k	  \le \tau^{k - 1}	\\
      \tau^{k - 1} &\le F_{k + 1} \le \tau^k
    \end{align*}
    Sumando las relaciones resultantes,
    y viendo de la ecuación~\eqref{eq:tau}
    que \(\tau^2 = 1 + \tau\):
    \begin{equation*}
      \begin{array}{rcccl}
	\tau^{k - 2} + \tau^{k - 1}
	  &\le & F_k + F_{k + 1}
	  &\le & \tau^{k - 1} + \tau^k \\
	\tau^{k - 2}(1 + \tau)
	  &\le & F_{k + 2}
	  &\le & \tau^{k - 1} (1 + \tau) \\
	\tau^k
	  &\le & F_{k + 2}
	  &\le & \tau^{k + 1}
      \end{array}
    \end{equation*}
    que es el caso siguiente.
  \end{description}

  Con estas estimaciones de \(F_k\),
  tenemos que si el algoritmo da a lo más \(k\) iteraciones:
  \begin{align*}
    b &\ge F_k \\
      &\ge \tau^{k - 2} \\
    k &\le \frac{\log b}{\log \tau} + 2
  \end{align*}
  O sea,
  \(k = O(\log b)\).

  Puede analizarse el comportamiento promedio del algoritmo,
  pero eso lleva a profundidades que escapan de este ramo.
  El detalle se encuentra en el texto de Knuth~%
    \cite{knuth97:_semin_algor}.

  Un algoritmo alternativo
  (máximo común divisor binario)%
    \index{maximo comun divisor@máximo común divisor!algoritmo binario}
  se obtiene de aplicar repetidas veces
  las siguientes observaciones:
  \begin{enumerate}
  \item
    \(\gcd(a, b) = \gcd(b, a)\) nos permite reordenar a gusto.
  \item
    \(\gcd(a, 0) = a\) da el resultado final.
  \item
    \(\gcd(a, b) = 2 \, \gcd(a / 2, b / 2)\)
    cuando \(a\) y \(b\) son pares.
  \item
    \(\gcd(a, b) = \gcd(a / 2, b)\)
    cuando \(a\) es par y \(b\) impar.
  \item
    \(\gcd(a, b) = \gcd(b, (a - b) / 2)\)
    cuando \(a\) y \(b\) son impares
    (en tal caso, \(a - b\) es par).
  \end{enumerate}
  En máquinas en las cuales la división es lenta
  este algoritmo puede ser más eficiente
  si se programa con cuidado aprovechando operaciones con bits.

  Nuestra versión~(\ref{alg:gcd-binario})
  primero extrae la máxima potencia de \(2\)
  que tienen en común \(a\) y \(b\),
  de allí en adelante trabaja solo con números impares,
  asegurándose de mantener siempre \(a \ge b\).
  \begin{algorithm}[htbp]
    \DontPrintSemicolon
    \SetKwFunction{Gcd}{gcd}

    \KwFunction \Gcd{\(a,\; b\)} \;
    \BlankLine
    \(u \leftarrow 1\) \;
    \While{\((2 \mid a) \wedge (2 \mid b)\)}{
      \((u, a, b) \leftarrow (2 u, a / 2, b / 2)\) \;
    }
    \While{\(2 \mid a\)}{
      \(a \leftarrow a / 2\) \;
    }
    \While{\(2 \mid b\)}{
      \(b \leftarrow b / 2\) \;
    }
    \If{\(a < b\)}{
      \((a, b) \leftarrow (b, a)\) \;
    }
    \Loop{
      \(t \leftarrow a - b\) \;
      \If{\(t = 0\)}{
	\Return \(a \cdot u\) \;
      }
      \Repeat{\(2 \centernot\mid t\)}{
	\(t \leftarrow t / 2\) \;
      }
      \((a, b) \leftarrow (b, t)\) \;
    }
    \caption{Máximo común divisor binario}
    \label{alg:gcd-binario}
  \end{algorithm}
  Un ejemplo del algoritmo binario
  sería el cálculo de \(\gcd(40\,902, 24\,140)\).
  Como \(40\,902 = 2 \cdot 20\,451\)
  y \(24\,140 = 4 \cdot 6\,035\),
  la máxima potencia de \(2\) que tienen en común \(a\) y \(b\)
  es \(u = 2\),
  y el algoritmo propiamente tal
  se inicia con \(a = 20451\), \(b = 6035\).
  \begin{table}[htbp]
    \centering
    \begin{tabular}{|>{\(}r<{\)}|>{\(}r<{\)}
		    |>{\(}r<{\)}@{${} - {}$}>{\(}r<{\)}
		       @{${} = {}$}>{\(}r<{\)}@{${} \cdot {}$}>{\(}r<{\)}|}
      \hline
      \multicolumn{1}{|c|}{\rule[-0.7ex]{0pt}{3ex}\(\boldsymbol{a}\)} &
	\multicolumn{1}{c|}{\(\boldsymbol{b}\)} &
	\multicolumn{4}{c|}{\(\boldsymbol{t}\)} \\
      \hline\rule[-0.7ex]{0pt}{3ex}%
      20\,451 & 6\,035 & 20\,451 & 6\,035 & 16 &    901 \\
       6\,035 &	   901 &  6\,035 &    901 &  2 & 2\,567 \\
       2\,567 &	   901 &  2\,567 &    901 &  2 &    833 \\
	  901 &	   833 &     901 &    833 &  4 &     17 \\
	  833 &	    17 &     833 &     17 & 16 &     51 \\
	   51 &	    17 &      51 &     17 &  2 &     17 \\
	   17 &	    17 &      17 &     17
		  & \multicolumn{2}{l|}{\(0\)} \\
      \hline
    \end{tabular}
    \caption{Traza del algoritmo binario para máximo común divisor}
    \label{tab:traza-gcd-binario}
  \end{table}
  La traza respectiva
  se reseña en el cuadro~\ref{tab:traza-gcd-binario}.
  El resultado final
  es \(\gcd(40\,902, 24\,140) = 2 \cdot 17 = 34\).

  El análisis completo del algoritmo parece ser intratable,
  Knuth~\cite{knuth97:_semin_algor}
  analiza en detalle un modelo aproximado
  y da algunos resultados exactos.

\section{Potencias}
\label{sec:potencias}
\index{algoritmo!potencia}

  Otra operación importante
  es calcular potencias.
  Un algoritmo eficiente para calcular potencias
  es el~\ref{alg:power}.%
    \index{algoritmo!potencia!binario}
  \begin{algorithm}[htbp]
    \DontPrintSemicolon
    \SetKwFunction{Pow}{pow}

    \KwFunction \Pow{\(a,\; n\)} \;
    \BlankLine
    \(r \leftarrow 1\) \;
    \While{\(n \ne 0\)}{
      \If{\(2 \centernot\mid n\)}{
	\(r \leftarrow r \cdot a\) \;
      }
      \(a \leftarrow a^2\) \;
      \(n \leftarrow \lfloor n / 2 \rfloor\) \;
    }
    \Return \(r\) \;
    \caption{Cálculo binario de potencias}
    \label{alg:power}
  \end{algorithm}
  Es fácil ver que con este algoritmo el cálculo de \(a^n\)
  toma \(O(\log n)\) multiplicaciones.
  En todo caso,
  el algoritmo~\ref{alg:power} no es lo mejor que se puede hacer,
  un tratamiento detallado de este espinudo tema
  da Knuth~\cite{knuth97:_semin_algor}.

  \begin{table}[htbp]
    \centering
    \begin{tabular}{|>{\(}r<{\)}|>{\(}r<{\)}|>{\(}r<{\)}|}
      \hline
      \multicolumn{1}{|c|}{\rule[-0.7ex]{0pt}{3ex}\(\boldsymbol{n}\)} &
	\multicolumn{1}{c|}{\(\boldsymbol{a}\)} &
	\multicolumn{1}{c|}{\(\boldsymbol{r}\)} \\
      \hline\rule[-0.7ex]{0pt}{3ex}%
	10 &	  3 &	 1 \\
	 5 &	  9 &	 9 \\
	 9 &	 81 &	 9 \\
	 1 & 6\,561 & 59\,049 \\
      \hline
    \end{tabular}
    \caption{Cálculo de $3^{10}$ por el método binario}
    \label{tab:3^10}
  \end{table}
  Un ejemplo del método binario
  da el cálculo de \(3^{10}\) en el cuadro~\ref{tab:3^10}.

\section{Factorizar}
\label{sec:factorizar}
\index{algoritmo de factorizacion@algoritmo de factorización}

  Una técnica básica para factorizar
  es la que hemos aprendido en el colegio:
  Para factorizar \(N\),
  intentamos los primos entre \(2\) y \(\lfloor \sqrt{N} \rfloor\).
  Si ninguno divide a \(N\),
  entonces \(N\) es primo.
  Si \(N\) es grande,
  esto definitivamente no es viable,
  pero sirve muy bien
  para eliminar factores primos chicos de algún número.

  Una alternativa que sirve bien cuando \(N\) tiene factores grandes
  es debida esencialmente a Fermat.%
    \index{algoritmo de factorizacion@algoritmo de factorización!Fermat}
  Supongamos \(N = U V\),
  donde podemos suponer que \(N\) es impar
  (y por tanto lo son \(U\) y \(V\)).
  Entonces,
  definiendo \(X\) e \(Y\) como sigue:
  \begin{align*}
    X &= (U + V) / 2 \\
    Y &= (U - V) / 2 \\
    N &= X^2 - Y^2
  \end{align*}
  La idea entonces es buscar sistemáticamente
  valores de \(X\) e \(Y\) según lo anterior.
  Aprovechando que la suma de los primeros \(n\) números impares
  cumple:
  \begin{equation*}
    \sum_{0 \le k \le n} (2 k + 1) = n^2
  \end{equation*}
  obtenemos el algoritmo~\ref{alg:factorizar-Fermat},
  \begin{algorithm}[htbp]
    \DontPrintSemicolon
    \SetKwFunction{Factor}{factor}

    \KwFunction \Factor{\(N\)} \;
    \BlankLine
    \(x \leftarrow 2 \left\lfloor \sqrt{N} \right\rfloor + 1\) \;
    \(y \leftarrow 1\) \;
    \(r \leftarrow \left\lfloor \sqrt{N} \right\rfloor^2 - N\) \;
    \While{\(r \ne 0\)}{
      \(r \leftarrow r + x\) \;
      \(x \leftarrow x + 2\) \;
      \While{\(r > 0\)}{
	\(r \leftarrow r -y\) \;
	\(y \leftarrow y + 2\) \;
      }
    }
    \Return \(N = \left((x - y)/2\right) \cdot \left((x + y - 2)/2\right)\) \;
    \caption{Factorizar según Fermat}
    \label{alg:factorizar-Fermat}
  \end{algorithm}
  donde usamos las variables \(x\), \(y\) y \(r\)
  para designar lo que en la exposición anterior
  llamamos \(2X + 1\), \(2Y + 1\) y \(X^2 - Y^2 -N\),
  respectivamente.
  Durante la ejecución tenemos \(\lvert r \rvert < x\) e \(y < x\).
  Lo más curioso
  es que no usa multiplicación ni división para factorizar.
  El lector podrá entretenerse aplicando a mano
  este algoritmo a \(377\).

  El método de Fermat en realidad era diferente,
  el algoritmo~\ref{alg:factorizar-Fermat}
  es muy eficiente en computadores
  pero no es muy adecuado para cálculo manual.
% Fixme: Add reference to TAoCP, where Knuth discusses this method/example
  Fermat no mantenía el valor de \(y\),
  miraba \(x^2 - N\)
  y descartaba no cuadrados viendo sus últimos dígitos
  (en base 10,
   deben ser \(00\), \(p1\), \(p4\), 25, \(i6\) o \(p9\),
   donde \(p\) es un dígito par e \(i\) uno impar);
  en caso de sospechar que fuera un cuadrado perfecto
  extraía una raíz.
  Esta misma idea puede extenderse a otras bases,
  como muestra Knuth~\cite{knuth97:_semin_algor}.
  Tomemos por ejemplo \(N = 8\,616\,460\,799\),
  y consideremos el cuadro~\ref{tab:condiciones-factores}.
  \begin{table}
    \centering
    \begin{tabular}{|>{\(}r<{\)}|*{3}{>{\(}l<{\)}}|}
      \hline
      \multicolumn{1}{|c|}{\rule[-0.7ex]{0pt}{3ex}\(\boldsymbol{m}\)} &
	\multicolumn{1}{c}{\textbf{Si $\boldsymbol{x \bmod m}$ es}} &
	\multicolumn{1}{c}{\textbf{$\boldsymbol{x^2 \bmod m}$ es}} &
	\multicolumn{1}{c|}{\textbf{$\boldsymbol{(x^2 - N) \bmod m}$ es}} \\
      \hline\rule[-0.7ex]{0pt}{3ex}%
       3 & 0, 1, 2
	 & 0, 1, 1
	 & \phantom{0}1, 2, 2 \\
       \rule[-0.7ex]{0pt}{3ex}%
       5 & 0, 1, 2, 3, 4
	 & 0, 1, 4, 4, 1
	 & \phantom{0}1, 2, 0, 0, 2 \\
       \rule[-0.7ex]{0pt}{3ex}%
       7 & 0, 1, 2, 3, 4, 5, 6
	 & 0, 1, 4, 2, 2, 4, 1
	 & \phantom{0}5, 6, 2, 0, 0, 2, 6 \\
       \rule[-0.7ex]{0pt}{3ex}%
       8 & 0, 1, 2, 3, 4, 5, 6, 7
	 & 0, 1, 4, 1, 0, 1, 4, 1
	 & \phantom{0}1, 2, 5, 2, 1, 5, 2 \\
       \rule[-0.7ex]{0pt}{3ex}%
      \multirow{2}*{11}
	 &  0,	1,  2,	3,  4,	5,  6,	7,
	 &  0,	1,  4,	9,  5,	3,  3,	5,
	 & 10,	0,  3,	8,  4,	2,  2,	4,\\
	 &  8,	9, 10
	 &  9,	4,  1
	 &  \phantom{0}8,  3,  0 \\
     \hline
    \end{tabular}
    \caption{Condiciones a $x$ e $y$
	     al factorizar $8\,616\,460\,799$}
    \label{tab:condiciones-factores}
  \end{table}
  Si \(x^2 - N\) es un cuadrado perfecto \(y^2\),
  entonces debe tener un residuo módulo \(m\) consistente con esto,
  para todo \(m\).
  Por ejemplo,
  con \(N = 8\,616\,460\,799\) y \(x \bmod 3 \ne 0\),
  entonces \((x^2 - N) \bmod 3 = 2\),
  y esto no puede ser un cuadrado perfecto,
  de forma que \(x\) debe ser un múltiplo de \(3\)
  para que \(N = x^2 - y^2\).
  Nuestro cuadro dice que:
  \begin{align*}
    x \bmod \phantom{0}3
      &= 0 \\
    x \bmod \phantom{0}5
      &= 0, 2 \text{\ ó\ } 3 \\
    x \bmod \phantom{0}7
      &= 2, 3, 4 \text{\ ó\ } 5 \\
    x \bmod \phantom{0}8
      &= 0 \text{\ ó\ } 4 \text{\ (o sea, \(x \bmod 4 = 0\))} \\
    x \bmod 11
      &= 0, 2, 4, 7, 9 \text{\ ó\ } 10
  \end{align*}
  Esto reduce la búsqueda en forma considerable.
  Por ejemplo,
  vemos que \(x\) debe ser múltiplo de \(12\).
  Debe ser \(x \ge \lceil \sqrt{N} \rceil = 92\,825\),
  y el menor múltiplo de \(12\) que cumple es \(92\,832\).
  Pero este valor
  tiene residuos \((2, 5, 3)\) módulos \((5, 7, 11)\),
  y falla nuestra condición respecto del módulo \(11\).
  Incrementar \(x\) en \(12\) aumenta los residuos
  módulo \(5\) en \(2\),
  módulo \(7\) en \(5\)
  y módulo \(11\) en \(1\).
  El primer \(x\) que cumple todas las condiciones es \(92\,880\),
  y \(92\,880^2 - N = 10\,233\,601\),
  que resulta ser el cuadrado de \(3\,199\).
  Hemos encontrado la solución \(x = 92\,880\) e \(y = 3\,199\),
  que entrega la factorización:
  \begin{equation*}
    8\,616\,460\,799 = (x - y) (x + y) = 89\,681 \cdot 96\,079
  \end{equation*}
  La importancia del número de marras
  es que el economista y lógico W.~S.~Jevons%
    \index{Jevons, W. S.}
  lo mencionó en un conocido libro en~1874~%
    \cite{jevons74:_princ_scien},
  diciendo que a pesar que es muy fácil multiplicar dos números,
  probablemente nunca nadie salvo él mismo conocería sus factores.
  Sin embargo,
  acabamos de demostrar
  que Fermat%
    \index{Fermat, Pierre de}
  podría haberlo factorizado en unos minutos.
  El punto central de que factorizar es difícil es correcto,
  siempre que los factores no sean tan cercanos.

  Un algoritmo curioso es rho de Pollard~%
    \cite{pollard75:_rho_factorization},%
    \index{algoritmo de factorizacion@algoritmo de factorización!rho de Pollard}
  que tiende a ser útil para factores más bien pequeños.
  La idea básica viene
  de lo que se conoce como \emph{la paradoja del cumpleaños}%
    \index{paradoja del cumpleanos@paradoja del cumpleaños}
  (\emph{\foreignlanguage{english}{birthday paradox}} en inglés):%
    \index{birthday paradox@\emph{\foreignlanguage{english}{birthday paradox}}|see{paradoja del cumpleaños}}
  En el año hay \(365\) días,
  si tomamos una persona la probabilidad
  que el cumpleaños de una segunda no coincida con la primera
  es \(1 - 1 / 365\),
  para que el de una tercera
  no coincida con ninguno de los dos anteriores
  es \(1 - 2 / 365\),
  y así sucesivamente.
  La probabilidad que en un grupo de \(n\) personas
  no hayan cumpleaños repetidos es:
  \begin{equation*}
    P(n)
      = \prod_{1 \le k \le n - 1} \, \left( 1 - \frac{k}{365} \right)
  \end{equation*}
  Resulta que para \(n = 24\)
  la probabilidad de que hayan dos (o más)
  personas con el mismo cumpleaños ya es mayor a \(1 / 2\),
  cuando intuitivamente uno pensaría que se requieren muchas más.
  Una manera alternativa de analizar aproximadamente el problema%
    \index{paradoja del cumpleanos@paradoja del cumpleaños!analisis aproximado@análisis aproximado}
  es considerar que hay
  \(n (n + 1) / 2\) pares de personas,
  basta que uno de los pares coincida,
  con lo que debiera ser suficiente
  que \(n (n + 1) / 2 \approx 365\)
  para que se produzca una coincidencia.
  Esto se traduce en \(n \approx \sqrt{2 \cdot 365} = 27\).
  Acá lo que se busca es generar rápidamente
  una gran colección de pares
  módulo \(N\)
  y buscar coincidencias módulo un primo que divide a \(N\).
  Un razonamiento como el anterior
  lleva a pensar que si \(p_1\) es el menor primo que divide a \(N\)
  con \(O(p_1^{\sfrac{1}{2}})\) pares hallaremos una coincidencia,
  que lleva a determinar \(p_1\).

  El nombre \(\rho\)
  viene de considerar una secuencia eventualmente periódica
  en la cual hay \(\mu\) elementos antes del primero que se repite
  (la cola de la letra \(\rho\)),
  y luego un ciclo de largo \(\lambda\)
  (la cabeza de \(\rho\)).
  Vale decir,
  sea \(f \colon \mathbb{N} \rightarrow \{0, 1, \dotsc, m - 1\}\)
  una función,
  y consideremos la secuencia definida por \(x_{i + 1} = f(x_i)\).
  Entonces hay \(\mu\) y \(\lambda\)
  tales que
    \(x_0, x_1, \dotsc, x_\mu, \dotsc, x_{\mu + \lambda - 1}\)
  son todos diferentes,
  pero \(x_\mu = x_{\mu + \lambda}\).
  Estas relaciones definen \(\mu\) y \(\lambda\).
  Tenemos \(0 \le \mu < m\),
  \(0 < \lambda \le m\) y \(\mu + \lambda \le m\).
  Así,
  \(x_j = x_k\) con \(j > k\)
  si y solo si \(j - k\) es múltiplo de \(\lambda\)
  y \(k \ge \mu\);
  con esto \(x_{2 k} = x_k\)
  si y solo si \(k\) es múltiplo de \(\lambda\)
  y \(k \ge \mu\).
  Vale decir,
  hay un \(k > 0\) con \(\mu \le k \le \mu + \lambda\)
  tal que \(x_k = x_{2 k}\),
  lo que lleva al algoritmo de Floyd~%
    \cite{floyd67:_non_deter_algor}%
    \index{Floyd, algoritmo de (deteccion de ciclos)@Floyd, algoritmo de (detección de ciclos)}
  (ver~\ref{alg:ciclo-Floyd})
  para detectar ciclos.
  \begin{algorithm}[htbp]
    \(x \leftarrow x_0\) \;
    \(y \leftarrow x_0\) \;
    \Repeat{\(x = y\)}{
      \(x \leftarrow f(x)\) \;
      \(y \leftarrow f(f(y))\) \;
    }
    \caption{Detectar ciclos (Floyd)}
    \label{alg:ciclo-Floyd}
  \end{algorithm}

  Sea ahora \(f(x)\) un polinomio con coeficientes enteros,
  \(p\) un factor primo de \(N\),
  y consideremos las secuencias
  definidas con un inicio \(A\) arbitrario:
  \begin{align*}
    x_0	      &= y_0 = A	\\
    x_{m + 1} &= f(x_m) \bmod N \\
    y_{m + 1} &= f(y_m) \bmod p
  \end{align*}
  Por el teorema chino de los residuos%
    \index{residuo!teorema chino de los}
  \(y_m \equiv x_m \pmod{p}\).
  La secuencia \(y_m\) debe repetirse con un período a lo más \(p\),
  digamos \(y_{k + \lambda} = y_k\).
  Entonces \(x_{k + \lambda} \equiv x_k \pmod{p}\),
  y \(\gcd(N, x_{k + \lambda} - x_k)\) da un factor de \(N\).
  Funciones de la forma \(f(x) = (\alpha x + \beta) \bmod N\)
  no sirven,
  ya que con ellas \(x_{k + \lambda} \equiv x_k \pmod{p}\)
  exactamente cuando \(x_{k + \lambda} \equiv x_k \pmod{N}\).
  Se usa lo siguiente más simple,
  \(f(x) = (x^2 + 1) \bmod N\);
  y si esto no tiene éxito,
  se intenta \(f(x) = (x^2 + c) \bmod N\)
  con \(c \ne 0\) y \(c \ne -2\)
  (porque estos caen en un ciclo de unos
   al toparse con \(x \equiv \pm 1 \pmod{N}\)).
  Todo esto lleva al algoritmo~\ref{alg:rho-Pollard}.
  \begin{algorithm}[htbp]
    \DontPrintSemicolon
    \SetKwFunction{Factor}{factor}

    \KwFunction \Factor{\(N\)} \;
    \BlankLine
    Elegir \(A\) al azar \;
    \(x \leftarrow A\) \;
    \(y \leftarrow A\) \;
    \Repeat{\(d \ne 1\)}{
      \(x \leftarrow f(x)\) \;
      \(y \leftarrow f(f(y))\) \;
      \(d \leftarrow \gcd(\lvert x - y \rvert, N)\) \;
    }
    \eIf{\(d = N\)}{
      \Return Falló
    }{
      \Return \(d \mid N\)
    }
    \caption{$\rho$ de Pollard}
    \label{alg:rho-Pollard}
  \end{algorithm}
  Brent~\cite{brent80:_improved_rho}
  da una variante usando un algoritmo de detección de ciclos
  más rápido.

  El método descrito,
  con \(f(x) = x^2 + 1\) y \(x_0 = 42\),
  factoriza \(16\,843\,009 = 257 \cdot 65\,537\)
  como muestra el cuadro~\ref{tab:ejemplo-rho}.
  \begin{table}[htbp]
    \centering
    \begin{tabular}{|*{4}{>{\(}r<{\)}|}}
      \hline
      \multicolumn{1}{|c|}{\rule[-0.7ex]{0pt}{3ex}\(\boldsymbol{i}\)} &
	\multicolumn{1}{c|}{\(\boldsymbol{x_i}\)} &
	\multicolumn{1}{c|}{\(\boldsymbol{x_{2 i}}\)} &
	\multicolumn{1}{c|}{\(\boldsymbol{\gcd}\)} \\
      \hline\rule[-0.7ex]{0pt}{3ex}%
	 0 &	       42 &		 &     \\
	 1 &	   1\,765 &  3\,115\,226 &   1 \\
	 2 &  3\,115\,226 &  4\,805\,758 &   1 \\
	 3 & 11\,262\,448 &  4\,817\,235 &   1 \\
	 4 &  4\,805\,758 &  9\,598\,062 &   1 \\
	 5 &  7\,583\,675 & 11\,476\,471 &   1 \\
	 6 &  4\,817\,235 &  2\,534\,841 &   1 \\
	 7 & 11\,064\,323 &	443\,204 &   1 \\
	 8 &  9\,598\,062 &  6\,015\,649 &   1 \\
	 9 &  6\,959\,372 &  6\,454\,177 &   1 \\
	10 & 11\,476\,471 & 15\,725\,109 &   1 \\
	11 &  3\,760\,417 &  9\,439\,232 &   1 \\
	12 &  2\,534\,841 &	574\,959 & 247 \\
      \hline
    \end{tabular}
    \caption{Ejemplo de Pollard $\rho$}
    \label{tab:ejemplo-rho}
  \end{table}

  Otra técnica es el método \(p - 1\) de Pollard~%
    \cite{pollard74:_thms_factor_prime_testing}.%
    \index{algoritmo de factorizacion@algoritmo de factorización!\(p - 1\) de Pollard}
  Recordemos el teorema de Fermat%
    \index{Fermat, pequeno teorema de@Fermat, pequeño teorema de}
  para \(p\) primo y \(p \centernot\mid a\):
  \begin{equation*}
    a^{p - 1} \equiv 1 \pmod{p}
  \end{equation*}
  Podemos elevar esta congruencia a una potencia cualquiera \(k\):
  \begin{equation*}
    a^{k (p - 1)} \equiv 1 \pmod{p}
  \end{equation*}
  Vale decir,
  \(p \mid a^{k (p - 1)} - 1\).
  Consideremos ahora el entero \(N\) a factorizar
  y \(p\) un factor primo (desconocido) de \(N\)
  y un valor \(M\) a determinar.
  Si \(p - 1 \mid M\)
  tenemos:
  \begin{equation*}
    p \mid (a^M \bmod N - 1)
  \end{equation*}
  y \(\gcd(a^M \bmod N - 1, N)\) dará un factor no trivial de \(N\).
  La idea es elegir \(a\) pequeño
  y hacer \(M\) el producto de muchos primos
  (ojalá más bien chicos).
  Si los factores de \(p - 1\) están en este conjunto,
  tendremos éxito.

  Otra colección de métodos,
  más apropiados para computación distribuida y números grandes,
  se deben a una idea de Maurice Kraitchik en los 1920s.%
    \index{Kraitchik, Maurice}
  Supongamos que podemos encontrar la congruencia:
  \begin{equation*}
    a^2 \equiv b^2 \pmod{N}
       \quad a \centernot\equiv \pm b \pmod{N}
  \end{equation*}
  Entonces \(N \mid (a^2 - b^2)\),
  pero \(a^2 - b^2 = (a + b) (a - b)\),
  y ninguno de estos dos factores es divisible por \(N\),
  por lo que si \(N = p q\),
  entonces \(p\) divide a uno de los factores y \(q\) al otro,
  y \(\gcd(a + b, N)\) y \(\gcd(a - b, N)\)
  son factores no triviales de \(N\).
  La manera de obtener esta ecuación
  es construir una \emph{gran} colección de relaciones de la forma
  \(a^2 \equiv q \pmod{N}\) en las cuales \(q\) es pequeño,
  e intentar factorizar tales \(q\),
  en particular en términos de primos chicos.
  Para la mayoría de los \(q\) esto no funcionará,
  pero bastan unos pocos que se puedan factorizar completamente.
  Si consideramos la expresión \(a^2 \equiv q \pmod{N}\),
  el lado izquierdo ya es un cuadrado,
  y lo hemos factorizado.
  Buscamos otros \(q\) factorizados
  que completen las potencias
  de los primos factores de \(q\) a pares,
  o sea,
  tenemos:
  \begin{align*}
    a_1^2
      &\equiv q_1 \phantom{q_2 \dotsm q_n} \pmod{N} \\
    a_2^2
      &\equiv q_2 \phantom{q_2 \dotsm q_n} \pmod{N} \\
      &\vdots \\
    a_n^2
      &\equiv q_n \phantom{q_2 \dotsm q_n} \pmod{N} \\
    (a_1 a_2 \dotso a_n)^2
      &\equiv q_1 q_2 \dotsm q_n \pmod{N}
  \end{align*}
  donde conocemos \(b^2 = q_1 q_2 \dotsm q_n\),
  y esto entrega una factorización de \(N\) por lo anterior,
  claro que la factorización puede ser trivial.
  La búsqueda de los \(q\)
  y su factorización
  en términos de un conjunto de primos predefinidos
  puede distribuirse;
  para factorizar luego planteamos
  un sistema de ecuaciones lineales módulo \(2\)
  buscando una combinación de \(q\) que sea un cuadrado.
  Hay varias variantes de esta idea general,
  que difieren en la estrategia usada
  para buscar cuadrados pequeños
  (y ojalá fácilmente factorizables)
  para combinar.
  No entraremos en ese detalle acá,
  Pommerance~\cite{pommerance96:_tale_two_sieves}%
    \index{algoritmo de factorizacion@algoritmo de factorización!criba}
  describe la historia con múltiples referencias.

\section{Factorización con curvas elípticas}
\label{sec:EC-factorizacion}
\index{algoritmo de factorizacion@algoritmo de factorización!curva eliptica@curva elíptica}

  Un método reciente es la factorización por curvas elípticas%
    \index{curva eliptica@curva elíptica}
  de Lenstra~\cite{lenstra87:_factor_integ_ellip_curves},
  basado en grupos de curvas elípticas
  (ver la sección~\ref{sec:curvas-elipticas}).
  Es un método cuyo tiempo de ejecución depende del factor primo más chico,
  por lo que se usa para eliminar factores pequeños
  para luego ir a un método general con los factores remanentes.
  Primero,
  si se cumple:
  \begin{equation}
    \label{eq:curva-eliptica-factorizacion}
    y^2
      \equiv x^3 + a x + b \pmod{n}
  \end{equation}
  por el (padre del) teorema chino de los residuos
  (corolario~\ref{cor:isomorfismo-anillo-Zm})
  también se cumple para los factores primos de \(n\).
  En el fondo,
  estamos efectuando
  cálculos simultáneos
  en los grupos elípticos para los factores primos de \(n\),
  y en alguno de ellos atinaremos al orden de \(P\)
  y obtenemos un factor de \(n\).
  Por el teorema de Hasse~%
    \cite{hasse36:_EC-I,hasse36:_EC-II,hasse36:_EC-III}
  el orden del grupo sobre \(\mathbb{Z}_p\) está entre
  \(p + 1 - 2 \sqrt{p}\) y \(p + 1 + 2 \sqrt{p}\),
  no depende directamente de \(p\).
  En este sentido,
  este método es un refinamiento de método \(p - 1\),
  que busca detectar
  el orden de un elemento en \(\mathbb{Z}_p^\times\)
  (pero allí el orden del grupo es \(p - 1\),
   y el método solo funciona si \(p - 1\) tiene factores chicos).

  Calcularemos múltiplos \(k \, \mathtt{P}\)
  para diversos valores de \(k\)
  para un punto \(\mathtt{P}\) de la curva
  usando la suma del grupo de la curva elíptica,
  ecuaciones~\eqref{eq:suma-curva-eliptica}
  y~\eqref{eq:doble-curva-eliptica}.
  Para valores grandes de \(k\)
  se puede usar un algoritmo afín al para calcular potencias,
  algoritmo~\ref{alg:power},%
    \index{algoritmo!potencia}
  pero como en estos grupos calcular restas es tan rápido como sumar
  se pueden usar variantes que las incluyen
  (por ejemplo,
   calcular \(15 \mathtt{P} = 2( 2 (2 (2 \mathtt{P}))) - \mathtt{P}\)
   son \(5\)~sumas/restas,
   calcular
     \(15 \mathtt{P}
	 = \mathtt{P} + 2 \mathtt{P}
	     + 2 (2 \mathtt{P}) + 2 (2 (2 \mathtt{P}))\)
   considera \(6\)).

  Las fórmulas de suma en el grupo
  involucran la ``pendiente'' \(s\),
  que requiere un inverso multiplicativo módulo \(n\).
  Si \(s = u / v\),
  con \(v \equiv 0 \pmod{n}\),
  el punto resultante es el punto en el infinito,
  el elemento neutro del grupo.%
    \index{operacion@operación!elemento neutro}
  Si es \(\gcd(v, n) \ne 1\) y \(\gcd(v, n) \ne n\),
  no se obtiene un punto válido en la curva,
  pero sí un factor no trivial de \(n\).

  El algoritmo contempla los siguientes pasos:
  \begin{enumerate}
  \item
    Elija una curva elíptica sobre \(\mathbb{Z}_n\),
    de la forma~\ref{eq:curva-eliptica-factorizacion}
    y un punto al azar \(\mathtt{P}\) sobre ella.
    Una posibilidad es elegir \(\mathtt{P} = (x, y)\)
    con coordenadas al azar diferentes de cero módulo \(n\),
    luego tomar un valor \(a \centernot\equiv 0 \pmod{n}\)
    y calcular:
    \begin{equation*}
      b = (y^2 - x^3 - a x) \bmod n
    \end{equation*}
  \item
    Calcule \(m \mathtt{P}\) para \(m\)
    un producto de muchos factores chicos,
    por ejemplo el producto de los primeros primos
    elevados a potencias chicas
    o \(B!\) para un \(B\) pequeño.
    Si en el proceso halla un factor de \(n\),
    deténgase.
    Si no halla factores o llega al punto en el infinito,
    intente con otra curva.
  \end{enumerate}

  Hay maneras de acelerar los cálculos usando curvas especiales
  o mediante descripciones alternativas de los puntos o las curvas,
  ver discusiones de implementación
  de Bernstein, Birkner, Lange y Peters~%
    \cite{cryptoeprint:2008:016}.
  Métodos relacionados usan grupos de curvas hiperelípticas,
  basadas en curvas de la forma \(y^2 = f(x)\)
  para polinomios \(f(x)\) de grado mayor a 4,
  ver Cosset~\cite{cosset10:_factor_genus_2_curves}.

\section{Determinar primalidad}
\label{sec:primalidad}
\index{algoritmo!primalidad}

  Para determinar si un número es primo,
  el teorema de Fermat%
    \index{Fermat, pequeno teorema de@Fermat, pequeño teorema de}
  es una herramienta poderosa.
  Por ejemplo,
  para \(2^{32} + 1 = 4\,294\,967\,297\)
  mediante \(32\) elevaciones al cuadrado módulo \(2^{32} + 1\)%
    \index{algoritmo!potencia}
  obtenemos que:
  \begin{equation*}
    3^{2^{32}} \equiv 3\,029\,026\,160 \pmod{2^{32} + 1}
  \end{equation*}
  lo que dice que \(2^{32} + 1\) no es primo.
  Claro que no da ninguna luz sobre sus factores.
  En general,
  para \(N\) compuesto es posible hallar \(a\)
  con \(0 < a < N\)
  tal que \(a^{N - 1} \centernot\equiv 1 \pmod{N}\),
  y la experiencia muestra que tales \(a\) se hallan rápidamente.
  Hay casos raros en los cuales frecuentemente se da
  \(a^{N - 1} \equiv 1 \pmod{N}\),
  pero entonces \(N\) tiene un factor menor que \(\sqrt[3]{N}\),
  como veremos más adelante.
  Para efectos prácticos basta considerar \(3^{N - 1} \bmod N\).

  La forma clásica de demostrar que \(N\) es primo
  para	\(N\) grande
  es hallar una raíz primitiva \(r\) de \(N\).%
    \index{raiz primitiva@raíz primitiva}
  Por suerte,
  las raíces primitivas de números primos son bastante numerosas.
  De la discusión de grupos cíclicos de orden \(m\)
  sabemos que las potencias relativamente primas a \(m\)
  del generador del grupo
  también son generadores,
  con lo que hay \(\phi(p - 1)\) raíces primitivas del primo \(p\).
  Recientemente,
  Agrawal, Kayal y~Saxena~\cite{agrawal04:_primes_in_P}%
    \index{AKS, algoritmo}
  describieron un algoritmo polinomial en el número de bits de \(n\)
  para determinar si es primo.
  La existencia de tal algoritmo se sospechaba hacía tiempo,
  pero el algoritmo en sí resultó sorprendente,
  su demostración requiere solo álgebra relativamente sencilla.

  Consideremos \(p\) primo,
  con lo que hay una raíz primitiva módulo \(p\),
  llamémosle \(r\).
  Tomemos \(k\) tal que \(0 \le k < p\),
  sabemos que si \(\ord_p(r^k) = n\)
  entonces \(k n\) es el mínimo común múltiplo de \(k\) y \(p - 1\),
  y será \(n = p - 1\) exactamente cuando \(\gcd(k, p - 1) = 1\).
  Si \(x\) es raíz primitiva módulo \(N\),
  para todo \(d\) que divide a \(N - 1\) debe ser:
  \begin{equation*}
    x^{(N - 1) / d} \centernot\equiv 1 \pmod{N}
  \end{equation*}
  porque esto asegura que \(\ord_N(x) = N - 1\).
  En todo caso,
  basta encontrar un \(x\)
  para cada primo \(p\) que divide a \(N - 1\),
  el producto de todos ellos será una raíz primitiva.
  Los cálculos involucrados
  (salvo posiblemente la factorización de \(N - 1\))
  son simples de efectuar con los algoritmos discutidos antes.
  Esto lo discutiremos en conexión con el algoritmo Diffie-Hellman%
    \index{Diffie-Hellman, algoritmo de}
  en la sección~\ref{sec:Diffie-Hellman}.
  El cuello de botella es factorizar \(N - 1\).

  En la práctica,
  se usan métodos que no \emph{garantizan} que el número es primo,
  pero que tienen alta probabilidad de detectar no-primos.
  El más usado actualmente es el test de Miller-Rabin~%
    \cite{miller76:_Riemann_hypot_tests_primality,
	  rabin80:_probab_algor_test_primality}.%
    \index{Miller-Rabin, prueba de}
  Monier~\cite{monier80:_evaluat_compar_two_effic_probab}
  compara en detalle dos algoritmos similares,
  y concluye que el de Miller-Rabin
  es más eficiente en todos los casos.

  \begin{algorithm}[htbp]
    \DontPrintSemicolon
    \SetKwFunction{IsPrime}{is\_prime}

    \KwFunction \IsPrime{\(N\)} \;
    \BlankLine
    \(s \leftarrow 0\) \;
    \(d \leftarrow N - 1\) \;
    \While{\(2 \mid d\)}{
      \(s \leftarrow s + 1\) \;
      \(d \leftarrow d / 2\) \;
    }
    \For{\(i \leftarrow 1\) \KwTo \(k\)}{
      Elija \(a\) al azar en el rango \([2, N - 2]\) \;
      \(x \leftarrow a^d \bmod N\) \;
      \If{\(x = 1\) o \(x = N - 1\)}{
	\KwContinue \;
      }
      \For{\(r \leftarrow 1\) \KwTo \(s - 1\)}{
	\(x \leftarrow x^2 \bmod N\) \;
	\uIf{\(x = 1\)}{
	  \Return Compuesto \;
	}
	\ElseIf{\(x = N - 1\)}{
	  \KwBreak \;
	}
      }
      \If{\(x \ne N - 1\)}{
	\Return Compuesto \;
      }
    }
    \Return Probablemente primo
    \caption{Prueba de primalidad de Miller-Rabin}
    \label{alg:Miller-Rabin}
  \end{algorithm}
  El test de Miller-Rabin
  se basa en la observación que módulo un primo \(p\)
  solo \(1\) y \(-1\) pueden ser raíces cuadradas de \(1\)
  (el polinomio \(x^2 - 1\) puede tener a lo más dos ceros
   en el campo \(\mathbb{Z}_p\),
   mientras por el teorema chino de los residuos
   en \(\mathbb{Z}_n\) con \(n\) compuesto
   hay un par diferente por cada factor primo de \(n\)).
  Si \(p\) es un primo impar,
  podemos escribir \(p - 1 = 2^s d\),
  con \(d\) impar.
  Con esta notación,
  del teorema de Fermat para \(a \centernot\equiv 0 \pmod{p}\)
  tenemos que \(a^{2^s d} \equiv 1 \pmod{p}\).
  Por la observación anterior
  sobre raíces cuadradas en \(\mathbb{Z}_p\),
  sacando raíz cuadrada sucesivamente partiendo de \(a^{p - 1} = 1\)
  debemos llegar a que \(a^d = \pm 1\)
  o que alguno de los \(a^{{2^r}d} = -1\) para \(1 \le r < s\).
  El test de Miller-Rabin se basa en el contrapositivo de esto.
  Puede demostrarse que a lo más \(1 / 4\) de los valores \(a\)
  para un número compuesto ``mienten'',
  con lo que repitiendo el proceso suficientes veces
  podemos tener gran confianza de que el número realmente es primo.
  El algoritmo~\ref{alg:Miller-Rabin} repite la prueba \(k\) veces.

\section{Números de Carmichael}
\label{sec:Carmichael}
\index{numero@número!Carmichael|see{Carmichael, número de}}
\index{Carmichael, numero de@Carmichael, número de}

  Queda la inquietud planteada antes
  sobre números para los cuales ``falla''
  el teorema de Fermat,%
    \index{Fermat, pequeno teorema de@Fermat, pequeño teorema de}
  en el sentido que \(a^{n - 1} \equiv 1 \pmod{n}\)
  se cumple con \(\gcd(a, n) = 1\),
  pero \(n\) no es primo.
  A tal número se le llama \emph{pseudoprimo}%
    \index{numero@número!pseudoprimo|textbfhy}%
    \index{pseudoprimo|see{número pseudoprimo}}
  (de Fermat con base \(a\)).
  El caso extremo lo dan los números de Carmichael~%
    \cite{carmichael10:_note_new_number_theor_funct},
  pseudoprimos de Fermat
  para todos los \(a\) relativamente primos a ellos.

  \begin{theorem}
    Todo número de Carmichael \(n\) es libre de cuadrados,
    y para \(p\) primo,
    si \(p \mid n\) entonces \(p - 1 \mid n - 1\).
  \end{theorem}
  \begin{proof}
    Consideremos un número de Carmichael
    \(n = p_1^{k_1} p_2^{k_2} \dotsm p_r^{k_r}\)
    donde los \(p_i\) son primos distintos,
    y un \(a\) relativamente primo a \(n\).
    Del padre del teorema chino de los residuos%
      \index{residuo!teorema chino de los!padre del}
    (corolario~\ref{cor:isomorfismo-anillo-Zm})
    sabemos que:
    \begin{equation*}
      \mathbb{Z}_n
	\cong \mathbb{Z}_{p_1^{k_1}}
		\times \mathbb{Z}_{p_2^{k_2}}
		\times \dotsb
		\times \mathbb{Z}_{p_r^{k_r}}
    \end{equation*}
    con lo que \(a^{n - 1} \equiv 1 \pmod{n}\)
    si y solo si \(a^{n - 1} \equiv 1 \pmod{p_i^{k_i}}\)
    para todos los \(i\).
    Esto a su vez para el primo \(p\)
    solo puede ser si \(\ord_{p^k}(a) \mid n - 1\).
    Sabemos por el teorema~\ref{theo:raices-primitivas}%
      \index{raiz primitiva@raíz primitiva}
    que si \(p\) es un primo impar,
    hay raíces primitivas módulo \(p^k\) para todo \(k\),
    vale decir,
    hay elementos de orden \(\phi(p) = p^{k - 1} (p - 1)\).
    (En realidad,
     basta con el teorema~\ref{theo:raiz-primitiva-p2},
     ya que si un elemento es de orden \(p (p - 1)\) módulo \(p^2\),
     tendrá que ser al menos de ese orden módulo \(p^k\);
     en particular,
     \(p\) divide a su orden.)
    Pero si \(p \mid n\),
    entonces \(p\) no puede dividir a \(n - 1\),
    y \(n\) no puede tener factores primos repetidos.
    Ahora,
    si \(n\) fuera par y \(p\) un primo impar que divide a \(n\),
    tendríamos que \(p - 1 \mid n - 1\),
    un número par dividiendo a uno impar,
    lo que es imposible.
  \end{proof}
  Esto fue demostrado por Korselt en 1899~%
    \cite{korselt99:_probl_chinois},
  los números llevan el nombre de Carmichael
  por ser el primero de hallar uno.

  Supongamos ahora que \(n = p q\), con \(p\) y \(q\) primos
  tales que \(p < q\).
  Entonces debe ser \(q - 1 \mid p q - 1\),
  pero esto es \(q - 1 \mid p (q - 1) + (p - 1)\),
  o sea \(q - 1 \mid p - 1\),
  también imposible.
  En resumen,
  un número de Carmichael
  tiene al menos tres factores primos diferentes,
  todos impares.

  Hay infinitos números de Carmichael,
  como demostraron Alford, Granville y Pommerance~%
    \cite{alford94:_there_infin_many_carmic_number},
  los primeros son:
  \begin{align*}
       561
      &= 3 \cdot 11 \cdot 17 \\
    1\,105
      &= 5 \cdot 13 \cdot 17 \\
    1\,729
      &= 7 \cdot 13 \cdot 19
  \end{align*}
  El primero con cuatro factores primos es:
  \begin{equation*}
    41\,041 = 7 \cdot 11 \cdot 13 \cdot 41
  \end{equation*}

%%% Local Variables:
%%% mode: latex
%%% TeX-master: "clases"
%%% End:


% criptografia.tex
%
% Copyright (c) 2009-2015 Horst H. von Brand
% Derechos reservados. Vea COPYRIGHT para detalles

\chapter{Criptología}
\label{cha:criptografia}
\index{criptologia@criptología|textbfhy}

  Debe distinguirse entre \emph{sistema criptográfico},%
    \index{sistema criptografico@sistema criptográfico|textbfhy}
  un conjunto de algoritmos diseñados para proteger secretos;
  la \emph{criptografía},%
    \index{criptografia@criptografía|textbfhy}
  el trabajo hecho para crear sistemas criptográficos;
  y finalmente \emph{criptoanálisis}%
    \index{criptoanalisis@criptoanálisis|textbfhy},
  trabajo hecho para burlar las protecciones de sistemas criptográficos.
  Se habla de \emph{criptología}%
    \index{criptologia@criptología|textbfhy}
  para referirse a la unión de criptografía y criptoanálisis.
  Es común que se confundan los términos \emph{criptografía}
  y \emph{criptología},
  nos preocuparemos de usar los términos precisos.

  La importancia práctica actual de la teoría de números%
    \index{teoria de numeros@teoría de números}
  está en sus aplicaciones a la criptografía,
  algunas de las cuales describiremos acá.
  Las técnicas mismas
  y los métodos que se han usado para romperlas
  hacen uso intensivo de conceptos de álgebra abstracta.
  Rutinariamente se hace necesario trabajar
  con enteros de cientos o miles de bits,
  o con elementos de grupos%
    \index{grupo}
  o campos finitos%
    \index{campo (algebra)@campo (álgebra)!finito}
  con números de elementos de similar envergadura.

\section{Referencias adicionales}
\label{sec:referencias-adicionales}

  Esta es un área muy amplia,
  hay algunos detalles adicionales sobre la teoría
  (y mucho sobre las aplicaciones prácticas)
  en el clásico de Anderson~\cite{anderson08:_secur_engin}.
  Sinkov~\cite{sinkov09:_elemen_crypt}
  describe métodos criptográficos elementales
  y cómo quebrarlos.
  Matt Curtin~\cite{curtin98:_snake_oil_warnin_signs}
  discute signos de alerta sobre criptografía poco confiable.
  Bernstein, Lange y Schwabe~%
    \cite{bernstein12:_sec_impact_new_crypt_lib}
  describen una biblioteca simple de usar
  para criptografía práctica,
  y discuten algunos de los ataques recientes
  basados en la operación de programas criptográficos.

  Las aplicaciones de criptología
  suelen discutirse en términos de personajes
  \(A\) (también llamada \foreignlanguage{english}{Alice})%
    \index{criptologia@criptología!Alice}
  y \(B\) (apodado \foreignlanguage{english}{Bob})%
    \index{criptologia@criptología!Bob}
  que desean intercambiar mensajes.%
    \index{mensaje}
  A veces aparecen otros actores,
  como \(C\) (alias \foreignlanguage{english}{Charlie})%
    \index{criptologia@criptología!Charlie}
  o \(E\) (\foreignlanguage{english}{Eve}),%
    \index{criptologia@criptología!Eve}
  quien desea interceptar el tráfico
  o intervenirlo de alguna forma
  (\emph{\foreignlanguage{english}{eavesdrop}}, en inglés),
  por ejemplo inyectando mensajes falsificados
  o modificando mensajes.
  Alice y Bob fueron presentados públicamente
  por Rivest, Shamir y Adleman~\cite{rivest83:_RSA},
  John Gordon~%
    \cite{gordon84:_alice_bob_after_dinner_speech}
  dio sus bibliografías definitivas.
  Schneier~\cite{schneier96:_applied_crypt}
  presenta una larga lista de otros personajes que suelen aparecer.

  No entraremos en más detalles en este amplio y complejo campo,
  nuestro interés
  es solo mostrar aplicaciones de la teoría de números
  vista hasta acá.
  Una referencia básica es el manual de Menezes y otros~%
    \cite{menezes96:_handb_applied_crypt},
  el texto de Anderson~%
    \cite{anderson08:_secur_engin}
  trata seguridad desde el punto de vista de ingeniería
  e incluye un capítulo accesible sobre el tema,
  mientras Schneier y coautores
  se concentran en aplicaciones prácticas~%
    \cite{ferguson10:_crypt_engin, schneier96:_applied_crypt}.

  Debe tenerse cuidado,
  se suele confundir la criptografía con seguridad.%
    \index{criptologia@criptología!y seguridad}
  La criptografía moderna es indispensable en muchas aplicaciones,
  pero es solo una de una variedad de herramientas
  requeridas para la seguridad computacional.

\section{Nomenclatura}
\label{sec:criptologia:nomenclatura}
\index{criptologia@criptología!nomenclatura|textbfhy}

  Si solo el destino previsto puede extraer el significado del mensaje
  se habla de \emph{confidencialidad}%
    \index{criptología!confidencialidad}.
  La \emph{integridad} del mensaje%
    \index{criptologia@criptología!integridad}
  se refiere a que el receptor puede asegurarse
  que el mensaje no ha sido alterado,
  \emph{autenticación}%
    \index{criptologia@criptología!autenticacion@autenticación}
  es que el receptor puede verificar la identidad de quien originó el mensaje,
  mientras \emph{no repudiación}%
    \index{criptologia@criptología!no repudiacion@no repudiación}
  asegura que el origen no pueda negar que envió el mensaje.

  Se habla de un mensaje en \emph{texto claro}%
    \index{criptologia@criptología!texto claro|textbfhy}
  (en inglés \emph{\foreignlanguage{english}{plaintext}})%
    \index{plaintext@\emph{\foreignlanguage{english}{plaintext}}|see{criptología!texto claro}}
  y su versión en \emph{texto cifrado}%
    \index{criptologia@criptología!texto cifrado|textbfhy}
  (en inglés \emph{\foreignlanguage{english}{cyphertext}}).%
    \index{ciphertext@\emph{\foreignlanguage{english}{ciphertext}}|see{criptología!texto cifrado}}
  Para nuestros efectos,
  podemos considerar los textos como números grandes
  (por ejemplo,
   tomando el texto claro codificado en UTF\nobreakdash-8),%
     \index{UTF-8@\texttt{UTF-8}}
  posiblemente dividido en \emph{bloques} de tamaño cómodo.
  La transformación de texto claro a texto cifrado se lleva a cabo
  mediante una \emph{función de cifrado} \(C\),%
    \index{criptologia@criptología!funcion de cifrado@función de cifrado|textbfhy}
  que toma el texto claro \(m\) y una \emph{clave} \(k\)%
    \index{criptologia@criptología!clave|textbfhy}
  para producir el respectivo texto cifrado \(c\):
  \begin{equation*}
    c = C_k(m)
  \end{equation*}
  Para descifrar el texto
  se usa la \emph{función de descifrado} \(D\)%
    \index{criptologia@criptología!funcion de descifrado@función de descifrado|textbfhy}
  con clave \(k'\):
  \begin{equation*}
    m = D_{k'}(c)
  \end{equation*}
  En el caso que \(k = k'\),
  se habla de un sistema \emph{simétrico}%
    \index{criptologia@criptología!sistema simetrico@sistema simétrico|see{criptología!clave privada}}
  o \emph{de clave privada}%
    \index{criptologia@criptología!clave privada|textbfhy}
  (claramente,
   en esta situación debe mantenerse secreta la clave).
  En el caso que \(k \ne k'\),
  obviamente habrá una relación entre las dos claves.
  Particularmente interesante
  es el caso en el que conociendo una de las dos
  es muy difícil obtener la otra.
  En tal caso,
  es perfectamente posible publicar \(k\),
  manteniendo secreta \(k'\).
  A estos sistemas se les llama \emph{de clave pública}.%
    \index{criptologia@criptología!clave publica@clave pública|textbfhy}
  Una aplicación interesante de sistemas de clave pública
  es \emph{firmas digitales}:%
    \index{criptologia@criptología!firma digital|textbfhy}%
    \index{criptologia@criptología!autenticacion@autenticación}
  Dado un mensaje \(m\),
  se calcula una función de \emph{\emph{\foreignlanguage{english}{hash}}}%
    \index{criptologia@criptología!funcion de hash@función de \emph{hash}}
  \(h(m)\) del mensaje,
  y se envía \(m\) junto con \(f = D_{k'}(h(m))\);
  quien lo recibe puede aplicar \(C_k(f)\),
  y comprobar que obtiene \(h(m)\).
  Si la función de \emph{\emph{\foreignlanguage{english}{hash}}}
  es tal que sea muy difícil construir un mensaje distinto
  que dé el mismo valor de la función,
  esto certifica
  que únicamente quien conoce \(k'\) puede originar la firma.

  El gran problema con los sistemas simétricos
  es que las partes deben tener algún canal de comunicación seguro
  mediante el cual distribuir las claves.%
    \index{criptologia@criptología!distribuir claves}
  Los sistemas de clave pública no tienen esta dificultad,
  pero por otro lado son muchísimo más demandantes en computación
  que los sistemas simétricos tradicionales.
  Luego lo que se hace normalmente
  es generar una clave para un sistema simétrico tradicional
  al azar,
  y luego usar un sistema de clave pública
  para enviarle esta clave al receptor.%
    \index{criptologia@criptología!sistemas hibridos@sistemas híbridos}

\section{Protocolo Diffie-Hellman de intercambio de claves}
\label{sec:Diffie-Hellman}
\index{Diffie-Hellman, algoritmo de|textbfhy}

  En rigor,
  el protocolo no sirve para intercambiar claves
  sino para acordar una clave entre las partes,
  pero el nombre es el tradicional.
  El algoritmo que discutiremos es de amplio uso~%
    \cite{carts01:_review_diffie_hellman},
  forma la base de mucho de lo que es seguridad en Internet.%
    \index{Internet}

  Supongamos que \foreignlanguage{english}{Alice}%
    \index{criptologia@criptología!Alice}
  y \foreignlanguage{english}{Bob}%
    \index{criptologia@criptología!Bob}
  desean acordar una clave \(K\),
  usando un medio de comunicación que no es seguro.
  La idea básica
  es que \foreignlanguage{english}{Alice} elige un primo \(p\)
  y una raíz primitiva \(g\) módulo \(p\).%
    \index{raiz primitiva@raíz primitiva}
  Ambos valores puede incluso publicarlos,
  mantenerlos en secreto
  no es necesario para la seguridad del esquema
  y pueden perfectamente reutilizarse muchas veces.
  Para generar una clave,
  \foreignlanguage{english}{Alice} elige un valor \(a\)
  (que mantiene en secreto),
  y envía \(A = g^a \bmod p\) a \foreignlanguage{english}{Bob}.
  A su vez,
  \foreignlanguage{english}{Bob} elige \(b\)
  (que también mantiene en secreto),
  y envía  \(B = g^b \bmod p\) a \foreignlanguage{english}{Alice}.
  \foreignlanguage{english}{Alice}
  calcula \(K = B^a \bmod p = g^{a b} \bmod p\),
  y \foreignlanguage{english}{Bob} obtiene \(K = A^b \bmod p\).
  Este valor puede usarse como clave por una sesión%
    \index{criptologia@criptología!clave de sesion@clave de sesión}
  y descartarse después.
  El punto es que con los algoritmos conocidos actualmente
  si \(p\) es un primo de unos \(300\)~dígitos,
  y \(a\) y \(b\) son números de \(100\)~dígitos
  es imposible hallar \(a\)
  si solo se conocen \(p\), \(g\) y \(g^a \bmod p\).
  Hay que tener cuidado con primos
  tales que \(p - 1\) tiene solo factores primos chicos,
  para ese caso hay algoritmos razonablemente eficientes
  que dan \(a\).
  Por esta razón suele elegirse un primo de Sophie Germain,%
    \index{Sophie Germain, primo de}%
    \glossary{Primo de Sophie Germain}
	     {Primo de la forma \(2 q + 1\), con \(q\) primo.}%
    \index{Germain, Sophie}
  vale decir,
  uno de la forma \(2 q + 1\) con \(q\) primo a su vez.
  Conviene trabajar en el subgrupo de orden \(q\),
  dado que de otra forma el valor de \(g^a \bmod p\)
  revela el último bit de \(a\)
  (hay formas eficientes de determinar
   si un número es o no un cuadrado
   en \(\mathbb{Z}_p\)).
  Está claro que exactamente lo mismo puede hacerse
  si \(g\) es un generador de algún otro grupo cíclico.
  La seguridad del esquema
  depende de la dificultad de calcular logaritmos discretos,%
    \index{logaritmo discreto}
  vale decir,
  dados \(g\) y \(g^a\) calcular \(a\).

  Sabemos
  (ver sección~\ref{sec:raices-primitivas})%
    \index{raiz primitiva@raíz primitiva}
  que hay \(\phi(p - 1)\) raíces primitivas módulo \(p\),
  con lo que son relativamente numerosas
  y es razonable buscar una raíz primitiva
  vía intentar valores al azar.
  Para un ejemplo numérico,
  tomemos \(p = 601\),
  con lo que \(p - 1 = 2^3 \cdot 3 \cdot 5^2\),
  y hay \(\phi(p - 1) = 160\) raíces primitivas.
  Una raíz primitiva \(g\) deberá cumplir:
  \begin{alignat*}{3}
    g^{\sfrac{600}{2}}
      &\centernot\equiv 1 \pmod{601}
    &
    g^{\sfrac{600}{3}}
      &\centernot\equiv 1 \pmod{601}
    &
    g^{\sfrac{600}{5}}
      &\centernot\equiv 1 \pmod{601} \\
  \intertext{Intentando con 31 tenemos:}
    31^{300}
      &\equiv 600 \pmod{601}
    \qquad&
    31^{200}
      &\equiv \phantom{00}1 \pmod{601}
    \qquad&
    31^{120}
      &\equiv 432 \pmod{601}
  \end{alignat*}
  La teoría anterior dice que aún requerimos algún valor \(u\)
  tal que \(u^{200} \centernot\equiv 1 \pmod{601}\)
  (ya tenemos cubiertas las potencias de \(2\) y \(5\),
   falta \(3\)),
  algunos intentos dan:
  \begin{equation*}
    357^{200}
      \equiv 576 \pmod{601}
  \end{equation*}
  con lo que \(g = 31 \cdot 357 \bmod 601 = 249\)
  es una raíz primitiva módulo \(p = 601\).

  Si ahora \foreignlanguage{english}{Alice} elige \(a = 17\),
  envía \(249^{17} \bmod 601 = 73\)
  a \foreignlanguage{english}{Bob},
  quien a su vez elige \(b = 58\)
  y envía \(249^{58} \bmod 601 = 149\)
  a \foreignlanguage{english}{Alice}.
  Ambos están ahora en condiciones de calcular \(K\):
  \foreignlanguage{english}{Alice}
  calcula \(149^{17} \bmod 601 = 366\),
  y \foreignlanguage{english}{Bob}
  obtiene \(73^{58} \bmod 601 = 366\).

  Este ejemplo muestra
  que se requiere la factorización completa de \(p - 1\)
  para obtener \(g\),
  razón por la que es imprescindible poder reusar estos valores.

\section{Sistema de clave pública de Rivest,
       Shamir y Adleman (RSA)}
\label{sec:RSA}
\index{criptologia@criptología!RSA}

  Es el sistema de clave pública más usado en la actualidad~%
     \cite{rivest78:_RSA}.
  Se eligen dos números primos \(p\) y \(q\),
  y se calcula el módulo \(n = p q\).
  Se elige además un exponente \(e\),
  y la clave pública es el par \((n, e)\).
  Para cifrar con RSA se usa:
  \begin{equation*}
    c = m^e \bmod n
  \end{equation*}
  Conociendo \(p\) y \(q\)
  podemos generar la correspondiente clave privada \((n, d)\)
  tal que:
  \begin{equation*}
    c^d \bmod n
      = \left(m^e\right)^d \bmod n
      = m
  \end{equation*}
  Por el teorema de Euler,
  si \(\gcd(m, n) = 1\)
  es \(m^{\phi(n)} \equiv 1 \pmod{n}\).
  Si \(d e \equiv 1 \pmod{\phi(n)}\),
  entonces \(c^d \equiv m^{d e} \equiv m \pmod{n}\).
  Más adelante discutiremos cómo elegir los parámetros del caso.

  Podemos elegir el exponente \(e\)
  como un número relativamente pequeño
  (recuérdese que estamos interesados en módulos grandes;
   hoy se recomienda usar módulos de \(4\,096\)~bits,
   unos \(1\,300\)~dígitos decimales)
  de forma que sea cómodo elevar a esa potencia al cifrar,%
    \index{algoritmo!potencia}
  pero el exponente de descifrado
  resultará ser un número muy grande.
  Por el teorema chino de los residuos%
    \index{residuo!teorema chino de los}
  podemos calcular módulos \(p\) y \(q\),
  o sea tenemos realmente:
  \begin{align*}
    c &\equiv m^e \pmod{p} \\
    c &\equiv m^e \pmod{q}
  \end{align*}
  Con esto requerimos que el exponente de descifrado \(d\) cumpla:
  \begin{align*}
    e d
      &\equiv 1 \pmod{(p - 1)} \\
    e d
      &\equiv 1 \pmod{(q - 1)}
  \end{align*}
  Para cumplir con ambas,
  por el teorema~\ref{theo:congruencia-mn}
  basta que:
  \begin{equation*}
    e d \equiv 1 \pmod{\lcm(p - 1, q - 1)}
  \end{equation*}

  Hoy típicamente se usan claves (módulos) de \(4\,096\) bits.
  Los primos \(p\) y \(q\)
  deben elegirse de forma de no ser demasiado cercanos,
  y ninguno de \(p - 1\), \(q - 1\), \(p + 1\) y \(q + 1\)
  debe tener muchos factores primos chicos,
  ya que de ser así \(n\) resulta relativamente fácil de factorizar.
  Puede demostrarse además
  (ver a Wiener~%
    \cite{wiener90:_crypt_short_rsa_secret_exp})
  que si \(d < n^{\sfrac{1}{4}} / 3\)
  es muy fácil recuperar \(d\) conociendo solo \(n\) y \(e\).

  Para elegir \(e\),
  una consideración importante es que sea relativamente pequeño
  y que su representación binaria tenga pocos unos,
  de forma que el cálculo de la potencia resulte simple
  (ver el algoritmo~\ref{alg:power}).
  Originalmente se recomendaba \(e = 3\),
  pero exponentes chicos hacen posibles
  ciertos ataques que consideraremos luego.
  Hoy se recomienda \(e = 2^{16} + 1 = 65\,537\)
  (un primo de Fermat,%
   \index{Fermat, primo de}%
   \glossary{Primo de Fermat}
	    {Primo de la forma \(2^{2^k} + 1\).}
   de la forma \(2^{2^k} + 1\)).
  Además de ser primo,
  este exponente tiene la virtud
  de tener solo dos unos en su expansión binaria;%
    \index{algoritmo!potencia}
  elevar a esta potencia involucra \(5\)~multiplicaciones,
  \(4\)~veces elevar al cuadrado
  y una multiplicación adicional por la base

  No hay similar control sobre \(d\),
  el exponente para descifrar.
  Usar el mínimo común múltiplo%
    \index{minimo comun multiplo@mínimo común múltiplo}
   \(\lcm(p - 1, q - 1)\) en vez de \(\phi(n)\)
  disminuye el valor,
  pero no significativamente.
  Podemos acelerar el proceso
  usando el teorema chino de los residuos,%
    \index{residuo!teorema chino de los}
  teorema~\ref{theo:chino-residuos}.
  Precalculamos valores \(d_1\), \(d_2\), \(q'\)
  (así, la clave privada es realmente \((p, q, d_1, d_2, q')\)),
  donde se cumplen las siguientes relaciones:
  \begin{equation*}
    e d_1
      \equiv 1 \pmod{(p - 1)}
    \hspace{3em}
    e d_2
      \equiv 1 \pmod{(q - 1)}
    \hspace{3em}
    q q'
      \equiv 1 \pmod{p}
  \end{equation*}
  Dado el mensaje cifrado \(c\)
  se obtiene el mensaje \(m\) mediante:
  \begin{align*}
    m_1
      &= c^{d_1} \bmod p \\
    m_2
      &= c^{d_2} \bmod q \\
    h
      &= ((m_1 - m_2) \cdot q') \bmod p \\
    m
      &= m_2 + q h
  \end{align*}
  Los cálculos de \(m_1\) y \(m_2\) se pueden efectuar en paralelo,
  e involucran exponentes y módulos mucho menores
  que en la formulación original,
  lo que hace más rápido el cálculo aún si es secuencial.

  Está claro que exactamente la misma idea es aplicable a módulos
  que son productos de más de dos primos,
  aunque en la práctica se usan solo dos
  (mientras menos factores tenga el módulo,
   más difícil es factorizarlo).

  La seguridad del sistema
  se basa en lo complejo que resulta factorizar números grandes,
  aunque hay algunas otras consideraciones~%
    \cite{boneh99:_twenty_years_attack_RSA,
	  durfee02:_crypt_rsa_using_algeb_lattice_method,
	  salah06:_mathem_attac_rsa_crypt}.
  El récord actual
  (a comienzos del 2012)
% http://www.crypto-world.com/FactorRecords.html
% http://www.loria.fr/~zimmerma/records/factor.html
  de factorización de números generales%
    \index{algoritmo de factorizacion@algoritmo de factorización!record@récord}
  es RSA\nobreakdash-\(768\)~\cite{cryptoeprint:2010:006},
  un número de \(768\)~bits
  (\(232\)~dígitos decimales)
  consumiendo el equivalente aproximado
  de \(2\,000\)~años de procesamiento
  en un Opteron a \(2,2\)\,GHz.

  Resulta que conocer \(d\)
  da una manera eficiente de factorizar \(n\).
  Podemos calcular \(k = d e - 1 = 2^s r\)
  con \(r\) impar y \(s > 0\).
  Entonces \(a^k \equiv 1 \pmod{n}\) para todo \(a\),
  y \(a^{k / 2}\) es raíz cuadrada de 1 módulo \(n\).
  Por el teorema chino de los residuos,
  \(1\) tiene cuatro raíces cuadradas módulo \(n = p q\):
  son \(x = u v\)
  donde \(u \equiv \pm 1 \pmod{p}\) y \(v \equiv \mp 1 \pmod{q}\).
  Como en la prueba de primalidad de Miller-Rabin,%
    \index{Miller-Rabin, prueba de}
  eligiendo \(a\) al azar e intentando
  \(a^r \bmod n\), \(a^{2r} \bmod n\), \ldots, \(a^{k / 2} \bmod n\)
  rápidamente hallaremos una raíz cuadrada \(x\) no trivial de 1,
  y obtenemos una factorización vía \(\gcd(x - 1, n)\).

  Un uso típico de RSA es enviar el mismo mensaje \(m\)
  a un grupo de \(k\) personas,
  donde la persona \(i\) usa clave pública \((n_i, e_i)\).
  Por simplicidad,
  supongamos \(e_i = 3\).
  Además,
  los \(n_i\) son relativamente primos a pares
  (en caso contrario,
   factorizar algunos es trivial).
  Recolectando tres mensajes cifrados \(c_i = m^3 \bmod n_i\),
  vía el teorema chino de los residuos podemos calcular
  \(c' \equiv m^3 \pmod{n_1 n_2 n_3}\),
  y como \(m < n_i\) esto da \(c' = m^3\) en \(\mathbb{Z}\),
  y basta calcular una raíz cúbica.
  Un ataque afín,
  debido a Franklin y Reiter~%
     \cite{franklin95:_linear_protoc_failure_rsa_exp_three},
  funciona si se tienen varios mensajes
  relacionados por una función lineal conocida
  (por ejemplo,
   si se ``rellena'' un mensaje corto
   agregando bits fijos o conocidos)
  cifrados con el mismo exponente.
  La complejidad del ataque es cuadrático en \(e\) y \(\log n\).
  Otro ataque,
  debido a Coppersmith, Franklin, Patarin y Reiter~%
    \cite{coppersmith96:_low_exp_rsa_relat_mesgs},
  es aplicable cuando se conocen \(e\) mensajes cifrados
  con el mismo módulo
  relacionados por polinomios conocidos.
  Por esto se sugiere usar \(e = 2^{16} + 1 = 65\,537\)
  (un primo de Fermat).

\subsection{Firma digital usando RSA}
\label{sec:firma-digital-RSA}
\index{criptologia@criptología!firma digital!RSA}

  Para firmar un mensaje usando RSA,
  se elige
  una función
    de \emph{\foreignlanguage{english}{hash}} criptográfica \(h\).
  Se envía el mensaje \(m\) junto con \(h(m)^d \bmod n\),
  cosa que solo puede hacer
  quien conozca el exponente secreto \(d\).
  Quien recibe el mensaje puede verificar la firma
  elevando al exponente público \(e\) módulo \(n\)
  y confirmando que el resultado coincide con el valor de \(h(m)\).

\section{El estándar de firma digital (DSS)}
\label{sec:DSS}
\index{criptologia@criptología!firma digital!DSA}
\index{criptologia@criptología!firma digital!DSS}

  El \emph{\foreignlanguage{english}{Digital Signature Algorithm}}
  (DSA)
  es un estándar del gobierno federal
  de Estados Unidos de Norteamérica
  para firmas digitales,
  a ser usado en el
  \emph{\foreignlanguage{english}{Digital Signature Standard}}
  (DSS),
  estándar FIPS~186~\cite{FIPS-186},
  adoptado en 1993.
  Es una modificación del esquema de ElGamal~%
    \cite{elgamal85:_public_key_crypt_signat_schem},
  Anderson y Vaudenay~\cite{anderson96:_minding_your_p_and_q}
  discuten algo de su diseño y algunos ataques.
  Hubo una revisión menor en 1996 como FIPS~186-1~\cite{FIPS-186-1},
  fue expandido en 2000
  (FIPS~186-2)
  y se rehizo completo en 2009,
  especificando algoritmos adicionales
  (FIPS~186-3~\cite{FIPS-186-3}).
  La versión actual data de 2013
  (FIPS~186-4~\cite{FIPS-186-4}).
  El algoritmo DSA tiene dos fases,
  en la primera se eligen los parámetros del algoritmo,
  que pueden compartirse entre diferentes usuarios;
  mientras la segunda calcula claves públicas y privadas
  para un usuario individual.
  Con la clave privada se firma un documento,
  y con la correspondiente clave pública
  se verifica que la firma es genuina.

\subsection{Selección de parámetros}
\label{sec:DSA-parametros}

  Se efectúan las siguientes operaciones:
  \begin{itemize}
  \item
    Se elige
    una función
    de \emph{\foreignlanguage{english}{hash}} criptográfica aprobada \(H\)
    (la versión original de DSA usaba SHA\nobreakdash-\(1\),
     actualmente también se especifica SHA\nobreakdash-\(2\)~%
       \cite{FIPS-180-3}).
    La salida de \(H\) puede truncarse al largo de la clave.
  \item
    Decida largo de clave,
    el par \((L, N)\),
    determinante para la seguridad del esquema.
    La versión actual~%
      \cite{FIPS-186-3}
    especifica pares
    \((1024, 224)\), \((2048, 224)\), \((2048, 256)\)
    y \((3072, 256)\).
  \item
    Elija un primo \(q\) de \(N\) bits.
    Nótese que \(N\) debe ser menor o igual
    al largo de la salida de \(H\).
  \item
    Elija un primo \(p\) de \(L\) bits
    tal que \(p - 1\) es múltiplo de \(q\).
  \item
    Elija \(g\),
    un número cuyo orden multiplicativo módulo \(p\) es \(q\).
    Esto se obtiene fácilmente como \(h^{(p - 1) / q}\)
    para \(1 < h < p - 1\) arbitrario,
    intentando nuevamente si el resultado es 1.
    La mayoría de los \(h\) producen lo buscado,
    suele simplemente usarse \(h = 2\).
  \end{itemize}
  Los parámetros \((p, q, g)\) pueden compartirse.

\subsection{Generar claves para un usuario}
\label{sec:DSA-claves}

  Dado un conjunto de parámetros,
  se calculan las claves pública y privada para un usuario.
  \begin{itemize}
  \item
    Elija \(x\) al azar,
    donde \(0 < x < q\).
  \item
    Calcule \(y = g^x \bmod p\).
  \end{itemize}
  La clave pública es \((p, q, g, y)\),
  la clave privada es \(x\).

\subsection{Firmar y verificar firma}
\label{sec:DSA-firmar}

  Sea \(m\) el mensaje.
  Para firmarlo,
  se procede como sigue:
  \begin{itemize}
  \item
    Genere un valor \(k\) al azar para este mensaje,
    donde \(0 < k < q\)
  \item
    Calcule \(r = (g^k \bmod p) \bmod q\),
    en el improbable caso que resulte \(r = 0\)
    elija un nuevo valor de \(k\)
  \item
    Calcule \(s = (k^{-1} \cdot (H(m) + x \cdot r)) \bmod q\).
    En el improbable caso que \(s = 0\),
    elija un nuevo valor de \(k\)
  \end{itemize}
  La firma es \((r, s)\).

  Para verificar la firma,
  se procede como sigue.
  Si no se cumplen \(0 < r < q\) y \(0 < s < q\),
  la firma se rechaza.
  Enseguida:
  \begin{itemize}
  \item
    Calcule \(w = s^{-1} \bmod q\)
  \item
    Calcule \(u_1 = H(m) \cdot w \bmod q\)
    y \(u_2 = r \cdot w \bmod q\)
  \item
    Calcule \(v = ((g^{u_1} \cdot y^{u_2} \bmod p) \bmod q)\)
  \end{itemize}
  La firma es válida si \(v = r\).

\subsection{Correctitud del algoritmo}
\label{sec:DSA-correcto}

  El algoritmo es correcto,
  en el sentido que quien verifica siempre acepta una firma válida.

  Primeramente,
  por el teorema de Fermat
  \(g^q \equiv h^{p - 1} \equiv 1 \pmod{p}\);
  como \(g > 1\) y \(q\) es primo,
  el orden de \(g\) es \(q\).
  Al firmar se calcula:
  \begin{equation*}
    s = k^{-1} \cdot (H(m) + x r) \bmod q
  \end{equation*}
  por lo que:
  \begin{align*}
    k
      &\equiv H(m) \cdot s^{-1} + x r s^{-1} \\
      &\equiv H(m) \cdot w + x r w \pmod{q}
  \end{align*}
  Como \(g\) es de orden \(q\) módulo \(p\),
  tenemos:
  \begin{align*}
    g^k
      &\equiv g^{H(m) w} y^{r w} \\
      &\equiv g^{u_1} y^{u_2} \pmod{p}
  \end{align*}
  y finalmente:
  \begin{equation*}
    r
      = (g^k \bmod p) \bmod q
      = (g^{u_1} y^{u_2} \pmod p) \bmod q
      = v
  \end{equation*}

\subsection{Ataques a DSS}
\label{sec:ataques-DSS}

  Lawson~\cite{lawson10:_dsa_req_random_value}
  indica que si tenemos dos firmas efectuadas con el mismo \(k\),
  en \(\mathbb{Z}_q\):
  \begin{align}
    r
      &= g^k \bmod p \label{eq:DSA:r} \\
    S_a
      &= k^{-1} (H(M_a) + x \cdot r) \label{eq:DSA:Sa} \\
    S_b
      &= k^{-1} (H(M_b) + x \cdot r) \label{eq:DSA:Sb}
  \end{align}
  De hallar dos firmas con el mismo \(r\),
  sabemos que se repitió \(k\);
  con \(r\) de~\eqref{eq:DSA:Sa} y~\eqref{eq:DSA:Sb}
  podemos despejar \(k\) y en consecuencia calcular \(x\).
  Un ataque similar
  es aplicable si se conocen algunos bits de \(k\),
  usando más firmas.

  El requerimiento de que \(k\) se elija al azar es crítico.
  Por ejemplo,
  el ampliamente publicitado problema de seguridad en Debian
  restringió el número de posibles \(k\) a \(32\,767\),
  lo que hace viable intentarlos todos para recuperar la clave.
  Nótese que esto no depende
  de lo seguro que haya sido el proceso de generarla,
  un único uso descuidado la revela.

% Fixme: Otros algoritmos: El Gamal
%	 Ataques a RSA ~~-> paper del caso
%	 Firmas digitales, ...

% Fixme: Recalcar que el algoritmo _no_ es todo (citar p.ej. Anderson)

\section{Otras consideraciones}
\label{sec:consideraciones-modulos}
\index{criptologia@criptología!consideraciones}

  Los algoritmos criptográficos basados en teoría de números
  usan números primos como partes de sus claves.
  En el caso de Diffie-Hellman,
  el primo usado puede publicarse,
  en caso de RSA
  es clave que los factores del módulo permanezcan secretos
  (deben generarse al azar,
   haciendo que sea difícil adivinarlos).
  Sin embargo,
  estudios recientes~%
    \cite{cryptoeprint:2012:064,
	  heninger12:_mining_your_ps_qs}
  han mostrado que una fracción no despreciable
  de las claves usadas en la práctica comparten factores,
  lo que las hace vulnerables.

  Para determinar factores comunes entre las claves,
  estos trabajos usan un truco debido a Dan Bernstein~%
    \cite{bernstein05:_factor_compr_essen_linear_time},
  quien luego da una versión mejorada~%
    \cite{bernstein04:_faster_factorization_coprimes}.
  Supongamos el conjunto de módulos
  \(m_1\), \(m_2\), \ldots, \(m_n\).
  En el caso de~\cite{cryptoeprint:2012:064}
  son \(4,7\)~millones de módulos de \(1\,024\)~bits,
  y calcular los máximos comunes divisores de todos los pares
  para detectar factores comunes está fuera de cuestión.
  Pero se puede proceder
  calculando primero \(M = m_1 \cdot m_2 \dotsm m_n\),
  y luego para cada \(m_i\) calcular
  \(\gcd(m_i, M \bmod m_i^2)\).
  Esto involucra un cálculo largo inicial para calcular \(M\),
  luego una operación costosa al calcular \(M \bmod m_i^2\)
  y determinar un máximo común divisor
  para cada módulo en el conjunto.
  Estas operaciones son razonablemente rápidas de efectuar
  en un computador común.

\section{Criptografía de curvas elípticas}
\label{sec:EC-criptography}
\index{criptologia@criptología!curvas elipticas@curvas elípticas}

  Recientemente se han introducido variantes
  de algunos métodos criptográficos
  que en vez de trabajar en el grupo \(\mathbb{Z}_p^\times\)
  usan el grupo de una curva elíptica~%
    \cite{hankerson04:_guide_ellip_curve_crypt}.
  La razón de usar curvas elípticas
  es que el problema de obtener \(k\) dados \(Q = k P\) y \(P\)
  en el grupo de una curva elíptica
  parece ser mucho más difícil de resolver
  que el problema equivalente en \(\mathbb{Z}_p^\times\).
  Eso sí que la selección de la curva
  y los demás parámetros no son triviales,
  por lo que hay curvas sugeridas~%
    \cite{secg09:_sec1,
	  secg10:_sec2,
	  NIST99:_recom_ellip_curves_feder_gover_use},
  mientras paranoicos terminales
  generarán sus propias curvas y parámetros.
  Para determinar el orden del grupo
  hay un algoritmo razonablemente eficiente
  ideado por Schoof~%
    \cite{schoof95:_count_point_ellip_curves_finit_field},
  con mejoras de Elkies y Atkin que solo circulan como borradores,
  Dewaghe~%
    \cite{dewaghe98:_remar_schoof_elkies_atkin_algor}
  describe la versión usada en la práctica.
  En~\cite{brainpool05:_stand_curves_curve_gener}
  se describe el proceso para generar curvas en detalle
  (y se dan curvas alternativas para uso criptográfico).
  En 1997 Certicom publicó una colección de desafíos~%
    \cite{certicom97:_ecc_chall},
  discusión de algoritmos relevantes
  y estimaciones del trabajo para resolverlos se dan en~%
    \cite{bailey09:_ecc2x},
  avances concretos
  respecto del menor problema
  aún sin resolver en 2012 se discuten en~%
    \cite{bailey09:_break_ecc2k-130}.

  Los algoritmos que siguen
  suponen que se acuerdan parámetros de dominio:
  El campo \(F\),
  los parámetros \(a\) y \(b\) de la curva,
  un generador \(g\) del grupo,
  el orden \(n\) de \(g\)
  (generalmente elegido como un primo),
  y el cofactor \(h\)
  (el orden del grupo de la curva dividido por el orden de \(g\),
   conviene que sea pequeño,
   menor a \(4\) y ojalá \(1\)).

\subsection{Intercambio de claves}
\label{sec:ECDH}

  Es usar la idea de Diffie-Hellman
  (ver la sección~\ref{sec:Diffie-Hellman})
  en una curva elíptica.
  En inglés
  le llaman
   \emph{\foreignlanguage{english}{Elliptic Curve Diffie-Hellman}},
  abreviado \emph{ECDH}.
  El funcionamiento es similar al algoritmo clásico.
  Para generar una clave compartida
  entre \foreignlanguage{english}{Alice}
  y \foreignlanguage{english}{Bob}
  proceden como sigue:
  \begin{enumerate}
  \item
    \foreignlanguage{english}{Alice}
    y \foreignlanguage{english}{Bob} tienen
    claves privadas \(d_A\) y \(d_B\)
    (enteros en el rango \(1\) a \(n - 1\)),
    y calculan \(Q_A = d_A g\) y \(Q_B = d_b g\)
  \item
    Intercambian \(Q_A\) y \(Q_B\) a través del medio inseguro
  \item
    \foreignlanguage{english}{Alice} calcula \(K = d_A Q_B\),
    mientras \foreignlanguage{english}{Bob}
    obtiene este valor como \(K = d_B Q_A\).
    Dado \(K = (x_k, y_k)\),
    la mayoría de los protocolos
    usan \(x_k\) para derivar la clave a ser usada.
  \end{enumerate}

  Una variante resistente a ciertos ataques es FHMQV~%
    \cite{cryptoeprint:2009:408}.

\subsection{Firmas digitales}
\label{sec:ECDSA}

  Esta es una variante del algoritmo DSA
  (sección~\ref{sec:DSS}).
  \foreignlanguage{english}{Alice} tiene una clave privada \(d_A\)
  (un entero al azar en el rango \(1\) a \(n - 1\))
  y su clave pública \(Q_A = d_A g\).
  Sea \(L_n\) el largo en bits del orden \(n\) del grupo.

  Si \foreignlanguage{english}{Alice}
  quiere enviar un mensaje \(m\)
  firmado a \foreignlanguage{english}{Bob},
  procede como sigue:
  \begin{enumerate}
  \item
    Calcula \(e = h(m)\),
    donde \(h\)
    es una función de \emph{\foreignlanguage{english}{hash}} criptográfica,
    y sea \(z\) los \(L_n\) bits más significativos de \(e\)
  \item
    Elija un entero \(k\) al azar entre \(1\) y \(n - 1\)
  \item
    Calcule \(r = x_1 \bmod n\),
    donde \((x_1, y_1) = k g\).
    Si \(r = 0\),
    vuelva al punto 2.
  \item
    Calcule \(s \equiv k^{-1}(z + r d_A) \pmod{n}\).
    Si \(s = 0\),
    vuelva al punto 2.
  \end{enumerate}
  La firma es el par \((r, s)\) resultante.

  Para verificar la firma,
  \foreignlanguage{english}{Bob}
  primero verifica la clave pública
  de \foreignlanguage{english}{Alice}:
  \begin{enumerate}
  \item
    Verifica que \(Q_A \ne 0\)
    y que sus coordenadas son válidas
  \item
    Verifica que \(Q_A\) está en la curva
  \item
    Verifica que \(n Q_A = 0\)
  \end{enumerate}
  Enseguida,
  para verificar la firma del mensaje \(m\),
  repite el cálculo que lleva a \(z\),
  luego:
  \begin{enumerate}
  \item
    Verifica que \(r\) y \(s\) estén en el rango \(1\) a \(n - 1\).
    En caso contrario,
    la firma no es válida.
  \item
    Calcula \(w = s^{-1}\) módulo \(n\),
    luego \(u_1 = z w \bmod n\) y \(u_2 = r w \bmod n\).
    Con esto determina \((x_1, y_1) = u_1 g + u_2 Q_A\).
  \end{enumerate}
  La firma es válida si \(r \equiv x_1 \pmod{n}\),
  en caso contrario no es válida.

%%% Local Variables:
%%% mode: latex
%%% TeX-master: "clases"
%%% End:


% combinatoria-elemental.tex
%
% Copyright (c) 2010-2014 Horst H. von Brand
% Derechos reservados. Vea COPYRIGHT para detalles

\chapter{Combinatoria elemental}
\label{cha:combinatoria-elemental}
\index{combinatoria}

  Consideremos el problema de contar sistemáticamente
  los elementos de colecciones de objetos.
  Nuestro interés en el tema es que por ejemplo
  el comportamiento de un algoritmo de ordenamiento
  dependerá del número de disposiciones distintas
  en que pueden venir los datos,
  y ciertas características de dicho orden.
  Contar éstos,
  particularmente para conjuntos de datos de tamaño interesante,
  generalmente no es posible manualmente.
  Acá nos concentraremos en algunas técnicas simples,
  de aplicabilidad sorprendentemente amplia.
  Más adelante veremos herramientas adicionales.

\section{Técnicas básicas}
\label{sec:tecnicas-basicas-conteo}
\index{combinatoria!tecnicas basicas@técnicas básicas}

  Las herramientas básicas son:
  \begin{description}
  \item[Biyecciones (funciones 1--1):]
    \index{combinatoria!biyecciones}
    Si hay una función \(1\) a \(1\)
    como \(f \colon \mathcal{X} \rightarrow \mathcal{Y}\!\),
    entonces \(\lvert \mathcal{X} \rvert
		 = \lvert \mathcal{Y} \rvert\)
    (esto incluso lo usamos para definir cardinalidades
     en el capítulo~\ref{cha:numerabilidad}).
    Más en general para funciones \(k\) a \(1\):
    Si hay tal función
      \(g \colon \mathcal{X} \rightarrow \mathcal{Y}\!\),
    entonces
      \(\lvert \mathcal{X} \rvert
	  = k \cdot \lvert \mathcal{Y} \rvert\).
  \item[Regla de la suma:]
    \index{combinatoria!regla de la suma}
    Si \(\mathcal{A} \cap \mathcal{B} = \varnothing\),
    entonces
      \(\lvert \mathcal{A} \cup \mathcal{B} \rvert
	   = \lvert \mathcal{A} \rvert
	       + \lvert \mathcal{B} \rvert\).
    Esto se generaliza en forma obvia a un número mayor
    de conjuntos disjuntos a pares:
    \begin{equation*}
      \lvert \mathcal{A}_1 \cup \mathcal{A}_2
	\cup \dotso \cup \mathcal{A}_r \rvert
	  = \lvert \mathcal{A}_1 \rvert
	      + \lvert \mathcal{A}_2 \rvert
	      + \dotsb + \lvert \mathcal{A}_r \rvert
    \end{equation*}
    Si la intersección no es vacía,
    al sumar los tamaños estamos contando la intersección dos veces,
    o sea para dos conjuntos debemos hacer:
    \begin{equation*}
      \lvert \mathcal{A} \cup \mathcal{B} \rvert
	 = \lvert \mathcal{A} \rvert + \lvert \mathcal{B} \rvert
	   - \lvert \mathcal{A} \cap \mathcal{B} \rvert
    \end{equation*}
    Más adelante
    (capítulo~\ref{cha:pie})%
      \index{inclusion y exclusion, principio de@inclusión y exclusión, principio de}
    veremos cómo se puede manejar esto
    si hay más de dos conjuntos involucrados.
  \item[Contar por filas y por columnas:]
    \index{combinatoria!contar por filas y columnas}
    Si \(\mathcal{S} \subseteq \mathcal{X} \times \mathcal{Y}\),
    y para \(x \in \mathcal{X}\) e \(y \in \mathcal{Y}\) definimos:
    \begin{align*}
      r_x(\mathcal{S})
	&= \lvert \{(x, y) \in \mathcal{S}
		      \colon y \in \mathcal{Y}\} \rvert \\
      c_y(\mathcal{S})
	&= \lvert \{(x, y) \in \mathcal{S}
		      \colon x \in \mathcal{X}\} \rvert
    \end{align*}
    Entonces:
    \begin{equation*}
      \lvert \mathcal{S} \rvert
	 = \sum_{x \in \mathcal{X}} r_x(\mathcal{S})
	 = \sum_{y \in \mathcal{Y}} c_y(\mathcal{S})
    \end{equation*}

    En forma más general,
    si hay dos (o más) maneras de contar algo,
    debieran coincidir los resultados.
  \item[Regla del producto:]
    \index{combinatoria!regla del producto}
    Si al contar por filas y columnas
    tomamos \(\mathcal{S} = \mathcal{X} \times \mathcal{Y}\),
    resulta:
    \begin{equation*}
      \lvert \mathcal{S} \rvert
	= \lvert \mathcal{X} \rvert
	    \cdot \lvert \mathcal{Y} \rvert
    \end{equation*}
    dado que en ese caso
    \(r_x(\mathcal{S}) = \lvert \mathcal{Y} \rvert \)
    y \(c_y(\mathcal{S}) = \lvert \mathcal{X} \rvert\).
  \end{description}

  Algunos ejemplos simples:
  \begin{itemize}
  \item
    ¿Cuántas patentes antiguas
    (\(2\) letras, pero no \(\mathtt{Q}\) ni \(\mathtt{W}\);
    y \(4\) dígitos)
    hay?

    Podemos considerarlo como una tupla.
    Como hay \(24\) letras permitidas y \(10\) dígitos,
    por la regla del producto esto corresponde a:
    \begin{equation*}
      24 \cdot 24 \cdot 10 \cdot 10 \cdot 10 \cdot 10
	= 5\,760\,000
    \end{equation*}
    posibilidades.
  \item
    Se estaban acabando los números con el esquema anterior,
    se agregó la letra \(\mathtt{W}\).
    ¿Cuántos números se agregan?

    Nuevamente una tupla,
    pero ahora de \(25\) letras y \(10\) dígitos.
    La regla del producto da para el nuevo total:
    \begin{equation*}
      25 \cdot 25 \cdot 10 \cdot 10 \cdot 10 \cdot 10
	= 6\,250\,000
    \end{equation*}
    Por la regla de la suma,
    las patentes actuales son las antiguas y las agregadas,
    con lo que las agregadas son:
    \begin{equation*}
      6\,250\,000 - 5\,760\,000
	= 490\,000
    \end{equation*}
  \item
    ¿Cuántas patentes nuevas
    (\(4\) consonantes y \(2\) dígitos)
    hay?

    Otra vez una tupla.
    Son \(21 \cdot 21 \cdot 21 \cdot 21 \cdot 10 \cdot 10
	    = 19\,448\,100\).
  \item
    ¿Cuántas patentes hay en total?

    Son el conjunto de patentes antiguas y las nuevas,
    lo que da por la regla de la suma:
    \begin{equation*}
      6\,250\,000 + 19\,448\,100
	= 25\,698\,100
    \end{equation*}
  \item
    En la Universidad de Miskatonic,
    el decano Halsey insiste en que
    todos los estudiantes
    deben tomar exactamente cuatro cursos por semestre.
    Pide a los profesores
    que le hagan llegar las listas de los alumnos en sus cursos,
    pero estos solo le informan los números de estudiantes,
    ver el cuadro~\ref{tab:numero-cursos}.
    \begin{table}[htbp]
      \centering
      \begin{tabular}[htbp]{|l|l|>{\(}r<{\)}|}
	\hline
	\multicolumn{1}{|c|}{\rule[-0.7ex]{0pt}{3ex}\textbf{Profesor}} &
	   \multicolumn{1}{c|}{\textbf{Materia}} &
	\multicolumn{1}{c|}{\textbf{Nº}} \\
	\hline
	\rule[-0.7ex]{0pt}{3ex}%
	Ashley	 & Física      &  45 \\
	Dexter	 & Zoología    &  29 \\
	Dyer	 & Geología    &  33 \\
	Ellery	 & Química     &   2 \\
	Lake	 & Biología    &  12 \\
	Morgan	 & Arqueología &   5 \\
	Pabodie	 & Ingeniería  & 103 \\
	Upham	 & Matemáticas &  95 \\
	Wilmarth & Inglés      &   7 \\
	\hline
      \end{tabular}
      \caption{Número de alumnos por curso}
      \label{tab:numero-cursos}
    \end{table}
    ¿Qué puede decir el decano Halsey con estos datos?

    Si consideramos los pares (alumno, curso),
    la suma de cada fila es el número de cursos que el alumno toma.
    Por tanto,
    si cada alumno toma cuatro cursos,
    la suma total debe ser divisible por cuatro.
    La suma de cada columna es el número de alumnos en el curso.
    Pero en este caso la suma total
    de los alumnos por curso es \(331\),
    así que la condición del decano no se está cumpliendo.
  \end{itemize}

  Como un ejemplo más complejo,
  usando estas ideas podemos demostrar nuevamente
  para la función \(\phi\) de Euler:
  \begin{theorem}[Identidad de Gauß]
    \index{Gauss, identidad de@Gauß, identidad de}
    \label{theo:Gauss-identity-2}
    Para todo entero \(n\),
    tenemos:
    \begin{equation*}
      n = \sum_{d \mid n} \phi(d)
    \end{equation*}
    donde la suma se extiende
    sobre los enteros \(d\) que dividen a \(n\).
  \end{theorem}
  La idea de la siguiente demostración
  viene del conjunto de fracciones:
  \begin{equation*}
    \left\{ \frac{1}{n}, \frac{2}{n},
	    \dotsc, \frac{n - 1}{n} \right\}
      = \left\{
	  \frac{a_1}{b_1}, \frac{a_2}{b_2}, \dotsc,
	    \frac{a_{n - 1}}{b_{n - 1}}
	\right\}
  \end{equation*}
  donde \(a_r / b_r\) está en mínimos términos,
  o sea con \(d_r = \gcd(r, n)\):
  \begin{equation*}
    \frac{a_r}{b_r}
      = \frac{r / d_r}{n / d_r}
  \end{equation*}
  Cada \(b_r\) aparece exactamente \(\phi(b_r)\) veces.
  \begin{proof}
    Sean \(\mathcal{S}\) los pares \((d, f)\)
    tales que \(d \mid n\), \(1 \le f \le d\) y \(\gcd(f, d) = 1\).
    Sumando por filas tenemos:
    \begin{equation*}
      \lvert \mathcal{S} \rvert
	= \sum_{d \mid n} \phi(d)
    \end{equation*}
    Para demostrar que \(n = \lvert \mathcal{S} \rvert\),
    construimos una biyección \(\beta\)
    entre \(\mathcal{S}\) y los enteros entre \(1\) y \(n\).

    Sea \(\beta(d, f) =	 f n / d\).
    Esto siempre es un entero positivo,
    ya que \(d \mid n\);
    y como \(1 \le f \le d\),
    es a lo más \(n\).
    Para demostrar que es una inyección,
    consideremos:
    \begin{align*}
      \beta(d, f)
	&= \beta(d', f') \\
      f n / d
	&= f' n / d' \\
      f d'
	&= f' d
    \end{align*}
    Esto último es \(d \mid f d'\),
    y como \(f\) y \(d\) son relativamente primos,
    por el lema~\ref{lem:gcd}
    significa que \(d \mid d'\).
    De la misma forma \(d' \mid d\),
    y resulta \(d = d'\).
    Con esto también es \(f = f'\).

    Para demostrar que es sobre,
    supongamos dado \(1 \le k \le n\),
    y sean:
    \begin{align*}
      g_k
	&= \gcd(k, n) \\
      d_k
	&= n / g_k    \\
      f_k
	&= k / g_k
    \end{align*}
    Tanto \(d_k\) como \(f_k\) son enteros,
    y además \(\gcd(d_k, f_k) = 1\).
    Resulta:
    \begin{align*}
      \beta(d_k, f_k)
	&= \frac{f_k n}{d_k} \\
	&= \frac{k n / g_k}{n / g_k} \\
	&= k
	   \qedhere
    \end{align*}
  \end{proof}

\section{Situaciones recurrentes}
\label{sec:conteos-recurrentes}

  Según Albert~%
    \cite{albert09:_basic_count_princ}
  algunas circunstancias comunes
  se organizan bajo las siguientes ideas:
  \begin{description}
  \item[Objetos distinguibles o no:]
    Al jugar cartas interesa fundamentalmente su pinta y valor,
    mientras al discutir un canasto de frutas
    no interesa la identidad de cada una de las manzanas.
  \item[Repeticiones o no:]
    En un juego de cartas
    se considera de bastante mal gusto
    que una misma carta aparezca varias veces;
    si nos preguntamos
    de cuántas formas pueden entregarse \$\,\(100\)
    usando monedas de \$\,\(1\), \$\,\(5\) y \$\,\(10\),
    claramente se permite que una moneda se repita.
  \item[Orden interesa:]
    Al jugar cartas,
    una mano queda determinada por el conjunto de cartas
    (el orden no importa),
    al discutir números escritos en decimal
    el orden de los dígitos es fundamental.
  \end{description}
  Esto da lugar a varias situaciones diferentes,
  ordenadas aproximadamente en orden de complejidad creciente
  del análisis:
  \begin{description}
  \item[Secuencias:]
    Se dan siempre que el orden interesa.
    Pueden darse tanto situaciones donde se permiten repeticiones
    como cuando no se permiten.
  \item[Conjuntos:]
    No hay repetición
    y no interesa el orden,
    solo si el elemento pertenece a la colección o no.
  \item[Multiconjuntos:]
    Se permiten repeticiones
    y no interesa el orden.
    Un elemento dado puede pertenecer varias veces a la colección.
  \end{description}
  Veamos las distintas situaciones por turno,
  buscando expresiones simples
  para el número total de posibilidades
  suponiendo que estamos tomando \(k\) elementos
  de entre \(n\) opciones.
  \begin{description}
  \item[Secuencias sin repeticiones:]
    Esta situación se conoce como \emph{permutaciones},
    suele anotarse \(P(n, k)\) para el número de permutaciones
    de \(k\) objetos tomados entre un total de \(n\).
    El primer elemento puede elegirse de \(n\)~formas,
    el segundo de \(n - 1\) maneras,
    y así hasta llegar al último,
    que se puede elegir de \(n - k + 1\) maneras.
    Aplicando la regla del producto,
    tenemos:
    \begin{align}
      P(n, k)
	&= n \cdot (n - 1) \dotsm (n - k + 1) \notag \\
	&= n^{\underline{k}} \label{eq:Perm=ff}
    \end{align}
    En el caso particular en que \(k = n\) resulta:
    \begin{align}
      P(n, n)
	&= n^{\underline{n}} \notag \\
	&= n! \label{eq:Perms=n!}
    \end{align}
  \item[Secuencias con repeticiones:]
    Generalmente se llaman
    usando el término inglés
      \emph{\foreignlanguage{english}{strings}}%
      \index{string@\emph{\foreignlanguage{english}{string}}|see{palabra}}
    (también \emph{palabras},%
      \index{palabra}
     o las podemos considerar como tuplas
     cuyos elementos se toman todos del mismo conjunto).
    No hay notación en uso común para este caso.
    Aplicando la regla de multiplicación,
    viendo que cada uno de los \(k\) elementos
    puede elegirse de \(n\) maneras independientemente,
    en este tenemos:
    \begin{equation*}
      n^k
    \end{equation*}
    Un caso de interés
    es contar todas las secuencias hasta cierto largo \(k\).
    Vimos que hay \(n^r\) secuencias de largo \(r\),
    con lo que por el teorema~\ref{theo:suma-geometrica}
    el número buscado es:
    \begin{equation*}
      \sum_{0 \le r \le k} n^r
	= \frac{n^{k + 1} - 1}{n - 1}
    \end{equation*}
  \item[Conjuntos:]
    Para elegir \(k\) elementos de entre \(n\)
    sin interesar el orden
    (lo que se llaman \emph{combinaciones},%
      \index{combinatoria!combinacion@combinación|textbfhy}
     y suele anotarse \(C(n, k)\))
    podemos elegirlos en orden
    (hay \(P(n, k)\) maneras de hacer esto),
    y luego considerar que hay \(P(k, k) = k!\) maneras de ordenar
    los \(k\) elementos elegidos
    (un mapa \(k!\) a \(1\)
     entre las secuencias ordenadas
     y el conjunto de \(k\) elementos elegidos).
    Vale decir,
    el número buscado es:
    \begin{align}
      C(n, k)
	&= \frac{P(n, k)}{P(k, k)} \label{eq:Comb=Perm/Perm} \\
	&= \frac{n^{\underline{k}}}{k!} \label{eq:Comb=ff/f} \\
	&= \frac{n!}{k! (n - k)!} \label{eq:Comb=f/f*f} \\
	&= \binom{n}{k} \label{eq:Comb=binomial}
    \end{align}
    Debido a esto suele leerse \(\binom{n}{k}\)
    como ``\(n\) elija \(k\)''%
      \index{coeficiente binomial}
    (en inglés
      \emph{\(n\) \foreignlanguage{english}{choose} \(k\)}).
    Nótese que:
    \begin{equation*}
      \binom{n}{k} = \binom{n}{n - k}
    \end{equation*}
    lo que puede interpretarse
    diciendo que al elegir
    los \(k\) elementos incluidos en el subconjunto,
    lo que en realidad estamos haciendo
    es elegir los \(n - k\) elementos
    que estamos dejando fuera.
    A esta clase de razonamiento
    se le llama \emph{demostración combinatoria}.%
      \index{demostracion@demostración!combinatoria|textbfhy}
  \item[Multiconjuntos:]
    \index{combinatoria!multiconjunto|see{multiconjunto!número}}
    \index{multiconjunto!numero@número|textbfhy}
    No hay una notación especial aceptada comúnmente para este caso.
    Algunos autores usan:
    \begin{equation*}
      \multiset{n}{k}
    \end{equation*}
    para el caso en que tenemos \(n\) tipos de elementos
    de los cuales tomamos en total \(k\).
    Una manera de representar esta situación
    es mediante variables \(x_r\),
    donde \(x_r\) representa el número de elementos
    de tipo \(r\) elegidos.
    Entonces el número de multiconjuntos de tamaño \(k\)
    tomando de entre \(n\) alternativas
    es el número de soluciones en números naturales
    a la ecuación:
    \begin{equation*}
      x_1 + x_2 + \dotsb + x_n = k
    \end{equation*}
    \begin{figure}[ht]
      \centering
      \pgfimage{images/stars-bars}
      \caption{Una distribución de \(6\) elementos en \(4\) grupos}
      \label{fig:stars-bars}
    \end{figure}
    La solución \(x_1 = 2\), \(x_2 = 0\), \(x_3 = 3\), \(x_4 = 1\)
    al caso \(n = 4\) y \(k = 6\)
    queda ilustrada en la figura~\ref{fig:stars-bars}.
    Esta distribución puede describirse
    con un total de \(n = 6\) asteriscos para la suma,
    separados por \(k - 1 = 3\) barras para marcar las separaciones
    (los extremos son fijos,
     y los omitimos):
    \begin{equation*}
      ** | | *** | *
    \end{equation*}
    Visto de esta forma,
    lo que estamos haciendo es distribuir \(n - 1\) barras
    en \(n + k - 1\) posiciones,
    un total de \(C(n + k - 1, n - 1)\).
    A este tipo de razonamiento se le conoce
    como \emph{\foreignlanguage{english}{stars and bars}}%
      \index{stars and bars@\emph{\foreignlanguage{english}{stars and bars}}}
    en inglés.
    Así,
    el número de soluciones se expresa:
    \begin{equation*}
      \multiset{n}{k}
	= \binom{n + k - 1}{n - 1}
    \end{equation*}
    Nótese que puede escribirse,
    de forma afín a los coeficientes binomiales:
    \begin{equation*}
      \multiset{n}{k}
	= \binom{n + k - 1}{k}
	= \frac{n^{\overline{k}}}{k!}
    \end{equation*}
    Como \(n^{\overline{k}} = (-1)^k \, (-n)^{\underline{k}}\)
    se cumple la curiosa identidad:
    \begin{equation}
      \label{eq:multiset=negative-binomial}
      \multiset{n}{k}
	= (-1)^k \binom{-n}{k}
    \end{equation}
  \end{description}

  Determinemos el número de subconjuntos
  de \(k\) elementos de \([n]\)
  que no contienen elementos consecutivos.%
    \index{combinatoria!subconjuntos sin elementos consecutivos}
  Es claro que si \(k = 0\) hay un único subconjunto
  (el vacío),
  si \(k = 1\) hay \(n\).
  Otros casos simples son:
  \begin{align*}
    n = 3, k = 2 &\colon 1 \quad \{1, 3\} \\
    n = 4, k = 2 &\colon 3 \quad \{1, 3\}, \{1, 4\}, \{2, 4\} \\
    n = 5, k = 2 &\colon 6 \quad \{1, 3\}, \{1, 4\}, \{1, 5\},
				 \{2, 4\}, \{2, 5\},
				 \{3, 5\} \\
    n = 5, k = 3 &\colon 1 \quad \{1, 3, 5\}
  \end{align*}

  Podemos nombrar un subconjunto de \([n]\)
  como \(\{ a_1, a_2, \dotsc, a_k \}\),
  con \(1 \le a_1 < a_2 < \dotsb < a_k \le n\).
  La restricción que no contenga elementos adyacentes
  se traduce en \(a_{r + 1} \ge a_r + 2\)
  para \(1 \le r < k\).
  Definamos nuevas variables:
  \begin{align*}
    d_1
      &= a_1 - 1 \\
    d_{r + 1}
      &= a_{r + 1} - a_r - 2 \quad \text{para \(1 \le r < k\)} \\
    d_{k + 1}
      &= n - a_k
  \end{align*}
  Es claro que la restricción es que \(d_r \ge 0\),
  y suman \(n - (k - 1) \cdot 2 - 1 = n - 2 k + 1\).
  Por lo anterior,
  el número de soluciones a esto es:
  \begin{equation*}
    \multiset{k + 1}{n - 2 k + 1}
      = \binom{n - k + 1}{k}
  \end{equation*}
  Esto coincide con los valores obtenidos antes.

  Una aplicación simple de los resultados anteriores
  es la siguiente:
  \begin{theorem}
    \index{combinatoria!numero de funciones@número de funciones|see{función!número}}
    \index{funcion@función!numero@número|textbfhy}
    \label{theo:numero-funciones}
    Sean \(\mathcal{X}\) e \(\mathcal{Y}\) conjuntos finitos.
    Entonces el número total
    de funciones
      \(f \colon \mathcal{X} \rightarrow \mathcal{Y}\) es:
    \begin{equation*}
      \lvert \mathcal{Y} \rvert^{\lvert \mathcal{X} \rvert}
    \end{equation*}
  \end{theorem}
  \begin{proof}
    Supongamos que \(\lvert \mathcal{X} \rvert = m\).
    Entonces podemos considerar esta situación
    como contar las tuplas
    \((f(1), f(2), \dotsc, f(m))\),
    en las cuales cada elemento
    toma un valor de entre \(\lvert \mathcal{Y} \rvert = n\),
    con lo que por la regla del producto hay \(n^m\) funciones.
  \end{proof}
  Es por este resultado que una notación común
  para el conjunto de funciones de \(\mathcal{X}\) a \(\mathcal{Y}\)
  es \(\mathcal{Y}^{\mathcal{X}}\).

  Una manera de describir un subconjunto \(\mathcal{S}\)
  de un conjunto \(\mathcal{U}\)
  es mediante su \emph{función característica}%
    \index{funcion caracteristica@función característica|see{conjunto!función característica}}%
    \index{conjunto!funcion caracteristica@función característica|textbfhy}
  \(\chi_{\mathcal{S}} \colon \mathcal{U} \rightarrow \{0, 1\}\),
  donde \(\chi_{\mathcal{S}}(u) = 0\)
  significa que \(u\) no pertenece al subconjunto,
  y \(\chi_{\mathcal{S}}(u) = 1\) que pertenece.
  Esta forma de ver las cosas lleva a:
  \begin{corollary}
    \index{combinatoria!numero de subconjuntos@número de subconjuntos|see{conjunto!subconjunto!número}}
    \index{conjunto!subconjunto!numero@número|textbfhy}
    \label{cor:numero-subconjuntos}
    Sea \(\mathcal{A}\) un conjunto finito.
    Entonces hay \(2^{\lvert \mathcal{A} \rvert}\)
    subconjuntos de \(\mathcal{A}\).
  \end{corollary}
  \begin{proof}
    Aplicar el teorema~\ref{theo:numero-funciones}
    al conjunto de funciones características.
  \end{proof}
  Es por esta razón que el conjunto
  de los subconjuntos de \(\mathcal{A}\)
  suele anotarse \(2^{\mathcal{A}}\).
  \begin{corollary}
    \index{combinatoria!numero de relaciones@número de relaciones|see{relación!número}}
    \index{relacion@relación!numero@número|textbfhy}
    \label{cor:numero-relaciones}
    Sean \(\mathcal{A}\) y \(\mathcal{B}\) conjuntos finitos.
    Entonces hay \(2^{\lvert \mathcal{A} \rvert
		      \cdot \lvert \mathcal{B} \rvert}\)
    relaciones de \(\mathcal{A}\) a \(\mathcal{B}\).
  \end{corollary}
  \begin{proof}
    \index{relacion@relación}
    Una relación entre \(\mathcal{A}\) y \(\mathcal{B}\)
    no es más que un subconjunto
    de \(\mathcal{A} \times \mathcal{B}\),
    aplicando la regla del producto
    y luego~(\ref{cor:numero-subconjuntos}) obtenemos lo prometido.
  \end{proof}

  Otro caso importante es contabilizar el número de inyecciones.
  \begin{theorem}
    \index{combinatoria!numero de inyecciones@número de inyecciones|see{función!inyectiva!nùmero}}
    \index{funcion@función!inyectiva!numero@número|textbfhy}
    \label{theo:numero-inyecciones}
    Sean \(\mathcal{X}\) e \(\mathcal{Y}\) conjuntos finitos,
    de cardinalidades \(\lvert \mathcal{X} \rvert = m\)
    e \(\lvert \mathcal{Y} \rvert = n\).
    Entonces el número total de funciones inyectivas
    \(i \colon \mathcal{X} \rightarrow \mathcal{Y}\)
    es \(n^{\underline{m}} = n! / (n - m)!\)
  \end{theorem}
  \begin{proof}
    Si es una inyección,
    no se repiten valores de la función
    (y por tanto \(m \le n\)).
    Si consideramos que \(\mathcal{X}\) son índices
    (definen las posiciones),
    estamos frente a permutaciones de \(n\) elementos
    de los que se eligen \(m\),
    vale decir es:
    \begin{equation*}
      P(n, m)
	= n^{\underline{m}}
	= \frac{n!}{(n - m)!}
      \qedhere
    \end{equation*}
  \end{proof}
  Directamente resulta:
  \begin{corollary}
    \index{combinatoria!numero de biyecciones@número de biyecciones|see{biyección!número}}
    \index{biyeccion@biyección!numero@número|textbfhy}
    \label{cor:numero-biyecciones}
    Sean \(\mathcal{X}\) e \(\mathcal{Y}\) conjuntos finitos
    tales que
      \(\lvert \mathcal{X} \rvert = \lvert \mathcal{Y} \rvert = n\).
    Entonces el número de biyecciones
      \(b \colon \mathcal{X} \rightarrow \mathcal{Y}\)
    es \(n!\).
  \end{corollary}
  \begin{proof}
    Para el caso \(n = m\)
    el teorema~\ref{theo:numero-inyecciones}
    da \(n^{\underline{n}} = n!\).
  \end{proof}

  Otra forma de interpretar
  el corolario~\ref{cor:numero-biyecciones}
  es que hay \(n!\) maneras de ordenar \(n\) elementos diferentes.

  Los números de combinaciones cumplen una colección inmensa
  de equivalencias curiosas.
  \begin{theorem}[Identidad de Pascal]
    \index{Pascal, identidad de|textbfhy}
    \label{theo:identidad-Pascal}
    Para \(n, k \in \mathbb{N}\) se cumplen:
    \begin{align*}
      \binom{n}{0}
	&= \binom{n}{n} = 1 \\
      \binom{n + 1}{k + 1}
	&= \binom{n}{k + 1} + \binom{n}{k}
     \end{align*}
  \end{theorem}
  \begin{proof}
    Primero:
    \begin{align*}
      \binom{n}{n}
	&= \binom{n}{n - n}
	 = \binom{n}{0} \\
      \binom{n}{0}
	&= \frac{n!}{n! \; 0!}
	 = 1
    \end{align*}
    Por el otro lado,
    podemos considerar que \(\binom{n + 1}{k + 1}\)
    corresponde a elegir \(k + 1\) elementos de entre \(n + 1\),
    cosa que se puede hacer fijando uno de los elementos,
    y luego considerar aquellos conjuntos de \(k + 1\) elementos
    que lo incluyen
    (corresponde a elegir los demás \(k\)
     de entre los \(n\) restantes,
     hay \(\binom{n}{k}\) casos de éstos),
    y los que no
    (corresponde a elegir \(k + 1\) elementos de entre los \(n\)
     que son elegibles,
     hay \(\binom{n}{k + 1}\) de estos casos).
    Como el conjunto de los subconjuntos
    que incluyen al elemento seleccionado
    y los que no son disjuntos,
    podemos aplicar la regla de la suma
    para obtener la recurrencia indicada.
  \end{proof}
  Una demostración alternativa es:
  \begin{proof}
    Primeramente,
    siempre es:
    \begin{align*}
      \binom{n}{0}
	&= \frac{n^{\underline{0}}}{0!}
	= 1 \\
      \binom{n}{n}
	&= \frac{n^{\underline{n}}}{n!}
	= \frac{n!}{n!}
	= 1
    \end{align*}
    Luego:
    \begin{align*}
      \binom{n}{k + 1} + \binom{n}{k}
	&= \frac{n^{\underline{k + 1}}}{(k + 1)!}
	     + \frac{n^{\underline{k}}}{k!} \\
	&= \frac{n^{\underline{k}} (n - k)
		   + (k + 1) n^{\underline{k}}}
		{(k + 1)!} \\
	&= \frac{n^{\underline{k}} (n + 1)}{(k + 1)!} \\
	&= \frac{(n + 1)^{\underline{k + 1}}}{(k + 1)!} \\
	&= \binom{n + 1}{k + 1}
      \qedhere
    \end{align*}
  \end{proof}
  \noindent
  Nótese que salvo en \(\binom{n}{n} = 1\)
  no presupone \(n \in \mathbb{N}_0\).

  Un resultado extremadamente importante
  es el que sigue:
  \begin{theorem}[Binomio]
    \index{binomio, teorema del|textbfhy}
    \label{theo:binomio}
    Para \(n \in \mathbb{N}\) tenemos:
    \begin{equation*}
      (a + b)^n
	= \sum_{0 \le k \le n} \binom{n}{k} a^k b^{n - k}
    \end{equation*}
  \end{theorem}
  \begin{proof}
    Por inducción sobre \(n\).%
      \index{demostracion@demostración!induccion@inducción}
    \begin{description}
    \item[Base:]
      Cuando \(n = 0\),
      tenemos:
      \begin{equation*}
	\sum_{0 \le k \le 0}
	  \binom{0}{k} a^k b^{0 - k}
	  = \binom{0}{0} a^0 b^0
	  = 1
      \end{equation*}
    \item[Inducción:]
      Nótese que en las sumatorias siguientes
      el rango de las sumas
      es exactamente los índices
      para los cuales
      no se anulan los coeficientes binomiales respectivos,
      con lo que podemos obviar los límites de las sumas.

      Tenemos:
      \begin{align*}
	(a + b)^{n + 1}
	  &= (a + b)^n \cdot (a + b) \\
	  &= \left(\sum_k \binom{n}{k} a^k b^{n - k}\right)
	       \cdot (a + b) \\
	  &= \sum_k \binom{n}{k} a^{k + 1} b^{n - k}
	       + \sum_k \binom{n}{k} a^k b^{n + 1 - k} \\
	  &= \sum_k \binom{n}{k - 1} a^k b^{n + 1 - k}
	       + \sum_k \binom{n}{k} a^k b^{n + 1 - k} \\
	  &= \sum_k
	       \left(
		 \binom{n}{k - 1}
		   + \binom{n}{k}
	       \right) \, a^k b^{n + 1} \\
	  &= \sum_k \binom{n + 1}{k} \, a^k b^{n + 1 - k} \\
	  &= \sum_{0 \le k \le n + 1}
	       \binom{n + 1}{k} \, a^k b^{n + 1 - k}
      \end{align*}
    \end{description}
    Por inducción,
    vale para \(n \in \mathbb{N}_0\).
  \end{proof}
  Por el teorema~\ref{theo:binomio}
  es que los números \(\binom{n}{k}\)
  se llaman \emph{coeficientes binomiales}.%
    \index{coeficiente binomial|textbfhy}

  Otro resultado importante es el siguiente.
  \begin{theorem}[Multinomio]
    \index{multinomio, teorema del|textbfhy}
    \label{theo:multinomio}
    Para \(n \in \mathbb{N}\)
    tenemos:
    \begin{equation*}
      (a_1 +a_2 + \dotsb a_r)^n
	 = \sum_{k_1 + k_2 + \dotsb + k_r = n}
	      \binom{n}{k_1, k_2, \dotsc, k_r}
		  a_1^{k_1} a_2^{k_2} \dotsb a_r^{k_r}
    \end{equation*}
    donde:
    \begin{equation*}
      \binom{n}{k_1, k_2, \dotsc, k_r}
	= \frac{n!}{k_1! k_2! \dotsm k_r!}
    \end{equation*}
    Esta expresión está definida
    solo si \(n = k_1 + k_2 + \dotsb + k_r\).
  \end{theorem}
  \begin{proof}
    Por inducción fuerte sobre \(r\).%
      \index{demostracion@demostración!induccion@inducción}
    \begin{description}
    \item[Base:]
      Cuando \(r = 2\),
      se reduce al teorema del binomio:
      \begin{equation*}
	\sum_{k_1 + k_2 = n}
	     \binom{n}{k_1, k_2} a_1^{k_1} a_2^{k_2}
	  = \sum_{0 \le k \le n}
	      \binom{n}{k, n - k} a_1^k a_2^{n - k}
	  = \sum_{0 \le k \le n} \binom{n}{k} a_1^k a_2^{n - k}
      \end{equation*}
    \item[Inducción:]
      Tenemos:
      \begin{align*}
	((a_1 &+ \dotsb + a_r) + a_{r + 1})^n \\
	  &= \sum_{0 \le k_{r + 1} \le n}
	       \binom{n}{k_{r + 1}} \,
		 \left(
		   \sum_{k_1 + k_2 + \dotsb + k_r = n - k_{r + 1}}
		   \binom{n - k_{r + 1}}{k_1, k_2, \dotsc, k_r}
		     a_1^{k_1} a_2^{k_2} \dotsm a_r^{k_r}
		 \right)
	       \cdot a_{r + 1}^{n - k_{r + 1}} \\
	  &= \sum_{k_1 + \dotsb + k_{r + 1} = n}
	       \binom{n}{k_{r + 1}}
		 \binom{n - k_{r + 1}}{k_1, k_2, \dotsc, k_r} \,
		   a_1^{k_1} a_2^{k_2} \dotsm a_{k_{r + 1}} \\
	  &= \sum_{k_1 + \dotsb + k_{r + 1} = n}
	       \binom{n}{k_1, k_2, \dotsc, k_{r + 1}} \,
		 a_1^{k_1} a_2^{k_2} \dotsm a_{k_{r + 1}}
      \end{align*}
      Acá usamos:
      \begin{align*}
	\binom{n}{k_{r + 1}}
	  \binom{n - k_{r + 1}}{k_1, k_2, \dotsc, k_r}
	  &= \frac{n!}{k_{r + 1}! (n - k_{r + 1})!}
	       \cdot \frac{(n - k_{r + 1})!}
			  {k_1! \; k_2! \dotsm k_r!} \\
	  &= \frac{n!}{k_1! \; k_2! \dotsm k_{r + 1}!} \\
	  &= \binom{n}{k_1, k_2, \dotsc, k_{r + 1}}
      \end{align*}
    \end{description}
    Por inducción es válido para \(r \ge 2\),
    y claramente es válido para \(r = 0\) y \(r = 1\),
    con lo que vale para \(r \in \mathbb{N}_0\).
  \end{proof}
  Por razones obvias,
  a los \(\binom{n}{k_1, k_2, \dotsc, k_r}\)
  se les llama \emph{coeficientes multinomiales},%
    \index{coeficiente multinomial|textbfhy}
  y tenemos también:
  \begin{equation*}
    \binom{n}{k, n - k} = \binom{n}{k}
  \end{equation*}

% manos-poker.tex
%
% Copyright (c) 2009-2014 Horst H. von Brand
% Derechos reservados. Vea COPYRIGHT para detalles

\section{Manos de poker}
\label{sec:manos-poker}
\index{manos de poker|see{combinatoria!manos de poker}}
\index{combinatoria!manos de poker|textbfhy}

% Fixme: Revisar las reglas, contabilizar _todas_ las manos
%	 (hay que restar alternativas repetidas, etc)
%	 ¿Algún juego más complejo (2 mazos, ...)?

  Nuestro siguiente tema de interés es contar subconjuntos
  que cumplen ciertas restricciones.
  Como conjuntos,
  siguiendo a Lehman, Leighton y Meyer~%
    \cite{lehman15:_mathem_comput_scien},
  usaremos manos de \foreignlanguage{english}{poker}.

  En \foreignlanguage{english}{poker}
  a cada jugador se le da una \emph{mano} de cinco cartas,
  elegidas del mazo inglés,
  formado por cuatro \emph{pintas}:%
    \index{carta!pinta}
  Pica (\(\spadesuit\)),
    \index{carta!pinta!pica (\(\spadesuit\))|textbfhy}
  corazón (\(\heartsuit\)),
    \index{carta!pinta!corazon (\(\heartsuit\))@corazón (\(\heartsuit\))|textbfhy}
  trébol (\(\clubsuit\))
    \index{carta!pinta!trebol (\(\clubsuit\))@trébol (\(\clubsuit\))|textbfhy}
  y diamante (\(\diamondsuit\));
    \index{carta!pinta!diamante (\(\diamondsuit\))|textbfhy}
  en cada pinta hay trece \emph{valores}:%
    \index{carta!valor|textbfhy}
  As, \(2\) a \(10\),
  \foreignlanguage{english}{Jack},
  \foreignlanguage{english}{Queen}
  y \foreignlanguage{english}{King}.
  El número total de manos posibles es:
  \begin{equation*}
    \binom{52}{5} = 2\,598\,960
  \end{equation*}

  Como estrategia general,
  buscaremos secuencias que describan las manos que queremos contar
  (porque contar secuencias es fácil),
  y nos aseguraremos que hay una biyección
  (o que haya alguna otra relación clara,
   como un mapa \(2\) a \(1\))
  entre descripciones
  y manos.

\subsection{Royal Flush}
\label{sec:royal-flush}

  Es la mano más alta en \foreignlanguage{english}{poker}.
  Consta de \foreignlanguage{english}{As},
  \foreignlanguage{english}{King},
  \foreignlanguage{english}{Queen},
  \foreignlanguage{english}{Jack}, \(10\) de la misma pinta,
  por ejemplo:
  \begin{equation*}
    \begin{array}{@{\{}*{4}{c@{\;\;}}c@{\}}}
      A \spadesuit & K \spadesuit & Q \spadesuit &
	J \spadesuit & 10 \spadesuit
    \end{array}
  \end{equation*}
  Está claro que hay una mano de éstas para cada pinta,
  con lo que hay exactamente \(4\).

\subsection{Straight Flush}
\label{sec:straight-flush}

  Consta de \(5\) cartas de la misma pinta en secuencia,
  donde \foreignlanguage{english}{As} cuenta como \(1\)
  (no después de \foreignlanguage{english}{King},
   como en el \foreignlanguage{english}{Royal Flush}).
  Ejemplos son:
  \begin{equation*}
    \begin{array}{@{\{}*{4}{c@{\;\;}}c@{\}}}
      8 \spadesuit & 9 \spadesuit & 10 \spadesuit &
	J \spadesuit & Q \spadesuit \\
      A \heartsuit & 2 \heartsuit & 3 \heartsuit  &
	4 \heartsuit & 5\heartsuit \\
      3 \clubsuit  & 4 \clubsuit  & 5 \clubsuit	  &
	6\clubsuit  & 7\clubsuit
    \end{array}
  \end{equation*}
  Estas manos podemos describirlas mediante
  una secuencia que indica:
  \begin{itemize}
  \item
    El valor de la primera carta en la secuencia.
    Este puede elegirse de \(9\) maneras
    (entre \(1\) y \(9\)).
  \item
    La pinta,
    que puede elegirse de \(4\) maneras.
  \end{itemize}
  En nuestros ejemplos:
  \begin{equation*}
    \begin{array}{l@{{} \longleftrightarrow \{}*{4}{c@{\;\;}}c@{\}}}
      (8, \spadesuit) &
	8 \spadesuit  & 9 \spadesuit & 10 \spadesuit &
	J \spadesuit  & Q \spadesuit \\
      (1, \heartsuit) &
	A \heartsuit  & 2 \heartsuit & 3 \heartsuit  &
	4 \heartsuit  & 5 \heartsuit \\
      (3, \clubsuit)  &
	3 \clubsuit   & 4 \clubsuit  & 5 \clubsuit   &
	6 \clubsuit   & 7 \clubsuit
    \end{array}
  \end{equation*}
  Por la regla del producto,%
    \index{combinatoria!regla del producto}
  el número total de estas manos es:
  \begin{equation*}
    9 \cdot \binom{4}{1} = 36
  \end{equation*}
  Como esto no describe un \foreignlanguage{english}{Royal Flush},
  no hace falta ningún ajuste adicional.

\subsection{Four of a Kind}
\label{sec:four-of-a-kind}

  Esta es una mano con cuatro cartas del mismo valor.
  Por ejemplo:
  \begin{align*}
    \begin{array}{@{\{}*{4}{c@{\;\;}}c@{\}}}
      8 \spadesuit  & 8 \heartsuit & 8 \clubsuit &
	8 \diamondsuit	 & 5 \diamondsuit \\
      2 \spadesuit  & 2 \heartsuit & 2 \clubsuit  &
	2 \diamondsuit	 & 3 \clubsuit
    \end{array}
  \end{align*}
  Para calcular cuántas de estas hay,
  armamos un mapa de secuencias a manos de este tipo
  y contamos las secuencias.
  En este caso,
  una mano queda descrita por:
  \begin{itemize}
  \item El valor que se repite.
  \item El valor de la quinta carta.
  \item La pinta de la quinta carta.
  \end{itemize}
  Hay una biyección entre secuencias de estos tres elementos
  y manos.
  En nuestros ejemplos,
  las correspondencias son:
  \begin{align*}
    \begin{array}{l@{{} \longleftrightarrow \{}*{4}{c@{\;\;}}c@{\}}}
      (8, 5, \diamondsuit) &
	8 \spadesuit & 8 \heartsuit & 8 \clubsuit & 8 \diamondsuit
	   & 5 \diamondsuit \\
      (2, 3, \clubsuit)	   &
	2 \spadesuit & 2 \heartsuit & 2 \clubsuit & 2 \diamondsuit
	   & 3 \clubsuit
    \end{array}
  \end{align*}
  Para el valor tenemos \(13\) posibilidades,
  para el valor de la quinta carta quedan \(12\) posibilidades,
  y hay \(4\) opciones para la pinta de la quinta carta.
  En total,
  usando la regla del producto,%
    \index{combinatoria!regla del producto}
  son \(13 \cdot 12 \cdot 4 = 624\) posibilidades.
  Hay \(1\) en \(2\,598\,960 / 624 = 4\,165\) manos,
  no sorprende que se considere muy buena.

\subsection{Full House}
\label{sec:full-house}

  Es una mano con tres cartas de un valor
  y dos de otro.
  Ejemplos:
  \begin{equation*}
    \begin{array}{@{\{}*{4}{c@{\;\;}}c@{\}}}
       2 \heartsuit & 2 \clubsuit & 2 \diamondsuit & Q \spadesuit
	  & Q \diamondsuit \\
       5 \spadesuit & 5 \clubsuit & 5 \diamondsuit & K \spadesuit
	  & K \heartsuit
    \end{array}
  \end{equation*}
  Nuevamente un mapa con secuencias:
  \begin{itemize}
  \item
    El valor del trío,
    que puede especificarse de \(13\) maneras.
  \item
    Las pintas del trío,
    que son elegir \(3\) de entre \(4\).
  \item
    El valor del par,
    que se puede tomar de \(12\) maneras.
  \item
    Las pintas del par,
    que se eligen \(2\) entre \(4\).
  \end{itemize}
  Las manos ejemplo corresponden con:
  \begin{equation*}
    \begin{array}{l@{{} \longleftrightarrow \{}*{4}{c@{\;\;}}c@{\}}}
      (2, \{\heartsuit, \clubsuit, \diamondsuit\},
       Q, \{\spadesuit, \diamondsuit\}) &
	 2 \heartsuit & 2 \clubsuit & 2 \diamondsuit & Q \spadesuit
	    & Q \diamondsuit \\
      (5, \{\spadesuit, \clubsuit, \diamondsuit\},
       K, \{\spadesuit, \heartsuit\})	&
	 5 \spadesuit & 5 \clubsuit & 5 \diamondsuit & K \spadesuit
	    & K \heartsuit
    \end{array}
  \end{equation*}
  Por la regla del producto%
    \index{combinatoria!regla del producto}
  el número de \foreignlanguage{english}{Full Houses} es entonces:
  \begin{equation*}
    13 \cdot \binom{4}{3} \cdot 12 \cdot \binom{4}{2} = 3\,744
  \end{equation*}

\subsection{Flush}
\label{sec:flush}

  Mano con \(5\) cartas de la misma pinta,
  como por ejemplo:
  \begin{equation*}
    \begin{array}{@{\{}*{4}{c@{\;\;}}c@{\}}}
      A \heartsuit & 3 \heartsuit & 4 \heartsuit &
	8 \heartsuit & K \heartsuit
    \end{array}
  \end{equation*}
  Esto se describe mediante la secuencia que da:
  \begin{itemize}
  \item
    Un conjunto de \(5\) valores,
    se eligen \(5\) de entre \(13\).
  \item
    Una pinta,
    se elige una entre \(4\).
  \end{itemize}
  En nuestro ejemplo:
  \begin{equation*}
    \begin{array}{l@{{} \longleftrightarrow \{}*{4}{c@{\;\;}}c@{\}}}
      (\{A, 3, 4, 8, K\}, \heartsuit) &
	 A \heartsuit & 3 \heartsuit & 4 \heartsuit &
	   8 \heartsuit & K \heartsuit
    \end{array}
  \end{equation*}
  De estas manos hay entonces:
  \begin{equation*}
    \binom{13}{5} \cdot \binom{4}{1} = 5\,148
  \end{equation*}
  Esto también describe al \foreignlanguage{english}{Royal Flush}
  y al \foreignlanguage{english}{Straight Flush},
  debemos restar éstos
  (regla de la suma):%
    \index{combinatoria!regla de la suma}
  \begin{equation*}
    5\,148 - 4 - 36 = 5\,108
  \end{equation*}

\subsection{Manos con dos pares}
\label{sec:dos-pares}

  Interesa calcular cuántas manos con dos pares hay,
  vale decir,
  dos cartas de un valor,
  dos cartas de otro valor,
  y una carta de un tercer valor.
  Ejemplos son:
  \begin{align*}
    \begin{array}{@{\{}*{4}{c@{\;\;}}c@{\}}}
      3 \heartsuit & 3 \diamondsuit & Q \spadesuit & Q \clubsuit
	 & 5 \diamondsuit \\
      9 \heartsuit & 9 \clubsuit    & K \spadesuit & K \diamondsuit
	 & 2 \spadesuit
    \end{array}
  \end{align*}
  Cada mano queda descrita por:
  \begin{itemize}
  \item
    El valor del primer par,
    puede elegirse de \(13\) maneras.
  \item
    Las pintas del primer par,
    se toman \(2\) de entre \(4\).
  \item
    El valor del segundo par,
    que se puede elegir de \(12\) maneras.
  \item
    Las pintas del segundo par,
    se eligen \(2\) entre \(4\).
  \item
    El valor de la carta extra,
    es uno de \(11\).
  \item
    La pinta de la carta extra,
    que es una de \(4\).
  \end{itemize}
  Se pensaría entonces que el número buscado es:
  \begin{equation*}
    13 \cdot \binom{4}{2} \cdot 12 \cdot \binom{4}{2}
     \cdot 11 \cdot \binom{4}{1}
  \end{equation*}
  ¡Esto es incorrecto!
  El mapa entre secuencias y manos no es una biyección,
  es \(2\) a \(1\)
  (podemos elegir cualquiera de los dos pares como primero).
  El valor correcto es:
  \begin{equation*}
    \frac{13 \cdot \binom{4}{2} \cdot 12 \cdot \binom{4}{2}
	    \cdot 11 \cdot \binom{4}{1}}{2}
      = 123\,552
  \end{equation*}
  No es una mano particularmente buena.

  Pero además es perturbadora:
  Es fácil omitir el detalle de que no es una biyección.
  Hay dos salidas:
  \begin{enumerate}
  \item
    Cada vez que se usa
    un mapa \(f \colon \mathcal{A} \rightarrow \mathcal{B}\),
    verifique que el mismo número de elementos de \(\mathcal{A}\)
    llevan a cada elemento de \(\mathcal{B}\);
    si este número es \(k\),
    aplique la regla de división con \(k\).%
      \index{combinatoria!regla de division@regla de división}
  \item
    Intente otra forma de resolver el problema.
    Muchas veces hay varias formas de enfrentarlo --
    y debieran dar el mismo resultado.
    Claro que suele ocurrir que métodos distintos
    dan resultados que se \emph{ven} diferentes,
    aunque resultan ser iguales.
  \end{enumerate}

  Arriba usamos un método,
  veamos un segundo:
  Hay una biyección entre estas manos y secuencias que especifican:
  \begin{itemize}
  \item
    Los valores de los dos pares,
    se pueden elegir \(2\) entre \(13\).
  \item
    Las pintas del par de menor valor,
    se eligen \(2\) entre \(4\).
  \item
    Las pintas del par de mayor valor,
    se eligen \(2\) entre \(4\).
  \item
    El valor de la carta extra,
    es \(1\) entre \(11\).
  \item
    La pinta de la carta extra,
    es \(1\) entre \(4\).
  \end{itemize}
  Para nuestro ejemplo:
  \begin{equation*}
    \begin{array}{l@{{} \longleftrightarrow \{}*{4}{c@{\;\;}}c@{\}}}
      (\{3, Q\}, \{\diamondsuit, \heartsuit\},
	 \{\spadesuit, \clubsuit\},
       5, \diamondsuit\}) &
	 3 \diamondsuit & 3 \heartsuit & Q \spadesuit &
	 Q \clubsuit  & 5 \diamondsuit \\
      (\{9, K\}, \{\clubsuit, \heartsuit\},
	 \{\spadesuit, \heartsuit\},
       2, \spadesuit\})	  &
	 9 \clubsuit	& 9 \heartsuit & K \spadesuit &
	 K \diamondsuit & 2 \spadesuit
    \end{array}
  \end{equation*}
  Esto lleva a:
  \begin{equation*}
    \binom{13}{2} \cdot \binom{4}{2} \cdot \binom{4}{2} \cdot 11
      \cdot \binom{4}{1}
  \end{equation*}
  Es el mismo resultado anterior,
  claro que escrito de forma ligeramente diferente.

\subsection{Manos con todas las pintas}
\label{sec:todas-las-pintas}

  Buscamos el número de manos con cartas de todas las pintas.
  Por ejemplo:
  \begin{equation*}
    \begin{array}{@{\{}*{4}{c@{\;\;}}c@{\}}}
      7 \heartsuit & 8 \diamondsuit & K \clubsuit &
	A \spadesuit & 3 \heartsuit
    \end{array}
  \end{equation*}
  Esto podemos describirlo mediante:
  \begin{itemize}
  \item
    Los valores de las cartas de cada pinta,
    o sea \(13 \cdot 13 \cdot 13 \cdot 13\) posibilidades.
  \item
    El valor de la carta extra,
    con \(12\) selecciones posibles.
  \item
    La pinta de la carta extra,
    \(4\) opciones.
  \end{itemize}
  La mano del ejemplo se describe mediante:
  \begin{equation*}
    \begin{array}{l@{{} \longleftrightarrow \{}*{4}{c@{\;\;}}c@{\}}}
      (A, 7, 8, K, 3, \heartsuit) &
	 7 \heartsuit & 8 \diamondsuit & K \clubsuit &
	 A \spadesuit & 3 \heartsuit
    \end{array}
  \end{equation*}
  El problema es nuevamente que esto no es una biyección,
  en el ejemplo podemos considerar \(3\heartsuit\) o \(7\heartsuit\)
  como la carta extra,
  y el mapa es \(2\) a \(1\).
  El número buscado es:
  \begin{equation*}
    \frac{13^4 \cdot 4 \cdot 12}{2}
      = 685\,464
  \end{equation*}

  Una forma alternativa
  es dar los valores del par de la misma pinta,
  y la pinta del par;
  y luego los valores de las tres cartas de las pintas restantes.
  Nuestro ejemplo se describe mediante:
  \begin{equation*}
    \begin{array}{l@{{} \longleftrightarrow \{}*{4}{c@{\;\;}}c@{\}}}
      (\{3, 7\}, \heartsuit, A, K, 8) &
	 7 \heartsuit & 8 \diamondsuit & K \clubsuit &
	 A \spadesuit & 3 \heartsuit
    \end{array}
  \end{equation*}
  Acá hemos supuesto
  el orden \(\spadesuit, \heartsuit, \clubsuit, \diamondsuit\)
  de las pintas.
  Esto da nuevamente:
  \begin{equation*}
    \binom{13}{2} \cdot \binom{4}{1} \cdot 13 \cdot 13 \cdot 13
      = 685\,464
  \end{equation*}

\subsection{Manos con valores diferentes}
\label{sec:poker-different-values}

  Nos interesa ahora contar el número de manos
  en las cuales todos los valores son diferentes.
  Una forma alternativa de describir estas manos
  es diciendo que no tienen pares.
  Veremos varias maneras para obtener esto.

  Una primera forma de enfrentar esto
  es considerar
  que la primera carta se puede elegir de entre \(52\),
  la segunda entre las \(48\) que no tienen el valor de la primera,
  y así sucesivamente.
  Pero el hablar de la primera, segunda y sucesivas cartas
  presupone orden,
  alerta del riesgo de contar demás.%
    \index{combinatoria!sobrecontar}
  Como son todas diferentes,
  basta dividir por el número de ordenamientos de \(5\) cartas,
  vale decir \(5! = 120\).
  O sea:
  \begin{equation*}
    \frac{52 \cdot 48 \cdot 44 \cdot 40 \cdot 36}{5!}
      = 1\,317\,888
  \end{equation*}

  Una solución alternativa
  consiste en seleccionar los valores de las \(5\) cartas,
  entre los \(13\) valores posibles,
  y luego a cada carta asignarle una pinta entre \(4\).
  Esto da:
  \begin{equation*}
    \binom{13}{5} \cdot 4^5
      = 1\,317\,888
  \end{equation*}

%%% Local Variables:
%%% mode: latex
%%% TeX-master: "clases"
%%% End:


% tao-bookkeeper.tex
%
% Copyright (c) 2009, 2011-2014 Horst H. von Brand
% Derechos reservados. Vea COPYRIGHT para detalles

\section{El tao de \texttt{BOOKKEEPER}}
\label{sec:tao-bookkeeper}
\index{Tao de \texttt{BOOKKEEPER}|see{combinatoria!secuencias con repeticiones}}
\index{combinatoria!secuencias con repeticiones|textbfhy}

  Veremos maneras de contar secuencias que incluyen elementos repetidos.
  Para llegar a la iluminación
  siguiendo los pasos de Lehman, Leighton y Meyer~%
    \cite{lehman15:_mathem_comput_scien},
  meditemos sobre la palabra \(\mathtt{BOOKKEEPER}\).
  \begin{enumerate}
  \item
    ¿De cuántas maneras se pueden ordenar las letras de \(\mathtt{POKE}\)?
  \item
    ¿De cuántas maneras se pueden ordenar las letras de
    \(\mathtt{B} \mathtt{O}_1 \mathtt{O}_2 \mathtt{K}\)?
    (Note que los subíndices
     hacen que las dos \(\mathtt{O}\) sean distintas).
  \item
    Pequeño saltamontes,
    mapea los ordenamientos de
    \(\mathtt{B} \mathtt{O}_1 \mathtt{O}_2 \mathtt{K}\)
    (las \(\mathtt{O}\) son diferentes)
    a \(\mathtt{BOOK}\)
    (las dos \(\mathtt{O}\) son idénticas).
    ¿Qué clase de mapa es este?
  \item
    ¡Muy bien,
    joven maestro!
    Dime ahora,
    ¿de cuántas maneras pueden ordenarse las letras de
    \(\mathtt{K} \mathtt{E}_1 \mathtt{E}_2 \mathtt{P}
      \mathtt{E}_3 \mathtt{R}\)?
  \item
    Mapea cada ordenamiento de
    \(\mathtt{K} \mathtt{E}_1 \mathtt{E}_2 \mathtt{P}
      \mathtt{E}_3 \mathtt{R}\)
    a un ordenamiento de \(\mathtt{KEEPER}\)
    tal que,
    borrando los subíndices,
    lista todos los que leen \(\mathtt{REPEEK}\).
    ¿Que clase de mapa es este?
  \item
    En vista de lo anterior,
    ¿cuántos ordenamientos de
    \(\mathtt{\foreignlanguage{english}{KEEPER}}\) hay?
  \item
    \emph{¡Ahora ya estás en posición de enfrentarte
      al terrible \(\mathtt{BOOKKEEPER}\)!}
    ¿Cuántos ordenamientos de
    \(\mathtt{B} \mathtt{O}_1 \mathtt{O}_2 \mathtt{K}_1
      \mathtt{K}_2 \mathtt{E}_1 \mathtt{E}_2 \mathtt{P}
      \mathtt{E}_3 \mathtt{R}\)
    hay?
  \item
    ¿Cuántos ordenamientos de
    \(\mathtt{BOO} \mathtt{K}_1
      \mathtt{K}_2 \mathtt{E}_1 \mathtt{E}_2 \mathtt{P}
      \mathtt{E}_3 \mathtt{R}\)
    hay?
  \item
    ¿Cuántos ordenamientos de
    \(\mathtt{BOOKKEEPER}\) hay?
  \item
    ¿Cuántos ordenamientos de
    \(\mathtt{VOODOODOLL}\) hay?
  \item
    Esta es muy importante,
    pequeño saltamontes.
    ¿Cuántas secuencias de \(n\) bits
    tienen \(k\) ceros y \(n - k\) unos?
  \end{enumerate}
  Prender subíndices,
  apagar subíndices.
  Ese es el tao de \(\mathtt{BOOKKEEPER}\).%
    \index{coeficiente multinomial}

%%% Local Variables:
%%% mode: latex
%%% TeX-master: "clases"
%%% End:


% juegos-completos.tex
%
% Copyright (c) 2011-2014 Horst H. von Brand
% Derechos reservados. Vea COPYRIGHT para detalles

\section{Juegos completos de poker}
\label{sec:poker-juegos}
\index{juegos de poker|see{combinatoria!juegos de poker}}
\index{combinatoria!juegos de poker|textbfhy}

  Los señores George G.~Akeley,
  Robert Blake,
  Randolph Carter
  y Edward P.~Davis
  juegan poker.
  Interesa saber
  de cuántas maneras se pueden repartir las \(52\)~cartas
  en \(4\)~manos de \(5\)~cartas,
  quedando \(32\)~cartas en el mazo.

  Podemos atacar el problema considerando que Akeley
  elige \(5\) cartas de las \(52\),
  que Blake elige \(5\) de las restantes,
  y así sucesivamente.
  El resultado es:
  \begin{align*}
    \binom{52}{5}
	\cdot \binom{52 - 5}{5}
	\cdot \binom{52 - 2 \cdot 5}{5}
	\cdot \binom{52 - 3 \cdot 5}{5}
      &= \frac{52!}{5! \, 5! \, 5! \, 5! \, 32!} \\
      &= \binom{52}{5 \; 5 \; 5 \; 5 \; 32}
  \end{align*}

  Si consideramos las cartas en un orden cualquiera,
  podemos representar la distribución
  asociando cada posición con quien la tiene.
  De esta forma,
  tenemos una biyección
  entre secuencias de \(52\)~dueños de las cartas respectivas
  y las distribuciones de las cartas.
  Para simplificar notación,
  denotamos a los caballeros
  por las primeras letras de sus apellidos,
  y el mazo por \(\mathtt{M}\).
  Buscamos entonces el número de secuencias de \(52\)~símbolos
  elegidos
  entre
    \(\{\mathtt{A}, \mathtt{B}, \mathtt{C}, \mathtt{D},
	\mathtt{M}\}\)
  formadas con \(5\)~\(\mathtt{A}\),
  \(5\)~\(\mathtt{B}\),
  \(5\)~\(\mathtt{C}\),
  \(5\)~\(\mathtt{D}\) y \(32\)~\(\mathtt{M}\).
  Esto nos lleva directamente al resultado anterior
  al aplicar el tao,
  sección~\ref{sec:tao-bookkeeper}.%
    \index{combinatoria!secuencias con repeticiones}

%%% Local Variables:
%%% mode: latex
%%% TeX-master: "clases"
%%% End:


% secuencias-restringidas.tex
%
% Copyright (c) 2009, 2012-2014 Horst H. von Brand
% Derechos reservados. Vea COPYRIGHT para detalles

\section{Secuencias con restricciones}
\label{sec:secuencias-restringidas}
\index{combinatoria!secuencias restringidas}

  También interesa poder contar reordenamientos
  en los cuales hay ciertas restricciones,
  como elementos en posiciones fijas
  o elementos en posiciones fijas relativas entre sí.

  Seguimos con nuestro ejemplo de \(\mathtt{BOOKKEEPER}\).
  \begin{enumerate}
  \item
    ¿De cuántas formas se puede escribir esta palabra
    si las dos \(\mathtt{O}\) siempre están juntas?
  \item
    ¿Cuántas formas hay de ordenar las letras
    si siempre están \(\mathtt{BPR}\) juntas en ese orden?
  \item
    ¿Si \(\mathtt{BPR}\) están juntas,
    pero en cualquier orden?
  \item
    ¿En cuántos aparecen \(\mathtt{BPR}\) en ese orden,
    no necesariamente juntas?
    % Podemos elegir las posiciones de BPR en \binom{10}{3} formas
    % (el orden está predeterminado), las demás por multinomio
  \item
    ¿Cuántas maneras hay de ordenar las letras
    si las \(\mathtt{O}\) están separadas por una letra?
  \item
    ¿Cuántas maneras hay de ordenarlas
    si las \(\mathtt{E}\) están separadas siempre por una letra?
  \item
    ¿De cuántas maneras se pueden ordenar las letras
    si las \(5\) vocales están al principio y las \(5\) consonantes al final?
  \item
    ¿Y si las vocales están en las posiciones impares?
  \item
    ¿Que pasa si las vocales
    ocupan las posiciones \(2\), \(3\), \(6\), \(7\), \(9\)?
  \item
    ¿Cuántos ordenamientos hay
    en los cuales las vocales están todas juntas?
  \item
    ¿Cuántos ordenamientos con \(\mathtt{B}\) en una posición impar hay?
  \item
    ¿Y si solo pedimos una \(\mathtt{O}\) en una posición par?
  \item
    ¿Cuántos ordenamientos hay con las tres \(\mathtt{E}\)
    en posiciones impares?
  \item
    ¿Cuántos órdenes tienen la \(\mathtt{B}\) separadas de la \(\mathtt{R}\)
    por dos letras?
  \item
    ¿Cuántos tienen la \(\mathtt{B}\) separadas de la \(\mathtt{R}\)
    por \(k\) letras?
  \end{enumerate}

  Otra situación,
  que puede enfrentarse mediante nuestra estrategia general
  de construir el objeto de interés
  en fases independientes,%
    \index{combinatoria!regla del producto}
  se presenta si queremos determinar
  el número de maneras de ordenar las letras de \(\mathtt{MISSISSIPPI}\)
  de forma que las vocales siempre estén separadas por consonantes.
  Vemos que hay \(4\) \(\mathtt{I}\),
  lo que deja \(5\) espacios en los cuales distribuir las consonantes.
  Si llamamos \(x_0\) al número de consonantes
  antes de la primera \(\mathtt{I}\),
  \(x_1\) a \(x_3\) al número de consonantes entre \(\mathtt{I}\)
  y finalmente \(x_4\) al número de consonantes
  después de la última \(\mathtt{I}\),
  quedan la ecuación:
  \begin{equation*}
    x_0 + x_1 + x_2 + x_3 + x_4
      = 7
  \end{equation*}
  Restricciones son que \(x_0 \ge 0\),
  \(x_k \ge 1\) para \(1 \le k \le 3\)
  y \(x_4 \ge 0\).
  Si definimos nuevas variables \(y_0 = x_0\),
  \(y_k = x_k - 1\) para \(1 \le k \le 3\)
  e \(y_4 = x_4\),
  queda:
  \begin{equation*}
    y_0 + y_1 + y_2 + y_3 + y_4
      = 4
  \end{equation*}
  lo que nos lleva a contar multiconjuntos:%
    \index{multiconjunto!numero@número}
  El número de soluciones es el número de multiconjuntos
  de \(5\) elementos de los que tomamos \(4\).
  Luego ordenamos el multiconjunto de consonantes
  \(\{\mathtt{M}, \mathtt{S}^4, \mathtt{P}^2\}\),
  distribuyendo las consonantes según los tramos definidos anteriormente.
  Como estas dos decisiones
  (número de consonantes en cada tramo
   y orden de las consonantes)
  son independientes,
  aplicamos la regla del producto:
  \begin{equation*}
    \multiset{5}{4} \cdot \binom{7}{1, 4, 2}
      = 7\,350
  \end{equation*}

% Fixme: Anunciar PIE, etc

%%% Local Variables:
%%% mode: latex
%%% TeX-master: "clases"
%%% End:


%%% Local Variables:
%%% mode: latex
%%% TeX-master: "clases"
%%% End:


% funciones-generatrices.tex
%
% Copyright (c) 2009-2014 Horst H. von Brand
% Derechos reservados. Vea COPYRIGHT para detalles

\chapter{Funciones generatrices}
\label{cha:funciones-generatrices}
\index{funcion@función!generatriz|see{generatriz}}
\index{generatriz|textbfhy}

  Veremos cómo usar series de potencias%
    \index{serie de potencias}
  (una herramienta del análisis,
   vale decir matemáticas de lo continuo)
  para resolver una variedad de problemas discretos.
  La idea de funciones generatrices
  permite resolver muchos problemas
  de forma simple y transparente.
  Incluso cuando no da soluciones puede iluminar,
  indicando relaciones entre problemas
  que a primera vista no son obvias.
  El aplicar herramientas analíticas
  (especialmente la teoría de funciones de variables complejas)%
    \index{analisis complejo@análisis complejo}
  permite deducir resultados
  que de otra forma serían muy difíciles de obtener.

  Nos centramos en aplicaciones y en uso de las técnicas discutidas
  más que en exponer la teoría,
  nuestros ejemplos frecuentemente
  llevan a resultados de interés independiente.

\section{Detalles adicionales}
\label{sec:detalles-adicionales}

  Para detalles de la teoría y aplicaciones adicionales
  véanse por ejemplo a Flajolet y Segdewick~%
    \cite{flajolet09:_analy_combin}
  o Wilf~%
    \cite{wilf06:_gfology},
  mientras Kauers~%
    \cite{kauers11:_concr_tetrah}
  se centra en el uso de paquetes de álgebra simbólica
  alrededor de esto.%
    \index{algebra simbolica@álgebra simbólica}
  Referencia obligatoria para todo lo que es combinatoria
  son los textos de Stanley~%
    \cite{stanley12:_enumer_combin-1,
	  stanley99:_enumer_combin-2}.
  Un recurso indispensable
  es la enciclopedia de secuencias de enteros~%
    \cite{sloane:_OEIS},%
    \index{OEIS@\texttt{OEIS}}
  que registra muchos miles de secuencias,
  cómo se generan y da referencias al respecto.

  Sea una secuencia
  \(\left\langle a_n \right\rangle_{n \ge 0}
     = \left\langle
	 a_0, a_1, a_2, \dotsc, a_n, \dotsc
       \right\rangle\).
  La \emph{función generatriz} (ordinaria) de la secuencia es
  la serie de potencias:%
    \index{generatriz!ordinaria}
  \begin{equation}
    \label{eq:definition-gf}
    A(z) = \sum_{n \ge 0} a_n z^n
  \end{equation}
  El punto es que la serie~\eqref{eq:definition-gf}
  representa en forma compacta y manejable
  la secuencia infinita.
  Wilf~\cite{wilf06:_gfology} expresa esto
  diciendo que la función generatriz es una línea de ropa
  de la cual se cuelgan los coeficientes para exhibición.
  Acá entendemos el exponente de \(z\) como un contador,
  índice del coeficiente correspondiente.
  Como veremos,
  operaciones sobre la función generatriz
  corresponden a actuar sobre la secuencia,
  en muchos casos resulta bastante más sencillo manipular la serie
  que trabajar directamente con la secuencia.

  Para un primer ejemplo,
  volveremos al problema
  de la Competencia de Ensayos de la Universidad de Miskatonic
  (ver la sección~\ref{sec:conjetura->teorema}),
  que llevó a la relación:
  \begin{equation}
    \label{eq:recurrence-UMEC-pre}
    b_{2 r + 1} = b_{2 r - 1} + r + 1 \qquad b_1 = 1
  \end{equation}
  con lo que tenemos,
  como antes
  (arbitrariamente dando el valor cero
  a los que no quedan definidos por la recurrencia):
  \begin{equation*}
    \left\langle b_n \right\rangle_{n \ge 0}
      = \left\langle
	  0, 1, 0, 3, 0, 6, 0, 10, 0, 15, \dotsc
	\right\rangle
  \end{equation*}
  Contar con algunos valores sirve para verificar
  (y para ``sentir'' cómo se comportan).

  Para anotar en forma compacta recurrencias%
    \index{recurrencia}
  como~\eqref{eq:recurrence-UMEC-pre}
  usaremos la notación:
  \begin{equation}
    \label{eq:recurrence-UMEC}
    b_{2 r + 1}
      = b_{2 r - 1} + r + 1
      \quad (r \ge 1)
      \qquad b_1 = 1
  \end{equation}
  O sea,
  damos la recurrencia misma,
  los índices para los que la recurrencia vale,
  y valores iniciales.
  Normalmente será
  que la recurrencia vale a partir del término siguiente
  al indicado como valor inicial,
  y omitiremos el rango de validez.

  Definamos la serie
  (note que el subíndice en \(b_{2 r + 1}\) avanza de a dos):
  \begin{equation}
    \label{eq:gf-UMEC}
    B(z) = \sum_{r \ge 0} b_{2 r + 1} z^r
  \end{equation}
  Si multiplicamos la recurrencia~\eqref{eq:recurrence-UMEC}
  por \(z^r\)
  y sumamos sobre \(r \ge 1\)
  (para considerar solo índices positivos)
  queda:
  \begin{equation}
    \label{eq:gf-UMEC-0}
    \sum_{r \ge 1} b_{2 r + 1} z^r
       = \sum_{r \ge 1} b_{2 r - 1} z^r
	   + \sum_{r \ge 1} (r + 1) z^r
  \end{equation}
  Expresando lo anterior en términos de \(B(z)\),
  reconocemos:
  \begin{align}
    \sum_{r \ge 1} b_{2 r + 1} z^r
      &= \sum_{r \ge 0} b_{2 r + 1} z^r - b_1
       = B(z) - 1 \label{eq:suma-b(2r+1)} \\
    \sum_{r \ge 1} b_{2 r - 1} z^r
      &= \sum_{r \ge 0} b_{2 r + 1} z^{r + 1}
       = z \sum_{r \ge 0} b_{2 r + 1} z^r
       = z B(z) \label{eq:suma-b(2r-1)}
  \end{align}
  En la ecuación~\eqref{eq:gf-UMEC-0}
  aparecen términos \((r + 1) z^r\),
  que sugieren la derivada de \(z^{r + 1}\).
  Por el otro lado,
  de la sección~\ref{sec:induccion-comun}
  sabemos que para \(\lvert z \rvert < 1\) vale la serie geométrica:
  \begin{equation}
    \label{eq:serie-geometrica}
    \frac{1}{1 - z} = \sum_{r \ge 0} z^r
  \end{equation}
  Derivando la serie geométrica respecto de \(z\) término a término,
  lo que es válido dentro del radio de convergencia%
    \index{serie de potencias!radio de convergencia}
  (no nos detendremos en este punto,
   para la teoría que lo justifica véanse por ejemplo a Chen~%
    \cite{chen08:_fundam_analy},
   a Trench~\cite{trench13:_introd_real_analy}
   o refiérase al capítulo~\ref{cha:analisis-complejo}),
  queda:
  \begin{align}
    \frac{\mathrm{d}}{\mathrm{d} z}
      \, \left( \frac{1}{1 - z} \right)
      &= \sum_{r \ge 0} \frac{\mathrm{d}}{\mathrm{d} z} \, z^r
	   \notag \\
    \frac{1}{(1 - z)^2}
      &= \sum_{r \ge 1} r z^{r - 1}
       = \sum_{r \ge 0} (r + 1) z^r
	   \label{eq:D-serie-geometrica}
  \end{align}
  También:
  \begin{equation}
    \label{eq:suma-r+1}
    \sum_{r \ge 1} (r + 1) z^r
       = \sum_{r \ge 0} (r + 1) z^r - 1
       = \frac{1}{(1 - z)^2} - 1
  \end{equation}
  Reemplazando~\eqref{eq:suma-b(2r+1)}, \eqref{eq:suma-b(2r-1)}
  y~\eqref{eq:suma-r+1} en~\eqref{eq:gf-UMEC-0} queda:
  \begin{equation*}
    B(z) - 1
      = z B(z) + \frac{1}{(1 - z)^2} - 1
  \end{equation*}
  Despejando \(B(z)\):
  \begin{equation}
    \label{eq:gf-UMEC-1}
    B(z) = \frac{1}{(1 - z)^3}
  \end{equation}
  Para algunas aplicaciones basta llegar hasta acá,
  puede extraerse bastante información sobre los coeficientes
  de la función.
  Por ejemplo,
  claramente la serie~\eqref{eq:gf-UMEC-1}
  converge para \(\lvert z \rvert < 1\),
  y por la prueba de la razón,%
    \index{serie de potencias!prueba de la razon@prueba de la razón}
  sabemos que:
  \begin{equation*}
    \lim_{r \rightarrow \infty}
	  \left\lvert
	    \frac{b_{2 r + 1}}{b_{2 r + 3}}
	  \right\rvert = 1
  \end{equation*}

  Pero interesa obtener una fórmula explícita
  (ojalá simple)
  para los coeficientes,
  de forma de poder determinar el tamaño requerido de las tarjetas.
  Derivando la serie geométrica por segunda vez,
  de forma de obtener la expresión \((1 - z)^{-3}\),
  resulta:
  \begin{align}
    \frac{\mathrm{d}^2}{\mathrm{d} z^2}
      \, \left( \frac{1}{1 - z} \right)
      &= \sum_{r \ge 1}
	   \frac{\mathrm{d}}{\mathrm{d} z} \, r z^{r - 1}
	   \notag \\
    \frac{2}{(1 - z)^3}
      &= \sum_{r \ge 2} r (r - 1) z^{r - 2}
       = \sum_{r \ge 0} (r + 2) (r + 1) z^r
	   \label{eq:DD-serie-geometrica}
  \end{align}
  Con~\eqref{eq:DD-serie-geometrica} y~\eqref{eq:gf-UMEC-1} resulta:
  \begin{equation}
    \label{eq:gf-UMEC-2}
    B(z)
      = \sum_{r \ge 0} b_{2 r + 1} z^r
      = \frac{1}{2} \, \sum_{r \ge 0} (r + 2) (r + 1) z^r
  \end{equation}
  Comparando coeficientes
  tenemos nuevamente
  la fórmula explícita~\eqref{eq:ensayos-valor-explicito}:
  \begin{equation*}
    b_{2 r + 1} = \frac{1}{2} \, (r + 2) (r + 1)
  \end{equation*}
  La ventaja frente al desarrollo
  de la sección~\ref{sec:conjetura->teorema}
  es que no tuvimos que ``adivinar'' esta solución
  y nos ahorramos la demostración por inducción.
  Nótese además que el valor de \(B(z)\)
  jamás fue del más mínimo interés en el desarrollo.

\subsubsection*{Receta}
\index{recurrencia!receta}

  Para resolver recurrencias se debe:
  \begin{enumerate}
  \item\label{item:plantear}
    Plantear la recurrencia.
  \item
    Recopilar valores iniciales.
  \item\label{item:valores}
    Aclarar para qué valores del índice vale la recurrencia.
  \item\label{item:GF}
    Definir la función generatriz de interés.
  \item\label{item:ecuacion}
    Multiplicar la recurrencia de~(\ref{item:plantear})
    por \(z^n\)
    y sumar sobre todos los valores~(\ref{item:valores}).
  \item
    Expresar (\ref{item:ecuacion})
    en términos de la función generatriz~(\ref{item:GF}).
  \item
    Despejar la función generatriz de~(\ref{item:ecuacion}).
  \item
    Extraer los coeficientes.
  \end{enumerate}

\section{Algunas series útiles}
\label{ref:series-utiles}
\index{serie de potencias!series utiles@series útiles}

  Al trabajar con funciones generatrices
  es importante tener algunas expansiones en serie
  conocidas a la mano.
  Las que más aparecen son las siguientes.

\subsection{Serie geométrica}
\label{sec:serie-geometrica}
\index{serie de potencias!geometrica@geométrica}

  Es la serie más común en aplicaciones.
  Si \(\lvert z \rvert < 1\),
  se cumple:
  \begin{equation}
    \label{eq:serie-geometrica-b}
    \sum_{n \ge 0} z^n
      = \frac{1}{1 - z}
  \end{equation}
  Una variante importante es la siguiente,
  expansión válida para \(\lvert a z \rvert < 1\)
  (siempre que usemos la convención \(0^0 = 1\)):
  \begin{equation}
    \label{eq:serie-geometrica-c}
    \sum_{n \ge 0} a^n z^n
      = \frac{1}{1 - a z}
  \end{equation}

\subsection{Teorema del binomio}
\label{sec:teorema-binomio}
\index{serie de potencias!binomio}

  Una de las series más importantes
  es la expansión de la potencia de un binomio
  (ver también el teorema~\ref{theo:binomio}):
  \begin{equation}
    \label{eq:serie-binomio}
    \sum_{n \ge 0} \binom{\alpha}{n} \, z^n
       = (1 + z)^\alpha
  \end{equation}
  Siempre que \(\lvert z \rvert < 1\)
  esto vale no solo para valores reales de \(\alpha\),
  sino incluso para \(\alpha\) complejos,
  si definimos:
  \begin{equation}
    \label{eq:coeficiente-binomial}
    \binom{\alpha}{k}
       = \frac{\alpha}{1} \cdot \frac{\alpha - 1}{2}
	    \cdot \frac{\alpha - 2}{3}
	    \cdot \dots
	    \cdot \frac{\alpha - k + 1}{k}
       = \frac{\alpha^{\underline{k}}}{k!}
  \end{equation}
  y (consistente con la convención que productos vacíos son \(1\))
  siempre es:
  \begin{equation}
    \label{eq:binomial(alpha,0)}
    \binom{\alpha}{0}
      = 1
  \end{equation}
  A los coeficientes~\eqref{eq:coeficiente-binomial}
  se les conoce como \emph{coeficientes binomiales}
  por su conexión con la potencia de un binomio.
  La expansión~\eqref{eq:serie-binomio}
  (también conocida como \emph{fórmula de Newton}
   si \(\alpha\) no es un natural)
  es fácil de demostrar por el teorema de Maclaurin.
  Incluso resulta que~\eqref{eq:serie-geometrica-b} es simplemente
  un caso particular de~\eqref{eq:serie-binomio}.

  Si \(\alpha\) es un entero positivo,
  la serie~\eqref{eq:serie-binomio} se reduce a un polinomio
  y la relación es válida para todo \(z\).
  Además,
  en caso que \(n \in \mathbb{N}\)
  podemos escribir:
  \begin{equation}
    \label{eq:coeficiente-binomial-factorial}
    \binom{n}{k}
       = \frac{n!}{k! (n - k)!}
  \end{equation}
  Es claro que:
  \begin{equation}
    \label{eq:coeficiente-binomial-contorno}
    \binom{n}{k}
      = 0 \text{\ si \(k < 0\) o \(k > n\)}
  \end{equation}
  Esto con~\eqref{eq:coeficiente-binomial-factorial}
  sugiere la convención:
  \begin{equation}
    \label{eq:1/k!-convention}
    \frac{1}{k!}
      = 0 \quad \text{si \(k < 0\)}
  \end{equation}
  Nótese la simetría:
  \begin{equation}
    \label{eq:coeficiente-binomial-simetria}
    \binom{n}{k}
      = \binom{n}{n - k}
  \end{equation}

  Casos especiales notables de coeficientes binomiales
  para \(\alpha \notin \mathbb{N}\) son los siguientes:
  \begin{description}
  \item[\boldmath Caso \(\alpha = 1 / 2\):\unboldmath]
    Tenemos,
    como siempre:
    \begin{equation}
      \label{eq:binomial(1/2,0)}
      \binom{1/2}{0}
	= 1
    \end{equation}
    Cuando \(k \ge 1\):
    \begin{align}
      \binom{1/2}{k}
	 &= \frac{\frac{1}{2} \cdot (\frac{1}{2}-1)
	       \dotsm (\frac{1}{2} - k + 1)}{k!} \notag \\
	 &= \frac{1}{2^k}
	       \cdot \frac{1 \cdot (1 - 2) \cdot (1 - 4)
			     \dotsm (1 - 2 k + 2)}{k!} \notag \\
	 &= \frac{(-1)^{k - 1}}{2^k k!}
	       \cdot (1 \cdot 3 \dotsm (2 k - 3)) \notag \\
	 &= \frac{(-1)^{k - 1}}{2^k k!}
	       \cdot \frac{1 \cdot 2 \cdot 3 \cdot 4
			      \cdot \dotsm
			      \cdot (2 k - 3) \cdot (2 k - 2)}
			  {2 \cdot 4 \cdot 6 \dotsm (2 k - 2)}
				\notag \\
	 &= \frac{(-1)^{k - 1}}{2^k k!}
	       \cdot \frac{(2 k - 2)!}{2^{k - 1} (k - 1)!}
		  \notag \\
	 &= \frac{(-1)^{k - 1}}{2^{2 k - 1} \cdot k}
	       \cdot \frac{(2 k - 2)!}{(k - 1)! \, (k - 1)!}
		  \notag \\
	 &= \frac{(-1)^{k - 1}}{2^{2 k - 1} \cdot k}
	       \cdot \binom{2 k - 2}{k - 1}
	    \label{eq:binomial(1/2,k)}
    \end{align}
    Hay que tener cuidado,
    la última fórmula no cubre el caso \(k = 0\).

    Una serie común es:
    \begin{align}
      \frac{1 - \sqrt{1 - 4 z}}{2 z}
	&= \frac{1}{2 z} \,
	     \left(
	       1 - \sum_{n \ge 0} \binom{1 / 2}{n} \, (-4 z)^n
	     \right) \notag \\
	&= \frac{1}{2 z} \,
	     \left(
	       1 - \left(
		     1 + \sum_{n \ge 1}
			   \frac{(-1)^{n - 1}}{n 2^{2 n - 1}} \,
			     \binom{2 n - 2}{n - 1} \, (-4 z)^n
		   \right)
	     \right) \notag \\
	&= \frac{1}{2 z}
	     \cdot 4 z \, \sum_{n \ge 0}
			    \frac{1}{2 (n + 1)}
			       \, \binom{2 n}{n} \, z^n
		\notag \\
	&= \sum_{n \ge 0} \frac{1}{n + 1} \, \binom{2 n}{n} \, z^n
	     \label{eq:gf-Catalan}
	     \index{Catalan, numeros de@Catalan, números de!generatriz|textbfhy}
    \end{align}
    Los coeficientes de~\eqref{eq:gf-Catalan} se conocen como
    \emph{números de Catalan}:%
      \index{Catalan, numeros de@Catalan, números de}%
      \index{Catalan, numeros de@Catalan, números de!formula@fórmula}%
    \begin{equation}
      \label{eq:Catalan-numbers}
      C_n
	= \frac{1}{n + 1} \, \binom{2 n}{n}
    \end{equation}
    La serie~\eqref{eq:gf-Catalan} aparece con regularidad,
    al igual que los coeficientes~\eqref{eq:Catalan-numbers}.
    Stanley~%
      \cite{stanley99:_enumer_combin-2,
	    stanley13:_catalan_addendum}
    lista un total de \(205\) interpretaciones combinatorias
    de los números de Catalan.
    Se la ha llamado la función generatriz más famosa de la combinatoria.
  \item[\boldmath Caso \(\alpha = -1/2\):\unboldmath]
    Mucha de la derivación es similar a la del caso anterior.
    Tenemos,
    para \(k > 0\):
    \begin{align}
      \binom{-1/2}{k}
	&= \frac{(-1/2) \cdot (-1/2 - 1) \cdot \dotsm
		   \cdot (-1/2 - k + 1)}
		{k!} \notag \\
	&= (-1)^k \frac{1}{2^k}
	     \cdot \frac{1 \cdot 3 \dotsm (2 k - 1)}{k!} \notag \\
	&= (-1)^k \frac{1}{2^k}
	     \cdot \frac{(2 k)!}{k! \, 2^k \, k!} \notag \\
	&= (-1)^k \frac{1}{2^{2 k}} \, \binom{2 k}{k}
	    \label{eq:binomial(-1/2,k)}
    \end{align}
    Esta fórmula con \(k = 0\) da:
    \begin{equation*}
      \binom{-1/2}{0} = 1
    \end{equation*}
    así no se necesita hacer un caso especial acá.

    Una expansión común es:
    \begin{equation}
      \label{eq:serie-reciproco-raiz}
      \frac{1}{\sqrt{1 - 4 z}}
	= \sum_{n \ge 0} \binom{2 n}{n} \, z^n
    \end{equation}
  \item[\boldmath Caso \(\alpha = -n\):\unboldmath]
    Cuando \(\alpha\) es un entero negativo,
    podemos escribir:
    \begin{equation}
      \label{eq:binomial(-n,k)}
      \binom{-n}{k}
	= \frac{(-n)^{\underline{k}}}{k!}
	= (-1)^k \, \frac{n^{\overline{k}}}{k!}
	= (-1)^k \, \frac{(n + k - 1)^{\underline{k}}}{k!}
	= (-1)^k \, \binom{k + n - 1}{n - 1}
    \end{equation}
    Nótense los casos particulares
    (aparecieron en nuestra derivación de la solución
     para la Competencia de Ensayos
     de la Universidad de Miskatonic):
    \begin{align*}
      \binom{-2}{k}
	&= (-1)^k \, \binom{k + 1}{1}
	 = (-1)^k (k + 1) \\
      \binom{-3}{k}
	&= (-1)^k \, \binom{k + 2}{2}
	 = (-1)^k \, \frac{(k + 1) (k + 2)}{2}
    \end{align*}
    Estos coeficientes binomiales
    son polinomios de grado \(n - 1\) en \(k\).

    En general,
    resulta:
    \begin{equation}
      \label{eq:serie-binomio-negativo}
      \frac{1}{(1 - z)^{n + 1}}
	= \sum_{k \ge 0} \binom{n + k}{n} \, z^k
    \end{equation}
  \end{description}
  Un par de series útiles son las sumas dobles:
  \begin{align}
    \label{eq:sum-binomial-double}
    \sum_{n, k} \binom{n}{k} \, x^k y^n
      &= \sum_{n} (1 + x)^n y^n
       = \frac{1}{1 - (1 + x) y} \\
    \label{eq:sum-multiset-double}
    \sum_{n, k} \multiset{n}{k} \, x^k y^n
      &= \sum_{n} \frac{y^n}{(1 - x)^n}
       = \frac{1 - x}{1 - x - y}
  \end{align}
  En~\eqref{eq:sum-multiset-double}
  usamos la identidad~\eqref{eq:multiset=negative-binomial}.

  Interesante resulta la serie:
  \begin{equation*}
    \sum_{n \ge 0} \binom{n}{k} \, z^n
  \end{equation*}
  Como \(n\) es un entero no-negativo,
  sabemos que \(\binom{n}{k} = 0\) si no es que \(0 \le k \le n\),
  podremos ahorrarnos los límites de las sumas para simplificar:
  \begin{align}
    \sum_n \binom{n}{k} \, z^n
      &= \sum_n \binom{n + k}{k} \, z^{n + k} \notag \\
      &= z^k \sum_n \binom{n + k}{n} \, z^n \notag \\
      &= \frac{z^k}{(1 - z)^{k + 1}}
	    \label{eq:serie-binomio-n}
  \end{align}
  Al final usamos~\eqref{eq:serie-binomio-negativo}.
  Omitir los rangos de los índices ahorró interminables ajustes.

  Para multiconjuntos,%
    \index{multiconjunto!generatriz}
  usando~(\ref{eq:sum-multiset-double}):
  \begin{align*}
    \sum_{n \ge 0} \multiset{n}{k} z^n
      &= \left[ x^k \right] \frac{1 - x}{1 - x - z} \\
      &= \frac{1}{1 - z}
	   \left[ x^k \right] \frac{1 - x}{1 - x / (1 - z)} \\
      &= \frac{1}{1 - z}
	   \left[ x^k \right] (1 - x)
	     \sum_{n \ge 0} \frac{x^n}{(1 - z)^n}
	      \\
      &= \frac{1}{1 - z}
	   \left(
	     \frac{1}{(1 - z)^k} - \frac{[k > 0]}{(1 - z)^{k - 1}}
	   \right)  \\
      &= \frac{1 - [k > 0] (1 - z)}{(1 - z)^{k + 1}} \\
      &= \frac{(1 - [k > 0]) + [k > 0] z}{(1 - z)^{k + 1}}
  \end{align*}
  Como el numerador es \(1\) si \(k = 0\) y \(z\) cuando \(k > 0\)
  podemos simplificar:
  \begin{equation}
    \label{eq:serie-multiset-n}
    \sum_{n \ge 0} \multiset{n}{k} z^n
      = \frac{z^{[k > 0]}}{(1 - z)^{k + 1}}
  \end{equation}

\subsection{Otras series}
\label{sec:otras-series}

  Una serie común es la exponencial:%
    \index{serie de potencias!exponencial}
  \begin{equation}
    \label{eq:exponencial}
    \mathrm{e}^z
      = \sum_{n \ge 0} \frac{z^n}{n!}
  \end{equation}
  con sus variantes:
  \begin{equation*}
    \mathrm{e}^{a z}
      = \sum_{n \ge 0} \frac{a^n z^n}{n!} \hspace{7em}
    \mathrm{e}^{-z}
      = \sum_{n \ge 0} \frac{(-1)^n z^n}{n!}
  \end{equation*}
  A veces aparecen funciones trigonométricas:%
    \index{serie de potencias!seno}%
    \index{serie de potencias!coseno}
  \begin{equation*}
    \sin z
      = \sum_{n \ge 0} (-1)^n \frac{z^{2 n + 1}}{(2 n + 1)!} \qquad
    \cos z
      = \sum_{n \ge 0} (-1)^n \frac{z^{2 n}}{(2 n)!}
  \end{equation*}
  o hiperbólicas:%
    \index{serie de potencias!seno hiperbolico@seno hiperbólico}%
    \index{serie de potencias!coseno hiperbolico@coseno hiperbólico}
  \begin{equation*}
    \sinh z
      = \sum_{n \ge 0} \frac{z^{2 n + 1}}{(2 n + 1)!} \qquad
    \cosh z
      = \sum_{n \ge 0} \frac{z^{2 n}}{(2 n)!}
  \end{equation*}
  Una relación útil es la fórmula de Euler:%
    \index{Euler, formula de (exponencial complejo)@Euler, fórmula de (exponencial complejo)}
  \begin{equation}
    \label{eq:formula-Euler-exponencial}
    \mathrm{e}^{u + \mathrm{i} v}
      = \mathrm{e}^u (\cos v + \mathrm{i} \sin v)
  \end{equation}

  Es frecuente la serie para el logaritmo:%
    \index{serie de potencias!logaritmo}
  \begin{align}
    \frac{\mathrm{d}}{\mathrm{d} z} \, \ln (1 - z)
      &= - \frac{1}{1 - z}
       = - \sum_{n \ge 0} z^n \notag \\
    \ln (1 - z)
      &= - \sum_{n \ge 1} \frac{z^n}{n}
	   \label{eq:ln(1-z)}
  \end{align}
  Muchos ejemplos adicionales de series útiles
  se hallan en el texto de Wilf~\cite{wilf06:_gfology}.

\section{Notación para coeficientes}
\label{sec:funciones-generatrices:notacion}
\index{serie de potencias!extraer coeficiente}

  Comúnmente extraeremos el coeficiente de un término de una serie.
  Para esto,
  dadas las series:
  \begin{equation*}
    A(z)
      = \sum_{n \ge 0} a_n z^n
    \hspace{3em}
    B(z)
      = \sum_{n \ge 0} b_n z^n
  \end{equation*}
  usaremos la notación:
  \begin{equation*}
     \left[ z^n \right] A(z) = a_n
  \end{equation*}
  Tenemos algunas propiedades simples:
  \begin{equation*}
    \left[ z^n \right] z^k A(z)
      = \left[ z^{n - k} \right] A(z)
  \end{equation*}
  Una vez dado cuenta de \(z^k\),
  queda por extraer el coeficiente de \(z^{n - k}\) de \(A(z)\).
  \begin{equation*}
    \left[ z^n \right] (\alpha A(z) + \beta B(z))
      = \alpha \left[ z^n \right] A(z)
	  + \beta \left[ z^n \right] B(z)
  \end{equation*}
  Generalmente no hay términos con potencias negativas de \(z\),
  tales coeficientes serán cero.

  En términos de esta notación
  el teorema de Maclaurin%
    \index{Maclaurin, teorema de}
  queda expresado como:
  \begin{equation*}
    \left[ z^n \right] \, A(z)
      = \frac{1}{n!} \, A^{(n)}(0)
  \end{equation*}
  La notación es de Goulden y Jackson~%
    \cite{goulden04:_combin_enumer}.
  Puede extenderse muchísimo,
  ver Knuth~\cite{knuth94:_brack_notat_coeff_operat}
  y Merlini, Sprugnoli y Verri~%
    \cite{merlini07:_method_coeff}.
  La idea se le atribuye a Egorychev~%
  \cite{egorychev84:_integ_repres_comput_combin_sums},
  aunque con una notación mucho más engorrosa.

  Consideremos secuencias \(\langle a_n \rangle_{n \ge 0}\)
  y \(\langle a_n \rangle_{n \ge 0}\)
  relacionadas por:
  \begin{equation}
    \label{eq:b=binomial-transform-a}
    \sum_{0 \le k \le n} \binom{n}{k} a_k
      = b_n
  \end{equation}
  Si multiplicamos ambos lados por \(z^n / n!\)
  y sumamos sobre \(n \ge 0\)
  resulta:
  \begin{align*}
    \mathrm{e}^z \cdot \sum_{n \ge 0} a_n \frac{z^n}{n!}
      &= \sum_{n \ge 0} b_n \frac{z^n}{n!} \\
    \sum_{n \ge 0} a_n \frac{z^n}{n!}
      &= \mathrm{e}^{-z} \cdot \sum_{n \ge 0} b_n \frac{z^n}{n!}
  \end{align*}
  Comparar coeficientes entrega:
  \begin{equation}
    \label{eq:a=inverse-binomial-transform-b}
    a_n
      = \sum_{0 \le k \le n} \binom{n}{k} (-1)^k b_k
  \end{equation}

  \begin{theorem}[Transformación de Euler]
    \index{Euler, transformacion de@Euler, transformación de|textbfhy}
    \label{theo:Euler-transformation}
    Sea \(A(z) = \sum a_n z^n\).
    Entonces:
    \begin{equation}
      \label{eq:Euler-transformation}
       \sum_{0 \le k \le n} \binom{n}{k} \, a_k
	 = \left[ z^n \right] \,
	     \frac{1}{1 - z} \,
	       A \left(
		   \frac{z}{1 - z}
		 \right)
    \end{equation}
  \end{theorem}
  \begin{proof}
    Como para \(k > n\) el coeficiente binomial se anula,
    podemos extender la suma a todo \(k \ge 0\).
    Consideremos:
    \begin{align*}
      \sum_{n \ge 0} z^n \sum_{k \ge 0} \binom{n}{k} a_k
	&= \sum_{k \ge 0} a_k \sum_{n \ge 0} \binom{n}{k} z^n \\
	&= \sum_{k \ge 0} a_k \frac{z^k}{(1 - z)^{k + 1}} \\
	&= \frac{1}{1 - z} \,
	    \sum_{k \ge 0} a_k \left( \frac{z}{1 - z} \right)^k \\
	&= \frac{1}{1 - z} A \left(  \frac{z}{1 - z} \right)
    \end{align*}
    Esto es equivalente a lo enunciado.
  \end{proof}
  Un ejemplo de la aplicación de la transformación de Euler
  es el tratamiento de una suma
  discutido por Greene y Knuth~\cite{greene10:_math_anal_algor},
  originalmente de Jonassen y Knuth~%
    \cite{jonassen78:_trivial_algorithm}:
  \begin{equation*}
    S_m
      = \sum_{0 \le k \le m}
	  \binom{m}{k} \, \left( - \frac{1}{2} \right)^k \, \binom{2 k}{k}
  \end{equation*}
  Del teorema del binomio sabemos que:
  \begin{equation*}
    \frac{1}{\sqrt{1 + 2 z}}
      = \sum_{n \ge 0}
	  \binom{2 n}{n} \, \left( - \frac{1}{2} \right)^n
  \end{equation*}
  Por la transformación de Euler~\eqref{eq:Euler-transformation}:
  \begin{equation*}
    \sum_{0 \le k \le m}
      \binom{m}{k} \, \binom{2 k}{k}
	\, \left( - \frac{1}{2} \right)^k
      = \left[ z^m \right] \,
	  \frac{1}{1 - z} \,
	    \left( 1 + 2 \frac{z}{1 - z} \right)^{-1/2}
      = \left[ z^m \right] \,
	  \frac{1}{\sqrt{1 - z^2}}
  \end{equation*}
  Resulta:
  \begin{equation*}
    S_m
      = \begin{cases}
	  \displaystyle \binom{2 k}{k} \, 2^{-2 k} & m = 2 k \\
	  0					   & m = 2 k + 1
	\end{cases}
  \end{equation*}
  Prodinger~\cite{prodinger94:_old_sum}
  incluso usa esta suma para mostrar diversas técnicas
  para obtener una fórmula cerrada.

\section{Decimar}
\label{sec:decimar}
\index{serie de potencias!decimar}

  Uno de los máximos castigos para una legión romana era la \emph{decimación},
  que consistía en ejecutar a uno de cada diez miembros.
  Nuestro objetivo acá es mucho más radical,
  aunque bastante menos sangriento.

  Sea una secuencia \(\langle a_n \rangle_{n \ge 0}\),
  con función generatriz ordinaria \(A(z)\).
  Es fácil ver que:
  \begin{align}
    \sum_{n \ge 0} a_{2 n} z^{2 n}
      &= \frac{A(z) + A(-z)}{2}
	     \label{eq:a_even} \\
    \sum_{n \ge 0} a_{2 n + 1} z^{2 n + 1}
      &= \frac{A(z) - A(-z)}{2}
	     \label{eq:a_odd}
  \end{align}
  Esto es útil si nos interesan términos alternos:
  \begin{align}
    A_e(z)
      &= \sum_{n \ge 0} a_{2 n} z^n
	     \notag \\
      &= \frac{A(z^{1/2}) + A(-z^{1/2})}{2}
	     \label{eq:A_even} \\
    A_o(z)
      &= \sum_{n \ge 0} a_{2 n + 1} z^n
	      \notag \\
      &= \frac{A(z^{1/2}) - A(-z^{1/2})}{2 z^{1/2}}
	      \label{eq:A_odd}
  \end{align}
  Interesa extender esto a extraer uno cada \(m\) términos.

  Sea \(\omega_m\) una raíz primitiva de \(1\),%
    \index{raiz primitiva de 1@raíz primitiva de \(1\)}
  o sea por ejemplo el complejo:
  \begin{align}
    \omega_m
      &= \mathrm{e}^{\frac{2  \pi \mathrm{i}}{m}}
	      \notag \\
      &= \cos \frac{2  \pi}{m} + \mathrm{i} \sin \frac{2  \pi}{m}
	      \label{eq:omega_m}
  \end{align}
  De ahora en adelante anotaremos simplemente \(\omega\) para simplificar,
  \(m\) quedará dado por el contexto.
  Los \(m\) ceros del polinomio \(x^m - 1\)
  son \(\omega^k\) para \(0 \le k < m\),
  ya que:
  \begin{align*}
    \omega^k
      &= \mathrm{e}^{\frac{2  k \pi \mathrm{i}}{m}} \\
    \left( \omega^k \right)^m
      &= \mathrm{e}^{\frac{2  m k \pi \mathrm{i}}{m}} \\
      &= \mathrm{e}^{2	k \pi \mathrm{i}} \\
      &= 1
  \end{align*}
  Como \(\omega \ne 1\),
  de la factorización:
  \begin{equation*}
    x^m - 1
      = (x - 1) (x^{m - 1} + x^{m - 2} + \dotsc + 1)
  \end{equation*}
  vemos que:
  \begin{equation*}
    \sum_{0 \le k < m} \omega^k
      = 0
  \end{equation*}
  Resulta la curiosa
  (y útil) identidad:
  \begin{equation}
    \label{eq:sum-omega-powers}
    \sum_{0 \le k < m} \omega^{k s}
      = \begin{cases}
	  0 & \text{si \(m \centernot\mid s\)} \\
	  m & \text{si \(m \mid s\)}
	\end{cases}
  \end{equation}

  En vista de lo anterior,
  consideremos:
  \begin{align*}
    \sum_{0 \le k < m} \omega^{- k r} A(\omega^k z)
      &= \sum_{0 \le k < m} \omega^{- k r}
	   \sum_{n \ge 0} a_n \omega^{k n} z^n \\
      &= \sum_{n \ge 0} a_n z^n \sum_{0 \le k < m} \omega^{k (n - r)}
  \end{align*}
  La suma interna es \(m\) si \(m \mid n - r\),
  \(0\) en caso contrario.
  Con esto podemos construir:
  \begin{equation}
    \label{eq:decimation}
    \sum_{n \ge 0} a_{m n + r} z^{m n + r}
      = \frac{1}{m} \sum_{0 \le k < m} \omega^{-k r} A(\omega^k z)
  \end{equation}
  de donde es sencillo extraer la función generatriz de la secuencia
  \(\langle a_{m n + r} \rangle_{n \ge 0}\).

\section{Algunas aplicaciones combinatorias}
\label{sec:combinatorial-applications}
\index{combinatoria!generatrices}

  Se buscan las formas de llenar un canasto con \(n\) frutas si:
  \begin{itemize}
  \item El número de manzanas tiene que ser par.
  \item El número de plátanos debe ser un múltiplo de \(5\).
  \item Hay a lo más \(4\) naranjas.
  \item Hay a lo más \(1\) sandía.
  \end{itemize}

  Consideremos primero solo manzanas y plátanos.
  Usamos \(z\)
  (a través de sus potencias)
  para contar el número total de frutas,
  \(\langle m_k \rangle_{k \ge 0}\)
  es la secuencia de formas de tener \(k\) manzanas
  mientras \(\langle p_k \rangle_{k \ge 0}\)
  corresponde a los plátanos;
  y sea \(\langle c_k \rangle_{k \ge 0}\)
  la secuencia de las maneras de juntar \(k\) de estas frutas.
  Para cuatro frutas:
  \begin{equation*}
    c_4
      = m_0 \cdot p_4
	 + m_1 \cdot p_3
	 + m_2 \cdot p_2
	 + m_3 \cdot p_1
	 + m_4 \cdot p_0
  \end{equation*}
  Esta es exactamente la forma
  en que calculamos el coeficiente de \(z^4\) en la serie:
  \begin{equation*}
    \sum_{k \ge 0} c_k z^k
      = \left( \, \sum_{k \ge 0} m_k z^k \right)
	  \cdot \left( \, \sum_{k \ge 0} p_k z^k \right)
  \end{equation*}
  Generalizando esta observación,
  la función generatriz para el número de canastos con \(n\) frutas
  es el producto de las funciones generatrices
  para cada tipo de fruta.
  Estas son:
  \begin{itemize}
  \item Para manzanas:
    \begin{equation*}
      1 + z^2 + z^4 + \dotsb
	= \frac{1}{1 - z^2}
    \end{equation*}
  \item Los plátanos se representan por:
    \begin{equation*}
      1 + z^5 + z^{10} + \dotsb
	= \frac{1}{1 - z^5}
    \end{equation*}
  \item Para las naranjas:
    \begin{equation*}
      1 + z + z^2 + z^3 + z^4
	= \frac{1 - z^5}{1 - z}
    \end{equation*}
  \item Las sandías aportan:
    \begin{equation*}
      1 + z
    \end{equation*}
  \end{itemize}
  Uniendo las anteriores,
  la función generatriz del número de formas
  de tener canastos con \(n\)~frutas resulta ser:
  \begin{equation*}
    C(z)
      = \frac{1}{1 - z^2}
	  \cdot \frac{1}{1 - z^5}
	  \cdot \frac{1 - z^5}{1 - z}
	  \cdot (1 + z)
      = \frac{1}{(1 - z)^2}
  \end{equation*}
  Hay
  \((-1)^n \binom{-2}{n} = \binom{n + 1}{1} = n + 1\)
  maneras de llenar el canasto con \(n\) frutas.

  Al lanzar dos dados
  las sumas \(2\) y \(12\) se pueden obtener de una única manera,
  mientras para \(4\) hay tres (\(1 + 3 = 2 + 2 = 3 + 1\)).
  Para calcular el número de maneras de lograr cada valor,
  representamos un dado mediante la función generatriz:
  \begin{equation}
    \label{eq:gf-dado}
    D(z)
      = z + z^2 + z^3 + z^4 + z^5 + z^6
  \end{equation}
  con lo cual:
  \begin{equation}
    \label{eq:dos-dados}
    D^2(z)
      = z^2 + 2 z^3 + 3 z^4 + 4 z^5 + 5 z^6 + 6 z^7
	  + 5 z^8 + 4 z^9 + 3 z^{10} + 2 z^{11} + z^{12}
  \end{equation}
  El coeficiente de \(z^n\) da
  el número de formas de obtener \(n\) lanzando dos dados.

  Nace entonces la pregunta
  de si hay dados marcados en forma diferente
  que den la misma distribución
  (``dados locos'').%
    \index{dados locos|see{Sicherman, dados de}}
  Para construirlos
  debemos hallar funciones generatrices \(D_1(z)\) y \(D_2(z)\)
  que den el producto~\eqref{eq:dos-dados}.
  Debemos además tener que ambas representen dados,
  o sea tengan \(6\) caras,
  y que cada cara debe tener al menos un punto.
  Que cada cara tenga al menos un punto
  se traduce en que la función generatriz sea divisible por \(z\),
  el número de caras
  es simplemente el valor de la función en \(z = 1\).
  O sea:
  \begin{equation}
    \label{eq:dados-locos-caras}
    D_1(1)
      = D_2(1)
      = 6
  \end{equation}
  Factorizamos:
  \begin{equation}
    \label{eq:gf-dado-factorizada}
    D(z)
      = z (z + 1) (z^2 - z + 1) (z^2 + z + 1)
  \end{equation}
  Evaluando los factores en \(1\):
  \begin{equation}
    \label{eq:gf-dado-factorizada_1}
    D(1)
      = 1 \cdot 2 \cdot 1 \cdot 3
  \end{equation}
  Tanto \(D_1(z)\) como \(D_2(z)\)
  deben tener los factores \(z\),
  \(z + 1\) y \(z^2 + z + 1\);
  solo queda por redistribuir \(z^2 - z + 1\):
  \begin{align}
    D_1(z)
      &= z (z + 1) (z^2 + z + 1) \notag \\
      &= z + 2 z^2 + 2 z^3 + z^4 \label{eq:Sicherman-1} \\
    D_2(z)
      &= z (z + 1) (z^2 - z + 1)^2 (z^2 + z + 1) \notag \\
      &= z + z^3 + z^4 + z^5 + z^6 + z^8 \label{eq:Sicherman-2}
  \end{align}
  Fuera de dados comunes hay una posibilidad adicional,
  dados marcados con los multiconjuntos \(\{1, 2^2, 3^2, 4\}\)
  y \(\{1, 3, 4, 5, 6, 8\}\).
  Estos se conocen como \emph{dados de Sicherman}~%
    \cite{gardner78_2:_mathem_games}.%
    \index{Sicherman, dados de}
  Broline~\cite{broline79:_renum_faces_dice} estudia el problema
  para dados de números distintos de caras
  y más de dos dados.
  Gallian y Rusin~\cite{gallian79:_cyclot_polyn_nonst_dice}
  tratan un problema más general.

  Un problema antiguo popularizado por Pólya~%
    \cite{polya56:_picture_writing}%
    \index{cambio de monedas}
  pide determinar de cuántas formas se puede dar un dólar,
  usando monedas de \(1\), \(5\), \(10\), \(25\) y~\(50\) centavos.
  \begin{figure}[ht]
    \centering
    \pgfimage{images/coins-52}
    \caption{52 centavos en monedas}
    \label{fig:coins-52}
  \end{figure}
  La figura~\ref{fig:coins-52}
  muestra una manera de dar \(52\)~centavos.
  \begin{figure}[ht]
    \centering
    \pgfimage{images/coins-product}
    \caption{Colección de monedas como producto}
    \label{fig:coins-product}
  \end{figure}
  Podemos representar una colección de monedas
  como el ``producto'' de las cantidades de cada denominación,
  véase la figura~\ref{fig:coins-product}
  para una manera de tener \(62\)~centavos
  (el cuadrado representa una mesa vacía,
   ninguna moneda).
  \begin{figure}[ht]
    \centering
    \pgfimage{images/coins-1+5s}
    \caption{Series para 1 o 5 centavos}
    \label{fig:coins-1+5s}
  \end{figure}
  Todas las cantidades posibles
  usando solo monedas de~\(1\) o \(5\)~centavos
  se ilustran en la figura~\ref{fig:coins-1+5s},
  donde el signo~\(+\) separa las alternativas.
  Si ``multiplicamos'' las series,
  \begin{figure}[ht]
    \centering
    \pgfimage{images/coins-1x5s}
    \caption{Serie para combinaciones de 1 y 5 centavos}
    \label{fig:coins-1x5s}
  \end{figure}
  como muestra la figura~\ref{fig:coins-1x5s}
  obviando las mesas vacías y los signos de multiplicación,
  resultan todas las opciones
  para entregar una cantidad usando esas monedas.
  Nos interesa el número de maneras de tener,
  digamos,
  \(12\)~centavos,
  sin importar las monedas mismas.
  Esto lo logramos poniendo la denominación como exponente,
  o sea representando la moneda de \(5\)~centavos como \(z^5\).
  Al multiplicar se suman los exponentes,
  y al juntar los términos con el mismo exponente
  en su coeficiente estamos contando las maneras de tener esa suma.
  Las series de la figura~\ref{fig:coins-1+5s} quedan como:
  \begin{alignat*}{2}
    &1 + z + z^2 + z^3 + z^4 + \dotsb
      &\,&= \frac{1}{1 - z} \\
    &1 + z^5 + z^{10} + z^{15} + z^{20} + \dotsb
      &&= \frac{1}{1 - z^5}
  \end{alignat*}
  El coeficiente de \(z^{12}\)
  en \((1 + z + z^2 + \dotsb) (1 + z^5 + z^{10} + \dotsb)\)
  da el número de maneras de entregar \(12\)~centavos
  usando solo monedas de \(1\) y \(5\)~centavos:
  \begin{equation*}
    \frac{1}{(1 - z) (1 - z^5)}
      = 1 + z + z^2 + z^3 + z^4 + 2 z^5 + 2 z^6 + 2 z^7
	  + 2 z^8 + 3 z^{10} + 3 z^{11} + 3 z^{12} + 3 z^{13}
	  + \dotsb
  \end{equation*}
  Hay \(3\)~maneras,
  a saber:
  Sólo monedas de \(1\)~centavo,
  una moneda de \(5\)~centavos y siete de \(1\)~centavo,
  o dos de \(5\) y dos de \(1\).

  Las cantidades que se pueden entregar
  con la moneda de denominación \(d\)
  se representan por:
  \begin{equation*}
    1 + z^d + z^{2 d} + z^{3 d} + \dotsb
      = \frac{1}{1 - z^d}
  \end{equation*}
  Como combinar denominaciones corresponde a multiplicar las series,
  para el conjunto completo de denominaciones
  tenemos la función generatriz:
  \begin{equation}
    \label{eq:gf-coins}
    P(z)
      = \frac{1}
	 {(1 - z) (1 - z^5) (1 - z^{10}) (1 - z^{25}) (1 - z^{50})}
  \end{equation}
  El valor buscado es el coeficiente de \(z^{100}\)
  en~\eqref{eq:gf-coins}.

  No es viable expandir~\eqref{eq:gf-coins} hasta \(z^{100}\),
  veremos un camino alternativo.
  La serie~\eqref{eq:gf-coins} es el producto de cinco factores,
  conocemos el primero
  (la serie geométrica)
  e iremos adicionando los demás uno a uno.
  Supongamos que ya tenemos el producto
  de los dos primeros factores:
  \begin{equation*}
    \frac{1}{(1 - z) (1 - z^5)}
      = a_0 + a_1 z + a_2 z^2 + \dotsb
  \end{equation*}
  y queremos añadir el tercero:
  \begin{equation*}
    \frac{1}{(1 - z) (1 - z^5) (1 - z^{10})}
      = b_0 + b_1 z + b_2 z^2 + \dotsb
  \end{equation*}
  Vemos que:
  \begin{equation*}
    (b_0 + b_1 z + b_2 z^2 + \dotsb) (1 - z^{10})
      = a_0 + a_1 z + a_2 z^2 + \dotsb
  \end{equation*}
  Comparando coeficientes
  (es \(b_n = 0\) si \(n < 0\)):
  \begin{equation*}
    b_n
      = b_{n - 10} + a_n
  \end{equation*}
  Esta relación
  permite calcular los \(b_n\) si ya conocemos los \(a_n\),
  y obtenemos la serie completa en cuatro pasos similares
  al que discutimos recién.
  El cuadro~\ref{tab:coin-change}
  resume el cálculo hasta \(50\)~centavos
  (solo se dan los valores necesarios para obtener \(p_{50} = 50\)),
  el amable lector completará el cuadro
  y verificará que hay un total
  de 292~maneras de dar un dólar en monedas.
  \begin{table}[ht]
    \centering
    \begin{tabular}{|>{\(}l<{\)}|*{11}{>{\(}r<{\)}}|}
      \hline
      \multicolumn{1}{|r}{\(n = {}\)} &
	\rule[-0.3ex]{0pt}{3ex}%
	0 & 5 & 10 & 15 & 20 & 25 & 30 & 35 & 40 & 45 & 50 \\
      \hline
      \rule[-0.5ex]{0pt}{3ex}%
      (1 - z)^{-1}
	  & 1 & 1 & 1 & 1 & 1 &	 1 &  1 &  1 &	1 &  1 &  1 \\
      (1 - z^5)^{-1}
	  & 1 & 2 & 3 & 4 & 5 &	 6 &  7 &  8 &	9 & 10 & 11 \\
      (1 - z^{10})^{-1}
	  & 1 & 2 & 4 & 6 & 9 & 12 & 16 &    & 25 &    & 36 \\
      (1 - z^{25})^{-1}
	  & 1 &	  &   &	  &   & 13 &	&    &	  &    & 49 \\
      (1 - z^{50})^{-1}
	  & 1 &	  &   &	  &   &	   &	&    &	  &    & 50
    \end{tabular}
    \caption{Tabla para calcular $p_{50}$}
    \label{tab:coin-change}
  \end{table}
  En el clásico de Graham, Knuth y~Patashnik~%
    \cite{graham94:_concr_mathem}
  continúan este desarrollo.
  Aprovechan la forma especial de las recurrencias resultantes
  y obtienen una fórmula cerrada para \(p_n\).

  Un problema clásico propuesto por Sylvester en~1884
  es el siguiente:
  Si solo se tienen estampillas de \(5\) y \(17\)~centavos,
  ¿cuál es el máximo monto
  que \emph{no} se puede franquear con ellas?%
    \index{problema de franqueo|see{Frobenius, problema de}}

  La solución de Bogomolny~%
    \cite{bogomolny12:_Sylvester_2nd_look}
  muestra cómo representar conjuntos.
  Para generalizar,
  digamos que los montos de las estampillas son \(p\) y \(q\),
  ambos mayores a \(1\),
  con \(\gcd(p, q) = 1\).
  Si no fueran relativamente primos,
  habrían infinitos valores imposibles de representar
  (solo es posible representar múltiplos de \(\gcd(p, q)\)
   mediante expresiones de la forma \(a p + b q\)).

  Por la identidad de Bézout%
    \index{Bezout, identidad de@Bézout, identidad de}
  (ver la sección~\ref{sec:GCD})
  sabemos que hay \(u, v\) tales que \(u p - v q = 1\),
  sin pérdida de generalidad podemos suponer que \(u, v > 0\).
  Si tomamos \(x q\) para algún \(x\) por determinar,
  para \(1 \le k < q\) podemos escribir:
  \begin{equation*}
    x q + k
      = x q + k (u p - v q)
      = k u p + (x - k v) q
  \end{equation*}
  El primer término es siempre positivo,
  interesa acotar \(k v\) para asegurar que ambos sean no negativos
  y \(x q + k\) siempre sea representable.
  Como \(v\) es el inverso de \(q\) en \(\mathbb{Z}_p\)
  es \(1 \le v < p\),
  y por tanto al menos
  a partir de \((q - 1) (p - 1) q\) todos son representables.

  Formemos la familia de secuencias aritméticas
  \(f_a = \langle a p + b q  \rangle_{b \ge 0}\):
  \begin{alignat*}{2}
    &f_0
      &\,&= \langle \phantom{0}0 + 0,
	       \phantom{p}0 + q,
	       \phantom{p}0 + 2 q,
	       \phantom{p}0 + 3 q, \dotsc \rangle \\
    &f_1
      &&= \langle \phantom{0}p + 0,
		  \phantom{0}p + q,
		  \phantom{0}p + 2 q,
		  \phantom{0}p + 3 q, \dotsc \rangle \\
    &f_2
      &&= \langle
	    2 p + 0, 2 p + q, 2 p + 2 q, 2 p + 3 q, \dotsc
	  \rangle \\
    & &&\vdots \\
    &f_{q - 1}
      &&= \langle
	    (q - 1) p + 0, (q - 1) p + q, (q - 1) p + 2 q,
	      (q - 1) p + 3 q, \dotsc
	  \rangle
  \end{alignat*}
  La idea es que la secuencia \(f_k\)
  representa los franqueos posibles
  con \(k\) estampillas de \(p\) centavos
  y algún número de estampillas de \(q\) centavos.
  Como \(\gcd(p, q) = 1\),
  estas secuencias son disjuntas,
  y cubren todas las posibilidades de \(a p + b q\)
  con \(a, b \ge 0\).
  Los elementos de \(f_a\) son congruentes módulo \(q\),
  siendo \(p\) y \(q\) relativamente primos
  el conjunto \(\{a p \bmod q \colon 0 \le a < q\}\)
  es simplemente \(\{k \colon 0 \le k < q\}\).
  Si las secuencias hubiesen comenzado
  con los residuos respectivos,
  las secuencias cubrirían todo \(\mathbb{N}\);
  pero como \(f_a\) parte de \(a p\)
  la unión de las secuencias deja espacios al comienzo.
  Interesa hallar el máximo número que no aparece en la unión,
  que llamaremos \(g(p, q)\).

  Los elementos de la unión de las secuencias
  indicadas son los exponentes
  de la siguiente función generatriz
  (los coeficientes de la suma son todos uno,
   no hay intersección entre las secuencias):
  \begin{equation*}
    f(z)
      = \frac{1}{1 - z^q}
	  \, (1 + z^p + z^{2 p} + \dotsb + z^{(q - 1) p})
      = \frac{1 - z^{p q}}{(1 - z^p) (1 - z^q)}
  \end{equation*}
  Por el otro lado,
  el conjunto completo de los enteros no negativos
  se representa por:
  \begin{equation*}
    h(z)
      = 1 + z + z^2 + z^3 + \dotsb
      = \frac{1}{1 - z}
  \end{equation*}
  La diferencia entre las dos es un polinomio,
  cuyos exponentes indican los números que no se pueden representar:
  \begin{equation}
    \label{eq:gf-Frobenius}
    h(z) - f(z)
      = \frac{1}{1 - z} - \frac{1 - z^{p q}}{(1 - z^p) (1 - z^q)}
      = \frac{(1 - z^p) (1 - z^q) - (1 - z) (1 - z^{p q})}
	     {(1 - z) (1 - z^p) (1 - z^q)}
  \end{equation}
  Restar el grado del denominador del grado del numerador
  da el grado del polinomio:
  \begin{equation}
    \label{eq:Frobenius:g(p,q)}
    g(p, q)
      = (p q + 1) - (p + q + 1)
      = p q - p - q
  \end{equation}
  Esta teoría
  nos dice que la respuesta al problema específico planteado
  es que el máximo valor que no puede franquearse es
  \begin{equation*}
    g(5, 17)
      = 5 \cdot 17 - 5 - 17
      = 63
  \end{equation*}

  Otra pregunta es cuántos son los valores
  que no pueden representarse,
  que no es más que la suma de los coeficientes
  del polinomio~\eqref{eq:gf-Frobenius},
  o sea,
  el valor del mismo evaluado en \(z = 1\).
  Aplicando l'Hôpital%
    \index{Hopital, regla de@l'Hôpital, regla de}
  tres veces a~\eqref{eq:gf-Frobenius}
  entrega:
  \begin{equation}
    \label{eq:Frobenius-not-representable}
    \lim_{z \rightarrow 1} \, \left( h(z) - f(z) \right)
      = \frac{p q - p - q + 1}{2}
      = \frac{g(p, q) + 1}{2}
  \end{equation}
  Los no representables
  resultan ser \(32\) en nuestro caso específico.

  Este es el caso particular \(n = 2\) del problema de Frobenius,%
    \index{Frobenius, problema de}
  determinar para un conjunto de naturales relativamente primos
  \(\{a_1, a_2, \dotsc, a_n\}\)
  cuál es el máximo entero que no puede representarse
  como combinación lineal con coeficientes naturales.
  A este número se le llama el \emph{número de Frobenius}%
    \index{Frobenius, numero de@Frobenius, número de}
  del conjunto,
  y se anota \(g(a_1, \dotsc, a_n)\).
  Para \(n > 2\) no se conocen fórmulas generales,
  solo soluciones en casos particulares.
  A pesar de parecer muy especializado,
  este problema y variantes aparecen en muchas aplicaciones.
  Un resumen reciente de la teoría y algoritmos
  presenta Ramírez~Alfonsín~%
    \cite{ramirez06:_dioph_frobenius_probl}.

\section{Manipulación de series}
\label{sec:manipulacion-series}

  Sea una secuencia
  \(\left\langle a_n \right\rangle_{n \ge 0}
     = \left\langle
	 a_0, a_1, a_2, \dotsc, a_n, \dotsc
       \right\rangle\).
  La \emph{función generatriz} (ordinaria) de la secuencia es
  la serie de potencias:%
    \index{generatriz!ordinaria|textbfhy}
  \begin{equation*}
    A(z)=\sum_{0 \le n} a_n z^n
  \end{equation*}
  Anotaremos
  \(A(z)
     \ogf \left\langle a_n\right\rangle_{n \ge 0}\) en este caso
  (\emph{ogf} es por
     \emph{\foreignlanguage{english}
			   {Ordinary Generating Function}}).

  La \emph{función generatriz exponencial}%
    \index{generatriz!exponencial|textbfhy}
  de la secuencia es la serie:
  \begin{equation*}
    \widehat{A}(z)
      = \sum_{0 \le n} a_n \, \frac{z^n}{n!}
  \end{equation*}
  Anotaremos
  \(\widehat{A}(z)
     \egf \left\langle a_n\right\rangle_{n \ge 0}\) en este caso
  (\emph{egf} es por
     \emph{\foreignlanguage{english}
			   {Exponential Generating Function}}).

  Por comodidad,
  a veces escribiremos estas relaciones
  con la función generatriz al lado derecho.

\subsection{Reglas OGF}
\label{sec:reglas-OGF}
\index{generatriz!ordinaria!reglas}

  Las propiedades siguientes de funciones generatrices ordinarias
  son directamente las definiciones del caso
  o son muy simples de demostrar,
  sus justificaciones detalladas quedarán de ejercicios.

  \begin{description}
  \item[Linealidad:]
    Si \(A(z) \ogf \left\langle a_n \right\rangle_{n \ge 0}\)
    y \(B(z) \ogf \left\langle b_n \right\rangle_{n \ge 0}\),
    y \(\alpha\) y \(\beta\) son constantes,
    entonces:
    \begin{equation*}
      \alpha A(z) + \beta B(z)
	 \ogf \left\langle
		\alpha a_n + \beta b_n
	      \right\rangle_{n \ge 0}
    \end{equation*}
  \item[Secuencia desplazada a la izquierda:]
    Si
    \(A(z) \ogf \left\langle a_n \right\rangle_{n \ge 0}\),
    entonces:
    \begin{equation*}
      \frac{A(z) - a_0 - a_1 z - \dotsb - a_{k - 1} z^{k - 1}}{z^k}
	\ogf \left\langle a_{n + k}\right\rangle_{n \ge 0}
    \end{equation*}
  \item[Multiplicar por \(n\):]
    Consideremos:
    \begin{align*}
      A(z)
	&\ogf \left\langle a_n\right\rangle_{n \ge 0} \\
      z \, \frac{\mathrm{d}}{\mathrm{d} z} A(z)
	&\ogf \left\langle n a_n\right\rangle_{n \ge 0}
    \end{align*}
    Esta operación
    se expresa en términos del operador \(z \mathrm{D}\)
    (acá \(\mathrm{D}\) es por derivada,
     para abreviar).
    Además:
    \begin{equation*}
      (z \mathrm{D})^2 A(z)
	= z D (z D A(z))
	\ogf \left\langle n^2 a_n\right\rangle_{n \ge 0}
    \end{equation*}
    Nótese que
      \((z \mathrm{D})^2 = z \mathrm{D} + z^2 \mathrm{D}^2\)
    es diferente de \(z^2 \mathrm{D}^2\).
  \item[Multiplicar por un polinomio en \(n\):]
    Si \(p(n)\) es un polinomio,
    entonces:
    \begin{align*}
      p(z \mathrm{D}) A(z)
	&\ogf \left\langle p(n) a_n \right\rangle_{n \ge 0}
    \end{align*}
  \item[Convolución:]
    Si \(A(z) \ogf \left\langle a_n \right\rangle_{n \ge 0}\)
    y \(B(z) \ogf \left\langle b_n \right\rangle_{n \ge 0}\)
    entonces:
    \begin{equation*}
      A(z) \cdot B(z)
	\ogf \left\langle
	       \sum_{0 \le k \le n} a_k b_{n - k}
	      \right\rangle_{n \ge 0}
    \end{equation*}
  \item
    Sea \(k\) un entero positivo
    y \(A(z) \ogf \left\langle a_n\right\rangle_{n \ge 0}\),
    entonces:
    \begin{equation*}
      (A(z))^k
	\ogf \left\langle \sum_{n_1 + n_2 + \dotsb + n_k = n}
	       \left( a_{n_1} \cdot a_{n_2} \dotsm a_{n_k} \right)
	     \right\rangle_{n \ge 0}
    \end{equation*}
    Vale la pena tener presente el caso especial:
    \begin{equation*}
      (A(z))^2
	\ogf \left\langle
	       \sum_{0 \le i \le n} a_i a_{n - i}
	     \right\rangle_{n \ge 0}
    \end{equation*}
  \item[Sumas parciales:]
    Supongamos:
    \begin{equation*}
      A(z) \ogf \left\langle a_n \right\rangle_{n \ge 0}
    \end{equation*}
    Podemos escribir:
    \begin{equation*}
      \sum_{0 \le k \le n} a_k
	= \sum_{0 \le k \le n} 1 \cdot a_k
    \end{equation*}
    Esto no es más que la convolución de las secuencias
    \(\left\langle 1 \right\rangle_{n \ge 0}\)
    y \(\left\langle a_n \right\rangle_{n \ge 0}\),
    y la función generatriz de la primera es nuestra vieja conocida,
    la serie geométrica,
    con lo que:
    \begin{equation}
      \label{eq:sumas-parciales}
      \frac{A(z)}{1 - z}
	\ogf \left\langle
	       \sum_{0 \le k \le n} a_k
	     \right\rangle_{n \ge 0}
    \end{equation}
  \end{description}

  Un primer ejemplo clásico
  (ver por ejemplo Knuth~\cite{knuth98:_sortin_searc})
  es el análisis de búsqueda binaria.%
    \index{analisis de algoritmos@análisis de algoritmos!busqueda binaria@búsqueda binaria}
  Supongamos que contamos con un arreglo ordenado de \(n\) claves
  \(k_1 < k_2 < \dotsb < k_n\),
  dada una clave \(k\) nos interesa identificar \(1 \le j \le n\)
  tal que \(k = k_j\)
  (búsqueda exitosa).
  Búsqueda binaria compara \(k\) con el elemento medio,
  en \(r = \lfloor (n + 1) / 2 \rfloor\).
  Si \(k = k_r\),
  estamos listos.
  En caso contrario,
  si \(k < k_r\) seguimos la búsqueda en \(k_1, \dotsc, k_{r - 1}\),
  mientras que si \(k > k_r\)
  seguimos la búsqueda en \(k_{r + 1}, \dotsc, k_n\).
  Interesa
  el número promedio \(b_n\) de comparaciones en búsquedas exitosas,
  si \(k\) se elige al azar.
  Hay un único elemento que puede encontrarse con una comparación,
  el elemento medio.
  Hay dos que pueden encontrarse con dos comparaciones,
  y así sucesivamente
  hasta llegar a un máximo
  de \(1 + \lfloor \log_2 n \rfloor\) comparaciones.
  Si sumamos el número de comparaciones
  para cada una de las \(n\) claves
  obtenemos el número promedio de comparaciones:
  \begin{equation*}
    b_n
      = \frac{1}{n} \,
	  \left(
	    1 + 2 + 2 + 3 + 3 + 3 + 3
	      + \dotsb
	      + (1 + \lfloor \log_2 n \rfloor)
	  \right)
  \end{equation*}
  Para calcular la suma,
  consideramos la secuencia infinita
    \(\langle 0, 1, 2, 2, 3, 3, 3, 3, \dotsc\rangle\),
  que se obtiene de sumar secuencias
  \(\langle 0, 1, 1, \dotsc \rangle\),
  \(\langle 0, 0, 1, 1, \dotsc \rangle\),
  y así sucesivamente,
  donde la \(k\)\nobreakdash-ésima secuencia
  comienza con \(2^k\) ceros.
  Podemos representar la secuencia
  como los coeficientes de la serie:
  \begin{equation*}
    L(z)
      = \frac{z}{1 - z}
	  + \frac{z^2}{1 - z}
	  + \dotsb
	  + \frac{z^{2^k}}{1 - z}
	  + \dotsb
      = \sum_{k \ge 0} \frac{z^{2^k}}{1 - z}
  \end{equation*}
  Nos interesan sumas parciales:
  \begin{align*}
    n b_n
      &= \left[ z^n \right] \,
	   \frac{1}{1 - z} \sum_{k \ge 0} \frac{z^{2^k}}{1 - z} \\
      &= \left[ z^n \right] \,
	   \sum_{k \ge 0} \frac{z^{2^k}}{(1 - z)^2} \\
      &= \sum_{k \ge 0} \left[ z^{n - 2^k} \right] (1 - z)^{-2} \\
      &= \sum_{k \ge 0} \binom{n - 2^k + 1}{n - 2^k} \\
      &= \sum_{0 \le k \le \lfloor \log_2 n \rfloor}
	   (n + 1 - 2^k) \\
      &= (n + 1) (\lfloor \log_2 n \rfloor + 1)
	   - \sum_{0 \le k \le \lfloor \log_2 n \rfloor} 2^k \\
      &= (n + 1) \lfloor \log_2 n \rfloor
	   + n - 2^{\lfloor \log_2 n \rfloor + 1} + 2
  \end{align*}
  Manipulaciones formales que dan directamente lo que buscamos.

\subsection{Reglas EGF}
\label{sec:reglas-EGF}
\index{generatriz!exponencial!reglas}

  Las siguientes resumen propiedades
  de las funciones generatrices exponenciales.
  Son simples de demostrar,
  y las justificaciones que no se dan acá quedarán de ejercicios.
  \begin{description}
  \item[Linealidad:]
    Si \(\widehat{A}(z)
	   \egf \left\langle a_n \right\rangle_{n \ge 0}\)
    y \(\widehat{B}(z)
	  \egf \left\langle b_n \right\rangle_{n \ge 0}\),
    y \(\alpha\) y \(\beta\) son constantes,
    entonces:
    \begin{equation*}
      \alpha \widehat{A}(z) + \beta \widehat{B}(z)
	\egf \left\langle
	       \alpha a_n + \beta b_n
	     \right\rangle_{n \ge 0}
    \end{equation*}
  \item[Secuencia desplazada a la izquierda:]
    Si \(\widehat{A}(z)
	   \egf \left\langle a_n \right\rangle_{n \ge 0}\),
    entonces:
    \begin{equation*}
      \mathrm{D}^k \widehat{A}(z)
	\egf \left\langle a_{n + k} \right\rangle_{n \ge 0}
    \end{equation*}
  \item[Multiplicación por un polinomio en \(n\):]
    Si es \(\widehat{A}(z)
	      \egf \left\langle a_n \right\rangle_{n \ge 0}\),
    y \(p\) es un polinomio,
    entonces:
    \begin{equation*}
      p(z \mathrm{D}) \widehat{A}(z)
	\egf \left\langle p(n) a_n \right\rangle_{n \ge 0}
    \end{equation*}
    Es la misma que en funciones generatrices ordinarias,
    ya que la operación \(z \mathrm{D}\)
    no altera el exponente en \(z^n\).
  \item[Convolución binomial:]
    Si \(\widehat{A}(z)
	   \egf \left\langle a_n \right\rangle_{n \ge 0}\) y
    \(\widehat{B}(z)
	\egf \left\langle b_n \right\rangle_{n \ge 0}\)
    entonces:
    \begin{align*}
      \widehat{A}(z) \cdot \widehat{B}(z)
	&= \sum_{n \ge 0}\biggl( \,
			   \sum_{0 \le k \le n}
			   \frac{a_k}{k!} \, \frac{b_{n - k}}
						  {(n - k)!}
			 \biggr) z^n \\
	&= \sum_{n \ge 0} \biggl( \,
			    \sum_{0 \le k \le n}
			       \binom{n}{k} \, a_k b_{n - k}
			  \biggr)
	       \frac{z^n}{n!}
    \end{align*}
    Vale decir:
    \begin{equation*}
      \widehat{A}(z) \cdot \widehat{B}(z)
	\egf \left\langle
	       \sum_{0 \le k \le n} \binom{n}{k} \, a_k b_{n - k}
	     \right\rangle_{n \ge 0}
    \end{equation*}
  \item[Términos individuales:]
    Es fácil ver que si
    \(\widehat{A}(z)
	\egf \left\langle a_n \right\rangle_{n \ge 0}\) entonces:
    \begin{equation*}
      a_n = \widehat{A}^{(n)}(0)
    \end{equation*}
    Esto en realidad no es más que el teorema de Maclaurin.%
      \index{Maclaurin, teorema de}
  \end{description}

\section[\texorpdfstring{El truco $z \mathrm{D} \log$}
			{Derivada logarítmica}]
	{\protect\boldmath
	   \texorpdfstring{El truco $z \mathrm{D}\log$}
			  {Derivada logarítmica}%
       \protect\unboldmath}
\index{derivada logaritmica@derivada logarítmica|textbfhy}

  Los logaritmos ayudan a simplificar expresiones con exponenciales
  y potencias.
  Pero terminamos con el logaritmo de una suma
  si el argumento es una serie,
  que es algo bastante feo de contemplar.
  Eliminar el logaritmo se logra derivando:
  \begin{equation*}
    \frac{\mathrm{d} \ln(A)}{\mathrm{d} z} = \frac{A'}{A}
  \end{equation*}
  Esto es mucho más decente.
  Multiplicamos por \(z\)
  para reponer la potencia ``perdida'' al derivar.

\subsubsection*{Receta:}
\index{derivada logaritmica@derivada logarítmica!receta}

  \begin{enumerate}
  \item
    Aplicar \(z \mathrm{D} \ln\).
  \item
    Multiplicar para eliminar fracciones.
  \item
    Igualar coeficientes.
  \end{enumerate}

\section{Ejemplos de manipulación de series}
\label{sec:gf-ejemplos}

  Un ejemplo inicial de aplicación de las ideas planteadas
  es obtener la suma de los primeros \(N\) cuadrados.%
    \index{suma!cuadrados}
  Por la suma de la serie geométrica,%
    \index{serie geometrica@serie geométrica!suma}
  teorema~\ref{theo:suma-geometrica}:
  \begin{align*}
    1 + z + z^2 + \dotsb + z^N
      &= \frac{1 - z^{N + 1}}{1 - z} \\
    (z \mathrm{D})^2 \, \left( 1 + z + z^2 + \dotsb + z^N \right)
      &= (z \mathrm{D})^2 \, \frac{1 - z^{N + 1}}{1 - z} \\
    \left.
      \left( 0^2 + 1^2 z + 2^2 z^2 + \dotsb + N^2 z^N \right)
    \right|_{z = 1}
      &= \lim_{z \rightarrow 1} \,
	   (z \mathrm{D})^2 \, \frac{1 - z^{N + 1}}{1 - z}
  \end{align*}
  Nótese que todas las expresiones involucradas son polinomios,
  con lo que cuestiones de convergencia y validez de las operaciones
  no son problema.

  El resto es derivar,
  calcular límites y álgebra:
  \begin{equation*}
    \sum_{1 \le k \le N} k^2 = \frac{N (N + 1) (2 N + 1)}{6}
  \end{equation*}
  La misma idea sirve para otras potencias.

  La maquinaria de funciones generatrices
  permite obtener en forma rutinaria
  resultados que de otra forma serían complicados de sospechar,
  y luego deberían ser demostrados por inducción.
  La operatoria suele ser tediosa,
  es útil tener un programa de álgebra simbólica
  (como \texttt{maxima}~\cite{maxima14b:_computer_algebra})%
    \index{maxima@\texttt{maxima}}
  a la mano.

  Otra aplicación es obtener la serie para \(A(z)^\alpha\),%
    \index{serie de potencias!potencia}
  una potencia arbitraria
  (\(\alpha \in \mathbb{C}\))
  de una serie \(A(z)\) que ya conocemos.
  Sea entonces:
  \begin{equation*}
    A(z)
      = \sum_{n \ge 0} a_n z^n
  \end{equation*}
  donde \(a_0 \ne 0\).
  Definimos:
  \begin{equation*}
    B(z)
      = A^\alpha (z) = \sum_{n \ge 0} b_n z^n
  \end{equation*}
  Aplicando la receta \(z \mathrm{D} \log\) obtenemos:%
    \index{derivada logaritmica@derivada logarítmica}
  \begin{align*}
    \frac{z B'(z)}{B(z)}
      &= \alpha z \, \frac{A'(z)}{A(z)} \\
    z B'(z) \cdot A(z)
      &= \alpha z A'(z) \cdot B(z) \\
    \biggl( \, \sum_{n \ge 0} n b_n z^n \biggr)
       \cdot \biggl( \, \sum_{n \ge 0} a_n z^n \biggr)
      &= \alpha \biggl( \,
		   \sum_{n \ge 0} n a_n z^n
		 \biggr)
	     \cdot \biggl( \,
		     \sum_{n \ge 0} b_n z^n
		   \biggr) \\
    \sum_{n \ge 0} \biggl( \,
		     \sum_{0 \le k \le n} k b_k a_{n - k}
		   \biggr) z^n
      &= \sum_{n \ge 0}
	   \biggl( \,
	     \sum_{0 \le k \le n} \alpha k a_k b_{n - k}
	   \biggr) z^n
  \end{align*}
  De acá sigue,
  igualando coeficientes:
  \begin{equation*}
    \sum_{0 \le k \le n} a_k (n - k) b_{n - k}
      = \sum_{0 \le k \le n} \alpha k a_k b_{n - k}
  \end{equation*}
  Nuevamente,
  esto involucra solo finitas operaciones.
  Finalmente:
  \begin{align*}
    \sum_{0 \le k \le n}
      \bigl(
	a_k (n - k) b_{n - k} - \alpha k a_k b_{n - k}
      \bigr)
      &= 0 \\
    \sum_{0 \le k \le n} (n - k - \alpha k) a_k b_{n - k}
      &= 0
  \end{align*}
  de donde resulta al separar el término con \(k = 0\):
  \begin{align*}
    n a_0 b_n
      &= -\biggl( \,
	    \sum_{1 \le k \le n} (n - k - \alpha k) a_k b_{n - k}
	  \biggr) \\
    b_n
      &= -\frac{1}{n a_0}
	  \, \sum_{1 \le k \le n} (n - k - \alpha k) a_k
    b_{n - k}
  \end{align*}
  Para comenzar la recurrencia,
  usamos:
  \begin{equation*}
    b_0 = a_0^\alpha
  \end{equation*}
  Compárese esta recurrencia con la expresión explícita
  para una potencia entera de una serie
  que derivamos antes.

\section{Funciones generatrices en combinatoria}
\label{sec:FG-combinatoria}

  De nuevo
  la Competencia de Ensayos de la Universidad de Miskatonic.
  Para simplificar notación,
  sea \(a_r = b_{2 r + 1}\).
  Resulta:
  \begin{equation}
    \label{eq:recurrence-UMEC-a-1}
    a_r
      = a_{r - 1} + r + 1
  \end{equation}
  La condición inicial es \(a_0 = b_1 = 1\).
  Llamemos \(A(z)\) a la función generatriz ordinaria
  de la secuencia \(\left\langle a_r\right\rangle_{r \ge 0}\):
  \begin{equation*}
    A(z)
      = \sum_{r \ge 0} a_r z^r
  \end{equation*}

  La recurrencia~\eqref{eq:recurrence-UMEC-a-1}%
    \index{recurrencia}
  es incómoda de manejar como está escrita,
  primero ajustamos los índices
  para no hacer referencia a términos previos:
  \begin{equation}
    \label{eq:recurrence-UMEC-a-2}
    a_{r + 1} - a_r
      = r + 2
  \end{equation}
  Las funciones generatrices de los términos al lado izquierdo
  de la recurrencia~\eqref{eq:recurrence-UMEC-a-2}
  son:%
    \index{generatriz!ordinaria}
  \begin{equation*}
    \left\langle a_{r + 1}\right\rangle_{r \ge 0}
      \ogf \frac{A(z) - a_0}{z}
      = \frac{A(z) - 1}{z}
    \hspace{3em}
    \left\langle a_r\right\rangle_{r \ge 0}
      \ogf A(z)
  \end{equation*}
  Necesitamos además la función generatriz de la secuencia \(r + 2\)
  que aparece al lado derecho,
  que no es más
  que la secuencia \(\left\langle 1\right\rangle_{r \ge 0}\)
  multiplicada por el polinomio \(r + 2\),
  con lo que:
  \begin{equation*}
    \left\langle r + 2\right\rangle_{r \ge 0}
      \ogf (z \mathrm{D} + 2) \, \frac{1}{1 - z}
      = \frac{z}{(1 - z)^2} + \frac{2}{1 - z}
  \end{equation*}
  Combinando las anteriores,
  tenemos:
  \begin{equation*}
    \frac{A(z) - 1}{z} - A(z)
      = \frac{z}{(1 - z)^2} + \frac{2}{1 - z}
  \end{equation*}
  Despejando \(A(z)\) se tiene:
  \begin{equation*}
    A(z)
      = \frac{1}{(1 - z)^3}
  \end{equation*}
  y los coeficientes del caso son inmediatos:
  \begin{equation*}
    a_r
      = (-1)^r \binom{-3}{r}
      = \binom{2 + r}{2}
      = \frac{(r + 2) (r + 1)}{2}
  \end{equation*}
  Nuevamente resulta:
  \begin{equation*}
    b_{2 r + 1}
      = \frac{(r + 2) (r + 1)}{2}
  \end{equation*}
  Esta derivación es aún más simple que la anterior.
  Siempre que sea posible
  se deben usar las propiedades de funciones generatrices,
  debe recurrirse a la receta general dada anteriormente
  solo cuando no es claro cómo aplicarlas.

  Podemos igualmente intentar
  con la función generatriz exponencial:%
    \index{generatriz!exponencial}
  \begin{equation*}
    \widehat{A}(z)
      = \sum_{r \ge 0} a_r \frac{z^r}{r!}
  \end{equation*}
  Aplicando las propiedades respectivas
  a~\eqref{eq:recurrence-UMEC-a-2}:
  (refiérase a la sección~\ref{sec:reglas-EGF}):
  \begin{align*}
    \widehat{A}'(z) - \widehat{A}(z)
      &= \left(
	   z \frac{\mathrm{d}}{\mathrm{d} z} + 2
	 \right) \mathrm{e}^z \\
      &= z \mathrm{e}^z + 2 \mathrm{e}^z
  \end{align*}
  Como condición inicial tenemos:
  \begin{equation*}
    \widehat{A}(0)
      = a_0
      = 1
  \end{equation*}
  La solución de la ecuación diferencial es:%
    \index{ecuacion diferencial@ecuación diferencial}
  \begin{equation*}
    \widehat{A}(z)
      = \frac{\mathrm{e}^z}{2} (z^2 + 4 z + 2)
  \end{equation*}

  Ahora tenemos dos caminos posibles:
  Expresar la solución mediante las propiedades,
  o calcular los términos mediante la expansión en serie.
  Para aplicar las propiedades,
  notamos:
  \begin{equation*}
    \frac{1}{2} \left( z^2 + 4 z + 2 \right) \mathrm{e}^z
      = \frac{1}{2}
	  \left( z^2 D^2 + 4 z D + 2 \right)
	  \mathrm{e}^z \\
  \end{equation*}
  Además:
  \begin{align*}
    (z D)^2
      &= z^2 D^2 + z D \\
    z^2 D^2 + 4 z D + 2
      &= (z D)^2 + 3 z D + 2
  \end{align*}
  O sea:
  \begin{equation*}
    \widehat{A}(z)
      = \frac{1}{2}
	  \left( (z D)^2 + 3 z D + 2 \right)
	  \mathrm{e}^z
  \end{equation*}
  Esto corresponde a:
  \begin{equation*}
    b_{2 r + 1}
      = a_r
      = \frac{1}{2} (r^2 + 3 r + 2)
      = \frac{(r + 2) (r + 1)}{2}
  \end{equation*}

  El otro camino es:
  \begin{align*}
    \mathrm{e}^z \frac{z^2 + 4 z + 2}{2}
      &= \sum_{r \ge 0}
	   \left(
	     \frac{z^{r + 2}}{2 r!}
	       + \frac{2 z^{r + 1}}{r!}
	       + \frac{z^r}{r!}
	   \right) \\
      &= \frac{1}{2} \sum_{r \ge 2} \frac{z^r}{(r - 2)!}
	   + 2 \sum_{r \ge 1} \frac{z^r}{(r - 1)!}
	   + \sum_{r \ge 0} \frac{z^r}{r!} \\
      &= \sum_{r \ge 0}
	   \left(
	     \frac{r (r - 1)}{2}
	       + 2 r + 1
	   \right) \frac{z^r}{r!} \\
      &= \sum_{r \ge 0}
	   \frac{(r + 2) (r + 1)}{2} \, \frac{z^r}{r!}
  \end{align*}
  y nuevamente:
  \begin{equation*}
    b_{2 r + 1}
      = a_r
      = \frac{(r + 2) (r + 1)}{2}
  \end{equation*}

  Los objetos colgados para exhibición en la función generatriz
  no tienen porqué ser números.
  Un polinomio \(f(x_1, x_2, \dotsc, x_n)\)
  se llama \emph{simétrico}%
    \index{polinomio!simetrico@simétrico|textbfhy}
  si para cualquier permutación \(\sigma\) de \([n]\):
  \begin{equation}
    \label{eq:symmetric-polynomial-definition}
    f(x_1, x_2, \dotsc, x_n)
      = f(x_{\sigma(1)}, x_{\sigma(2)}, \dotsc, x_{\sigma(n)})
  \end{equation}
  Vale decir,
  el polinomio se mantiene inalterado
  bajo cualquier reordenamiento de variables.

  Considerando polinomios homogéneos
  de grado \(m\) en \(n\) variables,
  están las familias:
  \begin{align}
    e_m(x_1, x_2, \dotsc, x_n)
      &= \sum_{k_1 < k_2 < \dotsb < k_m}
	   x_{k_1} x_{k_2} \dotsm x_{k_m}
	 \label{eq:symmetric-polynomial-e} \\
    h_m(x_1, x_2, \dotsc, x_n)
      &= \sum_{k_1 + k_2 + \dotsb + k_n = m}
	   x_1^{k_1} x_2^{k_2} \dotsm x_n^{k_n}
	 \label{eq:symmetric-polynomial-h} \\
    p_m(x_1, x_2, \dotsc, x_n)
      &= \sum_{1 \le k \le n} x_k^m
	 \label{eq:symmetric-polynomial-p}
  \end{align}
  Los \(e_m\) son los llamados \emph{elementales}%
    \index{polinomio!simetrico@simétrico!elemental|textbfhy}
  (ya nos tropezamos con ellos
   en las fórmulas de Vieta~\eqref{eq:Vieta-formulas}),
  los \(h_m\) se llaman \emph{homogéneos completos}%
    \index{polinomio!simetrico@simétrico!homogeneo completo@homogéneo completo|textbfhy}
  y los \(p_m\) simplemente \emph{sumas de potencias}.%
    \index{polinomio!simetrico@simétrico!suma de potencias|textbfhy}
  Por ejemplo,
  para \(n = 3\):
  \begin{align*}
    e_2(x_1, x_2, x_3)
      &= x_1 x_2 + x_1 x_3 + x_2 x_3 \\
    h_3(x_1, x_2, x_3)
      &= x_1^3 + x_1^2 x_2 + x_1^2 x_3
	     + x_1 x_2^2 + x_1 x_2 x_3 + x_1 x_3^2
	  + x_2^3 + x_2^2 x_3 + x_2 x_3^2
	  + x_3^3 \\
    p_3(x_1, x_2, x_3)
      &= x_1^3 + x_2^3 + x_3^3
  \end{align*}
  Tenemos los siguientes casos especiales:
  \begin{align}
    e_0(x_1, x_2, \dotsc, x_n)
      &= 1 \label{eq:e0=1} \\
    h_0(x_1, x_2, \dotsc, x_n)
      &= 1 \label{eq:h0=1} \\
    p_0(x_1, x_2, \dotsc, x_n)
      &= n \label{eq:p0=n}
  \end{align}
  Definamos las funciones generatrices:%
    \index{generatriz!ordinaria}%
    \index{polinomio!simetrico@simétrico!generatriz}
  \begin{align}
    E(t)
      &= \sum_{m \ge 0} e_m (x_1, x_2, \dotsc, x_n) t^m \\
      &= \prod_{1 \le k \le n} (1 + x_k t)
	 \label{eq:symmetric-polynomial-e-GF} \\
    H(t)
      &= \sum_{m \ge 0} h_m(x_1, x_2, \dotsc, x_n) t^m \\
      &= \prod_{1 \le k \le n} \frac{1}{1 - x_k t}
	 \label{eq:symmetric-polynomial-h-GF} \\
    P(t)
      &= \sum_{m \ge 0} p_{m + 1} (x_1, x_2, \dotsc, x_n) t^m \\
      &= \sum_{1 \le k \le n} \frac{x_k}{1 - x_k t}
	 \label{eq:symmetric-polynomial-p-GF}
  \end{align}
  Las fórmulas dadas debieran estar claras:
  En \(E(t)\) contribuyen al coeficiente de \(t^m\)
  los factores para \(m\) variables diferentes;
  en \(H(t)\)
  vemos que al expandir las series geométricas de cada factor
  estas dan la variable elevada a cada posible potencia,
  y las combinaciones posibles que dan \(t^m\)
  son exactamente
  las indicadas en~\eqref{eq:symmetric-polynomial-h};
  mientras en \(P(t)\) el coeficiente de \(t^m\)
  proviene de la suma de todas las variables elevadas a \(m + 1\).

  Hagamos uso de estas funciones generatrices ahora.
  De~\eqref{eq:symmetric-polynomial-e-GF}
  y~\eqref{eq:symmetric-polynomial-h-GF} está claro que:
  \begin{equation*}
    E(t) H(-t)
      = 1
  \end{equation*}
  Comparando coeficientes
  (al lado derecho tenemos \(1 + 0 z + 0 z^2 + \dotsb\)):
  \begin{equation}
    \label{eq:symmetric-polynomial-e-h}
    \sum_{0 \le r \le m}
      (-1)^{m - r} e_r (x_1, \dotsc, x_n) \, h_{m - r} (x_1, \dotsc, x_n)
      = [m = 1]
  \end{equation}
  También vemos que:
  \begin{align*}
    \ln E(t)
      &= \sum_{1 \le k \le n} \ln (1 + x_k t) \\
    \frac{E'(t)}{E(t)}
      &= \sum_{1 \le k \le n} \frac{x_k}{1 + x_k t} \\
      &= P(-t) \\
    E'(t)
      &= E(t) P(-t)
  \end{align*}
  De acá,
  comparando coeficientes:
  \begin{align}
    (-1)^m m e_m (x_1, \dotsc, x_n)
      &= \sum_{0 \le r \le m - 1}
	   e_r (x_1, \dotsc, x_n)
	     \cdot (-1)^{m - 1 - r} p_{m - r} (x_1, \dotsc, x_n)
	 \notag \\
    m e_m (x_1, \dotsc, x_n)
      &= \sum_{0 \le r \le m - 1}
	   (-1)^{r - 1} e_r (x_1, \dotsc, x_n)
	     \, p_{m - r} (x_1, \dotsc, x_n)
    \label{eq:symmetric-polynomial-e-p}
  \end{align}
  Similarmente:
  \begin{align*}
    \ln H(t)
      &= - \sum_{1 \le k \le n} \ln (1 - x_k t) \\
    \frac{H'(t)}{H(t)}
      &= P(t) \\
    H'(t)
      &= H(t) P(t)
  \end{align*}
  Igual que antes:
  \begin{equation}
    \label{eq:symmetric-polynomial-h-p}
    m h_m (x_1, \dotsc, x_n)
      = \sum_{0 \le r \le m - 1}
	   h_r (x_1, \dotsc, x_n) \, p_{m - r} (x_1, \dotsc, x_n)
  \end{equation}
  Siquiera sospechar
  las relaciones~\eqref{eq:symmetric-polynomial-e-h},
  \eqref{eq:symmetric-polynomial-e-p}
  y~\eqref{eq:symmetric-polynomial-h-p}
  de alguna otra forma sería sobrehumano.

  Otro ejemplo lo ofrecen las \emph{fuentes}
  (\emph{\selectlanguage{english}{fountain}} en inglés),%
    \index{fountain@\emph{\selectlanguage{english}{fountain}}|see{fuente}}
  formadas por filas de monedas
  de forma que cada moneda esté en contacto
  con dos monedas de la fila inferior.
  Si la fuente es tal que las monedas en cada fila están contiguas,
  se les llama \emph{fuentes de bloque}%
    \index{fuente}%
  (en inglés \emph{\foreignlanguage{english}{block fountain}}).%
    \index{block fountain@\emph{\selectlanguage{english}{block fountain}}|see{fuente}}
  \begin{figure}[ht]
    \centering
    \pgfimage{images/fountain}
    \caption{Una fuente de bloque}
    \label{fig:fountain}
  \end{figure}
  La figura~\ref{fig:fountain} ilustra una fuente de bloque.
  Interesa saber el número de fuentes de bloque
  cuya primera fila
  (su base)
  tiene \(n\)~monedas,
  llamémosle \(f_n\) a este número.

  Un poco de experimentación lleva a \(f_0 = 1\)
  (hay una única forma de armar una fuente con base 0),
  \(f_1 = 1\),
  \(f_2 = 2\)
  y \(f_3 = 5\).
  Es claro que si a una fuente con base \(n\)~monedas
  le quitamos la base,
  queda una fuente con base a lo más \(n - 1\)~monedas.
  Si no hay monedas en la segunda fila,
  hay una sola fuente;
  si es \(k \ge 1\) el largo de la segunda fila de monedas,
  tenemos una fuente de base~\(k\) a partir de la segunda fila
  y esta fuente puede ubicarse sobre la base
  en \((n - 1) - k + 1 = n - k\) posiciones.
  En consecuencia
  tenemos la recurrencia:%
    \index{fuente!recurrencia}
  \begin{equation}
    \label{eq:recurrence-fuentes}
    f_n
      = 1 + \sum_{1 \le k \le n} (n - k) f_k
      \quad (n \ge 1)
      \qquad
      f_0
	= 1
  \end{equation}
  Esto da los valores:
  \begin{equation}
    \label{eq:fountains-sequence}
    \langle 1, 1, 2, 5, 13, 34, 89, 233, 610, 1597, \dotsc \rangle
  \end{equation}
  La sumatoria en~\eqref{eq:recurrence-fuentes}
  es la convolución de \(\langle n \rangle_{n \ge 1}\)
  con \(\langle f_n \rangle_{n \ge 1}\).
  Definimos la función generatriz ordinaria:
  \begin{equation}
    \label{eq:gf-fountain}
    f(z)
      = \sum_{n \ge 0} f_n z^n
  \end{equation}
  Como:
  \begin{equation*}
    \frac{1}{1 - z} - 1
      \ogf \langle 1 \rangle_{n \ge 1}
    \hspace{3em}
    \frac{1}{(1 - z)^2} - 1
      \ogf \langle n \rangle_{n \ge 1}
    \hspace{3em}
    f(z) - 1
      \ogf \langle f_n \rangle_{n \ge 1}
  \end{equation*}
  aplicando las propiedades
  de las funciones generatrices ordinarias resulta:%
    \index{fuente!generatriz}
  \begin{equation}
    \label{eq:fe-fountains}
    f(z) - 1
      = \frac{z}{1 - z} + \frac{z}{(1 - z)^2} \cdot (f(z) - 1)
  \end{equation}
  Despejando \(f(z)\) obtenemos:
  \begin{equation}
    f(z)
      = \frac{1 - 2 z}{1 - 3 z + z^2}
      = \frac{5 + \sqrt{5}}{10}
	  \cdot \frac{1}{1 - z \frac{3 - \sqrt{5}}{2}}
	+ \frac{5 - \sqrt{5}}{10}
	    \cdot \frac{1}{1 - z \frac{3 + \sqrt{5}}{2}}
		  \label{eq:gf-fountain-pf}
  \end{equation}
  Ciertamente bastante feo,
  pero da lugar a la expansión explícita:
  \begin{equation}
    \label{eq:seq-fountain}
    f_n
      = \frac{5 + \sqrt{5}}{10}
	    \, \left( \frac{3 - \sqrt{5}}{2} \right)^n
	  + \frac{5 - \sqrt{5}}{10}
	      \, \left( \frac{3 + \sqrt{5}}{2} \right)^n
  \end{equation}

  Una forma instructiva
  de obtener el número de subconjuntos de \(k\) elementos%
    \index{conjunto!subconjunto!numero@número}
  de un conjunto de \(n\) elementos
  es partir de la recurrencia
  (ver el teorema~\ref{theo:identidad-Pascal}):
  \begin{equation}
    \label{eq:recurrence-subset}
    \binom{n + 1}{k + 1}
      = \binom{n}{k + 1} + \binom{n}{k}
    \hspace{3em} \binom{0}{k} = [k = 0]
    \hspace{3em} \binom{n}{0} = 1
  \end{equation}
  Si definimos la función generatriz bivariada:%
    \index{generatriz!bivariada}
  \begin{equation}
    \label{eq:gf-subset}
    C(x, y)
      = \sum_{\substack{k \ge 0 \\ n \ge 0}}
	  \binom{n}{k} \, x^k y^n
  \end{equation}
  Por las propiedades
  de las funciones generatrices ordinarias resulta:
  \begin{equation}
    \label{eq:fe-subsets}
    \frac{C(x, y) - C(0, y) - C(x, 0) + C(0, 0)}{x y}
      = C(x, y) + \frac{C(x, y) - C(0, y)}{x}
  \end{equation}
  El lado izquierdo es partir de la suma~\eqref{eq:gf-subset},
  restar la fila con \(k = 0\)
  y la columna con \(n = 0\);
  pero al hacerlo hay que reponer el coeficiente con \(k = n = 0\)
  que se restó dos veces.
  Luego se divide por \(x y\) para ajustar los exponentes.%
    \index{inclusion y exclusion, principio de@inclusión y exclusión, principio de}
  Por las condiciones de contorno:
  \begin{equation}
    \label{eq:binomial-boundary-gf}
    C(0, 0)
      = 1
    \hspace{3em}
    C(x, 0)
      = \sum_{k \ge 0} \binom{0}{k} \, x^k
      = 1
    \hspace{3em}
    C(0, y)
      = \sum_{n \ge 0} \binom{n}{0} \, y^n
      = \frac{1}{1 - y}
  \end{equation}
  Resulta nuevamente
  la función generatriz~\eqref{eq:sum-binomial-double}.

  La presente discusión se inspira en Bender~%
    \cite{bender06:_found_combin_applic}.
  Consideremos árboles binarios ordenados completos,%
    \index{arbol binario@árbol binario}
  definidos mediante:
  \begin{enumerate}[label=(\roman*), ref=\roman*]
  \item
    Un vértice aislado es un árbol binario ordenado completo
    (esta es la raíz del árbol y su única hoja)
  \item
    Si \(T_1\) y \(T_2\) son árboles binarios ordenados completos,
    lo es la estructura que agrega un nuevo nodo como raíz
    y pone la raíz de \(T_1\) como descendiente izquierdo de la raíz
    y la raíz de \(T_2\) como su descendiente derecho.
  \end{enumerate}
  Nos interesa determinar
  el número de estas estructuras con \(n\) hojas,
  que llamaremos \(b_n\).
  Claramente \(b_0 = 0\) y \(b_1 = 1\).
  Para \(n > 1\),
  tendremos dos subárboles;
  si el izquierdo aporta \(k\) hojas
  el derecho aporta \(n - k\),
  y el número de árboles que podemos crear en esta situación,
  por la regla del producto es
  \(b_k b_{n - k}\).
  Pero debemos considerar todos los posibles valores de \(k\),
  la regla de la suma nos dice para \(n > 1\):
  \begin{equation}
    \label{eq:UFBRPT-recurrence}
    b_n
      = \sum_{1 \le k \le n - 1} b_k b_{n - k}
  \end{equation}
  Fácilmente podemos calcular los primeros valores:
  \begin{equation}
    \label{eq:UFBRPT-values}
    \langle 0, 1, 1, 2, 5, 14, 42, 132, \dotsc \rangle
  \end{equation}
  Si definimos
  la función generatriz ordinaria \(B(z)\) de los \(b_n\),
  aplicando nuestra receta queda para \(n > 1\):
  \begin{align}
    \sum_{n \ge 2} b_n z^n
      &= \sum_{n \ge 2} z^n \sum_{1 \le k \le n - 1} b_k b_{n - k}
				  \notag \\
    B(z) - b_0 - b_1 z
      &= \sum_{n \ge 2}
	   \sum_{1 \le k \le n - 1}
	     b_k z^k \cdot b_{n - k} z^{n - 1 - k}
				  \notag \\
      &= \left( \sum_{k \ge 1} b_k z^k \right)
	    \cdot \left( \sum_{k \ge 1} b_k z^k \right)
				  \notag \\
      &= \left( B(z) - b_0 \right)^2
				  \notag \\
    B(z) - z
      &= B^2(z)
				  \label{eq:UFBRPT-functional}
  \end{align}
  De~\eqref{eq:UFBRPT-functional} resulta:
  \begin{equation*}
    B(z)
      = \frac{1 \pm \sqrt{1 - 4 z}}{2}
  \end{equation*}
  Como debe ser \(b_0 = 0\),
  el signo correcto es el negativo:%
    \index{generatriz!ordinaria}%
    \index{arbol binario@árbol binario!generatriz}
  \begin{equation}
    \label{eq:eq:UFBRPT-gf}
    B(z)
      = \frac{1 - \sqrt{1 - 4 z}}{2}
  \end{equation}
  Expandiendo la raíz mediante el teorema del binomio:
  \begin{align}
    B(z)
      &= \frac{1}{2}
	   \left(
	     1 - \left(
		   1 + \sum_{n \ge 1}
			 \frac{(-1)^{n - 1}}{n 2^{2 n - 1}} \,
			   \binom{2 n - 2}{n - 1}
			     \cdot (-1)^n \cdot 2^{2 n} \cdot z^n
		 \right)
	   \right) \notag \\
      &= \sum_{n \ge 1}
	   \frac{1}{n} \, \binom{2 n - 2}{n - 1} \, z^n \notag \\
      &= \sum_{n \ge 1} C_{n - 1} z^n
	   \label{eq:UFBRPT-series-gf}
  \end{align}
  Nuevamente números de Catalan.%
    \index{Catalan, numeros de@Catalan, números de}
  De la suma~\eqref{eq:UFBRPT-series-gf}
  y la condición \(b_0 = 0\) tenemos:
  \begin{equation}
    \label{eq:UFBRPT-number}
    b_n
      = \begin{cases}
	  0	    & \text{si \(n = 0\)} \\
	  C_{n - 1} & \text{si \(n \ge 1\)}
	\end{cases}
  \end{equation}

  Si contamos el número de maneras de crear palabras de \(n\) letras
  usando únicamente \(\mathrm{A}\) y \(\mathrm{B}\),
  es claro que esto corresponde
  a elegir \(k\) posiciones para las \(\mathrm{A}\)
  (y dejar las \(n - k\) restantes a llenar por \(\mathrm{B}\)).
  Si hay \(a_k\) maneras de tener \(k\) letras \(\mathrm{A}\)
  y \(b_k\) maneras de tener \(k\) letras \(\mathrm{B}\),
  vemos que el total es:
  \begin{equation*}
    \sum_{0 \le k \le n} \binom{n}{k} a_k b_{n - k}
  \end{equation*}
  Una convolución binomial.
  Deberemos multiplicar las funciones generatrices exponenciales
  de las secuencias \(\langle a_n \rangle_{n \ge 0}\)
  y \(\langle b_n \rangle_{n \ge 0}\)
  para obtener la función generatriz exponencial
  del número de palabras posibles.
    \index{generatriz!exponencial}
  Por ejemplo,
  si la restricción es que el número de \(\mathrm{A}\) es par
  y no hay restricciones para las \(\mathrm{B}\),
  las funciones generatrices respectivas son:
  \begin{equation*}
    \widehat{A}(z)
      = 1 + \frac{z^2}{2!} + \frac{z^4}{4!} + \dotsb
      = \cosh z
    \hspace{3em}
    \widehat{B}(z)
      = 1 + \frac{z}{1!} + \frac{z^2}{2!} + \dotsb
      = \mathrm{e}^z
  \end{equation*}
  Resulta:
  \begin{equation*}
    n! \left[ z^n \right] \mathrm{e}^z \cosh z
      = n! \left[ z^n \right] \frac{\mathrm{e}^{2 z} + 1}{2}
      = \frac{n!}{2} \left( \frac{2^n}{n!} + 1 \right)
      = 2^{n - 1} + \frac{n!}{2}
  \end{equation*}

  Si volvemos a enfrentarnos al temible \(\mathrm{BOOKKEEPER}\),
  para calcular cuántas palabras de \(n\) letras podemos formar,
  consideramos el multiconjunto
    \(\{ \mathrm{B}^1, \mathrm{E}^3, \mathrm{K}^2, \mathrm{O}^2,
	 \mathrm{P}^1, \mathrm{R}^1 \}\),
  las funciones generatrices exponenciales
  de cada letra dan los factores:
  \begin{equation*}
    P(z)
      = \left( 1 + z \right)
	  \cdot \left(
		  1 + z + \frac{z^2}{2!} + \frac{z^3}{3!}
		\right)
	  \cdot \left( 1 + z + \frac{z^2}{2!} \right)
	  \cdot \left( 1 + z + \frac{z^2}{2!} \right)
	  \cdot \left( 1 + z \right)
	  \cdot \left( 1 + z \right)
  \end{equation*}
  y para obtener el número de palabras de \(n\) letras es
  \(n! \left[ z^n \right] P(z)\).
  La función generatriz se expande a:
  \begin{equation*}
    P(z)
      = 1 + 6 z + \frac{33}{2} z^2 + \frac{83}{3} z^3
	  + \frac{379}{12} z^4 + \frac{155}{6} z^5
	  + \frac{371}{24} z^6 + \frac{27}{4} z^7
	  + \frac{25}{12} z^8 + \frac{5}{12} z^9
	  + \frac{1}{24} z^{10}
  \end{equation*}
  O sea,
  las posibles palabras de 6 letras son:
  \begin{equation*}
    6! \, \left[ z^6 \right] P(z)
      = 720 \cdot \frac{371}{24}
      = 11\,130
  \end{equation*}
  Un momento de reflexión muestra que el coeficiente principal%
    \index{polinomio!coeficiente principal}
  (del término de máximo grado)
  en esta expansión es una explicación alternativa del Tao
  (sección~\ref{sec:tao-bookkeeper}).

\section{Aceite de serpiente}
\label{sec:snake-oil}
\index{generatriz!aceite de serpiente}

  La manera tradicional de simplificar sumatorias
  (particularmente las que involucran coeficientes binomiales)
  es aplicar identidades u otras manipulaciones de los índices,
  como magistralmente exponen Knuth~%
    \cite{knuth97:_fundam_algor}
  y Graham, Knuth y~Patashnik~%
    \cite{graham94:_concr_mathem}.
  Acá mostramos un método alternativo,
  que no requiere saber y aplicar
  una enorme variedad de identidades.
  Wilf~%
    \cite{wilf06:_gfology}
  le llama \emph{\foreignlanguage{english}{Snake Oil Method}},
  por la cura milagrosa que se ve en las películas del viejo oeste.
  La técnica es bastante simple:
  \begin{enumerate}
  \item
    Identificar la variable libre,
    llamémosle \(n\),
    de la que depende la suma.
    Sea \(f(n)\) nuestra suma.
  \item
    Sea \(F(z)\) la función generatriz ordinaria
    de la secuencia \(\langle f(n) \rangle_{n \ge 0}\).
  \item
    Multiplique la suma por \(z^n\) y sume sobre \(n\).
    Tenemos \(F(z)\) expresado como una doble suma,
    sobre \(n\) y la variable de la suma original.
  \item
    Intercambie el orden de las sumas,
    y exprese la suma interna en forma simple y cerrada.
  \item
    Encuentre los coeficientes,
    son los valores de \(f(n)\) buscados.
  \end{enumerate}
  Sorprende la alta tasa de éxitos de la técnica.
  Tiene la ventaja de que no requiere mayor creatividad;
  resulta claro cuándo funciona
  y es obvio cuando falla.

  Usaremos la convención que toda suma sin restricciones
  es sobre el rango \(-\infty\) a \(\infty\).
  Como los coeficientes binomiales \(\binom{n}{k}\)
  que usaremos en los ejemplos se anulan cuando
  \(k\) no está en el rango \([0, n]\),
  esto evita interminables ajustes de índices.
  Por ejemplo,
  para \(n \ge 0\) tenemos:
  \begin{equation*}
    \sum_k \binom{n}{r + k} \, z^k
      = z^{-r} \, \sum_k \binom{n}{r + k} \, z^{r + k}
      = z^{-r} \, \sum_s \binom{n}{s} \, z^s
      = z^{-r} (1 + z)^n
  \end{equation*}

  \begin{example}
    Evaluar:
    \begin{equation*}
      \sum_k \binom{k}{n - k}
    \end{equation*}

    La variable libre es \(n\),
    llamamos \(g(n)\) a nuestra suma
    y a su función generatriz \(G(z)\).
    Multiplicamos por \(z^n\) y sumamos:
    \begin{align*}
      G(z)
	&= \sum_n \sum_k \binom{k}{n - k} \, z^n
	 = \sum_k \sum_n \binom{k}{n - k} \, z^n
	 = \sum_k z^k \, \sum_n \binom{k}{n - k} \, z^{n - k}
	 = \sum_k z^k \, \sum_r \binom{k}{r} \, z^r \\
	&= \sum_k z^k (1 + z)^k
	 = \frac{1}{1 - z (1 + z)}
	 = \frac{1}{1 - z - z^2}
    \end{align*}
    Esto se parece sospechosamente
    a la función generatriz~\eqref{eq:gf-Fibonacci}
    de los números de Fibonacci%
      \index{Fibonacci, numeros de@Fibonacci, números de}
    que discutiremos en la sección~\ref{sec:Fibonacci},
    como \(F_0 = 0\) vemos que:
    \begin{equation*}
      G(z)
	= \frac{F(z) - F_0}{z}
    \end{equation*}
    Por las propiedades de las funciones generatrices ordinarias,
    es \(g(n) = F_{n + 1}\).
  \end{example}
  Nuestro siguiente problema viene de Riordan~%
    \cite{riordan68:_combin_ident},
  donde se resuelve mediante delicadas maniobras.
  Nuestro desarrollo sigue a Dobrushkin~%
    \cite{dobrushkin10:_method_algor_analysis}.
  \begin{example}
    Evaluar:
    \begin{equation*}
      h_n
	= \sum_{0 \le k \le n}
	    (-1)^{n - k} \, 4^k \, \binom{n + k + 1}{2 k + 1}
    \end{equation*}

    Definimos \(H(z)\) como la función generatriz de los \(h_n\);
    multiplicamos por \(z^n\),
    sumamos para \(n \ge 0\)
    e intercambiamos orden de suma:
    \begin{align*}
      H(z)
	&= \sum_{n \ge 0} z^n
	     \sum_{0 \le k \le n}
	       (-1)^{n - k} \, 4^k \, \binom{n + k + 1}{2 k + 1} \\
	&= \sum_{n \ge 0}
	     \sum_{0 \le k \le n}
	       (-4)^k \, (-z)^n \, \binom{n + k + 1}{2 k + 1} \\
	&= \sum_{k \ge 0}
	     (-4)^k \,
	     \sum_{n \ge k}
	       \binom{n + k + 1}{2 k + 1} \, (-z)^n
    \end{align*}
    Para completar el trabajo necesitamos la suma interna.
    Haciendo el cambio de variable \(r = n - k\):
    \begin{equation*}
      \sum_{n \ge k} \binom{n + k + 1}{2 k + 1} \, (-z)^n
	= (-z)^k \, \sum_{r \ge 0}
		      \binom{r + 2 k + 1}{2 k + 1} \, (-z)^r
	= \frac{(-z)^k}{(1 + z)^{2 k + 2}}
    \end{equation*}
    Substituyendo en lo anterior:
    \begin{equation*}
      H(z)
	= \sum_{k \ge 0} \frac{(4 z)^k}{(1 + z)^{2 k + 2}}
	= \frac{1}{(1 + z)^2}
	    \cdot \frac{1}{1 - \frac{4 z}{(1 + z)^2}}
	= \frac{1}{(1 - z)^2}
    \end{equation*}
    Resta extraer los coeficientes,
    lo que da:
    \begin{equation*}
      h_n
	= (-1)^n \, \binom{-2}{n}
	= \binom{n + 1}{1}
	= n + 1
    \end{equation*}
  \end{example}
  La siguiente es una sumatoria que le dio problemas a Knuth,%
    \index{Knuth, Donald E.}
  como comenta en el prefacio
  del texto de Petkovšek, Wilf y Zeilberger~%
    \cite{petkovsek96:_AeqB}.
  \begin{example}
    Considere la suma:
    \begin{equation*}
      \sum_k \binom{2 n - 2 k}{n - k} \, \binom{2 k}{k}
    \end{equation*}
    Sabemos que la secuencia
    comienza \(\langle 1, 4, 16, 64, \dotsc\rangle\),
    por lo que sospechamos que la suma vale \(4^n\).

    Aplicando la receta,
    con \(s(n)\) la suma que nos interesa
    y \(S(z)\) la respectiva función generatriz:
    \begin{equation*}
      S(z)
	= \sum_{n \ge 0}
	    \sum_{0 \le k \le n}
	      \binom{2 n - 2 k}{n - k} \, \binom{2 k}{k} \, z^n
	= \left( \sum_{n \ge 0} \binom{2 n}{n} \, z^n \right)^2
    \end{equation*}
    En este caso
    (como en todas las convoluciones)
    la sumatoria externa simplemente se disuelve sola.
    La serie interna es~\eqref{eq:serie-reciproco-raiz}:
    \begin{equation*}
      S(z)
	= \left( \frac{1}{\sqrt{1 - 4 z}} \right)^2
	= \frac{1}{1 - 4 z}
    \end{equation*}
    Una serie geométrica,
    y el resultado \(s(n) = 4^n\) es inmediato.
  \end{example}
  \begin{example}
    Determine el valor de:
    \begin{equation*}
      \sum_{0 \le k \le n} \binom{n}{k}^2
    \end{equation*}

    Esto es esencialmente una convolución:
    \begin{equation*}
      \sum_{0 \le k \le n} \binom{n}{k} \, \binom{n}{n - k}
    \end{equation*}
    Acá producen problemas los distintos usos de \(n\),
    delimita el rango de la suma
    y aparece en los índices superiores
    de los coeficientes binomiales.
    Una solución en tales casos
    es intentar demostrar algo más general.
    Dividiendo los distintos usos de \(n\) en variables separadas
    queda:
    \begin{equation*}
      \sum_{0 \le k \le r} \binom{m}{k} \, \binom{n}{r - k}
    \end{equation*}
    Ahora hay varios índices libres,
    debemos elegir uno.
    Es una convolución,
    lo que hace sospechar que \(r\) es útil como variable libre.
    Así llamamos \(v(r)\) a nuestra suma,
    y su función generatriz \(V(z)\).
    \begin{align*}
      V(z)
	&= \sum_{r \ge 0}
	     z^r \, \sum_k \binom{m}{k} \, \binom{n}{r - k} \\
	&= \left( \sum_{k \ge 0} \binom{m}{k} \, z^k \right)
	     \cdot \left(
		     \sum_{k \ge 0} \binom{n}{k} \, z^k
		   \right) \\
	&= (1 + z)^m \, (1 + z)^n \\
	&= (1 + z)^{m + n}
    \end{align*}
    En consecuencia,
    tenemos la \emph{convolución de Vandermonde}%
      \footnote{Otro caso de injusticia histórica:
		Unos 400~años antes de Vandermonde
		la conocía Zhu Shije en China~%
		   \cite[páginas 59--60]
			{askey75:_orthogonal_poly_special_functions}.}:
    \begin{equation}
      \label{eq:Vandermonde-convolution}
      \index{Vandermonde, convolucion de@Vandermonde, convolución de}
      \sum_k \binom{m}{k} \, \binom{n}{r - k}
	= \binom{m + n}{r}
    \end{equation}
    que también puede escribirse en la forma simétrica:
    \begin{equation}
      \label{eq:Vandermonde-convolution-symmetric}
      \sum_k \binom{m}{r + k} \binom{n}{s - k}
	= \binom{m + n}{r + s}
    \end{equation}
    Acá la suma es sobre todo \(k \in \mathbb{Z}\),
    pero sólo para \(-r \le k \le s\) los términos no son cero.
    Indicarlo destruiría la simetría de la fórmula.
    Nótese además que nuestra demostración es aplicable
    también en caso que \(m\) o \(n\) no sean naturales.

    Nuestra suma original
    es simplemente el caso especial \(m = n = r\)
    de~\eqref{eq:Vandermonde-convolution}:
    \begin{equation}
      \label{eq:sum-square-binom}
      \sum_k \binom{n}{k}^2
	= \sum_k \binom{n}{k} \, \binom{n}{n - k}
	= \binom{2 n}{n}
    \end{equation}
    Nuevamente un caso de la paradoja del inventor.%
      \index{paradoja del inventor}
  \end{example}
  Un ejemplo propuesto por Liu~\cite{liu68:_introd_combin_mathem},
  que resuelve de forma afín a la nuestra:
  \begin{example}
    Calcular la suma:
    \begin{equation}
      \label{eq:so:binomial-convolution}
      S_r
	= \sum_{0 \le i \le r} \frac{r!}{(r - i + 1)! (i + 1)!}
    \end{equation}
    Vemos que los términos son sospechosamente similares
    a coeficientes binomiales:
    \begin{equation*}
      S_r
	= \sum_{0 \le i \le r}
	    \binom{r}{i} \frac{1}{r - i + 1} \frac{1}{i + 1}
    \end{equation*}
    Esta es una convolución binomial,%
      \index{convolucion binomial@convolución binomial}
    lo que sugiere la función generatriz exponencial:%
      \index{generatriz!exponencial}
    \begin{equation}
      \label{eq:so:binomial-convolution:egf}
      \widehat{S}(z)
	= \sum_{r \ge 0} S_r \frac{z^r}{r!}
    \end{equation}
    Vemos que:
    \begin{align*}
      \widehat{S}(z)
	&= \left(
	     \sum_{r \ge 0} \frac{1}{r + 1} \cdot \frac{z^r}{r!}
	   \right)^2 \\
	&= \left(
	     \sum_{r \ge 0} \frac{z^r}{(r + 1)!}
	   \right)^2 \\
	&= \left(
	     \frac{\mathrm{e}^z - 1}{z}
	   \right)^2 \\
	&= \frac{\mathrm{e}^{2 z} - 2 \mathrm{e}^z + 1}{z^2}
    \end{align*}
    Los coeficientes son inmediatos:
    \begin{align}
      S_r
	&= r! [z^r] \widehat{S}(z) \notag \\
	&= r! [z^r]
	   \frac{\mathrm{e}^{2 z} - 2 \mathrm{e}^z + 1}
		{z^2} \notag \\
	&= r! [z^{r + 2}]
	     \left(\mathrm{e}^{2 z} - 2 \mathrm{e}^z + 1 \right)
		 \notag \\
	&= r! \left(
		\frac{2^{r + 2}}{(r + 2)!}
		  - \frac{2}{(r + 2)!}
	      \right) \notag \\
	&= \frac{2^{r + 2} - 2}{(r + 1) (r + 2)}
	     \label{eq:so:binomial-convolution:coef}
    \end{align}
  \end{example}
  Finalmente,
  una identidad.
  \begin{example}
    Demostrar que para \(m, n \ge 0\)
    \begin{equation}
      \label{eq:so:binomial-identity}
      \sum_{k \ge 0} \binom{m}{k} \binom{n + k}{m}
	= \sum_{k \ge 0} \binom{m}{k} \binom{n}{k} \, 2^k
    \end{equation}
    Multiplicamos
    ambos lados de~\eqref{eq:so:binomial-identity} por \(z^n\)
    y sumamos,
    obteniendo la identidad \(L(z) = R(z)\)
    al igualar lado izquierdo con derecho:
    \begin{equation*}
      L(z)
	= \sum_{n \ge 0} z^n
	    \sum_{k \ge 0} \binom{m}{k} \binom{n + k}{m}
      \hspace{3em}
      R(z)
	= \sum_{n \ge 0} z^n
	    \sum_{k \ge 0} \binom{m}{k} \binom{n}{k} \, 2^k
    \end{equation*}
    Partimos por el lado izquierdo:
    \begin{align*}
      L(z)
	&= \sum_{k \ge 0} \binom{m}{k} z^{-k}
	     \sum_{n \ge 0} \binom{n + k}{m} z^{n + k} \\
    \intertext{Por la suma externa sabemos que \(0 \le k \le m\),
	       como en realidad la suma interna
	       es para \(n + k \ge m\)
	       podemos aplicar~\eqref{eq:serie-binomio-n}:}
      L(z)
	&= \sum_{k \ge 0} \binom{m}{k} z^{-k} \,
	     \frac{z^m}{(1 - z)^{m + 1}} \\
	&= \left( 1 + \frac{1}{z} \right)^m \,
	     \frac{z^m}{(1 - z)^{m + 1}} \\
	&= \frac{(1 + z)^m}{(1 - z)^{m + 1}}
    \end{align*}
    El lado derecho recibe un tratamiento similar:
    \begin{align*}
      R(z)
	&= \sum_{k \ge 0} \binom{m}{k} \, 2^k \,
	     \sum_{n \ge 0} \binom{n}{k} z^n \\
	&= \sum_{k \ge 0}
	     \binom{m}{k} \, 2^k \,\frac{z^k}{(1 - z)^{k + 1}} \\
	&= \frac{1}{1 - z} \,
	     \sum_{k \ge 0} \binom{m}{k}
	       \left( \frac{2 z}{1 - z} \right)^k \\
	&= \frac{1}{1 - z}
	     \, \left( 1 + \frac{2 z}{1 - z} \right)^m \\
	&= \frac{(1 + z)^m}{(1 - z)^{m + 1}}
    \end{align*}
    Se verifica la identidad.
  \end{example}

  Hay métodos complementarios,
  capaces de resolver automáticamente grandes clases de sumatorias,
  o demostrar que no hay expresiones simples para ellas.
  Petkovšek, Wilf y Zeilberger~%
    \cite{petkovsek96:_AeqB}
  los describen en detalle,
  y hay implementaciones de los mismos
  para los principales paquetes de álgebra simbólica.
  Cipra~\cite{cipra89:_grinch_stole_math}
  incluso se queja que estas demostraciones automatizadas
  quitan la entretención a las matemáticas.

%%% Local Variables:
%%% mode: latex
%%% TeX-master: "clases"
%%% End:


% principio-inclusion-exclusion.tex
%
% Copyright (c) 2009-2014 Horst H. von Brand
% Derechos reservados. Vea COPYRIGHT para detalles

\chapter{Principio de inclusión y exclusión}
\label{cha:pie}
\index{inclusion y exclusion, principio de@inclusión y exclusión, principio de}
\index{principio de inclusion y exclusion@principio de inclusión y exclusión|see{inclusión y exclusión, principio de}}

  Es común querer contar el número de objetos de una colección
  que cumplen con ciertos conjuntos de características.
  Si las características de interés son muchas,
  o la colección de objetos subyacente es grande,
  necesitamos un esquema que organice y simplifique
  los cálculos.
  Veremos el planteo de Wilf~%
    \cite{wilf06:_gfology},
  que además de ser mucho más simple que el tradicional
  aprovecha de buena forma
  lo que hemos aprendido de funciones generatrices.
  Con esto cerramos el estudio de las técnicas fundamentales
  de la combinatoria.

\section{El problema general}
\label{sec:PIE-problema-general}

  Concluimos en el capítulo~\ref{cha:combinatoria-elemental}
  que
  \(\lvert \mathcal{A} \cup \mathcal{B} \rvert
      = \lvert \mathcal{A} \rvert
	   + \lvert \mathcal{B} \rvert
	   - \lvert \mathcal{A} \cap \mathcal{B} \rvert
  \).
  Interesa generalizar para más conjuntos.
  La figura~\ref{fig:PIE} muestra tres conjuntos
  y sus intersecciones.
  \begin{figure}[htbp]
    \centering
    \pgfimage{images/PIE}
    \caption{Intersecciones entre tres conjuntos}
    \label{fig:PIE}
  \end{figure}
  Calcular
    \(\lvert \mathcal{A} \cup \mathcal{B} \cup \mathcal{C} \rvert\)
  es contar los elementos
  que pertenecen al menos a uno de los conjuntos.
  Comenzamos con \(\lvert \mathcal{A} \rvert
		     + \lvert \mathcal{B} \rvert
		     + \lvert \mathcal{C} \rvert\).
  Las intersecciones se cuentan dos veces,
  debemos restar
  \(\lvert \mathcal{A} \cap \mathcal{B} \rvert
      + \lvert \mathcal{A} \cap \mathcal{C} \rvert
      + \lvert \mathcal{B} \cap \mathcal{C} \rvert\).
  Hemos restado
    \(\lvert \mathcal{A}
	\cap \mathcal{B}
	\cap \mathcal{C} \rvert\) demás,
  debemos restituirlo:
  \begin{equation*}
    \lvert \mathcal{A} \cup \mathcal{B} \cup \mathcal{C} \rvert
      = \bigl(
	  \lvert \mathcal{A} \rvert
	     + \lvert \mathcal{B} \rvert
	     + \lvert \mathcal{C} \rvert
	 \bigr)
	   - \bigl(
	       \lvert \mathcal{A} \cap \mathcal{B} \rvert
		  + \lvert \mathcal{A} \cap \mathcal{C} \rvert
		  + \lvert \mathcal{B} \cap \mathcal{C} \rvert
	     \bigr)
	   + \lvert
	       \mathcal{A} \cap \mathcal{B} \cap \mathcal{C}
	     \rvert
  \end{equation*}
  El número de elementos
  que pertenecen exactamente a uno de los conjuntos es:
  \begin{equation*}
    \left(
      \lvert \mathcal{A} \rvert
	+ \lvert \mathcal{B} \rvert
	+ \lvert \mathcal{C} \rvert
    \right)
      - 2 \cdot \left(
	    \lvert \mathcal{A} \cap \mathcal{B} \rvert
	       + \lvert \mathcal{A} \cap \mathcal{C} \rvert
	       + \lvert \mathcal{B} \cap \mathcal{C} \rvert
	  \right)
      + 3 \cdot \lvert
		  \mathcal{A} \cap \mathcal{B} \cap \mathcal{C}
		\rvert
  \end{equation*}
  Al sumar los tamaños de los tres conjuntos
  incluimos dos veces las intersecciones en pares
  y tres veces la intersección entre los tres,
  debemos restarlas;
  al restar dos veces las tres intersecciones a pares
  estamos restando seis veces
  la intersección entre los tres conjuntos,
  debemos reponerla tres veces.
  Al resultado general
  se le llama \emph{principio de inclusión y exclusión},
  porque incluimos demás,
  y luego corregimos excluyendo.

  El tratamiento que sigue no es tradicional,
  seguimos a Wilf~\cite{wilf06:_gfology}%
    \index{Wilf, Herbert S.}
  Tomamos un conjunto universo,
  y los conjuntos que consideramos
  se representan mediante propiedades%
    \index{inclusion y exclusion, principio de@inclusión y exclusión, principio de!propiedades|textbfhy}
  (un elemento pertenece a uno de los conjuntos
   si tiene la propiedad que representa a ese conjunto).
  Las diversas intersecciones quedan expresadas
  a través de los elementos
  que tienen todas las propiedades
  correspondientes a los conjuntos intersectados.

  Sean:
  \begin{description}
  \item [\(\Omega\):]
    El universo.
    Un conjunto de objetos.
  \item [\(\mathcal{P}\):]
    Un conjunto de propiedades que los objetos pueden tener.
  \item [\(\mathcal{S}\):]
    Un subconjunto de las propiedades,
    \(\mathcal{S} \subseteq \mathcal{P}\).
  \item [\(N ( \supseteq \mathcal{S} )\):]
    Número de objetos con las propiedades en \(\mathcal{S}\)
    (puedan tener otras).
  \end{description}

  Para \(r \ge 0\) definimos:
  \begin{equation}
    \label{eq:PIE:definicion-Nr}
    N_r
      = \sum_{\lvert \mathcal{S} \rvert = r}
	  N(\supseteq \mathcal{S})
  \end{equation}
  Esto es la suma del tamaño de los conjuntos de objetos
  con al menos \(r\) de las propiedades.
  El conjunto de los objetos con al menos cero propiedades
  es el universo,
  o sea \(N_0 = \lvert \Omega \rvert\).
  Si hay \(r\) propiedades en total,
  \(N_r\) es el número de objetos con todas las propiedades.
  Estas cantidades,
  que suelen ser mucho más fáciles de calcular que lo que buscamos,
  las relacionaremos con el número de objetos
  que tienen exactamente \(t\) de las propiedades.

  Denote \(\omega \in \Omega\) un objeto,
  y llamemos \(P(\omega)\) al conjunto de propiedades de \(\omega\).
  Entonces:
  \begin{equation}
    \label{eq:PIE:Nr-objetos}
    N_r
      = \sum_{\lvert \mathcal{S} \rvert = r}
	  N(\supseteq \mathcal{S})
      = \sum_{\lvert \mathcal{S} \rvert = r}
	  \biggl(
	    \sum_{\substack{
	     \omega \in \Omega \\
	     \mathcal{S} \subseteq \mathcal{P}(\omega)
	  }} 1
	  \biggr)
      = \sum_{\omega \in \Omega}
	  \biggl(
	    \sum_{\substack{
	     \mathcal{S} \subseteq \mathcal{P}(\omega) \\
	     \lvert \mathcal{S} \rvert = r
	  }} 1
	  \biggr)
      = \sum_{\omega \in \Omega}
	   \binom{\lvert \mathcal{P}(\omega) \rvert}{r}
  \end{equation}
  En español dice:
  Si el objeto
  tiene un total
  de \(\lvert \mathcal{P}(\omega) \rvert\) propiedades,
  puedo elegir \(r\) de sus propiedades
  de \(\binom{\lvert \mathcal{P}(\omega) \rvert}{r}\) formas.

  Ahora sea \(e_t\)
  el número de objetos con exactamente \(t\) propiedades,
  es decir:
  \begin{equation}
    \label{eq:PIE:definiciom-et}
    e_t
      = \sum_{\lvert \mathcal{P}(\omega) \rvert = t} 1
  \end{equation}
  Cada uno de los \(e_t\) objetos con \(t\) propiedades
  aporta lo mismo a \(N_r\)
  (se considera una vez
   por cada subconjunto de \(r\) de sus propiedades):
  \begin{equation}
    \label{eq:PIE:Nr-et}
    N_r
      = \sum_{\omega \in \Omega}
	  \binom{\lvert \mathcal{P}(\omega) \rvert}{r}
      = \sum_{t \ge 0} \binom{t}{r} \, e_t
  \end{equation}
  Del sistema lineal~\eqref{eq:PIE:Nr-et}
  se busca
  despejar los \(e_t\).
  Para esta tarea definimos las funciones generatrices:%
    \index{generatriz!ordinaria}%
    \index{inclusion y exclusion, principio de@inclusión y exclusión, principio de!generatrices|textbfhy}
  \begin{align}
    \label{eq:PIE:definicion-E}
    E(z)
      &= \sum_{t \ge 0} e_t z^t \\
    \label{eq:PIE:definicion-N}
    N(z)
      &= \sum_{r \ge 0} N_r z^r
  \end{align}
  Substituyendo la expresión~\eqref{eq:PIE:Nr-et} para \(N_r\)
  en la definición~\eqref{eq:PIE:definicion-N} de \(N(z)\):
  \begin{equation*}
    N(z)
      = \sum_{r \ge 0} N_r z^r
      = \sum_{r \ge 0}
	  \biggl(
	    \sum_{t \ge 0} \binom{t}{r} \, e_t
	  \biggr) z^r
      = \sum_{t \ge 0} e_t
	  \biggl(\,
	    \sum_{r \ge 0} \binom{t}{r} z^r
	  \biggr)
      = \sum_{t \ge 0} e_t (1 + z)^t
      = E(1 + z)
  \end{equation*}
  De acá se tiene la fórmula central:
  \begin{equation}
    \label{eq:PIE:central}
    \index{inclusion y exclusion, principio de@inclusión y exclusión, principio de!formula central@fórmula central|textbfhy}
    E(z)
      = N(z - 1)
  \end{equation}
  De la expresión~\eqref{eq:PIE:central}
  podemos extraer el \(e_t\) que se quiera.
  Usando el teorema de Maclaurin:
  \begin{align}
    e_t
      &= \frac{1}{t!} E^{(t)}(0) \\
      &= \frac{1}{t!} N^{(t)}(-1) \\
      &= \frac{1}{t!} \sum_{r \ge 0}
			r^{\underline{t}} N_r (-1)^{r - t} \\
      &= \sum_{r \ge 0} (-1)^{r - t} \binom{r}{t} N_r
	   \label{eq:PIE:clasico}
	   \index{inclusion y exclusion, principio de@inclusión y exclusión, principio de!formula clasica@fórmula clásica}
  \end{align}
  La fórmula~\eqref{eq:PIE:clasico}
  expresa el celebrado principio de inclusión y exclusión.
  Además de ser mucho más simple que la demostración tradicional,
  nuestro desarrollo
  no hace necesario recordar esta engorrosa fórmula,
  da las herramientas para deducirla sin mayor esfuerzo
  cada vez que la necesitemos,
  y en muchos casos obtener los resultados buscados directamente
  sin tener que recurrir a ella explícitamente,
  usando la función generatriz \(E(z)\).

  Esta técnica es sencilla de aplicar
  cuando se buscan los que no tienen ninguna de las propiedades,
  conviene tratar de ajustar
  la definición de las propiedades de forma adecuada.
  También es crítico que el cálculo de los \(N(\supseteq S)\)
  y,
  en consecuencia,
  de los \(N_r\),
  sea simple,
  cosa que nuevamente depende de la elección de las propiedades.

  Volvamos al ejemplo de tres conjuntos,
  donde nos interesa saber cuántos elementos
  pertenecen exactamente a uno de ellos,
  o sea \(e_1\),
  como en la figura~\ref{fig:PIE}.
  En tal caso:
  \begin{alignat*}{2}
    N_0
      &= \lvert \Omega \rvert
    &
    N_1
      &= \lvert \mathcal{A} \rvert
	   + \lvert \mathcal{B} \rvert
	   + \lvert \mathcal{C} \rvert \\
    N_2
      &= \lvert \mathcal{A} \cap \mathcal{B} \rvert
	   + \lvert \mathcal{A} \cap \mathcal{C} \rvert
	   + \lvert \mathcal{B} \cap \mathcal{C} \rvert
    \hspace{4em}&
    N_3
      &= \lvert \mathcal{A} \cap \mathcal{B} \cap \mathcal{C} \rvert
  \end{alignat*}
  Resultan ser:
  \begin{align*}
    N(z)
      &= N_0 + N_1 z + N_2 z^2 + N_3 z^3 \\
    E(z)
      &= (N_0 - N_1 + N_2 - N_3)
	   + (N_1 - 2 N_2 + 3 N_3) z
	   + (N_2 - 3 N_3) z^2
	   + N_3 z^3
  \end{align*}
  Hay \(e_1 = N_1 - 2 N_2 + 3 N_3\) elementos
  que pertenecen a exactamente un conjunto,
  como dedujimos antes.

  Típicamente interesa saber cuántos de los objetos
  no tienen ninguna de las propiedades,
  lo que en nuestro caso es
  \(\overline{(\mathcal{A} \cup \mathcal{B} \cup \mathcal{C})}\).
  O sea,
  \(e_0 = E(0) = N(-1) = N_0 - N_1 + N_2 - N_3\).

  La unión de todos los conjuntos,
  en este caso \(\mathcal{A} \cup \mathcal{B} \cup \mathcal{C}\),
  la componen los que pertenecen al menos a uno de los conjuntos,
  vale decir,
  todos menos los que no pertenecen a ninguno:
  \begin{equation*}
    \lvert \mathcal{A} \cup \mathcal{B} \cup \mathcal{C} \rvert
      = \sum_{t \ge 0} e_t - e_0
      = E(1) - E(0)
      = N(0) - N(-1)
      = N_1 - N_2 + N_3
  \end{equation*}
  Nuevamente coincide con lo que obtuvimos antes.

  Si solo interesa calcular
  el número promedio de propiedades por objeto,
  como \(t = \binom{t}{1}\) resulta:
  \begin{equation}
     \label{eq:E-t}
     \index{inclusion y exclusion, principio de@inclusión y exclusión, principio de!numero promedio de propiedades@número promedio de propiedades}
    \E[t]
      = \frac{\sum_{t \ge 0} t e_t}{\sum_{t \ge 0} e_t}
      = \frac{N_1}{N_0}
  \end{equation}
  Para calcular la varianza del número de propiedades,
  partimos de:
  \begin{equation*}
    \var[t]
      = \E[t^2] - \left( \E[t] \right)^2
  \end{equation*}
  Como \(\binom{t}{2} = (t^2 - t) / 2\):
  \begin{align*}
     \label{eq:Var-t}
     \index{inclusion y exclusion, principio de@inclusión y exclusión, principio de!varianza del numero de propiedades@varianza del número de propiedades}
    \frac{N_2}{N_0}
      &= \frac{\sum_{t \ge 0} \binom{t}{2} e_t}{\sum_{t \ge 0} e_t} \\
      &= \frac{\sum_{t \ge 0} (t^2 - t) e_t}{2 \sum_{t \ge 0} e_t} \\
      &= \frac{1}{2} \left( \E[t^2] - \E[t] \right)
  \end{align*}
  con lo cual:
  \begin{equation}
    \var[t]
      = \frac{2 N_2}{N_0} + \frac{N_1}{N_0} - \frac{N_1^2}{N_0^2}
  \end{equation}

\subsubsection*{Receta}
\index{inclusion y exclusion, principio de@inclusión y exclusión, principio de!receta}

  \begin{enumerate}
  \item
    Definir \(\Omega\) y \(\mathcal{P}\),
    expresar lo que se busca en términos de \(e_t\).
  \item
    Calcular los \(N(\supseteq \mathcal{S})\).
  \item
    Calcular los \(N_r\),
    y en consecuencia obtener \(N(z)\).
  \item
    \(e_t = \left[ z^t \right] N(z - 1)\).
  \end{enumerate}

  Hay que tener cuidado con esto,
  acá las series deben converger
  para que nuestras operaciones tengan sentido.
  Normalmente el número de propiedades y objetos de interés
  es finito,
  así que en realidad estamos manipulando polinomios
  y no hay problemas.

% Fixme: Agregar ejemplos "típicos" (¿tarea?)

  \begin{example}
    \index{inclusion y exclusion, principio de@inclusión y exclusión, principio de!ejemplo!cursos}
    Un curso rinde pruebas
    con los profesores Ellery, Upham y Atwood.
    Nos dicen que \(10\) aprobaron la prueba de física,
    \(15\) la de matemáticas y
    \(12\) pasaron la prueba de química;
    \(6\) pasaron física y matemáticas,
    \(5\) pasaron física y química,
    mientras \(8\) pasaron matemáticas y química.
    El total de estudiantes
    que pasaron al menos una prueba es \(20\).
    ¿Cuántos pasaron las tres pruebas?

    Aplicamos nuestra receta:
    \begin{enumerate}
    \item
      El universo es el grupo de \(20\) estudiantes
      que aprobaron alguna de las pruebas,
      las propiedades son las pruebas aprobadas (\(F, M, Q\)).
      Interesan los que aprobaron todas las pruebas,
      o sea \(e_3\).
    \item
      Los \(N(\supseteq \mathcal{S})\) están dados
      en el enunciado.
      Por ejemplo,
      dice que \(N(\supseteq \{F, Q\}) = 5\).
    \item
      Como se comentó antes,
      al haber \(3\) propiedades es \(N_3 = e_3\).
      Tenemos:
      \begin{align*}
	N_0
	  &= \lvert \Omega \rvert = 20 \\
	N_1
	  &= N(\supseteq \{F\})
	       + N(\supseteq \{M\})
	       + N(\supseteq \{Q\})
	   = 10 + 15 + 12
	   = 37 \\
	N_2
	  &= N(\supseteq \{F, M\})
	       + N(\supseteq \{F, Q\})
	       + N(\supseteq \{M, Q\})
	   = 6 + 5 + 8
	   = 19 \\
	N_3
	  &= N(\supseteq \{F, M, Q\})
	   = e_3
      \end{align*}
      Resulta:
      \begin{equation*}
	N(z) = 20 + 37 z + 19 z^2 + e_3 z^3
      \end{equation*}
    \item
      Como todos los estudiantes del universo
      han aprobado al menos una de las pruebas:
      \begin{equation*}
	e_0 = 0 = E(0) = N(-1) = 2 - e_3
      \end{equation*}
      Con esto resulta \(e_3 = 2\).

      Pero también tenemos:
      \begin{align*}
	E(z)
	  &= N(z - 1) \\
	  &= 5 z + 13 z^2 + 2 z^3
      \end{align*}
      lo que dice que \(5\) aprobaron una única prueba
      y que \(13\) aprobaron dos.
    \end{enumerate}
  \end{example}

  \begin{example}
    \index{inclusion y exclusion, principio de@inclusión y exclusión, principio de!ejemplo!numeros con ceros par@números con ceros par}
    ¿Cuántos números de largo \(n\) escritos en decimal
    tienen un número par de ceros?

    Para tener valores con los cuales contrastar,
    analicemos algunos casos.
    Anotamos \(9\) para un dígito no cero,
    y \(0\) para un cero en el cuadro~\ref{tab:par-0},
    y contamos cuántos de cada tipo hay.
    \begin{table}[htbp]
      \centering
      \begin{tabular}{|>{\(}r<{\)}|>{\(}l<{\)}|>{\(}r<{\)}|}
	\hline
	\multicolumn{1}{|c|}
		    {\rule[-0.7ex]{0pt}{3ex}\(\boldsymbol{n}\)} &
	  \multicolumn{1}{c|}{\textbf{Descripción}} &
	  \multicolumn{1}{c|}{\textbf{Nº}} \\
	\hline\rule[-0.7ex]{0pt}{3ex}%
	  1 & 9				&      9 \\
	  2 & 99			&     81 \\
	  3 & 999 + 900			&    738 \\
	  4 & 9999 + 9900 + 9090 + 9009 & 6\,804 \\
	\hline
      \end{tabular}
      \caption{Posibilidades con un número par de ceros}
      \label{tab:par-0}
    \end{table}

    El universo
    es el conjunto de todos los números con \(n\) dígitos.
    La propiedad \(i\)
    es que el dígito \(i\)\nobreakdash-ésimo es cero.
    Lo que interesa entonces es:
    \begin{equation*}
      e_0 + e_2 + \dotsb
	= \sum_{r \ge 0} e_{2 r}
    \end{equation*}

    Podemos extraer
    únicamente los términos con potencia par mediante:%
      \index{serie de potencias!decimar}
    \begin{equation*}
      \frac{E(z) + E(-z)}{2}
	= \sum_{r \ge 0} e_{2 r} z^{2 r}
    \end{equation*}
    y nuestra suma no es más que:
    \begin{equation*}
      \frac{E(1) + E(-1)}{2}
    \end{equation*}
    Esto es válido,
    ya que estamos trabajando con polinomios.

    Un número decimal de \(n\) dígitos
    comienza con un dígito no cero,
    los demás \(n - 1\) dígitos pueden ser cualquiera.
    En este caso podemos calcular los \(e_r\) directamente,
    observando que hay \(r\) posiciones para los ceros,
    elegidas de entre \(n - 1\) posiciones,
    los otros \(n - r\) dígitos
    (incluyendo el primero)
    pueden tomar uno de los \(9\) valores restantes:
    \begin{align*}
      e_r
	&= \binom{n - 1}{r} \cdot 9^{n - r} \\
      E(z)
	&= 9 \cdot \sum_{r \ge 0}
		     \binom{n - 1}{r} \cdot 9^{n - 1 - r} \cdot z^r
	 = 9 \cdot (9 + z)^{n - 1}
    \end{align*}
    y el número buscado resulta ser:
    \begin{equation*}
      \frac{1}{2} \, \left( E(1) + E(-1) \right)
	= \frac{1}{2} \, \left(
		     9 \cdot (9 + 1)^{n - 1}
		       + 9 \cdot (9 - 1)^{n - 1}
		  \right)
	= \frac{9}{2} \, \left(10^{n - 1} + 8^{n - 1}\right)
    \end{equation*}
    Esto coincide con los valores calculados antes,
    cuadro~\ref{tab:par-0}.
  \end{example}

  \begin{example}
    \index{inclusion y exclusion, principio de@inclusión y exclusión, principio de!ejemplo!lanzar \(10\) dados}
    Se lanzan \(10\)~dados.
    ¿De cuántas maneras puede hacerse esto
    tal que aparezcan todas las caras?

    Es más fácil calcular el número de lanzamientos
    en los que una cara dada \emph{no} aparece,
    lo que a su vez lleva naturalmente a contar el número de maneras
    en que no falta ninguna cara.
    Aplicando la receta:
    \begin{description}
    \item[\boldmath\(\Omega\)\unboldmath:]
      El conjunto de todos los lanzamientos posibles de \(10\)~dados
    \item[Propiedades:]
      Un lanzamiento tiene la propiedad \(k\) si la cara \(k\) no aparece
    \item[Resultado:]
      Interesan los lanzamientos sin propiedades
    \end{description}
    Si en un lanzamiento las caras en \(\mathcal{S}\) no aparecen,
    quiere decir que es una secuencia de largo~\(10\)
    de las restantes \(6 - \lvert \mathcal{S} \rvert\) caras:
    \begin{equation*}
      N(\supseteq \mathcal{S})
	= (6 - \lvert \mathcal{S} \rvert)^{10}
    \end{equation*}
    Como \(\mathcal{S}\) se elige entre \(6\) posibilidades:
    \begin{equation*}
      N_r
	= \binom{6}{r} (6 - r)^{10}
    \end{equation*}
    Tenemos:
    \begin{equation*}
      N(z)
	= \sum_{r \ge 0} \binom{6}{r} (6 - r)^{10} z^r
    \end{equation*}
    De la fórmula mágica:
    \begin{align*}
      e_0
	&= E(0) \\
	&= N(-1) \\
	&= \sum_{r \ge 0} (-1)^r \binom{6}{r} (6 - r)^{10} \\
	&= 16\,435\,440
    \end{align*}
  \end{example}

\section{Desarreglos}
\label{sec:desarreglos-pie}
\index{desarreglo}

  Un \emph{punto fijo} de una permutación \(\pi\)%
    \index{permutacion@permutación!punto fijo}
  ocurre cuando su elemento número \(k\) es \(k\)
  (vale decir,
   \(\pi(k) = k\)).
  Un \emph{desarreglo}
  (en inglés \emph{\foreignlanguage{english}{derangement}})%
    \index{derangement@\emph{\foreignlanguage{english}{derangement}}|see{desarreglo}}
  es una permutación sin puntos fijos.
  \glossary{Desarreglo}{Permutación sin puntos fijos}
  \glossary{Punto fijo (de una permutación)}
    {Elemento que no cambia de posición}
  El primero en calcular el número de desarreglos
  fue \foreignlanguage{french}{Pierre R. de Montmort}
  (1678--1719)~%
    \index{Montmort, Pierre R.}%
    \cite{montmort08:_jeux_hazard}.
  Hathout~%
    \index{Hathout, Heba}%
    \cite{hathout03:_old_hats_probl, hathout04:_old_hats_probl_revis}
  presenta varias soluciones
  (incluyendo la presente,
   debida esencialmente a Nicolaus Bernoulli%
     \index{Bernoulli, Nicolaus}).

  Siguiendo nuestra receta:
  \begin{enumerate}
  \item \(\Omega\):
    El universo son las \(n!\) permutaciones de \(n\) elementos.
    La permutación \(\pi\) tiene la propiedad \(i\)
    si \(i\) es un punto fijo en ella.
    Interesa obtener \(e_0\).
  \item
    Sea \(\mathcal{S} \subseteq \{1, \dotsc, n\}\).
    Entonces \(N(\supseteq \mathcal{S})\)
    corresponde a las permutaciones
    para las cuales los elementos de \(\mathcal{S}\) son fijos,
    solo se pueden ``mover''
    los \(n - \lvert \mathcal{S} \rvert\) restantes:
    \begin{equation*}
      N(\supseteq \mathcal{S})
	= \left(n - \lvert \mathcal{S} \rvert\right)!
    \end{equation*}
  \item
    Como \(r\) puntos fijos
    pueden elegirse de \(\binom{n}{r}\) maneras,
    se tiene que:
    \begin{equation}
      \label{eq:fixed-points-Nr}
      N_r
	= \sum_{\lvert \mathcal{S} \rvert = r}
	    N(\supseteq \mathcal{S})
	= \sum_{\lvert \mathcal{S} \rvert = r} (n - r)!
	= \binom{n}{r} \cdot (n - r)!
    \end{equation}
    Con esto:
    \begin{equation}
      \label{eq:fixed-points-FG}
      N(z)
	= \sum_{0 \le r \le n} \binom{n}{r} (n - r)! \, z^r
	= \sum_{0 \le r \le n}
	     \frac{n!}{r!(n - r)!} \cdot (n - r)! \cdot z^r
	= n! \, \sum_{0 \le r \le n} \frac{z^r}{r!}
    \end{equation}
    En términos de la función exponencial truncada:%
      \index{serie de potencias!exponencial}
    \begin{equation}
      \label{eq:exp-truncada}
      \exp \rvert_n (z)
	= \sum_{0 \le k \le n} \frac{z^k}{k!}
    \end{equation}
    la ecuación~\eqref{eq:fixed-points-FG} es:
    \begin{equation*}
      N(z) = n! \cdot \exp \rvert_n (z)
    \end{equation*}
  \item
    En particular,
    \(e_0 = E(0) = N(-1)\)
    es el número de desarreglos de \(n\) elementos:
    \begin{align}
      D_n
	&=	   n! \exp \rvert_n (-1)
	    \label{eq:n-derangements} \\
	&\approx n! \, \mathrm{e}^{-1}
	    \label{eq:n-derangements-approx}
      \index{desarreglo!numero de, formula@número de, fórmula}
    \end{align}

    Consideremos la serie de Maclaurin%
      \index{Maclaurin, teorema de}
    para \(\mathrm{e}^x\)
    con el resto en la forma de Lagrange:%
      \index{Maclaurin, teorema de!Lagrange, forma del resto}
    \begin{equation*}
      \mathrm{e}^{-1}
	= \sum_{0 \le k \le n} \frac{(-1)^k}{k!}
	    + \frac{(-1)^{n + 1}}{(n + 1)!}
	    + \frac{\mathrm{e}^{-\xi}}{(n + 2)!} \, (-1)^{n + 2}
      \qquad (0 < \xi < 1)
    \end{equation*}
    Tenemos las cotas para el valor absoluto
    del error que comete
    la fórmula \(D_n \approx n! \mathrm{e}^{-1}\):
    \begin{align*}
      n ! \left(
	    \frac{1}{(n + 1)!} - \frac{\mathrm{e}^0}{(n + 2)!}
	  \right)
	&< \epsilon_n
	 < n! \left(
		\frac{1}{(n + 1)!}
		  - \frac{\mathrm{e}^{-1}}{(n + 2)!}
	      \right) \\
      \frac{1}{n + 2}
	&< \epsilon_n
	 < \frac{n + 2 - \mathrm{e}^{-1}}{(n + 1) (n + 2)}
	 < \frac{1}{n + 1}
    \end{align*}
    Para \(n \ge 1\) el error absoluto es menor a \(1 / 2\),
    y \(D_n\) es el entero más cercano a \(n! e^{-1}\).
    Algo como un \(37\)\%
    de las permutaciones no tienen puntos fijos.
    Es curioso que este resultado dependa tan poco de \(n\).

    Más en general,
    tenemos también:
    \begin{align}
      e_t
	&= n! \, \left[ z^t \right] \, \sum_{0 \le r \le n}
					 \frac{(z - 1)^r}{r!}
	 = n! \, \left[ z^t \right] \,
		   \sum_{0 \le r \le n}
		     \sum_{0 \le k \le r}
		       \frac{1}{r!} \,
			 \binom{r}{k} \, z^k \,
			 (-1)^{r - k} \notag \\
	&= n! \, \sum_{0 \le r \le n}
		   \frac{1}{r!} \binom{r}{t} \, (-1)^{r - t}
	 = n! \, \sum_{t \le r \le n}
		   \frac{(-1)^{r - t}}{t! (r - t)!}
	 = \frac{n!}{t!} \,
	      \sum_{0 \le r \le n - t} \frac{(-1)^r}{r!} \notag \\
	&= \frac{n!}{t!} \, \exp \rvert_{n - t} (-1)
	\label{eq:t-puntos-fijos}
    \end{align}
    Por~\ref{eq:E-t} y~\ref{eq:fixed-points-Nr}
    el número promedio de puntos fijos es:
    \begin{equation*}
      \bar{t}
	= \frac{N_1}{N_0}
	= \frac{\binom{n}{1} (n - 1)!}{n!}
	= 1
    \end{equation*}
    Curiosamente no depende de \(n\).
  \end{enumerate}

\section{El problema de ménages}
\label{sec:menages}
\index{problème des ménages@\emph{\foreignlanguage{french}{problème des ménages}}}

  Lucas~\cite{lucas91:_theo_nombres} en~1891
  planteó el problema de sentar \(n\) parejas en una mesa circular,
  alternando hombres y mujeres
  de forma que ninguna pareja se sentara junta.
  Este ``problème des ménages''
  fue resuelto recién en~1934 por Touchard~%
    \cite{touchard34:_permutations},
  pero sin dar una demostración.
  La primera demostración de la fórmula de Touchard
  fue dada por Kaplansky~\cite{kaplansky43:_menages} en~1943,
  claro que al insistir en ubicar primero a las damas
  y luego ordenar a los caballeros
  lleva a un desarrollo innecesariamente complicado.
  Seguimos a Bogart y~Doyle~%
    \cite{bogart86:_non_sexist_soln_menage_probl}
  en nuestra derivación.

  Observamos primero que \(n\) parejas
  pueden ubicarse de \(2 (n!)^2\)~maneras alrededor de la mesa
  (hay dos maneras de elegir los asientos de las mujeres,
   luego ordenamos a damas y caballeros).
  Aplicamos el principio de inclusión y exclusión,
  numerando las parejas de \(1\) a~\(n\),
  y definiendo que una distribución tiene la propiedad \(i\)
  si la pareja \(i\) se sienta junta.
  Si \(W_k\) es el número de maneras de sentar las parejas
  de manera que \(k\) parejas dadas se sientan juntas,
  tenemos:
  \begin{align}
    \label{eq:menage:N}
    N_r
      = \binom{n}{r} W_r
  \end{align}
  de la fórmula mágica resulta:
  \begin{equation}
    \label{eq:menages:M}
    M_n
      = \sum_{0 \le r \le n} (-1)^r \binom{n}{r} W_r
  \end{equation}
  donde:
  \begin{equation}
    \label{eq:menages:W}
    W_k
      = 2 \cdot d_k \cdot k! \cdot \left( (n - k)! \right)^2
  \end{equation}
  (debemos elegir asientos de las mujeres y hombres,
   dónde se ubican las parejas que se sientan juntas,
   y finalmente cómo se distribuyen las \(n - k\)~damas
   y los \(n - k\)~caballeros que no forman parte de las parejas).
  Acá \(d_k\)
  es el número de formas de ubicar \(k\)~piezas de dominó
  sobre un ciclo de~\(2 n\)
  sin que se traslapen,
  \begin{figure}[ht]
    \centering
    \pgfimage{images/menages}
    \caption{Dominós no traslapados en ciclo}
    \label{fig:menage}
  \end{figure}
  ver la figura~\ref{fig:menage} para un ejemplo.

  Si cortamos el círculo,
  hay dos posibilidades:
  Una de las piezas de dominó se corta en dos,
  queda ubicar \(k - 1\) de ellas en línea
  sobre \(2 n - 2\) asientos;
  o ninguna de las piezas se corta,
  debemos ubicar \(k\) piezas sobre una línea de \(2 n\) asientos.
  Si llamamos \(f(n, k)\) al número de formas
  de ubicar \(k\) piezas en una línea de \(n\) asientos,
  esto resulta de \(k + 1\) bloques entre piezas de dominó,
  entre los cuales se distribuyen
  las \(n - 2 k\) posiciones no cubiertas:
  \begin{align}
    f(n, k)
      &= \multiset{k + 1}{n - 2 k} \notag \\
      &= \binom{n - k}{k}
	    \label{eq:menage:f}
  \end{align}
  Con~\eqref{eq:menage:f} resulta:
  \begin{align}
    d_k
      &= \binom{2 n - k}{k} + \binom{2 n - 2 - (k - 1)}{k - 1}
	      \notag \\
      &= \frac{(2 n - k)!}{k! (2 n - 2 k)!}
	   + \frac{(2 n - k - 1)!}{(k - 1)! (2 n - 2 k)!}
	      \notag \\
      &= \frac{(2 n - k - 1)!}{(k - 1)! (2 n - 2 k)!}
	   \left(
	     \frac{2 n - k}{k}
	       + 1
	   \right)
	      \notag \\
      &= \frac{2 n}{2 n - k} \binom{2 n - k}{k}
	      \label{eq:menage:d}
  \end{align}
  Uniendo las relaciones~\eqref{eq:menages:M} a~\eqref{eq:menage:f},
  como por simetría \(M_n\) debe ser divisible por \(2 n!\),
  al simplificar resulta la fórmula de Touchard:%
    \index{Touchard, formula de@Touchard, fórmula de}
  \begin{align}
    M_n
      &= \sum_{0 \le k \le n}
	   (-1)^k \binom{n}{k}
	      \cdot 2
	      \cdot \frac{2 n}{2 n - k} \binom{2 n - k}{k}
	      \cdot k! \cdot \left( (n - k)! \right)^2 \notag \\
      &= 2 n! \sum_{0 \le k \le n}
		(-1)^k \cdot \frac{2 n}{2 n - k}
		       \cdot \binom{2 n - k}{k}
		       \cdot (n - k)!
	  \label{eq:menages}
  \end{align}

  El principio de inclusión y exclusión
  completa las herramientas elementales
  de la combinatoria vistas
  en el capítulo~\ref{cha:combinatoria-elemental}.

%%% Local Variables:
%%% mode: latex
%%% TeX-master: "clases"
%%% End:


 probabilidades.tex
%
% Copyright (c) 2013-2015 Horst H. von Brand
% Derechos reservados. Vea COPYRIGHT para detalles

\chapter{Rudimentos de probabilidades discretas}
\label{cha:probabilidad-discreta}
\index{probabilidad}

  Muchas aplicaciones involucran describir lo que ocurre
  cuando interviene el azar.
  Por ejemplo,
  cuál es la probabilidad
  con la que al lanzar dos dados la suma sea seis.
  Al analizar algoritmos
  suele ser de interés su rendimiento promedio,
  y poder cuantificar cuánto se espera pueda desviarse de él.

  Las situaciones que se presentan en casos de nuestro interés
  se pueden describir en forma discreta.
  Nos restringiremos a esta situación.
  Para profundizar en el tema
  (incluyendo probabilidades continuas)
  se recomienda el texto de Grinstead y Snell~%
    \cite{grinstead97:_introd_probab},
  una visión más completa da Feller en sus textos clásicos~%
    \cite{feller68:_intro_probab_theo_applic_1,
	  feller71:_intro_probab_theo_applic_2}.

\section{Probabilidades}
\label{sec:probabilidades}

  Primero consideraremos experimentos al azar
  en los cuales hay un número finito de resultados
  \(\omega_1, \omega_2, \dotsc, \omega_n\).
  El ejemplo tradicional es el lanzar un dado,
  con posibles resultados
    \(1\), \(2\), \(3\), \(4\), \(5\) y \(6\).
  Otro ejemplo es lanzar una moneda,
  con resultados cara o sello
  (generalmente anotados \(\mathrm{H}\)
   por \emph{\foreignlanguage{english}{head}}
   y \(\mathrm{T}\)
   por \emph{\foreignlanguage{english}{tail}} en inglés).

  Comúnmente nos referiremos a resultados de experimentos,
  como lanzar un dado cuatro veces
  y preguntarnos por la suma de los cuatro valores.
  En tal circunstancia podemos denotar el valor de cada lanzamiento
  por \(X_i\),
  con \(i = 1, 2, 3, 4\).
  La suma de interés entonces es:
  \begin{equation}
    \label{eq:suma-4-dados}
    X_1 + X_2 + X_3 + X_4
  \end{equation}
  Los \(X_i\) son \emph{variables aleatorias},%
    \index{variable aleatoria|textbfhy}
  simplemente expresiones
  cuyo valor es el resultado de un experimento.
  La suma~\eqref{eq:suma-4-dados} también es una variable aleatoria.

  Sea \(X\) la variable aleatoria
  que representa el resultado de un experimento.
  Asignaremos probabilidades
  a los posibles resultados de ese experimento.
  Esto lo hacemos a través de asignar un número no negativo
  \(f_X(\omega_j)\) a cada posible resultado \(\omega_j\)
  de manera que:
  \begin{equation}
    \label{eq:sum_probabilities=1}
    f_X(\omega_1) + f_X(\omega_2) + \dotsb + f_X(\omega_n)
      = 1
  \end{equation}
  A la función \(f_X\)
  se le llama la \emph{función de distribución}%
    \index{variable aleatoria!funcion de distribucion@función de distribución|textbfhy}
  de la variable aleatoria \(X\).
  Para el caso del lanzamiento de un dado
  asignaríamos iguales probabilidades de \(1 / 6\)
  a cada uno de los posibles resultados.
  Así podemos escribir para las probabilidades
  de que se cumplan las situaciones dadas:
  \begin{align*}
    \Pr(X = 1)
      &= \frac{1}{6} \\
    \Pr(X \le 4)
      &= \frac{2}{3} \\
    \Pr(X \in \{1, 3, 4\})
      &= \frac{1}{2}
  \end{align*}
  De la misma forma,
  al lanzar una moneda
  es natural asignar las probabilidades \(1 / 2\) a cara y a sello.

  En los ejemplos precedentes las probabilidades asignadas
  a los distintos resultados son iguales,
  pero esto no siempre será así.
  Si se ha determinado que cierto tratamiento
  tiene un \(70\)\% de éxitos,
  asignaríamos la probabilidad \(0,70\)
  a que el siguiente tratamiento resulte exitoso.
  Esto,
  con los casos anteriores,
  ilustra el concepto intuitivo
  de \emph{probabilidades como frecuencias}.%
    \index{probabilidad!como frecuencia}
  Vale decir,
  si hay una probabilidad \(p\)
  que el resultado de un experimento sea \(A\),
  si repetimos el experimento gran número de veces
  esperamos que una fracción \(p\) de los resultados sea \(A\).

  El lector alerta protestará que todo esto es circular:
  Estamos \emph{aseverando} que las probabilidades
  corresponden a frecuencias relativas ``en muchos experimentos''
  (pero también es posible que al lanzar una moneda mil veces
   resulte cara mil veces),
  para luego usar la idea
  que ciertos eventos son ``igualmente probables''
  y extraer probabilidades de ello.
  Para una base realmente rigurosa véase por ejemplo Ash~%
    \cite{ash08:_basic_probab_theo}.
  La teoría allí expuesta justifica nuestro tratamiento intuitivo,
  que sigue a Grinstead y Snell~%
    \cite{grinstead97:_introd_probab}.
  Para nuestros efectos generalmente bastará
  suponer que ciertos resultados son igualmente probables.

\section{Distribuciones discretas}
\label{sec:distribuciones-discretas}
\index{probabilidad!distribucion discreta@distribución discreta}

  Analizaremos múltiples experimentos
  desde el punto de vista probabilístico
  en lo que sigue.
  La idea global de lo que haremos se puede describir como sigue:
  Cada experimento tiene asociada una variable aleatoria,
  que representa los posibles resultados del experimento.
  El conjunto de posibles resultados es el \emph{espacio muestral}.%
    \index{espacio muestral|see{probabilidad!distribución discreta}}
  Primero consideraremos el caso de espacios muestrales finitos,
  para luego extender la discusión
  a espacios muestrales infinitos numerables.

  \begin{definition}
    Suponga un experimento cuyo resultado depende del azar.
    El resultado del experimento se representa por
    una \emph{variable aleatoria}.%
      \index{variable aleatoria|textbfhy}
    El \emph{espacio muestral} del experimento%
      \index{espacio muestral|see{probabilidad!distribución discreta}}
    es el conjunto de todos los posibles resultados.
    Si el espacio muestral es numerable,
    se dice que la variable aleatoria es \emph{discreta}.
  \end{definition}
  Completamos lo anterior con dos términos adicionales.
  \begin{definition}
    Los elementos del espacio muestral se llaman \emph{resultados}.
    Un \emph{evento} es un subconjunto del espacio muestral.%
      \index{probabilidad!evento}
  \end{definition}
  Usamos letras mayúsculas
  para representar el resultado del experimento,
  y generalmente anotaremos \(\Omega\) para el espacio muestral.
  Usaremos letras minúsculas para resultados
  y eventos por mayúsculas.
  Al lanzar un dado,
  si llamamos \(X\) al resultado del experimento
  (el número que muestra el dado),
  el espacio muestral es:
  \begin{equation*}
    \Omega
      = \{ 1, 2, 3, 4, 5, 6 \}
  \end{equation*}
  Un evento es que el resultado sea par,
  o sea \(E = \{ 2, 4, 6 \}\).
  Bajo la suposición que el dado no está cargado,
  es natural considerar
  que cada posible resultado tiene la misma probabilidad,
  \(f_X(i) = 1 / 6\) para \(1 \le i \le 6\).
  \begin{definition}
    Considere un experimento
    cuyo resultado es la variable aleatoria \(X\),
    con espacio muestral \(\Omega\).
    Una \emph{función de distribución} para \(X\)%
      \index{probabilidad!funcion de distribucion@función de distribución}
    es una función \(f_X \colon \Omega \rightarrow \mathbb{R}\)
    tal que:
    \begin{align}
      f_X(\omega)
	&\ge 0
	    \label{eq:f(omega)>=0} \\
      \sum_{\omega \in \Omega} f_X(\omega)
	&= 1
	    \label{eq:sum_f(omega)=1}
    \end{align}
    Para cualquier subconjunto \(E \subseteq \Omega\)
    definimos la \emph{probabilidad del evento \(E\)}%
      \index{probabilidad!evento|textbfhy}
    como:
    \begin{equation*}
      \Pr(E)
	= \sum_{\omega \in E} f_X(\omega)
    \end{equation*}
  \end{definition}
  Esto ya debiera alertar al lector que manipulaciones de conjuntos
  serán centrales en la discusión.
  La notación así introducida es consistente con la idea informal
  planteada antes.

  Una consecuencia inmediata de la definición
  es que para todo \(\omega \in \Omega\):
  \begin{equation*}
    \Pr(\{\omega\})
      = f_X(\omega)
  \end{equation*}

  Consideremos el experimento de lanzar una moneda dos veces.
  Podemos registrar el resultado de diversas maneras,
  dando el orden en que se dieron cara y cruz
  (\(\Omega
      = \{ \mathrm{HH}, \mathrm{HT}, \mathrm{TH}, \mathrm{TT} \}\)),
  el número de veces que salió cara
  (\(\Omega = \{0, 1, 2\}\))
  o como los pares sin importar el orden
  (\(\Omega = \{ \mathrm{HH}, \mathrm{HT}, \mathrm{TT} \}\)).
  Sea \(X\)
  la variable aleatoria que corresponde a este experimento,
  con el primer espacio muestral descrito.
  Asumiremos que cada uno de los resultados es igualmente probable,
  o sea la función de distribución \(f_X\) dada por:
  \begin{equation*}
    f_X(\mathrm{HH})
      = f_X(\mathrm{HT})
      = f_X(\mathrm{TH})
      = f_X(\mathrm{TT})
      = \frac{1}{4}
  \end{equation*}
  Para el evento \(E = \{\mathrm{HT}, \mathrm{TH}\}\)
  (una cara, una cruz)
  tenemos:
  \begin{equation*}
    \Pr(E)
      = \frac{1}{4} + \frac{1}{4}
      = \frac{1}{2}
  \end{equation*}

  Tenemos algunas propiedades simples.
  \begin{theorem}
    \index{probabilidad!evento!propiedades}
    \label{theo:properties-probabilities-events}
    Sea \(f\) una función de distribución
    sobre el espacio muestral \(\Omega\).
    Las probabilidades que \(f\)
    asigna a eventos \(E \subseteq \Omega\)
    cumplen:
    \begin{enumerate}
    \item
      \label{item:nonnegative}
      \(\Pr(E) \ge 0\) para todo \(E \subseteq \Omega\)
    \item
      \label{item:universe=1}
      \(\Pr(\Omega) = 1\)
    \item
      \label{item:subset}
      Si \(E \subseteq F \subseteq \Omega\)
      entonces \(\Pr(E) \le \Pr(F)\).
    \item
      \label{item:disjoint-union}
      Si \(A\) y \(B\) son subconjuntos disjuntos de \(\Omega\),
      entonces \(\Pr(A \cup B) = \Pr(A) + \Pr(B)\).
      En este caso se dice
      que \emph{\(A\) y \(B\) son mutuamente excluyentes}.%
	\index{probabilidad!evento!mutuamente excluyente}
    \item
      \label{item:complement}
      \(\Pr(\overline{A}) = 1 - \Pr(A)\)
    \end{enumerate}
  \end{theorem}
  \begin{proof}
    Cada propiedad por turno.
    \begin{enumerate}
    \item
      Por definición:
      \begin{equation*}
	\Pr(E)
	  = \sum_{\omega \in E} f(\omega)
      \end{equation*}
      Como a su vez \(f(\omega) \ge 0\),
      concluimos que \(\Pr(E) \ge 0\).
    \item
      Esto no es más que:
      \begin{equation*}
	\Pr(\Omega)
	  = \sum_{\omega \in \Omega} f(\omega)
	  = 1
      \end{equation*}
    \item
      Supongamos \(E \subseteq F\).
      Tenemos:
      \begin{equation*}
	\Pr(E)
	  =   \sum_{\omega \in E} f(\omega)
	  \le \sum_{\omega \in F} f(\omega)
	  =   \Pr(F)
      \end{equation*}
      ya que cada término de la primera suma está en la segunda,
      y los términos de la segunda suma que no estén en la primera
      no son negativos.
    \item
      Si \(A \cap B = \varnothing\),
      entonces:
      \begin{equation*}
	\Pr(A \cup B)
	  = \sum_{\omega \in A \cup B} f(\omega)
	  = \sum_{\omega \in A} f(\omega)
	     + \sum_{\omega \in B} f(\omega)
	  = \Pr(A)+ \Pr(B)
      \end{equation*}
    \item
      Aplicando las propiedades~\ref{item:disjoint-union}
      y~\ref{item:universe=1} a \(A \cup \overline{A} = \Omega\)
      resulta:
      \begin{equation*}
	\Pr(A) + \Pr(\overline{A})
	 = 1
      \end{equation*}
      que es equivalente a lo indicado.
    \qedhere
    \end{enumerate}
  \end{proof}
  Es común que sea más fácil calcular
  la probabilidad de que un evento no ocurra,
  en tal caso es útil la propiedad~\ref{item:complement}.

  La propiedad~\ref{item:disjoint-union}
  del teorema~\ref{theo:properties-probabilities-events}
  puede extenderse a uniones disjuntas finitas:
  \begin{theorem}
    \label{theo:disjoint-events}
    Sean \(A_1\), \(A_2\), \ldots, \(A_n\)
    subconjuntos de \(\Omega\),
    disjuntos a pares.
    Entonces:
    \begin{equation*}
      \Pr(A_1 \cup A_2 \cup \dotsb \cup A_n)
	= \sum_{1 \le k \le n} \Pr(A_k)
    \end{equation*}
  \end{theorem}
  Usaremos la siguiente consecuencia con frecuencia:%
    \index{probabilidad!evento!mutuamente excluyente}
  \begin{corollary}
    \label{cor:disjoint-events-intersection}
    Sean \(A_1\), \(A_2\), \ldots, \(A_n\)
    mutuamente excluyentes
    (vale decir,
     disjuntos a pares),
    tales que \(A_1 \cup A_2 \cup \dotsb \cup A_n = \Omega\),
    y \(E\) un evento cualquiera.
    Entonces:
    \begin{equation*}
      \Pr(E)
	= \sum_{1 \le k \le n} \Pr(E \cap A_k)
    \end{equation*}
  \end{corollary}
  \begin{proof}
    Los conjuntos \(E \cap A_k\) son disjuntos a pares,
    y su unión es \(E\).
  \end{proof}
  Una consecuencia útil es:
  \begin{corollary}
    \label{cor:event-intersection-and-complement}
    Para cualquier par de eventos \(A\) y \(B\):
    \begin{equation*}
      \Pr(A)
	= \Pr(A \cap B) + \Pr(A \cap \overline{B})
      \qedhere
    \end{equation*}
  \end{corollary}
  Usando las anteriores con~\eqref{eq:suma-union-interseccion}
  obtenemos:
  \begin{theorem}
    \label{theo:probability-union}
    Si \(A\) y \(B\) son subconjuntos de \(\Omega\):
    \begin{equation*}
      \Pr(A \cup B)
	= \Pr(A) + \Pr(B) - \Pr(A \cap B)
    \end{equation*}
  \end{theorem}
  Extender este resultado a más conjuntos
  es precisamente el principio de inclusión y exclusión,
  tema del capítulo~\ref{cha:pie}.

\subsection{Función generatriz de probabilidad}
\label{sec:PGF}
\index{probabilidad!funcion generatriz de@función generatriz de|textbfhy}

  Formalmente para la variable aleatoria \(X\)
  se define la \emph{función generatriz de probabilidades}
  (abreviada \emph{pgf},
  del inglés
  \emph{\foreignlanguage{english}
       {probability generating function}}) como:
  \begin{equation}
    \label{eq:pgf-def}
    G(z)
      = \E[z^X]
  \end{equation}
  En nuestro caso el espacio muestral es \(\Omega = \mathbb{N}_0\),
  con la función de distribución
  \(f \colon \mathbb{N}_0 \rightarrow \mathbb{R}\).
  La definición~\eqref{eq:pgf-def}
  se traduce en:
  \begin{equation}
    \label{eq:pgf}
    G(z)
      = \sum_{n \ge 0} f(n) z^n
  \end{equation}
  Esto no es más que la función generatriz ordinaria%
    \index{generatriz!ordinaria}
  de la secuencia \(\langle f(k) \rangle_{k \ge 0}\).
  La condición~\eqref{eq:sum_f(omega)=1}
  se traduce en:
  \begin{equation}
    \label{eq:pgf(1)=1}
    G(1)
      = 1
  \end{equation}
  Un dato interesante
  es que si tenemos variables independientes \(X\) e \(Y\),
  con distribuciones \(f_X\) y \(f_Y\),
  respectivamente;
  y cuyas funciones generatrices de probabilidad son
  respectivamente \(G_X\) y \(G_Y\),
  tenemos la función generatriz de probabilidad
  \(G_{X + Y}\) para la suma \(X + Y\):
  \begin{align}
    G_{X + Y}(z)
      &= \sum_{x, y} f_X(x) f_Y(y) z^{x + y}
	     \notag \\
      &= \left(\sum_x f_X(x) z^x\right)
	   \cdot \left(\sum_y f_Y(y) z^y\right)
	     \notag \\
      &= G_X(z) \cdot G_Y(z)
	     \label{eq:PGF-sum}
  \end{align}

  Planteado el modelo de un experimento al azar,
  queda el problema de determinar las probabilidades
  que mejor describen el experimento.
  Afortunadamente,
  en muchas situaciones simples
  cada evento elemental es igualmente probable.
  \begin{theorem}
    \label{theo:equilikely-principle}
    Si \(\Omega\) es finito,
    y cada resultado es igualmente probable,
    la probabilidad del evento \(E\)
    es:
    \begin{equation*}
      \Pr(E)
	= \frac{\lvert E \rvert}{\lvert \Omega \rvert}
    \end{equation*}
  \end{theorem}
  En tales situaciones el cálculo de probabilidades
  se reduce a contar los elementos de los eventos,
  lo que lleva a combinatoria
  como desarrollada en el capítulo~\ref{cha:combinatoria-elemental},
  involucra aplicaciones del principio de inclusión y exclusión
  discutido en el capítulo~\ref{cha:pie}
  o técnicas más avanzadas como las expuestas en el capítulo~%
    \ref{cha:metodo-simbolico} y siguientes.

\subsection{Función generatriz de momentos}
\label{sec:generatriz-momentos}

  Formalmente,
  se define la \emph{función generatriz de momentos}%
    \index{momentos!funcion generatriz de!función generatriz de}
  (abreviada \emph{mgf},
   del inglés \emph{\foreignlanguage{english}{moment generating function}})
  para la variable aleatoria \(X\)
  mediante:
  \begin{equation}
    \label{eq:mgf-def}
    M(z)
      = \E[\mathrm{e}^{z X}]
  \end{equation}
  Recordando~\eqref{eq:pgf-def} vemos que:
  \begin{equation}
    \label{eq:mgf-pgf}
    M(z)
      = G(\mathrm{e}^z)
  \end{equation}

  Podemos evaluar:
  \begin{align}
    M(z)
      &= \E[e^{z X}] \notag \\
      &= \E\left[ \sum_{r \ge 0} \frac{z^r X^r}{r!} \right] \notag \\
      &= \sum_{r \ge 0} \E[X^r] \frac{z^r}{r!}
	   \label{eq:mgf-moments}
  \end{align}
  Esta es la función generatriz exponencial
  de los momentos \(\E[X^n]\) de la variable.

\subsection{Problemas de urna}
\label{sec:urn-problems}
\index{probabilidad!modelo de urna}

  Muchas situaciones simples de probabilidades se pueden describir
  en el marco de una urna que contiene bolas,
  las cuales se extraen y se notan.
  Consideremos una urna que contiene \(n\) bolas,
  numeradas \(1\) a \(n\),
  de las que se extraen \(k\).
  Esto puede hacerse de diversas maneras.
  Primeramente,
  podemos extraer las bolas una a una
  (el orden en que se extraen importa)
  o sacarlas de una vez
  (el orden no importa).
  En el último caso
  resulta conveniente
  considerar igual que las bolas se sacan de a una,
  pero el orden no interesa.
  Enseguida,
  una vez que se extrae una bola podemos notarla
  y devolverla a la urna,
  o dejarla fuera
  (con y sin reemplazo).
  Parte del arte es reconocer estas cuatro situaciones
  aún si parecieran no ser aplicables.
  Para distinguir los casos
  anotamos elementos ordenados como tupla, \((1, 3, 2)\);
  y elementos no ordenados como (multi)conjunto, \(\{ 1, 2, 3 \}\).

  Podemos aplicar
  las técnicas del capítulo~\ref{cha:combinatoria-elemental}%
    \index{combinatoria}
  para contar las posibilidades en los cuatro casos resultantes:
  \begin{description}
  \item[Ordenadas, sin reemplazo:]
    \index{combinatoria!permutacion@permutación}
    Vemos que la primera bola se puede elegir de \(n\) maneras,
    la segunda entre las \(n - 1\) restantes,
    y así sucesivamente.
    Son permutaciones de \(k\) elementos tomados entre \(n\):
    \begin{equation*}
      G(n, k)
	= n^{\underline{k}}
    \end{equation*}
  \item[Ordenadas, con reemplazo:]
    \index{combinatoria!secuencia}
    En este caso
    cada una de las \(k\) bolas se elige entre las \(n\):
    \begin{equation*}
      n^k
    \end{equation*}
  \item[Sin orden, sin reemplazo:]
    \index{combinatoria!combinacion@combinación}
    Es elegir un subconjunto de \(k\) los \(n\) elementos,
    lo que llamamos combinaciones:
    \begin{equation*}
      C(n, k)
	= \binom{n}{k}
    \end{equation*}
  \item[Sin orden, con reemplazo:]
    \index{combinatoria!multiconjunto}
    En este caso lo relevante
    es cuántas veces aparece cada uno de los \(n\) elementos,
    es un multiconjunto de \(k\) elementos:
    \begin{equation*}
      \multiset{n}{k}
	= \binom{n + k - 1}{k}
    \end{equation*}
  \end{description}
  Ejemplos típicos son:
  \begin{enumerate}
  \item
    En una prueba de selección múltiple hay 20~preguntas,
    cada una con 5~alternativas.
    Considerando que se puede responder una de las opciones
    o dejar la pregunta en blanco,
    esto es elegir en orden \(k = 20\) bolas entre \(n = 6\)
    con reemplazo,
    el total de posibilidades es:
    \begin{equation*}
      6^{20}
	= 3\,656\,158\,440\,062\,976
    \end{equation*}
  \item
    En el campeonato mundial de fútbol juegan 32~equipos.
    Ordenar los ganadores
    (primer a tercer lugar)
    corresponde a elegir \(k = 3\) entre \(n = 32\)
    en orden sin reemplazo:
    \begin{equation*}
      32^{\underline{3}}
	= 32 \cdot 31 \cdot 30
	= 29\,760
    \end{equation*}
  \item
    De un mazo de 52~cartas se saca una mano de poker.
    Esto es tomar \(k = 5\) elementos entre \(n = 52\)
    sin orden y sin reemplazo,
    un subconjunto:
    \begin{equation*}
      \binom{52}{5}
	= 2\,598\,960
    \end{equation*}
  \item
    Si se lanzan tres dados,
    los resultados posibles corresponden a elegir \(k = 3\) valores
    de \(n = 6\) sin orden con reemplazo,
    un multiconjunto:
    \begin{equation*}
      \multiset{6}{3}
	= \binom{8}{3}
	= 56
    \end{equation*}
  \end{enumerate}
  El modelo de urna supone que todas las posibilidades así obtenidas
  son igualmente probables.
  Al menos en el caso del campeonato de fútbol deberemos acordar
  que esto no es realista
  (pocos apostarían que Chile resulte campeón el~2014).

\subsection{Distribuciones multivariadas}
\label{sec:multivariable-distributions}
\index{probabilidad!distribucion multivariada@distribución multivariada}

  Podemos manejar de forma similar
  distribuciones conjuntas de varias variables.
  Para concretar la discusión,
  veremos el caso de la variable aleatoria \((X, Y)\)%
    \index{variable aleatoria}
  compuesta por variables aleatorias \(X\) e \(Y\).
  La extensión a más variables es simple,
  y no nos detendremos en ello.

  Siendo consistentes con nuestra notación y definiciones previas,
  definimos la función de distribución \(f_{(X, Y)}\),
  que cumple:
  \begin{align*}
    f_{(X, Y)}(x, y)
      &\ge 0 \\
    \sum_{x, y} f_{(X, Y)}(x, y)
      &=   1
  \end{align*}
  Como:
  \begin{equation*}
    \Pr(X = x \wedge Y = y)
      = f_{(X, Y)}(x, y)
  \end{equation*}
  la probabilidad del evento \(X = x\) es simplemente:
  \begin{equation*}
    \Pr(X = x)
      = \sum_y f_{(X, Y)}(x, y)
  \end{equation*}
  Esto define la función de distribución de \(X\)
  (y similarmente la de \(Y\))
  en esta situación:
  \begin{align}
    f_X(x)
      &= \sum_y f_{(X, Y)}(x, y) \label{eq:f_X-from-f_XY} \\
    f_Y(y)
      &= \sum_x f_{(X, Y)}(x, y) \label{eq:f_Y-from-f_XY}
  \end{align}

\subsection{Diagramas de árbol}
\label{sec:diagramas-arbol}
\index{probabilidad!diagrama de arbol@diagrama de árbol}

  Muchos experimentos
  pueden considerarse que se llevan a cabo en etapas.
  Por ejemplo,
  el lanzar tres monedas podemos describirlo
  como lanzando una moneda después de la otra,
  dando lugar a un diagrama
  como el de la figura~\ref{fig:event-tree}
  (su \emph{diagrama de árbol}).
  \begin{figure}[ht]
    \centering
    \pgfimage{images/event-tree}
    \caption{Diagrama de árbol para lanzamiento de tres monedas}
    \label{fig:event-tree}
  \end{figure}
  Un \emph{camino}
  a través del árbol representa un posible resultado
  del experimento.
  En el caso ilustrado de lanzar tres monedas
  hay ocho caminos,
  suponiendo cada uno de los resultados igualmente probables
  le asignaríamos probabilidad \(1 / 8\) a cada uno de ellos.
  Si \(E\) es el evento ``hay al menos una cara'',
  el evento \(\overline{E}\) es ``no hay ninguna cara''.
  Es claro que solo si el resultado es tres veces sello
  (o sea \(\mathrm{TTT}\),
   que corresponde a \(\omega_8\))
  se da \(\overline{E}\),
  con lo que:
  \begin{equation*}
    \Pr(\overline{E})
      = \frac{1}{8}
  \end{equation*}
  Por lo tanto,
  usando el ítem~\ref{item:complement}
  del teorema~\ref{theo:properties-probabilities-events}:
  \begin{equation*}
    \Pr(E)
      = 1 - \Pr(\overline{E})
      = \frac{7}{8}
  \end{equation*}
  La utilidad del diagrama es que si en cada bifurcación
  anotamos la probabilidad de tomar los distintos caminos,
  la probabilidad de un evento es el producto de las probabilidades
  en el camino entre la raíz y ese evento.
  En nuestro caso,
  si asumimos
  que la probabilidad de cara y sello son ambas \(1 / 2\),
  independiente de la posición en el árbol,
  obtenemos \(1 / 8\) para \(f(\omega_8)\).
  Describir un experimento mediante un diagrama de árbol
  ayuda a organizar las ideas,
  y evita errores como omitir alguna de las posibilidades
  o asignación incoherente de probabilidades.

  Un ejemplo más complejo provee el dilema de Monty Hall,%
    \index{Monty Hall, dilema de}
  discutido originalmente
  en la popular columna ``Ask Marilyn'' de Marilyn vos Savant
  en la revista \foreignlanguage{english}{Parade},
  recogido luego en su libro~\cite{savant97:_power_logic_thinking}.
  La controversia se inició con la pregunta:
  \hybridblockquote{english}
     [Craig F. Whitaker,
      Columbia, MD]{%
     Suppose you're on a game show,
     and you're given the choice of three doors.
     Behind one door is a car,
     behind the others,
     goats.
     You pick a door,
     say number 1,
     and the host,
     who knows what's behind the doors,
     opens another door,
     say number 3,
     which has a goat.
     He says to you,
     ``Do you want to pick door number 2?''
     Is it to your advantage to switch your choice of doors?
  }
  El nombre viene de un popular presentador de televisión,
  en cuyo programa aparecía esta sección.
  El dilema dio lugar a encendidas discusiones,
  mientras aplicar las técnicas vistas lo resuelve sin ambigüedades.

  Primeramente,
  necesitamos describir la situación en forma precisa.
  Supondremos que el auto está con la misma probabilidad tras cada puerta,
  que el participante elige la puerta con la misma probabilidad,
  independiente de la ubicación del auto,
  y finalmente que Monty elige la puerta a abrir con la misma probabilidad
  entre las no elegidas por el participante y que no ocultan el auto.
  Por simetría,
  podemos designar por \(A\) la puerta elegida por el participante,
  cosa que dará un tercio de los casos
  (y probabilidades) a considerar,
  y nuestro diagrama considera solo este caso.
  Enseguida,
  consideramos las tres posibilidades para la ubicación del auto,
  y finalmente la elección de la puerta a abrir por Monty.
  \begin{figure}[ht]
    \centering
    \pgfimage{images/monty-hall}
    \caption{Árbol para el dilema de Monty Hall}
    \label{fig:monty-hall}
  \end{figure}
  Esto da el árbol de la figura~\ref{fig:monty-hall},
  del que obtenemos la probabilidad de ganar el auto
  al cambiar de puerta o no.
  Resulta que la probabilidad de ganar al cambiar de puerta
  es de \(2/3\),
  y de no cambiar de puerta solo \(1/3\),
  cosa a primera vista contradictoria.

\section{Probabilidad condicional}
\label{sec:probabilidad-condicional}
\index{probabilidad!condicional}

  Es frecuente querer determinar la probabilidad de un evento \(A\)
  sabiendo que un evento \(B\) ocurrió.
  Expresando esta situación como conjuntos,
  nos interesa la intersección entre \(A\) y \(B\);
  como sabemos que ocurrió \(B\),
  la probabilidad relativa es:
  \begin{equation*}
    \frac{\Pr(A \cap B)}{\Pr(B)}
  \end{equation*}
  Adoptamos esto como definición:
  \begin{definition}
    La \emph{probabilidad condicional} del evento \(A\)
    dado que ocurrió el evento \(B\) es:
    \begin{equation}
      \label{eq:def-conditional-probability}
      \Pr(A \vert B)
	= \frac{\Pr(A \cap B)}{\Pr(B)}
    \end{equation}
  \end{definition}
  Por ejemplo,
  si al lanzar dos dados la suma es diez,
  ¿cuál es la probabilidad de que haya un seis?
  En este caso,
  tenemos \(B\) como el evento que la suma es diez,
  vale decir:
  \begin{equation*}
    B = \{ (4, 6), (5, 5), (6, 4) \}
  \end{equation*}
  el evento \(A\) es que hay un único seis:
  \begin{equation*}
    A = \{ (1, 6), (2, 6), (3, 6), (4, 6), (5, 6),
	   (6, 1), (6, 2), (6, 3), (6, 4), (6, 5) \}
  \end{equation*}
  Resulta \(A \cap B = \{ (4, 6), (6, 4) \}\),
  con la suposición que todos los eventos elementales
  son igualmente probables es:
  \begin{equation*}
    \Pr(A \vert B)
      = \frac{2 / 36}{3 / 36}
      = \frac{2}{3}
  \end{equation*}

  Para el caso de una secuencia más larga de eventos,
  \(A_1 A_2 \dots A_n\),
  resulta:
  \begin{theorem}
    \label{theo:regla-producto}
    Tenemos la \emph{regla del producto}:
    \begin{equation}
      \label{eq:regla-producto}
      \Pr(A_1 A_2 \dots A_n)
	= \Pr(A_1)
	    \cdot \Pr(A_2 \vert A_1)
	    \cdot \Pr(A_3 \vert A_1 A_2)
	    \dotsm
	    \cdot \Pr(A_n \vert A_1 A_2 \dotsm A_{n - 1})
    \end{equation}
  \end{theorem}
  \begin{proof}
    La demostración formal es por inducción.%
      \index{demostracion@demostración!induccion@inducción}
    Ilustraremos la idea con el caso \(n = 3\):
    \begin{equation*}
      \Pr(A) \Pr(B \vert A) \Pr(C \vert A \cap B)
	= \Pr(A)
	    \, \frac{\Pr(A \cap B)}{\Pr(A)}
	    \, \frac{\Pr(A \cap B \cap C)}{\Pr(A \cap B)}
	= \Pr(A \cap B \cap C)
    \end{equation*}
    Esto corresponde al lado derecho e izquierdo,
    respectivamente,
    de~(\ref{eq:regla-producto}).
  \end{proof}

\section{Regla de Bayes}
\label{sec:regla-Bayes}
\index{probabilidad!Bayes, regla de|see{Bayes, regla de}}
\index{Bayes, regla de}

  Supongamos que \(B_1, B_2, \dotsc, B_n\)
  es una partición de \(\Omega\)
  (son eventos mutuamente excluyentes).
  Por el corolario~\ref{cor:disjoint-events-intersection}
  y la definición de probabilidad condicional
  obtenemos la \emph{ley de probabilidad total}:
  \begin{align}
    \Pr(A)
      &= \sum_{1 \le k \le n} \Pr(A \cap B_k) \notag \\
      &= \sum_{1 \le k \le n} \Pr(A \vert B_k) \, \Pr (B_k)
	   \label{eq:law-total-probability}
  \end{align}
  Combinando la definición de probabilidad condicional
  con la ley de probabilidad total
  da la importante \emph{ley de Bayes}:
  \begin{align}
    \Pr(B_k \vert A)
      &= \frac{\Pr(A \cap B_k)}{\Pr(A)} \notag \\
      &= \frac{\Pr(A \vert B_k) \, \Pr(B_k)}
	      {\sum_{1 \le i \le n}
		 \Pr(A \vert B_i) \, \Pr (B_i)}
	   \label{eq:Bayes-rule}
  \end{align}

  Para ilustrar la regla de Bayes,
  consideremos una empresa
  que tiene tres fábricas que producen chips,
  la planta~1 produce un \(20\)\% del total,
  la~2 un \(35\)\% y la~3 el \(45\)\%~restante.
  Las tasas de fallas en los chips de las distintas plantas son
  \(1\)\%,
  \(5\)\% y~\(3\)\%,
  respectivamente.
  Se tiene un chip defectuoso.
  ¿Cuál es la probabilidad
  de que haya sido fabricado en la planta~1?

  Sea \(A\) el evento que un chip es defectuoso,
  y sean los \(B_i\)
  los eventos que el chip haya sido fabricado en la planta~\(i\).
  Claramente los \(B_i\) particionan \(\Omega\).
  Las fracciones de la producción corresponden a los \(\Pr(B_i)\),
  las tasas de fallas por planta son los \(\Pr(A \vert B_i)\).
  Usando la regla de Bayes:
  \begin{equation*}
    \Pr(B_1 \vert A)
      = \frac{0,20 \cdot 0,01}
	     {0,20 \cdot 0,01 + 0,35 \cdot 0,05 + 0,45 \cdot 0,03}
      = 0,0606
  \end{equation*}
  Es un poco más del \(6\)\%.

\section{Independencia}
\label{sec:independencia}
\index{probabilidad!independencia}

  Dos eventos se dicen \emph{independientes}%
    \index{probabilidades!evento!independiente}
  si saber que ocurrió uno de ellos
  no altera la probabilidad del otro.
  Vale decir,
  \(A\) y \(B\) son independientes si:
  \begin{equation*}
    \Pr(A \vert B)
      = \Pr(A)
  \end{equation*}
  Alternativamente,
  incluyendo el caso \(B = \varnothing\):
  \begin{equation}
    \label{eq:def-independencia}
    \Pr(A \cap B)
      = \Pr(A) \, \Pr(B)
  \end{equation}
  Esto muestra que si \(A\) es independiente de \(B\)
  entonces \(B\) es independiente de \(A\).
  Esto claramente se puede extender a más eventos
  independientes a pares.

  De forma similar,
  si tenemos una variable aleatoria \((X, Y)\),
  se dice que las variables \(X\) e \(Y\) son independientes%
    \index{probabilidades!variable!independiente}
  si la función de distribución conjunta cumple:
  \begin{equation}
    \label{eq:independent-variables}
    f_{(X, Y)}(x, y)
      = f_X(x) \cdot f_Y(y)
  \end{equation}

\section{Las principales distribuciones discretas}
\label{sec:principales-distribuciones-discretas}

  Sabemos que cierta moneda da sello con probabilidad \(p\)
  y cara con probabilidad \(1 - p\)
  (no estamos suponiendo
   que ambos resultados son igualmente probables)
  al lanzarla.
  Al lanzarla una vez,
  considerando sello como ``éxito''
  (o \(1\))
  y cara como ``falla''
  (o \(0\))
  se habla de \emph{ensayo de Bernoulli}%
    \index{Bernoulli, ensayo de}
  (en inglés \emph{\foreignlanguage{english}{Bernoulli trial}}),
  la variable \(X\) que representa este experimento
  tiene probabilidad \(p\) de ser \(1\) y \(1 - p\) de ser \(0\).
  Se dice que \(X\) tiene \emph{distribución de Bernoulli},%
    \index{probabilidad!distribucion de Bernoulli@distribución de Bernoulli|see{Bernoulli, distribución de}}%
    \index{Bernoulli, distribucion de@Bernoulli, distribución de|textbfhy}
  y se anota:
  \begin{equation}
    \label{eq:distributed-Ber}
    X \sim \operatorname{\boldsymbol{\mathsf{Ber}}}(p)
  \end{equation}
  Su función generatriz de probabilidad es simplemente:
  \begin{equation}
    \label{eq:PGF-Ber}
    (1 - p) + z p
      = 1 + p (z - 1)
  \end{equation}

  Considerando nuevamente la moneda anterior,
  pero ahora lanzándola \(n\) veces,
  podemos describir el espacio muestral
  como el conjunto de \(n\)\nobreakdash-tuplas de \(0\) y \(1\),
  con \(0\) para cara y \(1\) para sello.
  ¿Cómo debiéramos definir las probabilidades
  para los eventos individuales?
  Es natural suponer que los lanzamientos son independientes,
  y que además la probabilidad de que resulte cara
  no cambia de un lanzamiento a otro.
  O sea,
  el número de caras es la suma de \(n\) ensayos de Bernoulli,
  independientes e idénticamente distribuidos.
  A esta situación común
  en que tenemos variables \(X_1\), \(X_2\), \ldots, \(X_n\)
  independientes e idénticamente distribuidas
  se suele abreviar como \emph{iid}
  (del inglés
   ``\emph{\foreignlanguage{english}{independent, identically distributed}}'',
   que casualmente sirve de abreviación del castellano también).%
     \index{iid|see{probabilidad!independientes e idénticamente distribuidas}}%
     \index{probabilidad!independientes e identicamente distribuidas@independientes e idénticamente distribuidas|textbfhy}
  Por~\eqref{eq:PGF-sum}
  y la función generatriz de probabilidad
  del ensayo de Bernoulli~\eqref{eq:PGF-Ber}
  esto significa que la función generatriz de probabilidad es:
  \begin{equation}
    \label{eq:PGF-Bin}
    (1 + p (z - 1))^n
  \end{equation}
  De~\eqref{eq:PGF-Bin}
  la distribución es directamente:
  \begin{equation}
    \label{eq:binomial-distribution}
    \Pr(X = k)
      = \binom{n}{k} p^k (1 - p)^{n - k}
  \end{equation}
  Esta es la \emph{distribución binomial}.%
    \index{probabilidad!distribucion binomial@distribución binomial|textbfhy}
  Si \(X\) representa el número de caras en este experimento,
  se anota:
  \begin{equation}
    \label{eq:distributed-Bin}
    X \sim \operatorname{\boldsymbol{\mathsf{Bin}}}(n, p)
  \end{equation}

  Si llamamos \(C_k\) al evento que al primer sello
  ocurre en el lanzamiento \(k\),
  sabemos que hay \(k - 1\) caras seguidas por un sello.
  Como discutido en la sección~\ref{sec:diagramas-arbol}
  esto resulta en:
  \begin{equation*}
    \Pr(C_k)
      = (p - 1)^{k - 1} p
  \end{equation*}
  Nótese que el espacio muestral es infinito en este caso.
  A esta distribución se le llama \emph{geométrica}.%
    \index{probabilidad!distribucion geometrica@distribución geométrica|textbfhy}
  Si \(X\) representa el número de lanzamientos,
  se anota:
  \begin{equation}
    \label{eq:distributed-G}
    X \sim \operatorname{\boldsymbol{\mathsf{G}}}(p)
  \end{equation}
  La función generatriz de probabilidades es:
  \begin{equation}
    \label{eq:PGF-G}
    \sum_{k \ge 1} (1 - p)^{k - 1} p z^k
      = \frac{z p}{1 - (1 - p) z}
  \end{equation}

  Otra distribución importante resulta de considerar una urna
  conteniendo un total de \(N\) bolas,
  \(r\) de las cuales son rojas
  y las demás negras.
  Se extraen \(n\) bolas sin orden y sin reposición,
  y nos preguntamos cuántas de ellas son rojas.
  Esto sirve por ejemplo para modelar encuestas.
  Estamos eligiendo \(k\) de las \(r\) bolas rojas
  y \(n - k\) de las \(N - r\) bolas negras,
  al aplicar la regla del producto
  y luego calcular la proporción del total de posibles subconjuntos
  de \(n\) elementos tomados del total de \(N\)
  resulta la \emph{distribución hipergeométrica}:%
    \index{probabilidad!distribucion hipergeometrica@distribución hipergeométrica|textbfhy}
  \begin{equation}
    \label{eq:Hyp-distribution}
    \Pr(X = k)
      = \frac{\binom{r}{k} \binom{N - r}{n - k}}{\binom{N}{r}}
  \end{equation}
  En este caso escribimos:
  \begin{equation}
    \label{eq:distributed-Hyp}
    X \sim \operatorname{\boldsymbol{\mathsf{Hyp}}}(n, r, N)
  \end{equation}
  La función generatriz de probabilidades
  no es una función elemental.

  Una distribución muy importante es la de Poisson.%
    \index{probabilidad!distribucion de Poisson@distribución de Poisson|see{Poisson, distribución de}}%
    \index{Poisson, distribucion de@Poisson, distribución de|textbfhy}
  Resulta de considerar un intervalo de tiempo
  en el cual ocurren eventos al azar
  en promedio a una tasa de \(\lambda\).
  Una manera de derivarla es considerar un intervalo de largo \(1\),
  en el cual ocurrirán en promedio \(\lambda\) eventos.
  Si dividimos el intervalo en \(n\) subintervalos del mismo largo,
  en cada subintervalo
  esperamos que ocurran \(\lambda / n\) eventos.
  Si \(n\) es grande,
  habrá a lo más un evento por subintervalo,
  y bajo el supuesto que los eventos ocurren al azar
  esto corresponde a una secuencia de \(n\)~ensayos de Bernoulli%
    \index{Bernoulli, ensayo de}
  con probabilidad \(\lambda / n\),
  o sea el número de eventos sigue una distribución binomial:%
    \index{probabilidad!distribucion binomial@distribución binomial}
  \begin{equation*}
    \Pr(X = k)
      = \lim_{n \rightarrow \infty}
	  \binom{n}{k}
	    \left( \frac{\lambda}{n} \right)^k
	    \left( 1 - \frac{\lambda}{n} \right)^{n - k}
  \end{equation*}
  En términos de la notación asintótica
  de la sección~\ref{sec:notacion-asintotica}%
    \index{notacion asintotica@notación asintótica}
  tenemos del límite clásico para \(n \rightarrow \infty\)
  y \(k\) fijo:
  \begin{equation*}
    \left( 1 - \frac{\lambda}{n} \right)^{n - k}
      \sim \mathrm{e}^{- \lambda}
  \end{equation*}
  Por otro lado:
  \begin{equation*}
    \frac{n!}{(n - k)!}
      = n (n - 1) \dotsm (n - k + 1)
      \sim n^{\underline{k}}
  \end{equation*}
  Uniendo las anteriores piezas queda en el límite:
  \begin{equation}
    \label{eq:Pois-distribution}
    \Pr(X = k)
      = \frac{\lambda^k}{k!} \mathrm{e}^{- \lambda}
  \end{equation}
  Escribimos:
  \begin{equation}
    \label{eq:distributed-Pois}
    X \sim \operatorname{\boldsymbol{\mathsf{Pois}}}(\lambda)
  \end{equation}
  Para la función generatriz de probabilidad:
  \begin{equation}
    \label{eq:PGF-Pois}
    \sum_{k \ge 0} \frac{\lambda^k}{k!} \mathrm{e}^{- \lambda} z^k
      = \mathrm{e}^{\lambda (z - 1)}
  \end{equation}

  Otra distribución que se encuentra ocasionalmente
  es la \emph{binomial negativa}%
    \index{probabilidad!distribucion binomial negativa@distribución binomial negativa|see{Pascal, distribución de}}%
    \index{Pascal, distribucion de@Pascal, distribución de|textbfhy}
  (o de Pascal).
  Suponga ensayos de Bernoulli independientes consecutivos,%
    \index{Bernoulli, ensayo de}
  nos interesa el número de experimentos con resultado uno
  antes de acumular \(r\) ceros
  (se suele hablar de ``éxitos'' y ``fallas'',
   interesa el número de éxitos para \(r\) fallas;
   pero las ``fallas'' no tienen porqué ser negativas,
   por ejemplo modela
   el número de penales antes de completar tres goles).
  Cuidado,
  hay una variedad de definiciones ligeramente diferentes.
  Si son \(k\) éxitos,
  hay \(k + r\) ensayos en total,
  y sabemos que el último resultado es \(0\).
  Quedan por distribuir \(r - 1\) fallas
  entre los primeros \(k + r - 1\) experimentos:
  \begin{equation}
    \label{eq:NB-distribution}
    \Pr(X = k)
      = \binom{k + r - 1}{k} (1 - p)^r p^k
  \end{equation}
  Se le llama binomial negativa por~\eqref{eq:binomial(-n,k)}:
  \begin{equation*}
    \binom{k + r - 1}{k}
      = (-1)^k \binom{-r}{k}
  \end{equation*}
  Si tomamos para \(r > 0\) cualquiera:
  \begin{equation*}
    \Pr(X = k)
      = (-1)^k \binom{-r}{k} (1 - p)^r p^k
  \end{equation*}
  Esta es una distribución de probabilidad,
  ya que:
  \begin{align*}
    \Pr(X = k)
      &\ge 0 \\
    \sum_{k \ge 0} \Pr(X = k)
      &=   \sum_{k \ge 0} (-1)^k \binom{-r}{k} (1 - p)^r p^k \\
      &=   (1 - p)^r \sum_{k \ge 0} \binom{-r}{k} (-p)^k \\
      &=   (1 - p)^r (1 - p)^{-r} \\
      &=   1
  \end{align*}
  Se anota:
  \begin{equation}
    \label{eq:distributed-NB}
    X \sim \operatorname{\boldsymbol{\mathsf{NB}}}(r, p)
  \end{equation}
  Por la discusión precedente \(r\) puede ser un real positivo,
  no solo un entero.
  Para la función generatriz de probabilidad tenemos:
  \begin{equation}
    \label{eq:PGF-NB}
    \sum_{k \ge 0} (-1)^k \binom{-r}{k} (1 - p)^r p^k z^k
      = (1 - p)^r \sum_{k \ge 0} \binom{-r}{k} (- p z)^k
      = \left( \frac{1 - p}{1 - p z} \right)^r
  \end{equation}

  Otra situación común es tener \(n\) posibilidades
  todas igualmente probables
  (\(1\) a \(n\),
   como en el caso de lanzar un dado).
  La distribución es simplemente:%
    \index{probabilidad!distribucion uniforme@distribución uniforme|textbfhy}
  \begin{equation}
    \label{eq:U-distribution}
    \Pr(X = k)
      = \begin{cases}
	  \displaystyle \frac{1}{n} & \text{si \(1 \le k \le n\)} \\
	  0			    & \text{caso contrario}
	\end{cases}
  \end{equation}
  La función generatriz de probabilidad es:
  \begin{equation}
    \label{eq:PGF-U}
    \sum_{1 \le k \le n} \frac{z^k}{n}
      = \frac{z (1 - z^n)}{n (1 - z)}
  \end{equation}

\section{Valor esperado}
\label{sec:valor-esperado}
\index{probabilidad!valor esperado}

  Aunque la distribución de una variable aleatoria
  contiene toda la información sobre probabilidades,
  suele ser más útil contar con características numéricas simples.
  Formalmente:
  \begin{definition}
    \label{def:expectation}
    Sea \(X\) una variable aleatoria discreta con valores reales
    y \(g \colon \mathbb{R} \rightarrow \mathbb{R}\)
    una función arbitraria.
    El \emph{valor esperado} de \(g(X)\)
    se define como:
    \begin{equation}
      \label{eq:def:expectation}
      \E[g(X])
	= \sum_{x \in \Omega} g(x) \, \Pr(X = x)
    \end{equation}
  \end{definition}
  El caso más importante es el valor esperado de \(X\),
  que amerita notación especial:
  \begin{equation}
    \label{eq:def:mu}
    \mu
      = \E[X]
  \end{equation}
  Por ejemplo,
  si \(X\)
  representa el número de puntos resultantes de lanzar un dado,
  suponiendo que todas las caras tienen la misma probabilidad:
  \begin{equation*}
    \E[X]
      = \sum_{1 \le k \le 6} k \, \Pr(X = k)
      = \frac{1}{6} \sum_{1 \le k \le 6} k
      = \frac{7}{2}
  \end{equation*}
  Este ejemplo incidentalmente muestra que \(\E[X]\)
  no tiene porqué ser un posible resultado del experimento.

  Una consecuencia extremadamente importante
  de la definición~\ref{def:expectation}
  es que el valor esperado es lineal:%
    \index{probabilidad!valor esperado!linealidad}
  \begin{theorem}
    \label{theo:expectation-linear}
    Sea la variable aleatoria \((X, Y)\)
    con distribución \(f_{(X, Y)}(x, y)\).
    Sean \(\alpha\) y \(\beta\) números reales
    y \(g\) y \(h\) funciones arbitrarias
    de las variables aleatorias \(X\) e \(Y\),
    respectivamente.
    Entonces:
    \begin{equation*}
      \E[\alpha g(X) + \beta h(Y)]
	= \alpha \, \E[g(X)] + \beta \, \E[h(Y])
    \end{equation*}
  \end{theorem}
  Nótese que no se hacen suposiciones
  sobre independencia de \(X\) e \(Y\).
  \begin{proof}
    Por definición:
    \begin{align*}
      \E[\alpha g(X) + \beta h(Y)]
	&= \sum_{(x, y)}
	    \left(
	      \alpha g(x) f_{(X, Y)}(x, y)
		+ \beta h(y) f_{(X, Y)} (x, y)
	      \right) \\
	&= \alpha \sum_{x, y} g(x) f_{(X, Y)}(x, y)
	     + \beta \sum_{x, y} h(y) f_{(X, Y)}(x, y) \\
	&= \alpha \sum_x g(x) \sum_y f_{(X, Y)}(x, y)
	     + \beta \sum_y h(y) \sum_x f_{(X, Y)}(x, y) \\
	&= \alpha \sum_x g(x) f_X(x)
	     + \beta \sum_y h(y) f_Y(y) \\
	&= \alpha \, \E[g(X)] + \beta \, \E[h(Y])
      \qedhere
    \end{align*}
  \end{proof}
  Nótese que esto vale incluso en caso que \(X = Y\),
  el caso más extremo de dependencia entre las variables.

  Resulta de interés acotar la dispersión de los posibles resultados.
  \begin{theorem}[Desigualdad de Markov]
    \label{theo:Markov-inequality}
    \index{Markov, desigualdad de}
    Sea \(X\) una variable aleatoria.
    Entonces para \(k > 0\):
    \begin{equation}
      \label{eq:Markov-inequality}
      \P(\lvert X \rvert \ge k)
	\le \E[ \lvert X \rvert ] / k
    \end{equation}
  \end{theorem}
  \begin{proof}
    La desigualdad se cumple trivialmente
    a menos que \(k > \E[ \lvert X \rvert ]\).
    Para tales \(k\):
    \begin{align*}
      k \P(\lvert X \rvert \ge k)
	&=  \sum_{r \ge k} k \P(\lvert X \rvert = r) \\
	&\le \sum_{r \ge k} r \P(\lvert X \rvert = r) \\
	&\le \sum_{r \ge 0} r \P(\lvert X \rvert = r) \\
	&=  \E[ \lvert X \rvert ]
    \qedhere
    \end{align*}
  \end{proof}

  Una importante medida
  de la dispersión de los datos es la \emph{varianza},%
    \index{probabilidad!varianza|textbfhy}%
    \index{varianza|see{probabilidad!varianza}}
  definida para una variable aleatoria \(X\)
  y una función \(g\) como:
  \begin{equation}
    \label{eq:definition-variance}
    \var[g(X)]
      = \E
	  \left[
	     \left( g(X) - \mathbb{E}\left( g(X) \right) \right)^2
	  \right]
  \end{equation}
  Podemos expresar:
  \begin{align}
    \var[g(X)]
      &= \E
	   \left[
	     \left( g(X) - \mathbb{E}\left( g(X) \right) \right)^2
	   \right]
	     \notag \\
      &= \E[g^2(X)]
	   - 2 \left(\E[g(X)] \right)^2
	   + \left(\E[g(X)] \right)^2
	     \notag \\
      &= \E\left[ g^2(X) \right]
	   - \left(\E\left[ g(X) \right] \right)^2
	     \label{eq:compute-variance}
  \end{align}
  Esto es más cómodo para cálculos.

  Comúnmente se usa la \emph{desviación estándar}%
    \index{desviacion estandar@desviación estándar|textbfhy},
  definida mediante:
  \begin{equation}
    \label{eq:standard-deviation-def}
    \sigma_X
      = \sqrt{\var[X]}
  \end{equation}

  Nuevamente podemos obtener una cota elemental:
  \begin{theorem}[Desigualdad de Chebychev]
    \label{theo:Chebychev-inequality}
    \index{Chebychev, desigualdad de}
    Sea \(X\) una variable aleatoria,
    y sea \(k > 0\) un número real.
    Si el valor esperado de \(X\) es \(\mu = \E[X]\)
    y su desviación estándar es \(\sigma = \sqrt{\var[X]}\)
    entonces:
    \begin{equation}
      \label{eq:Chebychev-inequality}
      \P(\lvert X - \mu \rvert \le k \sigma)
	\ge 1 - \frac{1}{k^2}
    \end{equation}
  \end{theorem}
  \begin{proof}
    Sea \(A = \{ r \colon \lvert x - \mu \rvert > k \sigma \}\).
    Entonces:
    \begin{align*}
      \var[X]
	&=   \E[(X - \mu)^2] \\
	&=   \sum_r (r - \mu)^2 \P(X = r) \\
	&\ge \sum_{r \in A} (r - \mu)^2 \P(X = r) \\
	&\ge k^2 \sigma^2 \sum_{r \in A} \P(X = r) \\
	&=   k^2 \sigma^2 \P(\lvert X - \mu \rvert > k \sigma)
    \end{align*}
    El resultado sigue de \(\var[X] = \sigma^2\).
  \end{proof}

  Supongamos una variable aleatoria \(X\)
  con distribución \(f_X\)
  y función generatriz de probabilidad \(G\).
  Es simple ver que:
  \begin{equation}
    \label{eq:PGF-expected-value}
    \E[X]
      = \sum_x x f_X(x)
      = G'(1)
  \end{equation}
  De forma similar:
  \begin{equation*}
    G''(1)
      = \sum_x x (x - 1) f_X(x)
      = \E[X^2] - \E[X]
  \end{equation*}
  Acá usamos el teorema~\ref{theo:expectation-linear};
  \(X^2\) y \(X\) definitivamente no son independientes,
  pero igual podemos sumar sus valores esperados.
  Combinando esto con~\eqref{eq:compute-variance}
  y recordando~\eqref{eq:PGF-expected-value}
  resulta:
  \begin{equation}
    \label{eq:PGF-variance}
    \var[X]
      = G''(1) + G'(1) - \left( G'(1) \right)^2
  \end{equation}
  Conociendo la función generatriz de probabilidad
  directamente tenemos los valores resumen
  más importantes de la variable.
  El lector curioso los tabulará para las distribuciones
  discutidas en la sección~\ref{sec:distribuciones-discretas}.

%%% Local Variables:
%%% mode: latex
%%% TeX-master: "clases"
%%% End:


% series-formales.tex
%
% Copyright (c) 2009, 2012-2014 Horst H. von Brand
% Derechos reservados. Vea COPYRIGHT para detalles

\chapter{Series formales de potencias}
\label{cha:series-formales}
\index{serie formal|textbfhy}

  Nuestro interés en las series de potencias%
    \index{serie de potencias}
  no es en su capacidad de definir funciones,
  sino simplemente como una representación compacta
  de una secuencia infinita.
  El desarrollo de la teoría de series formales
  nació de la observación que ciertas manipulaciones de series
  ``como si fueran polinomios''
  entregaban resultados correctos,
  incluso cuando una revisión más detallada demostraba
  que las operaciones no tenían validez.
  Resumiremos los resultados más importantes del área,
  que corroboran nuestras manipulaciones,
  a primera vista irresponsables y sin justificación,
  en los capítulos anteriores.
  Incluso veremos que las manipulaciones pueden justificarse
  si los coeficientes de la serie pertenecen a un anillo,
  no necesariamente son números reales o complejos.
  Esto es notable,
  estamos usando sumas infinitas
  en ámbitos en los cuales el concepto de límite
  necesario para justificar convergencia
  no es aplicable directamente.

\section{Un primer ejemplo}
\label{sec:un-primer-ejemplo}

  Si dejamos de lado el requerimiento de que la serie converja
  (y defina una función),
  podemos darle sentido incluso a series como:
  \begin{equation}
    \label{eq:gf-n!}
    f(z)
      =\sum_{n \ge 0} n! z^n
  \end{equation}
  que solo convergen para \(z = 0\),
  y para las que el análisis no tiene ningún uso.
  Podemos considerar la serie~\eqref{eq:gf-n!}
  como la función generatriz ordinaria de los factoriales.%
    \index{generatriz!ordinaria}%
    \index{factorial}
  Así:
  \begin{align*}
    z f(z)
      &= \sum_{n \ge 0} n! z^{n + 1} \\
    \mathrm{D}(z f(z))
      &= \sum_{n \ge 0} (n + 1)! z^n
       = \frac{f(z) - 1}{z} \\
  \intertext{Por el otro lado, derivando el producto:}
    \mathrm{D}(z f(z))
      &= f(z) + z f'(z) \\
    z f'(z)
      &= \frac{f(z) - 1}{z} - f(z) \\
  \intertext{El lado derecho es la función generatriz de \((n + 1)! - n!\),
	     e invita a sumar
	     (dividir por \(1 - z\) en funciones generatrices):}
    \frac{z f'(z)}{1 - z}
      &= \frac{1}{1 - z} \,
	   \left(
	     \frac{f(z) - 1}{z} - (f(z) - 1) - 1
	   \right) \\
      &= \frac{1}{1 - z} \,
	   \left(
	     (f(z) - 1) \, \left( \frac{1}{z} - 1 \right)
	   \right)
	   - \frac{1}{1 - z} \\
      &= \frac{1}{1 - z} \,
	   \left(
	     (f(z) - 1) \, \frac{1 - z}{z}
	   \right)
	   - \frac{1}{1 - z} \\
      &= \frac{f(z) - 1}{z} - \frac{1}{1 - z}
  \end{align*}
  Como:
  \begin{align*}
    z f'(z)
      &\ogf \langle n \, n! \rangle_{n \ge 0} \\
    \frac{z f'(z)}{1 - z}
      &\ogf \left\langle
	      \sum_{0 \le k \le n} k \cdot k!
	    \right\rangle_{n \ge 0}
  \end{align*}
  Aplicando nuevamente las propiedades,
  vemos que:
  \begin{equation}
    \label{eq:sn}
    \sum_{\mathclap{0 \le k \le n}} k \cdot k!
      = (n + 1)! - 1
  \end{equation}
  A pesar de su espeluznante derivación
  (no falta un paso en que no hagamos operaciones al menos dudosas
   con series infinitas que solo para \(z = 0\) convergen)
  la relación~\eqref{eq:sn} es correcta,
  cosa que el lector escéptico demostrará por inducción.%
    \index{demostracion@demostración!induccion@inducción}
  Resulta que estas operaciones pueden justificarse rigurosamente,
  tema que nos ocupará en este capítulo.

  Parte de lo que sigue viene de Shoup~%
    \cite{shoup09:_comput_introd_number_theor_algeb},
  las justificaciones siguen a Kauers~%
    \cite{kauers11:_concr_tetrah}.
  Operaciones con series pueden efectuarse con paquetes de álgebra simbólica,%
    \index{algebra simbolica@álgebra simbólica}
  como \texttt{maxima}~\cite{maxima14b:_computer_algebra},%
    \index{maxima@\texttt{maxima}}
  o aún mejor con sistemas especializados como \texttt{PARI/GP}~%
    \cite{PARI:2.7.2}.%
    \index{PARI/GP@\texttt{PARI/GP}}
  La biblioteca GiNaC~%
    \cite{bauer02:_ginac_fram_symbol_comput, GiNaC:1.6.2}%
    \index{GiNaC@\texttt{GiNaC}}
  permite manipular expresiones simbólicas,
  incluyendo series formales,
  y numéricas
  directamente en \cplusplus.%
    \index{C++ (lenguaje de programacion)@\cplusplus{} (lenguaje de programación)}

\section{Definición de serie formal}
\label{sec:serie-formal-def}

  Sea la serie:
  \begin{equation*}
    \sum_{n \ge 0} a_n z^n
  \end{equation*}
  donde los elementos \(a_n\) pertenecen a un anillo \(R\).%
    \index{anillo}

  Acá como en polinomios formales
  (capítulo~\ref{cha:anillos-polinomios})
  \(z\) es simplemente un \emph{símbolo}
  (también llamado \emph{indeterminada} o \emph{variable}).
  La consideramos como una construcción puramente formal,
  sin darle sentido a \(z\)
  ni preocuparse por convergencia.%
    \index{serie de potencias!convergencia}
  La única restricción que impone esto
  es que toda vez que se calcula un coeficiente
  deben efectuarse un número finito de operaciones
  (en un anillo arbitrario no son aplicables
   las ideas de límite y convergencia,
   necesarias para darle sentido a un número infinito de operaciones).
  Trataremos el caso en que \(R\) es un campo,%
    \index{campo (algebra)@campo (álgebra)}
  o al menos un dominio integral%
    \index{dominio integral}
  (un anillo conmutativo sin divisores de cero distintos de cero).
  Para evitar tener que mencionarlo infinidad de veces,
  usaremos la convención que \(R\) es un anillo general,
  \(D\) es un dominio integral
  y \(F\) es un campo.

  Definimos la igualdad entre series formales sobre el anillo \(R\):%
    \index{serie formal!igualdad}
  \begin{equation*}
    \sum_{n \ge 0} a_n z^n
       = \sum_{n \ge 0} b_n z^n \quad
	     \text{cuando \(a_n = b_n\) para todo \(n \ge 0\)}
  \end{equation*}
  Definimos además las operaciones:%
    \index{serie formal!operaciones}
  \begin{align*}
    \alpha \sum_{n \ge 0} a_n z^n
       &= \sum_{n \ge 0} (\alpha a_n) z^n
	    \qquad \text{para \(\alpha \in R\) o \(\alpha \in \mathbb{Z}\)} \\
    \sum_{n \ge 0} a_n z^n + \sum_{n \ge 0} b_n z^n
       &= \sum_{n \ge 0} (a_n + b_n) z^n \\
    \biggl( \, \sum_{r \ge 0} a_r z^r \biggr) \cdot
      \biggl( \, \sum_{s \ge 0} b_s z^s \biggr)
       &= \sum_{\substack{
		  r \ge 0 \\
		  s \ge 0
	       }} a_r b_s z^{r + s} \\
       &= \sum_{\mathclap{\substack{
			    n \ge 0 \\
			    0 \le k \le n
	       }}} a_k b_{n - k} z^n \\
       &= \sum_{n \ge 0} \biggl( \,
			    \sum_{0 \le k \le n} a_k b_{n - k}
			  \biggr) z^n
  \end{align*}
  Notar que en particular,
  para constantes \(\alpha\) y \(\beta\):
  \begin{equation*}
    \alpha \sum_{n \ge 0} a_n z^n + \beta \sum_{n \ge 0} b_n z^n
      = \sum_{n \ge 0} (\alpha a_n + \beta b_n) z^n
  \end{equation*}
  Las series formales como generalmente usadas hasta acá
  son un espacio vectorial de dimensión infinita%
    \index{espacio vectorial}
  sobre el campo \(\mathbb{R}\)
  (con base \(\{z^k\}_{k \ge 0}\)),
  con la operación adicional de multiplicación.

  Es rutina verificar que las series formales de potencias
  sobre el dominio integral \(D\) con variable \(z\)
  son un dominio integral,
  con:
  \begin{align*}
    0 &= \sum_{n \ge 0} 0 \, z^n \\
    1 &= \sum_{n \ge 0} [n = 0] \, z^n
  \end{align*}
  Al anillo de series formales sobre el anillo \(R\)
  lo llamaremos \(R \llbracket z \rrbracket\)
  (recuérdese que llamamos \(R[z]\)
   al anillo de polinomios en \(z\) sobre \(R\)).
  Para evitar distinciones inútiles
  consideramos \(R[z]\)
  subanillo de \(R\llbracket z \rrbracket\) de la forma natural.

\section{Unidades y recíprocos}
\label{sec:series:unidades-reciprocos}
\index{serie formal!unidad}
\index{serie formal!reciproco@recíproco}

  Sea \(F\) un campo.
  En el anillo de series formales \(F \llbracket z \rrbracket\)
  hay unidades que no son simplemente constantes
  (como ocurre en el correspondiente anillo de polinomios formales).
  Por ejemplo,
  en \(\mathbb{C}\llbracket z \rrbracket\):
  \begin{equation*}
    (1 - z) \cdot (1 + z + z^2 + z^3 + \dotsb)
      = (1 + z + z^2 + \dotsb) - (z + z^2 + z^3 + \dotsb)
      = 1
  \end{equation*}
  Si \(a_0 = 0\) la serie no puede tener recíproco,
  ya que no hay forma
  de crear un término constante del producto en tal caso.

  Por otro lado,
  supongamos que \(a_0 \ne 0\):
  \begin{align*}
    \biggl( \, \sum_{n \ge 0} a_n z^n \biggr) \cdot
      \biggl( \, \sum_{n \ge 0} b_n z^n \biggr)
       &= 1 \\
    \sum_{n \ge 0} \biggl( \,
		     \sum_{0 \le k \le n} a_{n -k} b_k
		   \biggr) z^n
       &= 1
  \end{align*}
  Para \(n = 0\) debe ser \(a_0 b_0 = 1\),
  o sea,
  \(b_0 = 1 / a_0\);
  luego para \(n > 0\) las sumas deben anularse:
  \begin{equation*}
    b_n
      = -\frac{1}{a_0} \, \sum_{0 \le k \le n - 1} a_{n - k} b_k
  \end{equation*}
  Con esto último se obtiene la secuencia de todos los \(b_n\),
  que es la secuencia de coeficientes del recíproco de la serie \(A(z)\).

\section{Secuencias de series}
\label{sec:series-secuencias}
\index{serie formal!secuencia}

  Para justificar rigurosamente el operar con series formales
  debemos desarrollar el marco adecuado.
  En términos generales,
  las operaciones son válidas siempre que el cálculo de cada coeficiente
  involucre un número finito de operaciones.
  Por ejemplo,
  en la sección~\ref{sec:series:unidades-reciprocos}
  vimos que el \(n\)\nobreakdash-ésimo coeficiente
  del recíproco de una serie cuyo término constante no es cero
  es una combinación lineal de los coeficientes \(0\) al \(n - 1\)
  de la serie,
  una suma finita.
  En contraste,
  ``evaluar'' la serie \(A(z)\) en algún ``punto'' implica una suma infinita.
  Usaremos \(A(0)\) como una notación cómoda
  para \(\left[ z^0 \right] A(z)\),
  eso sí.

  Por otro lado,
  sí tiene sentido reemplazar \(z\)
  por \(z + z^2\) en la serie formal \(A(z)\)
  (note que en la serie que estamos reemplazando
   el coeficiente de \(z^0\) se anula).
  Observamos que:
  \begin{equation*}
    (z + z^2)^n
      = z^n (z + 1)^n
      = z^n \sum_{0 \le k \le n} \binom{n}{k} \, z^k
  \end{equation*}
  Para:
  \begin{equation*}
    A(z)
      = \sum_{n \ge 0} a_n z^n
  \end{equation*}
  resulta:
  \begin{equation*}
    A(z + z^2)
      = \sum_{n \ge 0} a_n z^n \sum_{0 \le k \le n} \binom{n}{k} \, z^k
      = \sum_{n \ge 0}
	  \left(
	    \sum_{0 \le k \le n} \binom{n - k}{k} \, a_{n - 1}
	  \right) z^n
  \end{equation*}
  Como cada coeficiente de la nueva serie
  se calcula con un número finito de operaciones,
  tenemos una serie formal perfectamente definida.

  Para generalizar esto,
  requerimos definir
  la noción de límite de secuencias en \(R \llbracket z \rrbracket\).
  Informalmente,
  consideramos dos series como ``cercanas'' si coinciden sus primeros términos.
  \begin{definition}
    El \emph{orden} de una serie,%
      \index{serie formal!orden|textbfhy}
    \(\ord A(z)\),
    es el índice del primer coeficiente no cero.
  \end{definition}
  \begin{definition}
    La secuencia de series
    \(\left\langle A_k(z) \right\rangle_{k \ge 0}\)
    \emph{converge} a la serie \(A(z)\) si%
      \index{serie formal!secuencia!convergencia|textbfhy}
    \begin{equation*}
      \lim_{k \rightarrow \infty} \ord (A(z) - A_k(z))
	= \infty
    \end{equation*}
    En tal caso
    escribimos \(\lim_{k \rightarrow \infty} A_k(z) =  A(z)\).
  \end{definition}
  Si \(a_{n k} = \left[ z^n \right] A_k(z)\),
  hay un número finito de \(a_{n k}\)
  que difiere de \(\left[ z^n \right] A(z)\),
  y en un número finito de operaciones podemos calcular \(a_n\).

  Algunas consecuencias son las siguientes.
  \begin{theorem}
    \label{theo:series-operaciones}
    Sean \(\left\langle A_n(z) \right\rangle_{n \ge 0}\)
    y \(\left\langle B_n(z) \right\rangle_{n \ge 0}\) secuencias de series
    que convergen a \(A(z)\) y \(B(z)\) en \(R \llbracket z \rrbracket\),
    respectivamente.
    Entonces:
    \begin{enumerate}
    \item
     \(\displaystyle \left\langle A_n(z) + B_n(z) \right\rangle_{n \ge 0}\)
     converge,
     y \(\displaystyle \lim_{n \rightarrow \infty} A_n(z) + B_n(z)
	    = A(z) + B(z)\)
   \item
     \(\displaystyle
	  \left\langle A_n(z) \cdot B_n(z) \right\rangle_{n \ge 0}\)
     converge,
     y \(\displaystyle \lim_{n \rightarrow \infty} A_n(z) \cdot B_n(z)
	    = A(z) \cdot B(z)\)
    \end{enumerate}
  \end{theorem}
  \begin{proof}
    La demostración es aplicar hechos simples como:
    \begin{align*}
      \ord (A(z) + B(z))
	&\ge \min \{ \ord A(z), \ord B(z) \} \\
      \ord (A(z) \cdot B(z))
	&\ge \ord A(z) + \ord B(z)
    \end{align*}
    Omitiremos los detalles.
  \end{proof}

  Para series en \(R \llbracket z \rrbracket\):
  \begin{equation*}
    A(z)
      = \sum_{n \ge 0} a_n z^n
    \hspace{3em}
    B(z)
      = \sum_{n \ge 0} b_n z^n
  \end{equation*}
  Si \(b_0 = 0\),
  \(\ord (B(z))^k \ge k\).
  Consideremos la secuencia:
  \begin{align*}
    C_0(z)
      &= a_0 \\
    C_1(z)
      &= a_0 + a_1 B(z) \\
    C_2(z)
      &= a_0 + a_1 B(z) + a_2 (B(z))^2 \\
      &\vdots \\
    C_k(z)
      &= a_0 + a_1 B(z) + a_2 (B(z))^2 + \dotsb + a_k (B(z))^k \\
  \end{align*}
  Los coeficientes de \(C_k(z)\) coinciden hasta el de orden \(k\)
  con todos los sucesores en la secuencia,
  luego esta converge a una serie \(C(z)\).%
    \index{serie formal!secuencia!convergencia}
  \begin{definition}
    \index{serie formal!composicion@composición|textbfhy}
    Definimos la \emph{composición} de las series \(A(z)\) y \(B(z)\),
    donde \(b_0 = 0\),
    como:
    \begin{equation*}
      A(B(z))
	= \sum_{n \ge 0} a_n (B(z))^n
	= \lim_{k \rightarrow \infty} C_k(z)
    \end{equation*}
  \end{definition}

  Un teorema importante es:
  \begin{theorem}
    \label{theo:series-composicion-homomorfismo}
    \index{anillo!homomorfismo}
    Para todo \(U(z) \in R \llbracket z \rrbracket\)
    con \(U(0) = 0\),
    el mapa
      \(\Phi_U \colon
	  R \llbracket z \rrbracket \rightarrow R \llbracket z \rrbracket\)
    definido mediante \(\Phi_U(A(z)) = A(U(z))\)
    es un homomorfismo de anillo.
  \end{theorem}
  \begin{proof}
    Sean:
    \begin{equation*}
      A(z)
	= \sum_{n \ge 0} a_n z^n
      \hspace{3em}
      B(z)
	= \sum_{n \ge 0} b_n z^n
    \end{equation*}
    Demostrar que \(\Phi_U(A(z) + B(z)) = \Phi_U(A(z)) + \Phi_U(B(z))\)
    es simple,
    y queda de ejercicio.

    Para la multiplicación:
    \begin{align*}
      \Phi_U(A(z) \cdot &B(z)) - \Phi_U(A(z)) \cdot \Phi_U(B(z)) \\
	&= \sum_{n \ge 0}
	     \left(
	       \sum_{0 \le k \le n} a_k b_{n - k}
	     \right) U^n
	    - \left(
		\sum_{n\ge 0} a_n U^n
	      \right) \cdot
		\left(
		  \sum_{n\ge 0} b_n U^n
		\right) \\
	&= \lim_{N \rightarrow \infty}
	     \sum_{0 \le n \le N}
	       \left(
		 \sum_{0 \le k \le n} a_k b_{n - k}
	       \right) U^n \\
	&\hspace{3em}
	    - \lim_{N \rightarrow \infty}
		\left(
		  \sum_{0 \le n \le N} a_n U^n
		\right) \cdot
		  \lim_{N \rightarrow \infty}
		    \left(
		      \sum_{0 \le n \le N} b_n U^n
		    \right) \\
	 &= \lim_{N \rightarrow \infty}
	      \left(
		\sum_{0 \le n \le N}
		  \left(
		    \sum_{0 \le k \le n} a_k b_{n - k}
		  \right) U^n
		- \left(
		    \sum_{0 \le n \le N} a_n U^n
		  \right) \cdot
		    \left(
		      \sum_{0 \le n \le N} b_n U^n
		    \right)
	      \right) \\
	 &= - \lim_{N \rightarrow \infty}
		\sum_{N + 1 \le n \le 2 N}
		  \left(
		    \sum_{0 \le k \le n} a_k b_{n - k}
		  \right) U(z)^n
    \end{align*}
    En esto hemos usado el teorema~\ref{theo:series-operaciones}.
    Resta demostrar que el último límite es infinito:
    \begin{equation*}
      \ord
	\left(
	  \sum_{N + 1 \le n \le 2 N}
	    \left(
	      \sum_{0 \le k \le n} a_k b_{n - k}
	    \right) U(z)^n
	\right)
	\ge \ord U(z)^{N + 1}
    \end{equation*}
    Esto tiende a infinito cuando \(N \rightarrow \infty\),
    y es \(\Phi_U(A(z) \cdot B(z)) = \Phi_U(A(z)) \cdot \Phi_U(B(z))\).
  \end{proof}
  O sea,
  operar con las series \(A(z)\) y \(B(z)\)
  es lo mismo que operar con las series \(A(U(z))\) y \(B(U(z))\).
  La importancia del teorema~\ref{theo:series-composicion-homomorfismo}
  radica en que expresiones como:
  \begin{equation*}
    \frac{1}{1 - z - z^2}
  \end{equation*}
  son ambiguas:
  ¿Es el recíproco de la serie \(1 - z - z^2\),
  o es tal vez la composición de \(1 / (1 - u)\) con \(u = z + z^2\)?
  El teorema asegura que ambas son la misma serie.

  Con las mismas herramientas se pueden justificar sumas y productos infinitos
  de series formales.
  \begin{definition}
    Sea la secuencia \(\left\langle A_k(z) \right\rangle_{k \ge 0}\)
    de series formales en \(R \llbracket z \rrbracket\).
    Decimos que la suma infinita
    \begin{equation*}
      \sum_{k \ge 0} A_k(z)
    \end{equation*}
    \emph{converge} si la secuencia
    \begin{equation*}
      \sum_{0 \le k \le N} A_k(z)
    \end{equation*}
    converge en el sentido definido antes cuando \(N \rightarrow \infty\).
    Igualmente,
    decimos que el producto infinito
    \begin{equation*}
      \prod_{k \ge 0} A_k(z)
    \end{equation*}
    \emph{converge} si la secuencia
    \begin{equation*}
      \prod_{0 \le k \le N} A_k(z)
    \end{equation*}
    converge cuando \(N \rightarrow \infty\).
  \end{definition}
  Con esto:
  \begin{theorem}
    \label{theo:formal-series-convergence:sum+product}
    Sea \(\left\langle A_k(z) \right\rangle_{k \ge 0}\)
    una secuencia en \(R \llbracket z \rrbracket\).
    Entonces las siguientes son equivalentes:
    \begin{enumerate}
    \item\label{item:lfs:infty}
      \(\displaystyle \lim_{k \rightarrow \infty} \ord A_k(z) = \infty\)
    \item\label{item:lfs:sum}
      \(\displaystyle \sum_{k \ge 0} A_k(z)\) converge
    \item\label{item:lfs:prod}
      \(\displaystyle \prod_{k \ge 0} (1 + A_k(z))\) converge
    \end{enumerate}
  \end{theorem}
  \begin{proof}
    Demostraremos que (\ref{item:lfs:infty}) equivale a (\ref{item:lfs:sum}),
    la demostración que (\ref{item:lfs:infty}) equivale a (\ref{item:lfs:prod})
    es similar y se omite.

    Definamos:
    \begin{equation*}
      C_n(z)
	= \sum_{0 \le k \le n} A_k(z)
    \end{equation*}
    Primero demostramos
    (\ref{item:lfs:infty})~\(\implies\)~(\ref{item:lfs:sum}).
    Tenemos:
    \begin{equation*}
      \forall n \exists k_0 \forall k \ge k_0 \colon \ord A_k(z) > n
    \end{equation*}
    Esto es equivalente a decir:
    \begin{equation*}
      \forall n \exists k_0 \forall k \ge k_0
	\colon \ord (C_{k + 1}(z) - C_k(z)) > n
    \end{equation*}
    O sea,
    para cada \(n\) hay \(k_0\) tal que
    para todo \(k \ge k_0\) el valor de \(\left[ z^n \right] C_k(z)\)
    es fijo,
    llamémosle \(c_n\).
    Consideremos:
    \begin{equation*}
      C(z)
	= \sum_{r \ge 0} c_r z^r
    \end{equation*}
    Por construcción:
    \begin{equation*}
      \forall n \exists k_0 \forall k \ge k_0
	\colon \ord (C_k(z) - C(z)) > n
    \end{equation*}
    y la secuencia \(C_k(z)\) converge.

    Ahora demostramos
    (\ref{item:lfs:sum})~\(\implies\)~(\ref{item:lfs:infty}).
    Si \(C_k(z)\) converge a \(C(z)\),
    entonces:
    \begin{equation*}
      \forall n \exists k_0 \forall k \ge k_0
	\colon \ord (C_k(z) - C(z)) > n
    \end{equation*}
    Para tales \(n\) y \(k\)
    tenemos \(C_k(z) - C(z) > n\),
    y:
    \begin{equation*}
      \left[ z^n \right] A_k(z)
	= \left[ z \right] (C_{k + 1}(z) - C_k(z))
	= \left[ z \right] C_{k + 1}(z) - \left[ z \right] C_k(z))
	= 0
    \end{equation*}
    lo que es equivalente a:
    \begin{equation*}
      \forall n \exists k_0 \forall k \ge k_0
	\colon \ord A_k(z) > n
    \end{equation*}
    que es lo que había que demostrar.
  \end{proof}
  Nótese que en el ámbito de los reales \(a_n \rightarrow 0\)
  no asegura que \(\sum a_n\) ni \(\prod (1 + a_n)\) converjan.

\section{El principio de transferencia}
\label{sec:series-principio-transferencia}
\index{serie formal!principio de transferencia|textbfhy}
\index{serie de potencias}

  Cuando \(R = \mathbb{R}\) o \(R = \mathbb{C}\),
  es obvio preguntarse sobre la relación entre la serie formal
  y la función definida por la serie de potencias.
  Es claro que no toda serie formal corresponde a una función analítica,
  por ejemplo la serie
  \begin{equation*}
    \sum_{n \ge 0} n! z^n
  \end{equation*}
  converge únicamente para \(z = 0\).
  Pero como \(0! = 1\),
  tiene un recíproco como serie formal:
  \begin{equation*}
    \left( \sum_{n \ge 0} n! z^n \right)^{-1}
      = 1 - z - z^2 - 3 z^3 - 13 z^4 - 71 z^5 - 461 z^6 - \dotsb
  \end{equation*}
  Razonamientos válidos para series formales
  no necesariamente tienen sentido para series de potencias.

  Por el otro lado,
  si dos series de potencias son idénticas como funciones analíticas
  lo son como series formales:
  \begin{theorem}[Principio de transferencia]
    \label{theo:series-principio-transferencia}
    Sean:
    \begin{equation*}
      A(z)
	= \sum_{n \ge 0} a_n z^n
      \hspace{3em}
      B(z)
	= \sum_{n \ge 0} b_n z^n
    \end{equation*}
    funciones reales o complejas,
    analíticas en un vecindario \(\mathcal{U}\) no vacío de cero.%
      \index{funcion@función!analitica@analítica}
    Si \(A(z) = B(z)\) para todo \(z \in \mathcal{U}\),
    entonces \(a_n = b_n\) para todo \(n \in \mathbb{N}_0\).
  \end{theorem}
  \begin{proof}
    Bajo las suposiciones,
    \(C(z) = A(z) - B(z)\) es analítica
    e idénticamente \(0\) en \(\mathcal{U}\).
    Por el teorema de Taylor,%
      \index{Taylor, teorema de}
    esto significa que todos los coeficientes de \(C(z)\) son cero:
    \begin{equation*}
      \left[ z^n \right] C(z)
	= \left[ z^n \right] (A(z) - B(z))
	= \left[ z^n \right] A(z) - \left[ z^n \right] B(z)
	= a_n - b_n
	= 0
      \qedhere
    \end{equation*}
  \end{proof}
  Esto permite demostrar algunas identidades en forma simple.
  Por ejemplo,
  tenemos las expansiones:
  \begin{align*}
    \mathrm{e}^z
      &= \sum_{n \ge 0} \frac{z^n}{n!}
	   && \text{para todo \(z \in \mathbb{C}\)} \\
    \ln (1 + z)
      &= \sum_{n \ge 1} \frac{z^n}{n}
	   && \text{para \(\lvert z \rvert < 1\)}
  \end{align*}
  Como para las funciones analíticas respectivas
  \(\exp(\ln (1 + z)) = 1 + z\),
  esto vale para las series formales.
  Verificarlo en forma directa involucra largos y complicados cálculos.

\section{Derivadas e integrales formales}
\label{sec:series:derivadas-integrales}

  Definimos:%
    \index{serie formal!derivada}
  \begin{equation*}
    \frac{\mathrm{d}}{\mathrm{d} z} \,
      \biggl( \, \sum_{n \ge 0} a_n z^n \biggr)
      = \sum_{n \ge 1} n a_n z^{n - 1}
      = \sum_{n \ge 0} (n + 1) a_{n + 1} z^n
  \end{equation*}
  Esta es una definición puramente formal,
  no intervienen límites ni el significado de la serie como función.
  Definimos además para una serie formal \(f(z)\):
  \begin{align*}
    f^{(0)}(z)
      &= f(z) \\
    f^{(n + 1)}(z)
      &= \frac{\mathrm{d}}{\mathrm{d} z} \, f^{(n)}(z)
  \end{align*}
  Una notación alternativa útil es:
  \begin{equation*}
    \mathrm{D} f(z)
      = \frac{\mathrm{d}}{\mathrm{d} z} \, f(z)
  \end{equation*}
  bajo el entendido \(\mathrm{D}^n f(z) = f^{(n)}(z)\).
  Para las primeras derivadas se suele usar:
  \begin{equation*}
    f'(z)
      = \frac{\mathrm{d} f}{\mathrm{d} x}
    \hspace{3em}
    f''(z)
      = \frac{\mathrm{d}^2 f}{\mathrm{d} x^2}
    \hspace{3em}
    f'''(z)
      = \frac{\mathrm{d}^3 f}{\mathrm{d} x^3}
  \end{equation*}

  Tenemos también:
  \begin{theorem}
    \label{theo:derivadas-series}
    Sean \(f(z)\) y \(g(z)\) series formales.
    Entonces:
    \begin{align*}
      \mathrm{D}^n \left( a f(z) + b g(z)\right)
	&= a f^{(n)}(z) + b g^{(n)}(z) \\
      \mathrm{D}^n \left(f(z) \cdot g(z)\right)
	&= \sum_{0 \le k \le n} \binom{n}{k} \, f^{(k)}(z) \cdot g^{(n - k)}(z)
    \end{align*}
  \end{theorem}
  La fórmula para las derivadas de un producto
  se conoce bajo el nombre de Leibniz.
  Las demostraciones son rutinarias,
  y quedan de ejercicio.

  Para la composición de series definida antes
  \begin{theorem}[Regla de la cadena]
    \label{theo:cadena-series}
    Sean \(f(z)\) y \(g(z)\) series formales,
    con \(g(0) = 0\).
    Entonces:
    \begin{equation*}
      \frac{\mathrm{d}}{\mathrm{d} z} f(g(z))
	= f'(g(z)) \cdot g'(z)
    \end{equation*}
  \end{theorem}
  Para demostrarlo,
  primeramente se demuestra la derivada de una potencia de una serie,
  y usando esto se aplica término a término.
  Nuevamente es rutina,
  y nos ahorraremos los detalles.

  De la misma manera que obtenemos derivadas término a término,
  podemos calcular integrales:%
    \index{serie formal!integral}
  \begin{equation*}
    \int_0^z f(t) \, \mathrm{d} t
      = \sum_{n \ge 1} \frac{a_{n - 1}}{n} \, z^n
  \end{equation*}
  Es claro que se cumplen las relaciones fundamentales:
  \begin{equation*}
    \frac{\mathrm{d}}{\mathrm{d} z} \, \int_0^z f(t) \, \mathrm{d} t
      = \int_0^z \frac{\mathrm{d}}{\mathrm{d} t} \, f(t) \, \mathrm{d} t
      = f(z)
  \end{equation*}
  Podemos anotar para la antiderivada:
  \begin{equation*}
    \mathrm{D}^{-1} f(z)
      = \int_0^z f(z) \, \mathrm{d} z
  \end{equation*}

\section{Series en múltiples variables}
\label{sec:series-multivariables}
\index{serie formal!multivariada}

  Podemos también considerar series en más de una variable,
  si las variables son \(x\) e \(y\)
  anotaremos \(R\llbracket x, y \rrbracket\).
  Esto es,
  por ejemplo:
  \begin{equation*}
    A(x, y)
      = \sum_{\substack{
		r \ge 0 \\
		s \ge 0
	     }} a_{r, s} x^r y^s
  \end{equation*}
  Es simple
  (aunque engorroso)
  demostrar que \(R \llbracket x, y \rrbracket\)
  es isomorfo a \(R \llbracket x \rrbracket \llbracket y \rrbracket\)%
    \index{anillo!homomorfismo}
  y a \(R \llbracket y \rrbracket \llbracket x \rrbracket\),
  pero no nos detendremos en tales detalles.

  Para desarrollar una teoría de secuencias en series multivariables,
  definimos el \emph{orden total}%
    \index{serie formal!orden total|textbfhy}
  del término \(x_1^{n_1} x_2^{n_2} x_3^{n_3} \dotso x_m^{n_m}\)
  como \(n_1 + n_2 + \dotsb + n_m\).
  El orden (total) de la serie \(\ord A(x_1, x_2, \dotsc, x_m)\)
  es el orden total del término
  de orden total mínimo en \(A(x_1, x_2, \dotsc, x_m)\).
  Con esta definición
  si para una secuencia
  de series formales multivariadas \(A_k(x_1, x_2, \dotsc, x_n)\):
  \begin{equation*}
    \lim_{k \rightarrow \infty} \ord A_k (x_1, x_2, \dotsc, x_n)
      = \infty
  \end{equation*}
  sabemos que solo en un número finito de los \(A_k(x_1, x_2, \dotsc, x_n)\)
  el coeficiente de \(x_1^{m_1} x_2^{m_2} \dotsm x_n^{m_n}\)
  difiere,
  y lo podemos calcular en un número finito de pasos.
  Omitimos los detalles
  de teoremas análogos a los para el caso univariado,
  solo notamos que esto permite justificar
  en \(R \llbracket x, y \rrbracket\)
  series como:
  \begin{equation*}
    \exp(x + y)
      = \sum_{n \ge 0} \frac{(x + y)^n}{n!}
  \end{equation*}
  Esto no resulta del caso univariado,
  en \(R \llbracket x \rrbracket \llbracket y \rrbracket\)
  la serie \(x + y\) tiene término constante \(x\),
  que no es cero.
  Podemos definir derivadas (parciales) e integrales
  de forma similar al caso univariado.
  Usaremos la notación \(\mathrm{D}_x f\) para la derivada parcial
  respecto de \(x\) de la serie \(f\),
  también \(f_{x y}\) para la derivada respecto de \(x\) e \(y\).

  \begin{theorem}[Funciones implícitas]
    \index{serie formal!funcion implicita@función implícita}
    \label{theo:series-funciones-implicitas}
    Sea \(A(x, y) \in F \llbracket x, y \rrbracket\)
    tal que \(A(0, 0) = 0\)
    y \(\mathrm{D}_y A(0, 0) \ne 0\).
    Entonces existe
    una única serie formal \(f(x) \in F \llbracket x \rrbracket\)
    con \(f(0) = 0\)
    tal que \(A(x, f(x)) = 0\).
  \end{theorem}
  \begin{proof}
    Escribamos
    \begin{equation*}
      A(x, y)
	= \sum_{n \ge 0} a_n(x) y^n
	= \sum_{\substack{n \ge 0 \\ k \ge 0}} a_{n k} x^k y^n
    \end{equation*}
    donde \(a_n(x) \in F \llbracket x \rrbracket\).
    Las condiciones sobre \(A(x, y)\) resultan en
    \(a_0(x) = 0\) y \(a_1(x) \ne 0\).
    Mostraremos cómo calcular sucesivamente los coeficientes \(f_n\) de
    \begin{equation*}
      f(x)
	= \sum_{n \ge 0} f_n x^n
    \end{equation*}
    Como \(f(0) = 0\),
    ya tenemos \(f_0 = 0\).
    Enseguida:
    \begin{equation*}
      \left[ x \right] \sum_{n \ge 0} a_n(x) f(x)^n
	= 0
    \end{equation*}
    hace que baste considerar
    \begin{equation*}
      \left[ x \right] ( a_0(x) + a_1(x) f(x) )
	= \left[ x \right]
	     (a_{0 0} + a_{0 1} x + a_{1 0} f_1 x + a_{1 1} f_0 x + \dotsb)
	= 0
    \end{equation*}
    de donde despejamos
    \begin{equation*}
      f_1
	= - \frac{a_{0 1}}{a_{1 1}}
    \end{equation*}
    Esto es válido,
    ya que \(a_{1 1} = \mathrm{D}_y A(0, 0) \ne 0\).
    Continuamos:
    \begin{equation*}
      \left[ x^k \right] \sum_{n \ge 0} a_n(x) f(x)^n
	= 0
    \end{equation*}
    donde \(f(0) = 0\) permite truncar la suma,
    y se traduce en
    \begin{equation*}
      \left[ x^k \right]
	\sum_{0 \le n \le k} a_n(x) f(x)^n
    \end{equation*}
    Observamos que el único término en esto que depende de \(f_k\)
    viene de \(a_1(0) f_k\),
    todos los demás solo involucran \(f_0\), \ldots, \(f_{k - 1}\).
    Despejando,
    tenemos \(f_k\) en términos de los coeficientes anteriores,
    y \(f(x)\) queda determinada mediante un proceso convergente.
  \end{proof}

  Junto con el teorema~\ref{theo:series-composicion-homomorfismo},
  el teorema~\ref{theo:series-funciones-implicitas}
  nos dice que la substitución
  \(z \leadsto U(z)\) es un isomorfismo
  de \(F \llbracket z \rrbracket\) a sí mismo
  (un \emph{automorfismo})%
    \index{anillo!automorfismo}
  si \(\ord U(z) = 1\).
  Es claro que los coeficientes de tales funciones implícitas
  normalmente resultan bastante locos.

  Por ejemplo,
  podemos definir:
  \begin{equation*}
    W(z) \mathrm{e}^{W(z)}
      = z
  \end{equation*}
  con \(W(0) = 0\).
  Para esto tomamos
    \(A(x, y) = y \mathrm{e}^y - x \in \mathbb{R} \llbracket x, y \rrbracket\),
  como \(A(0, 0) = 0\) y \(\mathrm{D}_y A(0, 0) = 1 \ne 0\),
  se cumplen las condiciones del teorema~\ref{theo:series-funciones-implicitas}
  y tal serie de potencias \(W(z)\) existe.
  La demostración da una receta para obtener los coeficientes.

  Es relativamente sencillo el caso particular de ecuaciones de la forma
  \begin{equation*}
    u = t \phi(u)
  \end{equation*}
  donde \(\phi\) es una función dada de \(u\).
  Esta relación define \(u\) en función de \(t\),
  y ``estamos despejando \(u\) en términos de \(t\)''.
  Fue demostrada por Lagrange%
    \index{Lagrange, inversion de@Lagrange, inversión de|textbfhy}%
    \index{Lagrange-Burmann, inversion de@Lagrange-Bürmann, inversión de|see{Lagrange, inversión de}}
  y casi simultáneamente por Bürmann,%
    \index{Burmann, Hans Heinrich@Bürmann, Hans Heinrich}
  la forma dada acá es la de Bürmann.
  \begin{theorem}[Fórmula de inversión de Lagrange]
    \label{theo:LIF}
    Sean \(f(u)\) y \(\phi(u)\) series formales de potencias en \(u\)
    sobre un campo \(F\),
    con \(\phi(0) = 1\).
    Entonces hay una única serie formal \(u = u(t)\) que cumple:
    \begin{equation*}
      u = t \phi(u)
    \end{equation*}
    Además,
    el valor \(f(u(t))\) de \(f\) en el cero \(u = u(t)\),
    expandido en serie alrededor de \(t = 0\),
    cumple:
    \begin{equation*}
      \left[ t^n \right] \left\{ f(u(t)) \right\}
	 = \frac{1}{n} \, \left[ u^{n - 1} \right] \,
			    \left\{ f'(u) \phi(u)^n \right\}
    \end{equation*}
  \end{theorem}
  Dadas \(f\) y \(\phi\),
  esta fórmula da los coeficientes de \(f(u(t))\) en bandeja.
  No demostraremos este resultado,
  ya que nos llevaría demasiado fuera del rango de este ramo.
  La demostración del teorema puede encontrarse en el texto de Wilf~%
    \cite{wilf06:_gfology}.

  La razón del nombre es que si \(t = A(u)\),
  esta fórmula da \(u = u(t)\) mediante:
  \begin{equation*}
    u = t \frac{u}{A(u)}
  \end{equation*}

  Una aplicación entretenida provee la función de Cayley,%
    \index{Cayley, funcion de@Cayley, función de}
  definida por:
  \begin{equation}
    \label{eq:Cayley-function}
    C(z)
      = z \mathrm{e}^{C(z)}
  \end{equation}
  Con \(f(u) = u\)
  y \(\phi(u) = \mathrm{e}^u\) tenemos directamente los coeficientes:
  \begin{align}
    [z^n] C(z)
      &= \frac{1}{n} [u^{n - 1}] \mathrm{e}^{n u} \notag \\
      &= \frac{1}{n} \frac{n^{n - 1}}{(n - 1)!} \notag \\
      &= \frac{n^{n - 1}}{n!}
	    \label{eq:Cayley-coeff}
  \end{align}
  Por otro lado,
  con \(f(u) = \mathrm{e}^{\gamma u}\)
  y \(\phi(u) = \mathrm{e}^u\) resulta:
  \begin{align*}
    [z^n] \mathrm{e}^{\gamma C(z)}
      &= \frac{1}{n} [u^{n - 1}]
	   \left(
	     \frac{\mathrm{d}}{\mathrm{d} u} \mathrm{e}^{\gamma u}
	       \cdot \mathrm{e}^{n u}
	   \right) \\
      &= \frac{\gamma (\gamma + n)^{n - 1}}{n!}
  \end{align*}
  Comparando coeficientes de:
  \begin{equation*}
    \mathrm{e}^{(\alpha + \beta) C(z)}
      = \mathrm{e}^{\alpha C(z)} \cdot \mathrm{e}^{\beta C(z)}
  \end{equation*}
  se obtiene la fórmula binomial de Abel~\cite{abel26:_identity}:%
    \index{Abel, formula binomial de@Abel, fórmula binomial de}
  \begin{equation}
    \label{eq:Abel-binomial}
    (\alpha + \beta) (\alpha + \beta + n)^{n - 1}
      = \alpha \beta
	  \sum_{0 \le k \le n}
	    \binom{n}{k}
	      (\alpha + k)^{k - 1}
	      (\beta + n - k)^{n - k - 1}
  \end{equation}

%%% Local Variables:
%%% mode: latex
%%% TeX-master: "clases"
%%% End:


% euler-maclaurin.tex
%
% Copyright (c) 2010-2014 Horst H. von Brand
% Derechos reservados. Vea COPYRIGHT para detalles

\chapter{La fórmula de Euler-Maclaurin}
\label{cha:euler-maclaurin}

  Es común que interese el valor de alguna suma,
  particularmente alguna suma infinita.
  En muchos casos de interés la suma converge muy lentamente,
  y resulta indispensable
  contar con alguna técnica que permita acelerarla.
  En otros casos una expresión simple para un valor aproximado
  de una suma finita
  resulta mucho más útil que el valor exacto.
  Una de las técnicas principales para aproximar sumas infinitas
  es la fórmula de Euler-Maclaurin.
  En su desarrollo toman lugar central
  los polinomios y números de Bernoulli,
  que a su vez aparecen inesperadamente en muchas situaciones combinatorias.
  Entre otras aplicaciones,
  la fórmula de Euler-Maclaurin
  permite obtener aproximaciones simples para factoriales
  y números harmónicos,
  valores que a su vez son ubicuos en la combinatoria
  (y por tanto el análisis de algoritmos).

\section{Relación entre suma e integral}
\label{sec:relacion-suma-integral}

  Conceptualmente la suma y la integral
  están íntimamente relacionadas,
  ambas podemos representarlas como áreas bajo curvas
  como en la figura~\ref{fig:fz}.
  \begin{figure}[htbp]
    \centering
    \pgfimage{images/fz}
    \caption{Suma e integral como áreas}
    \label{fig:fz}
  \end{figure}
  Pareciera ser que la integral (área bajo la curva)
  y la suma (área bajo la escalera)
  tienden a tener una diferencia constante.
  Esto es exactamente lo que asegura nuestro siguiente teorema.
  \begin{theorem}[Maclaurin-Cauchy]
    \index{Maclaurin-Cauchy, teorema de|textbfhy}
    \label{theo:maclaurin-Cauchy}
    Sea \(f(z)\) una función continua,
    positiva
    y que tiende monotónicamente a cero.
    Entonces existe la constante de Euler:%
      \index{Euler, constante de (para una funcion)@Euler, constante de (para una función)}
    \begin{equation*}
      \gamma_f
	= \lim_{n \rightarrow \infty}
	    \left(
	      \sum_{1 \le k < n} f(k) - \int_1^n f(z)
					  \, \mathrm{d} z
	    \right)
    \end{equation*}
  \end{theorem}
  \begin{proof}
    Como \(f\) es continua,
    la integral existe para todo \(n \in \mathbb{N}\).
    Por ser decreciente:
    \begin{equation*}
      f(\lceil z \rceil)
	\le f(z)
	\le f(\lfloor z \rfloor)
    \end{equation*}
    Entonces:
    \begin{align*}
      \int_1^n	f(\lceil z \rceil) \, \mathrm{d} z
	&\le \int_1^n f(z) \, d z
	\le \int_1^n f( \lfloor z \rfloor ) \, \mathrm{d} z \\
      \sum_{1 \le k < n} f(k + 1)
	&\le \int_1^n f(z) \, \mathrm{d} z
	\le \sum_{1 \le k < n}	f(k) \\
      \sum_{2 \le k < n + 1} f(k)
	&\le \int_1^n f(z) \, \mathrm{d} z
	\le \sum_{1 \le k < n} f(k)
    \end{align*}
    Así,
    la diferencia:
    \begin{equation*}
      a_n
	= \sum_{1 \le k < n} f(k) - \int_1^n f(z) \, \mathrm{d} z
    \end{equation*}
    satisface \(0 \le f(n) \le a_n \le f(1)\).
    Además:
    \begin{equation*}
      a_{n + 1} - a_n
	= f(n + 1) - \int_n^{n + 1} f(z) \, \mathrm{d} z
	\le 0
    \end{equation*}
    O sea,
    la secuencia \(a_n\) es decreciente y acotada,
    y por tanto converge.
  \end{proof}
  Nótese que la demostración da cotas precisas:
  \(0 \le \gamma_f \le f(1)\).
  Consideremos la situación geométrica:
  La diferencia entre la línea continua y la escalera
  de la figura~\ref{fig:fz}
  es una serie de ``triángulos''
  de base \(1\) cuyas alturas suman \(f(1)\),
  con lo que su área total es aproximadamente \(f(1) / 2\),
  y esto suele no ser tan mala aproximación de \(\gamma_f\).

\section{Desarrollo de la fórmula}
\label{sec:desarrollo-euler-maclaurin}

  Con la intención de tomar límites \(b \rightarrow \infty\) luego,
  para calcular la suma entre \(1\) y \(a\)
  escribimos:
  \begin{align*}
    \sum_{1 \le k < b} f(k) - \int_1^b f(z) \, \mathrm{d} z
      &= \sum_{1 \le k < a} f(k) - \int_1^a f(z) \, \mathrm{d} z
	   + \sum_{a \le k < b} f(k) - \int_a^b f(z) \, \mathrm{d} z
  \end{align*}
  Nos falta aproximar los últimos términos.

  Tomemos el tramo entre un entero y el siguiente,
  para simplificar el rango entre \(0\) y \(1\).
  Nuestro resultado final
  se obtendrá sumando sobre los diferentes tramos.
  Integrando por partes:
  \begin{align*}
    \int_0^1 f(z) \, \mathrm{d} z
      &= z f(z) \, \biggr|_0^1
	  - \int_0^1 z f'(z) \, \mathrm{d} z \\
      &= z f(z) \, \biggr|_0^1
	   -  \frac{1}{2} \, z^2 f'(z) \, \biggr|_0^1
	   + \int_0^1 \frac{1}{2} \, z^2 f''(z) \, \mathrm{d} z \\
      &= z f(z) \, \biggr|_0^1
	   -  \frac{1}{2} \, z^2 f'(z) \, \biggr|_0^1
	   +  \frac{1}{2 \cdot 3} \, z^3 f''(z) \, \biggr|_0^1
	   - \int_0^1 \frac{1}{2 \cdot 3} \, z^3 f'''(z)
	       \, \mathrm{d} z
  \end{align*}
  Están apareciendo las derivadas sucesivas de \(f\)
  multiplicadas por polinomios.
  Si queremos polinomios mónicos,
  aparecerán divididos por factoriales.
  Llamemos \(B_n(z)\) al polinomio mónico de grado \(n\),
  partiendo con \(B_0(z) = 1\).
  Integrando por partes tenemos la relación básica:
  \begin{equation*}
    \int_0^1 B_n(z) f^{(n)}(z) \, \mathrm{d} z
      = \frac{B_{n + 1} (z)}{n + 1} \, f^{(n)} (z) \, \biggr|_0^1
	  - \int_0^1 \frac{B_{n + 1} (z)}{n + 1} \, f^{(n + 1)} (z)
	      \, \mathrm{d} z
  \end{equation*}
  De acá:%
    \index{Bernoulli, polinomios de!recurrencia|textbfhy}
  \begin{align}
    B_0(z)
      &= 1 \label{eq:B0} \\
    B'_{n + 1} (z)
      &= (n + 1) B_n(z)	 \quad n \ge 0
	    \label{eq:Bn}
  \end{align}
  Queda por definir la constante de integración en~\eqref{eq:Bn}.
  Tenemos primeramente:
  \begin{equation}
    \label{eq:Euler-Maclaurin-1}
    \int_0^1 f(z) \, \mathrm{d} z
      = \sum_{0 \le k \le n}
	   \frac{(-1)^k B_{k + 1}(z)}{(k + 1)!} \, f^{(k)}(z)
	     \, \biggr|_0^1
	 - \int_0^1 (-1)^n \frac{B_{n + 1}(z)}{(n + 1)!} \,
			     f^{(n + 1)}(z) \, \mathrm{d} z
  \end{equation}
  Interesa sumar la expresión~\eqref{eq:Euler-Maclaurin-1}
  para \([a, a + 1]\), \([a + 1, a + 2]\), \ldots, \([b - 1, b]\),
  conviene que se cumpla:
  \begin{equation}
    \index{Bernoulli, polinomios de|textbfhy}
    \label{eq:Bernoulli-polynomial-ends}
    B_n(0)
      = B_n(1)
  \end{equation}
  de forma que los términos intermedios se cancelen.
  Como \(B_0(z) = 1\),
  \(B_1(z)\) es una función lineal
  que solo si fuera constante cumpliría \(B_1(0) = B_1(1)\).
  La relación~\eqref{eq:Bernoulli-polynomial-ends}
  es válida siempre que \(n \ne 1\).
  Definimos en general:
  \begin{equation}
    \index{Bernoulli, numeros de@Bernoulli, números de|textbfhy}
    \label{eq:Bernoulli-number-definition}
    B_n
      = B_n(1)
  \end{equation}
  A los polinomios \(B_n(x)\)
  se les conoce como \emph{polinomios de Bernoulli},
  y las constantes \(B_n\) como \emph{números de Bernoulli},
  por razones
  que discutiremos en la sección~\ref{sec:suma-potencias}.
  Los números y polinomios de Bernoulli
  aparecen en una amplia gama de situaciones.
  Debe tenerse cuidado,
  hay autores que definen la secuencia
  (bajo el mismo nombre e incluso con la misma notación)
  de forma que todos los elementos son cero o positivos.

  En vista de la recurrencia~\eqref{eq:Bn},%
    \index{Bernoulli, polinomios de!recurrencia}
  si \(n \ge 2\) la condición~\eqref{eq:Bernoulli-polynomial-ends}
  puede expresarse también como:%
    \index{Bernoulli, polinomios de!integral}
  \begin{equation}
    \label{eq:Bernoulli-polynomial-integral}
    \int_0^1 B_n(z) \, \mathrm{d} z = 0
  \end{equation}
  Por el proceso que los produce,
  todos los coeficientes
  de los polinomios \(B_n(z)\) son racionales,
  por lo que también lo son las constantes \(B_n\).
  Los primeros polinomios y constantes
  registra el cuadro~\ref{tab:Bernoulli}.
  \begin{table}[htbp]
    \centering
    \begin{align*}
      \begin{array}{l@{\hspace*{2em}}l@{${} = {}$}c}
	\displaystyle
	  B_0(z) = 1 &
	  B_0 & \displaystyle \phantom{-}1 \\[1.5ex]
	\displaystyle
	  B_1(z) = z - \frac{1}{2} &
	  B_1 & \displaystyle -\frac{1}{2} \\[1.5ex]
	\displaystyle
	  B_2(z) = z^2 - z + \frac{1}{6} &
	  B_2 & \displaystyle \phantom{-}\frac{1}{6} \\[1.5ex]
	\displaystyle
	  B_3(z) = z^3 - \frac{3}{2} z^2 + \frac{1}{2} z &
	  B_3 & \displaystyle \phantom{-}0 \\[1.5ex]
	\displaystyle
	  B_4(z) = z^4 - 2 z^3 + z^2 - \frac{1}{30} &
	  B_4 & \displaystyle -\frac{1}{30} \\[1.5ex]
	\displaystyle
	  B_5(z) = z^5 - \frac{5}{2} z^4 + \frac{5}{3} z^3
		    - \frac{1}{6} z &
	  B_5 & \displaystyle \phantom{-}0 \\[1.5ex]
	\displaystyle
	  B_6(z) = z^6 - 3 z^5 + \frac{5}{2} z^4 - \frac{1}{2} z^2
		    + \frac{1}{42} &
	  B_6 & \displaystyle \phantom{-}\frac{1}{42} \\[1.5ex]
	\displaystyle
	  B_7(z) = z^7 - \frac{7}{2} z^6 + \frac{7}{2} z^5
		    - \frac{7}{6} z^3 + \frac{1}{6} z &
	  B_7 & \displaystyle \phantom{-}0 \\[1.5ex]
	\displaystyle
	  B_8(z) = z^8 - 4 z^7 + \frac{14}{3} z^6 - \frac{7}{3} z^4
		    + \frac{2}{3} z^2 - \frac{1}{30} &
	  B_8 & \displaystyle -\frac{1}{30} \\[1.5ex]
	\displaystyle
	  B_9(z) = z^9 - \frac{9}{2} z^8 + 6 z^7 - \frac{21}{5} z^5
		    + 2 z^3 -\frac{3}{10} z &
	  B_9 &	 \displaystyle \phantom{-}0 \\[1.5ex]
	\displaystyle
	  B_{10}(z) = z^{10} - 5 z^9 + \frac{15}{2} z^8
		    - 7 z^6 + 5 z^4 - \frac{3}{2} z^2
		    + \frac{5}{66} &
	  B_{10} &  \displaystyle \phantom{-} \frac{5}{66}
      \end{array}
    \end{align*}
    \caption[Polinomios y números de Bernoulli]
	    {Polinomios y números de Bernoulli~\cite{DLMF}}
    \label{tab:Bernoulli}
    \index{Bernoulli, polinomios de!cuadro}
    \index{Bernoulli, numeros de@Bernoulli, números de!cuadro}
  \end{table}
  Se observa que salvo \(B_1\)
  los valores de \(B_n\) para \(n\) impar
  son cero,
  y que los \(B_{2 n}\) alternan signo.
  Esto lo demostraremos en general más adelante.

  Para simplificar la derivación que sigue,
  definimos una extensión periódica del polinomio \(B_n(z)\):
  \begin{equation*}
    \widetilde{B}_n(z)
      = B_n(z - \lfloor z \rfloor)
  \end{equation*}
  La función \(\widetilde{B}_n(z)\) es continua
  dado que definimos \(B_n(0) = B_n(1) = B_n\)
  (salvo cuando \(n = 1\)).
  Así tenemos:
  \begin{align}
    \int_a^b f(z) \, \mathrm{d} z
      &= \sum_{1 \le k \le n}
	   \frac{(-1)^k \widetilde{B}_k(z)}{k!} \, f^{(k - 1)}(z)
	     \, \biggr|_a^b
	  + (-1)^{n + 1}
	      \int_a^b \frac{\widetilde{B}_{n + 1}(z)}{(n + 1)!} \,
				    f^{(n + 1)}(z)
		\, \mathrm{d} z \notag \\
      &= \frac{1}{2} f(a)
	   + \sum_{a < r < b} f(r)
	   + \frac{1}{2} f(b) \notag \\
      &\hspace{3.5em}
	    + \sum_{2 \le k \le n}
		\frac{(-1)^k B_k}{k!} \, f^{(k - 1)}(z)
		  \, \biggr|_a^b
	    + (-1)^{n + 1} \int_a^b
			     \frac{\widetilde{B}_{n + 1}(z)}
				  {(n + 1)!} \,
			     f^{(n + 1)}(z) \, \mathrm{d} z
	    \notag \\
      &= \sum_{a \le r < b} f(r)
	   + \sum_{1 \le k \le n}
	      \frac{(-1)^k B_k}{k!} \, f^{(k - 1)}(z)
		 \, \biggr|_a^b
	 + (-1)^{n + 1}
	     \int_a^b \frac{\widetilde{B}_{n + 1}(z)}{(n + 1)!} \,
				   f^{(n + 1)}(z)
	       \, \mathrm{d} z
	   \label{eq:Euler-Maclaurin-2}
  \end{align}
  Acá aprovechamos que \(B_1 = - 1 / 2\),
  absorbimos el término \(f(a)\)
  en la primera sumatoria y reorganizamos.

  Dividiendo el rango de la suma en~\eqref{eq:Euler-Maclaurin-2}
  \begin{multline}
    \label{eq:Euler-Maclaurin-3}
    \sum_{1 \le k < b} f(k) - \int_1^b f(z) \, \mathrm{d} z
      = \sum_{1 \le k < a} f(k) - \int_1^a f(z) \, \mathrm{d} z \\
	   - \sum_{1 \le k \le n}
	       \frac{(-1)^k B_k}{k!} \, f^{(k - 1)}(z)
		  \, \biggr|_a^b
	   + (-1)^n \int_a^b \frac{\widetilde{B}_{n + 1}(z)}
				  {(n + 1)!} \,
				     f^{(n + 1)}(z)
		      \, \mathrm{d} z
  \end{multline}
  Haciendo ahora \(b \rightarrow \infty\),
  y reorganizando~\eqref{eq:Euler-Maclaurin-3}
  bajo el entendido que:
  \begin{equation*}
    \lim_{z \rightarrow \infty} f^{(n)} (z) = 0
  \end{equation*}
  Recordando que salvo \(B_1\) todos los \(B_{2 k + 1} = 0\),
  obtenemos la fórmula de Euler-Maclaurin:
  \begin{equation}
    \index{Euler-Maclaurin, formula de@Euler-Maclaurin, fórmula de|textbfhy}
    \label{eq:Euler-Maclaurin}
    \sum_{1 \le k < a} f(k)
      = \int_1^a f(z) \, \mathrm{d} z
	  + \gamma_f
	  + B_1 f(a)
	  + \sum_{1 \le k \le n}
	       \frac{B_{2 k}}{(2 k)!} f^{(2 k - 1)}(a)
	  + R_n(f; a)
  \end{equation}
  En esto hemos escrito:
  \begin{align}
    \gamma_f
      &= \lim_{b \rightarrow \infty}
	   \left(
	     \sum_{1 \le k \le b} f(k) - \int_1^b f(z)
					   \, \mathrm{d} z
	   \right)
	      \label{eq:gamma-f} \\
    R_n(f; a)
      &= \int_a^\infty
	   \frac{\widetilde{B}_{2 n + 1}(z)}{(2 n + 1)!} \,
	     f^{(2 n + 1)}(z) \, \mathrm{d} z
	      \label{eq:euler-maclaurin-residue}
  \end{align}
  Resta encontrar mejores maneras
  de determinar los polinomios \(B_n(z)\),
  los coeficientes \(B_n = B_n(0)\),
  y finalmente acotar el resto \(R_n(f; a)\).
  Esto lo haremos en la sección~\ref{sec:resto-Euler-Maclaurin}.
  Lamentablemente,
  los \(B_n\) crecen muy rápidamente
  y~\eqref{eq:Euler-Maclaurin} rara vez converge,
  por lo que la constante \(\gamma_f\)
  debe determinarse de alguna otra forma.
  Las cotas que daremos indican que el error cometido
  es a lo más el último término incluido,
  la fórmula igual es útil para obtener valores numéricos precisos.

\section{Suma de potencias}
\label{sec:suma-potencias}

  Una aplicación obvia de la fórmula de Euler-Maclaurin%
    \index{Euler-Maclaurin, formula de@Euler-Maclaurin, fórmula de}
  es calcular las sumas:
  \begin{equation*}
    S_m(n)
      = \sum_{1 \le k \le n - 1} k^m
  \end{equation*}
  Acá tenemos:
  \begin{align*}
    f(z)
      &= z^m \\
    f^{(k)}(z)
      &= m^{\underline{k}} z^{m - k}
  \end{align*}
  La fórmula de Euler-Maclaurin sumando hasta el término \(m - 1\)
  (el resto es cero en este caso;
   en realidad estamos aplicando~\eqref{eq:Euler-Maclaurin-2},
   no hay constante \(\gamma\) porque es parte del resto)
  y tomando la suma desde \(0\) para simplificar
  da:
  \begin{align*}
    S_m(n)
      &= \int_0^n z^m \, \mathrm{d} z
	   + \sum_{0 \le k \le m - 1}
	       \frac{B_{k + 1}}{(k + 1)!} \,
		 m^{\underline{k}} \, n^{m - k} \\
      &= \frac{n^{m + 1}}{(m + 1)}
	   + \sum_{0 \le k \le m - 1}
	       B_{k + 1} \,
		  \frac{m^{\underline{k}}}{(k + 1)!} \,
		  n^{m - k} \\
      &= \frac{1}{m + 1}
	   \left(
	     n^{m + 1}
	       + \sum_{0 \le k \le m - 1} \binom{m + 1}{k + 1} \,
		   B_{k + 1} \, n^{m - k}
	   \right)
  \end{align*}
  Pero como \(B_0 = 1\),
  podemos incorporar el primer término a la suma,
  y luego de ajustar índices queda:%
    \index{suma!potencias|textbfhy}
  \begin{equation}
    \label{eq:Bernoulli-Smn}
    S_m(n)
      = \frac{1}{m + 1} \,
	  \left(
	    \sum_{0 \le k \le m}
	      \binom{m + 1}{k} \, B_k n^{m + 1 - k}
	  \right)
  \end{equation}
  Jakob Bernoulli%
    \index{Bernoulli, Jakob}
  había notado esta expansión,
  e incluso la usó para calcular \(S_{10}(1\,000)\).
  Es en honor a su descubrimiento
  que llevan su nombre estos números.

  La fórmula~\eqref{eq:Bernoulli-Smn}
  a veces se atribuye erróneamente a Faulhaber,%
    \index{Faulhaber, Johann}
  quien desarrolló fórmulas eficientes
  para expresar \(S_{2 m + 1}(n)\)
  en términos de \(n (n + 1)\).
  Una discusión detallada de sus resultados,
  reconstrucción de sus posibles métodos
  y una variedad de extensiones presenta Knuth~%
    \cite{knuth93:_johann_faulhaber_sums_powers}.

\section{Números harmónicos}
\label{sec:em-harmonicos}
\index{numeros harmonicos@números harmónicos}

  Usemos ahora nuestro nuevo juguete
  para aproximar los números harmónicos.
  Tenemos primeramente:
  \begin{equation*}
    \mathrm{D}^k z^{-1}
       = (-1)^{\underline{k}} \, z^{- k - 1}
       = (-1)^k k! z^{- k - 1}
  \end{equation*}
  Las derivadas tienden a cero cuando \(z \rightarrow \infty\),
  así que vamos bien.
  La fórmula de Euler-Maclaurin da:%
    \index{Euler-Maclaurin, formula de@Euler-Maclaurin, fórmula de}%
    \index{numeros harmonicos@números harmónicos!aproximacion@aproximación}
  \begin{align}
    H_n
      &= \frac{1}{n} + \sum_{1 \le r < n} \frac{1}{r} \notag \\
      &= \frac{1}{n} + \int_1^n \frac{\mathrm{d} z}{z}
	   + \gamma
	   + B_1 \cdot (-1) 1! n^{-1}
	   + \sum_{1 \le k \le s}
	       \frac{B_{2 k}}{(2 k)!}
		  \cdot (2 k - 1)! n^{- 2 k - 2}
	   + R_s(n) \notag \\
      &= \ln n + \frac{1}{n}
	   + \gamma
	   - \frac{1}{2 n}
	   + \sum_{1 \le k \le s}
	       \frac{B_{2 k}}{2 k}
		  \cdot n^{- 2 k - 2}
	   + R_s(n)
	      \label{eq:Hn-asy-Bn} \\
      &= \ln n + \gamma + \frac{1}{2 n}
	   - \frac{1}{12 n^2}
	   + \frac{1}{120 n^4}
	   - \frac{1}{252 n^6}
	   + O(n^{-8})
	      \label{eq:Hn-asy-numbers}
  \end{align}
  Por el teorema de Maclaurin-Cauchy,%
    \index{Maclaurin-Cauchy, teorema de}
  existe la constante:
  \begin{align*}
    \index{Euler, constante de|textbfhy}
    \index{\(\gamma\) (constante de Euler)|see{Euler, constante de}}
    \gamma
      &= \lim_{n \rightarrow \infty}
	   \left(
	     \sum_{1 \le k < n}
	       \frac{1}{k} - \int_1^n \frac{\mathrm{d} z}{z}
	   \right) \\
      &= \lim_{n \rightarrow \infty}
	   \left( H_n - \ln n \right) \\
      &\approx 0,5772156649
  \end{align*}
  La aproximación simple obtenida
  luego del teorema de Maclaurin-Cauchy
  da \(\gamma \approx 1 / 2\).
  Nada mal.

  Euler%
    \index{Euler, Leonhard}
  en 1736 obtuvo el valor de \(\gamma\) con \(16\) dígitos
  usando \(8\) términos de la expansión:
  \begin{equation*}
    \gamma
      = H_{10} - \ln 10 - \frac{1}{20} + \frac{1}{1200} - \dotsb
  \end{equation*}
  Para calcular el número de términos requeridos
  para una precisión similar directamente
  podemos usar la aproximación que derivamos.
  Para la precisión que obtuvo Euler
  requeriríamos \(n\) tal que:
  \begin{equation*}
    \left\lvert \gamma - (H_n - \ln n) \right\rvert
      \approx \frac{1}{2 n}
      < 5 \cdot 10^{-17} \\
  \end{equation*}
  Resulta \(10^{16}\) términos.

  Al número \(\gamma\)
  se le conoce como \emph{constante de Euler-Mascheroni}%
    \index{Euler-Mascheroni, constante de|see{Euler, constante de}}
  o simplemente como \emph{constante de Euler}.
  Gourdon y Sebah~%
    \cite{gourdon03:_euler_const}
  incluso lo consideran
  el tercer número más importante de la matemática,
  después de \(\pi\) y \(e\).
  Determinar si \(\gamma\) es racional,%
    \index{numero@número!irracional!\(\gamma\)}
  algebraico o trascendente
  es un problema abierto famoso.

\section{Fórmula de Stirling}
\label{sec:em-Stirling}

  Veamos cómo podemos aproximar factoriales
  con esta herramienta.
  Primero tenemos:
  \begin{equation*}
    \ln n!
      = \sum_{1 \le k < n} \ln k + \ln n
  \end{equation*}
  El teorema de Maclaurin-Cauchy no sirve si
  (como acá)
  tenemos una función creciente.
  Pero en caso que la función \(f(z)\) sea monótona creciente
  es claro que:
  \begin{align*}
    \int_1^n \lfloor f(z) \rfloor \, \mathrm{d} z
       \le \int _1^n f(z) \, \mathrm{d} z
      &\le \int_1^n \lceil f(z) \rceil \, \mathrm{d} z \\
    \sum_{1 \le k < n} f(k)
       \le \int _1^n f(z) \, \mathrm{d} z
      &\le \sum_{1 \le k < n} f(k + 1) \\
      &= \sum_{1 \le k < n} f(k) + f(n) - f(1)
  \end{align*}
  Promediando ambas cotas
  (la diferencia entre la curva y las escaleras son casi triángulos)
  queda:
  \begin{equation*}
    \sum_{1 \le k < n} f(k)
      \approx \int_1^n f(z) \, \mathrm{d} z
		- \frac{1}{2} \, (f(n) - f(1))
  \end{equation*}
  Hasta acá podemos decir que,
  burdamente:
  \begin{align*}
    \ln n!
      &= \sum_{1 \le k < n} \ln n + \ln n \\
      &\approx \int_1^n \ln z \, \mathrm{d} z
	  - \frac{1}{2} \, (\ln n - \ln 1)
	  + \ln z \\
      &= n \ln n - n + 1 + \frac{1}{2} \, \ln n \\
    n!
      &\approx e \sqrt{n} \, \left(\frac{n}{e}\right)^n
  \end{align*}
  Esto nos hace albergar la esperanza de obtener algo útil.

  Para una mejor aproximación usamos la fórmula de Euler-Maclaurin:
  \begin{align*}
    \int_1^n \ln z \, \mathrm{d} z
      &= n \ln n - n + 1 \\
    \mathrm{D}^k \ln z
      &= (-1)^{k - 1} (k - 1)! z^{-k}
  \end{align*}
  Si suponemos que existe la constante \(\ln \sigma\)
  (\emph{constante de Stirling}):%
    \index{Stirling, constante de}
  \begin{align}
    \ln n!
      &= \ln n + \sum_{1 \le k < n} \ln k \\
      &= \ln n + \int_1^n \ln z \, \mathrm{d} z + \ln \sigma
	   + B_1 \ln n \notag \\
      &\qquad
	   + \sum_{1 \le k \le s}
	       \frac{B_{2 k}}{(2 k)!}
		 \cdot (-1)^{2 k - 2} (2 k - 2)!
		 \cdot n^{- 2 k - 1}
	   + R_s(n) \notag \\
      &= \ln \sigma + (n + 1) \ln n - n
	   - \frac{1}{2} \ln n
	   - \sum_{1 \le k \le s}
	       \frac{B_{2 k}}{(2 k - 1) 2 k} \, n^{- 2 k - 1}
	   + R_s(n) \notag \\
      &= \ln \sigma + n \ln n + \frac{1}{2} \, \ln n - n
	   + \frac{1}{12 n}
	   - \frac{1}{360 n^3}
	   + \frac{1}{1260 n^5}
	   + R_3(n)
	      \label{eq:Stirling-1}
  \end{align}
  La constante \(\ln \sigma\)
  en~\eqref{eq:Stirling-1}
  queda determinada por el siguiente límite,
  si existe:
  \begin{align*}
    \ln \sigma
      &= \lim_{n \rightarrow \infty}
	   \left(
	     \sum_{1 \le k \le n}
	       \ln k - \int_1^n \ln z \, \mathrm{d} z
	   \right) \\
      &= \lim_{n \rightarrow \infty}
	   \left(
	     \ln n! - n \ln n - n
	   \right)
  \end{align*}
  Veremos más adelante que \(\sigma = \sqrt{2 \pi}\),
  usando este valor en~\eqref{eq:Stirling-1},
  y expandiendo la exponencial:
  \begin{equation}
    \label{eq:Stirling-asymptotic}
    n!
      = \sqrt{2 \pi n} \, \left( \frac{n}{e} \right)^n
	  \cdot \left(1
			+ \frac{1}{12 n}
			+ \frac{1}{288 n^2}
			- \frac{139}{51840 n^3}
			- \frac{571}{2488320 n^4}
			+ O(n^{-5})
		\right)
  \end{equation}
  Truncado en el primer término,
  queda:
  \begin{equation}
    \index{Stirling, formula de@Stirling, fórmula de|textbfhy}
    \label{eq:Stirling}
    n!
     \approx \sqrt{2 \pi n} \, \left( \frac{n}{e} \right)^n
  \end{equation}
  La ecuación~\eqref{eq:Stirling}
  es la fórmula de Stirling para el factorial.
  La aproximación dada antes dice que
  ya para \(n \ge 8\) el error es de cerca de \(1\)\%,
  cosa que cálculo directo confirma.

  Nótese que \(\sigma = \sqrt{2 \pi} = 2,5066\),
  nuestra burda aproximación \(\sigma \approx e\) no era tan mala.

  Falta el valor de \(\ln \sigma\).
  Por el producto de Wallis%
    \index{Wallis, producto de}
  (lo demostraremos más adelante):
  \begin{align*}
    \frac{\pi}{2}
      &= \prod_{k \ge 1}
	   \frac{2 k}{2 k - 1} \cdot \frac{2 k}{2 k + 1} \\
      &= \lim_{n \rightarrow \infty}
	   \prod_{1 \le k \le n}
	     \left(
	       \frac{2 k}{2 k - 1} \cdot \frac{2 k}{2 k}
	     \right) \cdot
	     \left(
	       \frac{2 k}{2 k} \cdot \frac{2 k}{2 k + 1}
	     \right) \\
      &= \lim_{n \rightarrow \infty}
	   \frac{(2 n!)^4}{(2 n)! (2 n + 1)!} \\
      &= \lim_{n \rightarrow \infty}
	   \frac{1}{2 n + 1} \cdot
	     \frac{2^{4 n} \, n!^4}{(2 n)!^2}
  \end{align*}
  Substituyendo la aproximación~\eqref{eq:Stirling-1}
  para los factoriales:
  \begin{align*}
    \frac{\pi}{2}
      &= \lim_{n \rightarrow \infty}
	   \frac{1}{2 n + 1} \cdot
	     \frac{2^{4 n} \sigma^4 n^2 \left( n / e \right)^{4 n}}
		  {\sigma^2 (2 n) \left( 2 n / e \right)^{4 n}} \\
    \sigma^2
      &= 2 \pi
  \end{align*}

  Para completar,
  demostraremos el producto de Wallis,
  siguiendo a Lynn~%
    \cite{lynn:_wallis_product}.
  \begin{theorem}[Producto de Wallis]
    \index{Wallis, producto de|textbfhy}
    \label{theo:producto-Wallis}
    Tenemos:
    \begin{equation*}
      \frac{\pi}{2}
	= \prod_{k \ge 1}
	    \frac{2 k}{2 k - 1} \cdot \frac{2 k}{2 k + 1}
    \end{equation*}
  \end{theorem}
  \begin{proof}
    Definamos:
    \begin{equation*}
      a_n
	= \int_0^{\pi / 2} \sin^n z \, \mathrm{d} z
    \end{equation*}
    Como \(0 \le \sin z \le 1\) en el rango \(0 \le z \le \pi / 2\),
      \(\left\langle a_n \right\rangle_{n \ge 0}\)
    es una secuencia positiva y monótona decreciente.
    Integrando por partes:
    \begin{align*}
      a_n
	&= -\sin^{n - 1} z \, \cos z \,
	   \biggr|_0^{\pi / 2}
	   + \int_0^{\pi / 2}
	       (n - 1) \sin^{n - 2} z \, \cos^2 z \, \mathrm{d} z \\
	&= \int_0^{\pi / 2}
	     (n - 1) \sin^{n - 2} z \, (1 - \sin^2 z)
	     \, \mathrm{d} z
    \end{align*}
    O sea \(a_n = (n - 1) a_{n - 2} - (n - 1) a_n\),
    que resulta en:
    \begin{equation}
      \label{eq:Wallis-recurrence}
      a_n
	= \frac{n - 1}{n} \, a_{n - 2}
    \end{equation}
    Por el otro lado,
    directamente tenemos \(a_0 = \pi / 2\) y \(a_1 = 1\).
    Con estos puntos de partida en~\eqref{eq:Wallis-recurrence}:
    \begin{align}
      a_{2 n}
	&= \frac{\pi}{2} \cdot \frac{1}{2} \cdot \dotsm
	     \cdot \frac{2 n - 1}{2 n} \label{eq:Wallis-an-even} \\
      a_{2 n + 1}
	&= \frac{2}{3} \cdot \frac{4}{5} \cdot \dotsm
	     \cdot \frac{2 n}{2 n + 1} \label{eq:Wallis-an-odd}
    \end{align}
    Como la secuencia \(a_n\) es decreciente,
    \(a_{2 n + 1} \le a_{2 n} \le a_{2 n - 1}\),
    y de la recurrencia~\eqref{eq:Wallis-recurrence}:
    \begin{equation*}
      1 \le \frac{a_{2 n}}{a_{2 n + 1}}
	\le \frac{a_{2 n - 1}}{a_{2 n + 1}}
	= 1 + \frac{1}{2 n}
    \end{equation*}
    En consecuencia:
    \begin{equation}
      \label{eq:Wallis-limit}
      \lim_{n \rightarrow \infty} \frac{a_{2 n}}{a_{2 n + 1}}
	= 1
    \end{equation}
    y por~\eqref{eq:Wallis-limit}
    con~\eqref{eq:Wallis-an-even} y~\eqref{eq:Wallis-an-odd}:
    \begin{equation*}
      \lim_{n \rightarrow \infty}
	\frac{\pi}{2} \cdot
	  \frac{1 \cdot 3 \dotsm (2 n - 1)}
	       {2 \cdot 4 \dotsm (2 n)} \cdot
	  \frac{3 \cdot 5 \dotsm (2 n + 1)}
	       {2 \cdot 4 \dotsm (2 n)}
	= 1
    \end{equation*}
    que es equivalente a lo planteado.
  \end{proof}
  Una muy bonita demostración alternativa
  (originalmente de Euler,%
     \index{Euler, Leonhard}
   quien de forma similar
   obtuvo muchos otros resultados sorprendentes)
  es la siguiente.
  Tiene el problema de basarse en la fórmula de Euler para el seno,
  que es muy sugestiva pero no es sencilla de demostrar.
  \begin{proof}
    La fórmula de Euler para el seno
    resulta de considerar esta función impar
    como un ``polinomio infinito'' con ceros \(0\) y \(\pm n \pi\),
    como también:
    \begin{equation*}
      \lim_{z \rightarrow 0} \frac{\sin z}{z}
	= 1
    \end{equation*}
    el coeficiente de \(z\) debe ser \(1\),
    por lo que puede expresarse:%
      \index{Euler, formula para seno@Euler, fórmula para seno}
    \begin{align*}
      \frac{\sin z}{z}
	&= \left( 1 - \frac{z^2}{\pi^2} \right)
	     \left( 1 - \frac{z^2}{4 \pi^2} \right)
	     \left( 1 - \frac{z^2}{9 \pi^2} \right) \dotsm \\
	&= \prod_{k \ge 1}
	     \left( 1 - \frac{z^2}{k^2 \pi^2} \right)
    \end{align*}
    Notamos que para \(z = \pi / 2\):
    \begin{align*}
      \frac{2}{\pi}
	&= \prod_{k \ge 1} \left( 1 - \frac{1}{4 k^2} \right) \\
      \frac{\pi}{2}
	&= \prod_{k \ge 1} \left( \frac{4 k^2}{4 k^2 - 1} \right) \\
	&= \prod_{k \ge 1} \frac{(2 k) (2 k)}{(2 k - 1) (2 k + 1)}
      \qedhere
    \end{align*}
  \end{proof}

\section{Propiedades de los polinomios
       y números de Bernoulli}
\label{sec:propiedades-Bernoulli}

  Derivaremos algunas propiedades adicionales
  de los polinomios y números de Bernoulli.%
    \index{Bernoulli, polinomios de}%
    \index{Bernoulli, numeros de@Bernoulli, números de}
  Algunas ya las usamos,
  otras las necesitaremos más adelante.
  En el proceso mostraremos algunas técnicas útiles
  para obtener información sobre los coeficientes de una serie.%
    \index{serie de potencias!coeficientes}

  De la recurrencia~\eqref{eq:Bn}:
  \begin{equation*}
    B'_n(z)
     = n B_{n - 1}(z)
  \end{equation*}
  Con esto,
  à la Maclaurin en \(y\):
  \begin{align}
    B_n(z + y)
      &= B_n(z)
	  + n B_{n - 1}(z) \, y
	  + \frac{n (n - 1)}{2} B_{n - 2}(z) \, y^2
	  + \dotsb \notag \\
      &= \sum_{0 \le k \le n} \binom{n}{k} B_{n - k}(z) \, y^k
	    \label{eq:Bn(z+y)}
  \end{align}
  Si ahora hacemos \(z = 0\) en~\eqref{eq:Bn(z+y)},
  recordamos \(B_n(0) = B_n\)
  y cambiamos variables \(y \leadsto z\) en el resultado:
  \begin{equation}
    \label{eq:Bn(z)-expansion}
    B_n(z)
      = \sum_{0 \le k \le n} \binom{n}{k} B_{n - k} \, z^k
  \end{equation}
  Para \(z = 1\),
  como \(B_n(1) = B_n\)
  salvo para \(n = 1\),
  resulta:
  \begin{equation}
    \label{eq:Bn-expansion}
    B_n
      = \sum_{0 \le k \le n} \binom{n}{k} B_k
  \end{equation}
  Si en~\eqref{eq:Bn-expansion}
  interpretamos \(\mathbf{B}^n\) como \(B_n\)
  tenemos la linda fórmula:%
    \index{Bernoulli, numeros de@Bernoulli, números de!linda formula@linda fórmula}
  \begin{equation}
    \label{eq:B-linda-formula}
    \mathbf{B}^n
      = (1 + \mathbf{B})^n
  \end{equation}
  En la linda fórmula~\eqref{eq:B-linda-formula}
  para \(B_{n + 1}\) se cancelan los \(B_{n + 1}\),
  y puede despejarse \(B_n\)
  dando la relación válida para \(n \ge 1\):
  \begin{equation*}
    B_n
      = - \frac{1}{n + 1}
	    \sum_{0 \le k \le n - 1} \binom{n + 1}{k} \, B_k
  \end{equation*}

  Por los factoriales definamos una función generatriz exponencial
  para los polinomios \(B_n(x)\)
  (estamos trabajando con series sobre el anillo \(\mathbb{Q}[x]\)):
  \begin{equation}
    \label{eq:definicion-B(x,z)}
    B(x, z)
      = \sum_{n \ge 0} B_n(x) \, \frac{z^n}{n!}
  \end{equation}
  De la recurrencia~\eqref{eq:Bn} para los polinomios:
  \begin{align}
    \frac{\partial B(x, z)}{\partial x}
      &= \sum_{n \ge 1} B'_n(x) \, \frac{z^n}{n!} \notag \\
      &= z \sum_{n \ge 1}
	     B_{n - 1}(x) \, \frac{z^{n - 1}}{(n - 1)!} \notag \\
      &= z B(x, z) \label{eq:B(x,z)}
  \end{align}
  La ecuación~\eqref{eq:B(x,z)} indica
  que para alguna función \(c(z)\) que no depende de \(x\):
  \begin{equation}
    \label{eq:B(x,z)-2}
    B(x, z)
      = c(z) \mathrm{e}^{x z}
  \end{equation}
  Usamos ahora la otra condición sobre los polinomios.
  Debe cumplirse:
  \begin{align*}
    \int_0^1 B(x, z) \, \mathrm{d} x
      &= \sum_{n \ge 0}
	   \frac{z^n}{n!} \, \int_0^1 B_n(x) \, \mathrm{d} x \\
      &= 1
  \end{align*}
  De nuestra expresión~\eqref{eq:B(x,z)-2} para \(B(x, z)\):
  \begin{align*}
    \int_0^1 B(x, z) \, \mathrm{d} x
      &= c(z) \int_0^1 \mathrm{e}^{x z} \, \mathrm{d} x \\
      &= c(z) \frac{1}{z} \, \left( \mathrm{e}^z - 1 \right)
  \end{align*}
  Comparando ambas expresiones para la integral obtenemos \(c(z)\),
  y finalmente:
  \begin{equation}
    \index{Bernoulli, polinomios de!generatriz}
    \label{eq:B(x,z)-3}
    B(x, z)
      = \sum_{n \ge 0} B_n(x) \, \frac{z^n}{n!}
      = \frac{z \mathrm{e}^{x z}}{\mathrm{e}^z - 1}
  \end{equation}
  Para justificar algunas de las manipulaciones que siguen,
  debemos asegurarnos
  que esta serie converge uniformemente para \(0 \le z \le 1\)%
    \index{convergencia uniforme}
  (no hay problemas con \(x\)).
  La función~\eqref{eq:B(x,z)-3} en \(z = 0\)
  tiene una singularidad removible,%
    \index{singularidad!removible}
  y tiene polos en \(z = \pm 2 n \pi \mathrm{i}\) para \(n \ge 1\),%
    \index{polo}
  por lo que el radio de convergencia es \(2 \pi > 1\).%
    \index{serie de potencias!radio de convergencia}
  Para detalles de estos conceptos
  véase el capítulo~\ref{cha:analisis-complejo}.%
    \index{analisis complejo@análisis complejo}

  Con \(x = 0\) obtenemos la función generatriz
  de los coeficientes \(B_n = B_n(0)\):%
    \index{Bernoulli, numeros de@Bernoulli, números de!generatriz}
  \begin{equation}
    \label{eq:B(0,z)}
    B(0, z)
      = \sum_{n \ge 0} B_n \, \frac{z^n}{n!}
      = \frac{z}{\mathrm{e}^z - 1}
  \end{equation}

  De los valores dados antes
  pareciera ser que los valores para índices impares son todos cero,
  salvo \mbox{\(B_1 = - 1 / 2\)}.
  Consideremos la función:
  \begin{align*}
    \frac{z}{\mathrm{e}^z - 1} + \frac{z}{2}
      &= \frac{z}{2}
	   \cdot \frac{\mathrm{e}^z + 1}{\mathrm{e}^z - 1} \\
      &= \frac{z}{2} \, \coth \frac{z}{2}
  \end{align*}
  Esta función es par,
  confirmando nuestra sospecha.

  Anotamos el resultado siguiente
  en términos de la función \(\zeta\) de Riemann:%
    \index{Riemann, funcion \(\zeta\) de@Riemann, función \(\zeta\) de|textbfhy}%
    \index{\(\zeta\) de Riemann|see{Riemann, función \(\zeta\) de}}
  \begin{equation}
    \label{eq:zeta}
    \zeta(z)
      = \sum_{n \ge 1} n^{-z}
  \end{equation}
  Uno de los resultados más sensacionales de Euler
  fue la solución en 1734 del problema de Basilea,%
    \index{Basilea, problema de|textbfhy}%
    \index{Euler, Leonhard}
  que venía siendo un tema recurrente desde 1650.
  Se buscaba el valor de la serie:
  \begin{equation}
    \label{eq:zeta(z)}
    \zeta(2)
      = \sum_{n \ge 1} \frac{1}{n^2}
  \end{equation}
  Hizo mucho más que esto,
  hallando los valores de \(\zeta(2 k)\) para \(k\) hasta \(13\).
  Luego halló la fórmula general,
  para la que hay hermosas demostraciones
  (ver Aigner y Ziegler~%
    \cite{aigner14:_proof_the_book}
   o Kalman~%
    \cite{kalman93:_six_ways_sum_series}).
  Podemos seguir uno de los razonamientos de Euler
  (Dunham~\cite{dunham09:_when_euler_met_lhopital}
   da otras de las demostraciones
   y algo del sabor del trabajo original)
  como sigue.

  Por la fórmula de Euler para la exponencial de un complejo:%
    \index{Euler, formula de (exponencial complejo)@Euler, fórmula de (exponencial complejo)}
  \begin{align}
    \cot z
      &= \frac{\cos z}{\sin z} \notag \\
      &= \mathrm{i} \, \frac{\mathrm{e}^{\mathrm{i} z}
			     + \mathrm{e}^{- \mathrm{i} z}}
		{\mathrm{e}^{\mathrm{i} z}
		  - \mathrm{e}^{- \mathrm{i} z}}
	     \notag \\
      &= \mathrm{i} + \frac{2 \mathrm{i}}
			   {\mathrm{e}^{2 \mathrm{i} z} - 1}
	     \notag \\
    \frac{z}{2} \, \cot \frac{z}{2}
      &= \frac{\mathrm{i} z}{2}
	   + \frac{\mathrm{i} z}{\mathrm{e}^{\mathrm{i} z} - 1}
	     \notag \\
    \intertext{Manejado los primeros dos términos
	       en forma especial queda:}
    \frac{z}{2} \, \cot \frac{z}{2}
      &= 1 + \sum_{k \ge 2} B_k \frac{(\mathrm{i} z)^k}{k!}
	   \label{eq:zcotz-1}
  \end{align}
  Por el otro lado tenemos la fórmula de Euler para el seno:%
    \index{Euler, formula para seno@Euler, fórmula para seno}
  \begin{equation}
    \label{eq:Euler-seno}
    \sin z
      = z \prod_{n \ge 1} \left( 1 - \frac{z^2}{n^2 \pi^2} \right)
  \end{equation}
  Aplicando \(z \mathrm{D} \log\) a~\eqref{eq:Euler-seno}:
  \begin{equation*}
    z \, \frac{\mathrm{d}}{\mathrm{d} z} \, \ln \sin z
      = z \, \frac{\cos z}{\sin z}
      = z \cot z
  \end{equation*}
  Con esto tenemos:
  \begin{align}
    z \cot z
      &= 1 - \sum_{n \ge 1} \frac{2 z^2}{1 - (z^2 / n^2 \pi^2)}
	  \notag \\
      &= 1 - 2 \sum_{n \ge 1}
		 \sum_{k \ge 0} \frac{z^{2 k + 2}}
				     {n^{2 k} \pi^{2 k}} \notag \\
      &= 1 - 2 \sum_{k \ge 0}
		 \frac{z^{2 k + 2}}{\pi^{2 k}}
		    \cdot \sum_{n \ge 1} \frac{1}{n^{2 k}} \notag \\
      &= 1 - 2 \sum_{k \ge 0} \frac{z^{2 k + 2} \zeta(2 k)}
				   {\pi^{2 k}}
						    \notag \\
    \frac{z}{2} \cot \frac{z}{2}
      &= 1 - 2 \sum_{k \ge 0}
		 \frac{z^{2 k + 2} \zeta(2 k)}
		      {2^{2 k + 2} \pi^{2 k}}
		    \label{eq:zcotz-2}
  \end{align}
  Comparando coeficientes de \(z^{2 k}\)
  entre~\eqref{eq:zcotz-1} y~\eqref{eq:zcotz-2} resulta:
  \begin{equation*}
    \zeta(2 k)
      = \frac{(-1)^{k + 1} 4^k \pi^{2 k} B_{2 k}}{2 (2 k)!}
  \end{equation*}
  Incidentalmente,
  esto demuestra que los números \(B_{2 k}\) alternan signo,
  ya que \(\zeta(2 k)\) claramente es positivo.
  Como \(\zeta(2 k) \sim 1\),
  usando la fórmula de Stirling~\eqref{eq:Stirling}:%
    \index{Stirling, formula de@Stirling, fórmula de}%
    \index{Bernoulli, numeros de@Bernoulli, números de!asintotica@asintótica|textbfhy}
  \begin{align}
    B_{2 k}
      &\sim (-1)^{k + 1} \frac{2 (2 k)!}{4^k \pi^{2 k}} \notag \\
      &\sim (-1)^{k + 1} \, 4 \sqrt{k \pi}
	      \left( \frac{k}{\pi e} \right)^{2 k}
		    \label{eq:Bernoulli-approximation}
  \end{align}
  Con este crecimiento de \(B_{2 k}\)
  las derivadas de \(f\) deben disminuir muy rápidamente
  para que la fórmula de Euler-Maclaurin converja.%
    \index{Euler-Maclaurin, formula de@Euler-Maclaurin, fórmula de!convergencia}

  La figura~\ref{fig:Bernoulli}
  grafica algunos polinomios de Bernoulli
  en el rango que nos interesa.
  \begin{figure}[htbp]
    \newlength{\ten}
    \settowidth{\ten}{\(10\)}
    \begin{tabular}{>{\raggedleft}m{\ten}@{\hspace{0.5ex}}
		     *{4}{@{}m{0.232\linewidth}}}
	     & \multicolumn{1}{c}{\(m\)}
	     & \multicolumn{1}{c}{\(m + 1\)}
	     & \multicolumn{1}{c}{\(m + 2\)}
	     & \multicolumn{1}{c}{\(m + 3\)} \\
       \(2\) & \pgfimage[width=\linewidth]{images/B2}
	     & \pgfimage[width=\linewidth]{images/B3}
	     & \pgfimage[width=\linewidth]{images/B4}
	     & \pgfimage[width=\linewidth]{images/B5} \\
       \(6\) & \pgfimage[width=\linewidth]{images/B6}
	     & \pgfimage[width=\linewidth]{images/B7}
	     & \pgfimage[width=\linewidth]{images/B8}
	     & \pgfimage[width=\linewidth]{images/B9} \\
      \(10\) & \pgfimage[width=\linewidth]{images/B10}
	     & \pgfimage[width=\linewidth]{images/B11}
	     & \pgfimage[width=\linewidth]{images/B12}
	     & \pgfimage[width=\linewidth]{images/B13}
    \end{tabular}
    \caption{Polinomios de Bernoulli en $[0, 1]$
	     (escalados de mínimo a máximo)}
    \label{fig:Bernoulli}%
    \index{Bernoulli, polinomios de!grafica@gráfica}
  \end{figure}
  Pareciera ser que \(B_{2 k}(z)\)
  es simétrica alrededor de \(1 / 2\),
  mientras \(B_{2 k + 1}(z)\) es antisimétrica.
  Para demostrar estos hechos
  consideramos:
  \begin{align*}
    B \left( 1 / 2 + u, z \right)
	+ B \left( 1 / 2 - u, z \right)
      &= \frac{z \mathrm{e}^{z (1 / 2 + u)}}{\mathrm{e}^z - 1}
	   + \frac{z \mathrm{e}^{z (1 / 2 - u)}}
		  {\mathrm{e}^z - 1} \\
      &= \frac{z \mathrm{e}^{z / 2}}{\mathrm{e}^z - 1}
	   \cdot \left(
		   \mathrm{e}^{z u} + \mathrm{e}^{- z u}
		 \right)
  \end{align*}
  Esta expresión es par en \(z\):
  \begin{align*}
    \frac{-z \mathrm{e}^{-z / 2}}{\mathrm{e}^{-z} - 1}
       \cdot \frac{\mathrm{e}^z}{\mathrm{e}^z}
      &= \frac{-z \mathrm{e}^{z / 2}}{1 - \mathrm{e}^z} \\
      &= \frac{z \mathrm{e}^{z / 2}}{\mathrm{e}^z - 1}
  \end{align*}
  Esto significa que los términos para \(z^{2 k + 1}\) se anulan:
  \begin{equation*}
    B_{2 k + 1}(1 / 2 - u)
      = - B_{2 k + 1}(1 / 2 + u) \\
  \end{equation*}
  En particular, \(B_{2 k + 1}(1 / 2) = 0\).

  De forma similar:
  \begin{align*}
    B \left( 1 / 2 + u, z \right)
	- B \left( 1 / 2 - u, z \right)
      &= \frac{z \mathrm{e}^{z (1 / 2 + u)}}{\mathrm{e}^z - 1}
	   - \frac{z \mathrm{e}^{z (1 / 2 - u)}}
		  {\mathrm{e}^z - 1} \\
      &= \frac{z \mathrm{e}^{z / 2}}{\mathrm{e}^z - 1}
	   \cdot \left(
		   \mathrm{e}^{z u} - \mathrm{e}^{- z u}
		 \right)
  \end{align*}
  Esta expresión es impar en \(z\),
  lo que significa que ahora se anularon los términos pares:
  \begin{equation*}
    B_{2 k}(1 / 2 - u)
      = B_{2 k}(1 / 2 + u)
  \end{equation*}

  De las gráficas~\ref{fig:Bernoulli}
  en el rango \([0, 1]\)
  se ve que el polinomio \(B_{2 k}(z)\) tiene dos ceros,
  mientras \(B_{2 k + 1}(z)\) tiene tres
  (0 y \(\pm 1\)).
  Esto vale en general.%
    \index{Bernoulli, polinomios de!ceros}
  \begin{theorem}
    \label{theo:Bernoulli-zeros}
    Para \(k > 0\),
    en el rango \([0, 1]\) el polinomio \(B_{2 k}(z)\)
    tiene exactamente dos ceros,
    mientras \(B_{2 k + 1} (z)\) tiene exactamente tres
    (\(0\), \(1 / 2\) y	 \(1\)).
  \end{theorem}
  \begin{proof}
    Demostramos por inducción que para \(k \ge 1\) en \([0, 1 / 2]\)
    el polinomio \(B_{2 k} (z)\) tiene exactamente un cero,%
      \index{demostracion@demostración!induccion@inducción}
    y que \(B_{2 k + 1}(z)\) no cambia de signo
    y se anula únicamente en los extremos.
    De partida,
    ninguno de los polinomios es idénticamente cero.
    \begin{description}
    \item[Base:]
      Para \(k = 1\) tenemos \(B_2(z) = z^2 - z + 1 / 6\)
      con ceros \(1 / 2 \pm \sqrt{3} / 6\)
      (estos dos están en el rango  \([0, 1]\),
       hay uno en \([0, 1/2]\)),
      y \(B_3(z) = z^3 - 3 z^2 / 2 + z / 2\) con ceros \(0, \pm 1\).
    \item[Inducción:]
      Supongamos que \(B_{2 k}(z)\)
      tiene un único cero en \([0, 1 / 2]\),
      y que \(B_{2 k + 1}(z)\)
      se anula únicamente en \(0\) y \(1 /2\).
      Debemos demostrar que vale para \(B_{2 k + 2}(z)\)
      y \(B_{2 k +3}(z)\) también.

      Como \(B_{2 k + 1}(z)\) no cambia de signo en el rango,
      \(B_{2 k + 2}(z)\) es monótona
      y por tanto puede tener a lo más un cero.
      Pero pusimos como condición
      que la integral de \(B_{2 k + 2}(z)\)
      de \(0\) a \(1\) se anule;
      como \(B_{2 k + 2}(z)\)
      es simétrica alrededor de \(z = 1 / 2\)
      se anula la integral de \(0\) a \(1 / 2\),
      por lo que deben haber valores positivos y negativos
      en el rango,
      y hay exactamente un cero en él.

      Por la recurrencia \(B'_n (z) = n B_{n - 1} (z)\),
      al tener un único cero \(B_{2 k + 2} (z)\),
      \(B_{2 k + 3}(z)\) tiene un único mínimo o máximo
      en el rango \([0, 1/2]\),
      y \(B_{2 k + 3}(z)\) puede tener a lo más dos ceros allí
      y conocemos dos (0 y \(1 / 2\)).
    \end{description}
    Por inducción vale para todo \(k \ge 1\).
  \end{proof}

  Como \(B'_{2 k}(1 / 2) = 2 k B_{2 k - 1} (1 / 2) = 0\),
  sabemos que \(B_{2 k}(z)\)
  tiene un mínimo o máximo en \(z = 1 / 2\).
  Tenemos:
  \begin{align*}
    B(0, z / 2)
      &= \frac{z}{2 (\mathrm{e}^{z / 2} - 1)} \\
      &= \frac{1}{2} \cdot \frac{z}{\mathrm{e}^{z / 2} - 1}
	   \cdot \frac{\mathrm{e}^{z / 2} + 1}
		      {\mathrm{e}^{z / 2} + 1} \\
      &= \frac{1}{2}
	   \cdot \frac{z (\mathrm{e}^{z / 2} + 1)}
		      {\mathrm{e}^z - 1} \\
      &= \frac{1}{2} \,
	   \left(
	     \frac{z \mathrm{e}^{z / 2}}{\mathrm{e}^z - 1}
	       - \frac{z}{\mathrm{e}^z - 1}
	   \right) \\
      &= \frac{1}{2} \,
	   \left(B(1 / 2, z) - B(0, z) \right)
  \end{align*}
  Comparando coeficientes de \(z^k\):
  \begin{align*}
    \frac{B_k}{2^k}
      &= \frac{1}{2} \, \left( B_k(1 / 2) - B_k \right) \\
    B_k(1 / 2)
      &= - \left( 1 - 2^{1 - k} \right) \, B_k
  \end{align*}
  Como \(z = 1 / 2\) es el único máximo (mínimo) de \(B_{2 k} (z)\)
  en el rango \([0, 1]\) por ser monótona en \([0, 1 / 2]\),
  al ser simétrica alrededor de \(1 / 2\) los mínimos (máximos)
  se dan en los extremos,
  y en este rango:
  \begin{equation*}
    \lvert B_{2 k}(z) \rvert
      \le \lvert B_{2 k} \rvert
  \end{equation*}

\section{El resto}
\label{sec:resto-Euler-Maclaurin}

  Nuestra fórmula maestra es:
  \begin{equation*}
    \sum_{1 \le k < a} f(k)
      = \int_1^a f(z) \, \mathrm{d} z
	  + \gamma_f
	  + B_1 f(a)
	  + \sum_{1 \le k \le n}
	       \frac{B_{2 k}}{(2 k)!} \, f^{(2 k - 1)}(a)
	  + \int_a^\infty
	      \frac{\widetilde{B}_{2 n + 1}(z)}{(2 n + 1)!} \,
			      f^{(2 n + 1)}(z) \, \mathrm{d} z
  \end{equation*}
  Interesa acotar la integral que determina el resto,
  la llamaremos \(R_n(f; a)\).
  Podemos volver a integrar por partes:
  \begin{equation*}
    R_n(f; a)
      = \int_a^\infty \frac{\widetilde{B}_{2 n + 2}(z)}
			   {(2 n + 2)!} \,
			      f^{(2 n + 2)}(z) \, \mathrm{d} z
  \end{equation*}
  Suponiendo que \(f^{(2 n + 2)}(z)\) y \(f^{(2 n + 1)}(z)\)
  tienden monótonamente a cero
  (con lo que en particular no cambian signo),
  la integral queda acotada
  por el valor extremo de \(\widetilde{B}_{2 n + 2}(z)\)
  y la integral del segundo factor.
  Para el valor extremo de \(\widetilde{B}_{2 n + 2}(z)\)
  tenemos la cota \(\lvert B_{2 n + 2} \rvert\):%
    \index{Euler-Maclaurin, formula de@Euler-Maclaurin, fórmula de!resto|textbfhy}
  \begin{align*}
    \lvert R_n(f; a) \rvert
      &\le \frac{\lvert B_{2 n + 2} \rvert}{(2 n + 2)!}
	      \cdot \left\lvert
		      \int_a^\infty f^{(2 n + 2)}(z)
			\, \mathrm{d} z
		    \right\rvert \\
      &= \frac{\lvert B_{2 n + 2} \rvert}{(2 n + 2)!}
	   \cdot \lvert f^{(2 n + 1)}(a) \rvert
  \end{align*}
  Esto es del orden del primer término omitido,
  como se indicó.

%%% Local Variables:
%%% mode: latex
%%% TeX-master: "clases"
%%% End:


% aplicaciones.tex
%
% Copyright (c) 2009-2014 Horst H. von Brand
% Derechos reservados. Vea COPYRIGHT para detalles

\chapter{Aplicaciones}
\label{cha:aplicaciones}

  Veremos varias aplicaciones concretas adicionales
  de la maquinaria de funciones generatrices,
  hallando funciones generatrices
  y también derivando (y demostrando) identidades.
  De particular interés para nosotros es la solución de recurrencias,
  que comúnmente aparecen en problemas combinatorios,
  en particular aplicaciones al análisis de algoritmos.
  En el camino estudiaremos algunas de las secuencias más comunes
  en la combinatoria.

\section{Números harmónicos}
\label{sec:numeros-harmonicos}
\index{numeros harmonicos@números harmónicos}

  Los números harmónicos se definen como:
  \begin{equation}
    \label{eq:harmonic-number}
    H_n = \sum_{1 \le k \le n} \frac{1}{k}
  \end{equation}
  Además definimos
  \(H_0 = 0\)
  (consistente con que sumas vacías son cero).
  Esta secuencia es importante en teoría de números,
  se requiere para calcular una variedad de funciones especiales,
  además que aparece con frecuencia al analizar algoritmos.

  Los primeros valores son:
  \begin{equation*}
    \left\langle
      0, 1, \frac{3}{2}, \frac{11}{6}, \frac{25}{12},
      \frac{137}{60},  \frac{49}{20}, \frac{363}{140},
      \frac{761}{280}, \frac{7\,129}{2\,520},
      \dotsc
    \right\rangle
  \end{equation*}
  Buscamos una expresión para:
  \begin{equation}
    \label{eq:harmonic-number-ogf}
    H(z)
      = \sum_{n \ge 0} H_n z^n
  \end{equation}
  Esto se reduce a:
  \begin{align*}
    H(z)
      &= \sum_{n \ge 1} \biggl( \,
			  \sum_{1 \le k \le n} \frac{1}{k}
			\biggr) z^n \\
      &= z \sum_{n \ge 1} \biggl( \,
			     \sum_{0 \le k \le n - 1}
			       \frac{1}{k + 1}
			   \biggr) z^{n - 1} \\
      &= z \sum_{n \ge 0} \biggl( \,
			     \sum_{0 \le k \le n} \frac{1}{k + 1}
			   \biggr) z^n
  \end{align*}
  Como tenemos una suma entre manos,
  usamos la regla de sumas parciales:%
    \index{generatriz!numeros harmonicos@números harmónicos}
  \begin{align}
    H(z)
      &= z \cdot \frac{1}{1 - z}
	   \cdot \sum_{k \ge 0} \frac{z^k}{k + 1}
	   \notag \\
      &= \frac{1}{1 - z} \, \ln \frac{1}{1 - z}
	   \label{eq:H(z)}
  \end{align}
  Acá usamos la suma~\eqref{eq:ln(1-z)} para el logaritmo
  derivada en la sección~\ref{sec:otras-series}.

  Aprovechando que la serie~\eqref{eq:H(z)}
  converge para \(\lvert z \rvert < 1\)
  (la expresión entra en problemas en \(z = 1\),
   el radio de convergencia es \(\lvert z \rvert = 1\)),
  podemos evaluar expresiones como:
  \begin{equation*}
    \sum_{n \ge 0} H_n \cdot 2^{-n}
      = H(1 / 2)
      = 2 \, \ln 2
  \end{equation*}

% logaritmos.tex
%
% Copyright (c) 2015 Horst H. von Brand
% Derechos reservados. Vea COPYRIGHT para detalles

\section{Funciones generatrices con logaritmos}
\label{sec:gf-logs}

  Requeriremos coeficientes de varias series involucrando logaritmos,
  de la forma:
  \begin{equation}
    \label{eq:ln:alpha-beta}
    \frac{1}{(1 - z)^\alpha} \ln^\beta \frac{1}{1 - z}
  \end{equation}
  para \(\alpha\) y \(\beta\) enteros.
  El caso más simple es \(\alpha = 0\) y \(\beta = 1\):
  \begin{equation}
    \label{eq:ln:0-1}
    \ln \frac{1}{1 - z}
      = \sum_{n \ge 1} \frac{z^n}{n}
  \end{equation}
  De~\eqref{eq:ln:0-1} resulta directamente
  la función generatriz de los números harmónicos:%
    \index{numeros harmonicos@números harmónicos!generatriz}
  \begin{equation}
    \label{eq:ln:1-1}
    \frac{1}{1 - z} \ln \frac{1}{1 - z}
      = \sum_{n \ge 1} H_n z^n
  \end{equation}
  Aplicando la fórmula de Leibnitz:%
    \index{Leibnitz!formula de@fórmula de}
  \begin{equation}
    \label{eq:Leibnitz-derivative}
    \frac{\mathrm{d}^m}{\mathrm{d} z^m} (f(z) \cdot g(z))
      = \sum_{0 \le r \le m} \binom{m}{r} f^{(r)}(z) \cdot g^{(m - r)}(z)
  \end{equation}
  a~\eqref{eq:ln:1-1} resulta:
  \begin{align}
    \frac{\mathrm{d}^{k - 1}}{\mathrm{d} z^{k - 1}}
      \left( \frac{1}{1 - z} \ln \frac{1}{1 - z} \right)
      &= \sum_{0 \le r \le k - 1}
	   \binom{k - 1}{r}
	      \frac{\mathrm{d}^r}{\mathrm{d} z^r} \ln \frac{1}{1 - z}
	      \cdot
	      \frac{\mathrm{d}^{k - 1 - r}}{\mathrm{d} z^{k - 1 - r}}
		 \frac{1}{1 - z}
		       \notag \\
      &= \frac{(k - 1)!}{(1 - z)^k} \ln \frac{1}{1 - z}
	   + \sum_{1 \le r \le k - 1}
	       \binom{k - 1}{r}
		 \frac{(r - 1)!}{(1 - z)^r}
		   \cdot \frac{(k - 1 - r)!}{(1 - z)^{k - r}}
		       \notag \\
      &= \frac{(k - 1)!}{(1 - z)^k}
	   \left(
	     \ln \frac{1}{1 - z}
	       + \sum_{1 \le r \le k - 1} \frac{1}{r}
	   \right)
		       \notag \\
      &= \frac{(k - 1)!}{(1 - z)^k}
	   \left(
	     \ln \frac{1}{1 - z}
	       + H_{k - 1}
	   \right)
  \end{align}
  Despejando el término que nos interesa:
  \begin{equation}
    \label{eq:ln:k-1}
    \frac{1}{(1 - z)^k} \ln \frac{1}{1 - z}
      = \frac{1}{(k - 1)!}
	  \frac{\mathrm{d}^{k - 1}}{\mathrm{d} z^{k - 1}}
	    \left( \frac{1}{1 - z} \ln \frac{1}{1 - z} \right)
	  - \frac{1}{(1 - z)^k} H_{k - 1}
\end{equation}
  Con~\eqref{eq:ln:1-1},
  las propiedades conocidas de las funciones generatrices%
    \index{funcion generatriz@función generatriz!propiedades}
  y el teorema del binomio:%
    \index{teorema del binomio}
  \begin{align}
    [z^n] \frac{1}{(1 - z)^k} \ln \frac{1}{1 - z}
      &= \frac{(n + k  - 1)^{\underline{k - 1}}}{(k - 1)!} H_{n + k - 1}
	   - \binom{n + k - 1}{k - 1} H_{k - 1} \notag \\
      &= \binom{n + k - 1}{k - 1} (H_{n + k - 1} - H_{k - 1})
		      \label{eq:ln:k-1:coef}
  \end{align}
  Vemos que para \(n = 0\) el coeficiente se anula
  y para \(n = 1\) es \(1\),
  tal como debiera ser.
%% Checked against the Taylor series for n = 1, 2, 4 and k = 1, 2, 10
%%   HvB 2015-01-16

  Otra colección de interés resulta partiendo con:
  \begin{align}
    [z^n] \ln^2 \frac{1}{1 - z}
      &= \sum_{1 \le r \le n - 1} \frac{1}{r (n - r)} \notag \\
  \intertext{Aplicamos fracciones parciales al sumando:}
      &= \frac{1}{n}
	   \sum_{1 \le r \le n - 1}
	     \left( \frac{1}{r} + \frac{1}{n - r} \right) \notag \\
      &= \frac{2}{n} H_{n - 1}
	     \label{eq:ln:0-2:coef}
  \end{align}
%% Checked against the Taylor series for n = 2, 3, 4, 7, 10
%%   HvB 2015-01-16
  De esto:
  \begin{align*}
    [z^n]\frac{1}{1 - z} \ln^2 \frac{1}{1 - z}
      &= \sum_{1 \le r \le n} \frac{2}{r} H_{r - 1} \\
      &= 2 \sum_{1 \le r \le n}
	     \frac{1}{r} \sum_{1 \le s \le r - 1} \frac{1}{s} \\
      &= 2 \sum_{1 \le s < r \le n} \frac{1}{r s} \\
  \intertext{Por simetría en \(r\) y \(s\) podemos escribir:}
      &= \sum_{1 \le s < r \le n} \frac{1}{r s}
	   + \sum_{1 \le r < s \le n} \frac{1}{r s} \\
      &= \sum_{\substack{1 \le r \le n \\
			 1 \le s \le n \\
			 r \ne s}} \frac{1}{r s} \\
      &= H^2_n - H^{(2)}_n
  \end{align*}
  Acá usamos la definición de números harmónicos generalizados:%
    \index{numeros harmonicos@números harmónicos!generalizados|textbfhy}
  \begin{equation}
    \label{eq:H(m)n}
    H^{(m)}_n
      = \sum_{1 \le k \le n} \frac{1}{k^m}
  \end{equation}
  Con esto el coeficiente que nos interesa es:
  \begin{equation}
    \label{eq:ln:1-2:coef}
    [z^n] \frac{1}{1 - z} \ln^2 \frac{1}{1 - z}
      = H^2_n - H^{(2)}_n
  \end{equation}
%% Checked against the Taylor series for n = 1, 2, 3, 4, 10
%%   HvB 2015-01-16

  Interesan un par de funciones generatrices adicionales.
  Primero:
  \begin{align*}
    \frac{\mathrm{d}}{\mathrm{d} z}
      \left( \frac{1}{1 - z} \ln^2 \frac{1}{1 - z} \right)
      &= \frac{1}{(1 - z)^2} \ln^2 \frac{1}{1 - z}
	   + 2 \frac{1}{(1 - z)^2} \ln \frac{1}{1 - z} \\
    \frac{\mathrm{d}^2}{\mathrm{d} z^2}
      \left( \frac{1}{1 - z} \ln^2 \frac{1}{1 - z} \right)
      &= 2 \frac{1}{(1 - z)^3} \ln^2 \frac{1}{1 - z}
	   + 6 \frac{1}{(1 - z)^3} \ln \frac{1}{1 - z}
	   + \frac{2}{(1 - z)^3}
  \end{align*}
  de donde despejamos:
  \begin{align*}
    \frac{1}{(1 - z)^2} \ln^2 \frac{1}{1 - z}
      &= \frac{\mathrm{d}}{\mathrm{d} z}
	   \left( \frac{1}{1 - z} \ln^2 \frac{1}{1 - z} \right)
	   - 2 \frac{1}{(1 - z)^2} \ln \frac{1}{1 - z} \\
    \frac{1}{(1 - z)^3} \ln^2 \frac{1}{1 - z}
      &=  \frac{1}{2} \frac{\mathrm{d}^2}{\mathrm{d} z^2}
			\left( \frac{1}{1 - z} \ln^2 \frac{1}{1 - z} \right)
	   - 3 \frac{1}{(1 - z)^3} \ln \frac{1}{1 - z}
	   - \frac{1}{(1 - z)^3}
  \end{align*}
  Derivando término a término
  la serie con coeficientes~\eqref{eq:ln:1-2:coef}
  tenemos los coeficientes de las derivadas,
  usamos los coeficientes~\eqref{eq:ln:k-1:coef} deducidos antes:
  \begin{align}
    [z^n] \frac{1}{(1 - z)^2} \ln^2 \frac{1}{1 - z}
      &= (n + 1) \left( H^2_{n + 1} - H^{(2)}_{n + 1} \right)
	  - 2 \binom{n + 1}{1} \left( H_{n + 1} - H_1 \right) \notag \\
      &= \binom{n + 1}{1}
	   \left(
	     H^2_{n + 1} - H^{(2)}_{n + 1} - 2 H_{n + 1} + 2
	   \right) \label{eq:ln:2-2:coef} \\
%% Checked against the Taylor series for n = 2, 3, 4, 5, 10
%%   HvB 2015-01-20
    [z^n] \frac{1}{(1 - z)^3} \ln^2 \frac{1}{1 - z}
      &= \frac{1}{2} (n + 2) (n + 1)
	   \left( H^2_{n + 2} - H^{(2)}_{n + 2} \right) \notag \\
	   &\qquad
	   - 3 \binom{n + 2}{2}
	       \left(
		 H_{n + 2}
		   - H_2
	       \right) \notag
	   - \binom{n + 2}{2} \notag \\
      &= \binom{n + 2}{2}
	   \left(
	     H^2_{n + 2} - H^{(2)}_{n + 2}
	       - 3 H_{n + 2} + \frac{7}{2}
	   \right)
%% Checked against the Taylor series for n = 2, 3, 4, 5, 10
%%   HvB 2015-01-20
  \end{align}

%%% Local Variables:
%%% mode: latex
%%% TeX-master: t
%%% End:


\section{Potencias factoriales}
\label{sec:potencias-factoriales}

  Definamos:
  \begin{equation*}
    G(z, u)
      = \sum_{n \ge 0} u^{\underline{n}} \, \frac{z^n}{n!}
  \end{equation*}
  Como:
  \begin{equation*}
    \frac{u^{\underline{n}}}{n!}
      = \binom{u}{n}
  \end{equation*}
  tenemos:
  \begin{equation*}
    G(z, u)
      = \sum_{n \ge 0} \binom{u}{n} z^n
      = (1 + z)^u
  \end{equation*}
  Esto implica:
  \begin{equation*}
    G(z, u) \cdot G(z, v)
      = G(z, u + v) \\
  \end{equation*}
  Podemos evaluar entonces de dos formas:
  \begin{align}
    G(z, u) \cdot G(z, v)
      &= \sum_{n \ge 0}
	   \biggl( \,
	     \sum_{0 \le k \le n}
	       \binom{n}{k}
		 \, u^{\underline{k}} v^{\underline{n - k}}
	   \biggr) \, \frac{z^n}{n!}
	       \label{eq:G(u)G(v)}\\
    G(z, u + v)
      &= \sum_{n \ge 0} (u + v)^{\underline{n}} \, \frac{z^n}{n!}
	       \label{eq:G(u+v)}
  \end{align}
  Comparando los coeficientes de \(z^n\)
  en~\eqref{eq:G(u)G(v)} y~\eqref{eq:G(u+v)} resulta:
  \begin{equation}
    \index{binomio, teorema del!potencias factoriales}
    \label{eq:binomial-falling}
    (u + v)^{\underline{n}}
      = \sum_{0 \le k \le n}
	  \binom{n}{k} \, u^{\underline{k}} v^{\underline{n - k}}
  \end{equation}
  Curioso equivalente de la fórmula
  para la potencia de un binomio.
  Aprovechando la relación
  entre potencias factoriales en subida y en bajada
  se puede derivar una relación similar
  para las potencias factoriales en subida.

\section{Números de Fibonacci}
\label{sec:Fibonacci}
\index{Fibonacci, numeros de@Fibonacci, números de}

  Consideremos la secuencia:
  \begin{equation}
    \label{eq:Fibonacci-sequence}
    \langle 0, 1, 1, 2, 5, 8, 13, 21, 34, \dotsc \rangle
  \end{equation}
  que se obtiene de la recurrencia válida para \(n \ge 0\):
  \begin{equation}
    \label{eq:recurrence-Fibonacci}
    F_{n + 2}
      = F_{n + 1} + F_n
      \qquad \text{\(F_0 = 0\), \(F_1 = 1\)}
  \end{equation}
  Esta la encontramos al analizar el algoritmo de Euclides%
    \index{Euclides, algoritmo de!analisis@análisis}
  para el máximo común divisor
  en la sección~\ref{sec:gcd}.
  Aparece en una gran variedad de situaciones
  relacionadas con nuestra área,
  y muchos fenómenos naturales,
  como el crecimiento de los árboles
  y las espirales que se observan en los girasoles,
  siguen aproximadamente esta secuencia.
  Véanse por ejemplo el libro de Dunlap~%
    \cite{dunlap98:_golden_ratio_fibonacci}
  para una variedad de situaciones donde aparecen,
  y la muy detallada discusión de Vajda~%
    \cite{vajda89:_fibonacci_lucas_number_golden_section}.

\subsection{Solución mediante funciones generatrices ordinarias}
\label{sec:Fibonacci-ordinarias}

  Definimos:
  \begin{equation*}
    F(z)
      = \sum_{n \ge 0} F_n z^n
  \end{equation*}
  Aplicando las propiedades de funciones generatrices ordinarias
  a~\eqref{eq:recurrence-Fibonacci}:
  \begin{equation*}
    \frac{F(z) - F_0 - F_1 \cdot z}{z^2}
      = \frac{F(z) - F_0}{z} + F(z)
  \end{equation*}
  Substituyendo los valores de \(F_0\) y \(F_1\)
  y despejando resulta:
  \begin{equation}
    \index{Fibonacci, numeros de@Fibonacci, números de!generatriz|textbfhy}
    \label{eq:gf-Fibonacci}
    F(z)
      = \frac{z}{1 - z - z^2}
  \end{equation}
  Necesitamos reducir~\eqref{eq:gf-Fibonacci}
  a fracciones con denominadores lineales,
  usando fracciones parciales.
  Buscamos factorizar de la siguiente manera:
  \begin{equation*}
    1 - z - z^2 = (1 - r_{+} z) (1 - r_{-} z)
  \end{equation*}
  Para obtener esta factorización
  realizamos el cambio de variable \(y = 1 / z\)
  y tenemos:
  \begin{align*}
    y^2 - y - 1
      &= (y - r_{+}) (y - r_{-}) \\
    r_{\pm}
      &= \frac{1 \pm \sqrt{5}}{2}
  \end{align*}
  y denotamos \(r_{+} = \tau\) y \(r_{-} = \phi\).
  El número \(\tau\) es la \emph{sección áurea}%
    \index{\(\tau\) (seccion aurea)@\(\tau\) (sección áurea)|see{sección áurea}}%
    \index{seccion aurea@sección áurea|textbfhy}
  (por la palabra griega para \emph{corte}),
  que ya habíamos encontrado antes
  al analizar el algoritmo de Euclides%
    \index{Euclides, algoritmo de!analisis@análisis}
  para el máximo común divisor.
  Una notación común para la sección áurea
  (particularmente en matemáticas recreativas)
  es \(\phi\) o \(\varphi\),
  en honor al escultor ateniense Fidias,%
    \index{Fidias}
  quien se dice usó esta razón extensamente en su trabajo.
  Otros usan \(\phi = - r_{-}\) y \(\Phi = r_{+}\),
  de forma de tener números positivos siempre.

  Podemos expresar:
  \begin{align*}
    y^2 - y - 1
      &= (y - \tau) (y - \phi) \\
      &= y^2 - (\tau + \phi) y + \tau \phi
  \end{align*}
  Comparando coeficientes
  resulta \(\phi = 1 - \tau = -1 / \tau\).
  Las fracciones parciales resultan ser:
  \begin{equation}
    \label{eq:gf-Fibonacci-fracciones-parciales}
    F(z) = \frac{1}{\tau - \phi} \cdot
	     \left(
		\frac{1}{1 - \tau z} - \frac{1}{1 - \phi z}
	     \right)
  \end{equation}
  En~\eqref{eq:gf-Fibonacci-fracciones-parciales}
  se reconocen dos series geométricas.
  Esto da la sorprendente relación,
  conocida como fórmula de Binet,%
    \index{Binet, formula de@Binet, fórmula de|textbfhy}%
    \index{Fibonacci, numeros de@Fibonacci, números de!formula de Binet@fórmula de Binet}
  que expresa los números de Fibonacci
  (enteros)
  en términos de números irracionales:%
    \index{numero@número!irracional}
  \begin{align}
    F_n
      &= \frac{\tau^n - \phi^n}{\tau - \phi} \\
      &= \frac{(1 + \sqrt{5})^n - (1 - \sqrt{5})^n}{2^n \sqrt{5}}
	   \label{eq:Binet-Fibonacci}
  \end{align}
  Ahora bien:
  \begin{equation*}
    \frac{1}{2 \tau - 1}
      = \frac{1}{\sqrt{5}}
      = 0,4472\dotso
    \hspace{3em}
    \tau
      = 1,618\dotso
    \hspace{3em}
    \phi
      = -0,6180\dotso
  \end{equation*}
  Resulta para todo \(n \ge 0\):
  \begin{equation*}
    \left| \frac{\phi^n}{2 \tau - 1} \right| < 0,5
  \end{equation*}
  Por lo tanto,
  \(F_n = \tau^n / \sqrt{5}\),
  redondeado al entero más cercano.
  De todas formas:%
    \index{Fibonacci, numeros de@Fibonacci, números de!asintotica@asintótica}
  \begin{equation}
    \label{eq:Fibonacci-asymptotic}
    F_n
      \sim \frac{\tau^n}{\sqrt{5}}
  \end{equation}

  Podemos obtener otras relaciones
  de la función generatriz~\eqref{eq:gf-Fibonacci}:
  \begin{equation*}
    F(z)
      = \frac{z}{1 - z - z^2}
      = \frac{z}{1 - z(1 + z)}
      = \sum_{r \ge 0} z^{r + 1} (1 + z)^r
      = \sum_{r, s \ge 0} \binom{r}{s} z^{r + s + 1}
  \end{equation*}
  De aquí,
  como con \(n = r + s\) tenemos \(r = n - s\):
  \begin{equation}
    \index{Fibonacci, numeros de@Fibonacci, números de!relacion con coeficientes binomiales@relación con coeficientes binomiales}
    \label{eq:Fibonacci-binomial}
    F_{n + 1}
      = \sum_{0 \le s \le n} \binom{n - s}{s}
  \end{equation}

\subsection{Solución mediante funciones generatrices exponenciales}
\label{sec:Fibonacci-exponenciales}

  Definimos la función generatriz exponencial:
  \begin{equation*}
    \widehat{F}(z)
      = \sum_{n \ge 0} \frac{F_n z^n}{n!}
  \end{equation*}
  Aplicando las propiedades de funciones generatrices exponenciales%
    \index{generatriz!exponencial}
  a~\eqref{eq:recurrence-Fibonacci}:
  \begin{equation*}
    \widehat{F}''(z)
      = \widehat{F}'(z) + \widehat{F}(z)
      \quad \text{\(\widehat{F}(0) = 0\), \(\widehat{F}'(0) = 1\)}
  \end{equation*}
  Resolvemos esta ecuación diferencial
  ordinaria lineal de segundo orden,
  homogénea y de coeficientes constantes
  por el método de la ecuación característica:
  \begin{equation*}
    r^2
      = r + 1
  \end{equation*}
  Los ceros son
  \(\tau\) y \(\phi\),
  con lo que tenemos:
  \begin{equation*}
    \widehat{F}(z)
      = \alpha \mathrm{e}^{\tau z} + \beta \mathrm{e}^{\phi z}
  \end{equation*}
  de donde:
  \begin{align*}
    \widehat{F}(0)  &= \alpha + \beta = 0 \\
    \widehat{F}'(0) &= \alpha \tau + \beta \phi = 1
  \end{align*}
  La solución de estas ecuaciones es:
  \begin{equation*}
    \alpha
      = \frac{1}{\sqrt{5}} \qquad
    \beta
      = -\frac{1}{\sqrt{5}}
  \end{equation*}
  y finalmente resulta
  la misma fórmula~\eqref{eq:Binet-Fibonacci} anterior:
  \begin{align*}
    F_n
      = \frac{1}{\sqrt{5}}
	  n! \left[ z^n \right] \,
	    \left(
	      \mathrm{e}^{\tau z} - \mathrm{e}^{\phi z}
	    \right)
      = \frac{\tau^n - \phi^n}{\sqrt{5}}
  \end{align*}
  Si comparamos las derivaciones,
  obtener la ecuación y sus condiciones de borde es más simple
  al usar funciones generatrices exponenciales,
  luego debemos resolver una ecuación diferencial,
  pero obtener el resultado
  de la solución de la ecuación diferencial es inmediato.
  En la derivación usando funciones generatrices ordinarias
  obtener la ecuación era algo más trabajo,
  y tuvimos que usar fracciones parciales
  para poder obtener la secuencia;
  pero tratar la ecuación misma era más simple.
  De todas formas,
  siempre tendremos las dos opciones.
  Cuál resulta más conveniente dependerá de la situación específica.

\subsection{Números de Fibonacci y fuentes}
\label{sec:Fibonacci-fuentes}

  Si comparamos la secuencia~\eqref{eq:fountains-sequence}
  de números de fuentes de base \(n\)%
    \index{fuente}
  con los números de Fibonacci,
  parecieran ser los términos alternos:
  \begin{equation*}
    \langle F_{2 n + 1} \rangle_{n \ge 0}
      = \langle 1, 2, 5, 13, 34, \dotsc \rangle
  \end{equation*}
  Una manera de verificar esto es extraer los términos impares
  de la función generatriz~\eqref{eq:gf-Fibonacci},%
    \index{serie de potencias!decimar}
  o sea encontrar una función generatriz
  para \(\langle F_{2 n + 1} \rangle_{n \ge 0}\)
  y comparar con~\eqref{eq:gf-fountain-pf}.

  Aplicamos la técnica descrita en la sección~\ref{sec:decimar}
  a la función generatriz~\eqref{eq:gf-Fibonacci}.
  Para los números de Fibonacci impares resulta:
  \begin{equation*}
    \frac{F(\sqrt{z}) - F(-\sqrt{z})}{2 \sqrt{z}}
      = \frac{1 - z}{1 - 3 z + z^2}
  \end{equation*}
  En nuestro caso tenemos la secuencia desplazada en uno,
  de~\eqref{eq:gf-fountain}:
  \begin{equation*}
    \frac{f(z) - f_0}{z}
      = \frac{1 - z}{1 - 3 z + z^2}
  \end{equation*}
  Coinciden,
  o sea:
  \begin{equation}
    \label{eq:fountain-Fibonacci}
    f_n
      = \begin{cases}
	  1	      & \text{si \(n = 0\)} \\
	  F_{2 n - 1} & \text{si \(n \ge 1\)}
	\end{cases}
  \end{equation}

\subsection{Búsqueda de Fibonacci}
\label{sec:busqueda-Fibonacci}
\index{Fibonacci, busqueda de@Fibonacci, búsqueda de|textbfhy}

  La \emph{búsqueda de Fibonacci}
  (ver por ejemplo a Kiefer~%
    \cite{kiefer53:_seq_minimax_search_maximum})
  es un método para encontrar el mínimo
  de una función en un rango dado.
  Resulta incluso que esta técnica es óptima
  en cuanto a número de veces que se evalúa la función
  siempre que nos restrinjamos a solo comparar valores.
  \begin{definition}
    Una función \(f : \mathbb{R} \rightarrow \mathbb{R}\)
    se dice \emph{unimodal} sobre el rango \([a, b]\)%
      \index{funcion@función!unimodal}
    si hay un único \(\xi\) con \(a \le \xi \le b\) tal que
    \(f\) es decreciente en \([a, \xi]\)
    y creciente en \([\xi, b]\).
  \end{definition}
  Esto describe el caso en que la función tenga un único mínimo
  en el rango,
  de forma muy similar se define el caso que tiene un único máximo,
  y en ambas situaciones se llama unimodal la función.

  Nos interesa acotar el mínimo en el rango \([a, b]\)
  de una función unimodal \(f(z)\)
  recurriendo únicamente a evaluar la función.
  Por ejemplo,
  la función está dada por una computación compleja
  y no hay forma de calcular su derivada.
  Supongamos que tenemos los valores de la función
  en los puntos \(a\) y \(b\).
  Elegimos dos puntos adicionales \(c < d\)
  dentro del rango \([a, b]\),
  y evaluamos la función en ellos,
  resultando la situación
  de la figura~\ref{fig:Fibonacci-search-step}.
  \begin{figure}[htbp]
    \centering
    \pgfimage{images/Fibonacci-search-step}
    \caption{Búsqueda de Fibonacci}
    \label{fig:Fibonacci-search-step}
  \end{figure}
  Al ser unimodal \(f\)
  sabemos que \(f(c)\) y \(f(d)\)
  son ambos menores que \(\max \{ f(a), f(b) \}\).
  Si \(f(c) < f(d)\),
  el mínimo está en el subintervalo \([a, d]\),
  descartamos el tramo \((d, b]\)
  y trabajamos con el nuevo rango \([a, d]\).
  De la misma manera,
  si \(f(c) > f(d)\),
  el mínimo está en el tramo \([c, b]\),
  descartamos el rango \([a, c)\).

  Consideremos el ejemplo
  de la figura~\ref{fig:Fibonacci-search-step}.
  Descartamos el rango \([a, c)\)
  y elegimos un nuevo punto \(e\) entre \(d\) y \(b\),
  y seguimos con nuevos puntos \(a'\), \(b'\), \(c'\) y \(d'\).
  Para solo calcular una vez la función en la iteración
  se reutilizan los valores calculados
  en los puntos \(a\), \(c\) y \(d\)
  (de descartar \((d, b]\))
  o en los puntos \(c\), \(d\) y \(b\)
  (de descartar \([a, c)\)).
  Queremos además
  que el método reduzca el tramo en la misma proporción
  en ambos casos.
  O sea,
  debe ser \(d - a = b - c\).
  En el siguiente paso queremos que se vuelva a repetir esto,
  debe ser también
  \(b - d = e - c\).

  Definamos \(r\) mediante \(d - a = r (b - a)\),
  con lo que \(c - a = (1 - r) (b - a)\).
  Restando:
  \begin{align*}
    d - c
      &= (d - a) - (c - a) \\
      &= r (b - a) - (1 - r) (b - a) \\
      &= (2 r  - 1) (b - a)
  \end{align*}
  Para el paso siguiente,
  \(a' = c\),
  \(c' = d\),
  \(d' = e\)
  y \(b' = b\).
  Elegimos \(r'\) mediante \(d' - a' = r' (b' - a')\),
  de donde resulta \(c' - a' = (1 - r') (b' - a')\),
  que es decir \(d - c = (1 - r') (b' - a')\).
  El intervalo \([a, b]\) se redujo a \([a', b']\),
  con \(b' - a' = r (b - a)\).
  Igualando los valores de \(d - c\),
  y substituyendo el valor de \(b' - a'\) resulta:
  \begin{align*}
    (2 r - 1) (b - a)
      &= (1 - r') (b ' - a') \\
    (2 r - 1) (b - a)
      &= (1 - r') r (b - a) \\
    2 r - 1
      &= (1 - r') r
  \end{align*}
  Despejando \(r\):
  \begin{equation}
    \label{eq:Fibonacci-search-recurrence-r}
    r = \frac{1}{r' + 1}
  \end{equation}
  Partiendo del final,
  esto permite calcular las razones previas.
  En el caso extremo reducimos el intervalo en una razón de \(1\)
  (vale decir, se mantiene el tamaño).
  \begin{figure}[htbp]
    \centering
     \pgfimage{images/Fibonacci-search-final}
    \caption{Búsqueda de Fibonacci: Juego final}
    \label{fig:Fibonacci-search-final}
  \end{figure}
  El paso final
  lo ilustra la figura~\ref{fig:Fibonacci-search-final}.
  El algoritmo retorna el rango marcado \(a\) a \(b\)
  como resultado final,
  la mejor aproximación a \(\xi\) es \(c = d = (a + b) / 2\).

  Esto sugiere la recurrencia:
  \begin{equation}
    \label{eq:Fibonacci-search-recurrence}
    r_{k + 1}
      = \frac{1}{1 + r_k}
      \quad (k \ge 1)
      \qquad r_0 = 1
  \end{equation}
  Intentando algunos valores obtenemos:
  \begin{equation*}
    \left\langle
      1, \frac{1}{2}, \frac{2}{3}, \frac{3}{5}, \frac{5}{8}, \dotsc
    \right\rangle
  \end{equation*}
  Da la impresión que:
  \begin{equation}
    \label{eq:Fibonacci-search-solution}
    r_k
      = \frac{F_{k + 1}}{F_{k + 2}}
  \end{equation}
  Esto es fácil de demostrar por inducción,
  los detalles los proveerá el amable lector.
  Con esto estamos en condiciones de plantear
  la búsqueda de Fibonacci,
  algoritmo~\ref{alg:Fibonacci-search}.
  Suponemos dados el intervalo \([a, b]\)
  y la tolerancia \(\epsilon\)
  (el largo del último intervalo).
  Para reducir el largo del intervalo en un factor \(F_n\)
  se calcula la función \(n + 4\) veces.
  En vista de~\eqref{eq:Fibonacci-asymptotic}
  para reducir el rango de largo \(L_0\) a \(L_f\)
  el número de llamadas de la función es:
  \begin{align*}
    n &\sim \frac{\ln (L_0 / L_f)}{\ln \tau}
	      + 4 + \frac{\ln 5}{2 \ln \tau} \\
      &\sim 4,78497 \, \ln \frac{L_0}{L_f} + 5,67723
  \end{align*}
  \begin{algorithm}[htbp]
    \DontPrintSemicolon
    \SetKwFunction{Fibonacci}{FibonacciSearch}

    \KwFunction \Fibonacci{\(f,\; a, \; b, \; \epsilon\)} \;
    \BlankLine
    \(L \leftarrow (b - a) / \epsilon\) \;
    \((F_a, F_b) \leftarrow (1, 1)\) \;
    \While{\(F_b < L\)}{
      \((F_a, F_b) \leftarrow (F_b, F_a + F_b)\) \;
    }
    \(c \leftarrow b - (b - a) \cdot F_a / F_b\) \;
    \(d \leftarrow a + (b - a) \cdot F_a / F_b\) \;
    \((f_a, f_b, f_c, f_d) \leftarrow (f(a), f(b), f(c), f(d))\) \;
    \While{\(F_a \ne 1\)}{
      \((F_a, F_b) \leftarrow (F_b - F_a, F_a)\) \;
      \eIf{\(f_d < f_c\)}{
	\(e \leftarrow c + (b - c) \cdot F_a / F_b\) \;
	\(f_e \leftarrow f(e)\) \;
	\((a, c, d, b) \leftarrow (c, d, e, b)\) \;
	\((fa, fc, fd, fb) \leftarrow (fc, fd, fe, fb)\) \;
      }{
	\(e \leftarrow d - (d - a) \cdot F_a / F_b\) \;
	\(f_e \leftarrow f(e)\) \;
	\((a, c, d, b) \leftarrow (a, e, c, d)\) \;
	\((fa, fc, fd, fb) \leftarrow (fa, fe, fc, fd)\) \;
      }
    }
    \eIf{\(f_a < f_b\)}{
      \Return \([a, d]\) \;
    }{
      \Return \([d, b]\) \;
    }
    \caption{Búsqueda de Fibonacci}
    \label{alg:Fibonacci-search}
  \end{algorithm}

  Un método relacionado es la búsqueda de sección áurea.%
    \index{seccion aurea@sección áurea!busqueda de@búsqueda de|see{Fibonacci, búsqueda de}}
  La idea es similar,
  solo que en vez de ir modificando \(r\) se usa el valor límite:
  \begin{equation*}
    \lim_{n \rightarrow \infty} \frac{F_{n + 1}}{F_n}
      = \tau
  \end{equation*}
  El programa es un poco más sencillo,
  pero algo menos eficiente
  (requiere más evaluaciones de la función).

\section{Coeficientes binomiales}
\label{sec:coeficientes-binomiales}
\index{coeficiente binomial}

  Hagamos como que nada sabemos\ldots
  ¿Cuántos subconjuntos de \(k\)
  elementos podemos obtener de un conjunto de \(n\) elementos?
  Obviamente,
  exactamente qué conjunto de \(n\) elementos tomemos da lo mismo,
  podemos usar el conjunto \(\{1, 2, \dotsc, n\}\)
  sin pérdida de generalidad.
  Podemos deducir algunas propiedades de estas:
  \begin{itemize}
  \item
    Tomar un número negativo de elementos
    del conjunto no tiene sentido,
    o sea,
    \(\binom{n}{k} = 0\) si \(k < 0\).
  \item
    Tomar más de \(n\) elementos es imposible,
    así que \(\binom{n}{k} = 0\) si \(k > n\).
  \item
    Hay una única forma de elegir cero elementos,
    y \(\binom{n}{0} = 1\).
  \item
    De la misma forma,
    hay una única manera de elegirlos todos,
    y \(\binom{n}{n} = 1\).
  \item
    Elegir \(k\) elementos a poner en el subconjunto
    es lo mismo que elegir los \(n - k\) que se dejan fuera,
    o sea \(\binom{n}{k} = \binom{n}{n - k}\).
  \end{itemize}
  Ahora buscamos encontrar
  una recurrencia para los \(\binom{n}{k}\).
  Podemos descomponer los \(\binom{n}{k}\) subconjuntos
  en dos grupos:
  \begin{description}
  \item[\boldmath Aquellos conjuntos
	que no contienen a \(n\):\unboldmath]
    Corresponden simplemente a tomar \(k\) elementos
    de los restantes \(n - 1\),
    de estos hay \(\binom{n - 1}{k}\).
  \item[\boldmath Aquellos conjuntos
	que contienen a \(n\):\unboldmath]
    Tomamos \(n\),
    y \(k - 1\) elementos más de entre los restantes \(n - 1\),
    de estos hay \(\binom{n - 1}{k - 1}\).
  \end{description}
  Como estas dos posibilidades son excluyentes,
  y corresponden a todas las formas
  de armar subconjuntos de \(k\) elementos:
  \begin{equation*}
    \binom{n}{k} = \binom{n - 1}{k} + \binom{n - 1}{k - 1}
  \end{equation*}
  Esto en principio es válido para \(1 \le k \le n - 1\).
  Pero si substituimos \(k = n\) bajo los entendidos de arriba
  resulta \(\binom{n}{n} = 1\),
  y con \(k > n\) se reduce a \(\binom{n}{k} = 0\),
  y la recurrencia en realidad es válida para \(k \ge 1\).
  Partiremos de \(k + 1\) y \(n + 1\)
  para poder sumar desde \(k = 0\) y \(n = 0\):
  \begin{equation}
    \label{eq:binomial-recurrencia}
    \binom{n + 1}{k + 1}
      = \binom{n}{k + 1} + \binom{n}{k}
  \end{equation}
  Como condiciones de contorno bastan:
  \begin{equation*}
    \binom{n}{0}
      = 1 \qquad
    \binom{0}{k}
      = [k = 0]
  \end{equation*}
  Ahora tenemos tres opciones de función generatriz ordinaria:
  \begin{equation*}
    A_n(x)
      = \sum_{k \ge 0} \binom{n}{k} x^k
      \qquad
    B_k(y)
      = \sum_{n \ge 0} \binom{n}{k} y^n
      \qquad
    C(x, y)
      = \sum_{\substack{k \ge 0 \\
			n \ge 0}}
	  \binom{n}{k} x^k y^n
  \end{equation*}
  Las primeras dos opciones
  llevarán a recurrencias en la función generatriz,
  cosa que la tercera resuelve automáticamente.
  Nada indica particulares complicaciones
  (más allá del uso de dos variables y no una
   como ha sido común hasta acá),
  por lo que optaremos por esta.
  Aplicando las propiedades de funciones generatrices ordinarias%
    \index{generatriz!ordinaria}
  a la recurrencia queda:
  \begin{equation}
    \label{eq:binomial-bivariada}
    \frac{C(x, y) - C(x, 0) - C(0, y) + C(0, 0)}{x y}
      = \frac{C(x, y) - C(0, y)}{x} + C(x, y)
  \end{equation}
  El numerador del lado izquierdo
  de~\eqref{eq:binomial-bivariada} corresponde a:
  \begin{equation*}
    \sum_{\substack{k \ge 0 \\
		    n \ge 0}}
      \binom{n + 1}{k + 1} x^{k + 1} y^{n + 1}
	= \sum_{\substack{k \ge 0 \\
			  n \ge 0}}
		\binom{n}{k} x^k y^n
	    - \sum_{k \ge 0} \binom{0}{k} x^k y^0
	    - \sum_{n \ge 0} \binom{n}{0} x^0 y^n
	    + \binom{0}{0}
  \end{equation*}
  que resulta de eliminar la primera fila y columna de la suma
  (de eso se hacen cargo los dos siguientes términos);
  al restarlas estamos restando dos veces \(C(0, 0)\),
  y debemos reponerlo,
  lo que da lugar al último término.
  Esto es el principio de inclusión y exclusión,%
    \index{inclusion y exclusion, principio de@inclusión y exclusión, principio de}
  capítulo~\ref{cha:pie},
  haciendo su trabajo.
  De nuestras condiciones de contorno:
  \begin{equation*}
    C(0, 0)
      = \binom{0}{0} = 1 \qquad
    C(x, 0)
      = \sum_{k \ge 0} \binom{0}{k} \, x^k = 1 \qquad
    C(0, y)
      = \sum_{n \ge 0} \binom{n}{0} \, y^n = \frac{1}{1 - y}
  \end{equation*}
  Despejando obtenemos:
  \begin{equation*}
    C(x, y)
      = \frac{1}{1 - (1 + x) y}
  \end{equation*}
  Expandiendo la serie geométrica:%
    \index{serie geometrica@serie geométrica}
  \begin{equation*}
    \binom{n}{k}
      = \left[ x^k y^n \right] C(x, y)
      = \left[ x^k y^n \right]
	  \sum_{r \ge 0} (1 + x)^r y^r
      = \left[ x^k \right] (1 + x)^n
  \end{equation*}
  Tenemos nuevamente la relación
  entre los coeficientes binomiales
  y el número de combinaciones de \(k\) elementos
  tomados entre \(n\).

  La recurrencia~\eqref{eq:binomial-recurrencia}
  mostrada de la siguiente manera:
  \begin{equation*}
    \xymatrix{
      \displaystyle \binom{n}{k} \ar[dr] &
	 & \displaystyle \binom{n}{k + 1} \ar[dl] \\
	 & \displaystyle\binom{n + 1}{k + 1} &
    }
  \end{equation*}
  y recordando \(\binom{n}{0} = \binom{n}{n} = 1\)
  da el famoso triángulo de Pascal,%
    \index{Pascal, triangulo de@Pascal, triángulo de|textbfhy}
  ver cuadro~\ref{tab:triangulo-Pascal}
  (comparar también con la sección~\ref{sec:conteos-recurrentes},
   en particular el teorema~\ref{theo:identidad-Pascal}).
  \begin{table}[htbp]
    \centering
    \begin{tabular}{>{\(}r<{\)}*{12}{>{\(}c<{\)}@{\hspace{1ex}}}>{\(}c<{\)}}
      n=0:& \phantom{00}
		& \phantom{00}
		    & \phantom{00}
			& \phantom{00}
			    & \phantom{00}
				 & \phantom{00}
				      &	 1 \\
	 \noalign{\smallskip\smallskip}
      n=1:&	&   &	&   &	 &  1 &	   &  1 \\
	 \noalign{\smallskip\smallskip}
      n=2:&	&   &	&   & 1	 &    &	 2 & \phantom{00}
					       &  1 \\
	 \noalign{\smallskip\smallskip}
      n=3:&	&   &	& 1 &	 &  3 &	   &  3 & \phantom{00}
						    &  1 \\
	 \noalign{\smallskip\smallskip}
      n=4:&	&   & 1 &   & 4	 &    &	 6 &	&  4 & \phantom{00}
							 &  1 \\
	 \noalign{\smallskip\smallskip}
      n=5:&	& 1 &	& 5 &	 & 10 &	   & 10 &    &	5 & \phantom{00}
							      &	 1
	      & \phantom{00} \\
	 \noalign{\smallskip\smallskip}
      n=6:& 1 &	  & 6	 &  & 15 &    & 20 &	& 15 &	  & 6 & \phantom{00}
								   & 1 \\
	 \noalign{\smallskip\smallskip}
    \end{tabular}
    \caption{Triángulo de Pascal}
    \label{tab:triangulo-Pascal}
  \end{table}

% recurrencia-2.tex
%
% Copyright (c) 2013-2014 Horst H. von Brand
% Derechos reservados. Vea COPYRIGHT para detalles

\section{Otras recurrencias de dos índices}
\label{sec:recurrencia-2-indices}

  Consideremos el problema de calcular
  cuántos subconjuntos de \(k\) elementos de \(\{1, 2, \dotsc, n\}\)
  hay tal que no contienen números consecutivos.
  Llamemos \(s(n, k)\) a este valor.
  Para construir una recurrencia para ellos,
  aplicamos el método general de ver qué pasa al incluir o excluir \(n\).
  \begin{description}
  \item[\boldmath No incluye \(n\):\unboldmath]
    Esto es simplemente elegir \(k\) de entre los primeros \(n - 1\),
    o sea \(s(n - 1, k)\).
  \item[\boldmath Incluye \(n\):\unboldmath]
    Quedan por agregar \(k - 1\) elementos,
    que no pueden incluir a \(n - 1\),
    o sea corresponde a \(s(n - 2, k - 1)\).
  \end{description}
  Esto nos da la recurrencia:
  \begin{equation*}
    s(n, k)
      = s(n - 1, k) + s(n - 2, k - 1)
  \end{equation*}
  Ajustando índices:
  \begin{equation}
    \label{eq:recurrence-s-1}
    s(n + 2, k + 1)
      = s(n + 1, k + 1) + s(n, k)
  \end{equation}
  Es claro que para \(n \ge 1\):
  \begin{equation}
    s(n, 1)
      = n
	   \label{eq:recurrence-s(n,1)-boundary}
  \end{equation}
  Requeriremos
  los valores \(s(0, 0)\), \(s(n, 0)\), \(s(0, k)\), \(s(1, k)\).
  De la recurrencia,
  con \(n \ge 0\):
  \begin{align*}
    s(n + 2, 1)
      &= s(n + 1, 1) + s(n, 0) \\
    n + 2
      &= n + 1 + s(n, 0)
  \end{align*}
  Por tanto,
  definimos \(s(n, 0) = 1\).
  Similarmente:
  \begin{equation*}
    s(2, k + 1)
      = s(1, k + 1) + s(0, k)
  \end{equation*}
  Para \(k \ge 1\) resulta \(s(0, k) = 0\),
  con lo que \(s(0, k) = [k = 0]\).
  Además,
  uniendo los casos \(s(1, 0) = s(1, 1) = 1\)
  con \(s(1, k) = 0\) para \(k > 1\):
  \begin{equation*}
    s(1, k)
      = [ 0 \le k \le 1 ]
  \end{equation*}

  Definamos la función generatriz:
  \begin{equation}
    \label{eq:recurrence-s-GF}
    S(x, y)
      = \sum_{n, k \ge 0} s(n, k) x^n y^k
  \end{equation}
  Para aplicar nuestra técnicas de solución de recurrencias,
  necesitaremos las sumas:
  \begin{align*}
    x^2 y \sum_{n, k \ge 0} s(n + 2, k + 1) x^n y^k
      &= S(x, y)
	   - \sum_{k \ge 0} s(0, k) y^k
	   - \sum_{k \ge 0} s(1, k) x y^k
	   - \sum_{n \ge 0} s(n, 0) x^n \\
      &\hspace{6em}
	   + s(0, 0)
	   + s(1, 0) x \\
      &= S(x, y)
	   - 1
	   - x (1 + y)
	   - \frac{1}{1 - x}
	   + 1
	   + x \\
      &= S(x, y) - x y - \frac{1}{1 - x} \\
    x y \sum_{n, k \ge 0} s(n + 1, k + 1) x^n y^k
      &= S(x, y)
	   - \sum_{k \ge 0} s(0, k) y^k
	   - \sum_{n \ge 0} s(n, 0) x^n
	   + s(0, 0) \\
      &= S(x, y)
	   - 1
	   - \frac{1}{1 - x}
	   + 1 \\
      &= S(x, y) - \frac{1}{1 - x}
  \end{align*}
  Los términos que se suman se han restado dos veces,
  y deben reponerse.
  Esto con la recurrencia da:
  \begin{equation*}
    \frac{S(x, y)
	    - x (1 + y)
	    - \frac{1}{1 - x}}
	 {x^2 y}
      = \frac{S(x, y) - \frac{1}{1 - x}}{x y}
	  + S(x, y)
  \end{equation*}
  Despejando:
  \begin{equation}
    \label{eq:GF-s}
    S(x, y)
      = \frac{1 + x y}{1 - x - x^2 y}
  \end{equation}
  Podemos escribir~\eqref{eq:GF-s} como:
  \begin{align*}
    S(x, y)
      &= \frac{1 + x y}{1 - x (1 + x y)} \\
      &= \sum_{r \ge 0} x^r (1 + x y)^{r + 1} \\
      &= \sum_{r \ge 0} x^r
	   \sum_{s \ge 0} \binom{r + 1}{s} \, x^s y^s \\
      &= \sum_{r, s \ge 0} \binom{r + 1}{s} \, x^{r + s} y^s
  \end{align*}
  De aquí:
  \begin{equation}
    \label{eq:s(n,k)}
    s(n, k)
      = \left[ x^n y^k \right] \, S(x, y)
      = \binom{n - k + 1}{k}
  \end{equation}

  Para \(y = 1\) la función generatriz~\eqref{eq:GF-s} da:
  \begin{equation*}
    S(x, 1)
      = \frac{1 + x}{1 - x - x^2}
      = \frac{F(x) - x}{x^2}
  \end{equation*}
  Acá \(F(x)\) es la función generatriz
  de los números de Fibonacci~\eqref{eq:gf-Fibonacci}.
  Esto concuerda con sumar la recurrencia~\eqref{eq:recurrence-s-1}
  sobre todo \(k\),
  como \(s(0, 0) = 1 = F_2\) y \(s(1, 0) + s(1, 1) = 2 = F_3\).

  Otro caso de interés son los números de Delannoy~%
    \cite{delannoy95:_quest_probab}
  (ver también la discusión de Banderier y Schwer~%
    \cite{banderier05:_why_delannoy_numbers}).
  Se define \(D(m, n)\)
  como el número de caminos entre \((0, 0)\) y \((m, n)\)
  en una cuadrícula,
  si se permiten únicamente pasos hacia el norte, el nordeste o este.
  De la definición es clara la recurrencia:
  \begin{equation}
    \label{eq:Delannoy-recurrence}
    D(m, n)
      = D(m - 1, n) + D(m, n - 1) + D(m - 1, n - 1)
    \qquad D(0, 0) = 1
  \end{equation}
  Para aplicar nuestra técnica requeriremos:
  \begin{equation}
    \label{eq:Delannoy-boundary}
    D(m, 0) = D(0, n) = 1
  \end{equation}
  Definiendo la función generatriz ordinaria:
  \begin{equation}
    \label{eq:Delannoy-GF-definition}
    d(x, y)
      = \sum_{m, n \ge 0} D(m, n) x^m y^n
  \end{equation}
  obtenemos la ecuación funcional:
  \begin{equation}
    \label{eq:Delannoy-FE}
    \frac{d(x, y) - d(0, y) - d(y, 0) + d(0, 0)}{x y}
      = \frac{d(x, y) - d(0, y)}{x} + \frac{d(x, y) - d(x, 0)}{y} + d(x, y)
  \end{equation}
  Las condiciones de contorno dan:
  \begin{equation}
    \label{eq:Delannoy-boundary-GF}
    d(0, y)
      = \frac{1}{1 - y}
    \hspace{3em}
    d(x, 0)
      = \frac{1}{1 - x}
  \end{equation}
  Substituyendo~\eqref{eq:Delannoy-boundary-GF} en~\eqref{eq:Delannoy-FE}
  y despejando \(d(x, y)\) resulta:
  \begin{equation}
    \label{eq:Delannoy-GF}
    d(x, y)
      = \frac{1}{1 - x - y - x y}
  \end{equation}
  El lector interesado verificará que expandir como serie geométrica
  y extraer los coeficientes respectivos resulta en:
  \begin{equation}
    \label{eq:Delannoy}
      D(m, n)
	= \sum_t \binom{m + n - t}{n} \, \binom{n}{t}
  \end{equation}
  La asimetría de~\eqref{eq:Delannoy}
  ofende las sensibilidades del autor.
  Puede escribirse en forma simétrica en términos de coeficientes trinomiales,
  eso sí.

  La siguiente idea da una expansión más simétrica:
  \begin{align*}
    d(x, y)
      &= \frac{1}{(1 - x) (1 - y) - 2 x y} \\
      &= \frac{(1 - x) (1 - y)}{1 - \frac{2 x y}{(1 - x) (1 - y)}} \\
      &= (1 - x) (1 - y)
	   \sum_{r \ge 0} \left( \frac{2 x y}{(1 - x) (1 - y)} \right)^r \\
      &= \sum_{r \ge 0}
	  \frac{2^r x^r y^r}{(1 - x)^{r - 1} \, (1 - y)^{r - 1}} \\
      &= \sum_{r \ge 0} 2^r \,
	   \sum_{s \ge 0} \binom{r + s}{s} \, x^{r + s}
	   \sum_{t \ge 0} \binom{r + t}{t} \, y^{r + t}
  \end{align*}
  Al extraer el coeficiente de \(x^m y^n\)
  solo sobreviven los términos con \(r + s = m\) y \(r + t = n\),
  aprovechando la simetría de los coeficientes binomiales:
  \begin{equation}
    \label{eq:Delannoy-alt}
    D(m, n)
      = \sum_{r \ge 0} 2^r \, \binom{m}{r} \, \binom{n}{r}
  \end{equation}
  En todo caso,
  ya habíamos demostrado esta identidad como~\eqref{eq:so:binomial-identity}
  usando aceite de serpiente.

%%% Local Variables:
%%% mode: latex
%%% TeX-master: "clases"
%%% End:


\section{Dividir y conquistar}
\label{sec:dividir-y-conquistar}

% Fixme: Agregar los algoritmos, eliminar comentarios abajo
% Fixme: Más ejemplos de algoritmos! Derivar alguno?

  Una de las estrategias más fructíferas para diseñar algoritmos
  es la que se llama \emph{dividir y conquistar}%
    \index{dividir y conquistar}
  (ver por ejemplo Cormen, Leiserson, Rivest y Stein~%
    \cite{cormen09:_introd_algor}).
  La idea es resolver un problema ``grande''
  por la vía de expresarlo
  en términos de varios problemas menores del mismo tipo,
  resolver estos (recursivamente)
  y luego combinar los resultados.
  Ejemplos típicos son el ordenamiento por intercalación%
    \index{ordenamiento!intercalacion@intercalación}
%  (ver algoritmo~\ref{alg:mergesort})
  y búsqueda binaria.%
    \index{busqueda binaria@búsqueda binaria}
%   (ver algoritmo~\ref{alg:busqueda-binaria}).

  Un ejemplo menos conocido
  es el algoritmo de Karatsuba
  para multiplicación de números enteros~%
    \cite{karatsuba62:_multiplication}.%
    \index{Karatsuba, algoritmo de}
%   (ver algoritmo~\ref{alg:Karatsuba}).
  Se desean multiplicar números de \(2 n\) dígitos,
  llamémosles \(A\) y \(B\),
  los dividimos en mitades más y menos significativas.
  Si la base es \(10\),
  escribimos:
  \begin{equation*}
    A = a \cdot 10^n + b
    \hspace{3em}
    B = c \cdot 10^n + d
  \end{equation*}
  donde \(0 \le a, b, c, d < 10^n\),
  y tenemos:
  \begin{equation*}
    A \cdot B
      = a c \cdot 10^{2 n}
	  + (a d + b c) \cdot 10^n
	  + b d
  \end{equation*}
  Esta fórmula permite calcular un producto de dos números
  de \(2 n\) dígitos
  mediante cuatro multiplicaciones de números de \(n\) dígitos
  (y algunas operaciones adicionales,
   como sumas de números de a lo más \(2 n\) dígitos).
  Si definimos:
  \begin{align*}
    u = a + b
    \hspace{3em}
    v = c + d
    \hspace{3em}
    u v
      = a c + a d + b c + b d
  \end{align*}
  podemos expresar:
  \begin{equation*}
    A \cdot B
      = a c \cdot 10^{2 n} + (u v - a c - b d) \cdot 10^n + b d
  \end{equation*}
  Esta fórmula significa usar tres
  (no cuatro)
  multiplicaciones,
  a costa de más operaciones de suma.
  Si comenzamos con números con \(2^n\) dígitos,
  podemos aplicar esta estrategia recursivamente,
  y los ahorros se suman.

  Un ejemplo lo da el producto \(23\,316\,384 \cdot 20\,936\,118\).
  Tenemos:
  \begin{align*}
    n &= 8 \\
    A &= 23\,316\,384  \\
    B &= 20\,936\,118 \\
    a &= 2\,331 \quad b = 6\,384
	 \qquad c = 2\,093 \quad d = 6\,118 \\
    u &= 8\,715 \quad v = 8\,211
  \end{align*}
  Debemos ahora calcular:
  \begin{align*}
    a c
      &= 2\,331 \cdot 2\,093 \\
      &= (23 \cdot 20) \cdot 10^4
	   + ((23 + 31) \cdot (20 + 93)
	   - 23 \cdot 20
	   - 31 \cdot 93) \cdot 10^2
	   + 31 \cdot 93 \\
      &= 460 \cdot 10^4 + 2\,759 \cdot 10^2 + 2\,883 \\
      &= 4\,878\,783
  \end{align*}
  En esto hemos calculado,
  por ejemplo:
  \begin{equation*}
    23 \cdot 20
      = 2 \cdot 2 \cdot 10^2
	  + ((2 + 3) \cdot (2 + 0)
	  - 2 \cdot 2
	  - 3 \cdot 0) \cdot 10
	  + 3 \cdot 0
      = 4 \cdot 10^2 + 6 \cdot 10 + 0
      = 460
  \end{equation*}
  Los otros valores intermedios a calcular son:
  \begin{equation*}
    b d
      = 39\,057\,312
    \hspace{3em}
    u v
      = 71\,558\,865
    \hspace{3em}
    u v - a c - b d
      = 27\,622\,770
  \end{equation*}
  Combinando los anteriores queda finalmente:
  \begin{align*}
     23\,316\,384 \cdot 20\,936\,118
       &= 4\,878\,783 \cdot 10^8
	    + 27\,622\,770 \cdot 10^4
	    + 39\,057\,312 \\
       &= 488\,154\,566\,757\,312
  \end{align*}
%   \begin{algorithm}[htbp]
%     \caption{Ordenamiento por intercalación}
%     \label{alg:mergesort}
%   \end{algorithm}
%   \begin{algorithm}[htbp]
%     \caption{Búsqueda binaria}
%     \label{alg:busqueda-binaria}
%   \end{algorithm}
%   \begin{algorithm}[htbp]
%     \caption{Multiplicación eficiente}
%     \label{alg:Karatsuba}
%   \end{algorithm}

  Otro ejemplo de esta estrategia es el algoritmo de Strassen~%
    \cite{strassen69:_matrix_multiplication}%
    \index{Strassen, multiplicacion de@Strassen, multiplicación de}
%   (ver algoritmo~\ref{alg:Strassen})%
  para multiplicar matrices.
  Consideremos primeramente
  el producto de dos matrices de \(2 \times 2\):
  \begin{equation*}
    \begin{pmatrix}
      c_{1 1} & c_{1 2} \\
      c_{2 1} & c_{2 2}
    \end{pmatrix}
      = \begin{pmatrix}
	  a_{1 1} & a_{1 2} \\
	  a_{2 1} & a_{2 2}
	\end{pmatrix}
	  \cdot
	    \begin{pmatrix}
	      b_{1 1} & b_{1 2} \\
	      b_{2 1} & b_{2 2}
	    \end{pmatrix}
  \end{equation*}
  Sabemos que:
  \begin{equation*}
    \begin{array}{l@{\qquad}l}
      c_{1 1}
	= a_{1 1} b_{1 1} + a_{1 2} b_{2 1} &
      c_{1 2}
	= a_{1 1} b_{1 2} + a_{1 2} b_{2 2} \\
      c_{2 1}
	= a_{2 1} b_{1 1} + a_{2 2} b_{2 1} &
      c_{2 2}
	= a_{2 1} b_{1 2} + a_{2 2} b_{2 2}
    \end{array}
  \end{equation*}
  Esto corresponde a \(8\) multiplicaciones.
  Definamos los siguientes productos:
  \begin{equation*}
    \begin{array}{l@{\qquad}l}
      m_1
	= (a_{1 1} + a_{2 2}) \, (b_{1 1} + b_{2 2}) &
      m_2
       = (a_{2 1} + a_{2 2}) \, b_{1 1} \\
      m_3
       = a_{1 1} \, (b_{1 2} - b_{2 2}) &
      m_4
       = a_{2 2} \, (b_{2 1} - b_{1 1}) \\
      m_5
       = (a_{1 1} + a_{1 2}) \, b_{2 2} &
      m_6
       = (a_{2 1} - a_{1 1}) \, (b_{1 1} + b_{1 2}) \\
      m_7
       = (a_{1 2} - a_{2 2}) \, (b_{2 1} + b_{2 2})
    \end{array}
  \end{equation*}
  Entonces podemos expresar:
  \begin{align*}
    \begin{array}{l@{\qquad}l}
      c_{1 1}
	= m_1 + m_4 - m_5 + m_7 &
      c_{1 2}
	= m_3 + m_5 \\
      c_{2 1}
	= m_2 + m_4 &
      c_{2 2}
	= m_1 - m_2 + m_3 + m_6
    \end{array}
  \end{align*}
  Con estas fórmulas se usan \(7\) multiplicaciones
  para evaluar el producto de dos matrices.
  Cabe hacer notar que estas fórmulas
  no hacen uso de conmutatividad,
  por lo que son aplicables también
  para multiplicar matrices de \(2 \times 2\)
  cuyos elementos son a su vez matrices.
  Podemos usar esta fórmula recursivamente
  para multiplicar matrices de \(2^n \times 2^n\).
%   \begin{algorithm}[htbp]
%     \caption{Multiplicación eficiente de matrices}
%     \label{alg:Strassen}
%   \end{algorithm}

  Tal vez el algoritmo más importante
  basado en dividir y conquistar
  es el que se conoce
  como \emph{transformada rápida de Fourier},%
    \index{Fourier, transformada rapida de@Fourier, transformada rápida de}%
    \index{FFT|see{Fourier, transformada rápida de}}
  generalmente abreviado \emph{FFT}
  (de \emph{\foreignlanguage{english}{Fast Fourier Transform}} en inglés);
  se acredita a Cooley y~Tukey~%
    \cite{cooley65:_FFT}
  (aunque para variar un poco,
   más de dos siglos antes Gauß%
     \index{Gauss, Carl Friedrich@Gauß, Carl Friedrich}
   ya lo empleaba,
   como relatan Heideman, Johnson y Burrus~%
     \cite{heideman84:_gauss_history_FFT}).
  Elegido
  como uno de los \(10\) algoritmos más importantes del siglo~XX~%
    \cite{dongarra00:_top10_algorithms},
  es la base de mucho de lo que es procesamiento de señales hoy día,
  es el corazón del algoritmo de Schönhage y Strassen,%
    \index{Schonhage y Strassen, algoritmo de@Schönhage y Strassen, algoritmo de}
  el mejor algoritmo conocido para multiplicar que resulta práctico
  para números muy grandes~%
    \cite{schoehage71:_schnel_multip_zahlen},
  y es central en el algoritmo de Fürer~%
    \cite{fuerer07:_faster_integ_multip},
  el mejor que se conoce
  (aunque este último solo sería ventajoso
   para números fuera del rango útil)

  Otro ejemplo clásico es el algoritmo Quicksort%
    \index{ordenamiento!quicksort}%
    \index{quicksort|see{ordenamiento}}%
  (ver la sección~\ref{sec:quicksort}),
  claro que en este la división no es equitativa
  (como en los otros que se mencionan).
  También fue considerado
  uno de los \(10\) algoritmos más importantes del siglo~XX~%
    \cite{dongarra00:_top10_algorithms}.

\subsection{Análisis de división fija}
\label{sec:d&c:division-fija}

  Consideraremos primero el caso en que el problema original
  se traduce en varios problemas
  de una fracción fija del tamaño del original.
  Si el tiempo de ejecución de un algoritmo de este tipo
  para una entrada de tamaño \(n\)
  lo denotamos por \(t(n)\),
  el problema se reduce a \(a\) problemas de tamaño \(n / b\),
  y el costo de reducir el problema y luego combinar las soluciones
  es \(f(n)\),
  al sumar el tiempo para resolver los subproblemas
  y las otras operaciones
  obtendremos recurrencias de la forma:
    \index{dividir y conquistar!recurrencia}
  \begin{equation*}
    t(n) = a \, t(n / b) + f(n) \qquad t(1) = t_1
  \end{equation*}
  El restringir el análisis a potencias de \(b\)
  es válido ya que interesa el comportamiento asintótico
  de la solución a la recurrencia.
  Intuitivamente es claro que los algoritmos considerados
  se ejecutan en un tiempo intermedio para tamaños intermedios,
  y en cualquier caso podemos ``rellenar''
  los datos hasta completar la potencia respectiva.
  Hacer esto no cambia nuestras conclusiones más abajo.

  En el caso de ordenamiento por intercalación,%
    \index{ordenamiento!intercalacion@intercalación}
  dividimos en dos partes iguales
  que se procesan recursivamente.
  El proceso de dividir
  puede implementarse vía tomar elementos alternativos
  y ubicarlos en grupos separados,
  el combinar las partes ordenadas
  toma tiempo proporcional a su tamaño.
  Por lo tanto,
  el crear los subproblemas y combinar sus soluciones
  toma un tiempo proporcional al número de elementos a ordenar.
  Así tenemos que
  \(a = b = 2\), \(f(n) = c n\) para alguna constante \(c\).
  Para búsqueda binaria,%
    \index{busqueda binaria@búsqueda binaria}
  se divide en dos partes iguales
  de las cuales se procesa recursivamente solo una,
  y el proceso de división es simplemente ubicar el elemento medio
  y comparar con él,
  y no hay combinación de subproblemas;
  todo esto toma un tiempo constante.
  En este caso es
  \(a = 1\), \(b = 2\), \(f(n) = c\) para alguna constante.
  En el algoritmo de Karatsuba%
    \index{Karatsuba, algoritmo de}
  se transforma la multiplicación de dos números de largo \(2 n\)
  en \(3\) multiplicaciones de números de \(n\) dígitos,
  las tareas adicionales son dividir los números en mitades
  y efectuar varias sumas y restas de números de \(2 n\) dígitos,
  y finalmente juntar las piezas.
  El costo de estas operaciones es simplemente proporcional a \(n\).
  Resulta \(a = 3\), \(b = 2\) y \(f(n) = c n\).
  En la multiplicación de matrices de Strassen%
    \index{Strassen, algoritmo de}
  el multiplicar matrices de \(2 n \times 2 n\)
  se traduce en \(7\) multiplicaciones de matrices de \(n \times n\)
  y algunas sumas de matrices.
  Tenemos entonces \(a = 7\),
  \(b = 2\),
  y las operaciones adicionales son básicamente sumas de matrices,
  lo que da \(f(n) = c n^2\).
  Para cubrir el patio de \(2^n \times 2^n\)
  de la Universidad de Miskatonic con losas en L%
    \index{pavimentacion@pavimentación}
  que vimos al discutir inducción fuerte
  (sección~\ref{sec:induccion-fuerte}),
  la demostración que dimos reduce el problema de \(2^n \times 2^n\)
  a 4~problemas de \(2^{n - 1} \times 2^{n - 1}\)
  haciendo una cantidad fija de trabajo,
  lo que hace \(a = 4\), \(b = 2\), \(d = 0\).
  Si aprovechamos simetrías
  con el cuadradito a cubrir siempre en una esquina,
  es un solo trabajo menor,
  con lo que \(a = 1\), \(b = 2\), \(d = 0\).
  En caso que la posición de August es arbitraria,
  hay 2~tipos de subproblemas
  (uno con el espacio libre en la esquina,
   el otro con el espacio para August en una posición arbitraria),
  y \(a = 2\), \(b = 2\), \(d = 0\).

  Estos ejemplos son bastante representativos.
  El análisis es simple si \(f(n) = c n^d\).
  Para búsqueda binaria tenemos \(d = 0\),
  para ordenamiento por intercalación%
    \index{ordenamiento!intercalacion@intercalación}
  y en Karatsuba \(d = 1\),
  Strassen da \(d = 2\).
  El cuadro~\ref{tab:dividir-conquistar} resume los parámetros
  para los algoritmos dados.

  Consideramos entonces la recurrencia,
  válida para \(n\) una potencia de \(b\):
  \begin{equation*}
    t(b n) = a t(n) + c n^d \qquad t(1) = t_1
  \end{equation*}

  Efectuamos el cambio de variables:
  \begin{equation*}
    \begin{array}{l@{\quad}l}
      n	   = b^k  & k	   = \log_b n \\
      t(n) = T(k) & t(b n) = T(k + 1)
    \end{array}
  \end{equation*}
  En estos términos,
  dadas las condiciones del problema
  para constantes \(c > 0\)
  (el costo de dividir y combinar no es nulo)
  y \(t_1 > 0\)
  (el resolver un problema de tamaño mínimo tiene algún costo)
  tenemos para \(k \ge 0\):
  \begin{equation*}
    T(k + 1) = a T(k) + c (b^d)^k \qquad T(0) = t_1
  \end{equation*}
  Para resolver la recurrencia
  definimos la función generatriz:
  \begin{equation*}
    g(z) = \sum_{k \ge 0} T(k) z^k
  \end{equation*}
  y aplicamos nuestra técnica a la recurrencia lineal resultante:
  \begin{align*}
    \frac{g(z) - t_1}{z}
      &= g(z) + c \, \frac{1}{1 - b^d z} \\
    g(z)
      &= \frac{t_1 - (b^d t_1 - c) z}{(1 - b^d z) (1 - a z)}
  \end{align*}
  Si \(a \ne b^d\):
  \begin{equation}
    \label{eq:DaC-ne}
    g(z)
      = \frac{c}{b^d - a} \cdot \frac{1}{1 - b^d z}
	  + \frac{(b^d - a) t_1 - c}{b^d - a}
	      \cdot \frac{1}{1 - a z}
  \end{equation}
  Cuando \(a = b^d\):
  \begin{equation}
    \label{eq:DaC-eq}
    g(z)
      = \frac{c}{a} \cdot \frac{1}{(1 - a z)^2}
	  + \frac{a t_1 - c}{a} \cdot \frac{1}{1 - a z}
  \end{equation}
  El comportamiento asintótico
  queda determinado por \(a\) y \(b^d\).
  Si \(a > b^d\),
  domina el segundo término de~\eqref{eq:DaC-ne}:
  \begin{equation}
    \label{eq:DaC-gt-asymp}
    T(k)
      \sim \left( t_1 + \frac{c}{a - b^d} \right) \cdot a^k
  \end{equation}
  Si \(a < b^d\),
  es el primer término de~\eqref{eq:DaC-ne} el dominante:
  \begin{equation}
    \label{eq:DaC-lt-asymp}
    T(k)
      \sim \frac{c}{b^d - a} \cdot b^{k d}
  \end{equation}
  En caso que \(a = b^d\) debemos recurrir a~\eqref{eq:DaC-eq},
  y es dominante el primer término:
  \begin{equation}
    \label{eq:DaC-eq-asymp}
    T(k)
      \sim \frac{c}{a} \cdot k a^k
  \end{equation}
  Las constantes indicadas son siempre diferentes de cero.

  En términos de las variables originales,
  es \(k = \log_b n\)
  y \(a^k = a^{\log_b n}
	  = b^{\log_b a \cdot \log_b n}
	  = n^{\log_b a}\):%
    \index{analisis de algoritmos@análisis de algoritmos!dividir y conquistar!asintotica@asintótica}
  \begin{equation*}
    t(n)
      \sim
      \begin{cases}
	\left( t_1 + \frac{c}{a - b^d} \right) \cdot n^{\log_b a}
	   & \text{si \(a > b^d\)} \\
	\frac{c}{a} \, n^{\log_b a} \log n)
	   & \text{si \(a = b^d\)} \\
	\frac{c}{b^d - a} \cdot n^d
	   & \text{si \(a < b^d\)}
      \end{cases}
  \end{equation*}
  Para los algoritmos que describimos
  tenemos las complejidades
  resumidas en el cuadro~\ref{tab:dividir-conquistar}.%
    \index{analisis de algoritmos@análisis de algoritmos!dividir y conquistar}
  \begin{table}[htbp]
    \centering
    \begin{tabular}{|l|*{3}{>{\(}r<{\)}|}>{\(}l<{\)}|}
      \hline
      \multicolumn{1}{|c|}
	 {\rule[-0.7ex]{0pt}{3ex}\textbf{Nombre}} &
	\multicolumn{1}{c|}{\(\boldsymbol{a}\)} &
	\multicolumn{1}{c|}{\(\boldsymbol{b}\)} &
	\multicolumn{1}{c|}{\(\boldsymbol{d}\)} &
	\multicolumn{1}{c|}{\textbf{Complejidad}} \\
      \hline\rule[-0.7ex]{0pt}{3.5ex}%
      Búsqueda binaria		     & 1 & 2 & 0
	& O(\log n)	      \\
      \hline\rule[-0.7ex]{0pt}{3.5ex}%
      Ordenamiento por intercalación & 2 & 2 & 1
	& O(n \log n)	      \\
      \hline\rule[-0.7ex]{0pt}{3.5ex}%
      Karatsuba			     & 3 & 2 & 1
	& O \left( n^{\log_2 3} \right) \\
      \hline\rule[-0.7ex]{0pt}{3.5ex}%
      \multirow{3}*{Pavimentación}   & 4 & 2 & 0
	& O \left( n^2 \right) \\
				     & 2 & 2 & 0
	& O( n) \\
				     & 1 & 2 & 0
	& O( n) \\
      \hline\rule[-0.9ex]{0pt}{3.5ex}%
      Strassen			     & 7 & 2 & 2
	& O \left( n^{\log_2 7} \right) \\
      \hline
    \end{tabular}
    \caption{Complejidad de algunos algoritmos}
    \label{tab:dividir-conquistar}
  \end{table}

  Este tipo de recurrencias puede resolverse exactamente.
  Por ejemplo,
  Sedgewick y Flajolet~\cite{sedgewick13:_introd_anal_algor}
  muestran que la solución a la recurrencia
  para el número de comparaciones
  en mergesort de \(n\) elementos diferentes:
    \index{ordenamiento!intercalacion@intercalación!analisis@análisis}
  \begin{equation}
    \label{eq:mergesort-compares-exact}
    C_n
      = C_{\lfloor n / 2 \rfloor} + C_{\lceil n / 2 \rceil} + n
    \qquad C_1 = 0
  \end{equation}
  es:
  \begin{equation}
    \label{eq:mergesort-compares-exact-solution}
    C_n
      = n \log_2 n + n \theta(1 - \{ \log_2 n \})
  \end{equation}
  donde:
  \begin{equation}
    \label{eq:mergesort-theta}
    \theta(x)
      = 1 + x - 2^x
  \end{equation}
  Resulta \(\theta(0) = \theta(1) = 0\)
  y \(0 < \theta(x) < 0,086\)
  para \(0 < x < 1\).
  Esta clase de comportamiento ``periódico''
  complica el análisis preciso de muchos algoritmos.

  Un desarrollo didáctico de resultados de este tipo
  se encuentra en el texto de Stein, Drysdale y Bogarth~%
    \cite[apéndice A]{stein10:_discr_mathem_comput_scien}.
  Una visión alternativa,
  incluyendo técnicas
  para acotar el caso de funciones forzantes diferentes,
  dan Bentley, Haken y Saxe~%
    \cite{bentley80:_gener_method_solvin_divid_conquer_recur}.
  Una extensión a estos resultados
  es el teorema de Akra-Bazzi~\cite{akra98:_sol_lin_recurr_eqs}.
  Leighton~\cite{leighton96:_notes_better_master_theo}
  da la variante que reseñamos,
  extensiones interesantes da Roura~%
     \cite{roura01:_improv_master_theor_divid_conquer_recur},
  soluciones más precisas para versiones discretas
  (con techos/pisos)
  ofrecen Drmota y Szpankowski~%
     \cite{Drmota:2011:MTD:2133036.2133064}.
  \begin{theorem}[Akra-Bazzi]
    \index{Akra-Bazzi, teorema de|textbfhy}
    \label{theo:Akra-Bazzi}
    Sea una recurrencia de la forma:
    \begin{equation*}
      T(z)
	= g(z) + \sum_{1 \le k \le n} a_k T(b_k z + h_k(z))
	   \quad \text{para \(z \ge z_0\)}
    \end{equation*}
    donde \(z_0\), \(a_k\) y \(b_k\) son constantes,
    sujeta a las siguientes condiciones:
    \begin{itemize}
    \item
      Hay suficientes casos base.
    \item
      Para todo \(k\) se cumplen \(a_k > 0\) y \(0 < b_k < 1\).
    \item
      Hay una constante \(c\)
      tal que \(\lvert g(z) \rvert = O(z^c)\).
    \item
      Para todo \(k\)
      se cumple \(\lvert h_k(z) \rvert = O(z / (\log z)^2)\).
    \end{itemize}
    Entonces,
    si \(p\) es tal que:
    \begin{equation*}
      \sum_{1 \le k \le n} a_k b_k^p
	= 1
    \end{equation*}
    la solución a la recurrencia cumple:
    \begin{equation*}
      T(z)
	= \Theta
	    \left(
	      z^p \left(
		     1 + \int_1^z \frac{g(u)}{u^{p + 1}}
			   \, \mathrm{d} u
		  \right)
	    \right)
    \end{equation*}
  \end{theorem}
  Frente a nuestro tratamiento tiene la ventaja
  de manejar divisiones desiguales
  (\(b_k\) diferentes),
  y explícitamente
  considera pequeñas perturbaciones en los términos,
  como lo son aplicar pisos o techos,
  a través de los \(h_k(z)\).
  Diferencias con pisos y techos están acotados por una constante,
  mientras la cota del teorema permite que crezcan.
  Por ejemplo,
  la recurrencia correcta para el número de comparaciones
  en ordenamiento por intercalación es:
  \begin{equation*}
    T(n)
      = T(\lfloor n / 2 \rfloor) + T(\lceil n / 2 \rceil) + n - 1
  \end{equation*}
  El teorema de Akra-Bazzi es aplicable.
  La recurrencia es:
  \begin{equation*}
    T(n)
      = T(n / 2 + h_{+}(n)) + T(n / 2 + h_{-}(n)) + n - 1
  \end{equation*}
  Acá \(\lvert h_{\pm}(n) \rvert \le 1/2\),
  además \(a_{\pm} = 1\) y \(b_{\pm} = 1/2\).
  Estos cumplen las condiciones del teorema,
  de:
  \begin{equation*}
    \sum_{1 \le k \le 2} a_k b_k^p = 1
  \end{equation*}
  resulta \(p = 1\),
  y tenemos la cota:
  \begin{equation*}
    T(z)
      = \Theta
	  \left(
	    z \left(
	  1 + \int_1^z \frac{u - 1}{u^2} \, \mathrm{d} u
	      \right)
	  \right)
      = \Theta
	  \left(
	    z \ln z + 1
	  \right)
      = \Theta(z \log z)
  \end{equation*}

  Otro ejemplo son los árboles de búsqueda aleatorizados
  (\emph{\foreignlanguage{english}{Randomized Search Trees}},
   ver por ejemplo Aragon y Seidel~%
     \cite{aragon89:_random_search_tree},
   Martínez y Roura~%
     \cite{martinez98:_random_binar_searc_trees}
   y Seidel y Aragon~%
     \cite{seidel96:_random_search_trees})
  en uno de ellos de tamaño~\(n\)
  una búsqueda toma tiempo aproximado:
  \begin{equation*}
    T(n)
      = \frac{1}{4} \, T(n / 4) + \frac{3}{4} \, T(3 n / 4) + 1
  \end{equation*}
  Nuevamente es aplicable el teorema~\ref{theo:Akra-Bazzi},
  de:
  \begin{equation*}
    \frac{1}{4} \, \left(\frac{1}{4}\right)^p
	+ \frac{3}{4} \left(\frac{3}{4}\right)^p
      = 1
  \end{equation*}
  obtenemos \(p = 0\),
  y por tanto la cota
  \begin{equation*}
    T(z)
      = \Theta \left(
	  z^0 \left( 1 + \int_1^z \frac{\mathrm{d} u}{u} \right)
	\right)
      = \Theta ( \log z )
  \end{equation*}

% quicksort.tex
%
% Copyright (c) 2010-2014 Horst H. von Brand
% Derechos reservados. Vea COPYRIGHT para detalles

\subsection{Quicksort}
\label{sec:quicksort}
\index{ordenamiento!quicksort|textbfhy}
\index{analisis de algoritmos@análisis de algoritmos!ordenamiento!quicksort}

  Quicksort,
  debido a Hoare~\cite{hoare62:_quicksort},
  es otro algoritmo basado en dividir y conquistar,
  pero en este caso la división no es fija.
  Dado un rango de elementos de un arreglo a ser ordenado,
  se elige un elemento \emph{pivote} de entre ellos
  y se reorganizan los elementos en el rango
  de forma que todos los elementos menores que el pivote
  queden antes de este,
  y todos los elementos mayores queden después.
  \begin{figure}[htbp]
    \centering
    \pgfimage{images/qsort-idea}
    \caption{Idea de Quicksort}
    \label{fig:qsort:idea}
  \end{figure}
  Con esto el pivote ocupa su posición final en el arreglo,
  y bastará ordenar recursivamente cada uno
  de los dos nuevos rangos generados
  para completar el trabajo.
  \begin{figure}[htbp]
    \centering
    \pgfimage{images/qsort-particionar}
    \caption{Particionamiento en Quicksort}
    \label{fig:qsort:particionamiento}
  \end{figure}
  La figura~\ref{fig:qsort:particionamiento} indica una manera popular
  de efectuar esta \emph{partición}:
  Se elige un pivote de forma aleatoria
  y el pivote elegido se intercambia con el primer elemento del rango
  (para sacarlo de en medio),
  luego se busca un elemento mayor que el pivote desde la izquierda
  y uno menor desde la derecha.
  Estos están fuera de orden,
  se intercambian y se continúa de la misma forma hasta agotar el rango.
  Después se repone el pivote en su lugar,
  intercambiándolo con el último elemento menor que él.
  El rango finalmente queda como indica la figura~\ref{fig:qsort:idea}.
  El listado~\ref{lst:quicksort} muestra una versión simple del programa,
  que elige siempre el primer elemento del rango como pivote.
  \lstinputlisting[language=C,
		   float,
		   caption={Versión simple de Quicksort},
		   label=lst:quicksort]
		   {code/quicksort.c}
% LaTeX bug: Leaving out the "float, ..." line out gives a crash
  Evaluaremos el tiempo promedio de ejecución del algoritmo.%
    \index{quicksort!analisis@análisis}
  Supondremos~\(n\) elementos todos diferentes,
  que las \(n!\)~permutaciones de los \(n\)~elementos
  son igualmente probables,
  y que el pivote se elige al azar en cada etapa.
  En este caso está claro que el método de particionamiento planteado
  no altera el orden de los elementos en las particiones
  respecto del orden que tenían originalmente.
  Luego,
  los elementos en cada partición también son una permutación al azar.

  Para efectos del análisis del algoritmo
  tomaremos como medida de costo el número promedio de comparaciones
  que efectúa Quicksort al ordenar un arreglo de \(n\)~elementos.
  El trabajo adicional que se hace en cada partición
  será aproximadamente proporcional a esto,
  por lo que esta es una buena vara de medida.
  Al particionar,
  cada uno de los \(n - 1\) elementos fuera del pivote
  se comparan con este exactamente una vez en el método planteado,
  y además es obvio que este es el mínimo número de comparaciones necesario
  para hacer este trabajo.
  Si llamamos \(k\) a la posición final del pivote,
  el costo de las llamadas recursivas que completan el ordenamiento
  será \(C(k - 1) + C(n - k)\).
  Si elegimos el pivote al azar
  la probabilidad de que \(k\) tenga un valor cualquiera entre~\(1\)
  y~\(n\) es la misma.
  Cuando el rango es vacío no se efectúan comparaciones.
  Estas consideraciones llevan a la recurrencia:
  \begin{equation*}
    C(n)
       =  n - 1 +
	   \frac{1}{n} \, \sum_{1 \le k \le n - 1}
	      \left(C(k - 1) + C(n - k)\right) \quad C(0)  = 0
  \end{equation*}
  Por simetría podemos simplificar la suma,
  dado que estamos sumando los mismos términos
  en orden creciente y decreciente.
  Cambiando el rango de la suma y multiplicando por \(n\) queda:
  \begin{equation*}
    n C(n)
      = n (n - 1) + 2 \sum_{0 \le k \le n - 1} C(k)
  \end{equation*}
  Ajustando los índices:
  \begin{equation*}
    (n + 1) C(n + 1)
      = n (n + 1) + 2 \sum_{0 \le k \le n} C(k) \quad C(0) = 0
  \end{equation*}

  Definimos la función generatriz ordinaria:
  \begin{equation*}
    c(z)
      = \sum_{n \ge 0} C(n) z^n
  \end{equation*}
  Aplicando las propiedades de funciones generatrices ordinarias
  a la recurrencia
  queda la ecuación diferencial:
  \begin{align*}
    \left( z \mathrm{D} + 1 \right) \, \frac{c(z)}{z}
      &= \left( (z \mathrm{D})^2 + z D \right) \, \frac{1}{1 - z}
	   + \frac{2 c(z)}{1 - z}
	   \qquad c(0) = 0 \\
    c'(z)
      &= \frac{2 c(z)}{1 - z} + \frac{2 z}{(1 - z)^3}
  \end{align*}
  La solución a esta ecuación es:
  \begin{equation*}
    c(z)
      = - 2 \, \frac{\ln (1 - z)}{(1 - z)^2}
	   - \frac{2 z}{(1 - z)^2}
  \end{equation*}
  El primer término corresponde
  a la suma parcial de la secuencia de números harmónicos
  (derivamos su función generatriz
   en la sección~\ref{sec:numeros-harmonicos}),
  el segundo término da un coeficiente binomial:
  \begin{align*}
    C(n)
      &= 2 \sum_{0 \le k \le n} H_k - 2 \, \binom{n}{1} \\
      &= 2 \sum_{0 \le k \le n} H_k - 2 n
  \end{align*}
  Interesa obtener una fórmula más simple
  para la suma de los números harmónicos.
  Por la fórmula para la función generatriz de las sumas parciales:
  \begin{equation*}
    H(z)
      = \sum_{n \ge 0} H_n z^n
      = \frac{1}{1 - z} \, \ln \frac{1}{1 - z}
  \end{equation*}
  con lo que la función generatriz de las sumas de números harmónicos es:
  \begin{equation}
    \label{eq:Hn-sum-gf}
    \frac{H(z)}{1 - z}
      = \frac{1}{(1 - z)^2} \, \ln \frac{1}{1 - z}
  \end{equation}
  Esto se parece a la derivada de \(H(z)\):
  \begin{equation*}
    H'(z)
      = \frac{1}{(1 - z)^2} \, \ln \frac{1}{1 - z} + \frac{1}{(1 - z)^2}
  \end{equation*}
  Pero sabemos que \(H'(z)\) es la función generatriz
  de la secuencia \(\left\langle (n + 1) H_{n + 1} \right\rangle_{n \ge 0}\),
  mientras \((1 - z)^{-2}\) corresponde simplemente
  a \(\left\langle n + 1 \right\rangle_{n \ge 0}\).
  De esta forma tenemos:
  \begin{align}
    \sum_{0 \le k \le n} H_k
      &= (n + 1) H_{n + 1} - (n + 1)
	   \notag \\
      &= (n + 1) H_n - n
	   \label{eq:Hn-sum}
  \end{align}
  Esto da finalmente:
  \begin{equation*}
    C(n)
      = 2 (n + 1) H_n - 4 n
  \end{equation*}
  Vimos en el capítulo~\ref{cha:euler-maclaurin}%
    \index{numeros harmonicos@números harmónicos!aproximacion@aproximación}
  que \(H_n = \ln n + O(1)\),
  con lo que \(C(n) = 2 n \ln n + O(n)\).

  Pero podemos hacer más.
  En el peor caso,%
    \index{quicksort!analisis@análisis!peor caso}
  al particionar en cada paso elegimos uno de los elementos extremos,
  con lo que las particiones son de largo 0 y \(n - 1\),
  lo que da lugar a la recurrencia:
  \begin{equation*}
    C_{\text{peor}}(n)
      = n - 1 + C_{\text{peor}}(n - 1) \quad C_{\text{peor}}(0) = 0
  \end{equation*}
  Las técnicas estándar dan como solución:
  \begin{align*}
    C_{\text{peor}}(n)
      &= \frac{n (n - 1)}{2} \\
      &= \frac{1}{2} n^2 + O(n)
  \end{align*}
  El mejor caso es cuando en cada paso la división es equitativa,
  lo que lleva casi a la situación de dividir y conquistar
  analizada antes
  (sección~\ref{sec:d&c:division-fija}),
  con \(a = 2\), \(b = 2\) y \(d = 1\),
  cuya solución sabemos es \(C_{\text{mejor}}(n) = O(n \log n)\).
  Un análisis más detallado,
  restringido al caso en que \(n = 2^k - 1\)
  de manera que los dos rangos siempre resulten del mismo largo,
  es como sigue.%
    \index{quicksort!analisis@análisis!mejor caso}
  La recurrencia original se reduce a:
  \begin{equation*}
    C_{\text{mejor}}(n)
      = n - 1 + 2 C_{\text{mejor}}((n - 1) / 2)
      \quad C_{\text{mejor}}(0) = 0
  \end{equation*}
  Con el cambio de variables:
  \begin{equation*}
    n = 2^k - 1 \quad F(k) = C_{\text{mejor}}(2^k - 1)
  \end{equation*}
  esto se transforma en:
  \begin{equation*}
    F(k)
      = 2^k - 2 + 2 F(k - 1) \quad F(0) = 0
  \end{equation*}
  cuya solución es:
  \begin{align*}
    F(k)
      &= k 2^k + 2^{k + 1} + 2 \\
    C_{\text{mejor}}(n)
      &= (n + 1) \log_2 (n + 1) + 2 (n + 1) + 2 \\
      &= \frac{1}{\ln 2} \, n \ln n + O(n)
  \end{align*}
  La constante en este caso es aproximadamente \(1,443\),
  el mejor caso no es demasiado mejor que el promedio;
  pero el peor caso es mucho peor.

  Una variante común
  es usar un método de ordenamiento simple para rangos chicos.
  Una opción es cortar la recursión
  no cuando el rango se reduce a un único elemento
  sino cuando cae bajo un cierto margen;
  y luego se ordena todo mediante inserción,
  que funciona muy bien si los datos vienen ``casi ordenados'',
  como resulta de lo anterior.
  Para analizar esto se requieren medidas más ajustadas
  del costo de los métodos,
  y se cambian las condiciones de forma que para valores de \(n\)
  menor que el límite se usa el costo del método alternativo.
  Esto puede hacerse,
  pero es bastante engorroso y no lo veremos acá.

  Para evitar el peor caso
  (que se da cuando el pivote es uno de los elementos extremos)
  una opción es tomar una muestra de elementos y usar la mediana
  (el elemento del medio de la muestra)
  como pivote.
  La forma más simple de hacer esto es tomar tres elementos.
  Como además es frecuente que se invoque el procedimiento con un arreglo
  ``casi ordenado''
  (o incluso ya ordenado),
  conviene tomar como muestra el primero,
  el último
  y un elemento del centro,
  de forma de elegir un buen pivote incluso en ese caso patológico.
  A esta idea se le conoce como \emph{mediana de tres}.
  Esta estrategia disminuye un tanto la constante
  por efecto de una división más equitativa.
  Tiene la ventaja adicional
  que tener elementos menor que el pivote al comienzo del rango
  y mayor al final
  no es necesario comparar índices para determinar
  si se llegó al borde del rango.
  El análisis detallado se encuentra por ejemplo en Sedgewick y Flajolet~%
    \cite{sedgewick13:_introd_anal_algor}.

  Por el otro lado,
  McIllroy~\cite{mcillroy99:_killer_adver_quicksort}
  muestra cómo lograr que siempre tome el máximo tiempo posible.
  Quicksort
  (haciendo honor a su nombre)
  es muy rápido
  ya que las operaciones en sus ciclos internos
  implican únicamente una comparación
  y un incremento o decremento de un índice.
  Es ampliamente usado,
  y como su peor caso es muy malo,
  vale la pena hacer un estudio detallado de la ingeniería del algoritmo,
  como hacen Bentley y McIllroy~%
    \cite{bentley93:_engin_sort_funct}.
  Debe tenerse cuidado con Quicksort por su peor caso,
  si un atacante puede determinar los datos
  puede hacer que el algoritmo consuma muchísimos recursos.
  Para evitar el peor caso se ha propuesto cambiar a Heapsort,%
    \index{ordenamiento!heapsort}
    debido a Williams~%
    \cite{williams64:_alg_heapsort}
  (garantizadamente \(O(n \log n)\),
   pero mucho más lento que Quicksort)
  si se detecta un caso malo,
  como propone Musser~\cite{musser97:_introsort}.

%%% Local Variables:
%%% mode: latex
%%% TeX-master: "clases"
%%% End:


%%% Local Variables:
%%% mode: latex
%%% TeX-master: "clases"
%%% End:


% recurrencias.tex
%
% Copyright (c) 2013-2015 Horst H. von Brand
% Derechos reservados. Vea COPYRIGHT para detalles

\chapter{Recurrencias}
\label{sec:recurrencias}
\index{recurrencia|textbfhy}

  Es común encontrarse con situaciones
  en las cuales debemos resolver alguna \emph{recurrencia},
  vale decir,
  tenemos una ecuación que relaciona valores de una secuencia
  (generalmente adicionando algunos valores iniciales).
  Esto aparece tanto en la solución de problemas combinatorios
  como en el análisis de diversos algoritmos.
  Algunas de las técnicas que discutiremos
  se desarrollaron inicialmente
  para su aplicación en campos diversos como la economía
  o el control de procesos.
  Quienes hayan profundizado en el estudio
  de la solución de ecuaciones diferenciales
  hallarán paralelos sorprendentes
  (y divergencias importantes)
  con esa área.
  No disponemos de espacio
  para estudiar ese fenómeno en más detalle
  (ni es de nuestro interés inmediato).

\section{Definición del problema}
\label{sec:definicion-problema-recurrencia}

  La situación general puede describirse:
  \begin{equation}
    \label{eq:recurrence-general}
    f(a_n, a_{n + 1}, \dotsc, a_{n + k}, n)
      = 0
  \end{equation}
  Si en~\eqref{eq:recurrence-general} aparecen \(a_n\) y \(a_{n + k}\),
  se habla de una \emph{recurrencia de orden \(k\)}.%
    \index{recurrencia!orden|textbfhy}
  En general harán falta \(k\) valores para determinar la secuencia,
  típicamente dados como \(a_0\) hasta \(a_{k - 1}\),
  de~\eqref{eq:recurrence-general} podremos obtener \(a_k\),
  con \(a_1\) a \(a_k\) tenemos \(a_{k + 1}\),
  y así sucesivamente.
  Es claro que hallar una expresión cerrada
  para los términos de tales secuencias
  será posible solo en situaciones especiales.

\section{Recurrencias lineales}
\label{sec:recurrencias-lineales-teoria}
 \index{recurrencia!lineal|textbfhy}

  Una recurrencia se dice \emph{lineal} si puede escribirse de la forma:
  \begin{equation}
    \label{eq:recurrence-linear}
    u_k(n) a_{n + k}
	+ u_{k - 1} (n) a_{n + k - 1}
	+ \dotsb
	+ u_0(n) a_n
      = f(n)
  \end{equation}
  donde \(u_i(n)\) y \(f(n)\) son funciones conocidas.
  Si tanto \(u_k\) como \(u_0\) son diferentes de cero,
  es una \emph{recurrencia de orden \(k\)}.
  Si \(f(n) = 0\),
  se dice que la recurrencia es \emph{homogénea},
    \index{recurrencia!lineal!homogenea@homogénea|textbfhy}
  en caso contrario \emph{no homogénea}.
  El estudio de las recurrencias lineales
  hace uso del álgebra lineal,
  para mayores detalles véanse por ejempo Strang~%
    \cite{strang09:_intr_linear_algebra}
  o Treil~%
    \cite{treil14:_linear_algeb_done_wrong}.

  Es claro que si las secuencias \(\langle x_n \rangle\)
  y \(\langle y_n \rangle\) satisfacen una recurrencia lineal homogénea,
  la combinación lineal \(\langle \alpha x_n + \beta y_n \rangle\)
  también la satisface.
  Esta es la razón del nombre.
  Si expresamos la recurrencia homogénea como:
  \begin{equation}
    \label{eq:recurrence-linear-homogeneous}
    a_{n + k}
      = u_{k - 1}(n) a_{n + k - 1}
	 + u_{k - 2}(n) a_{n + k - 2}
	 + \dotsb
	 + u_0(n) a_n
  \end{equation}
  con los vectores%
    \index{vector}
    \(\boldsymbol{a}_n = (a_{n + k - 1}, a_{n + k - 2}, \dotsc, a_n)\)
  podemos expresar la recurrencia~\eqref{eq:recurrence-linear-homogeneous}
  como:%
    \index{matriz}
  \begin{equation}
    \label{eq:recurrence-linear-homogeneous-matrix-form}
    \boldsymbol{a}_{n + 1}
      = \boldsymbol{U}_n \cdot \boldsymbol{a}_n
  \end{equation}
  donde:
  \begin{equation}
    \label{eq:recurrence-linear-homogeneous-matrix}
    \boldsymbol{U}_n
      = \begin{pmatrix}
	  u_{k - 1}(n) & u_{k - 2}(n) & \cdots & u_1(n) & u_0(n) \\
	  1	       & 0	      & \cdots & 0	&   0	 \\
	  0	       & 1	      & \cdots & 0	&   0	 \\
	  \vdots       & \vdots	      & \ddots & \vdots & \vdots \\
	  0	       & 0	      & \cdots & 1	&   0
	\end{pmatrix}
  \end{equation}
  lo que nos permite expresar:
  \begin{equation}
    \label{eq:recurrence-linear-homogeneous-matrix-soln}
    \boldsymbol{a}_n
      = \boldsymbol{U}_{n - 1}
	  \cdot \boldsymbol{U}_{n - 2}
	  \cdot \dotsb
	  \cdot \boldsymbol{U}_0
	  \cdot \boldsymbol{a}_0
  \end{equation}
  Esto dice que la solución de la ecuación lineal homogénea
  es la combinación lineal
  de \(k\) soluciones linealmente independientes:%
    \index{recurrencia!lineal!solucion@solución|textbfhy}
  Podemos elegir
  los \(k\) componentes de \(\boldsymbol{a}_0\) independientemente,
  si ninguna de las matrices \(\boldsymbol{U}_n\) es singular
  vectores iniciales linealmente independientes
  darán vectores finales linealmente independientes.
  La solución general de la recurrencia lineal no homogénea
  puede expresarse como una solución particular%
    \index{recurrencia!lineal!solucion particular@solución particular}
  y la combinación lineal de \(k\) soluciones linealmente independientes
  de la recurrencia homogénea.

  En caso que \(\boldsymbol{U}_n\) sea una matriz constante \(\boldsymbol{U}\)
  (la recurrencia tiene coeficientes constantes),
  la ecuación~\eqref{eq:recurrence-linear-homogeneous-matrix-soln}
  se reduce a:
  \begin{equation}
    \label{eq:recurrence-linear-homogeneous-matrix-const-soln}
    \boldsymbol{a}_n
      = \boldsymbol{U}^n \cdot \boldsymbol{a}_0
  \end{equation}
  Técnicas eficientes para el cálculo de potencias
  (ver la sección~\ref{sec:potencias})%
    \index{potencia!calculo@cálculo}
  permiten obtener el vector \(\boldsymbol{a}_n\)
  rápidamente de~\eqref{eq:recurrence-linear-homogeneous-matrix-const-soln}.
  Esto ofrece una alternativa a las técnicas expuestas
  en el capítulo~\ref{cha:funciones-generatrices}.%
    \index{generatriz}

  Una matriz \(\boldsymbol{A}\) se dice \emph{diagonalizable}%
    \index{matriz!diagonalizable|textbfhy}
  si hay una matriz diagonal \(\boldsymbol{D}\)%
    \index{matriz!diagonal}
  y una matriz invertible \(\boldsymbol{P}\)%
    \index{matriz!invertible}
  tales que
    \(\boldsymbol{A} = \boldsymbol{P}^{-1} \boldsymbol{D} \boldsymbol{P}\).
  Calcular potencias de una matriz expresada de esta forma
  es particularmente simple,
  ya que
    \(\boldsymbol{A}^n = \boldsymbol{P}^{-1} \boldsymbol{D}^n \boldsymbol{P}\),
  y la potencia de la matriz diagonal
  es simplemente las potencias de sus elementos.
  Los elementos de la matriz \(\boldsymbol{D}\)
  resultan ser los valores propios de \(\boldsymbol{A}\),%
    \index{matriz!valor propio}
  las soluciones de la ecuación
    \(\det(\boldsymbol{A} - \lambda \boldsymbol{I}) = 0\).
  Esto da una forma alternativa
  de expresar la solución de la recurrencia~\eqref{eq:recurrence-linear}
  para el caso de coeficientes constantes.

\section{Recurrencias lineales de primer orden}
\label{sec:rec-lineal-1er}

  Un caso de particular interés práctico son las recurrencias lineales
  de primer orden,%
    \index{recurrencia!lineal!primer orden}
  que podemos escribir:
  \begin{equation}
    \label{eq:recurrencia-lineal-1}
    a_{n + 1}
      = u_n a_n + f_n
  \end{equation}
  Vemos que si dividimos~\eqref{eq:recurrencia-lineal-1}
  por el \emph{factor sumador}:%
    \index{recurrencia!lineal!factor sumador}
  \begin{equation}
    \label{eq:recurrencia-lineal-2}
    s_n
      = \prod_{0 \le k \le n} u_k
  \end{equation}
  (lo que presupone que \(u_k \ne 0\) en el rango de interés)
  queda:
  \begin{equation*}
    \frac{a_{n + 1}}{s_n} - \frac{a_n}{s_{n - 1}}
      = \frac{f_n}{s_n}
  \end{equation*}
  Sumando ambos lados obtenemos la solución.

  La fórmula general es bastante engorrosa,
  ilustraremos la técnica mediante un ejemplo.
  % http://math.stackexchange.com 135803
  Sea:
  \begin{equation}
    \label{eq:recurrencia-ejemplo-1}
    a_{n + 1}
      = \frac{2 (n + 1) a_n + 5 (n + 1)!}{3}
      \qquad a_0 = 5
  \end{equation}
  Reordenando un poco:
  \begin{equation*}
    a_{n + 1} - \frac{2 (n + 1)}{3} \, a_n
      = \frac{5 (n + 1)!}{3}
  \end{equation*}
  Vemos que el factor sumador es:
  \begin{align*}
    s_n
      &= \prod_{0 \le k \le n} \frac{2 (n + 1)}{3} \\
      &= \left( \frac{2}{3} \right)^{n + 1} \, (n + 1)!
  \end{align*}
  Dividiendo la recurrencia por esto y sumando para \(0 \le k \le n - 1\)
  resulta:
  \begin{align*}
    \frac{a_n}{(2/3)^n n!} - \frac{a_0}{s_{-1}}
      &= \frac{5}{3} \, \sum_{0 \le k \le n - 1}
			  \left( \frac{3}{2} \right)^{k + 1} \\
    \frac{a_n}{(2/3)^n n!} - \frac{5}{1}
      &= \frac{5}{3} \cdot \frac{3}{2} \cdot \frac{(3/2)^n - 1}{3/2 - 1} \\
    \frac{a_n}{(2/3)^n n!}
      &= 5 + 5 \cdot \left( (3/2)^n - 1 \right) \\
      &= 5 \cdot (3/2)^n \\
    a_n
      &= 5 n!
  \end{align*}

  A veces esto sirve para simplificar sumas.
  Siguiendo esencialmente la estrategia de Rockett~%
    \cite{rockett81:_sums_inver_binom_coeff}
  calcularemos:
  \begin{equation*}
    \sum_{0 \le k \le n} \binom{n}{k}^{-1}
      = \frac{1}{n!} \sum_{0 \le k \le n} k! (n - k)!
  \end{equation*}
  Nos concentramos en la suma:
  \begin{align}
    S_n
      &= \sum_{0 \le k \le n} k! (n - k)!
	    \label{eq:sum-k!*(n-k)!-def} \\
      &= \sum_{0 \le k \le n - 1} k! (n - k)! + n! \notag \\
      &= \sum_{0 \le k \le n - 1} k! (n - 1 - k)! ((n + 1) - (k + 1))
	   + n! \notag \\
      &= (n + 1) \sum_{0 \le k \le n - 1} k! (n - 1 - k)!
	   - \sum_{0 \le k \le n - 1} k! (n - 1 - k)! (k + 1)
	   + n! \notag \\
      &= (n + 1) S_{n - 1}
	   - \sum_{0 \le k \le n - 1} (k + 1)! (n - (k + 1))!
	   + n! \notag \\
      &= (n + 1) S_{n - 1}
	   - \sum_{1 \le k \le n} k! (n - k)!
	   + n! \notag \\
      &= (n + 1) S_{n - 1}
	   - (S_n - n!)
	   + n! \notag \\
   2 S_n
      &= (n + 1) S_{n - 1} + 2 n!
	    \label{eq:sum-k!*(n-k)!-rec}
  \end{align}
  Esta es una recurrencia lineal de primer orden.%
    \index{recurrencia!lineal!primer orden}
  De la definición~\eqref{eq:sum-k!*(n-k)!-def} es \(S_0 = 1\).

  El factor sumador de~\eqref{eq:sum-k!*(n-k)!-rec} es \(2^{-n} (n + 1)!\):%
    \index{factor sumador|see{recurrencia!lineal!primer orden}}
  \begin{align*}
    \frac{2^{n + 1}}{(n + 1)!} S_n
      &= \frac{2^n}{n!} S_{n - 1} + \frac{2^{n + 1}}{n + 1} \\
    \frac{2^{n + 1}}{(n + 1)!} S_n - \frac{2}{1!} S_0
      &= \sum_{1 \le k \le n} \frac{2^{k + 1}}{k + 1} \\
  \intertext{Casualmente coincide con el término para \(k = 0\):}
    \frac{2^{n + 1}}{(n + 1)!} S_n
      &= \sum_{0 \le k \le n} \frac{2^{k + 1}}{k + 1}
  \end{align*}
  O sea:
  \begin{equation}
    \label{eq:sum-k!*(n-k)!-value}
    S_n
      = \frac{(n + 1)!}{2^{n + 1}}
	  \sum_{0 \le k \le n} \frac{2^{k + 1}}{k + 1}
  \end{equation}
  Con esto nuestra suma original es:
  \begin{equation}
    \label{eq:reciprocal-binomial-sum}
    \sum_{0 \le k \le n} \binom{n}{k}^{-1}
      = \frac{n + 1}{2^{n + 1}} \sum_{0 \le k \le n} \frac{2^{k + 1}}{k + 1}
  \end{equation}

\section{Recurrencias lineales de coeficientes constantes}
\label{sec:recurrencias-lineales}
\index{recurrencia!lineal!coeficientes constantes}

  Un caso particularmente importante
  es el de relaciones de recurrencias de la forma:
  \begin{equation*}
    a_k u_{n + k} + a_{k - 1} u_{n + k - 1} + \dotsb + a_0 u_n = f(n)
  \end{equation*}
  donde los \(a_i\) son constantes y \(f(n)\) es una función cualquiera.
  Esto se llama
  una \emph{relación de recurrencia lineal de coeficientes constantes}
  (de \emph{orden k}, si \(a_k \ne 0\)).
  Si \(f(n) = 0\),
  se dice \emph{homogénea}.
  Se requieren \(k\) condiciones adicionales para fijar la solución,
  que generalmente toman la forma de \emph{condiciones iniciales}
  dando los valores de \(u_0\) hasta \(u_{k - 1}\).
  Esto completa una \emph{recurrencia lineal}.
  La recurrencia de Fibonacci que resolvimos antes
  es una recurrencia de segundo orden,
  lineal,
  de coeficientes constantes,
  homogénea.
  La recurrencia a la que nos llevó
  la Competencia de Ensayos de la Universidad de Miskatonic
  (sección~\ref{sec:conjetura->teorema})
  es de primer orden,
  lineal de coeficientes constantes,
  no homogénea.

  Tratar el caso general es bastante engorroso,
  mostraremos el procedimiento mediante un ejemplo.
  De forma similar a la aplicación de funciones generatrices ordinarias
  presentada acá
  pueden aplicarse funciones generatrices exponenciales
  como lo hicimos en la sección~\ref{sec:Fibonacci-exponenciales}
  para los números de Fibonacci.
  Cuál se usa en un caso particular
  dependerá de lo que resulte más simple.

  Consideremos la recurrencia:
  \begin{equation*}
    u_{n + 2} + 4 u_n = 5 n^2 \qquad \text{\(u_0 = 1\), \(u_1 = 3\)}
  \end{equation*}
  Los primeros valores son:
  \begin{equation*}
    \left\langle
      1, 3, -4, -7, 36, 73, -64, -167, 436, 913, \dotsc
    \right\rangle
  \end{equation*}
  Aplicando nuestra estrategia general,
  definimos:
    \index{recurrencia!generatriz}
  \begin{equation*}
    U(z) = \sum_{n \ge 0} u_n z^n
  \end{equation*}
  Siguiendo la receta,
  y aplicando las propiedades de funciones generatrices ordinarias queda:
  \begin{equation*}
    \frac{U(z) - 3 z - 1}{z^2} + 4 \, U(z)
      = 5 (z \mathrm{D})^2 \left(\frac{1}{1 - z}\right) \\
  \end{equation*}
  Despejando y expresando en fracciones parciales:
  \begin{equation*}
    U(z)
      = \frac{82 z + 37}{25 (1 + 4 z^2)}
	  + \frac{2}{(1 - z)^3}
	  - \frac{19}{5 (1 - z)^2}
	  + \frac{33}{25 (1 - z)}
  \end{equation*}
  Luego basta ``leer'' los coeficientes en esto.
  Puntos interesantes ponen los términos:
  \begin{align*}
    \frac{1}{1 + 4 z^2}
      &= \sum_{n \ge 0} (-4)^n z^{2 n} \\
    \frac{z}{1 + 4 z^2}
      &= \sum_{n \ge 0} (-4)^n z^{2 n + 1} \\
    \frac{1}{(1 - z)^3}
      &= \sum_{n \ge 0} \binom{-3}{n} \, (-1)^n z^n \\
      &= \sum_{n \ge 0} \frac{(n + 2) (n + 1)}{2} \, z^n \\
    \frac{1}{(1 - z)^2}
      &= \sum_{n  \ge 0} \binom{-2}{n} \, (-1)^n z^n \\
      &= \sum_{n \ge 0} (n + 1) z^n
  \end{align*}
  Podemos entonces expresar la solución como:
  \begin{align*}
    u_{2 k}
      &= \frac{37}{25} \cdot (-4)^k
	   + 2 \cdot \frac{(2 k + 2) (2 k + 1)}{2}
	   - \frac{19}{5} \cdot (2 k + 1)
	   + \frac{33}{25} \\
      &= \frac{37}{25} \cdot (-4)^k
	   + 4 k^2 - \frac{8}{5} k - \frac{12}{25} \\
    u_{2 k + 1}
      &= \frac{82}{25} \cdot (-4)^k
	   + 2 \cdot \frac{(2 k + 3) (2 k + 2)}{2}
	   - \frac{19}{5} \cdot (2 k + 2)
	   + \frac{33}{25} \\
      &= \frac{82}{25} \cdot (-4)^k
	   + 4 k^2
	   + \frac{12}{5} k
	   - \frac{7}{25}
  \end{align*}
  Esta separación en términos pares e impares es incómoda.
  Usando números complejos podemos factorizar más:%
    \index{C (numeros complejos)@\(\mathbb{C}\) (números complejos)}
  \begin{align*}
    1 + 4 z^2
      &= (1 - 2 \mathrm{i} z) (1 + 2 \mathrm{i} z) \\
    \frac{1}{1 + 4 z^2}
      &= \frac{1}{2} \, \left(
			  \frac{1}{1 + 2 \mathrm{i} z}
				     + \frac{1}{1 - 2 \mathrm{i} z}
			\right) \\
    \frac{z}{1 + 4 z^2}
      &= \frac{\mathrm{i}}{4} \,
	   \left(
	     \frac{1}{1 + 2 \mathrm{i} z}
	       - \frac{1}{1 - 2 \mathrm{i} z}
	   \right)
  \end{align*}
  y podemos entonces expresar:
  \begin{align*}
    u_n
      &= \frac{82}{25} \cdot
	   \frac{\mathrm{i}}{4} \,
	     \bigl(
	       (-2 \mathrm{i})^n - (2 \mathrm{i})^n
	     \bigr)
	   + \frac{37}{25} \cdot
		\frac{1}{2} \,
		  \bigl(
		    (-2 \mathrm{i})^n + (2 \mathrm{i})^n
		  \bigr) \\
      &\hspace{3em}
	   + 2 \cdot \frac{(n + 2) (n + 1)}{2}
	   - \frac{19}{5} \cdot (n + 1)
	   + \frac{33}{25} \\
      &= \frac{\left(
		 (37 - 41 \mathrm{i}) \mathrm{i}^n
		    + (37 + 41 \mathrm{i}) (-\mathrm{i})^n
	       \right) 2^n
	      }{50}
	   + n^2
	   - \frac{4 n}{5}
	   - \frac{12}{25}
  \end{align*}
  Es bien poco probable que hubiéramos adivinado esta solución\ldots

  Alternativamente,
  para el término:
  \begin{equation*}
      \frac{87 z + 37}{25 (1 + 4 z^2)}
      = \frac{37 - 41 \mathrm{i}}{2} \cdot \frac{1}{1 - 2 \mathrm{i} z}
	  + \frac{37 + 41 \mathrm{i}}{2} \cdot \frac{1}{1 + 2 \mathrm{i} z}
  \end{equation*}
  Podemos expresar el aporte mediante exponenciales complejas%
    \index{C (numeros complejos)@\(\mathbb{C}\) (números complejos)!exponencial}
  (conjugar el exponente corresponde a conjugar):
  \begin{equation*}
    \exp(u) \cdot \exp(n v)
      + \exp(\overline{u}) \cdot \exp(n \overline{v})
  \end{equation*}
  Tenemos:
  \begin{align*}
    \frac{37 + 41 \mathrm{i}}{2}
      &= \exp(u) \\
    2 \mathrm{i}
      &= \exp (v)
  \end{align*}
  Tomando en cuenta que:%
    \index{C (numeros complejos)@\(\mathbb{C}\) (números complejos)!funciones trigonometricas@funciones trigonométricas}
  \begin{align*}
    \cos v
      &= \frac{\mathrm{e}^{\mathrm{i} v}
		  + \mathrm{e}^{ - \mathrm{i} v}}
	      {2} \\
    \sin v
      &= - \mathrm{i} \, \frac{\mathrm{e}^{\mathrm{i} v}
				 - \mathrm{e}^{ - \mathrm{i} v}}
			      {2}
  \end{align*}
  si escribimos \(u = a + b \mathrm{i}\) y \(v = c + d \mathrm{i}\)
  después de simplificar la solución:
  \begin{align*}
    \exp(u) \cdot \exp(n v)
      + \exp(\overline{u}) \cdot \exp(n \overline{v})
      &= \exp(a + b \mathrm{i} + n (c + d \mathrm{i}))
	   + \exp(a - b \mathrm{i} + n (c - d \mathrm{i})) \\
      &= \exp(a + n c) (\exp((b + n d) \mathrm{i})
			  + \exp(- (b + n d) \mathrm{i})) \\
      &= 2 \exp(a + n c) \cos (b + n d)
  \end{align*}
  En detalle,
  para este caso tenemos:
  \begin{align*}
    e^a
      &= \frac{\sqrt{37^2 + 41^2}}{2}
       = \frac{5 \sqrt{122}}{2} \\
    \cos b
      &= \frac{37}{\sqrt{37^2 + 41^2}}
       = \frac{37 \sqrt{122}}{610} \\
    e^c
      &= 2 \\
    \cos d
      &= \frac{0}{\sqrt{0^2 + 2^2}}
       = 0
      \qquad d = \frac{\pi}{2}
  \end{align*}
  y la solución puede expresarse:
  \begin{equation*}
    u_n
      = 5 \sqrt{122} \cdot 2^{n - 1}
	  \cdot \cos \left( b + \frac{n \pi}{2} \right)
	   + n^2
	   - \frac{4 n}{5}
	   - \frac{12}{25}
\end{equation*}

% ricatti.tex
%
% Copyright (c) 2013-2014 Horst H. von Brand
% Derechos reservados. Vea COPYRIGHT para detalles

\section{Recurrencia de Ricatti}
\label{sec:Ricatti}
\index{recurrencia!Ricatti|see{Ricatti, recurrencia de}}
\index{Ricatti, recurrencia de}

  Una recurrencia de la forma:
  \begin{equation}
    \label{eq:Ricatti}
    w_{n + 1}
      = \frac{a w_n + b}{c w_n + d}
  \end{equation}
  donde \(c \ne 0\)
  y \(a d - b c \ne 0\) se llama \emph{recurrencia de Ricatti}
  (si \(c = 0\) es simplemente una recurrencia lineal de primer orden;
   si \(a d = b c\) se reduce a \(w_{n + 1} = \text{constante}\)).
  Incidentalmente,
  la recurrencia~\eqref{eq:Fibonacci-search-recurrence-r}
  para~\(r\)
  que hallamos al analizar la búsqueda de Fibonacci
  en la sección~\ref{sec:busqueda-Fibonacci}%
    \index{Fibonacci, busqueda de@Fibonacci, búsqueda de!analisis@análisis}
  es una recurrencia de Ricatti.

  Para resolver este tipo de recurrencias hay varias opciones,
  que exploraremos en lo que sigue.

\subsection{Vía recurrencia de segundo orden}
\label{sec:Ricatti-2nd}

  Seguimos el esquema de Brand~\cite{brand55:_seq_def_difference_eq}.
  Si en~\eqref{eq:Ricatti} substituimos \(y_n \mapsto c w_n + d\),
  queda una recurrencia de la forma:
  \begin{equation}
    \label{eq:Ricatti-2nd-aux-1}
    y_n
      = \alpha - \frac{\beta}{y_{n - 1}}
  \end{equation}
  donde:
  \begin{align*}
    \alpha
      &= a + d \\
    \beta
      &= a d - b c
  \end{align*}
  Claramente eso solo vale si \(a d - b c \ne 0\).
  Substituyendo ahora:
  \begin{equation}
    \label{eq:Ricatti-2nd-y}
    y_n
      = \frac{x_{n + 1}}{x_n}
  \end{equation}
  resulta:
  \begin{equation}
    \label{eq:Ricatti-2nd-aux}
    x_{n + 2} - \alpha x_{n + 1} + \beta x_n
      = 0
  \end{equation}
  Esta es una recurrencia lineal de segundo orden,
  de coeficientes constantes y homogénea.
  Necesitamos dos valores iniciales para resolverla,
  podemos elegir bastante arbitrariamente \(x_0 = 1\),
  dando \(x_1 = y_0\),
  que a su vez podemos obtener de la condición inicial original.

  Para un ejemplo,
  tomemos \(w_0 = 3\) y:
  \begin{equation}
    \label{eq:Ricatti-ex}
    w_{n + 1}
      = \frac{5 w_n + 2}{3 w_n + 4}
  \end{equation}
  Siguiendo los pasos indicados:
  \begin{align*}
    3 w_{n + 1} + 4
      &= 3 \cdot \frac{5 w_n + 2}{3 w_n + 4} + 4 \\
      &= 9 - \frac{14}{3 w_n + 4}
  \end{align*}
  Substituyendo:
  \begin{equation*}
    3 w_n + 4
      = \frac{x_{n + 1}}{x_n}
  \end{equation*}
  y reordenando resulta:
  \begin{equation}
    \label{eq:Ricatti-ex-2nd}
    x_{n + 2} - 9 x_{n + 1} + 14 x_n
      = 0
  \end{equation}
  Con las condiciones iniciales \(x_0 = 1\), \(x_1 = 3 w_0 + 4 = 13\)
  la solución de~\eqref{eq:Ricatti-ex-2nd} es:
  \begin{equation*}
    x_n
      = \frac{11 \cdot 7^n - 6 \cdot 2^n}{5}
  \end{equation*}
  y finalmente:
  \begin{equation}
    \label{eq:Ricatti-ex-2nd-sol}
    w_n
      = \frac{11 \cdot 7^n + 2^{n + 2}}{11 \cdot 7^n - 3 \cdot 2^{n + 1}}
  \end{equation}

\subsection{Reducción a una recurrencia de primer orden}
\label{sec:Ricatti-1}

  Siguiendo a Mitchell~%
    \cite{mitchell00:_riccati_solution}
  definamos la secuencia auxiliar:
  \begin{equation}
    \label{eq:Ricatti-1st-x}
    x_n
      = \frac{1}{1 + \eta w_n}
  \end{equation}
  Expresamos la recurrencia~\eqref{eq:Ricatti}
  en términos de \(x_n\),
  y despejamos \(x_{n + 1}\):
  \begin{equation*}
    x_{n + 1}
      = \frac{(d \eta - c) x_n + c}
	     {(b \eta^2 - (a - d) \eta - c) x_n + a \eta + c}
  \end{equation*}
  Buscamos que esta recurrencia sea lineal,
  o sea:
  \begin{equation*}
    b \eta^2 - (a - d) \eta - c
      = 0
  \end{equation*}
  Arbitrariamente elegimos el signo positivo:
  \begin{equation}
    \label{eq:Ricatti-1st-aux-eta}
    \eta
      = \frac{a - d + \sqrt{(a - d)^2 + 4 b c}}{2 b}
  \end{equation}
  Esto lleva a la ecuación auxiliar:
  \begin{equation}
    \label{eq:Ricatti-1st-aux}
    x_{n + 1}
      = \frac{(d \eta - c) x_n + c}{a \eta + c}
  \end{equation}
  Esta es simple de resolver.

  Aplicado al mismo ejemplo anterior,
  tenemos:
  \begin{align*}
    \eta
      &= \frac{5 - 4 + \sqrt{(5 - 4)^2 + 4 \cdot 2 \cdot 3}}{2 \cdot 2}
       = \frac{3}{2}
  \end{align*}
  La recurrencia auxiliar es:
  \begin{align}
    x_{n + 1}
      &= \frac{(4 \cdot \frac{3}{2} - 3) x_n + 3}
	      {5 \cdot \frac{3}{2} + 3} \notag \\
      &= \frac{2 x_n + 2}{7}
	       \label{eq:Ricatti-ex-1st-aux}
  \end{align}
  De la condición inicial tenemos:
  \begin{equation}
    \label{eq:Ricatti-ex-1st-initial}
    x_0
      = \frac{1}{1 + \eta w_0}
      = \frac{2}{11}
  \end{equation}
  La tradicional danza para resolver recurrencias entrega:
  \begin{equation}
    \label{eq:Ricatti-ex-1st-aux-sol}
    x_n
      = \frac{2}{5} - \frac{84}{385} \cdot \left( \frac{2}{7} \right)^n
  \end{equation}
  De acá con~\eqref{eq:Ricatti-1st-x} resulta:
  \begin{equation}
    \label{eq:Ricatti-ex-1st-sol}
    w_n
      = \frac{11 \cdot 7^n + 2^{n + 2}}{11 \cdot 7^n - 3 \cdot 2^{n + 1}}
  \end{equation}
  Tal como antes.

\subsection{Transformación de Möbius}
\label{sec:Ricatti-Moebius}

  A una transformación de la forma:
  \begin{equation}
    \label{eq:Moebius-tranform}
    w = \frac{a_{1 1} z + a_{1 2}}{a_{2 1} z + a_{2 2}}
  \end{equation}
  donde \(a_{11} a_{22} - a_{12} a_{21} \ne 0\)
  (de lo contrario,
   la expresión se reduce a una constante)
  se le llama \emph{transformación de Möbius}.%
    \index{Mobius, transformacion de@Möbius, transformación de}%
    \index{recurrencia!Ricatti!transformacion de Möbius@transformación de Möbius}
  Una de sus características interesantes es que forman un grupo
  con la composición de funciones,%
    \index{grupo}
  como es fácil demostrar.
  A nosotros puntualmente nos interesa lo siguiente:
  Sean \(A(z)\), \(B(z)\) transformaciones de Möbius.
  \begin{align}
    A(z)
     &= \frac{a_{1 1} z + a_{1 2}}{a_{2 1} z + a_{2 2}}
	  \label{eq:Moebius-A} \\
    B(z)
     &= \frac{b_{1 1} z + b_{1 2}}{b_{2 1} z + b_{2 2}}
	  \label{eq:Moebius-B}
  \end{align}
  La composición es:
  \begin{equation}
    \label{eq:Moebius-AB}
    A(B(z))
      = \frac{(a_{11} b_{11} + a_{12} b_{21}) z
		 + (a_{11} b_{12} + a_{12} b_{22})}
	     {(a_{21} b_{11} + a_{22} b_{21}) z
		 + (a_{21} b_{12} + a_{22} b_{22})}
  \end{equation}
  Si representamos las transformaciones
  por las respectivas matrices de coeficientes,
  vemos que la composición corresponde al producto de las matrices.
  Lo que hace la recurrencia~\eqref{eq:Ricatti}
  es aplicar la transformación de Möbius repetidas veces:
  \begin{equation}
    \label{eq:Ricatti-Moebius}
    w_n
      = A^n(w_0)
  \end{equation}
  a lo que naturalmente corresponde
  el calcular la potencia de la matriz respectiva.
  Para cálculo numérico puede usarse
  entonces una técnica eficiente para calcular potencias,
  como las dadas en la sección~\ref{sec:potencias}.%
    \index{potencia!calculo eficiente@cálculo eficiente}

  En nuestro caso de matrices de \(2 \times 2\)
  los valores propios son simples de obtener:%
    \index{matriz!valor propio}
  \begin{equation}
    \label{eq:diagonalizable-lambda-eqn}
    (a_{11} - \lambda) (a_{22} - \lambda) - a_{12} a_{21}
      = 0
  \end{equation}
  La fórmula cuadrática da:
  \begin{equation}
    \label{eq:diagonalizable-lambda}
    \lambda
      = \frac{a_{11} + a_{22}
		\pm \sqrt{(a_{11} + a_{22})^2
			     - 4 (a_{11} a_{22} - a_{12} a{21})}}
	     {2}
  \end{equation}
  Las columnas de la matriz \(\boldsymbol{P}\) a su vez
  son los vectores propios correspondientes:
  \begin{equation}
    \label{eq:diagonalizable-p-eqn}
    \begin{pmatrix}
      a_{11} - \lambda_i & a_{12} \\
      a_{21}		 & a_{22} - \lambda_i
    \end{pmatrix}
    \cdot
    \begin{pmatrix}
      p_{1 i} \\
      p_{2 i}
    \end{pmatrix}
      =
      \begin{pmatrix}
	0 \\
	0
      \end{pmatrix}
  \end{equation}
  Por construcción,
  la matriz del sistema~\eqref{eq:diagonalizable-p-eqn}
  es singular,
  y solo da una relación entre \(p_{1 i}\) y \(p_{2 i}\).
  Podemos imponer cualquier condición que resulte cómoda.
  Conociendo \(\boldsymbol{D}\) y \(\boldsymbol{P}\)
  podemos calcular las potencias
  y así obtener la solución buscada.

  Volviendo a nuestro manoseado ejemplo,
  tenemos:
  \begin{equation}
    \label{eq:Ricatti-ex-Moebius-A}
    \boldsymbol{A} =
    \begin{pmatrix}
      5 & 2 \\
      3 & 4
    \end{pmatrix}
  \end{equation}
  Los valores propios son \(\lambda_1 = 7\) y \(\lambda_2 = 2\),
  eligiendo valores \(p_{1 i} = 1\):
  \begin{equation}
    \label{eq:eq:Ricatti-ex-Moebius-P}
    \boldsymbol{D} =
    \begin{pmatrix}
	7  & 0 \\
	0  & 2
    \end{pmatrix} \qquad
    \boldsymbol{P} =
    \begin{pmatrix}
	1  & 1 \\
	1  & - \sfrac{3}{2}
    \end{pmatrix} \qquad
    \boldsymbol{P}^{-1} =
    \begin{pmatrix}
      \sfrac{3}{5} & \sfrac{2}{5} \\
      \sfrac{2}{5} & -\sfrac{2}{5}
    \end{pmatrix}
  \end{equation}
  Podemos entonces calcular:
  \begin{align}
    \boldsymbol{A}^n
     &= \boldsymbol{P}^{-1}
	  \cdot \boldsymbol{D}^n
	  \cdot \boldsymbol{P} \notag \\
     &=
     \begin{pmatrix}
       \displaystyle \frac{3 \cdot 7^n + 2^{n + 1}}{5} &
	 \displaystyle \frac{3 \cdot 7^n - 3 \cdot 2^n}{5} \\
       \\
       \displaystyle \frac{2 \cdot 7^n - 2^{n + 1}}{5} &
	 \displaystyle \frac{2 \cdot 7^n + 3 \cdot 2^n}{5}
     \end{pmatrix}
	\label{eq:Ricatti-ex-Moebius-An}
  \end{align}
  Con esto es:
  \begin{equation}
    \label{eq:Ricatti-ex-Moebius-sol-gral}
    w_n
      = \frac{(3 \cdot 7^n + 2^{n + 1}) w_0
		+ (2 \cdot 7^n - 2^{n + 1})}
	     {(2 \cdot 7^n - 2^{n + 1}) w_0
		+ (2 \cdot 7^n + 3 \cdot 2^n)}
  \end{equation}
  Nótese que esta solución muestra explícitamente la dependencia
  de la condición inicial.
  Con nuestra condición inicial \(w_0 = 3\) resulta nuevamente:
  \begin{equation}
    \label{eq:Ricatti-ex-Moebius-sol}
    w_n
      = \frac{11 \cdot 7^n + 2^{n + 2}}{11 \cdot 7^n - 3 \cdot 2^{n + 1}}
  \end{equation}

%%% Local Variables:
%%% mode: latex
%%% TeX-master: "clases"
%%% End:


%%% Local Variables:
%%% mode: latex
%%% TeX-master: "clases"
%%% End:


% metodo-simbolico.tex
%
% Copyright (c) 2011-2014 Horst H. von Brand
% Derechos reservados. Vea COPYRIGHT para detalles

\chapter{El método simbólico}
\label{cha:metodo-simbolico}
\index{metodo simbolico@método simbólico|textbfhy}

  Vimos antes
  (capítulo~\ref{cha:funciones-generatrices})
  que las operaciones aritméticas entre funciones generatrices
  dan funciones generatrices
  que corresponden a combinaciones
  de los objetos que éstas representan.%
    \index{generatriz!combinatoria}
  Expandiendo esta observación,
  veremos un marco en el cual derivar ecuaciones
  para las funciones generatrices de interés
  en problemas combinatorios
  es casi automático,
  como sistematizado por Flajolet y Sedgewick~%
    \cite{flajolet09:_analy_combin}
  y aplicado a análisis de algoritmos por Sedgewick y Flajolet~%
    \cite{sedgewick13:_introd_anal_algor}.
  Otra visión en la misma línea presenta Wilf~%
    \cite{wilf06:_gfology}.
  Claro está que igual queda la tarea
  de extraer la información buscada
  de la ecuación resultante.
  Más adelante discutiremos herramientas para esta segunda tarea.

\section{Un primer ejemplo}
\label{sec:MS-ejemplo}

  Consideremos primero el derivar una ecuación generatriz
  para el número de árboles binarios con \(n\) nodos,
  definidos diciendo que un árbol binario es una de las siguientes:
  \begin{itemize}
  \item
    Es \emph{vacío}.
  \item
    Consta de un \emph{nodo raíz} y dos subárboles binarios
    (izquierdo y derecho).
  \end{itemize}
  Una manera de modelar esto es usar la variable \(z\)
  para marcar los nodos,
  vía usar \(\beta\) para representar un árbol binario
  y \(\lvert \beta \rvert\) para su número de nodos,
  definir la función generatriz:
  \begin{equation*}
    B(z)
      = \sum_{\beta} z^{\lvert \beta \rvert}
  \end{equation*}
  El coeficiente de \(z^n\)
  en \(B(z)\) es el número que nos interesa.

  Directamente de la definición de árbol binario%
    \index{arbol binario@árbol binario}
  sabemos que hay un árbol binario vacío,
  que al no tener nodos aporta \(1\) a \(B(z)\).
  Los demás se pueden dividir en un nodo raíz
  y dos subárboles binarios,
  vale decir:
  \begin{align*}
    B(z)
      &= 1 + \sum_{\beta_1, \beta_2}
	       z^{1 + \lvert \beta_1 \rvert
		    + \lvert \beta_2 \rvert} \\
      &= 1 + z B^2 (z)
  \end{align*}
  Hay una íntima relación entre la definición recursiva
  y nuestra ecuación para la función generatriz.
  Lo que buscamos es sistematizar y extender esta observación.

  Estamos interesados
  en colecciones de objetos.
  Formalmente:
  \begin{definition}
    \index{metodo simbolico@método simbólico!clase|textbfhy}
    Una \emph{clase} \(\mathcal{A}\)
    es un conjunto numerable
    de \emph{objetos} \(\alpha \in \mathcal{A}\).
    A cada objeto \(\alpha\) se le asocia un \emph{tamaño},
    \(\lvert \alpha \rvert \in \mathbb{N}_0\).
    El conjunto de objetos con un tamaño dado es finito.
  \end{definition}
  Usaremos consistentemente letra caligráfica,
  como \(\mathcal{A}\),
  para clases,
  y la misma letra para identificar nociones relacionadas.
  Así,
  para la clase \(\mathcal{A}\)
  generalmente usaremos \(\alpha\) para un elemento de la clase
  y llamaremos \(a_n\)
  al número de objetos de la clase de tamaño \(n\).
  Usaremos \(\mathcal{A}_n\)
  para referirnos al conjunto de objetos de la clase \(\mathcal{A}\)
  de tamaño \(n\),
  con lo que \(a_n = \lvert \mathcal{A}_n \rvert\).
  A las funciones generatrices ordinaria y exponencial
  correspondientes
  les llamaremos \(A(z)\) y \(\widehat{A}(z)\),
  respectivamente:
  \begin{align*}
    A(z)
      &= \sum_{\alpha \in \mathcal{A}} z^{\lvert \alpha \rvert}
       = \sum_{n \ge 0} a_n z^n \\
    \widehat{A}(z)
      &= \sum_{\alpha \in \mathcal{A}}
	   \frac{z^{\lvert \alpha \rvert}}{\lvert \alpha \rvert !}
       = \sum_{n \ge 0} a_n \, \frac{z^n}{n!}
  \end{align*}

  Nuestro siguiente objetivo es construir nuevas clases
  a partir de las que ya tenemos.
  Debe tenerse presente que como lo que nos interesa
  es contar el número de objetos de cada tamaño,
  basta construir objetos con distribución de tamaños adecuada
  (o sea,
   relacionados con lo que deseamos contar por una biyección).
  Comúnmente el tamaño de los objetos es el número de alguna clase de átomos
  que lo componen.

  Las clases más elementales son \(\varnothing\),
  la clase que no contiene objetos;
  \(\mathcal{E} = \{\epsilon\}\),
  la clase que contiene únicamente el objeto vacío \(\epsilon\)
  (de tamaño nulo);
  y la clase que comúnmente llamaremos \(\mathcal{Z}\),
  conteniendo un único objeto de tamaño uno
  (que llamaremos \(\zeta\) por consistencia).
  Luego definimos operaciones que combinan
  las clases \(\mathcal{A}\) y \(\mathcal{B}\)
  mediante \emph{unión combinatoria} \(\mathcal{A} + \mathcal{B}\),
  en que aparecen los \(\alpha\) y los \(\beta\) con sus tamaños
  (los objetos individuales se ``decoran'' con su proveniencia,
   de forma que \(\mathcal{A}\) y  \(\mathcal{B}\)
   no necesitan ser disjuntos;
   pero generalmente nos preocuparemos
   que \(\mathcal{A}\) y \(\mathcal{B}\)
   sean disjuntos,
   o podemos usar el principio de inclusión y exclusión
   para contar los conjuntos de interés).
  Ocasionalmente restaremos objetos de una clase,
  lo que debe interpretarse sin decoraciones
  (estamos dejando fuera ciertos elementos,
   simplemente).
  Usaremos \emph{producto cartesiano}
  \(\mathcal{A} \times \mathcal{B}\),%
    \index{producto cartesiano}
  cuyos elementos son pares \((\alpha, \beta)\)
  y el tamaño del par
  es \(\lvert \alpha \rvert + \lvert \beta \rvert\).
  Otras operaciones son formar \emph{secuencias}%
    \index{secuencia}
  de elementos de \(\mathcal{A}\)
  (se anota \(\Seq(\mathcal{A})\)),
  formar \emph{conjuntos}%
    \index{conjunto}
  \(\Set(\mathcal{A})\)
  y \emph{multiconjuntos}
    \index{multiconjunto}
  \(\MSet(\mathcal{A})\)
  de elementos de \(\mathcal{A}\).
  Consideramos también la operación \(\Cyc(\mathcal{A})\),
  que consiste en ordenar elementos de \(\mathcal{A}\) en un círculo
  (una secuencia conectando inicio y fin).%
    \index{ciclo}
  Usaremos también la operación de \emph{composición},
  que anotaremos \(\mathcal{A} \circ \mathbf{B}\),
  definida mediante para cada objeto \(\alpha \in \mathcal{A}\)
  construir un nuevo objeto substituyendo
  \(\lvert \alpha \rvert\) elementos de \(\mathcal{B}\) por sus átomos.
  Otra operación útil es \emph{marcar} uno de los átomos de cada objeto,
  cosa que anotaremos \(\mathcal{A}^\bullet\).
  El tamaño de un objeto compuesto
  es simplemente la suma de los tamaños de los componentes.
  De incluir objetos de tamaño cero
  en estas construcciones pueden crearse infinitos objetos de un tamaño dado,
  lo que no es una clase según nuestra definición.
  Por ello estas construcciones son aplicables
  sólo si \(\mathcal{A}_0 = \varnothing\).

  Otro juego popular de notaciones para estas operaciones es
  \(\mathfrak{S}(\mathcal{A})\) para secuencia,
  \(\mathfrak{P}(\mathcal{A})\) para conjunto
  (de \emph{\foreignlanguage{english}{powerset}}),
  \(\mathfrak{M}(\mathcal{A})\) para multiconjunto
  \(\mathfrak{C}(\mathcal{A})\) para ciclo,
  y \(\theta \mathcal{A}\) para marcar un átomo.

  Es importante recalcar las relaciones y diferencias
  entre las estructuras.
  En una secuencia es central
  el orden de las piezas que la componen.
  Ejemplo son las palabras,
  interesa el orden exacto de las letras
  (y estas pueden repetirse).
  En un conjunto solo interesa si el elemento está presente o no,
  no hay orden.
  En un conjunto un elemento en particular está o no presente,
  a un multiconjunto puede pertenecer varias veces.

  En lo que sigue haremos distinción
  entre objetos rotulados y sin rotular.
  Para algunos ejemplos de la distinción véase la sección~%
    \ref{sec:combinatorial-applications}.%
    \index{combinatoria!objetos rotulados}
  Generalmente los rótulos se refieren a algún orden externo
  u otra marca que distingue a los elementos.
  Si un objeto se considera creado de átomos idénticos
  (intercambiables)
  corresponde considerarlos no rotulados;
  si un objeto está compuesto de átomos diferenciables
  podemos considerarlos rotulados secuencialmente,
  y estamos frente a objetos rotulados.
  Un punto que produce particular confusión
  es que tiene perfecto sentido
  hablar de secuencias de elementos sin rotular.
  La secuencia impone un orden,
  pero elementos iguales se consideran indistinguibles
  (en una palabra interesa el orden de las letras,
   pero al intercambiar dos letras iguales
   la palabra sigue siendo la misma).
  Recuerde la discusión de la sección~\ref{sec:tao-bookkeeper}.

\section{Objetos sin rotular}
\label{sec:sin-rotular}
\index{metodo simbolico@método simbólico!objetos no rotulados}

  Nuestro primer teorema
  relaciona las funciones generatrices ordinarias
  respectivas para algunas de las operaciones entre clases
  definidas antes.
  Las funciones generatrices de las clases \(\varnothing\),
  \(\mathcal{E}\) y \(\mathcal{Z}\)
  son,
  respectivamente,
  \(0\), \(1\) y \(z\).
  En las derivaciones
  de las transferencias de ecuaciones simbólicas
  a ecuaciones para las funciones generatrices
  lo que nos interesa es contar los objetos entre manos,
  recurriremos a biyecciones para ello en algunos de los casos.
  \begin{theorem}[Método simbólico, OGF]
    \index{metodo simbolico@método simbólico!teorema de transferencia!objetos no rotulados|textbfhy}
    \label{theo:ms-OGF}
    Sean \(\mathcal{A}\) y \(\mathcal{B}\) clases de objetos,
    con funciones generatrices ordinarias
    respectivamente \(A(z)\) y \(B(z)\).
    Entonces tenemos
    las siguientes funciones generatrices ordinarias:
    \begin{enumerate}
    \item
      Para enumerar \(\mathcal{A}^\bullet\):
      \begin{equation*}
	z \mathrm{D} A(z)
      \end{equation*}
    \item
      Para enumerar \(\mathcal{A} + \mathcal{B}\):
      \begin{equation*}
	A(z) + B(z)
      \end{equation*}
    \item
      Para enumerar \(\mathcal{A} \times \mathcal{B}\):
      \begin{equation*}
	A(z) \cdot B(z)
      \end{equation*}
    \item
      Para enumerar \(\Seq(\mathcal{A})\):
      \begin{equation*}
	\frac{1}{1 - A(z)}
      \end{equation*}
    \item
      Para enumerar \(\mathcal{A} \circ \mathcal{B}\):
      \begin{equation*}
	A(B(z))
      \end{equation*}
    \item
      Para enumerar \(\Set(\mathcal{A})\):
      \begin{equation*}
	\prod_{\alpha \in \mathcal{A}}
	   \left( 1 + z^{\lvert \alpha \rvert} \right)
	  = \prod_{n \ge 1} (1 + z^n)^{a_n}
	  = \exp \left(
		   \sum_{k \ge 1} \frac{(-1)^{k + 1}}{k} \, A(z^k)
		 \right)
      \end{equation*}
    \item
      Para enumerar \(\MSet(\mathcal{A})\):
      \begin{equation*}
	\prod_{\alpha \in \mathcal{A}}
	   \frac{1}{1 - z^{\lvert \alpha \rvert}}
	  = \prod_{n \ge 1} \frac{1}{(1 - z^n)^{a_n}}
	  = \exp\left(
		   \sum_{k \ge 1} \frac{A(z^k)}{k}
		\right)
      \end{equation*}
    \item
      Para enumerar \(\Cyc(\mathcal{A})\):
      \begin{equation*}
	\sum_{n \ge 1} \frac{\phi(n)}{n} \, \ln \frac{1}{1 - A(z^n)}
      \end{equation*}
    \end{enumerate}
  \end{theorem}
  \begin{proof}
    Usamos libremente resultados sobre funciones generatrices,
    capítulo~\ref{cha:funciones-generatrices}%
      \index{generatriz},
    en las demostraciones de cada caso.
    Usaremos casos ya demostrados en las demostraciones sucesivas.
    \begin{enumerate}
    \item % mark A
      El objeto \(\alpha \in \mathcal{A}\)
      da lugar a \(\lvert \alpha \rvert\) objetos
      al marcar cada uno de sus átomos,
      lo que da la función generatriz:
      \begin{equation*}
	\sum_{\alpha \in \mathcal{A}}
	  \lvert \alpha \rvert z^{\lvert \alpha \rvert}
      \end{equation*}
      Esto es lo indicado.
    \item % A + B
      Si hay \(a_n\) elementos de \(\mathcal{A}\) de tamaño \(n\)
      y \(b_n\) elementos de \(\mathcal{B}\) de tamaño \(n\),
      habrán \(a_n + b_n\) elementos
      de \(\mathcal{A} + \mathcal{B}\)
      de tamaño \(n\).

      Alternativamente,
      usando la notación de Iverson
      (ver la sección~\ref{sec:sumatorias-productorias}):%
	\index{Iverson, convencion de@Iverson, convención de}
      \begin{equation*}
	\sum_{\mathclap{\gamma \in \mathcal{A} \cup \mathcal{B}}}
	  z^{\lvert \gamma \rvert}
	  = \; \sum_{\mathclap{\gamma \in \mathcal{A}
					    \cup \mathcal{B}}}
		 \left(
		   [\gamma \in \mathcal{A}]
		     z^{\lvert \gamma \rvert}
		      + [\gamma \in \mathcal{B}]
			  z^{\lvert \gamma \rvert}
		 \right)
	  = \sum_{\alpha \in \mathcal{A}} z^{\lvert \alpha \rvert}
	      + \sum_{\beta \in \mathcal{B}}
		  z^{\lvert \beta \rvert}
	  = A(z) + B(z)
      \end{equation*}
    \item % A x B
      Hay:
      \begin{equation*}
	\sum_{0 \le k \le n} a_k b_{n - k}
      \end{equation*}
      maneras de combinar elementos de \(\mathcal{A}\)
      con elementos de \(\mathcal{B}\) cuyos tamaños sumen \(n\),
      y este es precisamente
      el coeficiente de \(z^n\) en \(A(z) \cdot B(z)\).

      Alternativamente:
      \begin{align*}
	\sum_{\mathclap{\gamma \in \mathcal{A} \times \mathcal{B}}}
	     z^{\lvert \gamma \rvert}
	  = \sum_{\mathclap{\substack{
			      \alpha \in \mathcal{A} \\
			      \beta \in \mathcal{B}
		 }}} z^{\lvert \alpha \rvert + \lvert \beta \rvert}
	  = \left(
	       \sum_{\alpha \in \mathcal{A}}
		 z^{\lvert \alpha \rvert}
	     \right)
	       \cdot \left(
			\sum_{\beta \in \mathcal{B}}
			  z^{\lvert \beta \rvert}
		     \right)
	  = A(z) \cdot B(z)
      \end{align*}
    \item % Seq(A)
      Hay una manera de obtener la secuencia de largo 0
      (aporta el objeto vacío \(\epsilon\)),
      las secuencias de largo \(1\)
      son simplemente los elementos de \(\mathcal{A}\),
      las secuencias de largo \(2\)
      son elementos de \(\mathcal{A} \times \mathcal{A}\),
      y así sucesivamente.
      O sea,
      las secuencias se representan mediante:
      \begin{equation*}
	\mathcal{E}
	  + \mathcal{A}
	  + \mathcal{A} \times \mathcal{A}
	  + \mathcal{A} \times \mathcal{A} \times \mathcal{A}
	  + \dotsb
      \end{equation*}
      Por la segunda parte
      y la serie geométrica~\eqref{eq:serie-geometrica},
      la función generatriz correspondiente es:
      \begin{equation*}
	1 + A(z) + A^2(z) + A^3(z) + \dotsb
	  = \frac{1}{1 - A(z)}
      \end{equation*}
    \item % A o B
      Si elegimos \(\alpha \in \mathcal{A}\),
      vemos que debemos reemplazar sus \(\lvert \alpha \rvert\)~átomos
      por elementos de \(\mathcal{B}\)
      en el orden indicado por la estructura de \(\alpha\).
      Pero tal secuencia es enumerada por \(B(z)^{\lvert \alpha \rvert}\),
      con lo que al sumar:
      \begin{equation*}
	\sum_{\alpha \in \mathcal{A}} B(z)^{\lvert \alpha \rvert}
	  = A(B(z)
      \end{equation*}
    \item % Set(A)
      La clase de los subconjuntos finitos de \(\mathcal{A}\)
      queda representada por el producto simbólico:
      \begin{equation*}
	\prod_{\alpha \in \mathcal{A}} (\mathcal{E} + \{\alpha\})
      \end{equation*}
      ya que
      al distribuir los productos de todas las formas posibles
      aparecen todos los subconjuntos de \(\mathcal{A}\).
      Directamente obtenemos entonces:
      \begin{equation*}
	\prod_{\alpha \in \mathcal{A}}
	    \left( 1 + z^{\lvert \alpha \rvert} \right)
	  = \prod_{n \ge 0} (1 + z^n)^{a_n}
      \end{equation*}
      Otra forma de verlo es que cada elemento de tamaño \(n\)
      aporta un factor \(1 + z^n\),
      si hay \(a_n\) de estos
      el aporte total es \((1 + z^n)^{a_n}\).
      Esta es la primera parte de lo aseverado.
      Aplicando logaritmo:
      \begin{align*}
	\sum_{\alpha \in \mathcal{A}}
	    \ln \left(1 + z^{\lvert \alpha \rvert} \right)
	  &= -\sum_{\alpha \in \mathcal{A}}
		\sum_{k \ge 1}
		  \frac{(-1)^k z^{\lvert \alpha \rvert k}}{k}  \\
	  &= -\sum_{k \ge 1} \frac{(-1)^k}{k} \,
		\sum_{\alpha \in \mathcal{A}}
		  z^{\lvert \alpha \rvert k} \\
	  &= \sum_{k \ge 1} \frac{(-1)^{k + 1} \, A(z^k)}{k}
      \end{align*}
      Exponenciando lo último
      resulta equivalente a la segunda parte.
    \item % A o B
      Para cada objeto \(\alpha \in \mathcal{A}\)
      esto se traduce
      en una secuencia de \(\lvert \alpha \rvert\) elementos de \(\mathcal{B}\)
      a ser reemplazados por los átomos de \(\alpha\),
      con lo que la función generatriz respectiva es:
      \begin{equation*}
	\sum_{\alpha \in \mathcal{A}} B(z)^{\lvert \alpha \rvert}
      \end{equation*}
      que es lo prometido.
    \item % MSet(A)
      Podemos considerar un multiconjunto finito
      como la combinación de una secuencia
      para cada tipo de elemento:
      \begin{equation*}
	\prod_{\alpha \in \mathcal{A}} \Seq(\{ \alpha \})
      \end{equation*}
      La función generatriz buscada es:
      \begin{equation*}
	\prod_{\alpha \in \mathcal{A}}
	  \frac{1}{1 - z^{\lvert \alpha \rvert}}
	  = \prod_{n \ge 0} \frac{1}{(1 - z^n)^{a_n}}
      \end{equation*}
      Esto provee la primera parte.
      Nuevamente aplicamos logaritmo para simplificar:
      \begin{align*}
	\ln \prod_{\alpha \in \mathcal{A}}
	       \frac{1}{1 - z^{\lvert \alpha \rvert}}
	  &= - \sum_{\alpha \in \mathcal{A}}
		 \ln \left( 1 - z^{\lvert \alpha \rvert} \right) \\
	  &= \sum_{\alpha \in \mathcal{A}}
	       \sum_{k \ge 1}
		 \frac{z^{k \lvert \alpha \rvert}}{k} \\
	  &= \sum_{k \ge 1}
	       \frac{1}{k} \,
		 \sum_{\alpha \in \mathcal{A}}
		   z^{k \lvert \alpha \rvert} \\
	  &= \sum_{k \ge 1}
	       \frac{A(z^k)}{k}
      \end{align*}
    \item % Cyc(A)
      Esta situación es más compleja de tratar,
      la discutiremos en la sección~\ref{sec:ogf-ciclo} más abajo.
      \qedhere
    \end{enumerate}
  \end{proof}
  La utilidad del teorema~\ref{theo:ms-OGF}
  es que de cómo construir
  la clase de objetos que nos interesa
  da directamente una ecuación satisfecha por la función generatriz.
  Claro que igual resta extraer los coeficientes,
  tarea en la cual la fórmula de inversión de Lagrange
  (teorema~\ref{theo:LIF})%
    \index{Lagrange, inversion de@Lagrange, inversión de}
  es invaluable.

\subsection{Algunas aplicaciones}
\label{sec:ms-ogf-aplicaciones}

  La clase de los árboles binarios \(\mathcal{B}\)
  es por definición es la unión disjunta del árbol vacío
  y la clase de tuplas de un nodo (la raíz)
  y dos árboles binarios.%
    \index{arbol binario@árbol binario}
  O sea:
  \begin{equation*}
    \mathcal{B}
      = \mathcal{E}
	  + \mathcal{Z} \times \mathcal{B} \times \mathcal{B}
  \end{equation*}
  de donde directamente
  igual que antes obtenemos:
  \begin{equation*}
    B(z)
      = 1 + z B^2(z)
  \end{equation*}
  Con el cambio de variable \(u(z) = B(z) - 1\) queda:
  \begin{equation*}
    u(z)
      = z (1 + u(z))^2
  \end{equation*}
  Es aplicable la fórmula de inversión de Lagrange,%
    \index{Lagrange, inversion de@Lagrange, inversión de}
  teorema~\ref{theo:LIF},
  con \(\phi(u) = (u + 1)^2\) y \(f(u) = u\):
  \begin{align*}
    \left[ z^n \right] u(z)
      &= \frac{1}{n} \, \left[ u^{n - 1} \right] \, \phi(u)^n \\
      &= \frac{1}{n} \, \left[ u^{n - 1} \right] \, (u + 1)^{2 n} \\
      &= \frac{1}{n} \, \left[ u^{n - 1} \right] \,
	   \sum_{k \ge 0} \binom{2 n}{k} u^k \\
      &= \frac{1}{n} \, \binom{2 n}{n - 1}
  \end{align*}
  Tenemos,
  como \(u(z) = B(z) - 1\) y sabemos que \(b_0 = 1\):
  \begin{equation*}
    b_n =
    \begin{cases}
      \displaystyle
	\frac{1}{n}\binom{2 n}{n - 1}
	  = \frac{1}{n + 1} \, \binom{2 n}{n}
	       & \text{si \(n \ge 1\)} \\
      1
	       & \text{si \(n = 0\)}
    \end{cases}
  \end{equation*}
  Casualmente la expresión simplificada para \(n \ge 1\)
  da el valor correcto \(b_0 = 1\).
  A estos números ya los habíamos mencionado
  en~\eqref{eq:Catalan-numbers},%
    \index{Catalan, numeros de@Catalan, números de}
  son los números de Catalan.
  Es \(b_n = C_n\).

  Sea ahora \(\mathcal{A}\)
  la clase de \emph{árboles con raíz ordenados},%
    \index{arbol con raiz@árbol con raiz!ordenado}
  formados por un nodo raíz
  conectado a las raíces de una secuencia de árboles ordenados.
  La idea es que la raíz tiene hijos en un cierto orden.
  Simbólicamente:
  \begin{equation*}
    \mathcal{A}
      = \mathcal{Z} \times \Seq(\mathcal{A})
  \end{equation*}
  El método simbólico entrega directamente la ecuación:
  \begin{equation*}
    A(z)
      = \frac{z}{1 - A(z)}
  \end{equation*}
  Nuevamente es aplicable la fórmula de inversión de Lagrange,%
    \index{Lagrange, inversion de@Lagrange, inversión de}
  con \(\phi(A) = (1 - A)^{-1}\) y \(f(A) = A\):
  \begin{align*}
    \left[ z^n \right] A(z)
      &= \frac{1}{n} \, \left[ A^{n - 1} \right] \, \phi(A)^n \\
      &= \frac{1}{n} \, \left[ A^{n - 1} \right] \, (1 - A)^{-n} \\
      &= \frac{1}{n} \, \binom{2 n - 2}{n - 1} \\
      &= C_{n - 1}
  \end{align*}
  Otra vez números de Catalan.%
    \index{Catalan, numeros de@Catalan, números de}
  Un combinatorista de verdad considerará esto
  como el desafío de encontrar una biyección entre árboles binarios
  y árboles ordenados,
  nosotros nos contentaremos con consignar el resultado.

  La manera obvia de representar \(\mathbb{N}_0\)
  es mediante secuencias de marcas,
  como \(||||\) para 4;
  simbólicamente \(\mathbb{N}_0 = \Seq(\mathcal{Z})\).
  Para calcular el número de multiconjuntos de \(k\) elementos
  tomados entre \(n\),%
    \index{multiconjunto!numero@número}
  un multiconjunto queda representado
  por las cuentas de los \(n\) elementos de que se compone,
  y eso corresponde a:
  \begin{equation*}
    \mathbb{N}_0 \times \dotsb \times \mathbb{N}_0
      = (\Seq(\mathcal{Z}))^n
  \end{equation*}
  Para obtener el número que nos interesa:
  \begin{equation*}
    \multiset{n}{k}
      = \left[ z^k \right] (1 - z)^{-n}
      = (-1)^n \binom{-n}{k}
      = \binom{n + k - 1}{n}
  \end{equation*}
  Este resultado ya lo dedujimos
  en el capítulo~\ref{cha:combinatoria-elemental}.

  Consideremos \emph{árboles~\(2\)-\(3\)},%
     \index{arbol 2-3@árbol \(2\)-\(3\)}
  constando de un único nodo,
  o de un nodo conectado a \(2\) o \(3\) árboles~\(2\)-\(3\).
  Estos son de interés como estructuras de datos,
  dado que es fácil mantenerlos balanceados,
  de forma que todas las hojas estén a la misma distancia de la raíz.
  Vemos que de un árbol~\(2\)-\(3\) balanceado obtenemos uno mayor
  reemplazando simultáneamente todas las hojas por dos o tres nodos.
  Si consideramos el tamaño del árbol \(2\)-\(3\) como el número de sus hojas,
  descritas por la clase \(\mathcal{D}\),
  esto lleva a la ecuación simbólica:
  \begin{equation*}
    \mathcal{D}
      = \mathcal{Z}
	  + \mathcal{D}
	      \circ \left( \mathcal{Z}^2 + \mathcal{Z}^3 \right)
  \end{equation*}
  que nos da la ecuación funcional:
  \begin{equation}
    \label{eq:balanced-2-3-trees}
    D(z)
      = z + D(z^2 + z^3)
  \end{equation}
  Es claro que substituir \(s\) veces
  partiendo de la estimación inicial \(D(z) = z\)
  nos entrega hasta el coeficiente de \(z^{2 s}\).
  Resulta:
  \begin{equation*}
    D(z)
      = z + z^2 + z^3 + z^4 + 2 z^5 + 2 z^6 + 3 z^7 + 4 z^8 + 5 z^9
	 + 8 z^{10} + 14 z^{11} + 23 z^{12} + 32 z^{13} + 43 z^{14}
	 + \dotsb
  \end{equation*}
  El comportamiento de los coeficientes
  de soluciones de ecuaciones como~\eqref{eq:balanced-2-3-trees}
  es bastante extraño,
  Odlyzko~\cite{odlyzko82:_period_oscil_coeff_power_series}
  demuestra que oscilan y da rangos.

  Los \emph{árboles con raíz} constan de un nodo raíz
  conectado a una colección de árboles con raíz.%
    \index{arbol con raiz@árbol con raiz}
  La clase \(\mathcal{R}\) correspondiente cumple:
  \begin{equation*}
    \mathcal{R}
      = \mathcal{Z} \times \MSet(\mathcal{R})
  \end{equation*}
  Para la función generatriz queda:
  \begin{equation}
    \label{eq:rooted-tree-fe}
    R(z)
      = z \, \exp \left( \sum_{k \ge 1} \frac{R(z^k)}{k} \right)
  \end{equation}
  Ciertamente es una ecuación harto fea,
  pero puede usarse para obtener sucesivamente los \(r_n\).

  Una manera general
  de atacar ecuaciones como~\eqref{eq:rooted-tree-fe}
  es ver que da \(R\) en términos de \(R\),
  y comenzar con alguna aproximación
  para ir refinándola.
  De partida,
  vemos que \(r_0 = 0\),
  con lo que tenemos una aproximación inicial \(R^{(0)}(z) = 0\).
  Substituyendo en~\eqref{eq:rooted-tree-fe}
  obtenemos \(R^{(1)}(z) = z\).
  Como al substituir \(R^{(1)}(z)\) en~\eqref{eq:rooted-tree-fe}
  ya no aparecerán nuevos términos en \(z\),
  sabemos que \(r_1 = 1\).
  De la misma forma,
  cuando substituyamos \(R^{(n)}(z)\) en~\eqref{eq:rooted-tree-fe}
  ya no aparecerán nuevas contribuciones al coeficiente de \(z^n\),
  y este proceso converge
  según definimos en el capítulo~\ref{cha:series-formales}.
  No tiene sentido retener más términos al calcular \(R^{(n)}(z)\),
  los demás no influyen sobre el coeficiente de \(z^n\).
  Armado con un paquete de álgebra simbólica%
    \index{algebra simbolica@álgebra simbólica}
  o mucha paciencia
  se pueden calcular términos adicionales:
  \begin{equation}
    \label{eq:rooted-tree-gf}
    R(z)
      = z + z^2 + 2 z^3 + 4 z^4 + 9 z^5 + 20 z^6 + 48 z^7
	  + 115 z^8 + \dotsb
  \end{equation}
  Se pueden extraer estimaciones asintóticas
  de~\eqref{eq:rooted-tree-fe},
  ver por ejemplo a Flajolet y Sedgewick~%
    \cite{flajolet09:_analy_combin}
  o a Knuth~\cite{knuth11:_combin_alg_1},
  pero las técnicas a emplear escapan con mucho a nuestro ámbito.
  Ya Pólya~%
    \cite{polya87:_combin_enumer_group_graph_chemic_compoun}
  encontró:
  \begin{equation*}
    r_n
     \sim 0,4399 \cdot 2,9558^n \cdot n^{- 3 / 2}
  \end{equation*}

  Una \emph{combinación} de \(n\) es expresarlo como una suma.%
    \index{numero natural@número natural!combinacion@combinación}
  Por ejemplo,
  hay \(8\) combinaciones de \(4\):
  \begin{equation*}
      4
	= 3 + 1
	= 2 + 2
	= 2 + 1 + 1
	= 1 + 3
	= 1 + 2 + 1
	= 1 + 1 + 2
	= 1 + 1 + 1 + 1
  \end{equation*}
  Llamemos \(c(n)\) al número de combinaciones de \(n\).

  A la clase de los naturales
  podemos representarla como de secuencias de marcas.
  Por ejemplo, \(5\) es \(|||||\).
  Son secuencias no vacías,
  el primer natural es \(1\).
  Así:
  \begin{equation*}
    \mathbb{N}
      = \mathcal{Z} \times \Seq(\mathcal{Z})
  \end{equation*}
  Otra forma de representar al natural \(n\)
  es mediante una bolsa de \(n\) piedritas,
  que sugiere:
  \begin{equation*}
    \mathbb{N}
      = \mathcal{Z} \times \MSet(\mathcal{Z})
  \end{equation*}
  Las reglas de transferencia del teorema~\ref{theo:ms-OGF}
  en este caso particular
  dan la misma función generatriz para ambas:
  \begin{equation*}
    N(z)
      = \frac{z}{1 - z}
  \end{equation*}

  A su vez,
  una combinación no es más que una secuencia de naturales:
  \begin{equation*}
    \mathcal{C}
      = \Seq(\mathbb{N})
  \end{equation*}
  Directamente resulta:
  \begin{align*}
    C(z)
      &= \sum_{n \ge 0} c(n) z^n \\
      &= \frac{1}{1 - N(z)} \\
      &= \frac{1}{2} + \frac{1}{2} \cdot \frac{1}{1 - 2 z} \\
    c(n)
      &= \frac{1}{2} \, [n = 0] + \frac{1}{2} \cdot 2^n \\
      &= \frac{1}{2} \, [n = 0] + 2^{n - 1}
  \end{align*}
  Esto es consistente con \(c(4) = 8\) obtenido arriba.

% palabras-sin-patrones.tex
%
% Copyright (c) 2013-2014 Horst H. von Brand
% Derechos reservados. Vea COPYRIGHT para detalles

\subsection{Palabras que no contienen un patrón dado}
\label{sec:strings-excluding-pattern}
\index{palabra!numero@número}

  Nos interesan secuencias que no contengan un patrón dado.%
    \index{palabra!patron@patrón}
  Un ejemplo simple es secuencias binarias sin ceros seguidos.
  Llamemos \(\mathcal{B}_{00}\) a esta clase.
  Un elemento de \(\mathcal{B}_{00}\) puede ser vacío o \(0\),
  o es \(1\) o \(01\)
  seguido por un elemento de \(\mathcal{B}_{00}\).
  O sea:
  \begin{equation*}
    \mathcal{B}_{00}
      = \mathcal{E} + \{0\} + \{1, 01\} \times \mathcal{B}_{00}
  \end{equation*}
  Si \(z\) marca cada símbolo,
  para la respectiva función generatriz ordinaria \(B_{00}(z)\):%
    \index{generatriz}
  \begin{align*}
    B_{00}(z)
      &= 1 + z + (z + z^2) B_{00}(z) \\
  \intertext{Despejando:}
    B_{00}(z)
      &= \frac{1 + z}{1 - z - z^2}
  \end{align*}
  Resulta ser
    \(\left[ z^n \right] B_{00}(z) = F_n + F_{n + 1} = F_{n + 2}\),
  un número de Fibonacci.%
    \index{Fibonacci, numeros de@Fibonacci, números de}

  Si ahora buscamos que no contenga \(k\) ceros seguidos,
  podemos expresar:
  \begin{equation*}
    \mathcal{B}_{0^k}
      = \mathcal{P}_{< k}
	  + \mathcal{P}_{< k}
	      \times \{ 1 \} \times \mathcal{B}_{0^k}
  \end{equation*}
  Acá \(\mathcal{P}_{< k}\)
  es la clase de secuencias de menos de \(k\) ceros:
  \begin{equation*}
    \mathcal{P}_{< k}
      = \mathcal{E} + \{0\} + \{0\}^2 + \dotsb + \{0\}^{k - 1}
  \end{equation*}
  Las respectivas funciones generatrices ordinarias cumplen:
  \begin{align}
    P_{< k} (z)
      &= 1 + z + z^2 + \dotsb + z^{k - 1} \notag \\
      &= \frac{1 - z^k}{1 - z} \notag \\
    B_{0^k} (z)
      &= \frac{1 - z^k}{1 - z} (1 + z B_{0^k} (z)) \notag \\
  \intertext{Despejando:}
    B_{0^k} (z)
      &= \frac{1 - z^k}{1 - 2 z + z^{k + 1}}
	   \label{eq:B0^k}
  \end{align}
  Podemos extraer información adicional de acá.
  Los coeficientes de \(B_{0^k}(z)\)
  son el número de \foreignlanguage{english}{strings}
  que no contienen \(0^k\),
  el coeficiente de \(z^n\)
  dividido por \(2^n\) es la proporción del total:
  \begin{align*}
    B_{0^k}(z)
      &= \sum_{n \ge 0}
	   \text{\{\# de largo \(n\) sin \(0^k\)\}} z^n \\
    B_{0^k}(z / 2)
      &= \sum_{n \ge 0}
	   \text{\{\# de largo \(n\) sin \(0^k\)\}} / 2^n z^n \\
    B_{0^k}(1 / 2)
      &= \sum_{n \ge 0}
	   \text{\{\# de largo \(n\) sin \(0^k\)\}} / 2^n \\
      &= \sum_{n \ge 0}
	   \Pr( \text{No hay \(0^k\) en los primeros \(n\)}) \\
      &= \sum_{n \ge 0}
	   \Pr( \text{Primer \(0^k\)
		      termina después de \({} > n\)} ) \\
      &= \text{Posición esperada
	       del fin de los primeros \(k\) ceros}
  \end{align*}
  A esto se le llama \emph{tiempo de espera}%
    \index{palabra!tiempo de espera|textbfhy}
  (en inglés,
   \emph{\foreignlanguage{english}{waiting time}}).%
     \index{palabra!waiting time@\emph{\foreignlanguage{english}{waiting time}}|see{palabra!tiempo de espera}}
  Resulta:
  \begin{theorem}
    \label{theo:waiting-time-2^k}
    El tiempo de espera para los primeros \(k\) ceros
    en un \foreignlanguage{english}{string} binario al azar es:
    \begin{equation}
      \label{eq:waiting-time-2^k}
      B_{0^k}(1/2)
	= 2^{k + 1} - 2
    \end{equation}
  \end{theorem}
  O sea,
  en promedio hay que esperar \(30\)~bits hasta hallar \(0000\).
  La pregunta obvia es si esto vale también para otros patrones
  de largo cuatro,
  por ejemplo \(0001\).
  Resulta que no es así.
  Consideremos la primera vez que aparece \(000\).
  Es igualmente probable que continúe \(0000\) o \(0001\).
  Si \(0000\) no calza es \(0001\),
  y para \(0000\) debemos esperar al menos \(4\)~bits más.
  Si \(0001\) no calza,
  es porque es \(0000\) y el próximo bit puede completar \(0001\).

  Consideremos un patrón \(p\) de largo \(k\) arbitrario
  tomados entre \(s\) símbolos entonces.
  Sea \(\mathcal{B}_p\)
  la clase de \foreignlanguage{english}{strings}
  que no contienen \(p\),
  y sea \(\mathcal{T}_p\)
  la clase de \foreignlanguage{english}{strings}
  que terminan en \(p\),
  pero en los cuales \(p\) no aparece nunca antes del final.
  Es claro que \(\mathcal{B}_p\) y \(\mathcal{T}_p\) son disjuntos.
  Si agregamos un símbolo a un \foreignlanguage{english}{string}
  en \(\mathcal{B}_p\),
  el resultado es un \foreignlanguage{english}{string} no vacío
  en \(\mathcal{B}_p\) o en \(\mathcal{T}_p\).
  O sea:%
    \index{metodo simbolico@método simbólico}
  \begin{equation*}
    \mathcal{B}_p + \mathcal{T}_p
      = \mathcal{E} + \mathcal{B}_p \times \{ 0, 1, \dotsc, s - 1 \}
  \end{equation*}
  Esto nos da la ecuación funcional
  para las respectivas funciones generatrices ordinarias:%
    \index{generatriz}
  \begin{equation}
    \label{eq:B+T-fe}
    B_p(z) + T_p(z)
      = 1 + s z B_p(z)
  \end{equation}
  Hace falta determinar \(\mathcal{T}_p\).
  Es similar a \(\mathcal{B}_p \times \{ p \}\),
  pero debemos considerar que un elemento de \(\mathcal{B}_p\)
  puede terminar en ``casi'' \(p\),
  con lo que solo le falta una cola.

  El desarrollo que sigue
  es básicamente de Odlyzko~\cite{odlyzko85:_enum_strings}.
  Escribiremos \(\lvert x \rvert\)
  para el largo del \foreignlanguage{english}{string} \(x\)
  (el número de símbolos que lo componen).
  Para describir la manera
  en que dos \foreignlanguage{english}{strings} se traslapan
  definimos la \emph{correlación}%
    \index{palabra!correlacion@correlación|textbfhy}
  entre los \foreignlanguage{english}{string} \(x\) e \(y\)
  (posiblemente de distinto largo)
  como el polinomio \(c_{x y}(t)\) de grado \(\lvert x \rvert - 1\)
  tal que el coeficiente de \(t^k\)
  se determina ubicando \(y\) bajo \(x\)
  de manera que el primer caracter de \(y\)
  cae bajo el \(k\)\nobreakdash-ésimo caracter de \(x\).
  El coeficiente es \(1\) si ambos son iguales donde traslapan,
  \(0\) en caso contrario.
  Por ejemplo,
  si \(x = \mathtt{c a b c a b c}\)
  e \(y = \mathtt{a b c a b c d e}\),
  resulta \(c_{x y}(t) = t^4 + t\),
  como muestra el cuadro~\ref{tab:cxy}.
  \begin{table}[ht]
    \centering
    \begin{tabular}{>{\(}l<{\)}*{16}{>{\(\mathtt}c<{\)}}>{\(}r<{\)}}
      x: & c & a & b & c & a & b & c &	 &   &	 &   &	 &   &	 &   \\
      y: & a & b & c & a & b & c & d & e &   &	 &   &	 &   &	 & 0 \\
	 &   & a & b & c & a & b & c & d & e &	 &   &	 &   &	 & 1 \\
	 &   &	 & a & b & c & a & b & c & d & e &   &	 &   &	 & 0 \\
	 &   &	 &   & a & b & c & a & b & c & d & e &	 &   &	 & 0 \\
	 &   &	 &   &	 & a & b & c & a & b & c & d & e &   &	 & 1 \\
	 &   &	 &   &	 &   & a & b & c & a & b & c & d & e &	 & 0 \\
	 &   &	 &   &	 &   &	 & a & b & c & a & b & c & d & e & 0
    \end{tabular}
    \caption{Cálculo de $c_{x y} (t) = t^4 + t$}
    \label{tab:cxy}
  \end{table}
  Nótese que en general \(c_{x y}(t) \ne c_{y x}(t)\)
  (en el ejemplo es \(c_{y x}(t) = 0\)).
  De particular interés
  es la \emph{autocorrelación} \(c_x(t) = c_{x x}(t)\),
  la correlación de un \foreignlanguage{english}{string}
  consigo mismo.%
    \index{palabra!autocorrelacion@autocorrelación|textbfhy}
  En el ejemplo,
  \(c_x (t) = t^6 + t^3 + 1\).

  Fijemos un patrón \(p\) de largo \(k\),
  y escribamos:
  \begin{align*}
    B_p(z)
      &= \sum_{n \ge 0} b_n z^n \\
    T_p(z)
      &= \sum_{n \ge 0} t_n z^n
  \end{align*}
  Consideremos uno de los \(b_n\)
  \foreignlanguage{english}{strings} de largo \(n\)
  que no terminan en \(p\),
  y adosemos \(p\) al final.
  Sea \(n + r\) la posición
  en la cual por primera vez termina \(p\) en el resultado,
  donde \(0 < r \le k\).
  Como \(p\) también aparece al final,
  deben coincidir el prefijo de largo \(k - r\) de \(p\)
  y el sufijo de largo \(k - r\) de \(p\),
  o sea,
  \(\left[ t^{k - r} \right] c_p(t) = 1\).

  Para un ejemplo,
  sea el patrón \(p = \mathtt{a a b a}\)
  y el \foreignlanguage{english}{string}
    \(x = \mathtt{a b a b b a a b} \in \mathcal{B}_p\).
  Es \(k = \lvert p \rvert = 4\)
  y \(n = \lvert x \rvert = 8\).
  Vemos que
    \(x p = \mathtt{a b a b b
		    \textcolor{red}{a a b a}
		    \textcolor{blue}{a b a}}\),
  o sea,
  \(r = 1\)
  (hay un traslapo de \(k - r = 4 - 1 = 3\)
   entre el principio del patrón
   y el final del \foreignlanguage{english}{string}).
  Tenemos un miembro de \(\mathcal{T}_p\) de largo \(n + r = 9\)
  y una cola de largo \(k - r = 3\),
  determinados en forma única por \(x\) y \(p\).
  Esta descomposición solo es posible cuando
  \(\left[ t^{k - r} \right] c_p (t) = 1\).

  Nos interesa contar estos \foreignlanguage{english}{string}.
  Hay \(t_{n + r}\) de ellos,
  la descomposición descrita es una biyección.
  Vale decir,
  como los coeficientes de \(c_p\) son cero o uno:
  \begin{equation}
    \label{eq:bn=tnc}
    b_n
      = \sum_{0 < r \le k}
	  t_{n + r} \left[ t^{k - r} \right] c_p (t)
  \end{equation}
  Multiplicando~\eqref{eq:bn=tnc}
  por \(z^{n + k}\) y sumando para \(n \ge 0\)
  (recordar que \(k = \lvert p \rvert \ge \deg(c_p(t)) + 1\))
  da:
  \begin{align}
    B_p(z) z^k
      &= \sum_{n \ge 0}
	   z^{n + k}
	   \sum_{0 < r \le k}
	     t_{n + r}
	     \left[ t^{k - r} \right] c_p (t)
		 \notag \\
  \intertext{Esta es la convolución de \(T_p(z)\) con \(c_p(z)\):}
    B_p(z) z^k
      &= T_p(z) c_p(z) \label{eq:T-fe}
  \end{align}

  Uniendo las piezas anteriores
  tenemos un resultado de Solov'ev~%
    \cite{solovev66:_combin_ident_its_applic_probl}:
  \begin{theorem}
    \label{theo:Bp-gf}
    Sea \(p\) un patrón de largo \(k\) formado por \(s\) símbolos,
    con autocorrelación \(c_p(z)\).
    Entonces
    el número de \foreignlanguage{english}{strings} de largo \(n\)
    que no contienen el patrón \(p\)
    tiene función generatriz ordinaria:
    \begin{equation}
      \label{eq:Bp-gf}
      B_p(z)
	= \frac{c_p(z)}{(1 - s z) c_p(z) + z^k}
    \end{equation}
    El tiempo de espera para el patrón \(p\)%
      \index{palabra!tiempo de espera}
    está dado por:
    \begin{equation}
      \label{eq:WT-p}
      W_p
	= s^k c_p (1 / s)
    \qedhere
    \end{equation}
  \end{theorem}
  \begin{proof}
    La ecuación~\eqref{eq:Bp-gf}
    es la solución del sistema de ecuaciones~\eqref{eq:B+T-fe}
    y~\eqref{eq:T-fe}.
    El tiempo de espera
    es como se discutió antes para el patrón \(0^k\),
    la expresión dada
    resulta de substituir \(z = 1 / s\) en~\eqref{eq:Bp-gf}.
  \end{proof}
  Incidentalmente,
  al ser \(c_p\)
  un polinomio de coeficientes enteros de grado menor a \(k\),
  por~\eqref{eq:WT-p} el tiempo de espera siempre es un entero.

  Completando la discusión previa,
  tenemos \(c_{0000} (z) = 1 + z + z^2 + z^3\)
  y \(c_{0001} (z) = 1\).
  El tiempo de espera para el patrón \(p\)
  está dado por \(2^4 c_p(1/2)\).
  Para nuestros dos patrones,
  por las fórmulas desarrolladas antes:
  \begin{align}
    B_{0000}(z)
      &= \frac{1 - z^4}{1 - 2 z + z^5}
	      \label{B0000-fg} \\
    W_{0000}
      &= 30   \label{B0000-wt} \\
    B_{0001}(z)
      &= \frac{1}{1 - 2 z + z^4}
	      \label{B0001-fg} \\
    W_{0001}
      &= 16   \label{B0001-wt}
  \end{align}
  Por técnicas similares se pueden manejar conjuntos de patrones.

%%% Local Variables:
%%% mode: latex
%%% TeX-master: "clases"
%%% End:


\subsection{Construcción ciclo}
\label{sec:ogf-ciclo}

  El tratamiento de \(\Cyc(\mathcal{A})\)
  en el teorema~\ref{theo:ms-OGF}
  requiere considerar simetrías en la secuencia subyacente.%
    \index{ciclo!simetria@simetría}
  Por ejemplo,
  el ciclo \((a b c d)\) resulta de las cuatro secuencias
  \(a b c d\), \(b c d a\), \(c d a b\) y \(d a b c\);
  pero el ciclo \((a b a b)\) resulta solo de las dos
  \(a b a b\) y \(b a b a\).
  Por ahora
  hablaremos de secuencias y ciclos de símbolos,
  para luego aplicar lo aprendido a clases
  y sus funciones generatrices.
  Usaremos conceptos de funciones aritméticas
  y el anillo de Dirichlet%
    \index{Dirichlet, anillo de}
  en lo sucesivo,
  véase la sección~\ref{sec:estructura-Un}.

  Nuestro desarrollo sigue a Flajolet y Soria~%
    \cite{flajolet91:_cycle_constr}.
  Llamemos \emph{secuencia primitiva} a una secuencia
  que no es la repetición de una secuencia más corta.%
    \index{secuencia!primitiva|textbfhy}
  O sea,
  \(a b a a b\) es primitiva,
  \(a b a b = (a b)^2\) no lo es.
  La secuencia más corta
  tal que la secuencia dada
  se puede escribir como una repetición de ella
  la llamamos la \emph{raíz} de la secuencia.%
    \index{secuencia!raiz@raíz|textbfhy}
  En el ejemplo,
  la raíz de \(a b a a b\) es \(a b a a b\),
  ya que es primitiva;
  la raíz de \(a b a b\) es \(a b\).
  Sea \(s\) el número de símbolos en el alfabeto
  sobre el cual se consideran las secuencias.
  Si llamamos \(p_n\)
  al número de secuencias primitivas de largo \(n\),
  como toda secuencia es la repetición de una secuencia primitiva,
  debe ser:
  \begin{equation*}
    s^n
      = \sum_{d \mid n} p_d
  \end{equation*}
  El lado izquierdo es el número total de secuencias de largo \(n\),
  el lado derecho cuenta este mismo número
  como secuencias primitivas de largo \(d\)
  que se repiten para dar el largo \(n\).
  Inversión de Möbius,
  teorema~\ref{theo:Moebius-inversion},
  entrega:
  \begin{equation*}
    p_n
      = \sum_{d \mid n} \mu(d) s^{n / d}
  \end{equation*}
  Necesitamos extender este resultado
  a las funciones generatrices respectivas.
  \begin{lemma}[Inversión de Möbius]
    \index{Mobius, inversion de@Möbius, inversión de|textbfhy}
    \label{lem:GF-Moebius-inversion}
    Sean secuencias \(\langle u_n \rangle_{n \ge 0}\)
    y \(\langle v_n \rangle_{n \ge 0}\)
    tales que para \(n \ge 1\):
    \begin{equation*}
      u_n
	= \sum_{d \mid n} v_d
    \end{equation*}
    Entonces las funciones generatrices ordinarias
    respectivas \(U(z)\) y \(V(z)\)
    cumplen:
    \begin{align*}
      U(z)
	&= \sum_{n \ge 1} V(z^n) \\
      V(z)
	&= \sum_{n \ge 1} \mu(n) U(z^n)
    \end{align*}
  \end{lemma}
  \begin{proof}
    Primero,
    usando la convención de Iverson:%
      \index{Iverson, convencion de@Iverson, convención de}
    \begin{align*}
      U(z)
	&= \sum_{n \ge 1} u_n z^n \\
	&= \sum_{n \ge 1}
	     \sum_{a \ge 1} \sum_{b \ge 1} [a b = n] v_a z^n \\
	&= \sum_{a \ge 1} \sum_{b \ge 1} v_a z^{a b}
	     \sum_{n \ge 1} [a b = n] \\
	&= \sum_{b \ge 1} \sum_{a \ge 1} v_a (z^b)^a \\
	&= \sum_{b \ge 1} V(z^b)
     \end{align*}
     De la misma forma:
     \begin{align*}
       V(z)
	 &= \sum_{n \ge 1} \sum_{a \ge 1} \sum_{b \ge 1} [a b = n]
	      \mu(a) u_b z^n \\
	 &= \sum_{a \ge 1} \mu(a) \sum_{b \ge 1} u_b z^{a b} \\
	 &= \sum_{a \ge 1} \mu(a) U(z^a)
       \qedhere
     \end{align*}
  \end{proof}
  Queda claro que
  si \(\alpha(n)\) tiene inversa de Dirichlet \(\alpha^{-1}(n)\) y:
  \begin{align}
    u_n
      &= \sum_{d \mid n} \alpha(n / d) v_d \notag \\
    v_n
      &= \sum_{d \mid n} \alpha^{-1}(n / d) u_d \notag \\
    \intertext{entonces:}
    U(z)
      &= \sum_{n \ge 1} \alpha(n) V(z^n)
	     \label{eq:Dirichlet-sum-gf} \\
    V(z)
      &= \sum_{n \ge 1} \alpha^{-1}(n) U(z^n)
	     \label{eq:Dirichlet-inversion-gf}
  \end{align}

  Igual que en el caso de secuencias
  podemos hablar de \emph{ciclos primitivos}%
    \index{ciclo!primitivo}
  y sus \emph{raíces},
  nos interesa la relación entre secuencias y ciclos.
  \begin{figure}[ht]
    \centering
    \subfloat[Raíz \(a\)]{\pgfimage{images/6cycle-1}}%
    \hspace{3em}%
    \subfloat[Raíz \(a b\)]{\pgfimage{images/6cycle-2}}%
    \hspace{3em}%
    \subfloat[Raíz \(a a b\)]{\pgfimage{images/6cycle-3}}%
    \hspace{3em}%
    \subfloat[Primitivo]{\pgfimage{images/6cycle-6}}
    \caption{Ciclos de largo seis}
    \label{fig:6cycles}
  \end{figure}
  La figura~\ref{fig:6cycles} muestra algunos ciclos de largo seis
  con sus raíces.

  Sea \(\omega \) una secuencia primitiva de largo \(l\),
  y consideremos el ciclo formado por \(r\) copias de \(\omega\),
  \(\omega^r\).
  Si lo rotamos en un múltiplo de \(l\) posiciones
  obtenemos el original.
  Si lo rotamos en menos de \(l\) posiciones,
  el efecto es dividir \(\omega = \alpha \beta\)
  y trasladar \(\alpha\) al final,
  queda \(\beta (\alpha \beta)^{r - 1} \alpha = (\beta \alpha)^r\),
  nuevamente un ciclo con raíz de largo \(l\).
  Esto ocurre al rotar en cualquier número de posiciones
  que no es un múltiplo de \(l\).
  Si la secuencia es primitiva,
  todas las rotaciones de la misma también lo son.
  Vale decir,
  hay un mapa \(1\) a \(l\) de secuencias primitivas de largo \(l\)
  a ciclos primitivos de largo \(l\).
  A su vez,
  todo ciclo es la repetición de un ciclo primitivo
  (su raíz).

  Traduzcamos lo anterior a funciones generatrices ahora.
  Consideremos primeramente
  secuencias de al menos un \(\mathcal{A}\).
  Si además del tamaño total
  nos interesa el número de \(\mathcal{A}\) componentes,
  usando una clase auxiliar \(\mathcal{U}\)
  con un único elemento de tamaño uno
  podemos representar la clase \(\mathcal{S}\)
  que nos interesa como:
  \begin{equation*}
    \mathcal{S}
      = \Seq_{\ge 1} (\mathcal{U} \times \mathcal{A})
  \end{equation*}
  De esto,
  usando \(u\)
  para contar el número de \(\mathcal{U}\) participantes
  (vale decir,
   el número de \(\mathcal{A}\) que componen nuestra secuencia)
  el método simbólico da:
  \begin{equation*}
    S(z, u)
      = \frac{u A(z)}{1 - u A(z)}
  \end{equation*}
  La función generatriz \(S_p(z, u)\)
  del número de secuencias primitivas formadas por \(\mathcal{A}\)
  queda determinado por la ecuación implícita:
  \begin{equation*}
    S(z, u)
      = \sum_{n \ge 1} S_p(z^n, u^n)
  \end{equation*}
  de donde por el lema~\ref{lem:GF-Moebius-inversion}:
  \begin{equation}
    \label{eq:ms-primitive-sequences}
    S_p(z, u)
      = \sum_{n \ge 1} \mu(n) \, \frac{u^n A(z^n)}{1 - u^n A(z^n)}
  \end{equation}
  Pero nos interesa la función generatriz de los ciclos primitivos,
  \(C_p(z, u)\).
  Por lo discutido antes,
  esto se obtiene de \(S_p(z, u)\)
  haciendo el reemplazo \(u^l \mapsto u^l / l\),
  y esto a su vez integrando término a término se obtiene como:
  \begin{align}
    C_p(z, u)
      &= \int_0^u S_p(z, v) \, \frac{\mathrm{d} v}{v} \notag \\
      &= \sum_{n \ge 1}
	   \frac{\mu(n)}{n} \, \ln \frac{1}{1 - u^n A(z^n)}
	    \label{eq:gf-primitive-cycle}
  \end{align}
  Construimos ciclos completos repitiendo ciclos primitivos,
  lo que corresponde a la inversa de Dirichlet
  de~\eqref{eq:gf-primitive-cycle}:
  \begin{equation}
    \label{eq:Cycle-binary}
    C(z, u)
      = \sum_{n \ge 1}
	  \frac{\phi(n)}{n} \, \ln \frac{1}{1 - u^n A(z^n)}
  \end{equation}
  Substituyendo \(u = 1\) obtenemos la ecuación prometida.

  Resta demostrar que en el anillo de Dirichlet:%
    \index{Dirichlet, anillo de}
  \begin{equation*}
    \left( \frac{\mu(n)}{n} \right)^{-1}
      = \frac{\phi(n)}{n}
  \end{equation*}
  La identidad de Gauß%
    \index{Gauss, identidad de@Gauß, identidad de}
  (teorema~\ref{theo:Gauss-identity}) dice:
  \begin{equation*}
    n = \sum_{d \mid n} \phi(d)
  \end{equation*}
  Por inversión de Möbius:%
    \index{Mobius, inversion de@Möbius, inversión de}
  \begin{equation*}
    \phi(n)
      = \sum_{d \mid n} \mu(d) \, \frac{n}{d}
  \end{equation*}
  que es equivalente a lo que buscábamos demostrar.

% irreducible-polynomials.tex
%
% Copyright (c) 2013-2014 Horst H. von Brand
% Derechos reservados. Vea COPYRIGHT para detalles

\subsection{Polinomios irreductibles en \(\mathbb{F}_q\)}
\label{sec:count-irreductible-polynomials}

  Recordamos del capítulo~\ref{cha:campos-finitos}
  que los polinomios
  \(\mathbb{F}_q[x]\) para \(q\) la potencia de un primo
  son un dominio euclidiano,
  por lo que
  por la teoría de la sección~\ref{sec:dominios-euclidianos},
  en particular el teorema~\ref{theo:PID=>UFD},
  nos asegura que hay factorización única
  (salvo unidades)
  en \(\mathbb{F}_q[x]\).
  Para obviar las unidades,
  consideremos polinomios mónicos.

  Si para el polinomio \(\alpha(x) \in \mathbb{F}_q[x]\)
  consideramos su grado como tamaño,
  vemos que multiplicar polinomios es simplemente sumar sus tamaños.
  Podemos entonces considerar
  la clase \(\mathcal{P}\)
  de polinomios mónicos en \(\mathbb{F}_q[x]\)
  con \(\lvert \alpha(x) \rvert = \deg(\alpha)\),
  y combinar polinomios corresponde a multiplicarlos.
  Es claro que hay \(q^n\) polinomios mónicos de grado \(n\),
  o sea la función generatriz ordinaria
  que cuenta polinomios mónicos es:
  \begin{equation*}
    P(z)
      = \sum_{n \ge 0} q^n z^n
      = \frac{1}{1 - q z}
  \end{equation*}

  Consideremos la clase \(\mathcal{I}\)
  de polinomios mónicos irreductibles,
  contados por la función generatriz ordinaria:
  \begin{equation*}
    I(z)
      = \sum_{n \ge 0} N_n z^n
  \end{equation*}
  Factorización única significa que todo polinomio
  corresponde a un multiconjunto de polinomios irreductibles:
  \begin{equation}
    \label{eq:polynomials=MSet(irreducibles)}
    \mathcal{P}
      = \MSet(\mathcal{I})
  \end{equation}
  Resulta interesante
  contar con una forma de resolver ecuaciones implícitas
  como~\eqref{eq:polynomials=MSet(irreducibles)}.
  \begin{theorem}
    \label{theo:A=MSet(B)}
    Sean \(\mathcal{A}\) y \(\mathcal{B}\)
    clases de objetos no rotulados
    relacionadas mediante:
    \begin{equation*}
      \mathcal{A}
	= \MSet(\mathcal{B})
    \end{equation*}
    Entonces las funciones generatrices ordinarias
    respectivas cumplen:
    \begin{equation}
      \label{eq:A=MSet(B)-->B}
      B(z)
	= \sum_{k \ge 1} \frac{\mu(k)}{k} \, \ln A(z^k)
    \end{equation}
  \end{theorem}
  \begin{proof}
    El método simbólico da:
    \begin{equation*}
      A(z)
	= \exp \left( \sum_{k \ge 1} \frac{B(z^k)}{k} \right)
    \end{equation*}
    Tomando logaritmos:
    \begin{align*}
      \ln A(z)
	&= \sum_{r \ge 1} \frac{B(z^r)}{r} \\
	&= \sum_{r \ge 1} \frac{1}{r} \, \sum_{s \ge 1} b_s z^{r s}
    \end{align*}
    Extraemos coeficientes:
    \begin{align*}
      n \left[ z^n \right] \ln A(z) \\
	&= \sum_{r \ge 1} \frac{n}{r} \, \sum_{s \ge 1} b_s z^{r s} \\
	&= \sum_{r s = n} s b_s
    \end{align*}
    Este es exactamente el caso
    del lema~\ref{lem:GF-Moebius-inversion},
    lo que entrega lo enunciado.
  \end{proof}

  Con el teorema~\ref{theo:A=MSet(B)}
  queda de~\eqref{eq:polynomials=MSet(irreducibles)}:
  \begin{align*}
    I(z)
      = \sum_{k \ge 1} \frac{\mu(k)}{k} \, \frac{1}{1 - q z^k}
  \end{align*}
  Tenemos nuevamente el resultado
  del teorema~\ref{theo:number-irreducible-polynomials}:
  \begin{equation*}
    N_n
      = \frac{1}{n} \, \sum_{d \mid n} \mu(n / d)  \, q^d
  \end{equation*}

%%% Local Variables:
%%% mode: latex
%%% TeX-master: "clases"
%%% End:


\section{Objetos rotulados}
\label{sec:rotulados}
\index{metodo simbolico@método simbólico!objetos rotulados}

  En la discusión previa solo interesaba el tamaño de los objetos,
  no su disposición particular.
  Consideraremos ahora objetos rotulados,
  donde importa cómo se compone el objeto de sus partes
  (los átomos están numerados,
   o se ubican en orden).

  El objeto más simple con partes rotuladas son las permutaciones
  (biyecciones \(\sigma \colon [n] \rightarrow [n]\),
   podemos considerarlas secuencias de átomos numerados).
  Para la función generatriz exponencial tenemos,
  ya que hay \(n!\) permutaciones de \(n\) elementos:
  \begin{equation*}
    \sum_{\sigma}
	\frac{z^{\lvert \sigma \rvert}}{\lvert \sigma \rvert !}
      = \sum_{n \ge 0} n! \, \frac{z^n}{n!}
      = \frac{1}{1 - z}
  \end{equation*}

  Lo siguiente más simple de considerar
  es colecciones de ciclos rotulados.
  Por ejemplo,
  escribimos \((1\;3\;2)\) para el objeto
  en que viene \(3\) luego de \(1\),
  \(2\) sigue a \(3\),
  y a su vez \(1\) sigue a \(2\).
  Así \((2\;1\;3)\) es solo otra forma de anotar el ciclo anterior,
  que no es lo mismo que \((3\;1\;2)\).
  Interesa definir formas consistentes
  de combinar objetos rotulados.
  Por ejemplo,
  al combinar el ciclo \((1\;2)\) con el ciclo \((1\;3\;2)\)
  resultará un objeto con \(5\) rótulos,
  y debemos ver cómo los distribuimos entre las partes.
  El cuadro~\ref{tab:ciclo+ciclo}
  reseña las posibilidades al respetar
  el orden de los elementos asignados a cada parte.
  \begin{table}[htbp]
    \centering
    \begin{tabular}{*{4}{>{\(}l<{\)}}}
      (1\;2) (3\;5\;4) & (2\;3) (1\;5\;4) & (3\;4) (1\;5\;2) &
	  (4\;5) (1\;3\;2) \\
      (1\;3) (2\;5\;4) & (2\;4) (1\;3\;5) & (3\;5) (1\;4\;2) \\
      (1\;4) (2\;5\;3) & (2\;5) (1\;3\;4) \\
      (1\;5) (2\;4\;3)
    \end{tabular}
    \caption{Combinando los ciclos $(1\;2)$ y $(1\;3\;2)$}
    \label{tab:ciclo+ciclo}
  \end{table}
  Es claro que lo que estamos haciendo es elegir
  un subconjunto de \(2\) rótulos
  de entre los \(5\) para asignárselos al primer ciclo.
  El combinar
  dos clases de objetos \(\mathcal{A}\) y \(\mathcal{B}\)
  de esta forma lo anotaremos \(\mathcal{A} \star \mathcal{B}\).
  Otra operación común es la \emph{composición},
  anotada \(\mathcal{A} \circ \mathcal{B}\).
  La idea es elegir un elemento \(\alpha \in \mathcal{A}\),
  luego elegir \(\lvert \alpha \rvert\) elementos
  de \(\mathcal{B}\),
  y reemplazar los \(\mathcal{B}\) por las partes de \(\alpha\),
  en el orden que están rotuladas;
  para finalmente asignar rótulos a los átomos
  que conforman la.estructura completa
  respetando el orden de los rótulos
  al interior de los \(\mathcal{B}\).
  Ocasionalmente es útil \emph{marcar}
  uno de los componentes del objeto,
  operación que anotaremos \(\mathcal{A}^\bullet\).
  Otra notación común para esta operación es \(\Theta \mathcal{A}\).
  Usaremos también la construcción \(\MSet(\mathcal{A})\),
  que podemos considerar como una secuencia de elementos numerados
  obviando el orden.
  Cuidado,
  muchos textos le llaman \(\Set()\) a esta operación.

  Tenemos el siguiente teorema:
  \begin{theorem}[Método simbólico, EGF]
    \index{metodo simbolico@método simbólico!teorema de transferencia!objetos rotulados|textbfhy}
    \label{theo:ms-EGF}
    Sean \(\mathcal{A}\) y \(\mathcal{B}\) clases de objetos,
    con funciones generatrices exponenciales
    \(\widehat{A}(z)\) y \(\widehat{B}(z)\),
    respectivamente.
    Entonces tenemos
    las siguientes funciones generatrices exponenciales:
    \begin{enumerate}
    \item
      Para enumerar \(\mathcal{A}^\bullet\):
      \begin{equation*}
	z \mathrm{D} \widehat{A}(z)
      \end{equation*}
    \item
      Para enumerar \(\mathcal{A} + \mathcal{B}\):
      \begin{equation*}
	\widehat{A}(z) + \widehat{B}(z)
      \end{equation*}
    \item
      Para enumerar \(\mathcal{A} \star \mathcal{B}\):
      \begin{equation*}
	\widehat{A}(z) \cdot \widehat{B}(z)
      \end{equation*}
    \item
      Para enumerar \(\mathcal{A}^\bullet\):
      \begin{equation*}
	z \widehat{A}'(z)
      \end{equation*}
    \item
      Para enumerar \(\mathcal{A} \circ \mathcal{B}\):
      \begin{equation*}
	\widehat{A}(\widehat{B}(z))
      \end{equation*}
    \item
      Para enumerar \(\Seq(\mathcal{A})\):
      \begin{equation*}
	\frac{1}{1 - \widehat{A}(z)}
      \end{equation*}
    \item
      Para enumerar \(\MSet(\mathcal{A})\):
      \begin{equation*}
	\mathrm{e}^{\widehat{A}(z)}
      \end{equation*}
    \item
      Para enumerar \(\Cyc(\mathcal{A})\):
      \begin{equation*}
	-\ln(1 - \widehat{A}(z))
      \end{equation*}
    \end{enumerate}
  \end{theorem}
  \begin{proof}
    Usaremos casos ya demostrados en las demostraciones sucesivas.
    \begin{enumerate}
    \item % mark A
      El objeto \(\alpha \in \mathcal{A}\)
      da lugar a \(\lvert \alpha \rvert\) objetos
      al marcar cada uno de sus átomos,
      lo que da la función generatriz exponencial:
      \begin{equation*}
	\sum_{\alpha \in \mathcal{A}}
	  \lvert \alpha \rvert
	    \frac{z^{\lvert \alpha \rvert}}{\lvert \alpha \rvert !}
      \end{equation*}
      Esto es lo indicado.
    \item % A + B
      Nuevamente trivial.
    \item % A x B
      El número de objetos \(\gamma\) que se obtienen
      al combinar \(\alpha \in \mathcal{A}\)
      con \(\beta \in \mathcal{B}\) es:
      \begin{equation*}
	\binom{\lvert \alpha \rvert + \lvert \beta \rvert}
	      {\lvert \alpha \rvert}
      \end{equation*}
      y tenemos la función generatriz exponencial:
      \begin{equation*}
	\sum_{\gamma \in \mathcal{A} \star \mathcal{B}}
	    \frac{z^{\lvert \gamma \rvert}}{\lvert \gamma \rvert !}
	  = \sum_{\substack{
		     \alpha \in \mathcal{A} \\
		     \beta \in \mathcal{B}
		 }}
	       \binom{\lvert \alpha \rvert + \lvert \beta \rvert}
		     {\lvert \alpha \rvert}
		  \frac{z^{\lvert \alpha \rvert
			    + \lvert \beta \rvert}}
		       {(\lvert \alpha \rvert
			    + \lvert \beta \rvert)!}
	  = \left(
	      \sum_{\alpha \in \mathcal{A}}
		\frac{z^{\lvert \alpha \rvert}}
		     {\lvert \alpha \rvert !}
	    \right)
	      \cdot \left(
		\sum_{\beta \in \mathcal{B}}
		  \frac{z^{\lvert \beta \rvert}}
		       {\lvert \beta \rvert !}
		    \right)
	  = \widehat{A}(z) \cdot \widehat{B}(z)
      \end{equation*}
    \item % Mark(A)
      Si tomamos un objeto \(\alpha \in \mathcal{A}\)
      de tamaño \(\lvert \alpha \rvert\),
      estamos creando \(\lvert \alpha \rvert\) nuevos objetos
      al marcar cada uno de sus componentes.
      La función generatriz resultante es:
      \begin{equation*}
	\sum_{\alpha \in \mathcal{A}}
	  \lvert \alpha \rvert \frac{z^{\lvert \alpha \rvert}}
				    {\lvert \alpha \rvert !}
	  = z \widehat{A}'(z)
      \end{equation*}
    \item % A circ B
      Tomemos \(\alpha \in \mathcal{A}\),
      de tamaño \(n = \lvert \alpha \rvert\),
      y \(n\) elementos de \(\mathcal{B}\) en orden
      a ser reemplazados por las partes de \(\alpha\).
      Esa secuencia de \(\mathcal{B}\) es representada por:
      \begin{equation*}
	\mathcal{B}
	  \star \mathcal{B}
	  \star \dotsb
	  \star \mathcal{B}
      \end{equation*}
      con función generatriz exponencial:
      \begin{equation*}
	\widehat{B}^n (z)
      \end{equation*}
      Sumando sobre las contribuciones:
      \begin{equation*}
	\sum_{\alpha \in \mathcal{A}}
	   \frac{\widehat{B}^{\lvert \alpha \rvert}(z)}
		{\lvert \alpha \rvert \, !}
      \end{equation*}
      Esto es lo prometido.
    \item % Seq(A)
      Primeramente,
      para \(\Seq(\mathcal{Z})\),
      como hay \(n!\) secuencias de largo \(n\):
      \begin{equation*}
	\sum_{n \ge 0} n! \frac{z^n}{n!}
	  = \frac{1}{1 - z}
      \end{equation*}
      Aplicando composición se obtiene lo indicado.
    \item % MSet(A)
      Hay un único multiconjunto de \(n\) elementos rotulados
      (se rotulan simplemente de 1 a  \(n\)),
      con lo que \(\MSet(\mathcal{Z})\) corresponde a:
      \begin{equation*}
	\sum_{n \ge 0} \frac{z^n}{n!}
	  = \exp(z)
      \end{equation*}
      Al aplicar composición resulta lo anunciado.

      Otra demostración es considerar el multiconjunto de \(\mathcal{A}\),
      descrito por \(\mathcal{M} = \MSet(\mathcal{A})\).
      Si marcamos uno de los átomos de \(\mathcal{M}\)
      estamos marcando uno de los \(\mathcal{A}\),
      el resto sigue formando un multiconjunto de \(\mathcal{A}\):
      \begin{equation*}
	\mathcal{M}^\bullet
	  = \mathcal{A}^\bullet \star \mathcal{M}
      \end{equation*}
      Por lo anterior:
      \begin{equation*}
	z M'(z)
	  = z A'(z) M(z)
      \end{equation*}
      Hay un único multiconjunto de tamaño \(0\),
      o sea \(M(0) = 1\);
      y hemos impuesto la condición
      que no hay objetos de tamaño \(0\) en \(\mathcal{A}\),
      vale decir,
      \(A(0) = 0\).
      Así la solución a la ecuación diferencial es:
      \begin{equation*}
	M(z)
	  = \exp(A(z))
      \end{equation*}
    \item % Cyc(A)
      Consideremos un ciclo de \(\mathcal{A}\),
      o sea \(\mathcal{C} = \Cyc(\mathcal{A})\).
      Si marcamos los \(\mathcal{C}\),
      estamos marcando uno de los \(\mathcal{A}\),
      y el resto es una secuencia:
      \begin{equation*}
	\mathcal{C}^\bullet
	  = \mathcal{A}^\bullet \star \Seq(\mathcal{A})
      \end{equation*}
      Esto se traduce en la ecuación diferencial:
      \begin{equation*}
	z \widehat{C}'(z)
	  = z A'(z) \frac{1}{1 - A(z)}
      \end{equation*}
      Integrando bajo el entendido \(C(0) = 0\) con \(A(0) = 0\)
      se obtiene lo indicado.
      \qedhere
    \end{enumerate}
  \end{proof}

\subsection{Rotulado o no rotulado, esa es la cuestión\ldots}
\label{sec:rotulado-o-no}

  Después de las exposiciones anteriores
  el amable lector estará comprensiblemente confundido
  respecto de cuándo considerar rotulados los objetos entre manos.
  Como regla general,
  se deben considerar no rotulados los objetos
  en los cuales piezas iguales son intercambiables.
  Al considerar un canasto de frutas,
  como en el primer ejemplo
  de la sección~\ref{sec:combinatorial-applications},
  consideramos que solo es relevante el número
  de las frutas de los distintos tipos.
  En los términos presentes,
  son multiconjuntos de objetos no rotulados,
  y la clase canasto
  (número par de manzanas,
   número de plátanos divisible por cinco,
   a lo más cuatro naranjas,
   opcionalmente una sandía)
  se representa simbólicamente como:
  \begin{equation*}
    \mathcal{C}
      = \MSet(\mathcal{Z} \times \mathcal{Z})
	  \times \MSet(\mathcal{Z}
			 \times \mathcal{Z}
			 \times \mathcal{Z}
			 \times \mathcal{Z}
			 \times \mathcal{Z})
	  \times \MSet_{\le 4}(\mathcal{Z})
	  \times \MSet_{\le 1}(\mathcal{Z})
  \end{equation*}
  Aplicando las reglas de transferencia
  del teorema~\ref{theo:ms-OGF}
  resulta la función generatriz ordinaria:
  \begin{align*}
    C(z)
      &= \frac{1}{1 - z^2}
	   \cdot \frac{1}{1 - z^5}
	   \cdot (1 + z + z^2 + z^3 + z^4)
	   \cdot (1 + z) \\
      &= \frac{1}{(1 - z)^2}
  \end{align*}
  y en consecuencia el número de canastos con \(n\) frutas es:
  \begin{equation*}
    \left[ z^n \right] C(z)
      = n + 1
  \end{equation*}
  como habíamos deducido antes.

  Las permutaciones son secuencias de elementos distinguibles,%
    \index{permutacion@permutación!generatriz}
  por lo que se consideran objetos rotulados.
  La clase de permutaciones
  queda entonces representada por la expresión simbólica:
  \begin{equation*}
    \mathcal{P}
      = \Seq(\mathcal{Z})
  \end{equation*}
  Las reglas del teorema~\ref{theo:ms-EGF}
  dan la función generatriz exponencial:
  \begin{equation*}
    \widehat{P}(z)
      = \frac{1}{1 - z}
  \end{equation*}
  de donde el número de permutaciones de \(n\) elementos es:
  \begin{equation*}
    n! \left[ z^n \right] \widehat{P}(z)
      = n!
  \end{equation*}
  como ya sabíamos.

  Hay situaciones en las cuales los objetos en sí
  son indistinguibles,
  pero los consideramos rotulados por sus posiciones.
  Un ejemplo popular considera un buque que hace señales
  mediante \(12\)~banderas de colores blanco, rojo, azul y negro.
  Se restringen las señales
  a tener un número par de banderas blancas
  e impar de rojas.
  Se pregunta cuántas señales diferentes puede dar,
  suponiendo que tiene banderas suficientes de cada color.
  Esto puede responderse por las técnicas
  de la sección~\ref{sec:tao-bookkeeper},
  pero resulta engorroso.
  Si consideramos cada color de bandera
  como el multiconjunto de esa bandera rotuladas por su posición,
  la regla de distribución de rótulos de la operación \(\star\)
  exactamente corresponde
  a barajar las banderas de los distintos colores.
  Considerando entonces objetos rotulados,
  la clase de señales queda descrita simbólicamente por:
  \begin{equation*}
    \mathcal{S}
      = \MSet_{\text{even}} (\mathcal{Z})
	  \star \MSet_{\text{odd}} (\mathcal{Z})
	  \star \MSet(\mathcal{Z})
	  \star \MSet(\mathcal{Z})
  \end{equation*}
  El teorema~\ref{theo:ms-EGF} da la función generatriz exponencial
  (el multiconjunto de número par de elementos da los términos pares
   de la serie para la exponencial,
   multiconjuntos con número impar
   de elementos da los términos impares;
   y esto a su vez da coseno y seno hiperbólicos):
  \begin{align*}
    \widehat{S}(z)
      &= \cosh z \, \sinh z \, \mathrm{e}^{2 z} \\
      &= \frac{\mathrm{e}^{4 z} - 1}{4}
  \end{align*}
  y el número de señales que pueden formarse con \(12\)~banderas es:
  \begin{equation*}
    12! \left[ z^{12} \right] \widehat{S}(z)
      = 12! \, \frac{1}{4} \frac{4^{12}}{12!}
      = 4^{11}
  \end{equation*}

\subsection{Algunas aplicaciones de objetos rotulados}
\label{sec:ms-egf-aplicaciones}

  Un primer ejemplo simple
  es determinar
  el número de organizaciones circulares de \(n\) elementos.
  Al ser diferentes,
  podemos considerarlos rotulados.
  Quedan representadas simbólicamente por \(\Cyc(\mathcal{Z})\),
  con función generatriz exponencial:
  \begin{equation*}
    \ln \frac{1}{1 - z}
  \end{equation*}
  Como es una función generatriz exponencial,
  interesa:
  \begin{equation*}
    n! \left[ z^n \right] \ln \frac{1}{1 - z}
      = n! \left[ z^n \right] \sum_{n \ge 1} \frac{z^n}{n}
      = (n - 1)!
  \end{equation*}

  Las permutaciones
  podemos representarlas como secuencias rotuladas,
  \(\Seq(\mathcal{Z})\),
  con función generatriz exponencial:
  \begin{equation*}
    \frac{1}{1 - z}
  \end{equation*}
  La función generatriz exponencial
  para colecciones de ciclos \(\MSet(\Cyc(\mathcal{Z}))\) es:
  \begin{equation*}
    \exp( -\ln(1 - z))
      = \frac{1}{1 - z}
  \end{equation*}
  Vale decir,
  hay tantas maneras de distribuir \(n\) elementos en ciclos
  como hay permutaciones de esos \(n\) elementos.
  Volveremos a esto en el capítulo~\ref{cha:permutaciones}.

  Sea \(\mathcal{D}\) la clase de los desarreglos.%
    \index{metodo simbolico@método simbólico!desarreglos}
  Como las permutaciones son elementos que se mantienen fijos
  (podemos representarlos como su conjunto)
  y elementos que no están en sus posiciones
  (desarreglos),
  podemos expresar:
  \begin{equation*}
    \Seq(\mathcal{Z})
      = \mathcal{D} \star \Set(\mathcal{Z})
  \end{equation*}
  O sea:
  \begin{equation*}
    \frac{1}{1 - z}
      = \widehat{D}(z) \cdot \mathrm{e}^z
  \end{equation*}
  Como antes.

  Podemos modificar los operadores,
  por ejemplo anotar \(\MSet_{\ge 1}(\mathcal{A})\)
  para conjuntos de al menos un \(\mathcal{A}\),
  con ajustes a sus expansiones
  sugeridas por la demostración del teorema del caso.
  Hay identidades evidentes,
  como
  \(\Seq_{\ge 1}(\mathcal{A})
      = \mathcal{A} \star \Seq(\mathcal{A})\)
  que pueden simplificar los desarrollos.

  Una permutación consta de sus puntos fijos
  y el desarreglo de los restantes.
  Si tiene exactamente \(k\) puntos fijos:
  \begin{equation*}
    \mathcal{D} \star \MSet_k(\mathcal{Z})
  \end{equation*}
  Representando un multiconjunto de \(k\) elementos
  como secuencia obviando el orden,
  esto da la función generatriz exponencial:%
    \index{generatriz!exponencial}
  \begin{equation*}
    \frac{\mathrm{e}^{-z}}{1 - z} \cdot \frac{z^k}{k!}
  \end{equation*}
  Extraemos coeficientes:
  \begin{align*}
    n! \left[ z^n \right] \frac{z^k \mathrm{e}^{-z}}{k! (1 - z)}
      &= \frac{n!}{k!} \left[ z^{n - k} \right]
			 \frac{\mathrm{e}^{-z}}{1 - z} \\
      &= \frac{n!}{k!} \exp \rvert_{n - k} (-1)
  \end{align*}

  Para contar todas las maneras de particionar un conjunto
  tenemos la expresión simbólica:
  \begin{equation*}
    \MSet(\MSet_{\ge 1}(\mathcal{Z}))
  \end{equation*}
  que se traduce directamente en la función generatriz exponencial
  de los \emph{números de Bell}~%
    \cite{bell34:_expon_number}:%
    \index{Bell, numeros de@Bell, números de}
  \begin{equation}
    \label{eq:Bell-egf}
    \widehat{B}(z)
      = \mathrm{e}^{\mathrm{e}^z - 1}
  \end{equation}

  Para obtener una fórmula explícita para \(B_n\)
  salimos del espacio estricto de las series formales.
  Las manipulaciones se justifican ya que las series involucradas
  convergen uniformemente para todo \(z\).
  \begin{align*}
    \widehat{B}(z)
      &= \frac{\mathrm{e}^{\mathrm{e}^z}}{e} \\
      &= \frac{1}{e}
	   \, \sum_{r \ge 0} \frac{\mathrm{e}^{r z}}{r!} \\
      &= \frac{1}{e} \, \sum_{r \ge 0} \frac{1}{r!}
			  \sum_{s \ge 0} \frac{(r z)^s}{s!} \\
      &= \frac{1}{e} \, \sum_{s \ge 0} \frac{z^s}{s!}
			  \sum_{r \ge 0} \frac{r^s}{r!}
  \end{align*}
  El número de Bell \(B_n\)
  es el coeficiente de \(z^n / n!\) en esto:
  \begin{equation}
    \label{eq:Dobinski}
    B_n
      = \frac{1}{e} \, \sum_{r \ge 0} \frac{r^n}{r!}
  \end{equation}
  El resultado~\eqref{eq:Dobinski}
  se conoce como \emph{ecuación de Dobiński}~%
    \cite{dobinski77:_summir_reihe}.%
    \index{Dobinski, ecuacion de@Dobiński, ecuación de}%
    \index{Bell, numeros de@Bell, números de!formula@fórmula|see{Dobiński, ecuación de}}

  Derivando~\eqref{eq:Bell-egf}
  obtenemos la ecuación diferencial,
  cuyo valor inicial resulta directamente de la función generatriz:
  \begin{equation}
    \label{eq:Bell-egf-ode}
    \widehat{B}'(z)
      = \mathrm{e}^z \widehat{B}(z)
      \qquad \widehat{B}(0) = 1
  \end{equation}
  El lado izquierdo es un desplazamiento,
  el derecho corresponde a una convolución binomial:
  \begin{equation}
    \label{eq:Bell-recurrence}
    \index{Bell, numeros de@Bell, números de!recurrencia}
    B_{n + 1}
      = \sum_{0 \le k \le n} \binom{n}{k} B_k
      \qquad B_0 = 1
  \end{equation}
  El lector interesado verificará
  que el truco \(z \mathrm{D} \log\)
  aplicado a~\eqref{eq:Bell-egf}%
    \index{derivada logaritmica@derivada logarítmica}
  lleva a la misma recurrencia~\eqref{eq:Bell-recurrence}.

  Un ejemplo clásico es considerar árboles rotulados,
  formados por un nodo raíz conectados a un conjunto de árboles.%
    \index{arbol con raiz@árbol con raiz!rotulado}
  Esto lleva directamente a:
  \begin{equation*}
    \mathcal{T}
      = \mathcal{Z} \star \MSet(\mathcal{T})
  \end{equation*}
  que se traduce en la ecuación
  para la función generatriz \(\widehat{T}(z)\):
  \begin{equation*}
    \widehat{T}(z)
      = z \mathrm{e}^{\widehat{T}(z)}
  \end{equation*}
  Inversión de Lagrange%
    \index{Lagrange, inversion de@Lagrange, inversión de}
  da directamente la afamada fórmula de Cayley:%
    \index{Cayley, formula de@Cayley, fórmula de}
  \begin{align*}
    \frac{t_n}{n!}
      &= \frac{1}{n} \left[ u^{n - 1} \right] \mathrm{e}^{n u} \\
      &= \frac{1}{n} \cdot \frac{n^{n - 1}}{(n - 1)!} \\
     t_n
       &= n^{n - 1}
  \end{align*}

  \begin{figure}[ht]
    \centering
    \pgfimage{images/function-graph}
    \caption{Una función de \([21]\) a \([21]\)}
    \label{fig:function-graph}
  \end{figure}
  Consideremos una función de \([n] \rightarrow [n]\),
  como por ejemplo
  la graficada en la figura~\ref{fig:function-graph}
  vía indicar por flechas el valor de la función.
  Vemos que los valores se organizan en árboles,
  y a su vez estos en ciclos.
  Esto se describe mediante las ecuaciones simbólicas:
  \begin{align*}
    \mathcal{T}
      &= \mathcal{Z} \star \MSet(\mathcal{T}) \\
    \mathcal{F}
      &= \MSet(\Cyc(\mathcal{T}))
  \end{align*}
  Esto lleva a las ecuaciones funcionales:
  \begin{align*}
    \widehat{T}(z)
      &= z \mathrm{e}^{\widehat{T}(z)} \\
    \widehat{F}(z)
      &= \exp(- \ln (1 - \widehat{T}(z))) \\
      &= \frac{1}{1 - \widehat{T}(z)}
  \end{align*}
  Podemos aplicar inversión de Lagrange,
  teorema~\ref{theo:LIF},
  con \(\phi(u) = \mathrm{e}^u\) y \(f(u) = (1 - u)^{-1}\):%
    \index{Lagrange, inversion de@Lagrange, inversión de}
  \begin{align*}
    \frac{f_n}{n!}
      &= \frac{1}{n}
	   \left[ u^{n - 1} \right]
	      \left( (1 - u)^{-2} \mathrm{e}^{n u} \right) \\
      &= \frac{1}{n}
	   \left[ u^{n - 1} \right]
	     \sum_{k \ge 0} (k + 1) u^k \mathrm{e}^{n u} \\
      &= \frac{1}{n}
	   \sum_{k \ge 0} (k + 1)
	     \left[ u^{n - k - 1} \right] \mathrm{e}^{n u} \\
      &= \frac{1}{n}
	   \sum_{k \ge 0} (k + 1)
	     \frac{n^{n - k - 1}}{(n - k - 1)!} \\
    f_n
      &= (n - 1)! \, \sum_{k \ge 0} (k + 1)
	   \frac{n^{n - k - 1}}{(n - k - 1)!}
  \end{align*}
  No es precisamente una fórmula bonita,
  pero no fue difícil de deducir.

\subsection{Operaciones adicionales}
\label{sec:ms-egf-operaciones-extra}
\index{metodo simbolico@método simbólico!operaciones adicionales}

  Hay operaciones adicionales que son de interés ocasional.
  La definición de la siguiente operación es un tanto bizarra,
  pero pronto la aplicaremos.
  Sean \(\mathcal{A}\) y \(\mathcal{B}\) clases de objetos,
  con \(\alpha \in \mathcal{A}\) y \(\beta \in \mathcal{B}\).
  Definimos el \emph{producto cajonado}%
    \index{metodo simbolico@método simbólico!producto cajonado}
  (en el inglés original
     \emph{\foreignlanguage{english}{boxed product}},
   término bastante poco descriptivo)
  entre \(\alpha\) y \(\beta\),
  que se anota \(\alpha^{\square} \star \beta\),
  combinando \(\alpha\) y \(\beta\)
  y rotulando el resultado de forma que el mínimo rótulo
  se asigna a la parte \(\alpha\).
  Por ejemplo:
  \begin{equation*}
    (2, 1, 3)^\square \star (2, 1)
      = \{(2, 1, 3, 5, 4),
	  (2, 1, 4, 5, 3),
	  (2, 1, 5, 4, 3),
	  (3, 1, 4, 5, 2),
	  (3, 1, 5, 4, 2),
	  (4, 1, 5, 3, 2)
	\}
  \end{equation*}
  Lo que estamos haciendo
  es elegir \(\lvert \alpha \rvert - 1\) rótulos
  de entre \(\lvert \alpha \rvert + \lvert \beta \rvert - 1\).
  Extendemos esta operación a las clases respectivas
  uniendo los conjuntos resultantes.
  Si llamamos \(n = \lvert \alpha \rvert + \lvert \beta \rvert\),
  \(k = \lvert \alpha \rvert\)
  (y por tanto \(n - k = \lvert \beta \rvert\))
  el número de nuevos objetos de tamaño \(n\) creados así es:
  \begin{equation*}
    \sum_{1 \le k \le n}
      \binom{n - 1}{k - 1} a_k b_{n - k}
      = \sum_{0 \le k \le n - 1}
	  \binom{n - 1}{k} a_{k + 1} b_{n - 1 - k}
  \end{equation*}
  Esta es la convolución binomial
  de las secuencias \(\langle a_{n + 1} \rangle_{n \ge 0}\)
  y \(\langle b_n \rangle_{n \ge 0}\),
  pero desplazada en una posición a la derecha.
  Desplazamiento a la derecha es derivar,
  con lo que desplazar a la izquierda es integrar:
  \begin{equation*}
    \mathcal{A}^\square \star \mathcal{B}
      \egf \int_0^z \mathrm{D} \widehat{A}(u)
		  \cdot \widehat{B}(u) \, \mathrm{d} u
  \end{equation*}

  Pongamos en uso esta operación.
  Una \emph{permutación alternante}%
    \index{permutacion@permutación!alternante}
  es tal que:
  \begin{equation*}
    a_1 < a_2 > a_3 < \dotsb
  \end{equation*}
  Consideremos primero las que tienen un número impar de elementos,
  clase \(\mathcal{T}\).
  Si nos fijamos en su máximo,
  vemos que divide la permutación en forma única
  en una permutación alternante de largo impar,
  el máximo,
  y una permutación alternante de largo impar.

  La operación \(a_i \mapsto n + 1 - a_i\)
  es claramente una biyección entre permutaciones,
  y hace que el máximo pase a ser el mínimo.
  Podemos además interpretar la combinación
  \([\square,
       (\square, \dotsc, \square),
       (\square, \dotsc, \square)]\)
  como ubicando el primer elemento entre las secuencias.
  Con estas biyecciones en mente,
  podemos contar la clase \(\mathcal{T}\)
  de permutaciones alternantes de largo impar mediante:
  \begin{equation}
    \label{eq:class-alternating-odd}
    \mathcal{T}
      = \mathcal{Z}
	  + \mathcal{Z}^\square
	      \star (\mathcal{T} \star \mathcal{T})
  \end{equation}
  lo que lleva a la ecuación:
  \begin{equation*}
    \widehat{T}(z)
      = z + \int_0^z u' \cdot \widehat{T}^2(u) \, \mathrm{d} u
  \end{equation*}
  y a la ecuación diferencial:
  \begin{equation*}
    \widehat{T}'(z)
      = 1 + \widehat{T}^2(z)
    \quad \widehat{T}(0) = 0
  \end{equation*}
  de donde:
  \begin{equation}
    \label{eq:egf-alternating-odd}
    \widehat{T}(z)
      = \tan z
  \end{equation}

  Si consideramos las permutaciones alternantes de largo par
  (terminan en una subida),
  clase \(\mathcal{S}\),
  vemos que su máximo las divide en una permutación alternante de largo impar
  que termina en una bajada
  (descrita por \(\mathcal{T}\)),
  el máximo,
  y una permutación alternante de largo par
  que termina en subida
  (descrita por \(\mathcal{S}\)):
  \begin{equation}
    \label{eq:class-alternating-even}
    \mathcal{S}
      = \mathcal{E}
	  + \mathcal{Z}^\square
	      \star (\mathcal{T} \star \mathcal{S})
  \end{equation}
  Esto da lugar a la ecuación:
  \begin{equation*}
    \widehat{S}(z)
      = 1 + \int_0^z u' \cdot \widehat{S}(u) \widehat{T}(u)
	      \, \mathrm{d} u
  \end{equation*}
  con solución:
  \begin{equation}
    \label{eq:egf-alternating-even}
    \widehat{S}(z)
      = \sec z
  \end{equation}
  Obtenemos el curioso resultado de André~%
    \cite{andre81:_sur_permut_alter}
  (ver también Stanley~\cite{stanley09:_survey_alter_permut})
  para los números \(E_n\) que cuentan permutaciones alternantes:
  \begin{equation}
    \label{eq:egf-Euler-numbers}
    \widehat{E}(z)
      = \sec z + \tan z
  \end{equation}
  Los coeficientes son los \emph{números de Euler},%
    \index{Euler, numeros de@Euler, números de}
  quien había llegado a los \(E_{2 n + 1}\) por otro camino.

  Una manera de obtener una recurrencia es observar:
  \begin{align*}
    \widehat{E}'(z)
      &= \sec^2 z + \tan z \sec z \\
    \widehat{E}^2(z)
      &= \tan^2 z + 2 \tan z \sec z + \sec^2 z \\
      &= 2 \tan z \sec z + 2 \sec^2 z - 1 \\
      &= 2 \widehat{E}'(z) - 1
  \end{align*}
  Usando las propiedades de funciones generatrices exponenciales%
    \index{generatriz!exponencial}
  vemos que cumplen la recurrencia:
  \begin{equation}
    \label{eq:recurrence-Euler}
    2 E_{n + 1}
      = [n = 1] + \sum_{0 \le k \le n} \binom{n}{k} E_k E_{n - k}
    \quad E_0 = 1
  \end{equation}

%%% Local Variables:
%%% mode: latex
%%% TeX-master: "clases"
%%% End:


% numeros-combinatorios.tex
%
% Copyright (c) 2011-2014 Horst H. von Brand
% Derechos reservados. Vea COPYRIGHT para detalles

\chapter{Números combinatorios}
\label{cha:numeros-combinatorios}
\index{combinatoria}

  Las funciones generatrices tienen muchas aplicaciones en combinatoria,%
    \index{generatriz}
  por las razones que quedarán claras
  a través de variados ejemplos.
  En muchos casos es directo obtener recurrencias
  para los números que interesan,
  y funciones generatrices
  dan entonces forma de llegar a expresiones explícitas
  o relaciones adicionales.
  Aplicamos el método simbólico
  (desarrollado en el capítulo~\ref{cha:metodo-simbolico})
  donde es posible,
  dado que simplifica inmensamente los desarrollos.

\section{Subconjuntos y multiconjuntos}
\label{sec:sub+multi-conjuntos}

  Para obtener el número de subconjuntos%
    \index{conjunto!subconjunto}
  de \(k\) elementos tomados entre \(n\),
  podemos razonar en forma afín a la demostración
  del teorema~\ref{theo:ms-OGF} para conjuntos:
  Cada elemento aporta \(0\) o \(1\) al tamaño de un subconjunto,
  por lo que la función generatriz
  para los subconjuntos de un conjunto de \(n\) elementos no es más que:
  \begin{equation*}
    (1 + z)^n
  \end{equation*}
  y el número de interés es entonces directamente:
  \begin{equation*}
    \binom{n}{k}
      = \left[ z^k \right] (1 + z)^n
  \end{equation*}
  como ya sabíamos.

  Otra situación interesante son los multiconjuntos.%
    \index{multiconjunto!subconjunto}
  Siguiendo nuevamente el razonamiento de la demostración
  del teorema~\ref{theo:ms-OGF},
  el aporte de cada elemento es:
  \begin{equation*}
    1 + z + z^2 + \dotsb
      = \frac{1}{1 - z}
  \end{equation*}
  con lo que el número de multiconjuntos
  de \(k\) elementos tomados entre \(n\)
  es simplemente:
  \begin{equation*}
    \multiset{n}{k}
      = \left[ z^k \right] \, \left( \frac{1}{1 - z} \right)^n
      = (-1)^k \, \binom{-n}{k}
      = \binom{n + k - 1}{n - 1}
  \end{equation*}

\section[¿Cuántas secuencias
	   de \texorpdfstring{$2 n$ }{}paréntesis balanceados hay?]
	{\protect\boldmath
	   ¿Cuántas secuencias
	   de \texorpdfstring{$2 n$ }{}paréntesis balanceados
	   hay?%
       \protect\unboldmath}
\label{sec:Catalan}
\index{Catalan, numeros de@Catalan, números de}
\index{parentesis balanceados@paréntesis balanceados|see{Catalan, números de}}

  Llamemos \(C_n\)
  al número de secuencias
  de \(2 n\) paréntesis balanceados.
  Los primeros valores son:
  \begin{align*}
    1 & \quad \epsilon \\
    1 & \quad () \\
    2 & \quad () \quad (()) \\
    5 & \quad ()()() \quad ()(()) \quad (())() \quad (()()) \quad ((()))
  \end{align*}

  Toda secuencia
  de paréntesis balanceados se puede dividir en dos partes:
  El \(2 k\)\nobreakdash-ésimo paréntesis
  cierra el primer paréntesis (\(1 \le k \le n\)),
  y el resto.
  Marcando este comienzo en rojo
  está la secuencia
  \textcolor{red}{(()())}(()(()))()(()()()).
  Llamamos \emph{perfecta} a una secuencia
  con \(n = k\).
  Hay \(C_{n - 1}\) secuencias perfectas
  de largo \(2 n\),
  como muestra la siguiente biyección:
  \begin{itemize}
  \item
    Tomar una secuencia balanceada cualquiera
    de largo \(2 n - 2\),
    y encerrarla entre paréntesis
    da una secuencia perfecta de largo \(2 n\).
  \item
    Tomar una secuencia perfecta,
    y eliminar los paréntesis de más afuera
    da una secuencia balanceada.
  \end{itemize}

  Si llamamos \(\mathcal{C}\)
  la clase de secuencias de paréntesis balanceados,
  tenemos la relación simbólica:%
    \index{metodo simbolico@método simbólico}
  \begin{equation*}
    \mathcal{C}
      = \mathcal{E} + ( \mathcal{C} ) \mathcal{C}
  \end{equation*}
  Al usar \(z\) para marcar un par de paréntesis
  (sus posiciones exactas realmente no interesan),
  esto da:
  \begin{equation}
    \label{eq:Catalan}
    C(z)
      = 1 + z C^2(z)
  \end{equation}
  Ya nos habíamos tropezado con la ecuación~\eqref{eq:Catalan}
  al derivar la función generatriz para el número de árboles binarios
  en el método simbólico en el capítulo~\ref{cha:metodo-simbolico},
  y vimos que los \(C_n\)
  son los números de Catalan~\eqref{eq:Catalan-numbers}:
    \index{Catalan, numeros de@Catalan, números de}%
    \index{Catalan, numeros de@Catalan, números de!formula@fórmula}
  \begin{equation*}
    C_n
      = \frac{1}{n + 1} \, \binom{2 n}{n}
  \end{equation*}
  Expandiendo la serie:
  \begin{equation*}
    C(z)
      = 1 + z + 2 z^2 + 5 z^3 + 14 z^4 + 42 z^5 + 132 z^6
	  + 429 z^7 + 1\,430 z^8 + 4\,862  z^9 + 16\,796 z^{10} + \dotsb
  \end{equation*}

  Consideremos ahora la manera de dividir un polígono convexo
  (vale decir,
   toda diagonal cae completamente en su interior)
  en triángulos mediante diagonales.%
    \index{poligono@polígono!triangulacion@triangulación}
  La figura~\ref{fig:triangulations-pentagon}
  muestra que para el pentágono hay cinco triangulaciones.
  \begin{figure}[ht]
    \centering
    \subfloat{\pgfimage{images/triangulation-1}}
    \hspace{2em}
    \subfloat{\pgfimage{images/triangulation-2}}
    \hspace{2em}
    \subfloat{\pgfimage{images/triangulation-3}}
    \\
    \subfloat{\pgfimage{images/triangulation-4}}
    \hspace{2em}
    \subfloat{\pgfimage{images/triangulation-5}}
    \caption{División del pentágono en triángulos}
    \label{fig:triangulations-pentagon}
  \end{figure}
  Podemos considerar un polígono convexo triangulado
  como un polígono triangulado,
  un triángulo y otro polígono triangulado.
  El caso extremo es el ``polígono'' con dos vértices
  (una línea)
  que tiene una única triangulación
  (en cero triángulos).
  Si representamos la clase de triangulaciones por \(\mathcal{T}\)
  y los triángulos por \(\mathcal{Z}\),
  tenemos la expresión simbólica:%
    \index{metodo simbolico@método simbólico}
  \begin{equation}
    \label{eq:triangulation-se}
    \mathcal{T}
      = \mathcal{E} + \mathcal{T} \times \mathcal{Z} \times \mathcal{T}
  \end{equation}
  con la correspondiente ecuación funcional:
  \begin{equation}
    \label{eq:triangulation-fe}
    T(z)
      = 1 + z T^2(z)
  \end{equation}
  La solución es nuevamente los números de Catalan;%
    \index{Catalan, numeros de@Catalan, números de}
  solo que expresa el número de triangulaciones
  en términos del número de triángulos,
  no de lados del polígono.
  Vemos que un polígono de \(n\) lados se divide en \(n - 2\) triángulos,
  con lo que finalmente el número de triangulaciones
  de un polígono de \(n\)~lados es \(C_{n - 2}\).

\section{Números de Motzkin}
\label{sec:numeros-motzkin}
\index{Motzkin, numeros de@Motzkin, números de|textbfhy}

  Donaghey y Shapiro~\cite{donaghey77:_motzkin_numbers}
  indican una estrecha relación entre los números de Motzkin
  y los de Catalan,
    \index{Catalan, numeros de@Catalan, números de}%
  por lo que debieran aparecer con frecuencia similar.
  Una de las tantas estructuras que cuentan
  es el número de maneras en que pueden dibujarse cuerdas
  entre puntos sobre una circunferencia
  de manera que no se intersecten al interior ni sobre la circunferencia.
  La figura~\ref{fig:Motzkin} muestra que \(m_5 = 21\).
  \begin{figure}[ht]
    \centering
    \subfloat{\pgfimage{images/Motzkin-01}}
    \hspace{1em}
    \subfloat{\pgfimage{images/Motzkin-02}}
    \hspace{1em}
    \subfloat{\pgfimage{images/Motzkin-03}}
    \hspace{1em}
    \subfloat{\pgfimage{images/Motzkin-04}}
    \hspace{1em}
    \subfloat{\pgfimage{images/Motzkin-05}}
    \hspace{1em}
    \subfloat{\pgfimage{images/Motzkin-06}}
    \hspace{1em}
    \subfloat{\pgfimage{images/Motzkin-07}}
    \\
    \subfloat{\pgfimage{images/Motzkin-08}}
    \hspace{1em}
    \subfloat{\pgfimage{images/Motzkin-09}}
    \hspace{1em}
    \subfloat{\pgfimage{images/Motzkin-10}}
    \hspace{1em}
    \subfloat{\pgfimage{images/Motzkin-11}}
    \hspace{1em}
    \subfloat{\pgfimage{images/Motzkin-12}}
    \hspace{1em}
    \subfloat{\pgfimage{images/Motzkin-13}}
    \hspace{1em}
    \subfloat{\pgfimage{images/Motzkin-14}}
    \\
    \subfloat{\pgfimage{images/Motzkin-15}}
    \hspace{1em}
    \subfloat{\pgfimage{images/Motzkin-16}}
    \hspace{1em}
    \subfloat{\pgfimage{images/Motzkin-17}}
    \hspace{1em}
    \subfloat{\pgfimage{images/Motzkin-18}}
    \hspace{1em}
    \subfloat{\pgfimage{images/Motzkin-19}}
    \hspace{1em}
    \subfloat{\pgfimage{images/Motzkin-20}}
    \hspace{1em}
    \subfloat{\pgfimage{images/Motzkin-21}}
    \caption{Cuerdas entre cinco puntos sobre la circunferencia}
    \label{fig:Motzkin}
  \end{figure}
  Analizando la situación de la figura~\ref{fig:Motzkin}
  podemos construir una recurrencia para \(m_n\).
  Si elegimos uno de los \(n\) puntos
  este puede participar en alguna cuerda o no.
  Si no participa,
  es como si no existiera,
  esa situación aporta \(m_{n - 1}\) casos.
  Si ese punto participa en una cuerda,
  sus dos extremos quedan excluidos,
  y quedan por tender cuerdas entre los demás \(n - 2\) puntos,
  de forma que no crucen la cuerda entre manos.
  Vale decir,
  la cuerda corta el círculo en dos,
  una parte de \(k\) nodos y otra de \(n - k - 2\) nodos,
  las formas de tender cuerdas entre ellas se pueden combinar a gusto:%
    \index{recurrencia}
  \begin{equation*}
    m_n
      = m_{n - 1} + \sum_{0 \le k \le n - 2} m_k m_{n - k - 2}
  \end{equation*}
  Como condiciones de contorno tenemos \(m_0 = m_1 = 1\)
  (con la recurrencia entregan los valores conocidos
   \(m_2 = 2\) y \(m_3 = 4\)).
  Resulta:
  \begin{equation}
    \label{eq:Motzkin-recurrence}
    m_{n + 2}
      = m_{n + 1} + \sum_{0 \le k \le n} m_k m_{n - k}
    \qquad m_0 = m_1 = 1
  \end{equation}
  Aplicando las reglas con la función generatriz ordinaria \(M(z)\)
  y simplificando queda:%
    \index{generatriz}
  \begin{equation}
    \label{eq:Motzkin-fuctional}
    M(z)
      = 1 + z M(z) + z^2 M^2(z)
  \end{equation}
  de donde:
  \begin{equation}
    \label{eq:Motzkin-gf}
    M(z)
      = \frac{1 - z - \sqrt{1 - 2 z - 3 z^2}}{2 z^2}
  \end{equation}
  (elegimos el signo negativo ya que \(M(0) = m_0 = 1\)).
  Expandiendo en serie:
  \begin{equation*}
    M(z)
      = 1 + z + 2 z^2 + 4 z^3 + 9 z^4 + 21 z^5 + 51 z^6 + 127 z^8
	  + 323 z^8 + 835 z^9 + 2\,188 z^{10} + \dotsb
  \end{equation*}

  Alternativamente,
  si \(\mathcal{M}\) es la clase de las maneras de tender cuerdas,%
    \index{metodo simbolico@método simbólico}
  podemos descomponerla
  en ningún punto
  (aporta \(\mathcal{E}\));
  agregar un punto que no participa en ninguna cuerda
  (aporta \( \mathcal{Z} \times \mathcal{M}\)
  y tender una cuerda,
  lo que da cuenta de dos puntos y divide en dos conjuntos de puntos
  entre los cuales tender cuerdas
  (aporta
     \(\mathcal{Z} \times \mathcal{Z} \times \mathcal{M} \times \mathcal{M}\)).
  En total:
  \begin{equation}
    \label{eq:Motzkin-se}
    \mathcal{M}
      = \mathcal{E}
	  + \mathcal{Z} \times \mathcal{M}
	  + \mathcal{Z} \times \mathcal{Z}
	      \times \mathcal{M} \times \mathcal{M}
  \end{equation}
  Esto lleva nuevamente a la ecuación funcional~(\ref{eq:Motzkin-fuctional}).

\section{Números de Schröder}
\label{sec:numeros-Schroeder}
\index{Schroder, numeros de@Schröder, números de|textbfhy}

  De interés ocasional son los números de Schröder,
  quien los planteó como el segundo de sus cuatro problemas~%
    \cite{schroeder70:_vier_probleme},
  ver también a Stanley~\cite{stanley97:_hippar_plutar_schroed_hough}.
  Tenemos \(n\) símbolos
  (por ejemplo, \(x\))
  e interesa saber de cuántas formas se pueden ``parentizar'',
  donde la regla es más fácil de explicar recursivamente:
  \(x\) mismo es una parentización;
  y si lo son \(\sigma_1\) a \(\sigma_k\) con \(k \ge 2\),
  también lo es \((\sigma_1 \sigma_2 \dotsm \sigma_k)\).
  Por ejemplo,
  \((((x x) x (x x x)) (x x))\)
  es una parentización de \(x x x x x x x x\).
  El método simbólico aplicado a esta descripción recursiva
  lleva a:%
    \index{metodo simbolico@método simbólico}
  \begin{equation}
    \label{eq:Schroeder-symbolic}
    \mathcal{S}
      = \mathcal{Z} + \Seq_{\ge 2}(\mathcal{S})
  \end{equation}
  y a la ecuación funcional:
  \begin{equation*}
    S(z)
      = z + \frac{S^2(z)}{1 - S(z)}
  \end{equation*}
  No es aplicable directamente la fórmula de inversión de Lagrange,%
    \index{Lagrange, inversion de@Lagrange, inversión de}
  pero podemos resolver la cuadrática:
  \begin{equation}
    \label{eq:Schroeder-functional}
    2 S^2(z) - (z + 1) S(z) + z
      = 0
  \end{equation}
  lo que al descartar la solución espuria da:
  \begin{equation}
    \label{eq:Schroeder-gf}
    S(z)
      = \frac{1}{4} \, \left( 1 + z - \sqrt{1 - 6 z + z^2} \right)
  \end{equation}
  Expandiendo en serie:
  \begin{equation*}
    S(z)
      = z + z^2 + 3 z^3 + 11 z^4 + 45 z^5 + 197 z^6 + 903 z^7
	  + 4\,279 z^8 + 20\,793 z^9 + 103\,049 z^{10} + \dotsb
  \end{equation*}

  La ecuación~\eqref{eq:Schroeder-gf} es incómoda.
  Derivando~\eqref{eq:Schroeder-functional}
  y despejando \(S'(z)\):
  \begin{equation}
    \label{eq:Schroeder-differential}
    S'(z)
      = \frac{S(z) - 1}{4 S(z) - z - 1}
  \end{equation}
  Observamos de~\eqref{eq:Schroeder-gf} que:
  \begin{equation*}
    4 S(z) - z - 1
      = \sqrt{1 - 6 z + z^2}
  \end{equation*}
  Amplificando la fracción en~\eqref{eq:Schroeder-differential} por esto,
  substituyendo el resultado de despejar \(S^2(z)\)
  de~\eqref{eq:Schroeder-functional}
  y simplificando:
  \begin{equation}
    \label{eq:Schroeder-differential-2}
    (z^2 - 6 z + 1) S'(z) - (z - 3) S(z)
      = -z + 1
  \end{equation}
  Substituyendo la serie de potencias \(S(z)\)
  en~\eqref{eq:Schroeder-differential-2}
  e igualando coeficientes de \(z^n\)
  resulta la recurrencia para \(n \ge 1\)
  (esto permite evitar las situaciones especiales
   que introduce el lado derecho de~\eqref{eq:Schroeder-differential-2}):
  \begin{equation}
    \label{eq:Schroeder-recurrence}
    (n + 2) s_{n + 2} - 3 (2 n + 1) s_{n + 1} + (n - 1) s_n
      = 0
  \end{equation}
  Expandiendo~\eqref{eq:Schroeder-gf} tenemos \(s_1 = s_2 = 1\)
  como puntos de partida.

  Acá obtuvimos una ecuación diferencial lineal%
    \index{ecuacion diferencial@ecuación diferencial}
  manipulando la ecuación funcional para la función generatriz,
  y de ella extrajimos una recurrencia,
  más cómoda para calcular los coeficientes que la función generatriz.
  Esto puede extenderse
  al caso general de ecuaciones funcionales algebraicas,
  como muestra Bostan, Chyzak, Salvy, Lecerf y Schost~%
    \cite{bostan07:_differ_equat_algeb_funct}.

  Otra opción es el camino siguiente:
  \begin{align*}
    \frac{S(z)}{z}
      &= 1 + \frac{S(z)}{z} \, \frac{S(z)}{1 - S(z)} \\
    \intertext{Despejando \(S / z\) resulta:}
    S(z)
      &= z \, \frac{1 - S(z)}{1 - 2 S(z)}
  \end{align*}
  donde sí es aplicable inversión de Lagrange.
  El resultado es complicado,
  y lo omitiremos.

\section{Números de Stirling de segunda especie}
\label{sec:Stirling-2}
\index{Stirling, numeros de@Stirling, números de!segunda especie|textbfhy}

% Fixme: Agregar más manipulación, identidades, manipulación (GKP, TAoCP)

  Interesa el número de maneras
  de dividir el conjunto \(\{1, 2, 3, 4\}\) en dos clases,
  como ilustra el cuadro~\ref{tab:S-4-2}.
  \begin{table}[htbp]
    \centering
    \begin{tabular}{*{2}{>{\(}l<{\)}}}
      \{1\}	  & \{2, 3, 4\} \\
      \{1, 2\}	  & \{3, 4\}	\\
      \{1, 3\}	  & \{2, 4\}	\\
      \{1, 4\}	  & \{2, 4\}	\\
      \{1, 2, 3\} & \{4\}	\\
      \{1, 2, 4\} & \{3\}	\\
      \{1, 3, 4\} & \{2\}
    \end{tabular}
    \caption{Las 7 particiones de 4 elementos en 2 clases}
    \label{tab:S-4-2}
  \end{table}
  Esto muestra que hay 7 particiones de 4 elementos en 2 clases.
  El número de maneras de dividir un conjunto en clases
  lo da el \emph{número de Stirling de segunda especie},
  se anota \(\classes{n}{k}\) para el número de formas
  de dividir un conjunto de \(n\) elementos en \(k\) particiones.
  Cabe hacer notar que esta notación,
  originada por Karamata~%
    \cite{karamata35:_sommab_exp},
  es relativamente común
  (uno de sus campeones es Knuth,%
     \index{Knuth, Donald E.}
   por ejemplo~\cite{graham94:_concr_mathem, knuth92:_two_notes_notat}),
  aunque hay una gran variedad de notaciones,
  algunas con signo.
  Claramente para los números de Bell%
    \index{Bell, Eric Temple}%
    \index{Bell, numeros de@Bell, números de}
  vistos en la sección~\ref{sec:rotulados}:
  \begin{equation}
    \label{eq:Bell-Stirling}
    B_n
      = \sum_{1 \le k \le n} \classes{n}{k}
  \end{equation}

  Una aplicación es
  contar el número de funciones sobre de \([n]\) a \([k]\):%
    \index{funcion@función!sobreyectiva!numero@número}
  Corresponde
  a particionar el dominio en las preimágenes de cada elemento del rango,
  y podemos asignar valores de la función
  a cada una de las \(k\)~particiones
  de \(k!\)~maneras,
  con lo que \(\classes{n}{k} k!\) es el valor buscado.

  Para obtener \(\classes{n}{k}\),
  consideremos dos grupos de particiones:%
    \index{Stirling, numeros de@Stirling, números de!segunda especie!recurrencia}
  \begin{description}
  \item[\boldmath Aquellas en que \(n\) está solo:\unboldmath]
    Corresponden a tomar \(k - 1\) particiones
    de los demás \(n - 1\) elementos,
    hay \(\classes{n - 1}{k - 1}\) de estas.
  \item[\boldmath Aquellas en que \(n\) está con otros elementos:\unboldmath]
    Se construyen en base a \(k\) clases
    de los restantes \(n - 1\) elementos
    vía agregar \(n\) a cada clase,
    hay \(k \cdot \classes{n - 1}{k}\) de estas.
  \end{description}
  Estas dos opciones son excluyentes y exhaustivas,
  y:
  \begin{equation}
    \label{eq:recurrence-Stirling-2}
    \classes{n}{k} = \classes{n - 1}{k - 1} + k \classes{n - 1}{k}
  \end{equation}
  Donde:
  \begin{equation*}
    \classes{n}{0}
      = [n = 0]
    \qquad
    \classes{n}{n}
      = 1
  \end{equation*}
  Si además decretamos:
  \begin{equation*}
    \classes{n}{k} =
      \begin{cases}
	0 & n < 0 \\
	0 & k < 0 \\
	0 & k > n
      \end{cases}
  \end{equation*}
  la recurrencia \emph{siempre} se cumple.
  Una tabla de los números de Stirling de segunda especie
  en forma de triángulo
  (como el triángulo de Pascal del cuadro~\ref{tab:triangulo-Pascal})
  comienza como ilustra el cuadro~\ref{tab:triangulo-Stirling-2}.
  \begin{table}[htbp]
    \centering
    \begin{tabular}{>{\(}r<{\)}*{12}{>{\(}c<{\)}@{\hspace{1ex}}}>{\(}c<{\)}}
      n=0:& \phantom{00}
		& \phantom{00}
		    & \phantom{00}
			& \phantom{00}
			    & \phantom{00}
				 & \phantom{00}
				      & 1 \\
	 \noalign{\smallskip\smallskip}
      n=1:&	&   &	&   &	 &  0 & \phantom{00}
					  &  1 \\
	 \noalign{\smallskip\smallskip}
      n=2:&	&   &	&   &  0 &    &	 1 & \phantom{00}
					       &  1 \\
	 \noalign{\smallskip\smallskip}
      n=3:&	&   &	& 0 &	 &  1 &	   &  3 & \phantom{00}
						    &  1 \\
	 \noalign{\smallskip\smallskip}
      n=4:&	&   & 0 &   &  1 &    &	 7 &	&  6 & \phantom{00}
							 &  1 \\
	 \noalign{\smallskip\smallskip}
      n=5:&	& 0 &	& 1 &	 & 15 &	   & 25 &    & 10 & \phantom{00}
							      &	 1
	      & \phantom{00} \\
	 \noalign{\smallskip\smallskip}
      n=6:& 0 &	  & 1 &	  & 31 &    & 90 &	& 65 &	  & 15 & \phantom{00}
								    &  1 \\
	 \noalign{\smallskip\smallskip}
    \end{tabular}
    \caption{Números de Stirling de segunda especie}
    \label{tab:triangulo-Stirling-2}
    \index{Stirling, numeros de@Stirling, números de!segunda especie!cuadro}
  \end{table}
  Podemos optar entre tres funciones generatrices:
  Multiplicar por \(u^k\) y sumar sobre \(k\),
  multiplicar por \(z^n\) y sumar sobre \(n\)
  o multiplicar por \(u^k z^n\) sumar sobre ambos índices.
  Al sumar sobre \(k\) el factor \(k\) resulta en una derivada,
  se optaría por sumar solo sobre \(n\),
  que da ecuaciones más simples de tratar.
  Resulta eso sí una recurrencia para la función generatriz del caso.
  El desarrollo es largo,
  y lo omitiremos,
  vea el texto de Wilf~\cite{wilf06:_gfology}.

  Es más simple aplicar el método simbólico.%
    \index{metodo simbolico@método simbólico}
  Para obtener el número de particiones de \(n\) elementos en \(k\) clases
  partimos de la expresión simbólica:
  \begin{equation*}
    \mathcal{S}
      = \MSet(\mathcal{U} \times \MSet_{\ge 1}(\mathcal{Z}))
  \end{equation*}
  en la que \(\mathcal{U}\) contiene un único elemento de tamaño uno
  (usado para contabilizar el número de clases
   asociándolo a la variable \(u\),
   mientras \(z\) cuenta el número de elementos total),
  lo que lleva a la función generatriz mixta%
    \index{generatriz!multivariada}
  (exponencial en \(z\),
   los elementos están rotulados;
   y ordinaria en \(u\),
   las clases no lo están):
  \begin{equation}
    \label{eq:Stirling-2-EGF}
    \index{Stirling, numeros de@Stirling, números de!segunda especie!funcion generatriz@función generatriz}
    S(z, u)
      = \mathrm{e}^{u (\mathrm{e}^z - 1)}
  \end{equation}
  De acá:
  \begin{align}
    \classes{n}{k}
      &= n! \left[ z^n u^k \right] S(u, z) \notag \\
      &= n! \left[ z^n \right] \, \frac{(\mathrm{e}^z - 1)^k}{k!} \notag \\
      &= \frac{n!}{k!} \left[ z^n \right] \,
	    \sum_{0 \le r \le k} (-1)^{k - r} \,
	       \binom{k}{r} \, \mathrm{e}^{r z} \notag \\
      &= \frac{n!}{k!}
	    \sum_{0 \le r \le k} (-1)^{k - r} \, \binom{k}{r} \, \frac{r^n}{n!}
		\notag \\
      &= \sum_{0 \le r \le k} \frac{(-1)^{k - r} r^n}{r! (k - r)!}
		\label{eq:Stirling-2-explicit}
  \end{align}

\section{Números de Stirling de primera especie}
\label{sec:Stirling-1}
\index{Stirling, numeros de@Stirling, números de!primera especie|textbfhy}

  Interesa el número de maneras
  de organizar \(n\) elementos en \(k\) ciclos.
  Esto queda representado por la expresión simbólica:%
    \index{metodo simbolico@método simbólico}
  \begin{equation*}
    \mathcal{C}
      = \MSet(\mathcal{U} \times \Cyc(\mathcal{Z}))
  \end{equation*}
  de donde el método simbólico entrega directamente
  la función generatriz mixta%
    \index{generatriz!multivariada}
  (exponencial en \(z\),
   los elementos están rotulados;
   y ordinaria en \(u\),
   los ciclos no lo están):
  \begin{equation}
    \label{eq:Stirling-1-EGF}
    \index{Stirling, numeros de@Stirling, números de!primera especie!funcion generatriz@función generatriz}
    C(z, u)
      = \exp \left( u \ln \frac{1}{1 - z} \right)
      = (1 - z)^{-u}
  \end{equation}
  Los coeficientes
  se conocen como \emph{números de Stirling de primera especie}
  y se anota \(\cycle{n}{k}\)
  (nuevamente notación impulsada por Knuth~%
    \index{Knuth, Donald E.}%
    \cite{knuth92:_two_notes_notat}).
  Se lee ``\(n\) ciclo \(k\)''
  (en inglés
   se expresa \emph{\(n\) \foreignlanguage{english}{cycle} \(k\)}).

  Para derivar una recurrencia para ellos,%
    \index{Stirling, numeros de@Stirling, números de!primera especie!recurrencia}
  consideremos cómo podemos construir
  una organización de \(n\) objetos con \(k\) ciclos
  a partir de \(n - 1\) objetos.
  Al agregar el nuevo objeto,
  podemos ponerlo en un ciclo por sí mismo,
  lo que puede hacer de una única manera
  partiendo
  de cada una de las \(\cycle{n - 1}{k - 1}\) organizaciones
  de \(n - 1\) elementos con \(k - 1\) ciclos.
  La otra opción
  es insertarlo en alguno de los \(k\) ciclos ya existentes.
  Si suponemos \(n - 1\) elementos y \(k\) ciclos:
  \begin{equation*}
    (a_1 \, a_2 \dotso a_{j_1})
    (a_{j_1 + 1} \, a_{j_1 + 2} \dotso a_{j_2})
    \dotso
    (a_{j_{k - 1} + 1} \, a_{j_{k - 1} + 2} \dotso a_{n - 1})
  \end{equation*}
  En ella podemos insertar el nuevo elemento antes de cada elemento,
  agregándolo al ciclo al que este pertenece;
  insertar el elemento al final del ciclo
  es lo mismo que ubicarlo al comienzo de éste,
  por lo que esto no aporta nuevas opciones.
  De esta forma para cada caso
  hay \(n - 1\) posibilidades de insertar el nuevo elemento:
  \begin{equation}
    \label{eq:recurrence-Stirling-1}
    \cycle{n}{k}
      = (n - 1) \cycle{n - 1}{k} + \cycle{n - 1}{k - 1}
  \end{equation}
  Para condiciones de contorno,
  tenemos:
  \begin{equation*}
    \cycle{n}{0}
      = [n = 0]
    \qquad
    \cycle{n}{n}
      = 1
  \end{equation*}
  Si además decretamos:
  \begin{equation*}
    \cycle{n}{k} =
      \begin{cases}
	0 & n < 0 \\
	0 & k < 0 \\
	0 & k > n
      \end{cases}
  \end{equation*}
  la recurrencia \emph{siempre} se cumple.

  En forma de triángulo à la Pascal
  tenemos el cuadro~\ref{tab:triangulo-Stirling-1}.
  \begin{table}[htbp]
    \centering
    \begin{tabular}{>{\(}r<{\)}*{12}{>{\(}c<{\)}@{\hspace{1ex}}}>{\(}c<{\)}}
      n=0:& \phantom{000}
		  & \phantom{000}
		       & \phantom{000}
			    & \phantom{000}
				 & \phantom{000}
				      & \phantom{000}
					   &  1 \\
	 \noalign{\smallskip\smallskip}
      n=1:&	  &    &    &	 &    &	 0 & \phantom{000}
						&  1 \\
	 \noalign{\smallskip\smallskip}
      n=2:&	  &    &    &	 &  0 &	   &  1 & \phantom{000}
						     &	1 \\
	 \noalign{\smallskip\smallskip}
      n=3:&	  &    &    &  0 &    &	 2 &	&  3 & \phantom{000}
							  &  1 \\
	 \noalign{\smallskip\smallskip}
      n=4:&	  &    &  0 &	 &  6 &	   & 11 &    &	6 & \phantom{000}
							       &  1 \\
	 \noalign{\smallskip\smallskip}
      n=5:&	  &  0 &    & 24 &    & 50 &	& 35 &	  & 10 & \phantom{000}
								    &  1
	       & \phantom{000} \\
	 \noalign{\smallskip\smallskip}
      n=6:& 0	  &    &120 &	 &274 &	   &225 &    & 85 &    & 15
	       & \phantom{000} &  1 \\
	 \noalign{\smallskip\smallskip}
    \end{tabular}
    \caption{Números de Stirling de primera especie}
    \label{tab:triangulo-Stirling-1}
    \index{Stirling, numeros de@Stirling, números de!primera especie!cuadro}
  \end{table}
  Hay fórmulas explícitas,
  pero son complicadas y las omitiremos.

\section{Números de Lah}
\label{sec:numeros-lah}
\index{Lah, numeros de@Lah, números de|textbfhy}
\index{Stirling, numeros de@Stirling, números de!tercera especie|see{ Lah, números de}}

  Los números de Lah~%
    \cite{lah54:_new_kind_number}
  cuentan el número de maneras
  de ordenar \(n\) elementos en \(k\) secuencias.
  Se les ha llamado ``números de Stirling de tercera especie''
  por analogía a los anteriores,
  y Petkovšek y Pisanski~%
    \cite{petko02:_combin_lah_stirling}
  les dan la notación \(\lah{n}{k}\) que usaremos.
  También es común \(L_{n, k}\).
  Queda representado por la expresión simbólica:%
    \index{metodo simbolico@método simbólico}
  \begin{equation*}
    \mathcal{L}
      = \MSet(\mathcal{U} \times \Seq_{\ge 1}(\mathcal{Z}))
  \end{equation*}
  y el método simbólico da la función generatriz mixta%
    \index{generatriz!multivariada}
  (elementos rotulados,
   secuencias sin rotular):
  \begin{equation}
    \label{eq:Lah-EGF}
    L(z, u)
      = \mathrm{e}^{u ((1 - z)^{-1} - 1)}
      = \mathrm{e}^{u z (1 - z)^{-1}}
  \end{equation}
  Podemos extraer una fórmula explícita de~\eqref{eq:Lah-EGF}:%
    \index{Lah, numeros de@Lah, números de!formula@fórmula}
  \begin{align}
    \lah{n}{k}
      &= n! \left[ u^k z^n \right]
	      \, \exp \left( u z (1 - z)^{-1} \right) \\
      &= n! \left[ z^n \right] \, \frac{z^k (1 - z)^{-k}}{k!} \\
      &= \frac{n!}{k!} \, \left[ z^{n - k} \right] (1 - z)^{-k} \\
      &= \frac{n!}{k!} \, (-1)^{n - k} \binom{-k}{n - k} \\
      &= \frac{n!}{k!} \, \binom{n - 1}{k - 1}
  \end{align}

  Para derivar una recurrencia para estos números,%
    \index{Lah, numeros de@Lah, números de!recurrencia}
  usamos la misma técnica anterior.
  Veamos cómo podemos construir
  \(k\) secuencias tomadas entre \(n\) elementos
  partiendo con las configuraciones de \(n - 1\) elementos.
  Hay dos posibilidades exhaustivas y excluyentes:
  \(n\) forma una secuencia por sí solo,
  cosa que puede hacerse de una única forma
  partiendo
  con cada una de las \(\lah{n - 1}{k - 1}\) configuraciones
  con \(n - 1\) elementos y \(k - 1\) secuencias;
  o podemos agregar \(n\) a alguna de \(k\) secuencias
  en configuraciones de \(n - 1\) elementos.
  Podemos agregar un nuevo elemento a una secuencia de largo \(e\)
  de \(e + 1\) formas
  (antes del primero,
   o después de cada uno de los \(e\) elementos).
  Como los largos de las secuencias suman \(n - 1\),
  en total
  creamos \((n + k - 1) \lah{n - 1}{k}\) nuevas configuraciones.
  Resulta:
  \begin{equation}
    \label{eq:recurrence-Lah}
    \lah{n}{k}
      = (n + k - 1) \lah{n - 1}{k} + \lah{n - 1}{k - 1}
  \end{equation}
  Para condiciones de contorno tenemos:
  \begin{equation*}
    \lah{n}{0}
      = [n = 0]
    \qquad
    \lah{n}{n}
      = 1
  \end{equation*}
  Si además decretamos:
  \begin{equation*}
    \lah{n}{k} =
      \begin{cases}
	0 & n < 0 \\
	0 & k < 0 \\
	0 & k > n
      \end{cases}
  \end{equation*}
  la recurrencia \emph{siempre} se cumple.

  Al estilo del triángulo de Pascal
  tenemos el cuadro~\ref{tab:triangulo-Lah}.
  \begin{table}[htbp]
    \centering
    \begin{tabular}{>{\(}r<{\)}*{12}{>{\(}c<{\)}@{\hspace{1ex}}}>{\(}c<{\)}}
      n=0:& \phantom{0000}
		  & \phantom{0000}
		       & \phantom{0000}
			    & \phantom{0000}
				 & \phantom{0000}
				      & \phantom{0000}
					   &  1 \\
	 \noalign{\smallskip\smallskip}
      n=1:&	  &    &     &	  &	 &   0 & \phantom{0000} &  1 \\
	 \noalign{\smallskip\smallskip}
      n=2:&	  &    &     &	  &    0 &     &  1 & \phantom{0000} &	 1 \\
	 \noalign{\smallskip\smallskip}
      n=3:&	  &    &     &	 0 &	 &   6 &	 &   6
	       & \phantom{0000} &  1 \\
	 \noalign{\smallskip\smallskip}
      n=4:&	  &    &   0 &	  &   24 &     & 36 &	  & 12
	       & \phantom{0000} &  1 \\
	 \noalign{\smallskip\smallskip}
      n=5:&	  &  0 &     & 120 &	 & 240 &	  & 120 &    &	20
	       & \phantom{0000} & 1 & \phantom{0000} \\
	 \noalign{\smallskip\smallskip}
      n=6:& 0	  &    & 720 &	   &1800 &	   &1200 &    & 300 &	 & 30
	       & \phantom{0000} &  1 \\
	 \noalign{\smallskip\smallskip}
    \end{tabular}
    \caption{Números de Lah}
    \label{tab:triangulo-Lah}
    \index{Lah, numeros de@Lah, números de!cuadro}
  \end{table}

\section{Potencias, números de Stirling y de Lah}
\label{sec:potencias-Stirling-Lah}
\index{Stirling, numeros de@Stirling, números de!y potencias}
\index{Lah, numeros de@Lah, números de!y potencias}

  Las potencias factoriales aparecen con bastante regularidad%
    \index{potencia!factorial}
  al calcular con diferencias finitas,%
    \index{diferencias finitas}
  como muestran entre otros Graham, Knuth y~Patashnik~%
    \cite{graham94:_concr_mathem}.
  Como vimos
  en la sección~\ref{sec:preliminares-potencias-factoriales}
  hay paralelos entre las diferencias finitas
  y la derivada,
  y entre la sumatoria y la integral.
  Este tipo de relaciones se explotan en el cálculo umbral~%
    \cite{roman78:_umbral_calculus}.%
    \index{calculo umbral@cálculo umbral}

  Interesa expresar la potencia \(z^n\)
  en términos de los \(z^{\underline{k}}\),
  o sea obtener los coeficientes \(S(n, k)\)
  en la expansión siguiente:
  \begin{equation*}
    z^n = \sum_{0 \le k \le n} S(n, k) z^{\underline{k}}
  \end{equation*}
  Cuando \(k < 0\) o \(k > n\) debe ser \(S(n, k) = 0\),
  con lo que los límites en realidad son superfluos.
  Además,
  para \(n = 0\)
  resulta \(S(0, 0) = 1\),
  es \(S(n, 0) = 0\) si \(n > 0\)
  y es claro que \(S(n, n) = 1\).

  Escribamos:
  \begin{align*}
    z^{n + 1}
      &= \sum_k S(n, k) z^{\underline{k}} \cdot z \\
      &= \sum_k
	   S(n, k) (z^{\underline{k + 1}} + k z^{\underline{k}}) \\
      &= \sum_k (S(n, k - 1) + k S(n, k)) z^{\underline{k}}
  \end{align*}
  Comparando coeficientes de esto con la expansión de \(z^{n + 1}\):
  \begin{equation*}
    S(n + 1, k) = k S(n, k) + S(n, k - 1)
  \end{equation*}
  Tenemos las condiciones de contorno:
  \begin{equation*}
    S(n, 0)
      = [n = 0]
    \qquad
    S(n, n)
      = 1
  \end{equation*}
  El lector astuto reconocerá esto
  como la recurrencia~\eqref{eq:recurrence-Stirling-2}
  que obtuvimos para los números de Stirling de segunda especie,
  tenemos:
  \begin{equation}
    \label{eq:Stirling-2-down}
    z^n = \sum_k \classes{n}{k} z^{\underline{k}}
  \end{equation}

  Veamos ahora los coeficientes \(C(n, k)\) en:
  \begin{equation*}
    z^{\overline{n}}
      = \sum_k C(n, k) z^k
  \end{equation*}
  Con esto:
  \begin{align*}
    z^{\overline{n + 1}}
      &= \sum_k C(n, k) z^k (z + n) \\
      &= \sum_k (C(n, k) z^{k + 1} + n C(n, k) z^k) \\
      &= \sum_k (C(n, k - 1) + n C(n, k)) z^k
  \end{align*}
  Comparando coeficientes
  con la expansión de \(z^{\overline{n + 1}}\):
  \begin{equation*}
    C(n + 1, k)
      = n C(n, k) + C(n, k - 1)
  \end{equation*}
  con condiciones de contorno:
  \begin{equation*}
    C(n, 0) = [n = 0] \qquad C(n, n) = 1
  \end{equation*}
  Esto coincide con los números de Stirling de primera especie,
  o sea es:%
    \index{potencia!factorial}
  \begin{equation}
    \label{eq:Stirling-1-up}
    z^{\overline{n}}
      = \sum_k \cycle{n}{k} z^k
  \end{equation}

  En vista que nos ha ido tan bien con esto,
  consideremos los coeficientes \(L(n, k)\) en:
  \begin{equation*}
    z^{\overline{n}}
      = \sum_k L(n, k) z^{\underline{k}}
  \end{equation*}
  Aplicando la misma estrategia:
  \begin{align*}
    z^{\overline{n + 1}}
      &= \sum_k L(n, k) z^{\underline{k}} (z + n) \\
      &= \sum_k L(n, k) (z^{\underline{k + 1}} + n + k) \\
      &= \sum_k (L(n, k - 1) + (n + k) L(n, k)) z^{\underline{k}}
  \end{align*}
  Comparar coeficientes da:
  \begin{equation*}
    L(n + 1, k)
      = (n + k) L(n, k) + L(n, k - 1)
  \end{equation*}
  con condiciones de borde:
  \begin{equation*}
    L(n, 0) = [n = 0] \qquad L(n, n) = 1
  \end{equation*}
  Esta es la recurrencia~\eqref{eq:recurrence-Lah}
  de los números de Lah:
  \begin{equation}
    \label{eq:Lah-up}
    z^{\overline{n}}
      = \sum_k \lah{n}{k} z^{\underline{k}}
  \end{equation}

  Podemos usar
  la identidad \((-z)^{\underline{m}} = (-1)^m z^{\overline{m}}\)
  y viceversa para obtener de~\eqref{eq:Stirling-2-down},
  \eqref{eq:Stirling-1-up}
  y~\eqref{eq:Lah-up}:
  \begin{align}
    z^n
      &= \sum_k (-1)^{n - k} \classes{n}{k} z^{\overline{k}}
	      \label{eq:Stirling-2-up} \\
    z^{\underline{n}}
      &= \sum_k (-1)^{n - k} \cycle{n}{k} z^k
	      \label{eq:Stirling-1-down} \\
    z^{\underline{n}}
      &= \sum_k (-1)^{n - k} \lah{n}{k} z^{\overline{k}}
	      \label{eq:Lah-down}
  \end{align}

  La impresionante colección de identidades~%
    \eqref{eq:Stirling-2-down},
    \eqref{eq:Stirling-1-up},
    \eqref{eq:Lah-up},
    \eqref{eq:Stirling-2-up},
    \eqref{eq:Stirling-1-down}
    y \eqref{eq:Lah-down}
  cumple la promesa dada por el título.

\section{Desarreglos}
\label{sec:desarreglos}
\index{desarreglo}

  Ya calculamos el número de desarreglos
  (permutaciones sin puntos fijos)
  como ejemplo del uso del principio de inclusión y exclusión
  en el capítulo~\ref{cha:pie}
  y como ejemplo del método simbólico
  en la sección~\ref{sec:rotulados}.
  Acá veremos técnicas alternativas.

  Llamamos \(D_n\) al número de desarreglos
  de \(n\) elementos.
  Valores de \(D_n\) para \(n\) pequeños son interesantes
  para contrastar nuestros resultados luego:
  \(D_0 = 1\)
  (hay una única manera de ordenar cero elementos,
   y en esa ningún elemento está en su posición),
  \(D_1 = 0\)
  (un elemento puede ordenarse de una manera solamente,
   y ese siempre está en su posición),
  \(D_2 = 1\)
  (solo \((2\,1)\)),
  \(D_3 = 2\)
  (son \((3\,1\,2)\) y \((2\,3\,1)\)).

  Interesa obtener una recurrencia para los \(D_n\).%
    \index{desarreglo!recurrencia}
  Comenzaremos considerando una permutación de 7 elementos
  como \((2 \; 6 \; \mathbf{3} \; \mathbf{4} \; 1 \; 5 \; \mathbf{7})\).
  Esta permutación tiene tres puntos fijos
  (marcados con negrilla):
  El \(3\) está en la posición \(3\),
  el \(4\) en la posición \(4\)
  y el \(7\) en la posición \(7\).
  Hay \(D_{7 - 3}\) formas
  de ordenar los demás \(4 = 7 - 3\) elementos
  sin introducir puntos fijos adicionales
  (debemos dejarlos desordenados
   con respecto a sus posiciones propias,
   que estas se llamen \(1, 2, 3, 4\) o \(1, 2, 5, 6\) da lo mismo);
  a su vez los \(3\) elementos fijos
  se pueden elegir de \(\binom{7}{3}\) formas.
  O sea,
  hay un total de \(\binom{7}{3} \cdot D_{7 - 3}\) permutaciones
  de \(7\) elementos con \(3\) puntos fijos.

  En general,
  si hay \(k\) puntos fijos en una permutación de \(n\) elementos,
  estos pueden elegirse de \(\binom{n}{k}\) maneras;
  por lo tanto
  hay exactamente \(\binom{n}{k} \cdot D_{n - k}\) permutaciones
  con \(k\) puntos fijos.
  Toda permutación tiene puntos fijos (\(0, 1, \dotsc, n\) de ellos)
  y hay un total de \(n!\)~permutaciones,
  por lo que concluimos:
  \begin{equation}
    \label{eq:relacion-desarreglos}
    n! = \sum_{0 \le k \le n} \binom{n}{k} \cdot D_{n - k}
  \end{equation}
  Esta relación es válida para \(n \ge 0\).

  En la recurrencia~\eqref{eq:relacion-desarreglos}
  se observa que el lado derecho
  es la convolución binomial
  de la secuencia \(\left\langle 1 \right\rangle_{n \ge 0}\)
  con la secuencia \(\left\langle D_n \right\rangle_{n \ge 0}\).
  En consecuencia definimos:
  \begin{equation*}
    \widehat{D}(z)
      = \sum_{n \ge 0} D_n \, \frac{z^n}{n!}
  \end{equation*}
  Aplicamos las propiedades de funciones generatrices exponenciales%
    \index{generatriz!exponencial}
  a~\eqref{eq:relacion-desarreglos}.
  Al lado izquierdo queda una suma geométrica
  al simplificarse los factoriales:
  \begin{align}
    \frac{1}{1 - z}
      &= \mathrm{e}^z \cdot \widehat{D}(z) \notag \\
    \widehat{D}(z)
      &= \frac{\mathrm{e}^{-z}}{1 - z}
	\label{eq:EGF-desarreglos}
	\index{desarreglo!funcion generatriz@función generatriz}
  \end{align}
  De~\eqref{eq:EGF-desarreglos}:
  \begin{align*}
    D_n
      &= n! \left[ z^n \right] \widehat{D}(z) \\
      &= n! \sum_{0 \le k \le n} \! \frac{(-1)^k}{k!}
  \end{align*}
  En términos de la exponencial truncada
  definida por~\eqref{eq:exp-truncada}
  resulta~\eqref{eq:n-derangements}:%
    \index{desarreglo!expresion@expresión}
  \begin{equation*}
    D_n
      = n! \cdot \exp \rvert_n (-1)
  \end{equation*}
  Tenemos:
  \begin{equation*}
    \widehat{D}(z)
      = 1 + \frac{1}{2!} z^2 + \frac{2}{3!} z^3
	  + \frac{9}{4!} z^4 + \frac{44}{5!} z^5
	  + \frac{265}{6!} z^6 + \frac{1\,854}{7!} z^7
	  + \frac{14\,833}{8!} z^8 + \frac{133\,496}{9!} z^9
	  + \dotsb
  \end{equation*}

  Otra forma de resolver esto es partir derivando directamente
  una recurrencia para los \(D_n\).%
    \index{desarreglo!recurrencia}
  Consideremos \(n\) personas
  que eligen entre \(n\) sombreros
  de manera que ninguna se lleva el suyo.
  Numeramos a las personas
  y los respectivos sombreros de~\(1\) a~\(n\).
  La persona~\(n\)
  puede elegir el sombrero equivocado de~\(n - 1\) maneras,
  supongamos que elige el sombrero~\(k\).
  Ahora hay dos posibilidades:
  Si la persona~\(k\) toma el sombrero~\(n\),
  podemos eliminar los sombreros
  (y las personas)~\(n\) y~\(k\) de consideración,
  y el problema se reduce a distribuir
  los~\(n - 2\) sombreros restantes
  entre las otras~\(n - 2\) personas.
  Si la persona~\(k\) no toma el sombrero~\(n\),
  podemos renombrar ese como~\(k\)
  (no lo toma~\(k\) porque ahora le corresponde)
  y quedan por distribuir los \(n - 1\) sombreros restantes
  sin que a nadie le toque el suyo.
  Al revés,
  de un par de desarreglos de \(n - 1\) y \(n - 2\) elementos
  podemos construir \(n - 1\) desarreglos de \(n\) elementos
  (podemos elegir \(k\) arriba de \(n - 1\) maneras en ambos casos).
  Obtenemos:
  \begin{equation}
    \label{eq:recurrence-desarreglos}
    D_n
      = (n - 1) (D_{n - 1} + D_{n - 2})
    \quad (n \ge 2)
    \qquad \text{\(D_0 = 1\), \(D_1 = 0\)}
    \index{desarreglo!recurrencia}
  \end{equation}

  Para resolver la recurrencia~\eqref{eq:recurrence-desarreglos},
  definimos una función generatriz exponencial.%
    \index{generatriz!exponencial}
  La razón de intentar una función generatriz exponencial
  en este caso
  es que el factor \(n - 1\)
  se compensa parcialmente con \((n - 1)!\) en el denominador.
  Para \(n \ge 1\) podemos escribir:
  \begin{equation}
    \label{eq:recurrence-desarreglos-1}
    D_{n + 1}
      = n D_n + n D_{n - 1}
  \end{equation}
  Las propiedades de las funciones generatrices exponenciales
  dan:
  \begin{align*}
    \widehat{D}'(z)
      &\egf \left\langle D_{n + 1} \right\rangle_{n \ge 0} \\
    z \widehat{D}'(z)
      &\egf \left\langle n D_n \right\rangle_{n \ge 0}
  \end{align*}
  Falta el segundo término del lado derecho
  de~\eqref{eq:recurrence-desarreglos-1}:
  \begin{equation*}
    \sum_{n \ge 1} n D_{n - 1} \, \frac{z^n}{n!}
      = z \sum_{n \ge 1} D_{n - 1} \, \frac{z^{n - 1}}{(n - 1)!}
      = z \widehat{D}(z)
  \end{equation*}
  Otra forma de ver esto
  es que la secuencia
    \(\left\langle D_{n - 1} \right\rangle_{n \ge 0}\)
  corresponde a la antiderivada de \(\widehat{D}(z)\)
  (correr una posición a la derecha),
  y al multiplicar por \(n\)
  (que corresponde al operador \(z \mathrm{D}\))
  la derivada y la antiderivada se cancelan.
  Combinando las anteriores:
  \begin{align*}
    \widehat{D}'(z)
      &= z \widehat{D}'(z) + z \widehat{D}(z)
	   \qquad \widehat{D}(0) = D_0 = 1 \\
    \frac{\widehat{D}'(z)}{\widehat{D}(z)}
      &= \frac{z}{1 - z} \\
  \end{align*}
  La solución de esta ecuación diferencial es:%
    \index{ecuacion diferencial@ecuación diferencial}
  \begin{equation*}
    \widehat{D}(z)
      = \frac{\mathrm{e}^{-z}}{1 - z}
  \end{equation*}
  Con esto nuevamente tenemos~\eqref{eq:n-derangements}:
  \begin{equation*}
    D_n
      = n! \cdot  \exp \rvert_n (-1)
  \end{equation*}

  Aún otra manera de enfrentar este problema
  es masajear la recurrencia~\eqref{eq:recurrence-desarreglos-1}
  para obtener una recurrencia más simple de resolver:%
    \index{desarreglo!recurrencia}
  \begin{align}
    D_n - n D_{n - 1}
      &= - D_{n - 1} + (n - 1) D_{n - 2} \notag \\
      &= - \left( D_{n - 1} - (n - 1) D_{n - 2} \right) \notag \\
    (-1)^n \left( D_n - n D_{n - 1} \right)
      &= (-1)^{n - 1} \left( D_{n - 1} - (n - 1) D_{n - 2} \right)
	\label{eq:identity-derangements}
  \end{align}
  Lo que dice~\eqref{eq:identity-derangements}
  es que la expresión indicada es independiente de \(n\).
  Sabemos que \(D_0 = 1\) y \(D_1 = 0\),
  evaluándola para \(n = 1\) obtenemos:
  \begin{equation}
    \label{eq:identity-derangements-value}
    (-1)^1 \left( D_1 - 1 \cdot D_0 \right)
      = 1
  \end{equation}
  Esto nos da la recurrencia:
  \begin{equation}
    \label{eq:recurrence-derangements-3}
    \index{recurrencia!desarreglo}
    D_n
      = n D_{n - 1} + (-1)^n
    \qquad D_0 = 1
  \end{equation}
  Esta es una recurrencia lineal de primer orden.
  Nuevamente el factor \(n\)
  sugiere una función generatriz exponencial.%
    \index{generatriz!exponencial}
  Multiplicando por \(z^n / n!\) y sumando:
  \begin{align*}
    \sum_{n \ge 1} D_n \, \frac{z^n}{n!}
      &= \sum_{n \ge 1} n D_{n - 1} \, \frac{z}{n!}
	   + \sum_{n \ge 1} (-1)^n \, \frac{z^n}{n!} \\
    \widehat{D}(z) - D_0
      &= z \sum_{n \ge 1} D_{n - 1} \, \frac{z^{n - 1}}{(n - 1)!}
	   + \mathrm{e}^{-z} - 1 \\
  \intertext{Como \(D_0 = 1\):}
    \widehat{D}(z)
      &= z \widehat{D}(z) + \mathrm{e}^{-z} \\
    \widehat{D}(z)
      &= \frac{\mathrm{e}^{-z}}{1 - z}
  \end{align*}
  Esta es nuevamente~\eqref{eq:EGF-desarreglos},
  y tenemos~\eqref{eq:n-derangements} una vez más.

  Queda de ejercicio intentar las anteriores
  con una función generatriz ordinaria.

\section{Resultados de competencias con empate}
\label{sec:campeonatos-empate}
\index{Bell, numeros de (ordenados)@Bell, números de (ordenados)|textbfhy}

  Interesa saber cuántos resultados finales
  pueden producirse en un campeonato entre \(n\) participantes
  si pueden producirse empates,
  vale decir pueden haber varios primeros puestos,
  y en general varios en cada puesto.
  No hay puestos vacantes,
  si hay participantes en el puesto \(j\),
  los hay en todos los puestos \(1 \le i \le j\).
  A estos números se les llama \emph{números de Bell ordenados},
  como los números de Bell
  vistos en la sección~\ref{sec:rotulados}%
    \index{Bell, numeros de@Bell, números de}
  cuentan el número total de particiones
  sin especificar el orden de las mismas,
  acá las estamos ordenando.
  Good~\cite{good75:_number_orders_cand_ties_permit}
  trata estos números en gran detalle.

  Llamemos \(R_n\) al número de posibilidades indicado.
  Claramente \(R_0 = 1\), \(R_1 = 1\), \(R_2 = 3\).

  Si hay \(k\) en primer lugar,
  habrán \(n - k\) que se distribuyen de la misma forma
  desde el segundo lugar,
  o sea,
  hay \(R_{n - k}\) distribuciones de los demás.
  Como a los \(k\) campeones los estamos eligiendo entre los \(n\),
  y pueden haber de \(1\) a \(n\) en primer lugar,
  resulta la recurrencia:%
    \index{Bell, numeros de (ordenados)@Bell, números de (ordenados)!recurrencia}
  \begin{equation}
    \label{eq:ranking-recurrence-1}
    R_n
      = \sum_{ 1 \le k \le n} \binom{n}{k} \, R_{n - k}
  \end{equation}
  Para \(n \ge 1\)
  podemos completar la suma en~\eqref{eq:ranking-recurrence-1}:
  \begin{equation*}
    2 R_n
      = \sum_{0 \le k \le n} \binom{n}{k} \, R_{n - k}
  \end{equation*}
  Para \(n = 0\) tenemos:
  \begin{equation*}
    \sum_{0 \le k \le 0} \binom{0}{k} \, R_{0 - k}
      = R_0
  \end{equation*}
  En vista de esto,
  como \(R_0 = 1\),
  la recurrencia completa,
  válida para \(n \ge 0\),
  es:
  \begin{equation}
    \label{eq:ranking-recurrence-2}
    2 R_n
      = [n = 0] + \sum_{0 \le k \le n} \binom{n}{k} \, R_{n - k}
  \end{equation}
  Como la sumatoria en~\eqref{eq:ranking-recurrence-2}
  es la convolución binomial
  de las secuencias \(\langle 1 \rangle_{n \ge 0}\)
  y \(\langle R_n \rangle_{n \ge 0}\),
  definimos la función generatriz exponencial:%
    \index{generatriz!exponencial}
  \begin{equation}
    \label{eq:ranking-gf-def}
    \widehat{R}(z)
       = \sum_{n \ge 0} R_n \, \frac{z^n}{n!}
  \end{equation}
  Con~\eqref{eq:ranking-recurrence-2} tenemos así:
  \begin{align}
    2 \widehat{R}(z)
      &= 1 + \widehat{R}(z) \mathrm{e}^z \notag \\
    \widehat{R}(z)
      &= \frac{1}{2 - \mathrm{e}^z}
	    \label{eq:ranking-gf}
  \end{align}
  Expandiendo en serie:
  \begin{equation*}
    \widehat{R}(z)
      = 1 + z + \frac{3}{2!} z^2 + \frac{13}{3!} z^3
	  + \frac{75}{4!} z^4 + \frac{541}{5!} z^5
	  + \frac{4\,683}{6!} z^6 + \frac{47\,293}{7!} z^7
	  + \frac{545\,835}{8!} z^8
	  + \dotsb
  \end{equation*}

  Alternativamente,
  por el método simbólico:%
    \index{metodo simbolico@método simbólico}
  \begin{equation*}
    \mathcal{R}
      = \Seq(\Set_{\ge 1}(\mathcal{Z}))
  \end{equation*}
  y directamente:
  \begin{equation*}
    \widehat{R}(z)
      = \frac{1}{2 - \mathrm{e}^z}
  \end{equation*}

  Podemos expandir~\eqref{eq:ranking-gf} como serie geométrica:
  \begin{align*}
    \widehat{R}(z)
      &= \frac{1}{2} \, \sum_{k \ge 0}
			  \frac{\mathrm{e}^{k z}}{2^k} \\
      &= \frac{1}{2} \, \sum_{k \ge 0}
			  \frac{1}{2^k} \,
			     \left(
			       \sum_{n \ge 0} \frac{(k z)^n}{n!}
			     \right) \\
      &= \sum_{n \ge 0}
	   \frac{z^n}{n!} \,
	   \sum_{k \ge 0}
	     \frac{k^n}{2^{k + 1}}
  \end{align*}
  De acá tenemos:
  \begin{equation}
    \label{eq:ranking-explicit}
    R_n
      = \sum_{k \ge 0} \frac{k^n}{2^{k + 1}}
  \end{equation}
  Difícilmente habríamos adivinado esta expresión
  vía calcular valores
  (con la idea de demostrarla luego por inducción).%
    \index{demostracion@demostración!induccion@inducción}

\section{Particiones de enteros}
\label{sec:funciones-generatrices:particiones}
\index{numero natural@número natural!particion@partición}

  En el siglo~XX
  tema de investigaciones importantes en teoría de números
  involucraban la teoría de las particiones de enteros,
  un área en la que Euler fue pionero.
  Acá veremos solo un par de resultados curiosos,
  que ya demostró Euler.

\subsection{Particiones en general}
\label{sec:particiones-generales}

  Sea \(p(n)\) el número de formas de escribir \(n\) como suma.
  Tenemos:
  \begin{align*}
    p(1)
      &= 1 \qquad 1 \\
    p(2)
      &= 2 \qquad 2 = 1 + 1 \\
    p(3)
      &= 3 \qquad 3 = 1 + 1 + 1 = 2 + 1 \\
    p(4)
      &= 4 \qquad 4 = 1 + 1 + 1 + 1 = 2 + 1 + 1 = 2 + 2 = 3 + 1 \\
    \vdots
  \end{align*}
  Tratamos esto mediante el método simbólico,
  una partición de \(n\)
  corresponde a un multiconjunto de \(\mathbb{N}\),
  o sea nos interesa \(\MSet(\mathbb{N})\).
  Cada número aparece una vez en \(\mathbb{N}\),
  así:
  \begin{equation}
    \label{eq:partition-gf}
    P(z)
      = \sum_{n \ge 1} p(n) z^n
      = \prod_{n \ge 1} \frac{1}{1 - z^n}
  \end{equation}

\subsection{Sumandos diferentes e impares}
\label{sec:particiones-diferentes}

  Poniendo como condición
  que los sumandos no se pueden repetir
  tenemos subconjuntos de \(\mathbb{N}\),
  interesa \(\Set(\mathbb{N})\).
  Anotamos \(p_d(n)\) para el número de estas particiones,
  por \emph{\foreignlanguage{english}{different}} en inglés,
  y distinguimos así la función generatriz también:
  \begin{align}
    P_d(z)
      &= \sum_{n \ge 0} p_d (n) z^n \notag \\
      &= \prod_{n \ge 1} (1 + z^n)
	    \label{eq:partition-different-gf}
  \end{align}
  Este producto podemos expresarlo de otra forma:
  \begin{align}
    P_d(z)
      &= \prod_{n \ge 1} \frac{1 - z^{2 n}}{1 - z^n} \notag \\
      &= \prod_{n \ge 0} \frac{1}{1 - z^{2 n + 1}}
	    \label{eq:partition-odd-gf}
  \end{align}
  Este producto corresponde a sumandos impares,
  si escribimos \(p_o(n)\)
  por el número de particiones en sumandos impares
  (de \emph{\foreignlanguage{english}{odd}} en inglés),
  tenemos:
  \begin{equation}
    \label{eq:partition-odd=different}
    p_d(n) = p_o(n)
  \end{equation}
  Curioso resultado~\eqref{eq:partition-odd=different},
  obtenido únicamente
  considerando las funciones generatrices del caso.
  Y ni siquiera evaluamos ninguna de~\eqref{eq:partition-gf},
  \eqref{eq:partition-different-gf} o~\eqref{eq:partition-odd-gf}.

\subsection{Un problema de Moser y Lambek}
\label{sec:Moser-Lambek}

  En 1959,
  Leo Moser y Joe Lambek plantearon~%
    \cite{novakovic07:_gener_func}:
  \begin{theorem}
    \label{theo:Moser-Lambek}
    Hay una única forma de particionar \(\mathbb{N}_0\)
    en conjuntos \(\mathcal{A}\) y \(\mathcal{B}\)
    tales que el número de maneras
    en que se puede representar \(n \in \mathbb{N}_0\)
    como sumas \(n = a_1 + a_2\)
    con \(a_1 \ne a_2\) y \(a_1, a_2 \in \mathcal{A}\)
    es igual al número de representaciones como
    \(n = b_1 + b_2\)
    con \(b_1 \ne b_2\) y \(b_1, b_2 \in \mathcal{B}\).
  \end{theorem}
  \begin{proof}
    Definamos funciones generatrices:%
      \index{generatriz}
    \begin{equation*}
      A(z)
	= \sum_{a \in \mathcal{A}} z^a \qquad
      B(z)
	= \sum_{b \in \mathcal{B}} z^b
    \end{equation*}
    Como \(\mathcal{A}\) y \(\mathcal{B}\)
    particionan \(\mathbb{N}_0\),
    los coeficientes son \(0\) o \(1\):
    \begin{equation*}
      A(z) + B(z)
	= \frac{1}{1 - z}
    \end{equation*}
    Para las maneras en que se puede escribir \(n\)
    como suma de dos \(\mathcal{A}\) distintos:
    \begin{equation*}
      \sum_{\substack{
	      a_1 \ne a_2 \\
	      a_1, a_2 \in A
	   }}
	     z^{a_1 + a_2}
	= \frac{1}{2} \, \left( A^2(z) - A(z^2) \right)
    \end{equation*}
    De la misma forma podemos expresar el caso de \(\mathcal{B}\),
    e interesa que sean las mismas:
    \begin{align*}
      A^2(z) - A(z^2)
	&= B^2(z) - B(z^2) \\
      A(z^2) - B(z^2)
	&= A^2(z) - B^2(z) \\
	&= (A(z) - B(z)) \cdot (A(z) + B(z)) \\
	&= \frac{A(z) - B(z)}{1 - z}
    \end{align*}
    O equivalentemente:
    \begin{equation*}
      A(z) - B(z)
	= \left( A(z^2) - B(z^2) \right) (1 - z)
    \end{equation*}
    Substituyendo
      \(z \rightsquigarrow z^2, z^4, \dotsc, z^{2^{n - 1}}\)
    obtenemos:
    \begin{equation}
      \label{eq:Moser-Lambek-recurrence}
      A(z) - B(z)
	= \left( A(z^{2^n}) - B(z^{2^n}) \right)
	    \prod_{0 \le k \le n - 1} \left( 1 - z^{2^k} \right)
    \end{equation}
    lo que indica:
    \begin{equation}
      \label{eq:Moser-Lambek}
      A(z) - B(z)
	= \prod_{k \ge 0} \left( 1 - z^{2^k} \right)
    \end{equation}
    Como series formales
    (ver la sección~\ref{sec:series-secuencias})%
      \index{serie formal!convergencia}
    la secuencia
    al lado derecho de~\eqref{eq:Moser-Lambek-recurrence}
    converge a la expresión~\eqref{eq:Moser-Lambek}.

    Los términos del lado derecho de~\eqref{eq:Moser-Lambek}
    tienen coeficientes \(\pm 1\),
    con lo que determinan en forma única
    los coeficientes de \(A(z)\) y \(B(z)\),
    que son cero o uno.
    El conjunto \(\mathcal{A}\)
    son los que tienen un número par de unos
    en su expansión binaria.
  \end{proof}

\section{Contando secuencias}
\label{sec:ejemplos-secuencias}
\index{secuencia}

  Muchas situaciones
  llevan a enfrentar problemas de contar secuencias
  con ciertas restricciones.
  Veremos algunos ejemplos representativos.

  ¿Cuántas palabras de largo \(n\)
  formadas únicamente por las \(5\) vocales
  pueden formarse,
  si deben contener un número par de vocales fuertes
  (`a', `e' y `o')?

  De las solicitadas podemos formar \(1\) de largo \(0\),
  \(2\) de largo \(1\),
  \(2 \cdot 2 + 3 \cdot 3 = 13\) de largo \(2\).
  Estos valores sirven para verificar luego.

  Definamos las clases \(\mathcal{U}\)
  para palabras con un número par
  de vocales fuertes,
  y \(\mathcal{V}\) si tienen un número impar,
  con las respectivas funciones generatrices \(U(z)\) y \(V(z)\)
  donde \(z\) cuenta el número de vocales.
  Podemos definir el sistema de ecuaciones simbólicas:
  \begin{align*}
    \mathcal{U}
      &= \mathcal{E}
	    + \{i, u\} \times \mathcal{U}
	    + \{a, e, o\} \times \mathcal{V} \\
    \mathcal{V}
      &= \{a, e, o\} \times \mathcal{U}
	    + \{i, u\} \times \mathcal{V}
  \end{align*}
  El método simbólico entrega:%
    \index{metodo simbolico@método simbólico}
  \begin{align*}
    U(z)
      &= 1 + 2 z U(z) + 3 z V(z) \\
    V(z)
      &= 3 z U(z) + 2 z V(z)
  \end{align*}
  Resolvemos el sistema de ecuaciones para \(U(z)\),
  que es lo único que realmente interesa,
  y descomponemos en fracciones parciales:
  \begin{equation*}
    U(z)
      = \frac{1}{2} \cdot \frac{1}{1 - 5 z}
	  + \frac{1}{2} \cdot \frac{1}{1 + z}
  \end{equation*}
  Podemos leer los \(u_n\) de esto último,
  que son simplemente dos series geométricas:
  \begin{equation*}
    u_n = \frac{1}{2} \,\left( 5^n + (-1)^n \right)
  \end{equation*}
  Esto coincide con los valores obtenidos antes.

  Volvamos nuevamente
  a la situación de las secuencias de unos y ceros
  sin ceros seguidos,
  pero ahora interesa contar no solo el largo
  sino simultáneamente el número de unos.
  Para la clase \(\mathcal{S}\) de tales secuencias obtuvimos:
  \begin{equation*}
    \mathcal{S}
      = \mathcal{E} + \{0\} + \mathcal{S} \times \{1, 10\}
  \end{equation*}
  Si usamos \(x\) para marcar los ceros,
  e \(y\) para el número total de símbolos,
  obtenemos para la función generatriz \(S(x, y)\):%
    \index{generatriz!multivariada}
  \begin{equation*}
    S(x, y)
      = 1 + y + (x y + x y^2) S(x, y)
  \end{equation*}
  Despejando:
  \begin{equation*}
    S(x, y)
      = \frac{1 + y}{1 - x (y + y^2)}
  \end{equation*}
  Podemos expandir como serie geométrica en \(x (y + y^2)\):
  \begin{equation*}
    S(x, y)
      = \sum_{r \ge 0} x^r y^r (1 + y)^{r + 1}
  \end{equation*}
  El número de estas secuencias de largo \(n\) con \(k\) unos es:
  \begin{align*}
    \left[ x^k y^n \right] \sum_{r \ge 0} x^r y^r (1 + y)^{r + 1}
      &= \left[ y^n \right] y^k (1 + y)^{k + 1} \\
      &= \left[ y^{n - k} \right] (1 + y)^{k + 1} \\
      &= \binom{k + 1}{n - k}
  \end{align*}
  Pero \(n - k\) es el número de ceros.
  Podemos interpretar esto
  como diciendo que debemos distribuir \(n - k\) ceros
  en las \(k + 1\) posiciones separadas por los \(k\) unos.

%%% Local Variables:
%%% mode: latex
%%% TeX-master: "clases"
%%% End:


% propiedades.tex
%
% Copyright (c) 2014-2015 Horst H. von Brand
% Derechos reservados. Vea COPYRIGHT para detalles

\chapter{Propiedades adicionales}
\label{cha:propiedades-adicionales}
\index{propiedad|textbfhy}

  De manera muy similar
  a como contabilizamos las estructuras de un tamaño dado
  mediante funciones generatrices
  podemos representar el total de alguna característica.
  Dividiendo por el número de estructuras del tamaño respectivo
  tenemos el promedio del valor de interés.%
    \index{propiedad!promedio}
  Esto suele ser relevante como medida del rendimiento promedio
  de algún algoritmo o estructura de datos.%
    \index{analisis de algoritmos@análisis de algoritmos}

  Veremos dos maneras complementarias de atacar esta clase de situaciones.
  Partiremos por una representación directa,
  más sencilla de aplicar en muchos casos,
  pero que entrega información limitada.
  Luego mostraremos una técnica que permite obtener estadísticas detalladas.

\section{Funciones generatrices cumulativas}
\label{sec:generatrices-cumulativas}
\index{generatriz!cumulativa|textbfhy}

  Para precisar,
  consideremos una clase de objetos \(\mathcal{A}\).
  Como siempre el número de objetos de tamaño \(n\)
  lo anotaremos \(a_n\),
  con función generatriz:
  \begin{align}
    A(z)
      &= \sum_{\alpha \in \mathcal{A}} z^{\lvert \alpha \rvert}
	    \label{eq:A-def} \\
      &= \sum_{n \ge 0} a_n z^n
	    \label{eq:A-an}
  \end{align}
  Consideremos no sólo el número de objetos,
  sino alguna característica,
  cuyo valor para el objeto \(\alpha\) anotaremos \(\chi(\alpha)\).
  Es natural definir la \emph{función generatriz cumulativa}:%
    \index{generatriz!cumulativa|textbfhy}
  \begin{equation}
    \label{eq:cogf-def}
    C(z)
      = \sum_{\alpha \in \mathcal{A}} \chi(\alpha) z^{\lvert \alpha \rvert}
  \end{equation}
  Vale decir,
  los coeficientes son la suma de la medida \(\chi\)
  para un tamaño dado:
  \begin{equation}
    \label{eq:cogf-coefficient}
    [z^n] C(z)
      = \sum_{\lvert \alpha \rvert = n} \chi(\alpha)
  \end{equation}
  Así tenemos el valor promedio para objetos de tamaño \(n\):%
    \index{valor esperado}
    \index{propiedad!promedio}
  \begin{equation}
    \label{eq:chi-expected}
    \E_n[\chi]
      = \frac{[z^n] C(z)}{[z^n] A(z)}
  \end{equation}

  La discusión precedente es aplicable
  si tenemos objetos no rotulados entre manos.
  Si corresponden objetos rotulados,
  podemos definir las respectivas funciones generatrices exponenciales:
  \begin{align}
    \widehat{A}(z)
      &= \sum_{\alpha \in \mathcal{A}}
	   \frac{z^{\lvert \alpha \rvert}}{\lvert \alpha \rvert !}
		\label{eq:Ahat-def} \\
      &= \sum_{n \ge 0} a_n \frac{z^n}{n!}
		\label{eq:Ahat-an} \\
    \widehat{C}(z)
      &= \sum_{\alpha \in \mathcal{A}}
	   \chi(\alpha) \frac{z^{\lvert \alpha \rvert}}{\lvert \alpha \rvert !}
		\label{eq:Chat-def}
  \end{align}
  Nuevamente,
  como los factoriales en los coeficientes se cancelan:
  \begin{equation}
    \label{eq:chi-expected-hat}
    \E_n[\chi]
      = \frac{[z^n] \widehat{C}(z)}{[z^n] \widehat{A}(z)}
  \end{equation}

  Para un primer ejemplo trivial,
  consideremos secuencias binarias%
    \index{secuencia!binaria}
  y determinemos el número promedio de ceros
  en las secuencias de largo \(n\).
  Estos son secuencias de objetos sin rotular
  (intercambiar un par de ceros no cambia la secuencia).
  Podemos describir la clase de las secuencias de interés como:
  \begin{equation}
    \label{eq:binary-sequence}
    \mathcal{S}
      = \mathcal{E} + \mathcal{S} \times \{0, 1\}
  \end{equation}
  Para la función generatriz respectiva:
  \begin{equation}
    \label{eq:S-def}
    S(z)
      = \sum_{\sigma \in \mathcal{S}} z^{\lvert \sigma \rvert}
  \end{equation}
  el método simbólico lleva directamente a:%
    \index{metodo simbolico@método simbólico}
  \begin{equation*}
    S(z)
       = 1 + 2 z S(z)
  \end{equation*}
  con solución:
  \begin{equation}
    \label{eq:S-soln}
    S(z)
      = \frac{1}{1 - 2 z}
  \end{equation}
  de donde vemos que:%
    \index{propiedad!promedio}
  \begin{equation*}
    [z^n] S(z)
      = 2^n
  \end{equation*}
  Como esperábamos.

  Si llamamos \(\zeta(\sigma)\) al número de ceros
  en la secuencia \(\sigma\),
  vemos que el número total de ceros
  en todas las secuencias de largo \(\lvert \sigma \rvert + 1\)
  que pueden crearse a partir de \(\sigma\)
  es simplemente \(2 \zeta(\sigma) + 1\)
  (añadir \(1\) aporta \(\zeta(\sigma)\) ceros al total,
   agregar \(0\) aporta \(\zeta(\sigma) + 1\)).
  Siguiendo la descripción de la clase
  podemos derivar una ecuación para \(C(z)\):
  \begin{align}
    C(z)
      &= \sum_{\sigma \in \mathcal{S}}
	   \zeta(\sigma) z^{\lvert \sigma \rvert}
		     \label{eq:C-def} \\
      &= \zeta(\epsilon)
	   + \sum_{\sigma \in \mathcal{S}}
	       (2 \zeta(\sigma) + 1) z^{\lvert \sigma \rvert + 1}
		      \notag \\
      &= 2 z \sum_{\sigma \in \mathcal{S}}
	     \zeta(\sigma)  z^{\lvert \sigma \rvert}
	   + z \sum_{\sigma \in \mathcal{S}} z^{\lvert \sigma \rvert}
		      \notag \\
      &= 2 z C(z) + z S(z)
		      \label{eq:C-fe}
  \end{align}
  Con~\eqref{eq:S-soln} podemos resolver~\eqref{eq:C-fe}:
  \begin{equation}
    \label{eq:C-soln}
    C(z)
      = \frac{z}{(1 - 2 z)^2}
  \end{equation}
  De acá:
  usando la convención de Iverson
  (ver la sección~\ref{sec:sumatorias-productorias}):%
     \index{Iverson, convencion de@Iverson, convención de}
  \begin{align*}
    [z^n] C(n)
       &= [z^n] \frac{z}{(1 - 2 z)^2} \\
       &= \begin{cases}
	     0				& \text{si \(n = 0\)} \\
	     [z^{n - 1}] (1 - 2 z)^{-2} & \text{si \(n > 0\)}
	  \end{cases} \\
       &= [n > 0] n \cdot 2^{n - 1} \\
       &= n \, 2^{n - 1}
  \end{align*}
  Combinando con lo anterior:%
    \index{propiedad!promedio}
  \begin{align}
    \E_n[\zeta]
       &= \frac{[z^n] C(z)}{[z^n] S(z)} \notag \\
       &= \frac{n}{2}  \label{eq:E(zeta)}
  \end{align}
  Tal como esperábamos.

  \lstinputlisting[language=C,
		   xleftmargin=3em, numbers=left,
		   caption={Ordenamiento por inserción},
		   label=lst:insercion-2]
		  {code/insertion.c}
  Una de las áreas principales de aplicación de la combinatoria
  es el análisis detallado de algoritmos,%
    \index{analisis de algoritmos@análisis de algoritmos}
  como ilustra nuestro siguiente ejemplo.
  Analizaremos el algoritmo de ordenamiento por inserción,
  mostrado en el listado~\ref{lst:insercion-2}.%
    \index{ordenamiento!insercion@inserción}%
    \index{analisis de algoritmos@análisis de algoritmos!ordenamiento!insercion@inserción}
  Interesa particularmente el número de veces que se ejecuta la línea~9.
  Es claro que este número es \(O(n)\),
  pero interesa una descripción más precisa.

  Si suponemos que todos los valores son diferentes,
  y que todos los órdenes de los datos de entrada son igualmente probables,
  esto se reduce a analizar la permutación de los valores.
  Vemos que para un valor dado de \lstinline!i!
  los valores previos ya han sido ordenados,
  con lo que el valor de \lstinline!a[i]!
  se compara con los valores anteriores que son mayores a él,
  y éstos se mueven una posición hacia arriba en el arreglo
  en la línea~9.

  En una permutación \(\pi\) se dice que hay una \emph{inversión}%
    \index{permutacion@permutación!inversion@inversión|textbfhy}
  si \(\pi(i) > \pi(j)\) con \(i < j\).
  La parte central del análisis
  es entonces determinar el número promedio de inversiones
  en permutaciones de \(n\) elementos.
  Anotemos \(\iota(\pi)\) para el número de inversiones
  de la permutación \(\pi\),
  y definamos la función generatriz cumulativa:
  \begin{equation}
    \label{eq:I-def}
    I(z)
      = \sum_{\pi \in \mathcal{P}}
	  \iota(\pi) \frac{z^{\lvert \pi \rvert}}{\lvert \pi \rvert !}
  \end{equation}
  Nos interesa particularmente el número promedio de inversiones
  para permutaciones de tamaño~\(n\).

  Podemos describir permutaciones mediante la expresión simbólica:
  \begin{equation}
    \label{eq:P-class}
    \mathcal{P}
      = \mathcal{E} + \mathcal{P} \star \mathcal{Z}
  \end{equation}
  Vale decir,
  una permutación es vacía
  o es una permutación combinada con un elemento adicional.
  Dada la permutación \(\pi\)
  construimos permutaciones de largo \(\lvert \pi \rvert + 1\)
  añadiendo un nuevo elemento vía la operación \(\star\).
  Estamos creando \(\lvert \pi \rvert + 1\) nuevas permutaciones,
  cada una de las cuales conserva las inversiones que tiene,
  y agrega entre \(0\) y \(\lvert \pi \rvert\) nuevas inversiones
  dependiendo del valor elegido como último.
  El total de inversiones en el conjunto de permutaciones así creado
  a partir de \(\pi\) es:
  \begin{equation}
    \label{eq:iota-decomposed}
    (\lvert \pi \rvert + 1) \iota(\pi)
      + \sum_{0 \le k \le \lvert \pi \rvert} k
      = (\lvert \pi \rvert + 1) \iota(\pi)
	  +  \frac{\lvert \pi \rvert ( \lvert \pi \rvert + 1)}{2}
  \end{equation}
  Con esto tenemos la descomposición para la función generatriz cumulativa
  (la permutación de cero elementos no tiene inversiones):
  \begin{align}
    I(z)
      &= \sum_{\pi \in \mathcal{P}}
	   \iota(\pi) \frac{z^{\lvert \pi \rvert}}{\lvert \pi \rvert !}
	       \label{eq:Inv-definition} \\
      &= \iota(\epsilon)
	   + \sum_{\pi \in \mathcal{P}}
	       \left(
		 (\lvert \pi \rvert + 1) \iota(\pi)
		     +	\frac{\lvert \pi \rvert ( \lvert \pi \rvert + 1)}{2}
	       \right)
	       \frac{z^{\lvert \pi \rvert + 1}}{(\lvert \pi \rvert + 1)!}
		   \label{eq:Inv-decomposed} \\
      &= \sum_{\pi \in \mathcal{P}}
	   \iota(\pi) \frac{z^{\lvert \pi \rvert + 1}}{\lvert \pi \rvert !}
	   + \frac{1}{2}
	       \sum_{\pi \in \mathcal{P}}
		 \frac{z^{\lvert \pi \rvert + 1}}{\lvert \pi \rvert !}
		 \lvert \pi \rvert \notag \\
      &= z I(z) + \frac{1}{2} z \sum_{k \ge 0} k z^k \notag \\
      &= z I(z) + \frac{z^2}{2 (1 - z)^2}
  \end{align}
  Despejando:
  \begin{equation}
    \label{eq:Inv-explicit}
    I(z)
      = \frac{1}{2} \frac{z^2}{(1 - z)^3}
  \end{equation}
  Obtenemos el número promedio de inversiones directamente:%
    \index{permutacion@permutación!inversion@inversión!numero promedio@número promedio}%
    \index{propiedad!promedio}
  \begin{align}
    \E_n[\iota]
      &= [z^n] I(z)
	   \label{En-iota} \\
      &= \frac{1}{2} \binom{n}{2}
	   \notag \\
      &= \frac{n (n - 1)}{4}
	  \label{En-iota-explicit}
  \end{align}
  En consecuencia,
  en promedio al ordenar \(n\)~elementos
  el método de inserción mueve \(n (n - 1) / 4\)~elementos.
  Esto también resulta ser el número promedio de comparaciones de elementos,
  ver el listado~\ref{lst:insercion-2}.

  Podemos definir árboles binarios como un \emph{nodo externo}%
    \index{arbol binario@árbol binario|textbfhy}%
    \index{arbol binario@árbol binario!nodo externo|textbfhy}
  (simbolizado por \(\Box\))
  o un \emph{nodo interno}%
    \index{arbol binario@árbol binario!nodo interno|textbfhy}
  (simbolizado por \(\Circle\))
  conectado a dos árboles binarios
  (izquierdo y derecho).
  Así podemos expresar la clase de árboles binarios como:
  \begin{equation}
    \label{eq:A-class}
    \mathcal{A}
      = \Box + \Circle \times \mathcal{A} \times \mathcal{A}
  \end{equation}
  Con \(\lvert \alpha \rvert\) el número de nodos internos%
    \index{arbol binario@árbol binario!nodo interno}
  del árbol binario \(\alpha\)
  y \(\boxed{\alpha}\) su número de nodos externos,%
    \index{arbol binario@árbol binario!nodo externo}
  por inducción estructural%
    \index{induccion@inducción!estructural}:
  \begin{equation}
    \label{eq:A-internal-external-size}
    \boxed{\alpha}
      = \lvert \alpha \rvert + 1
  \end{equation}

  Si consideramos como medida de tamaño el número de nodos internos,
  la descripción simbólica~\eqref{eq:A-class} da la ecuación funcional:
  \begin{equation}
    \label{eq:A-fe}
    A(z)
      = 1 + z A^2(z)
  \end{equation}
  que entrega:
  \begin{equation}
    \label{eq:A-explicit}
    A(z)
      = \frac{1 - \sqrt{1 - 4 z}}{2 z}
  \end{equation}
  que sabemos de~\eqref{eq:gf-Catalan} da los números de Catalan:%
    \index{Catalan, numeros de@Catalan, números de}
  \begin{equation}
    \label{eq:A-coeff}
    a_n
      = C_n
      = \frac{1}{n + 1} \binom{2 n}{n}
  \end{equation}

  En un árbol binario,
  la \emph{altura} de un nodo es la distancia de la raíz.%
    \index{arbol binario@árbol binario!altura de un nodo|textbfhy}
  Se define el \emph{largo de camino interno}%
    \index{arbol binario@árbol binario!largo de camino interno|textbfhy}
  (en inglés \emph{\foreignlanguage{english}{internal path length}})
  del árbol
  como la suma de las alturas de los nodos internos,
  lo anotamos \(\pi(\alpha)\).
  El \emph{largo de camino externo}%
    \index{arbol binario@árbol binario!largo de camino externo|textbfhy}
  (en inglés \emph{\foreignlanguage{english}{external path length}})
  del árbol
  es la suma de las alturas de los nodos externos,
  que anotamos \(\xi(\alpha)\).
  Si buscamos en un árbol binario
  en el que los nodos en el subárbol izquierdo son menores que la raíz,
  y ésta a su vez menor que los nodos en el subárbol derecho,
  \(\pi(\alpha)\) es la suma de los costos
  para buscar los \(\lvert \alpha \rvert\)~nodos internos,
  mientras \(\xi(\alpha)\) es la suma de los costos
  para buscar los \(\lvert \alpha \rvert + 1\)~nodos externos,
  partiendo cada vez de la raíz.
  Si almacenamos datos en los nodos internos
  en la forma de árboles binarios de búsqueda,
  nodos externos corresponden a búsquedas fallidas

  Calculemos el promedio de los largos de camino
  en árboles binarios de \(n\)~nodos internos.
  De partida,
  ambas medidas son cero para el árbol que sólo tiene un nodo externo.
  De la descripción del árbol binario \(\alpha\)
  como nodo raíz y subárboles izquierdo y derecho
  (\(\alpha_l\) y \(\alpha_r\),
   respectivamente),
  como al agregar una raíz la altura de cada nodo aumenta en uno
  (y la suma de las alturas aumenta en el número de nodos considerados),
  podemos escribir:
  \begin{align}
    \pi(\alpha)
      &= \pi(\alpha_l) + \lvert \alpha_l \rvert
	  + \pi(\alpha_r) + \lvert \alpha_r \rvert
	      \label{eq:pi-decomposed} \\
    \xi(\alpha)
      &= \xi(\alpha_l) + \boxed{\alpha_l}
	  + \xi(\alpha_r) + \boxed{\alpha_r} \notag \\
      &= \xi(\alpha_l) + \lvert \alpha_l \rvert + 1
	  + \xi(\alpha_r) + \lvert \alpha_r \rvert + 1
	      \label{eq:xi-decomposed}
  \end{align}
  Esto lleva directamente a las ecuaciones funcionales
  para las funciones generatrices cumulativas:
  \begin{align}
    I(z)
      &= \sum_{\alpha \in \mathcal{A}} \pi(\alpha) z^{\lvert \alpha \rvert}
	   \label{eq:Ipi-def} \\
      &= \sum_{\substack{\alpha_l \in \mathcal{A} \\
			 \alpha_r \in \mathcal{A}}}
	       (\pi(\alpha_l) + \lvert \alpha_l \rvert
		  + \pi(\alpha_r) + \lvert \alpha_r \rvert)
	       z^{\lvert \alpha_l \rvert + \lvert \alpha_r \rvert + 1}
	   \label{eq:Ipi-decomposed} \\
    E(z)
      &= \sum_{\alpha \in \mathcal{A}} \xi(\alpha) z^{\lvert \alpha \rvert}
	   \label{eq:E-def} \\
      &= \sum_{\substack{\alpha_l \in \mathcal{A} \\
			 \alpha_r \in \mathcal{A}}}
	       (\xi(\alpha_l) + \lvert \alpha_l \rvert + 1
		  + \xi(\alpha_r) + \lvert \alpha_r \rvert + 1)
	       z^{\lvert \alpha_l \rvert + \lvert \alpha_r \rvert + 1}
	   \label{eq:E-decomposed}
  \end{align}
  Consideremos las sumas resultantes,
  por ejemplo:
  \begin{align*}
    \sum_{\substack{\alpha_l \in \mathcal{A} \\
		    \alpha_r \in \mathcal{A}}}
      \pi(\alpha_l) z^{\lvert \alpha_l \rvert + \lvert \alpha_r \rvert + 1}
      &= z \sum_{\alpha_l \in \mathcal{A}}
	     \pi(\alpha_l) z^{\lvert \alpha_l \rvert}
	   \cdot \sum_{\alpha_r \in \mathcal{A}} z^{\lvert \alpha_r \rvert} \\
      &= z I(z) A(z)
  \end{align*}
  Otro tipo de suma es:
  \begin{align*}
    \sum_{\substack{\alpha_l \in \mathcal{A} \\
		    \alpha_r \in \mathcal{A}}}
      \lvert \alpha_l \rvert
      z^{\lvert \alpha_l \rvert + \lvert \alpha_r \rvert + 1}
    &= z \sum_{\alpha_l \in \mathcal{A}}
	   \lvert \alpha_l \rvert z^{\lvert \alpha_l \rvert}
	 \cdot \sum_{\alpha_r \in \mathcal{A}} z^{\lvert \alpha_r \rvert} \\
    &= z^2 A'(z) A(z)
  \end{align*}
  Acá usamos:
  \begin{equation*}
    z A'(z)
      = \sum_{\alpha \in \mathcal{A}}
	  \lvert \alpha \rvert z^{\lvert \alpha \rvert}
  \end{equation*}
  Finalmente:
  \begin{align*}
    \sum_{\substack{\alpha_l \in \mathcal{A} \\
		    \alpha_r \in \mathcal{A}}}
      z^{\lvert \alpha_l \rvert + \lvert \alpha_r \rvert + 1}
      &= z \sum_{\alpha_l \in \mathcal{A}} z^{\lvert \alpha_l \rvert}
	    \cdot \sum_{\alpha_r \in \mathcal{A}} z^{\lvert \alpha_r \rvert} \\
      &= z A^2(z)
  \end{align*}
  En~\eqref{eq:Ipi-decomposed}
  los primeros tipos de suma se repiten dos veces
  (una vez al sumar sobre \(\alpha_l\) y una vez al sumar sobre \(\alpha_r\)):
  \begin{equation}
    \label{eq:Ipi-functional}
    I(z)
      = 2 z I(z) A(z) + 2 z^2 A(z) A'(z)
  \end{equation}
  Despejando \(I(z)\):
  \begin{align}
    I(z)
      &= \frac{2 z^2 A(z) A'(z)}{1 - 2 z A(z)} \notag \\
      &= \frac{1 - 3 z - (1 - z) \sqrt{1 - 4 z}}{z (1 - 4 z)}
	    \label{eq:Ipi-explicit}
  \end{align}
  Por el teorema de Bender
  (teorema~\ref{theo:Bender})%
    \index{Bender, teorema de}
  tenemos que:
  \begin{align}
    [z^n] I(z)
      &\sim \lim_{z \to 1/4}
	      \left(
		\frac{1 - 3 z - (1 - z) \sqrt{1 - 4 z}}{z}
	      \right)
	      \cdot 4^n \notag \\
      &\sim 4^n
	 \label{eq:In-asy}
  \end{align}
  Nos interesa el promedio,%
    \index{propiedad!promedio}
  para lo que según~\eqref{eq:En-value} requerimos además:
  \begin{align*}
    [z^n] A(z)
      &=    C_n \\
      &=    \frac{1}{n + 1} \binom{2 n}{n} \\
      &\sim \frac{4^n n^{-3/2}}{\sqrt{\pi}}
  \end{align*}
  (lo último de la fórmula de Stirling~\eqref{eq:Stirling}%
     \index{Stirling, formula de@Stirling, fórmula de}
   para factoriales con la expresión~\eqref{eq:Comb=f/f*f}
   para coeficientes binomiales).
  Con esto el promedio buscado es:
  \begin{align}
    \E_n[\pi]
      &=    \frac{[z^n] I(z)}{[z^n] A(z)} \notag \\
      &\sim \sqrt{\pi} n^{3/2}
	  \label{eq:En-pi-value}
  \end{align}
  El costo promedio de búsquedas exitosas en el árbol
  resulta así:%
    \index{arbol binario@árbol binario!busqueda exitosa@búsqueda exitosa}
  \begin{equation}
    \frac{\E_n[\pi]}{n}
      \sim \sqrt{\pi n}
	  \label{eq:A-success-asy}
  \end{equation}

  Para \(E(z)\) tenemos de forma similar:
  \begin{equation*}
    \label{eq:E-functional}
    E(z)
      = 2 z E(z) A(z) + 2 z^2 A(z) A'(z) + 2 z A^2(z)
  \end{equation*}
  Despejando:
  \begin{align}
    E(z)
      &= \frac{2 z^2 A(z) A'(z) + 2 z A^2(z)}{1 - 2 z A(z)}  \notag \\
      &= \frac{1 - \sqrt{1 - 4 z}}
	      {1 - 4 z}
	    \label{eq:E-explicit}
  \end{align}
  Esto es sencillo de manejar usando~\eqref{eq:binomial(-1/2,k)}
  y aproximando el coeficiente binomial
  mediante la fórmula de Stirling~\eqref{eq:Stirling}:
  \begin{align}
    [z^n] E(z)
      &=    [z^n] \frac{1}{1 - 4 z} - [z^n] (1 - 4 z)^{-1/2} \notag \\
      &=    4^n - \binom{-1/2}{n}(-4)^n \notag \\
      &=    4^n - \frac{1}{4^n} \binom{2 n}{n} \cdot 4^n \notag \\
      &\sim 4^n \left(1 - \sqrt{\frac{2}{\pi n}} \right)
	   \label{eq:E-asy}
  \end{align}
  Para el costo promedio de las \(n + 1\)~posibles búsquedas fallidas%
    \index{arbol binario@árbol binario!busqueda fallida@búsqueda fallida}
  resulta:
  \begin{align}
    \frac{\E_n[\xi]}{n + 1}
      &=    \frac{[z^n] E(z)}{(n + 1) [z^n] A(z)} \notag \\
      &\sim \sqrt{\pi n}
	  \label{eq:A-fail-asy}
  \end{align}

  Interesa analizar el comportamiento de árboles binarios de búsqueda,%
    \index{arbol binario de busqueda@árbol binario de búsqueda|textbfhy}%
    \index{analisis de algoritmos@análisis de algoritmos!arbol binario de busqueda@árbol binario de búsqueda}
  particularmente el costo promedio de búsquedas exitosas y fallidas.
  Árboles binarios de búsqueda
  normalmente se crean insertando sucesivamente los elementos a buscar,
  con lo que un modelo razonable
  es considerar que se insertan elementos de claves diferentes
  y que todas las permutaciones de los datos
  son igualmente probables.
  Nótese que estas son las mismas estructuras que consideramos antes,
  pero la distribución es diferente.

  Al elegir la raíz
  estamos dividiendo los restantes valores en dos subárboles,
  estos valores están intercalados.
  Si los tamaños de los subárboles izquierdo y derecho
  son \(\lvert \alpha_l \rvert\) y \(\lvert \alpha_r \rvert\),
  respectivamente,
  el mismo árbol binario de búsqueda
  resulta del siguiente número de permutaciones diferentes:
  \begin{equation*}
    \binom{\lvert \alpha_l \rvert + \lvert \alpha_r \rvert}
	  {\lvert \alpha_l \rvert}
  \end{equation*}
  y sabemos que los coeficientes binomiales%
    \index{coeficiente binomial}
  tienen su máximo cuando \(\lvert \alpha_l \rvert = \lvert \alpha_r \rvert\).
  Vale decir,
  construir árboles binarios de búsqueda insertando elementos
  en orden aleatorio da resultados más balanceados.

  Suponemos además que la probabilidad de buscar cada uno de los datos
  es la misma,
  y además que búsquedas fallidas tienen la misma probabilidad
  para cada rango de claves
  antes,
  entre cada par de elementos
  y luego del último.
  El costo de una búsqueda exitosa es la altura del nodo interno
  que contiene el dato buscado,
  el de una búsqueda fallida es la altura del nodo externo
  en que termina.

  Bajo los supuestos indicados,
  el primer elemento de la permutación de \(n\) elementos
  (digamos que es \(k\))
  es la raíz del árbol,
  los elementos menores que \(k\) forman el subárbol izquierdo
  mientras los elementos mayores integran el subárbol derecho.
  Dado que todas las permutaciones se suponen igualmente probables,
  también lo son
  las permutaciones de los elementos que forman los subárboles.
  Sabemos que la función generatriz para permutaciones es:
  \begin{align}
    \widehat{P}(z)
      &= \sum_{\sigma \in \mathcal{P}}
	   \frac{z^{\lvert \sigma \rvert}}{\lvert \sigma \rvert !}
	      \label{eq:P-def} \\
      &= \frac{1}{1 - z}
	      \label{eq:P-explicit}
  \end{align}
  Tenemos la función generatriz cumulativa del largo de camino interno:
  \begin{align}
    \widehat{I}(z)
      &= \sum_{\sigma \in \mathcal{P}}
	   \pi(\sigma)
	   \frac{z^{\lvert \sigma \rvert}}{\lvert \sigma \rvert !}
	      \label{eq:BST-I-def} \\
  \intertext{Descomponemos según la raíz:}
    \widehat{I}(z)
      &= \sum_{\substack{\sigma_l \in \mathcal{P} \\
			 \sigma_r \in \mathcal{P}}}
	   \binom{\lvert \sigma_l \rvert + \lvert \sigma_r \lvert}
		 {\lvert \sigma_l \rvert}
	   \frac{z^{\lvert \sigma_l \rvert + \lvert \sigma_r \rvert + 1}}
		{(\lvert \sigma_l \rvert + \lvert \sigma_r \rvert + 1)!}
	   (\pi(\sigma_l) + \pi(\sigma_r)
	      + \lvert \sigma_l \rvert + \lvert \sigma_r \rvert)
	      \label{eq:BST-I-decomposed} \\
  \intertext{Derivamos para simplificar la suma:}
    \widehat{I}'(z)
      &= \sum_{\substack{\sigma_l \in \mathcal{P} \notag \\
			 \sigma_r \in \mathcal{P}}}
	   \frac{z^{\lvert \sigma_l \rvert}}{\lvert \sigma_l \rvert !}
	   \frac{z^{\lvert \sigma_r \rvert}}{\lvert \sigma_r \rvert !}
	   (\pi(\sigma_l) + \lvert \sigma_l \rvert
	      + \pi(\sigma_r) + \lvert \sigma_r \rvert) \\
      &= 2 \widehat{I}(z) \widehat{P}(z)
	   + 2 z \widehat{P}(z) \widehat{P}'(z) \notag \\
      &= \frac{2 \widehat{I}(z)}{1 - z} + \frac{2 z}{(1 - z)^2}
	      \label{eq:BST-I-ode}
  \end{align}
  Condición inicial es que \(\widehat{I}(0) = 0\),
  ya que el árbol con un único nodo externo tiene \(\pi(\Box) = 0\).
  La solución de la ecuación diferencial~\eqref{eq:BST-I-ode} es:
  \begin{equation}
    \label{eq:BST-I-explicit}
    \widehat{I}(z)
      = \frac{2}{(1 - z)^2} \ln \frac{1}{1 - z} - \frac{2 z}{(1 - z)^3}
  \end{equation}
  Esta es esencialmente la función generatriz~\eqref{eq:Hn-sum-gf}
  de la suma de números harmónicos,
  \eqref{eq:Hn-sum} da los coeficientes:
  \begin{align}
    \E_n[\pi]
      &=    2 (n + 1) (H_{n + 1} - 1) - 2 n
	      \label{eq:BST-En-pi-explicit} \\
      &\sim 2 n \ln n
	      \label{eq:BST-En-pi-asy}
  \end{align}
  El costo promedio de una búsqueda exitosa es así:%
    \index{arbol binario de busqueda@árbol binario de búsqueda!busqueda exitosa@búsqueda exitosa|textbfhy}%
    \index{propiedad!promedio}
  \begin{equation}
    \label{eq:BST-success}
    \frac{\E_n[\pi]}{n}
      \sim 2 \ln n
  \end{equation}

  De forma similar tratamos búsquedas fallidas:
  \begin{align}
    \widehat{E}(z)
      &= \sum_{\sigma \in \mathcal{P}}
	   \xi(\sigma)
	   \frac{z^{\lvert \sigma \rvert}}{\lvert \sigma \rvert !}
	      \label{eq:BST-E-def} \\
      &= \sum_{\substack{\sigma_l \in \mathcal{P} \\
			 \sigma_r \in \mathcal{P}}}
	   \binom{\lvert \sigma_l \rvert + \lvert \sigma_r \lvert}
		 {\lvert \sigma_l \rvert}
	   \frac{z^{\lvert \sigma_l \rvert + \lvert \sigma_r \rvert + 1}}
		{(\lvert \sigma_l \rvert + \lvert \sigma_r \rvert + 1)!}
	   (\xi(\sigma_l) + \lvert \sigma_l \rvert + 1
	      + \xi(\sigma_r) + \lvert \sigma_r \rvert + 1)
	      \label{eq:BST-E-decomposed} \\
    \widehat{E}'(z)
      &= \sum_{\substack{\sigma_l \in \mathcal{P} \notag \\
			 \sigma_r \in \mathcal{P}}}
	   \frac{z^{\lvert \sigma_l \rvert}}{\lvert \sigma_l \rvert !}
	   \frac{z^{\lvert \sigma_r \rvert}}{\lvert \sigma_r \rvert !}
	   (\xi(\sigma_l) + \xi(\sigma_r)
	      + \lvert \sigma_l \rvert + 1
	      + \lvert \sigma_r \rvert + 1) \notag \\
      &= 2 \widehat{E}(z) \widehat{P}(z)
	   + 2 z \widehat{P}(z) \widehat{P}'(z)
	   + 2 \widehat{P}^2(z) \notag \\
      &= \frac{2 \widehat{E}(z)}{1 - z} + \frac{2}{(1 - z)^3}
	      \label{eq:BST-E-ode}
  \end{align}
  Nuevamente,
  como \(\xi(\Box) = 0\) es \(\widehat{E}(0) = 0\).
  Solución de la ecuación diferencial es:
  \begin{equation}
    \label{eq:BST-E-explicit}
    \widehat{E}(z)
      = \frac{2}{(1 - z)^2} \ln \frac{1}{1 - z}
  \end{equation}
  De~\eqref{eq:Hn-sum} tenemos los coeficientes:
  \begin{equation*}
    [z^n] E(z)
      = 2 (n + 1) (H_{n + 1} - 1)
  \end{equation*}
  El costo promedio de una búsqueda fallida resulta ser:%
    \index{arbol binario de busqueda@árbol binario de búsqueda!busqueda fallida@búsqueda fallida|textbfhy}%
    \index{propiedad!promedio}
  \begin{align}
    \frac{\E_n[\xi]}{n + 1}
      &=   2 (H_{n + 1} - 1)
		      \label{eq:BST-En-xi-explicit} \\
      &\sim 2 \ln n
		      \label{eq:BST-En-xi-asy}
  \end{align}

\section{Generatrices multivariadas}
\label{sec:multivariable-GF}
\index{generatriz!multivariada|textbfhy}

  Una manera distinta de atacar el problema general que hemos planteado
  es usar funciones generatrices multivariadas.
    Consideraremos una clase \(\mathcal{A}\),
  con objetos \(\alpha \in \mathcal{A}\) de tamaño \(\lvert \alpha \rvert\);
  y a su vez un parámetro,
  cuyo valor para \(\alpha\) es \(\chi(\alpha)\).
  Como hasta ahora usaremos la indeterminada \(z\) para contabilizar tamaños,
  y usaremos la indeterminada \(u\)
  para marcar el valor del parámetro de interés.
  Si los átomos que componen \(\alpha\) son indistinguibles,
  es natural definir la función generatriz ordinaria:
  \begin{equation}
    \label{eq:obgf-def}
    A(z, u)
      = \sum_{\alpha \in \mathcal{A}} z^{\lvert \alpha \rvert} u^{\chi(\alpha)}
  \end{equation}
  De la misma forma,
  si los átomos son distinguibles
  es apropiada la función generatriz exponencial:
  \begin{equation}
    \label{eq:ebgf-def}
    \widehat{A}(z, u)
      = \sum_{\alpha \in \mathcal{A}}
	  \frac{z^{\lvert \alpha \rvert}}{\lvert \alpha \rvert !}
	     u^{\chi(\alpha)}
  \end{equation}
  Es común que nos interese el valor promedio de \(\chi(\alpha)\)
  para objetos de tamaño dado.%
    \index{propiedad!promedio}
  Nótese que:
  \begin{equation}
    \label{eq:obgf-partial}
    \frac{\partial A}{\partial u}
      = \sum_{\alpha \in \mathcal{A}}
	  \chi(\alpha) u^{\chi(\alpha) - 1} z^{\lvert \alpha \rvert}
  \end{equation}
  Así podemos calcular los valores promedios a partir de
  los coeficientes de las siguientes sumas:
  \begin{align}
    \sum_{\alpha \in \mathcal{A}} z^{\lvert \alpha \rvert}
      &= A(z, 1)
	  \label{eq:obgf|u=1} \\
    \sum_{\alpha \in \mathcal{A}} \chi(\alpha) z^{\lvert \alpha \rvert}
      &= \left. \frac{\partial A}{\partial u} \right\rvert_{u = 1}
	  \label{eq:obgf_u|u=1}
  \end{align}
  Vemos que~\eqref{eq:obgf|u=1}
  no es más que la función generatriz~\eqref{eq:A-def}
  del número de objetos,
  mientras~\eqref{eq:obgf_u|u=1}
  es la función generatriz cumulativa~\eqref{eq:cogf-def}.

  En aras de brevedad,
  usaremos la notación:%
    \index{derivada parcial, notacion@derivada parcial, notación}
  \begin{equation}
    \label{eq:partial-notation}
    A_z(z, u)
      = \frac{\partial A}{\partial z}
    \qquad
    A_u(z, u)
      = \frac{\partial A}{\partial u}
  \end{equation}
  En esto el subíndice indica el argumento de la función
  (o sea,
   primer y segundo argumento respectivamente en el ejemplo).
  Para derivadas parciales superiores anotamos por ejemplo:
  \begin{equation*}
    A_{z z}(z, u)
      = \frac{\partial^2 A}{\partial z^2}
    \hspace*{3em}
    A_{u z}(z, u)
      = \frac{\partial^2 A}{\partial z \partial u}
    \hspace*{3em}
    A_{z u}(z, u)
      = \frac{\partial^2 A}{\partial u \partial z}
  \end{equation*}
  Nótese que el orden de los subíndices es el orden en que se deriva.

  En particular,
  extraer los coeficientes de \(z^n\) de las sumas mencionadas
  entrega los valores necesarios:%
    \index{valor esperado}
  \begin{align}
    \E_n[\chi]
      &= \frac{\sum_{\lvert \alpha \rvert = n} \chi(\alpha)}{a_n}
	   \label{eq:En-def} \\
      &= \frac{[z^n] A_u(z, 1)}{[z^n] A(z, 1)}
	   \label{eq:En-value}
  \end{align}
  Exactamente el mismo razonamiento
  se aplica a funciones generatrices exponenciales
  (los denominadores \(n!\) se cancelan).

  La ventaja de esta línea de desarrollo
  frente al de la sección~\ref{sec:generatrices-cumulativas}
  es que la función generatriz multivariada contiene la distribución completa
  de los valores.
  En particular,
  vemos que la función generatriz de probabilidad
  (ver~\ref{sec:PGF})
  de la medida \(\chi\) para objetos de tamaño \(n\) está dada por:
  \begin{equation}
    \label{eq:PGF-chi}
    G_n(u)
      = \frac{[z^n] A(z, u)}{[z^n] A(z, 1)}
  \end{equation}
  Aplicando~\eqref{eq:PGF-variance} vemos que:%
    \index{propiedad!varianza}
  \begin{align}
    \var_n[\chi]
      &= G_n''(1) + G_n'(1) - \left( G_n'(1) \right)^2 \\
      &= \frac{[z^n] A_{u u} (z, 1)}{[z^n] A(z, 1)}
	    + \frac{[z^n] A_u (z, 1)}{[z^n] A(z, 1)}
	    - \left( \frac{[z^n] A_u (z, 1)}{[z^n] A(z, 1)} \right)^2
		 \label{eq:varn-chi}
  \end{align}
  Cabe resaltar que en~\eqref{eq:varn-chi}
  se producirán importantes cancelaciones por la resta,
  se requieren valores precisos para los coeficientes
  para poder usarla.
  Esta fórmula se aplica sin cambios a funciones generatrices exponenciales,
  los factoriales de los denominadores se cancelan
  como ocurría en~\eqref{eq:chi-expected-hat}.

  Repitiendo el primer ejemplo,
  obtengamos el número promedio de \(0\) en secuencias binarias de largo \(n\).%
    \index{secuencia!binaria}
  Estas secuencias quedan descritas por la expresión simbólica
  (compare con~\eqref{eq:binary-sequence}):
  \begin{equation}
    \label{eq:S-class}
    \mathcal{S}
      = \Seq(\{0\} + \{1\})
  \end{equation}
  Usando \(z\) para tamaño
  (número total de símbolos)
  y \(u\) para el número de ceros,
  la función generatriz correspondiente a \(\{0\} + \{1\}\) es:
  \begin{equation}
    \label{eq:01-bgf}
    z u + z
      = z (1 + u)
  \end{equation}
  con lo que al aplicar el método simbólico resulta:
  \begin{equation}
    \label{eq:S-obgf}
    S(z, u)
      = \frac{1}{1 - z (1 + u)}
  \end{equation}
  Aplicando la técnica explicitada por~\eqref{eq:En-value}
  a~\eqref{eq:S-obgf} tenemos:
  \begin{align*}
    S(z, 1)
      &= \frac{1}{1 - 2 z} \\
    S_u(z, u)
      &= \frac{z}{(1 - z (1 + u))^2} \\
    S_u(z, 1)
      &= \frac{z}{(1 - 2 z)^2}
  \end{align*}
  Nuevamente tenemos~\eqref{eq:E(zeta)} para el promedio:%
    \index{propiedad!promedio}
  \begin{align*}
    [z^n] S(z, 1)
      &= [z^n] \frac{1}{1 - 2 z}
       = 2^n \\
    [z^n] S_u(z, 1)
      &= [z^n] \frac{z}{(1 - 2 z)^2}
       = n \, 2^{n - 1}
  \end{align*}
  En consecuencia,
  el número promedio de ceros es:
  \begin{equation}
    \label{eq:S-ave}
    \frac{[z^n] S_u(z, 1)}{[z^n] S(z, 1)}
      = \frac{n \, 2^{n - 1}}{2^n}
      = \frac{n}{2}
  \end{equation}
  Tal como esperábamos.

  Vamos por la varianza:%
    \index{propiedad!varianza}
  \begin{align*}
    S_{u u}(z, u)
      &= \frac{2 z^2}{(1 - z - z u)^3} \\
    S_{u u}(z, 1)
      &= \frac{2 z^2}{(1 - 2 z)^3} \\
    [z^n] S_{u u}(z, 1)
      &= \frac{1}{2} \binom{n}{2} 2^n \\
      &= n (n - 1) 2^{n - 2}
  \end{align*}
  Con~\eqref{eq:varn-chi}
  resulta la varianza del número de ceros
  en secuencias binarias de largo \(n\):
  \begin{align}
    \var_n[\zeta]
      &= \frac{n (n - 1) 2^{n - 2}}{2^n}
	   + \frac{n 2^{n - 1}}{2^n}
	   - \left( \frac{n 2^{n - 1}}{2^n} \right)^2 \notag \\
      &= \frac{n}{4} \label{eq:S-var}
  \end{align}

  Un ejemplo más complejo es el algoritmo obvio
  para hallar el máximo de un arreglo,%
    \index{maximo@máximo}%
    \index{analisis de algoritmos@análisis de algoritmos!maximo@máximo}
  ver el listado~\ref{lst:maximo}.
  Es evidente que el número de veces que se actualiza la variable
  \lstinline!m! es \(O(n)\),
  pero interesa dar una respuesta más precisa.
  \lstinputlisting[language=C,
		   xleftmargin=3em, numbers=left,
		   caption={Hallar el máximo},
		   label=lst:maximo]
		   {code/maximum.c}
  Necesitamos un modelo para responder a la pregunta.
  Si suponemos que todos los valores son diferentes,
  y que todas las maneras de ordenarlos son igualmente probables,
  estamos buscando el número promedio de máximos de izquierda a derecha
  de permutaciones.%
    \index{permutacion@permutación!maximos izquierda a derecha@máximos izquierda a derecha|textbfhy}
  Podemos describir la clase de permutaciones simbólicamente
  como en~\eqref{eq:P-class}:
  \begin{equation}
    \label{eq:P-class-again}
    \mathcal{P}
      = \mathcal{E} + \mathcal{P} \star \mathcal{Z}
  \end{equation}
  Si llamamos \(\chi(\sigma)\) al número de máximos de izquierda a derecha
  en la permutación \(\sigma\),
  la función generatriz de probabilidad de que una permutación de tamaño \(n\)
  tenga \(k\) máximos de izquierda a derecha es:
  \begin{equation}
    \label{eq:M-pgf}
    M(z, u)
      = \sum_{\sigma \in \mathcal{P}}
	  \frac{z^{\lvert \sigma \rvert}}{\lvert \sigma \rvert !}
	    u^{\chi(\sigma)}
  \end{equation}
  Esto casualmente es la función generatriz exponencial bivariada
  correspondiente a la clase~\eqref{eq:P-class-again}.

  Como el último elemento de la permutación
  es un máximo de izquierda a derecha
  si es el máximo de todos ellos,
  usando la convención de Iverson%
     \index{Iverson, convencion de@Iverson, convención de}
  (ver la sección~\ref{sec:sumatorias-productorias})
  podemos expresar el número de máximos de izquierda a derecha
  en la permutación resultante de \(\sigma \star (1)\)
  si se asigna el rótulo \(j\) al elemento nuevo como:
  \begin{equation}
    \label{eq:chi+1}
    \chi(\sigma) + [j = \lvert \sigma \rvert + 1]
  \end{equation}
  con lo que:
  \begin{align}
    M(z, u)
      &= \sum_{\sigma \in \mathcal{P}}
	   \sum_{1 \le j \le \lvert \sigma \rvert + 1}
	     \frac{z^{\lvert \sigma \rvert + 1}}{(\lvert \sigma \rvert + 1)!}
	       u^{\chi(\sigma) + [j = \lvert \sigma \rvert + 1]}
		  \label{eq:M-decomposed} \\
      &= \sum_{\sigma \in \mathcal{P}}
	   \frac{z^{\lvert \sigma \rvert + 1}}{(\lvert \sigma \rvert + 1)!}
	      u^{\chi(\sigma)}
	   \sum_{1 \le j \le \lvert \sigma \rvert + 1}
	     u^{[j = \lvert \sigma \rvert + 1]}
		  \notag  \\
      &= \sum_{\sigma \in \mathcal{P}}
	   \frac{z^{\lvert \sigma \rvert + 1}}{(\lvert \sigma \rvert + 1)!}
	      u^{\chi(\sigma)} (\lvert \sigma \rvert + u)
		  \label{eq:M-decomposed-result}
  \end{align}
  Derivando respecto de \(z\):
  \begin{align*}
    M_z(z, u)
      &= \sum_{\sigma \in \mathcal{P}}
	   \frac{z^{\lvert \sigma \rvert}}{\lvert \sigma \rvert !}
	   u^{\chi(\sigma)}
	   (\lvert \sigma \rvert + u) \\
      &= z M_z(z, u) + u M(z, u)
  \end{align*}
  Vale decir:
  \begin{equation}
    \label{eq:M-pde}
    (1 - z) M_z(z, u) - u M(z, u)
      = 0
  \end{equation}
  En~\eqref{eq:M-pde} la variable~\(u\) interviene como parámetro,
  esta es una ecuación diferencial ordinaria.
  Como \(M(0, u) = 1\),
  la solución es:
  \begin{align}
    M(z, u)
      &= \left( \frac{1}{1 - z} \right)^u
	    \label{eq:M-solution} \\
      &= \sum_{n, k} \cycle{n}{k} \frac{z^n}{n!} u^k
	    \label{eq:M-solution-Stirling1}
  \end{align}
  Aparecen los números de Stirling de primera especie~%
    \eqref{eq:Stirling-1-EGF},%
    \index{Stirling, numeros de@Stirling, números de!primera especie}
  o sea,
  hay tantas permutaciones de \(n\)~elementos
  con \(k\)~máximos de izquierda a derecha
  como permutaciones con \(k\)~ciclos.%
    \index{permutacion@permutación!ciclos}
  Derivando respecto de \(u\):
  \begin{align*}
    M_u(z, 1)
      &= \frac{1}{1 - z} \ln \frac{1}{1 - z}
  \end{align*}
  Reconocemos la función generatriz~\eqref{eq:H(z)}
  de los números harmónicos,%
    \index{numeros harmonicos@números harmónicos}
  y el número promedio de asignaciones a \lstinline!m!
  buscando el máximo entre \(n\) elementos resulta ser:%
    \index{propiedad!promedio}
  \begin{align}
    \E_n[\chi]
      &= [z^n] M_u(z, 1) \notag \\
      &= H_n  \label{eq:En-max} \\
      &= \ln n + \gamma + O(1 / n) \label{eq:En-max-asy}
  \end{align}
  Lo último de la expansión asintótica~\eqref{eq:Hn-asy-Bn}.

  Para usar~\eqref{eq:PGF-variance} calculamos:
  \begin{equation}
    \label{eq:Muu}
    M_{u u}(z, 1)
      = \frac{\ln^2 (1 - z)}{1 - z}
  \end{equation}
  Esta es la función~\eqref{eq:ln:alpha-beta}
  que analizamos en la sección~\ref{sec:gf-logs}.
  Con la definición de números harmónicos generalizados~\eqref{eq:H(m)n}%
    \index{numeros harmonicos@números harmónicos!generalizados|textbfhy}
  resulta la elegante fórmula:
  \begin{equation}
    \label{eq:Muu-coef}
    [z^n] M_{u u}(z, 1)
      = H^2_n - H^{(2)}_n
  \end{equation}
  Tenemos lo necesario para calcular la varianza:%
    \index{propiedad!varianza}
  \begin{align}
    \var_n[\chi]
      &= [z^n] M_{u u}(z, 1) + [z^n] M_u(z, 1) - ([z^n] M_u(z, 1))^2 \notag \\
      &= H^2_n - H^{(2)}_n + H_n - H^2_n \notag \\
      &= H_n - H^{(2)}_n
	  \label{eq:varn-max}
  \end{align}

  Ilustramos el cálculo de los largos promedio
  de camino interno y externo en árboles binarios.%
    \index{abrol binario@árbol binario}%
    \index{arbol binario@árbol binario!largo de camino interno}%
    \index{arbol binario@árbol binario!largo de camino externo}%
  Expresamos la clase de árboles binarios
  como en~\eqref{eq:A-class}:
  \begin{equation}
    \label{eq:A-class-again}
    \mathcal{A}
      = \Box + \Circle \times \mathcal{A} \times \mathcal{A}
  \end{equation}
  Como al agregar una raíz la altura de cada nodo aumenta en uno
  (y la suma de las alturas aumenta en el número de nodos)
  podemos escribir para el largo de camino interno:
  \begin{align}
    I(z, u)
      &= 1 + z \sum_{\substack{\alpha_l \in \mathcal{A} \\
			       \alpha_r \in \mathcal{A}}}
		 z^{\lvert \alpha_l \rvert}
		   u^{\pi(\alpha_l) + \lvert \alpha_l \rvert}
		 z^{\lvert \alpha_r \rvert}
		   u^{\pi(\alpha_r) + \lvert \alpha_r \rvert}
	   \label{A-bgf-decomposed} \\
      &= 1 + z \left(
		 \sum_{\alpha \in \mathcal{A}}
		   (z u)^{\lvert \alpha \rvert} u^{\pi(\alpha)}
	       \right)^2 \notag \\
      &= 1 + z I^2(z u, u)
	   \label{eq:A-bgf-decomposed-result}
  \end{align}
  Derivando respecto de \(u\):
  \begin{align*}
    I_u(z, u)
      &= 2 z I(z u, u) (z I_z(z u, u) + I_u(z u, u)) \\
    I_u(z, 1)
      &= 2 z I(z, 1) (z I_z(z, 1) + I_u(z, 1)) \\
    I_u(z, 1)
      &= \frac{2 z^2 I(z, 1) I_z(z, 1)}{1 - 2 z I(z, 1)}
  \end{align*}
  Pero \(I(z, 1) = A(z)\),
  donde \(A(z)\) está dada por~\eqref{eq:A-explicit},
  con lo que \(I_z(z, 1) = A'(z)\),
  y resulta:
  \begin{align}
    I_u(z, 1)
      &= \frac{2 z^2 A(z) A'(z)}{1 - 2 z A(z)} \notag \\
      &= \frac{1 - 3 z - (1 - z) \sqrt{1 - 4 z}}{z - 4 z^2}
	   \label{eq:Ipi-bgf_u|u=1}
  \end{align}
  Esto es nuevamente~\eqref{eq:Ipi-explicit}.

  De forma parecida podemos tratar el largo de camino externo.%
    \index{arbol binario@árbol binario!largo de camino externo}
  Siguiendo la misma descomposición del árbol:
  \begin{align}
    E(z, u)
      &= \sum_{\alpha \in \mathcal{A}}
	   z^{\lvert \alpha \rvert} u^{\xi(\alpha)}
	   \label{E-bgf-decomposed} \\
      &= 1 + z \sum_{\substack{\alpha_l \in \mathcal{A} \\
			       \alpha_l \in \mathcal{A}}}
		 z^{\lvert \alpha_l \rvert + \lvert \alpha_r \rvert}
		 u^{\xi(\alpha_l) + \boxed{\alpha_l}
		     + \xi(\alpha_r) + \boxed{\alpha_r}} \notag \\
      &= 1 + z \left(
		 \sum_{\alpha \in \mathcal{A}}
		   z^{\lvert \alpha \rvert}
		   u^{\xi(\alpha) + \boxed{\alpha}}
	       \right)^2 \notag \\
      &= 1 + z \left(
		 \sum_{\alpha \in \mathcal{A}}
		   z^{\lvert \alpha \rvert}
		   u^{\xi(\alpha) + \lvert \alpha \rvert + 1}
	       \right)^2 \notag \\
      &= 1 + z u^2 \left(
		     \sum_{\alpha \in \mathcal{A}}
		       (z u)^{\lvert \alpha \rvert}
		       u^{\xi(\alpha)}
		   \right)^2 \notag \\
      &= 1 + z u^2 E^2(z u, u)
	   \label{eq:E-bgf-decomposed-result}
  \end{align}
  Como antes,
  derivando ambos lados respecto de \(u\):
  \begin{align*}
    E_u(z, u)
      &= 2 z u E^2(z u, u)
	  + 2 z u^2 E(z u, u) ( z E_z(z u, u) + E_u(z u, u)) \\
    E_u(z, 1)
      &= 2 z E^2(z, 1)
	  + 2 z E(z, 1) ( z E_z(z, 1) + E_u(z, 1))
  \end{align*}
  Despejando,
  como \(E(z, 1) = A(z)\):
  \begin{align}
    E_u(z, 1)
      &= \frac{2 z A^2(z) + 2 z^2 A(z) A'(z)}{1 - 2 z A(z)} \notag \\
      &= \frac{1 - \sqrt{1 - 4 z}}{1 - 4 z}
	 \label{eq:E-bgf_u|u=1-asy}
  \end{align}
  Nuevamente~\eqref{eq:E-explicit},
  que da el promedio asintótico~\eqref{eq:A-fail-asy}..

  Si construimos árboles binarios insertando claves en orden aleatorio,
  la distribución es diferente.
  Para el largo de camino interno tenemos:
  \begin{align}
    \widehat{I}(z, u)
      &= \sum_{\alpha \in \mathcal{P}}
	   \frac{z^{\lvert \alpha \rvert}}{\lvert \alpha \rvert !}
	   u^{\pi(\alpha)}
		 \label{eq:BST-IPL-mgf-def} \\
      &= 1 + \sum_{\substack{\alpha_l \in \mathcal{P} \\
			     \alpha_r \in \mathcal{P}}}
	       \binom{\lvert \alpha_l \rvert + \lvert \alpha_r \rvert}
		     {\lvert \alpha_l \rvert}
		 \frac{z^{\lvert \alpha_l \rvert + \lvert \alpha_r \rvert + 1}}
		      {(\lvert \alpha_l \rvert + \lvert \alpha_r \rvert + 1)!}
		 u^{\pi(\alpha_l) + \alpha_l + \pi(\alpha_r) + \alpha_r}
		 \label{eq:BST-IPL-mgf-decomposed}
  \end{align}
  Para simplificar,
  derivamos respecto de \(z\):
  \begin{align}
    \widehat{I}_z(z, u)
      &= \sum_{\substack{\alpha_l \in \mathcal{P} \\
			 \alpha_r \in \mathcal{P}}}
	   \binom{\lvert \alpha_l \rvert + \lvert \alpha_r \rvert}
		 {\lvert \alpha_l \rvert}
	     \frac{z^{\lvert \alpha_l \rvert + \lvert \alpha_r \rvert}}
		  {(\lvert \alpha_l \rvert + \lvert \alpha_r \rvert)!}
	     u^{\pi(\alpha_l) + \alpha_l + \pi(\alpha_r) + \alpha_r}
		 \notag \\
      &= \sum_{\substack{\alpha_l \in \mathcal{P} \\
			 \alpha_r \in \mathcal{P}}}
	   \frac{(z u)^{\lvert \alpha_l \rvert}}
		{\lvert \alpha_l \rvert !}
	   u^{\pi(\alpha_l)}
	     \cdot \frac{(z u)^{\lvert \alpha_r \rvert}}
			{\lvert \alpha_r \rvert !}
		   u^{\pi(\alpha_r)} \notag \\
      &= \widehat{I}^2(z u, u)
		 \label{eq:BST-IPL-mgf-z-decomposed}
  \end{align}
  Nos interesan las derivadas \(\widehat{I}_u(z, 1)\)
  y \(\widehat{I}_{u u}(z, 1)\),
  ya sabemos que \(\widehat{I}(z, 1) = (1 - z)^{-1}\).
  Tenemos:
  \begin{align}
    \widehat{I}_{z u}(z, u)
      &= 2 \widehat{I}(z u, u)
	   (z \widehat{I}_z(z u, u) + \widehat{I}_u(z u, u))
	     \label{eq:I_zu} \\
    \widehat{I}_{z u u}(z, u)
      &= 2 (z \widehat{I}_z(z u, u) + \widehat{I}_u(z u, u))^2 \notag \\
      &\qquad  + 2 \widehat{I}(z u, u)
		 (z^2 \widehat{I}_{z z}(z u, u)
		     + z \widehat{I}_{z u}(z u, u)
		     + z \widehat{I}_{u z}(z u, u)
		     + \widehat{I}_{u u}(z u, u))
	     \label{eq:I_zuu}
  \end{align}
  Por el teorema de Schwartz%
    \index{Schwartz, teorema de}
  (también conocido como teorema de Clairaut,%
    \index{Clairaut, teorema de}
   ver textos de cálculo,
   como Zakon~\cite{zakon09:_mathem_analy_ii}
   o Thomson, Bruckner y Bruckner~%
     \cite[teorema 12.5]{thomson08:_elemen_real_analy})
  si las derivadas son continuas
  se cumple que:
  \begin{equation*}
    f_{x y} (x, y)
      = f_{y x}(x, y)
  \end{equation*}
  En general,
  podemos permutar el orden de derivación a gusto
  si alguna de las derivadas de interés es continua.
  En nuestro caso,
  al ser \([z^n] \widehat{I}(z, u)\) un polinomio en~\(u\)
  y \(\widehat{I}(z, 1) = (1 - z)^{-1}\),
  que tiene infinitas derivadas continuas en \(z = 0\),
  las derivadas que nos interesan en \(z = 0\), \(u = 1\)
  siempre existen y son continuas.
  Reordenando derivadas en~\eqref{eq:I_zu}
  y~\eqref{eq:I_zuu},
  evaluado para \(u = 1\) resulta:
  \begin{align*}
    \widehat{I}_{u z}(z, 1)
      &= 2 \widehat{I}(z, 1)
	   (z \widehat{I}_z(z, 1) + \widehat{I}_u(z, 1)) \\
    \widehat{I}_{u u z}(z, 1)
      &= 2 (z \widehat{I}_z(z, 1) + \widehat{I}_u(z, 1))^2
	   + 2 \widehat{I}(z, 1)
	       (z^2 \widehat{I}_{z z}(z, 1)
		   + 2 z \widehat{I}_{u z}(z, 1)
		   + \widehat{I}_{u u}(z, 1))
  \end{align*}
  Despejando:
  \begin{equation}
    \label{eq:Iu-ode}
    \widehat{I}_{u z}(z, 1)
      = \frac{2 I_u(z, 1)}{1 - z} + \frac{2 z}{(1 - z)^3}
  \end{equation}
  Condiciones iniciales para las ecuaciones diferenciales~\eqref{eq:Iu-ode}
  y su similar de~\eqref{eq:I_zuu}
  da la condición \([z^0] \widehat{I}(z, u) = 1\),
  de donde \(\widehat{I}_u(0, 1) = 0\),
  con lo que:
  \begin{equation}
    \label{eq:Iu|u=1}
    \widehat{I}_u(z, 1)
      = 2 \frac{1}{(1 - z)^2} \ln \frac{1}{1 - z} - \frac{z}{(1 - z)^2}
  \end{equation}
  y también,
  reemplazando~\eqref{eq:Iu|u=1} en~\eqref{eq:I_zuu}
  y resolviendo la ecuación diferencial resultante,
  donde nuestra condición anterior ahora da \(\widehat{I}_{u u}(0, 1) = 0\):
  \begin{equation}
    \label{eq:Iuu|u=1}
    \widehat{I}_{u u}(z, 1)
      = \frac{1}{(1 - z)^3}
	  \left(
	    4 (1 + z) \ln^2 \frac{1}{1 - z}
	      - 4 (1 + z) \ln \frac{1}{1 - z}
	      + 2 z^2 + 4 z
	  \right)
  \end{equation}

%%% Local Variables:
%%% mode: latex
%%% TeX-master: "clases"
%%% End:


% grafos.tex
%
% Copyright (c) 2009-2014 Horst H. von Brand
% Derechos reservados. Vea COPYRIGHT para detalles

\chapter{Grafos}
\label{cha:grafos}

  Un grafo corresponde a una abstracción
  de la situación en la cual hay objetos (\emph{vértices}),
  algunos de los cuales están conectados entre sí
  (mediante \emph{arcos}).
  El interés es razonar solo con el hecho
  que existen o no conexiones entre los vértices.
  Esta área es una de las más antiguas entre lo que se conoce
  como matemáticas discretas,
  con un amplio rango de aplicaciones.
  Los grafos
  (y estructuras afines)
  sirven para abstraer y representar objetos del más variado tipo,
  desde redes de transporte
  hasta rangos de validez de valores al analizar el código de un programa,
  pasando por aplicaciones como asignación de horarios.

  Como modelan una variedad de situaciones,
  estudiamos algoritmos para efectuar operaciones comunes sobre grafos.
  Demostraremos que son correctos,
  y daremos un somero análisis de su rendimiento.

\section{Algunos ejemplos de grafos}
\label{sec:ejemplos-grafos}

  Aplicaciones de grafos son circuitos eléctricos,
  donde interesa cómo están conectados entre sí los componentes
  (ver un ejemplo en la figura~\ref{fig:circuito-electrico})
  \begin{figure}[htbp]
    \centering
    \pgfimage[width=0.6\textwidth]{images/LowPass3poleCauer}
    % http://commons.wikimedia.org/wiki/File:LowPass3poleCauer.png
    % Public domain
    \caption[Diagrama de circuito de un filtro de paso bajo
	     de tercer orden]
	    {Diagrama de circuito de un filtro de paso bajo
	     de tercer orden~\cite{wikimedia06:LowPass3poleCauer}}
    \label{fig:circuito-electrico}
  \end{figure}
  y representaciones de redes de transporte,
  como el esquema~\ref{fig:tube-1908}
  de la red de metro de Londres en 1908.
  \begin{figure}[htbp]
    \centering
    \pgfimage[width=0.9\textwidth]{images/Tube_map_1908-2}
    % http://upload.wikimedia.org/wikipedia/commons/9/90/Tube_map_1908-2.jpg
    % Public domain
    \caption[Esquema del metro de Londres (1908)]
	    {Esquema del metro de Londres (1908)~%
	       \cite{London_tube1908}}
    \label{fig:tube-1908}
  \end{figure}

  En computación los grafos son ubicuos
  porque son una forma cómoda
  de representar relaciones entre objetos,
  como programas o personas.
  Un arco puede representar que dos personas se llevan bien
  (o no),
  que un ramo debe tomarse antes de otro,
  o que una función llama a otra.
  Muchas estructuras de datos,
  particularmente las enlazadas,
  pueden representarse mediante grafos,
  y muchos problemas de optimización importantes
  se modelan mediante grafos.
  También sirven como notación gráfica
  de algunos modelos de computación.
  Una advertencia:
  A pesar de ser un área bastante antigua de las matemáticas,
  aún no hay consenso en la notación o la nomenclatura.
  Curiosamente,
  recién en 1936 Kőnig
  publicó el primer texto sobre teoría de grafos~%
    \cite{koenig36:_theor_graph},
  cuando el área data de la época de Euler (1707\nobreakdash--1783).%
    \index{Euler, Leonhard}
  En caso de duda,
  revise las definiciones dadas por el autor.
  Acá usaremos básicamente la notación y nomenclatura
  de uno de los textos estándar del área,
  el de Diestel~%
    \cite{diestel10:_graph_theor}.
  Un tratamiento más accesible para no especialistas es el de Ore~%
    \cite{ore90:_graph_uses}.

  Formalmente:
  \begin{definition}
    \index{grafo|textbfhy}
    Un \emph{grafo} \(G = (V, E)\) consta de:
    \begin{center}
      \begin{description}
	\item[\boldmath \(V\):\unboldmath]
	  Conjunto finito no vacío de \emph{vértices}.%
	    \index{grafo!vertice@vértice|textbfhy}
	\item[\boldmath \(E\):\unboldmath]
	  Conjunto de \emph{arcos},%
	    \index{grafo!arco|textbfhy}
	  pares de vértices pertenecientes a \(V\).
	  Un arco \(\{a, b\} \in E\)
	  consta de \(a, b \in V\).
      \end{description}
    \end{center}
  \end{definition}
  Para nuestros efectos no interesa
  el caso de conjuntos infinitos de vértices.
  Al número de vértices de un grafo se le llama su \emph{orden.}%
    \index{grafo!orden|textbfhy}
  Consideramos en esta definición que un arco
  conecta un par de vértices diferentes
  (sin importar el orden),
  y no pueden haber varios arcos uniendo el mismo par de vértices.
  A veces para abreviar se anota \(a b\) por el arco \(\{a, b\}\).
  En tal caso \(a b = b a\).

  Dos vértices contenidos en un arco se llaman \emph{adyacentes}%
    \index{grafo!vertices adyacentes@vértices adyacentes}
  o \emph{vecinos}.%
    \index{grafo!vertices vecinos@vértices vecinos}
  Al número de arcos en que participa un vértice se llama su \emph{grado},%
    \index{grafo!vertice@vértice!grado}
  Si todos los vértices del grafo tienen el mismo grado,
  el grafo se llama \emph{regular}.%
    \index{grafo!regular}
  que para el vértice \(v\) se anota \(\delta(v)\).%
    \index{\(\delta\) (grado de un vertice)@\(\delta\) (grado de un vértice)}
  Un arco que contiene al vértice \(v\) se dice \emph{incide} en él.
  Dos arcos que tienen un vértice en común
  también se llaman \emph{adyacentes}.%
    \index{grafo!arcos adyacentes}
  Si de un grafo \(G = (V, E)\)
  se eliminan arcos o vértices
  (con los arcos que los contienen)
  el resultado \(G' = (V', E')\)
  es un \emph{subgrafo} de \(G\).%
    \index{grafo!subgrafo}
  El \emph{vecindario} de \(v\) son los vértices adyacentes a él,%
    \index{grafo!vecindario}
  anotamos \(N_G(v) = \{v_1, v_2, \dotsc, v_k\}\)
  si \(\{v, v_i\} \in E\).
  Normalmente omitiremos el subíndice que identifica al grafo
  cuando es claro del contexto.

  Para evitar notación engorrosa,
  identificaremos el grafo \(G = (V, E)\)
  con su conjunto de vértices o arcos.
  Así,
  diremos simplemente \(v \in G\) para indicar \(v \in V\),
  \(u v \in G\) cuando \(\{u, v\} \in E\)
  o \(G \smallsetminus u v\)
  para el grafo \(G' = (V, E \smallsetminus \{u, v\})\).

  Se dice que el grafo \(G' = (V', E')\)
  es un \emph{subgrafo} del grafo \(G = (V, E)\)
  si \(V' \subseteq V\) y \(E' \subseteq E\).
  Decir que \(G'\) es un grafo
  hace que los vértices que aparecen en \(E'\)
  están en \(V'\),
  y hace también que \(V' \ne \varnothing\).

  Variantes de grafos son \emph{multigrafos},%
    \index{multigrafo|textbfhy}
  en los cuales se permiten varios arcos
  entre el mismo par de vértices,
  e incluso arcos que comienzan y terminan en el mismo vértice.
  Muchas de nuestras conclusiones se aplican a ellos también,
  pero no los trataremos explícitamente.

  Como los grafos son empleados en muchas áreas,
  los nombres suelen ajustarse al área bajo estudio,
  por ejemplo a veces se les llama \emph{redes}.%
    \index{red|see{grafo}}
  Los vértices pueden llamarse también \emph{nodos}%
    \index{grafo!nodo|see{grafo!vértice}}
  o \emph{puntos},%
    \index{grafo!punto|see{grafo!vértice}}
  y hay quienes hablan de \emph{aristas} en vez de arcos.%
    \index{grafo!arista|see{grafo!arco}}
  Nosotros excluimos la posibilidad \(V = \varnothing\),
  dado que resulta un contraejemplo trivial
  a muchos teoremas importantes.
  Al dejar fuera este caso muchos resultados
  se pueden expresar en forma más simple,
  sin embargo esta convención no es universal.
  En general,
  hay consenso en el significado de los términos,
  pero el tratamiento de casos excepcionales varía.

  Al dibujar grafos se representan los vértices mediante puntos
  y los arcos mediante líneas que los unen.
  Los vértices normalmente no se identifican.
  Notar que \(u\) conectado con \(v\) o \(v\) conectado con \(u\),
  para el caso significa lo mismo.
  No importa la ruta o el largo del arco,
  ni si accidentalmente cruza otros.
  Sólo el hecho que un par de vértices están conectados importa.

  \begin{example}
    Definición de un grafo.

    \(G\) dado por:
    \begin{align*}
      V &= \{a, b, c, d, z\} \\
      E &= \{\{a, d\}, \{b, z\}, \{c, d\}, \{d, z\}\}
    \end{align*}
    Gráficamente está dado por la figura~\ref{fig:grafo-a}.
    \begin{figure}[htbp]
      \centering
      \pgfimage{images/grafo-ejemplo}
      \caption{Un grafo}
      \label{fig:grafo-a}
    \end{figure}
  \end{example}

\section{Representación de grafos}
\label{sec:representacion}
\index{grafo!representacion@representación}

  Un dibujo es útil para seres humanos,
  pero bastante inútil para razonar con él
  o para el uso en computadoras.
  Veremos algunas opciones adicionales.

\subsection{Lista de adyacencia}
\label{sec:lista-adyacencia}
\index{grafo!representacion@representación!lista de adyacencia}

  La \emph{lista de adyacencia}
  es una tabla donde para cada vértice
  se listan los vértices adyacentes.
  Para el caso del grafo de la figura~\ref{fig:grafo-a}
  se tiene la lista de adyacencia en el cuadro~\ref{tab:la-grafo-a}.
  \begin{table}[htbp]
    \centering
    \begin{tabular}{|>{\(}r<{\)}|>{\(}l<{\)}|}
      \hline
      \multicolumn{1}{|c|}{\rule[-0.7ex]{0pt}{3ex}\textbf{V}} &
	\multicolumn{1}{c|}{\textbf{Ady}} \\
      \hline
	\rule[-0.7ex]{0pt}{3ex}%
      a & d \\
      b & z \\
      c & d \\
      d & a, c, z \\
      z & b, d \\
      \hline
    \end{tabular}
    \caption{Lista de adyacencia
	     para el grafo de la figura~\ref{fig:grafo-a}}
    \label{tab:la-grafo-a}
  \end{table}

\subsection{Matriz de adyacencia}
\label{sec:matriz-adyacencia}
\index{grafo!representacion@representación!matriz de adyacencia}

  Representar un grafo mediante una \emph{matriz de adyacencia}
  corresponde a definir una matriz cuyos índices son los vértices,
  y los elementos son 1 o 0 dependiendo de si los vértices del caso
  están conectados o no.
  El grafo de la figura~\ref{fig:grafo-a}
  queda representado en el cuadro~\ref{tab:ma-grafo-a}.
  \begin{table}[htbp]
    \centering
    \begin{tabular}{l|*{5}{c}}
	  & $a$ & $b$ & $c$ & $d$ & $z$ \\
      \hline
	\rule[-0.7ex]{0pt}{3ex}%
      $a$ &   0 &   0 &	  0 &	1 &   0 \\
      $b$ &   0 &   0 &	  0 &	0 &   1 \\
      $c$ &   0 &   0 &	  0 &	1 &   0 \\
      $d$ &   1 &   0 &	  1 &	0 &   1 \\
      $z$ &   0 &   1 &	  0 &	1 &   0
    \end{tabular}
    \caption{Matriz de adyacencia
	     del grafo de la figura~\ref{fig:grafo-a}}
    \label{tab:ma-grafo-a}
  \end{table}
  Es claro que esta matriz es simétrica
  (vale decir,
   \(a[i, j] = a[j, i]\))
  ya que \(i\) conectado a \(j\)
  es lo mismo que \(j\) conectado a \(i\).
  Los elementos en la diagonal son todos cero
  porque no hay arcos que conectan vértices consigo mismos.

  Los multigrafos permiten rizos%
    \index{multigrafo!rizo}
  (en inglés \emph{\foreignlanguage{english}{loops}})%
    \index{multigrafo!loop@\emph{\foreignlanguage{english}{loop}}|see{multigrafo!rizo}}
  que conectan vértices consigo mismos.
  En tal caso la diagonal no necesariamente es ceros.
  Si hay más de un arco entre un par de vértices,
  es natural considerar que la entrada de la matriz
  es el número de arcos entre los vértices.
  Igualmente,
  si consideramos que el arco tiene dirección
  (va de \(u\) a \(v\)),
  resulta una matriz no necesariamente simétrica
  (estamos representando \emph{grafos dirigidos},%
    \index{grafo!dirigido|see{digrafo}}%
    \index{digrafo}
   que se discuten en mayor detalle
   en el capítulo~\ref{cha:digrafos}).

\subsection{Representación enlazada}
\label{sec:grafo-punteros}
\index{grafo!representacion@representación!enlazada}

  Una opción es representar los vértices por nodos
  con punteros que lo conectan a sus vecinos.
  Como el número de vecinos no necesariamente es el mismo
  (o siquiera razonablemente acotado)
  es natural que cada nodo tenga una lista de punteros a los vecinos
  (terminan siendo las listas de adyacencia).

\subsection{Representación implícita}
\label{sec:grafo-implicita}
\index{grafo!representacion@representación!implicita@implícita}

  En muchas aplicaciones
  el grafo nunca existe como estructura de datos,
  se van generando
  (y descartando)
  los vértices vecinos conforme se requieren.
  Un ejemplo de esta situación se da
  cuando un programa juega al ajedrez:
  Los nodos son posiciones de las piezas,
  y dos nodos son adyacentes
  si son posiciones relacionadas mediante una movida.
  El grafo del caso es finito,
  pero tan grande que es totalmente impracticable generarlo completo
  (y aún menos almacenarlo).
  Se van generando los vértices conforme los requiera el programa.

\section{Isomorfismo entre grafos}
\label{sec:isomorfismo}
\index{grafo!isomorfismo}

  Intuitivamente,
  si dos grafos pueden dibujarse de la misma forma,
  los consideraremos iguales.
  Sin embargo,
  como los conjuntos de vértices
  (y en consecuencia,
   arcos)
  no serán los mismos,
  esta idea debe interpretarse de otra forma.
  \begin{definition}
    Si \(G_1 = (V_1, E_1)\) y \(G_2 = (V_2, E_2)\) son grafos
    se dice que son \emph{isomorfos}
    si existe una biyección \(\alpha \colon V_1 \rightarrow V_2\)
    tal que \(\{\alpha(u), \alpha(v)\} \in E_2\)
    exactamente cuando \(\{u, v\} \in E_1\).
    En tal caso se anota \(G_1 \cong G_2\).
  \end{definition}
  Nótese que esto es coherente con lo que indicamos antes,
  en que los vértices no se identifican.
  Cuando hablamos de un grafo en realidad estamos refiriéndonos
  a una clase de grafos isomorfos.

  Una regla simple que resulta de la definición
  es que el número de vértices
  y arcos es el mismo entre grafos isomorfos.
  Al buscar isomorfismos solo deben considerarse como candidatos
  vértices del mismo grado,
  y vértices adyacentes deberán mapear a vértices adyacentes.
  La figura~\ref{fig:isomorfismo}
  muestra un par de grafos isomorfos,
  indicando las correspondencias entre vértices.
  \begin{figure}[htbp]
    \setbox1=\hbox{\pgfimage{images/isomorfo-a}}
    \setbox2=\hbox{\pgfimage{images/isomorfo-b}}
    \centering
    \subfloat{
      \copy1
    }%
    \hspace{3em}%
    \subfloat{
      \raisebox{0.5\ht1-0.5\ht2}{\copy2}
    }
    \caption{Ejemplo de isomorfismo entre grafos}
    \label{fig:isomorfismo}
  \end{figure}
  Otro par de grafos isomorfos
  dan la figura~\ref{fig:isomorfismo-2}.
  \begin{figure}[htbp]
    \setbox1=\hbox{\pgfimage{images/Petersen1}}
    \setbox2=\hbox{\pgfimage{images/Petersen2}}
    \centering
    \subfloat{
      \copy1
    }%
    \hspace{2em}%
    \subfloat{
      \raisebox{0.5\ht1-0.5\ht2}{\copy2}
    }
    \caption[Dos formas de dibujar el grafo de Petersen]
	    {Dos formas de dibujar el grafo de Petersen~%
	      \cite{kempe86:_memoir_theor_mathem_form,
		    petersen98:_sur_tait}}
    \label{fig:isomorfismo-2}
    \index{Petersen, grafo de|textbfhy}
  \end{figure}
  De los ejemplos se nota que incluso para grafos más bien chicos
  no es posible determinar a simple vista si son isomorfos.
  Resulta que el problema de determinar si dos grafos son isomorfos
  es \NP\nobreakdash-completo,%
    \index{NP-completo, problema@\NP-completo, problema}
  una categoría de problemas difíciles introducida por Cook~%
    \cite{cook71:_compl_theor_provin_proced}
  para los cuales no se conocen algoritmos
  que tomen un tiempo razonable.
  Para la definición precisa véanse textos de algoritmos
  y teoría de autómatas,
  como~%
    \cite{aho74:_design_anal_comp_algor,
	  hopcroft06:_intro_autom_theor_languag_comput,
	  parberry94:_probl_algor},
  y a Garey y Johnson~%
    \cite{garey79:_comput_intrac}
  para un tratamiento detallado.
  El problema de isomorfismo de grafos
  fue uno de los primeros problemas clásicos
  demostrado \NP\nobreakdash-completo.

\section{Algunas familias de grafos especiales}
\label{sec:grafos-especiales}

  Algunos grafos se repiten en aplicaciones,
  o son útiles para ejemplos y casos de estudio.
  Se les dan nombres y notación especiales.
  \begin{itemize}
  \item
    \index{grafo!camino|textbfhy}
    \(P_n\): Camino simple de \(n\) vértices,
    donde \(n \ge 2\),
    ver la figura~\ref{fig:Ps}.
    Tiene \(n - 1\) arcos,
    los vértices son de grados \(1\) y \(2\).
    \begin{figure}[htbp]
      \centering
      \setbox2=\hbox{\pgfimage{images/P2}}
      \setbox3=\hbox{\pgfimage{images/P3}}
      \setbox4=\hbox{\pgfimage{images/P4}}
      \setbox5=\hbox{\pgfimage{images/P5}}
      \centering
      \subfloat[\(P_2\)]{
	\raisebox{0.5\ht5-0.5\ht2}{\copy2}
      }%
      \hspace{2.5em}%
      \subfloat[\(P_3\)]{
	\raisebox{0.5\ht5-0.5\ht3}{\copy3}
      }%
      \hspace{2.5em}%
      \subfloat[\(P_4\)]{
	\raisebox{0.5\ht5-0.5\ht4}{\copy4}
      }%
      \hspace{2.5em}%
      \subfloat[\(P_5\)]{
	\copy5
      }
      \caption{Algunos grafos $P_n$}
      \label{fig:Ps}
    \end{figure}
  \item
    \index{grafo!ciclo|textbfhy}
    \(C_n\): Ciclo de \(n\) vértices,
    donde el vértice \(i\) está conectado
    con los vértices \(i - 1\) e \(i + 1\)
    módulo \(n\).
    Para que sea realmente un ciclo,
    es \(n \ge 3\).
    Tiene \(n\) arcos,
    es regular de grado \(2\).
    Ver figura~\ref{fig:Cs}.
    \begin{figure}[htbp]
      \setbox3=\hbox{\pgfimage{images/C3}}
      \setbox4=\hbox{\pgfimage{images/C4}}
      \setbox5=\hbox{\pgfimage{images/C5}}
      \setbox6=\hbox{\pgfimage{images/C6}}
      \centering
      \subfloat[\(C_3\)]{
	\raisebox{0.5\ht6-0.5\ht3}{\copy3}
      }%
      \hspace{2.5em}%
      \subfloat[\(C_4\)]{
	\raisebox{0.5\ht6-0.5\ht4}{\copy4}
      }%
      \hspace{2.5em}%
      \subfloat[\(C_5\)]{
	\raisebox{0.5\ht6-0.5\ht5}{\copy5}
      }%
      \hspace{2.5em}%
      \subfloat[\(C_6\)]{
	\copy6
      }
      \caption{Algunos grafos $C_n$}
      \label{fig:Cs}
    \end{figure}
  \item
    \index{grafo!completo|textbfhy}
    \(K_n\): Grafo completo de \(n\) vértices
    cada uno conectado con todos los demás,
    con \(n \ge 1\).
    Tiene \(n (n - 1) / 2\) arcos,
    es regular de grado \(n - 1\).
    La figura~\ref{fig:Ks} muestra algunos ejemplos.
    \begin{figure}[htbp]
      \setbox3=\hbox{\pgfimage{images/K3}}
      \setbox4=\hbox{\pgfimage{images/K4}}
      \setbox5=\hbox{\pgfimage{images/K5}}
      \setbox6=\hbox{\pgfimage{images/K6}}
      \centering
      \subfloat[\(K_3\)]{
	\raisebox{0.5\ht6-0.5\ht3}{\copy3}
      }%
      \hspace{2.5em}%
      \subfloat[\(K_4\)]{
	\raisebox{0.5\ht6-0.5\ht4}{\copy4}
      }%
      \hspace{2.5em}%
      \subfloat[\(K_5\)]{
	\raisebox{0.5\ht6-0.5\ht5}{\copy5}
      }%
      \hspace{2.5em}%
      \subfloat[\(K_6\)]{
	\copy6
      }
      \caption{Algunos grafos $K_n$}
      \label{fig:Ks}
    \end{figure}
  \item
    \index{grafo!rueda|textbfhy}
    \(W_n\): Rueda
    (en inglés \emph{\foreignlanguage{english}{wheel}})
    de \(n\) vértices,
    que consiste en un grafo \(C_{n - 1}\)
    más un ``centro'' conectado a cada vértice del ciclo.
    Nótese que algunos autores llaman \(W_n\) a \(W_{n + 1}\)
    (solo cuentan los vértices de afuera).
    Tiene \(2 (n - 1)\) arcos,
    \(n - 1\) vértices de grado \(3\)
    y uno de grado \(n - 1\).
    Algunas ruedas muestra la figura~\ref{fig:Wn}.
    \begin{figure}[htbp]
      \setbox4=\hbox{\pgfimage{images/W4}}
      \setbox5=\hbox{\pgfimage{images/W5}}
      \setbox6=\hbox{\pgfimage{images/W6}}
      \setbox7=\hbox{\pgfimage{images/W7}}
      \centering
      \subfloat[\(W_4\)]{
	\raisebox{0.5\ht7-0.5\ht4}{\copy4}
      }%
      \hspace{2.5em}%
      \subfloat[\(W_5\)]{
	\raisebox{0.5\ht7-0.5\ht5}{\copy5}
      }%
      \hspace{2.5em}%
      \subfloat[\(W_6\)]{
	\raisebox{0.5\ht7-0.5\ht6}{\copy6}
      }%
      \hspace{2.5em}%
      \subfloat[\(W_7\)]{
	\copy7
      }
      \caption{Algunos grafos $W_n$}
      \label{fig:Wn}
    \end{figure}
  \item
    \index{grafo!cubo|textbfhy}
    \(Q_n\): Cubo de orden \(n\).
    Donde:
    \begin{center}
      \begin{description}
      \item[Vértices:]
	  Secuencias de \(n\) símbolos \(\{0, 1\}\).
	\item[Arcos:]
	  Conectan a todos los pares de vértices
	  que difieren en un símbolo.
      \end{description}
    \end{center}
    Nótese que el número de vértices
    es \(2^n\).
    Tiene \(n \, 2^{n - 1}\) arcos,
    es regular de grado \(n\).
    La figura~\ref{fig:Qn} muestra algunos grafos \(Q_n\).
    \begin{figure}[htbp]
      \setbox2=\hbox{\pgfimage{images/Q2}}
      \setbox3=\hbox{\pgfimage{images/Q3}}
      \centering
      \subfloat[\(Q_2\)]{
	\raisebox{0.5\ht3-0.5\ht2}{\copy2}
      }%
      \hspace{5em}%
      \subfloat[\(Q_3\)]{
	\copy3
      }
      \caption{Algunos cubos}
      \label{fig:Qn}
    \end{figure}
  \end{itemize}
  Resulta que \(C_3 \cong K_3\),
  \(Q_2 \cong C_4\)
  y  \(W_4 \cong K_4\).

\section{Algunos resultados simples}
\label{sec:resultados-simples}

  Algunos teoremas simples de demostrar son sorprendentemente útiles.
  \begin{theorem}
    \label{theo:sum-degree=2edges}
    Sea \(G = (V, E)\) un grafo,
    entonces:
    \begin{equation*}
      \sum_{v \in V} \delta(v) = 2 \cdot \lvert E \rvert
    \end{equation*}
  \end{theorem}
  \begin{proof}
    Consideremos \(S \subseteq V \times E\)
    tal que \((v, e) \in S\) siempre que \(v \in e\).
    Contando los elementos de \(S\) ``por filas'' y ``por columnas''
    (ver la discusión respectiva
     en el capítulo~\ref{cha:combinatoria-elemental})
    tenemos:
    \begin{description}
    \item[Por filas:]
      Cada vértice aparece una vez
      por cada arco en el cual participa:
      \begin{equation*}
	\lvert S \rvert = \sum_{v \in V} \delta(v)
      \end{equation*}
    \item[Por columnas:]
      Cada vértice aparece dos veces
      (una por cada extremo del arco):
      \begin{equation*}
	\lvert S \rvert
	  = \sum_{e \in E} 2 = 2 \cdot \lvert E \rvert
      \end{equation*}
    \end{description}
    Estas dos expresiones deben ser iguales,
    lo que corresponde precisamente a lo que se quería demostrar.
  \end{proof}

  \begin{lemma}[Handshaking]
    \label{lem:handshaking}
    El número de vértices de grado impar en un grafo es par.
  \end{lemma}
  \begin{proof}
    Sean \(V_o\) los vértices de grado impar
    y \(V_e\) los vértices de grado par del grafo.
    Entonces:
    \begin{equation*}
      \sum_{v \in V_o} \delta(v) + \sum_{v \in V_e} \delta(v)
	= 2 \cdot \lvert E \rvert
    \end{equation*}
    El lado derecho de esta ecuación es par.
    El segundo término del lado izquierdo
    es una suma de números pares,
    por lo que es par.
    Con esto,
    el primer término debe ser par,
    pero es la suma de números impares.
    Esto significa que hay un número par de estos,
    que es exactamente lo que se quería demostrar.
  \end{proof}

\section{Secuencias gráficas}
\label{sec:secuencias-graficas}

  Nos interesa resolver problemas como el siguiente:
  \begin{example}
    ¿Es posible tener grafos con vértices
    de grados \(1, 2, 2, 3, 4\)?

    Es factible dibujar este grafo,
    ver figura~\ref{fig:12234}.
    \begin{figure}[htbp]
      \centering
      \pgfimage{images/grafo-12234}
      \caption{Un grafo con grados 1, 2, 2, 3 y 4}
      \label{fig:12234}
    \end{figure}
  \end{example}

  Diremos que una secuencia de enteros es \emph{gráfica}%
    \index{grafo!secuencia grafica@secuencia gráfica|textbfhy}
  si corresponde a los grados de los vértices de un grafo.
  Hay algunas condiciones simples,
  como que el número de vértices de grado impar debe ser par
  (lema~\ref{lem:handshaking})
  y que el grado máximo debe ser menor que el número de vértices.
  Para determinar si una secuencia es gráfica
  veremos primero cómo reorganizar los arcos
  sin afectar los grados de los vértices.
  Esto lo usaremos para modificar grafos
  de forma que responder la pregunta sea más fácil.
  \begin{definition}
    En un grafo \(G = (V, E)\)
    un \emph{2\nobreakdash-switch}
    respecto de los arcos \(u v, x y \in E\)
    (donde \(u x, v y \notin E\))
    reemplaza esos arcos por \(u x\) y \(v y\).
    Denotamos \(G \stackrel{2s}{\longrightarrow} H\)
    si se puede obtener \(H\) de \(G\)
    mediante una secuencia finita de 2\nobreakdash-switch.
  \end{definition}
  Para ilustración véase la figura~\ref{fig:2-switch}.
  \begin{figure}[htbp]
    \centering
    \subfloat{\pgfimage{images/example-2switch-a}}%
    \hspace{4em}%
    \subfloat{\pgfimage{images/example-2switch-b}}
    \caption{La operación 2-switch entre los arcos $u v$ y $x y$}
    \label{fig:2-switch}
  \end{figure}
  Nótese que si \(G \stackrel{2s}{\longrightarrow} H\)
  entonces también \(H \stackrel{2s}{\longrightarrow} G\),
  ya que podemos aplicar la secuencia en el orden inverso.
  En realidad,
  esta es una relación de equivalencia%
    \index{relacion@relación!equivalencia}
  (es reflexiva,
     ya que no hacer nada
     equivale a aplicar \(0\) \(2\)\nobreakdash-switch;
   es claro que es transitiva;
   y es simétrica
     porque podemos aplicar
     una secuencia de \(2\)\nobreakdash-switch
   al revés en orden inverso).
  Antes de demostrar el teorema de Berge,
  vemos que podemos ordenar los arcos
  de forma que conecten los vértices en orden de grado decreciente:
  \begin{lemma}
    \label{lem:Berge}
    Sea \(G\) un grafo de orden \(n\),
    con \(\delta_G(v_i) = d_i\)
    tal que \(d_1 \ge d_2 \ge \dotso \ge d_n\).
    Entonces hay un grafo \(G'\)
    tal que \(G \stackrel{2s}{\longrightarrow} G'\)
    con \(N_{G'} (v_1) = \{v_2, v_3, \dotsc, v_{d_1 + 1}\}\).
  \end{lemma}
  \begin{proof}
    Consideremos un grafo
    con los vértices ordenados según grado decreciente
    en que lo indicado no se cumple,
    o sea,
    hay un primer vértice \(v_i\) con \(2 \le i \le d_1 + 1\)
    tal que \(v_1 v_i \notin E\).
    Demostraremos que un \(2\)\nobreakdash-switch
    corrige esto para ese vértice,
    repitiendo el proceso logramos lo prometido.

    Como \(\delta_G(v_1) = d_1\),
    hay \(v_j\) con \(j \ge d_1 + 2\) tal que \(v_1 v_j \in E\).
    Debe ser \(d_i \ge d_j\),
    ya que \(i < j\).
    Como \(v_1 v_j \in E\),
    es \(d_j = \delta_G(v_j) \ge 1\).
    Así sabemos que \(v_j\)
    es adyacente a \(v_1\) y a \(d_j - 1\) otros vértices,
    mientras \(v_i\) es adyacente a \(d_i\) vértices
    que no incluyen a \(v_1\).
    Como por el orden de los vértices \(d_i \ge d_j > d_j - 1\),
    los conjuntos \(N(v_i)\) y \(N(v_j) \smallsetminus \{v_1\}\)
    no pueden coincidir
    (por ser de tamaños diferentes);
    o sea hay algún \(t \ne 1\)
    tal que \(v_i v_t \in E\) pero \(v_j v_t \notin E\).
    Aplicando un 2\nobreakdash-switch a \(v_1 v_j\) y \(v_i v_t\)
    (son arcos \(v_1 v_j\) y \(v_i v_t\)
     y no están unidos \(v_1\) con \(v_i\) ni \(v_j\) con \(v_t\),
     con lo que esta operación es válida)
    ninguno de los vértices involucrados cambia de grado
    y se intercambian \(v_i\) con \(v_j\) en el vecindario de \(v_1\).
    Aplicando repetidas veces
    esta operación obtendremos lo prometido.
  \end{proof}
  Ahora podemos demostrar:
  \begin{theorem}[Berge, 1973]
    \index{Berge, teorema de}
    \label{theo:Berge}
    Dos grafos \(G\) y \(H\)
    sobre el mismo conjunto de vértices \(V\)
    satisfacen \(\delta_G(v) = \delta_H(v)\) para todo \(v \in V\)
    si y solo si \(G \stackrel{2s}{\longrightarrow} H\).
  \end{theorem}
  \begin{proof}
    Demostramos implicancia en ambos sentidos.
    Ya vimos que al aplicar un \(2\)\nobreakdash-switch
    el grado de los vértices no cambia,
    con lo que si \(G \stackrel{2s}{\longrightarrow} H\)
    entonces los vértices tienen los mismos grados en ambos.

    Para demostrar necesidad,
    aplicamos inducción sobre el número de vértices en \(G\).%
      \index{demostracion@demostración!induccion@inducción}
    La base,
    \(\lvert V \rvert = 1\) es trivial.
    Para inducción,
    por el lema~\ref{lem:Berge},
    si elegimos el vértice \(v\) de grado máximo en \(G\) y \(H\),
    hay grafos \(G'\) y \(H'\) tales que
    \(G \stackrel{2s}{\longrightarrow} G'\)
    y \(H \stackrel{2s}{\longrightarrow} H'\)
    y tales que \(N_{G'}(v) = N_{H'}(v)\).
    Si eliminamos \(v\) de \(G'\) y \(H'\)
    obteniendo grafos \(G''\) y \(H''\),
    ambos tienen los mismos grados
    ya que estamos eliminando los mismos arcos de \(G'\) y \(H'\).
    Por la hipótesis de inducción,
    \(G'' \stackrel{2s}{\longrightarrow} H''\),
    y por tanto también \(G' \stackrel{2s}{\longrightarrow} H'\).
    Esto basta para demostrar lo aseverado,
    dado que es una relación de equivalencia.
  \end{proof}

  Con estas herramientas
  estamos en posición de atacar nuestro problema original.
  \begin{definition}
    Sea \(\left\langle d_1, d_2, d_3, \dotsc, d_n \right\rangle\)
    una secuencia descendente de números naturales,
    o sea, \(d_1 \ge d_2 \ge d_3 \ge \dotso \ge d_n\).
    Tal secuencia se dice \emph{gráfica}
    si hay un grafo \(G = (V, E)\)
    con \(V = \{v_1, v_2, \dotsc, v_n\}\)
    tal que \(d_i = \delta(v_i)\).
  \end{definition}
  Entonces\footnote{%
    Según Allenby y Slomson~\cite[página~159]{allenby11:_how_count},
    Havel~\cite{havel55:_poznamka}
    publicó este resultado en checo,
    Hakimi~\cite{hakimi62:_realiz_set_int_degrees_vertices_linear_graph}
    es independiente del resultado previo.}:
  \begin{theorem}[Havel-Hakimi]
    \index{Havel-Hakimi, teorema de}
    \label{theo:Havel}
    Una secuencia \(d_1 \ge d_2 \ge d_3 \ge \dotso \ge d_n\)
    (con  \(d_1 \ge 1\) y \(n \ge 2\))
    es gráfica si y solo si lo es la secuencia siguiente
    ordenada de mayor a menor:
    \begin{equation*}
      d_2 - 1, d_3 - 1, \dotsc, d_{d_1 + 1} - 1,
	d_{d_1 + 2}, d_{d_1 + 3}, \dotsc d_n
    \end{equation*}
  \end{theorem}
  \begin{proof}
    Demostramos implicancias en ambas direcciones.
    Para el recíproco,
    consideremos un grafo \(G\)
    de orden \(n - 1\) con vértices y grados:
    \begin{equation*}
      \delta_G(v_2) = d_2 - 1,
	\dotsc, \delta_G(v_{d_1 + 1}) = d_{d_1 + 1} - 1,
		   \delta_G(v_{d_1 + 2}) = d_{d_1 + 2},
	\dotsc, \delta_G(v_n) = d_n.
    \end{equation*}
    Agregue el vértice \(v_1\)
    con arcos \(v_1 v_i\) para \(2 \le i \le d_1 + 1\) para dar el grafo \(H\),
    que cumple \(\delta_H(v_1) = d_1\),
    y \(\delta_H(v_i) = d_i\) para todo \(2 \le i \le n\).

    Para el directo,
    suponga un grafo \(G\) tal que \(\delta_G(v_i) = d_i\)
    para \(1 \le i \le n\).
    Por el lema~\ref{lem:Berge}
    podemos suponer que \(N_G(v_1) = \{v_2, \dotsc, v_{d_1 + 1}\}\).
    Si eliminamos el vértice \(v_1\) de \(G\)
    obtenemos un grafo con la secuencia de grados indicada.
  \end{proof}

  Aplicando repetidas veces el teorema de Havel-Hakimi,
  teorema~\ref{theo:Havel},
  podemos determinar rápidamente si una secuencia es o no gráfica.
  Por ejemplo,
  considérese la secuencia
    \(\left\langle 4, 4, 4, 3, 2, 1 \right\rangle\).
  Tenemos:
  \begin{align*}
    \left\langle 4, 4, 4, 3, 2, 1 \right\rangle
      &\text{\ es gráfica si y solo si\ }
	 \left\langle 3, 3, 2, 1, 1 \right\rangle
	 \text{\ es gráfica} \\
      &\text{\ es gráfica si y solo si\ }
	 \left\langle 2, 1, 1, 0 \right\rangle
	 \text{\ es gráfica} \\
      &\text{\ es gráfica si y solo si\ }
	 \left\langle 0, 0, 0 \right\rangle
	 \text{\ es gráfica}
  \end{align*}
  Esta última corresponde al grafo de tres vértices y sin arcos,
  por lo que es gráfica.

  El teorema de Havel-Hakimi da una forma de construir un grafo
  con los grados prescritos:
  Se ordenan los grados de mayor a menor,
  luego el vértice \(v_1\)
  está conectado con \(v_2\) a \(v_{d_1} + 1\),
  el vértice \(v_2\)
  se conecta con los siguientes desde \(v_{d_1 + 2}\)
  hasta completar su grado,
  y así sucesivamente.
  Donde en el proceso se reordenan los grados
  deben reordenarse los vértices
  de la misma manera.
  Para un ejemplo de este proceso,
  tomemos:
  \begin{align*}
    \langle 8, 8, 6, 5, 4, 3, 3, 3, 1, 1 \rangle
      &\rightarrow \langle 7, 5, 4, 3, 2, 2, 2, 0, 1 \rangle
	 \hspace*{3.5em} \text{\(v_{10}\) pasa a \(v_9\)} \\
      &\phantom{{} \rightarrow {}}
		   \langle 7, 5, 4, 3, 2, 2, 2, 1, 0 \rangle \\
      &\rightarrow \langle \phantom{7,}
			      4, 3, 2, 1, 1, 1, 0, 0 \rangle \\
      &\rightarrow \langle \phantom{7, 4,}
				 2, 1, 0, 0, 1, 0, 0 \rangle
	 \hspace*{3.5em} \text{\(v_8\) pasa a \(v_6\)} \\
      &\phantom{{} \rightarrow {}}
		   \langle \phantom{7, 5,}
				 2, 1, 1, 0, 0, 0, 0 \rangle \\
      &\rightarrow \langle \phantom{7, 5, 4,}
				    0, 0, 0, 0, 0, 0 \rangle
  \end{align*}
  La última secuencia es gráfica
  (son seis vértices aislados).
  Los comentarios
  indican los cambios de posición que sufrieron los vértices:
  Se intercambiaron \(v_{9}\) con \(v_{10}\);
  luego \(v_6\) fue a \(v_7\),
  \(v_7\) a \(v_8\) y \(v_8\) pasó a \(v_6\).
  \begin{figure}[ht]
    \centering
    \pgfimage{images/Havel}
    \caption{Un grafo con vértices
	     de grados $\langle 8, 8, 6, 5, 4, 3, 3, 3, 1, 1 \rangle$}
    \label{fig:ex-Havel}
  \end{figure}
  La figura~\ref{fig:ex-Havel} muestra el grafo resultante.

  \begin{definition}
    \(G = (V,E)\) es un grafo.
    Entonces se define:
    \begin{description}
      \item[Camino:]
	Es una secuencia de vértices
	  \(\left\langle v_1, v_2, \dots, v_n \right\rangle\)
	tal que \(v_i, v_{i + 1}\) son adyacentes
	(\emph{\foreignlanguage{english}{walk}} en inglés).
      \item[Camino simple:]
	Es un camino en los que los \(v_i\) son todos distintos
	(en inglés, \emph{\foreignlanguage{english}{path}}).
      \item[Ciclo:]
	Es un camino
	  \(\left\langle v_1, v_2, \dots, v_n, v_1 \right\rangle\),
	en el cual no se repite más que el primer y último vértice.
	Se llama \emph{\(r\)\nobreakdash-ciclo}
	(ciclo de largo \(r\))
	si tiene \(r\) arcos y \(r\) vértices.
      \item[Circuito:]
	Un camino cerrado
	\(\left\langle v_1, v_2, \dots, v_n, v_1 \right\rangle\)
	(pueden repetirse vértices).
	Algunos les llaman ciclos,
	y llaman \emph{ciclos simples}
	a lo que nosotros llamamos ciclos.
    \end{description}
  \end{definition}

  \begin{definition}
    Sea \(G = (V,E)\) un grafo.
    Definimos la relación \(\sim\) entre vértices,
    tal que \(x \sim y\) si \(x\) e \(y\)
    están en un camino de \(G\),
    o sea \(x = v_1, v_2, \dots, v_k = y\) es un camino.
  \end{definition}

  Es fácil ver que \(\sim\) es una relación de equivalencia:
  \begin{description}
  \item[Reflexiva:]
    \(x \sim x\).
    Un camino de \(0\) arcos cumple con la definición.
  \item[Simétrica:]
    \(x \sim y \implies y \sim x\):
    Esto es
    \(x = v_1, \dots, v_k = y \implies y = v_k, \dots, v_1 = x\),
    que claramente es cierto.
  \item[Transitiva:]
    \((x \sim y) \wedge (y \sim z) \implies x \sim z\).
    Esto es decir:
    \begin{equation*}
      x = v_1, \dots, v_k = y = u_1, \dots, u_k = z
    \end{equation*}
    Esto es un camino que de \(x\) va a \(z\),
    \(x \sim z\).
  \end{description}

  \begin{definition}
    \index{grafo!componente conexo|textbfhy}
    \index{grafo!conexo|textbfhy}
    Sea \(G= (V, E)\) un grafo.
    Si \(V_1, V_2, \dotsc, V_k\)
    son las clases de equivalencia de \(\sim\),
    y \(E_1, E_2, \dots, E_k\)
    conjuntos de arcos tales que \(E_i\)
    contiene solo vértices de \(V_i\),
    a los grafos \(G_i = (V_i, E_i)\)
    se les llama \emph{componentes conexos} de \(G\).
    Si \(G\) tiene un único componente conexo
    es llamado \emph{conexo}.
  \end{definition}

  En el grafo de la figura~\ref{fig:2componentes}
  se distinguen vértices a los cuales
  no se puede acceder desde algunos de los otros vértices.
  Este grafo tiene dos componentes conexos.
  \begin{figure}[htbp]
    \centering
    \pgfimage{images/2componentes}
    \caption{Un grafo con dos componentes conexos}
    \label{fig:2componentes}
  \end{figure}

  Un resultado simple
  es la relación entre el número de vértices y arcos
  en grafos conexos.
  \begin{theorem}
    \label{theo:VEC}
    Todo grafo \(G = (V, E)\) tiene
    a lo menos \(\lvert V \rvert - \lvert E \rvert\)
    componentes conexos.
  \end{theorem}
  Nótese que para \(K_n\) esto nos dice
  que tiene al menos \(n (3 - n) / 2\)~componentes conexos,
  y para \(n > 3\) esto es negativo.
  La cota no es para nada ajustada.
  \begin{proof}
    Usamos inducción sobre el número de arcos.%
      \index{demostracion@demostración!induccion@inducción}
    \begin{description}
    \item[Base:]
      En un grafo con 0 arcos,
      cada vértice es un componente conexo,
      y hay
      \(\lvert V \rvert - 0 = \lvert V \rvert\) componentes conexos.
    \item[Inducción:]
      Suponemos que la hipótesis
      vale para todo grafo de \(n\) arcos,
      y demostramos que vale para todo grafo de \(n + 1\) arcos,
      con \(n \ge 0\).
      Considérese un grafo \(G = (V, E)\) con \(n + 1\) arcos.
      Eliminamos un arco arbitrario \(a b\) del grafo,
      dejando el grafo \(G'\) con \(n\) arcos.
      Por la hipótesis,
      \(G'\) tiene a lo menos \(\lvert V \rvert - n\)
      componentes conexos.
      Reponemos el arco eliminado,
      con lo que tenemos de vuelta el grafo original \(G\).
      Si \(a\) y \(b\) pertenecían al mismo componente conexo
      de \(G'\),
      \(G\) tiene el mismo número de componentes conexos de \(G'\),
      que es a lo menos \(\lvert V \rvert - n\) por hipótesis.
      Si \(a\) y \(b\) pertenecen
      a componentes conexos distintos de \(G'\),
      \(G\) tiene un componente conexo menos que \(G'\),
      ya que el arco \(a b\)
      une esos dos componentes conexos de \(G'\)
      en uno solo en \(G\).
      Como \(G'\)
      tenía a lo menos \(\lvert V \rvert - n\) componentes conexos,
      \(G\) tiene entonces a lo menos uno menos que esto,
      vale decir
	\(\lvert V \rvert - n - 1 = \lvert V \rvert - (n + 1)\).
      Esto demuestra el paso de inducción.
    \qedhere
    \end{description}
  \end{proof}

  Algunos puntos se deben notar de esta demostración.
  Primeramente,
  usamos inducción sobre el número de arcos.
  Esto es común en demostraciones en grafos,
  al igual que inducción sobre el número de vértices.
  Sólo si ninguna de estas dos estrategias sirve
  vale la pena considerar otras opciones.

  Por otro lado,%
    \index{grafo!eliminar y reponer}
  usamos la táctica de eliminar un arco
  y reponerlo en nuestra demostración.
  Esta es la forma más sencilla de evitar errores lógicos comunes,
  ya que asegura que el elemento que queremos agregar es posible
  y lleva en la dirección correcta.
  Si se usa inducción en grafos
  (ya sea sobre arcos o vértices),
  siempre conviene usar esta idea de encoger-expandir.

  \begin{corollary}
    \label{cor:grafo-conexo-vertices-arcos}
    Todo grafo conexo de \(n\) vértices
    tiene a lo menos \(n - 1\) arcos.
  \end{corollary}
  \begin{proof}
    Usamos la misma estrategia
    de la demostración del teorema~\ref{theo:VEC}:
    Partiendo con \(\lvert V \rvert\) vértices aislados
    (\(\lvert V \rvert\) componentes conexos),
    cada vez que agregamos un arco
    disminuye el número de componentes conexos
    en \(0\) o \(1\).
    Si siempre elegimos un arco
    que conecta componentes conexos distintos,
    al agregar \(n - 1\) arcos queda un único componente conexo.
  \end{proof}

  \begin{definition}
    Sea \(G = (V, E)\) un grafo.
    Entonces:
    \begin{itemize}
      \item
	Un camino simple que visita todos los vértices
	es un \emph{camino hamiltoniano}.%
	  \index{grafo!camino hamiltoniano|textbfhy}
	Un ciclo que contiene todos los vértices del grafo
	es llamado \emph{ciclo hamiltoniano}.%
	  \index{grafo!ciclo hamiltoniano|textbfhy}
      \item
	Un camino que pasa exactamente una vez por cada arco
	es denominado \emph{camino de Euler}.%
	  \index{grafo!camino de Euler|textbfhy}
	Un circuito que pasa exactamente una vez por cada arco
	se llama \emph{circuito de Euler}.%
	  \index{grafo!circuito de Euler|textbfhy}
    \end{itemize}
  \end{definition}

  Determinar si hay un camino o ciclo hamiltoniano
  es \NP\nobreakdash-completo.%
    \index{NP-completo, problema@\NP-completo, problema}
  Incluso tiene la distinción de ser uno de los 21~problemas
  identificados inicialmente como tales por Karp~%
    \cite{karp72:_reduc_among_combin_prob}.

  En cambio,
  un camino (o circuito) de Euler es sencillo de hallar.
  Si consideramos vértices cualquiera hay dos opciones:
  \begin{enumerate}
  \item
    Comienzo en un vértice,
    termino en otro.
  \item
    Comienzo en un vértice,
    termino en el mismo.
  \end{enumerate}

  Si inicio y fin son diferentes
  (es un camino de Euler):
  \begin{itemize}
  \item
    Inicio:
    \emph{Salgo} una vez,
    \emph{paso} por él (entro y salgo) varias veces,
    lo que significa que \(\delta(\mathit{inicio})\) es impar.
  \item
    Fin:
    \emph{Llego} una vez,
    \emph{paso} por él (entro y salgo) varias veces,
    con lo que también \(\delta(\mathit{fin})\) es impar.
  \item
    Otros vértices:
    \emph{Paso} por él (entro y salgo) varias veces,
    por lo que \(\delta(\mathit{otro})\) es par.
  \end{itemize}
  Si inicio y fin son el mismo
  (es un circuito de Euler):
  \begin{itemize}
  \item
    Inicio (y fin):
    \emph{Salgo} una vez,
    \emph{paso} por él (entro y salgo) varias veces,
    \emph{llego} una vez,
    y \(\delta(\mathit{inicio})\) es par.
  \item
    Otros vértices:
    \emph{Paso} por él (entro y salgo) varias veces,
    con lo que \(\delta(\mathit{otro})\) es siempre par.
  \end{itemize}
  Estas dos son las únicas posibilidades,
  y por tanto es condición necesaria
  para la existencia de un camino de Euler
  en un grafo conexo el que o todos los vértices sean de grado par
  (en tal caso podemos comenzar en cualquiera de ellos,
  terminamos en el mismo,
  es un \emph{circuito de Euler}),
  o que hayan exactamente dos vértices de grado impar
  (comenzamos en uno de ellos,
  terminamos en el otro,
  es un \emph{camino de Euler}).
  Más adelante demostraremos que estas condiciones son suficientes,
  y daremos algoritmos
  para encontrar un camino (o circuito) de Euler.

  \begin{example}
    \index{Koenigsberg, puentes de@Königsberg, puentes de}
    Puentes de \foreignlanguage{german}{Königsberg}

    Supóngase que se desea dar un paseo
    por la ciudad de \foreignlanguage{german}{Königsberg},
    pasando una única vez por cada uno de los siete puentes,
    situados según la figura~\ref{fig:puentes}.
    ¿Es posible realizar esta tarea?

    La respuesta es no,
    como demostró Euler en 1735,%
      \index{Euler, Leonhard}
    dando inicio al estudio de lo que hoy es la teoría de grafos.
    Representando los sectores unidos por los puentes en un grafo
    como en la misma figura~\ref{fig:puentes},
    se aprecia claramente
    que hay más de dos vértices de grado impar.
    Debido a esto no es posible que exista un camino de Euler
    que permita cumplir con la tarea requerida.
    Si bien la representación no es un grafo propiamente tal%
      \index{multigrafo}
    -- es un multigrafo
	pues hay vértices que están conectados por más de un arco --
    aún así los principios son aplicables.
    \begin{figure}[htbp]
      \centering
      \subfloat[Königsberg en tiempos de Euler]
	       {\pgfimage[width=0.3\textwidth]{images/Konigsberg}}%
      \hspace*{1.5em}%
      \subfloat[Königsberg y sus puentes]
	       {\pgfimage[width=0.3\textwidth]{images/Konigsberg_colour}}%
      \hspace*{2.5em}%
      \subfloat[Puentes de Königsberg como multigrafo]
	       {\pgfimage{images/puentesdeK}}
      \caption[Puentes de Königsberg]
	      {Puentes de Königsberg~%
		 \cite{oconnor00:_koenigsberg_bridges}}
      \label{fig:puentes}
      % http://www-history.mcs.st-andrews.ac.uk/Diagrams/Konigsberg.jpeg
      % http://www-history.mcs.st-andrews.ac.uk/Diagrams/Konigsberg_colour.jpeg
    \end{figure}
  \end{example}

  Del resultado siguiente Euler demostró la necesidad en 1736,
  Hierholzer~%
    \cite{hierholzer73:_ueber_moegl_linien}
  demostró suficiencia recién en~1873.
  Igual se atribuye a Euler.
  \begin{theorem}[Euler]
    \label{theo:Euler-circuito-camino}
    \index{Euler, teorema de}
    Sea \(G\) un grafo conexo.
    Entonces hay un camino de Euler si y solo si
    hay exactamente dos vértices de grado impar
    (y todo camino de Euler comienza en uno de ellos
     y termina en el otro),
    y hay un circuito de Euler
    si y solo si todos los vértices son de grado par.
  \end{theorem}
% Fixme: Paulina Silva <pasilva@alumnos.inf.utfsm.cl> dice que esta
%	 demostración es muy larga y confusa. Reorganizarla.
  \begin{proof}
    Demostramos implicancia en ambas direcciones.
    Que las condiciones son necesarias ya lo vimos antes,
    demostramos ahora que son suficientes por inducción fuerte
    sobre el número de arcos de \(G\).%
      \index{demostracion@demostración!induccion@inducción!fuerte}
    \begin{description}
    \item[Base:]
      Si \(G\) tiene un único arco,
      la conclusión es trivial.
    \item[Inducción:]
      Sea un grafo \(G\) con \(n + 1\) arcos,
      todos cuyos vértices son de grado par
      o hay exactamente dos vértices de grado impar.
      La estrategia general es eliminar un arco,
      y analizar por separado las situaciones
      en las cuales esta operación
      divide el grafo en dos componentes conexos
      y aquellas en que sigue siendo conexo.%
	\index{grafo!componente conexo}

      Consideremos primero el caso en que todos los vértices
      de \(G\) son de grado par.
      Elijamos un vértice \(x\) y un arco \(e = x y\).
      Si eliminamos el arco \(e\),
      obtenemos un nuevo grafo \(G'\),
      en el cual ahora \(x\) e \(y\)
      son los únicos vértices de grado impar.
      Entonces \(G'\) es conexo,
      ya que si no fuera conexo
      \(x\) e \(y\) en \(G'\)
      pertenecerían a componentes conexos diferentes,
      y en \(G'\) los vértices \(x\) e \(y\)
      serían los únicos de grado impar
      en sus respectivos componentes conexos.
      Esto es absurdo,
      contradice al lema~\ref{lem:handshaking}.
      Por inducción,
      como \(G'\) es conexo y tiene \(n\) arcos,
      hay un camino de Euler
      que comienza en \(x\) y termina en \(y\);
      al reponer el arco \(x y\) hay entonces un circuito de Euler
      (el camino anterior junto con este arco).

      Supongamos ahora que \(G\)
      tiene exactamente dos vértices de grado impar,
      llamémosles \(x\) e \(y\).
      Consideremos primero el caso
      en que \(x\) e \(y\) son adyacentes.
      Eliminando el arco \(x y\)
      tenemos un grafo \(G'\) con \(n\) arcos,
      y todos sus vértices son de grado par.
      Si \(G'\) es conexo,
      tiene un circuito de Euler,
      y agregando el arco \(x y\) a este tenemos un camino de Euler
      que comienza en \(x\) y termina en \(y\).
      Si \(G'\) no es conexo,
      tiene dos componentes conexos,
      llamémosles \(G_1\) y \(G_2\).
      Pero tanto \(G_1\) como \(G_2\)
      tienen solo vértices de grado par,
      y tienen menos de \(n\) arcos,
      con lo que cada uno de ellos tiene un circuito de Euler,
      que podemos suponer comienza y termina en \(x\)
      (respectivamente \(y\)).
      Conectando estos dos circuitos mediante el arco \(x y\)
      obtenemos un camino de Euler para \(G\),
      que comienza en \(x\) y termina en \(y\).
      Si no hay un arco que conecte a \(x\) e \(y\),
      debe haber un arco \(x z\) para algún vértice \(z\).
      Eliminando este arco,
      tenemos un grafo \(G'\) con \(n\) arcos
      en el cual hay exactamente dos vértices de grado impar,
      \(z\) e \(y\).
      Por inducción,
      si \(G'\) es conexo
      hay un camino de Euler
      que comienza en \(z\) y termina en \(y\),
      reponiendo el arco \(x z\)
      tenemos un camino de Euler
      que comienza en \(x\) y termina en \(y\).
      Si \(G'\) no es conexo,
      tendrá componentes conexos \(G_1\)
      (que contiene a \(x\))
      y \(G_2\).
      Entonces \(z\) estará en \(G_2\)
      (en caso contrario, \(G'\) sería conexo),
      e \(y\) estará en \(G_2\) también
      (de otra forma,
       sería el único vértice de grado impar en \(G_1\)).
      O sea,
      \(G_1\) tiene solo vértices de grado par,
      y \(G_2\) tiene exactamente dos vértices de grado impar
      (\(y\) y \(z\)).
      Por inducción,
      hay un circuito de Euler en \(G_1\),
      que podemos suponer comienza y termina en \(x\),
      y un camino de Euler en \(G_2\),
      que comienza en \(z\) y termina en \(y\).
      Reponiendo el arco \(x z\)
      tenemos un camino de Euler que comienza en \(x\),
      recorre \(G_1\) para volver a \(x\),
      luego pasa a \(G_2\) por \(x z\)
      y sigue el camino de Euler en \(G_2\) para terminar en \(y\).
    \qedhere
    \end{description}
  \end{proof}

  La demostración del teorema~\ref{theo:Euler-circuito-camino}
  no da muchas luces sobre cómo hallar el camino
  (circuito) de Euler.
  Curiosamente,
  Euler mismo nunca dio un método para hallar tal camino o circuito.
  Una técnica elegante da el algoritmo de Fleury~%
    \cite{fleury83:_deux_probl_geomet_situac}:%
    \index{Fleury, algoritmo de}
  Si hay vértices de grado impar,
  comience en uno de ellos,
  en caso contrario elija uno cualquiera.
  En cada paso,
  elija un arco desde el vértice actual y atraviéselo,
  luego lo elimina del grafo.
  Al hacer esto,
  debe tener cuidado que el grafo resultante sea conexo
  (salvo que no tenga alternativa).

  Una técnica más eficiente para hallar un circuito de Euler
  se debe a Hierholzer~%
    \cite{hierholzer73:_ueber_moegl_linien}:%
    \index{Hierholzer, algoritmo de}
  Parta de un vértice \(v\) cualquiera
  y siga un camino sin repetir arcos a través del grafo
  hasta volver a \(v\).
  Es imposible quedar sin posibilidades de continuar,
  ya que los vértices son de grado par,
  de llegar a un vértice
  tiene que haber al menos un arco que permita salir;
  y como hay un número finito de vértices
  tarde o temprano retornaremos al inicio.
  Si esto no visita todos los arcos,
  elija algún vértice \(v'\) en el circuito construido
  que tenga arcos no visitados,
  y comience el proceso nuevamente desde \(v'\),
  integrando luego el nuevo circuito en el anterior.

% Fixme: Correctitud de los algoritmos

  \begin{example}
    \begin{figure}[htbp]
     \centering
      \pgfimage{images/casita}
      \caption{Dibuja una casita}
      \label{fig:casita}
   \end{figure}
    Se pide dibujar la figura~\ref{fig:casita} en un papel,
    de manera que el lápiz no se levante en ningún momento del papel
    y no dibuje dos veces el mismo trazo.
    ¿Es posible realizar esto?

    La respuesta es sí,
    puesto que hay exactamente dos vértices de grado impar.
    Debemos elegir uno de ellos como punto de partida,
    y terminaremos en el otro.
  \end{example}

  \begin{example}
    \label{ex:queso}
    Un cubo de queso cortado en \(3 \times 3\).

    Un ratón comienza en una de las esquinas,
    come ese cubito y sigue con uno de los vecinos (no en diagonal).
    ¿Puede comerse todo el queso terminando con el cubo del centro?
    Los arcos en la figura~\ref{fig:cubo-queso}
    muestran las movidas legales.
    \begin{figure}[htbp]
      \centering
      \pgfimage{images/cubo-queso}
      \caption{Queso cortado en nueve cubitos}
      \label{fig:cubo-queso}
    \end{figure}

    La respuesta a esto es no.
    Considere el grafo de la figura~\ref{fig:cubo-queso-bipartito},
    \begin{figure}[htbp]
      \centering
      \pgfimage{images/cubo-queso-bipartito}
      \caption{Cubitos de queso de colores}
      \label{fig:cubo-queso-bipartito}
    \end{figure}
    en el cual se han coloreado de rojo y azul vértices adyacentes.
    Se ve que hay 14~vértices azules y 13~rojos.
    En cualquier camino que nuestro roedor siga
    irá alternando colores,
    por lo que si comienza en una de las esquinas,
    que son azules,
    necesariamente terminará en un cubito azul
    de comerse todo el queso.
    Pero el cubito central es rojo.
  \end{example}

\section{Árboles}
\label{sec:arboles}
\index{grafo!arbol@árbol|textbfhy}
\index{arbol@árbol|seealso{grafo!árbol}}

  En muchas aplicaciones aparecen grafos conexos
  sin enlaces redundantes
  (sin ciclos).
  Esta idea es capturada por la definición siguiente.
  \begin{definition}
    El grafo \(T = (V, E)\) es un \emph{árbol} si:
    \begin{enumerate}[label=\textbf{T\arabic{*}:}, ref=T\arabic{*}]
    \item
      \label{T:conexo}
      \(T\) es conexo.
    \item
      \label{T:aciclico}
      No hay ciclos en \(T\).
    \end{enumerate}
  \end{definition}
  Aclaramos que los árboles binarios%
    \index{arbol binario@árbol binario}
  vistos en el ramo Estructuras de Datos,
  \emph{no son árboles}.
  Acá no hay raíz,
  hijos ni descendientes,
  y aún menos ``hijos izquierdos'' y ``derechos'',
  solo \emph{vecinos}.
  Y como estos son grafos,
  no hay árboles sin nodos.

  \begin{definition}
    \index{grafo!arbol@árbol!hoja}
    \index{grafo!arbol@árbol!vertice interno@vértice interno}
    En un árbol \(T = (V, E)\)
    un vértice \(v \in V\) se llama \emph{hoja}
    si tiene grado uno.
    En caso contrario es un \emph{vértice interno}.
  \end{definition}

  Buena parte de la importancia de los árboles
  reside en que tienen una colección de propiedades interesantes,
  como las siguientes.
  \begin{theorem}
    \index{grafo!arbol@árbol!propiedades}
    \label{theo:arbol-propiedades}
    Si \(T = (V, E)\) es un árbol entonces:
    \begin{enumerate}[label=\textbf{T\arabic{*}:},
		      ref=T\arabic{*},
		      resume]
      \item
	\label{T:unico-camino}
	Para cualquier par de vértices en \(V\)
	hay un único camino simple entre ellos.
      \item
	\label{T:nuevo-arco}
	Al agregar un arco a \(T\) se forma un ciclo.
      \item
	\label{T:quitar-arco}
	Al eliminar un arco de \(T\),
	quedan dos componentes conexos que son árboles.
      \item
	\label{T:hojas}
	Un árbol con al menos dos vértices tiene al menos dos hojas.
      \item
	\label{T:vertices-arcos}
	\(\lvert E \rvert = \lvert V \rvert - 1\).
    \end{enumerate}
  \end{theorem}
  \begin{proof}
    Demostramos cada una de las aseveraciones por turno.
    \begin{enumerate}[label=\textbf{T\arabic{*}:},
		      ref=T\arabic{*},
		      start=3]
    \item
      Supongamos que \(T = (V, E)\) es un árbol
      y que hay dos vértices \(x, y\)
      con más de un camino simple que los conecta,
      digamos:
      \begin{align*}
	x &= u_1, u_2, \dots, u_r =  y \\
	x &= v_1, v_2, \dots, v_s =  y
      \end{align*}
      Sea ahora \(i\) el \emph{menor} índice
      tal que \(v_{i + 1} \ne u_{i + 1}\);
      y sea \(j\) el \emph{mayor} índice
      tal que \(u_{j - 1} \ne v_{k - 1}\),
      pero \(u_j = v_k\) para algún \(k\).
      Vea la figura~\ref{fig:teo-T3}.
      \begin{figure}[htbp]
	\centering
	\pgfimage{images/teo3}
	\caption{Esquema de vértices
		 en la parte~\ref{T:unico-camino}
		 el teorema~\ref{theo:arbol-propiedades}}
	\label{fig:teo-T3}
      \end{figure}
      Se nota que
      \(v_i, v_{i + 1}, \dots,
	 v_{k - 1}, v_k, u_{j - 1}, u_{j - 2}, u_{i + 1}, u_i\)
      son un ciclo,
      pero siendo \(T\) un árbol no tiene ciclos.
      Esta contradicción completa la demostración de esta parte.
    \item
      Al agregar un arco \(x y\) al árbol,
      este junto con el camino entre \(x\) e \(y\)
      (que existe porque \(T\) es conexo)
      forman un ciclo.
    \item
      Consideremos un arco \(x y\) del árbol.
      Si lo eliminamos,
      ya no hay caminos entre \(x\) e \(y\)
      (por~\ref{T:unico-camino}
       hay un único camino entre \(x\) e \(y\),
       precisamente este arco).
      Luego el grafo resultante tiene dos componentes conexos,
      cada uno conexo y sin ciclos.
      Ambos son árboles.
    \item
% Fixme: Paulina Silva <pasilva@alumnos.inf.utfsm.cl> dice que esto
% no se entiende bien
      Consideremos un camino de largo máximo en \(T\),
      con vértices \(v_1\), \(v_2\), \ldots, \(v_m\).
      Entonces \(m \ge 2\),
      dado que un árbol con al menos dos vértices
      tiene que tener al menos un arco.
      No pueden haber arcos \(v_1 v_i\) para \(i \ge 2\),
      ya que de otra forma
      tendríamos un ciclo \(v_1, \dotsc, v_i, v_1\).
      Tampoco puede haber un arco \(u v_1\),
      ya que de haberlo
      tendríamos un camino más largo \(u, v_1, \dotsc, v_m\).
      O sea,
      \(v_1\) es una hoja.
      De forma similar,
      \(v_m\) es una hoja,
      y hay al menos dos hojas.

      Nótese que el caso extremo de dos hojas se da en un camino.
    \item
      Queremos demostrar
      que \(\lvert E \rvert = \lvert V \rvert - 1\).
      Usamos inducción sobre el número de vértices.%
	\index{demostracion@demostración!induccion@inducción}
      En un árbol con un único vértice,
      la aseveración se cumple.
      Supongamos ahora que la aseveración se cumple
      para todos los árboles con \(n\) vértices,
      y consideremos un árbol con \(n + 1\) vértices.
      Elijamos una hoja \(x\)
      (por~\ref{T:hojas} hay al menos dos hojas),
      hay un único arco \(x y\) que incluye a \(x\).
      Al eliminar el vértice \(x\) de \(T\)
      junto con el arco \(x y\)
      queda un árbol de \(n\) vértices,
      que por inducción tiene \(n - 1\) arcos.
      Al reponer el vértice y el arco,
      el número de arcos y el de vértices aumenta en uno,
      y tenemos el resultado.
    \qedhere
    \end{enumerate}
  \end{proof}
  La parte~\ref{T:vertices-arcos}
  y el corolario~\ref{cor:grafo-conexo-vertices-arcos}
  dicen que un árbol es el grafo conexo con mínimo número de arcos%
    \index{grafo!conexo}
  para ese conjunto de vértices.
  Es común querer conectar los vértices de un grafo
  con el mínimo número de arcos:
  \begin{definition}
    \index{grafo!arbol recubridor@árbol recubridor|textbfhy}
    \index{grafo!spanning tree@\emph{\foreignlanguage{english}{spanning tree}}|see{grafo!árbol recubridor}}
    Sea \(G = (V, E)\) un grafo conexo.
    A un árbol \(T = (V, E')\),
    donde \(E' \subseteq E\)
    se le llama \emph{árbol recubridor}
    (en inglés, \emph{\foreignlanguage{english}{spanning tree}})
    de \(G\).
  \end{definition}

  \begin{example}
    Dibujar los árboles no isomorfos de \(6\) vértices.

    La mejor forma de solucionar esto
    es empezar a dibujar los grafos,
    partiendo por el caso
    en que se encuentre un vértice de grado máximo,
    es decir,
    de grado~\(5\).
    Véase la figura~\ref{subfig:a6-d5}.
    Luego los árboles con grado máximo~\(4\)
    (figura~\ref{subfig:a6-d4}),
    los de grado~\(3\)
    (figura~\ref{subfig:a6-d3}),
    y finalmente los de grado máximo~\(2\)
    (figura~\ref{subfig:a6-d2}).
    \begin{figure}[htbp]
      \centering
      \subfloat[Grado máximo \(5\)]{
	\pgfimage{images/a6-d5}
	\label{subfig:a6-d5}
      }%
      \hspace*{4em}%
      \subfloat[Grado máximo \(4\)]{
	\pgfimage{images/a6-d4}
	\label{subfig:a6-d4}
      }

      \subfloat[Grado máximo \(3\)]{
	\pgfimage{images/a6-d3a}%
	\hspace{2.5em}%
	\pgfimage{images/a6-d3b}%
	\hspace{2.5em}%
	\pgfimage{images/a6-d3c}
	\label{subfig:a6-d3}
      }

      \subfloat[Grado máximo \(2\)]{
	\pgfimage{images/a6-d2}
	\label{subfig:a6-d2}
      }
      \caption{Los \(6\) árboles con \(6\) vértices}
      \label{fig:arbol6}
    \end{figure}
    Estos son la solución a nuestro problema.
    Hay un total de \(6\) árboles no isomorfos de \(6\) vértices.
    Obtener el número de árboles para cualquier número de vértices
    es uno de los problemas abiertos famosos de la teoría de grafos.
  \end{example}

\section{Árboles con raíz}
\label{sec:arboles-raiz}
\index{grafo!arbol con raiz@árbol con raíz|textbfhy}

  Veremos algunas aplicaciones de árbol
  con un vértice especial designado como raíz.
  Esto aparece en aplicaciones en las cuales hay una jerarquía,
  como al representar un organigrama.
  Así \emph{no} son isomorfos los árboles con raíz
  (el vértice en blanco marca el distinguido como raíz)
  mostrados en la figura~\ref{fig:arbol-raiz},
  a pesar de ser isomorfos si los consideramos como árboles
  (no distinguimos raíces).
  \begin{figure}[htbp]
    \setbox1=\hbox{\pgfimage{images/arbol-raiz-1}}
    \setbox2=\hbox{\pgfimage{images/arbol-raiz-2}}
    \centering
    \subfloat{
      \raisebox{0.5\ht2-0.5\ht1}{\copy1}
    }%
    \hspace{3.5em}%
    \subfloat{
      \copy2
    }
    \caption{Ejemplos de árbol con raíz}
    \label{fig:arbol-raiz}
  \end{figure}
  Aparte de la raíz distinguimos \emph{vértices internos}
  con \(\delta(v) \ge 2\),
  y \emph{hojas} con \(\delta(v) = 1\).
  Normalmente dibujaremos la raíz y debajo de ella sus vecinos,
  y así sucesivamente hasta llegar a las hojas.

  En muchas aplicaciones encontraremos
  que la raíz y los vértices internos tienen el mismo grado.
  Si tienen grado \(m\)
  se habla de \emph{árboles \(m\)\nobreakdash-arios}.

  Podemos enumerar los vértices de un árbol con raíz,
  analizando su distancia desde la raíz,
  donde la distancia es el largo
  (número de arcos)
  del camino entre la raíz y el vértice considerado:
  \begin{description}
  \item[\boldmath Nivel \(0\):\unboldmath]
    La raíz.
  \item[\boldmath Nivel \(1\):\unboldmath]
    Los vecinos de la raíz.
  \item[\boldmath Nivel \(2\):\unboldmath]
    Los vecinos de vértices en el nivel \(1\),
    salvo los que están en el nivel \(0\).
  \item[\boldmath\(\dotsb\)\unboldmath]
  \item[\boldmath Nivel \(n\):\unboldmath]
    Los vecinos de los vértices en el nivel \(n - 1\),
    salvo los que están en nivel \(n - 2\).
  \end{description}
  Esto motiva la siguiente definición:
  \begin{definition}
    \index{grafo!arbol con raiz@árbol con raíz!altura|textbfhy}
    La \emph{altura} del árbol con raíz
    es el máximo \(k\) para el que el nivel \(k\) no es vacío.
  \end{definition}
  La interpretación como una jerarquía
  similar a una genealogía sugiere:
  \begin{definition}
    \index{grafo!arbol con raiz@árbol con raíz!ancestro}
    \index{grafo!arbol con raiz@árbol con raíz!descendiente}
    \index{grafo!arbol con raiz@árbol con raíz!padre}
    \index{grafo!arbol con raiz@árbol con raíz!hijo}
    Sea \(T\) un árbol con raíz \(r\).
    Si hay un camino de \(r\) a \(v\) que pasa por \(u\),
    se dice que \(u\) es \emph{ancestro} de \(v\),
    y \(v\) es un \emph{descendiente} de \(u\).
    Si \(u\) y \(v\) son vecinos,
    se dice que \(u\) es el \emph{padre} de \(v\),
    y que \(v\) es \emph{hijo} de \(u\).
  \end{definition}

  La enumeración en niveles que da lugar a la definición de altura
  sugiere el algoritmo~\ref{alg:recorrer-arbol}
  \begin{algorithm}[htbp]
    \DontPrintSemicolon
    \SetKwFunction{Recorrer}{recorrer}

    \KwProcedure \Recorrer{\(v\)} \;
    \BlankLine
    \eIf{\(v\) es hoja}{
      Visitar \(v\) \;
    }{
      Visitar \(v\) en preorden \;
      \For{\(x\) hijo de \(v\)}{
	\Recorrer{\(x\)} \;
      }
      Visitar \(v\) en postorden \;
    }
    \caption{Recorrer árboles con raíz}
    \label{alg:recorrer-arbol}
  \end{algorithm}
  para recorrer un árbol con raíz.
  Se invoca el procedimiento recorrer inicialmente con la raíz.
  En este algoritmo podemos considerar visitar
  (procesar de alguna forma)
  cada vértice la primera o la última vez que pasamos por él,
  dando lugar a recorridos en \emph{preorden}
  o en \emph{postorden},%
    \index{grafo!arbol con raiz@árbol con raíz!recorrer}
  alternativas que suelen presentarse por separado.
  Cual se elija
  (o incluso si se usan ambos)
  dependerá de la aplicación.
  Como no hay orden definido entre los hijos de un vértice,
  en caso de haber varios elegimos uno arbitrariamente.

  \begin{theorem}
    \label{theo:maximo-hojas}
    Si el número máximo de hijos
    de los vértices de un árbol con raíz es \(d\)
    y su altura es \(h\)
    entonces el árbol tiene a lo más \(d^h\) hojas.
  \end{theorem}
  \begin{proof}
    La demostración es por inducción fuerte sobre \(h\).%
      \index{demostracion@demostración!induccion@inducción!fuerte}
    \begin{description}
    \item[Base:]
      Cuando \(h = 0\) hay un único vértice
      (la raíz es hoja)
      y hay \(1 \le d^0 = 1\) hojas.
    \item[Inducción:]
      Supongamos que todos los árboles
      de altura menor o igual a \(h\)
      tienen a lo más \(d^h\) hojas.
      Consideremos un árbol de altura \(h + 1\).
      Este es la raíz
      y a lo más \(d\) árboles de altura a lo más \(h\),
      cada uno de los cuales aporta a lo más \(d^h\) hojas,
      para un total de	a lo más \(d \cdot d^h = d^{h + 1}\) hojas.
    \qedhere
    \end{description}
  \end{proof}

  \begin{corollary}
    \label{cor:arbol-hojas}
    Un árbol en el cual cada nodo tiene a lo más \(d\) hijos
    y que tiene \(r\) hojas tiene altura a lo menos de \(\log_d r\).
  \end{corollary}

  En particular,
  árboles binarios%
    \index{arbol binario@árbol binario}
  (que como ya se comentó realmente no son árboles,
   pero tienen suficiente en común con ellos
   para los efectos presentes)
  con \(r\) hojas tienen altura a lo menos \(\log_2 r\),
  y árboles binarios de altura \(h\) tienen a lo más \(2^h\) hojas.

\section{Árboles ordenados}
\label{sec:arbol-ordenado}%
\index{grafo!arbol ordenado@árbol ordenado}

  Una situación afín a los árboles con raíz
  se da cuando hay un orden entre los hijos de un vértice.
  Así,
  hay un primer,
  segundo,
  tercer,
  etc.~hijo.
  Muchas situaciones son naturales de modelar de esta forma.

\subsection{Árboles de decisión}
\label{sec:arbol-decision}
\index{arbol de decision@árbol de decisión}

  Un árbol de decisión representa una secuencia de decisiones
  y los resultados de estas.
  Se comienza en la raíz,
  cada vértice interno representa una decisión,
  y las hojas son resultados finales.
  Un camino entre la raíz y una hoja representa
  una ejecución del procedimiento,
  a través de la secuencia de decisiones y sus resultados
  que el camino representa.

  \begin{example}
    Búsqueda de monedas falsas.

    Tenemos una moneda \(O\) (la sabemos buena)
    y \(r\) otras monedas,
    una de las cuales puede ser falsa
    (puede que sea más pesada o más liviana que la moneda \(O\)).
    ¿Cuál es el número mínimo de pesadas
    (comparar el peso de dos colecciones de monedas)
    para determinar si hay una falsa y saber exactamente cuál es?

    Un nodo del árbol de decisiones puede representarse
    como en la figura~\ref{fig:pesar-monedas},
    \begin{figure}[htbp]
      \centering
      \pgfimage{images/decision-monedas}
      \caption{Vértice del árbol de decisión al pesar monedas}
      \label{fig:pesar-monedas}
    \end{figure}
    para cada pesada hay tres opciones:
    La izquierda es más liviana,
    son iguales,
    la derecha es más liviana.

    Las hojas que debemos obtener
    (resultados finales del proceso)
    son las siguientes:
    \begin{itemize}
      \item
	Todas buenas.
      \item
	\#1 Pesada.
      \item
	\#1 Liviana.
      \item
	(Muchas alternativas omitidas)
      \item
	\#\(r\) Pesada.
      \item
	\#\(r\) Liviana.
    \end{itemize}
    Hay \(2 r + 1\) resultados,
    con lo que se requieren
    a lo menos \(\lceil \log_3 (2 r + 1) \rceil\) pesadas.
    Pueden ser más que esto,
    solo hemos demostrado que es imposible hacerlo con menos.
  \end{example}

% Fixme: ¿Árboles AND/OR?

\subsection{Análisis de algoritmos de ordenamiento}
\label{sec:arbol_ordenamiento}
\index{ordenamiento!cotas|textbfhy}

  Supongamos un método de ordenamiento basado en comparaciones.
  Una pregunta obvia es:
  ¿Cuántas comparaciones se requieren para ordenar \(n\) elementos?

  El suponer que todos los elementos son diferentes
  hace más duro resolver el problema,
  con lo que nos concentramos en ese caso.
  El ordenar \(n\) elementos involucra determinar en qué orden están
  (o, lo que es lo mismo, han de ubicarse).
  Modelamos un algoritmo de ordenamiento
  especificando cuáles de los elementos originales
  se comparan en cada paso,
  y organizamos los distintos caminos que sigue el algoritmo
  como ramas de un árbol con raíz.
  Cada vértice representa
  el resultado de comparar dos elementos con las opciones \(<,>\).
  Pueden aparecer comparaciones redundantes
  o incluso contradictorias en el árbol.
  Las hojas son órdenes de los \(n\) elementos de entrada
  (aunque también es posible que aparezcan entre las hojas
   situaciones imposibles,
   al especificar el camino desde la raíz
   situaciones contradictorias).

  \begin{example}
    Comparamos tres elementos \(\{a, b, c\}\),
    todos distintos.

    Como se ve en la figura~\ref{fig:arbol_decision},
    tomando cada comparación \(x \colon y\) como un vértice
    cuyos resultados son \(x < y\) o \(x > y\),
    el árbol tiene \(6\)~hojas,
    que corresponden a las \(3!\)~formas de ordenar \(3\)~objetos.
    \begin{figure}[htbp]
      \centering
      \pgfimage{images/ordenamiento}
      \caption{Árbol de decisión al ordenar $3$ objetos}
      \label{fig:arbol_decision}
    \end{figure}
  \end{example}
  Esto muestra que la altura del árbol de decisión
  (el número de comparaciones requeridas en el peor caso)
  es \(\lceil \log_2 n! \rceil\).
  Aproximamos el factorial en la sección~\ref{sec:em-Stirling}
  como \(\ln n! = n \ln n - n + O(1)\).
  Por nuestro análisis
  el número de comparaciones requerido
  es entonces \(\Omega(n \log n)\).

\subsection{Generar código}
\label{sec:Sethi-Ullman}
\index{generar codigo@generar código}

% Fixme: ¿Agregar un programa (parser + generador)?
% Conversado con Paulina Silva <pasilva@alumnos.inf.utfsm.cl>,
% sugiere un programa completo (parser + construir árbol
% sintáctico + mostrar este + calcular números de Sethi
% + generar código) como ejemplo externo, tal vez mostrar en
% el texto solo los pasos discutidos acá.

% Fixme: Revisar en Aho-Ullman (citar)

  Una aplicación interesante de árboles ordenados
  se da al generar código
  para expresiones aritméticas
  mediante la técnica de Sethi y Ullman~%
    \cite{sethi70:_gener_optim_code_arith_expres}.
  Considerando un modelo de máquina
  que tiene cierto número de registros de propósito general,
  con operaciones aritméticas tradicionales
  que toman sus argumentos de dos registros cualquiera
  y dejan el resultado en alguno cualquiera.
  Lo que interesa
  es generar código óptimo para expresiones aritméticas
  en esta clase de máquinas.

   Por ejemplo,
   la expresión
     \lstinline[language=C]!a * x + b * y + c * (-(u + v))!
   queda representada
   por el árbol de la figura~\ref{fig:Sethi-Ullman}.
  \begin{figure}[htbp]
    \centering
    \pgfimage{images/Sethi}
    \caption{Árbol sintáctico de una expresión}
    \label{fig:Sethi-Ullman}
  \end{figure}%
    \index{Sethi-Ullman, algoritmo de}
  Los números que adornan los vértices del árbol
  representan el número de registros requeridos
  para calcular el valor de ese vértice
  (se les llama \emph{números de Sethi-Ullman},%
     \index{Sethi-Ullman, numeros de@Sethi-Ullman, números de}
   en honor a los inventores del algoritmo que discutiremos).

  Al calcular una expresión
  vista de esta forma como un árbol
  nada puede ganarse calculando parte de una rama,
  seguir con otra,
  para luego volver a completar la primera.
  El caso más simple es el de una variable o constante,
  basta cargar el valor en un registro libre.
  Consideremos ahora una operación cualquiera dentro del árbol,
  suponiendo operaciones con dos argumentos
  (las ideas se pueden extender sin problemas
   a casos en que hay más de dos parámetros).
  Toma los valores de sus argumentos de dos registros.
  Para calcular estas expresiones más complejas,
  supongamos que evaluar los operandos
  requiere \(m\) y \(n\) registros respectivamente.
  Se dan dos casos:
  \begin{description}
  \item[\boldmath Caso \(m = n\):\unboldmath]
    De ser así,
    la estrategia consiste en calcular el argumento izquierdo
    (se usan \(m\) registros en el proceso,
     queda el resultado en uno de ellos).
    Luego calculamos el argumento derecho,
    usando \(m\) registros más,
    lo que con el registro usado para almacenar temporalmente
    el valor del operando izquierdo
    hace un total de \(m + 1\) registros.
    Finalmente calculamos el valor buscado,
    sin usar registros adicionales.
  \item[\boldmath Caso \(m \ne n\):\unboldmath]
    Consideremos \(m < n\),
    el otro caso es totalmente simétrico.
    En este caso calculamos primero aquel argumento
    que requiere más registros,
    (el derecho en nuestro caso),
    usando \(n\) registros,
    y dejando el resultado en uno de ellos.
    Luego calculamos el otro argumento
    (el izquierdo en nuestro caso),
    para lo que se requieren \(m < n\) registros,
    reusando los que quedaron libres del cálculo anterior.
    En total,
    se requieren \(n\) registros.
  \end{description}
  El algoritmo para generar código óptimo en este caso particular
  (este tipo de arquitecturas
   y expresiones sin subexpresiones repetidas)
  resulta inmediato de la discusión anterior:
  \begin{enumerate}
  \item
    Calcule los números de Sethi-Ullman para los vértices del árbol
    en un recorrido en postorden.%
      \index{grafo!arbol@árbol!recorrido}
  \item
    Considere cada vértice en un recorrido en postorden.
    Si es una variable o constante,
    cargue su valor en algún registro libre.
    Si es una operación,
    calcule primero aquel argumento que requiere más registros,
    luego el otro,
    y efectúe la operación.
  \end{enumerate}
  Si faltan registros,
  la solución es guardar en una variable temporal
  el contenido de aquel registro que no se usará por más tiempo,
  para reponerlo cuando se necesite.
  Esto resulta óptimo,
  ya que minimiza el número de instrucciones adicionales
  (cada operación requiere al menos una instrucción,
   y nuestro algoritmo usa exactamente una instrucción por operación
   si no quedamos cortos de registros).
  Nótese también que basta con operaciones
  que trabajen entre dos registros
  (origen y destino),
  no hacen falta operaciones con dos orígenes
  y un destino para el resultado.
  Eso sí pueden hacer falta operaciones simétricas,
  con efectos por ejemplo \(a \leftarrow a - b\)
  y \(a \leftarrow b - a\)
  (aunque pueden obtenerse sus efectos
   con una secuencia de dos operaciones,
   por ejemplo calculando \(a \leftarrow a - b\)
   y cambiando el signo).

% Fixme: ¿Ejemplo en descenso recursivo y/o bison/flex?

  \begin{table}[htbp]
    \begin{align*}
      r0 &\leftarrow u	     \\
      r1 &\leftarrow v	     \\
      r0 &\leftarrow r0 + r1 \\
      r0 &\leftarrow -r0     \\
      r1 &\leftarrow c	     \\
      r0 &\leftarrow r0 * r1 \\
      r1 &\leftarrow b	     \\
      r2 &\leftarrow y	     \\
      r1 &\leftarrow r1 * r2 \\
      r0 &\leftarrow r0 + r1 \\
      r1 &\leftarrow a	     \\
      r2 &\leftarrow x	     \\
      r1 &\leftarrow r1 * r2 \\
      r0 &\leftarrow r0 + r1
    \end{align*}
    \caption{Código óptimo para la expresión ejemplo}
    \label{tab:Sethi-Ullman}
  \end{table}
  Para el árbol de la figura~\ref{fig:Sethi-Ullman}
  resulta el código del cuadro~\ref{tab:Sethi-Ullman}.
  El formato de las instrucciones
  es por ejemplo \(r0 \leftarrow r1 - r2\),
  para indicar que al registro \(r0\)
  se le asigna el valor \(r1 - r2\).
  Sólo se aceptan operaciones entre registros,
  traer datos de memoria a un registro
  y llevar datos de un registro a memoria.
  Lamentablemente en situaciones más realistas
  (registros de uso específico,
   instrucciones que no solo operan entre registros,
   subexpresiones comunes,
   aplicar identidades algebraicas)
  la situación es bastante más compleja
  y no hay algoritmos tan simples y eficientes.

\section{Grafos planares}
\label{sec:grafos-planares}
\index{grafo!planar|textbfhy}

  Un grafo se dice \emph{planar} si puede dibujarse en un plano
  sin que se crucen arcos.
  Un caso particular del siguiente resultado
  dio Descartes en 1639,%
    \index{Descartes, Rene@Descartes, René}
  el caso general se debe a Euler en 1751.%
    \index{Euler, Leonhard}
  Eppstein lista 19 demostraciones en~%
    \cite{eppstein05:_nineteen_proofs_Euler},
  la brillante demostración siguiente es de von~Staudt~%
    \cite{staudt47:_geometrie_lage}.
  \begin{theorem}[Fórmula de Euler]
    \index{Euler, formula de (grafos planares)@Euler, fórmula de (grafos planares)}
    Sea \(G = (V, E)\) un grafo planar conexo dibujado en el plano.
    Definimos \(f\) como el número de caras del grafo
    (las áreas separadas por arcos,
     incluyendo el área infinita fuera del grafo),
    \(e = \lvert E \rvert\) y \(v = \lvert V \rvert\).
    Entonces:
    \begin{equation*}
      v - e + f
	= 2
    \end{equation*}
  \end{theorem}
  \begin{proof}
    Sea \(G\) un grafo planar conexo,
    dibujado en el plano.
    Definimos el \emph{dual} de \(G\) como el multigrafo \(G^*\),
    cuyos vértices son las caras de \(G\)
    y cuyos arcos pasan por los puntos medios de los arcos de \(G\)
    que separan las caras,
    vea la figura~\ref{fig:Euler-formula}.
    \begin{figure}[ht]
      \centering
      \pgfimage{images/Euler-formula}
      \caption{Ilustración de la demostración de la fórmula de Euler }
      \label{fig:Euler-formula}
    \end{figure}
    El multigrafo \(G^*\) es conexo,%
      \index{multigrafo}
    ya que podemos ir de cualquiera de las caras de \(G\)
    a cualquiera otra atravesando arcos.
    Consideremos un árbol recubridor de \(G^*\),%
      \index{grafo!arbol recubridor@árbol recubridor}
    llamémosle \(T^*\).
    Si eliminamos los arcos de \(G\) cortados por arcos de \(T^*\),
    queda un grafo que llamaremos \(T\).
    Como \(T^*\) es un árbol,
    no tiene ciclos y no puede cortar \(G\) en componentes conexos,
    así que \(T\) es conexo.
    Por el otro lado,
    \(T\) no tiene ciclos,
    ya que podemos ir de cualquiera de los vértices de \(G^*\)
    a cualquier otro a través de arcos de \(T^*\).
    O sea,
    \(T\) también es un árbol.
    Por construcción,
    el número de arcos de \(G\)
    es la suma del número de arcos de \(T\)
    con el número de arcos de \(T^*\):
    \begin{equation*}
      (v - 1) + (f - 1)
	= e
    \end{equation*}
    Esto es lo que se quería probar.
  \end{proof}

  Llamemos \emph{\(k\)\nobreakdash-cara}
  a una cara acotada por \(k\) arcos,
  y sea \(f_k\) el número de \(k\)\nobreakdash-caras en \(G\).
  Del lema~\ref{lem:handshaking} sabemos que:
  \begin{equation*}
    2 e
      = \sum_{x \in V} \delta(x)
  \end{equation*}
  De forma similar tenemos:
  \begin{equation*}
    f
      = \sum_k f_k \qquad
    2 e
      = \sum_k k f_k
  \end{equation*}
  El número promedio de lados por cara es:
  \begin{equation*}
    \overline{f}
      = \frac{2 e}{f}
  \end{equation*}
  Con esto podemos demostrar que \(K_5\) y \(K_{3, 3}\)
  (\(K_{3, 3}\)
   es el grafo formado por dos conjuntos de tres vértices,
   en el cual todos los vértices de un conjunto están conectados
   con todos los vértices del otro
   pero no hay conexiones dentro de los conjuntos,
   ver la figura~\ref{fig:K33})
   \begin{figure}[htbp]
     \centering
     \pgfimage{images/K33}
     \caption{El grafo $K_{3, 3}$}
     \label{fig:K33}
   \end{figure}
  no son planares.
  Consideremos \(K_5\),
  con \(v = 5\) y \(e = \binom{5}{2} = 10\).
  Para un supuesto dibujo de \(K_5\) en el plano
  calculamos \(f = 2 + e - v = 7\);
  el número promedio de lados por cara
  sería \(\overline{f} = 2 \cdot 10 / 7 < 3\),
  lo que es ridículo.
  De forma similar,
  para \(K_{3, 3}\) resultan \(v = 6\), \(e = 9\) y \(f = 5\),
  con lo que \(\overline{f} = 2 \cdot 9 / 5 < 4\).
  Es fácil verificar
  (ver la figura~\ref{fig:K33})
  que \(K_{3, 3}\) no tiene ciclos de largo tres,
  así que esto es imposible.

  Como toda cara tiene al menos tres lados,
  y al sumar sobre todas las caras los arcos se cuentan dos veces,
  tenemos:
  \begin{equation*}
    2 e
      \ge 3 f
  \end{equation*}
  Substituyendo en la fórmula de Euler:
  \begin{align}
    e + 2
      &\le v + \frac{2 e}{3} \notag \\
    e
      &\le 3 v - 6
	 \label{eq:planar-arcs-vertices}
  \end{align}
  O sea,
  los grafos planares son ralos,
  tienen mucho menos que los arcos posibles.

  Otra conclusión inmediata
  viene de combinar el teorema~\ref{theo:sum-degree=2edges}
  con~\eqref{eq:planar-arcs-vertices}:
  \begin{align}
    \sum_i \delta(v_i)
      &= 2 e \notag \\
    \overline{\delta}
      &=   \frac{2 e}{v} \notag \\
      &\le \frac{6 v - 12}{v} \notag \\
      &<   6
	 \label{eq:planar-average-degree}
  \end{align}
  Por lo tanto,
  tienen que haber vértices de grado menor a seis
  en todo grafo planar.

  \begin{theorem}
    \index{solido regular@sólido regular|see{poliedro!regular}}
    \index{poliedro!regular}
    Hay cinco sólidos regulares.
  \end{theorem}
  \begin{proof}
    Si tomamos el sólido,
    eliminamos una de sus caras
    y extendemos sus aristas en un plano,
    obtenemos un grafo planar
    al tomar los vértices del sólido como vértices del grafo,
    y las aristas como arcos.
    Un ejemplo lo da el dodecaedro
    (sólido con \(12\)~caras pentagonales),
    figura~\ref{fig:dodecaedro}.
    \begin{figure}[htbp]
      \centering
      \pgfimage{images/dodecaedro}
      \caption{El grafo del dodecaedro}
      \label{fig:dodecaedro}
    \end{figure}
    Podemos aplicar la fórmula de Euler al grafo resultante.%
      \index{Euler, formula de (grafos planares)@Euler, fórmula de (grafos planares)}
    Sean las caras del sólido de \(p\) lados,
    y se encuentran \(q\) aristas en cada vértice.
    Por lo anterior,
    \(p f = 2 e\).
    Además,
    como cada arista conecta dos vértices,
    \(q v = 2 e\).
    Substituyendo estos en la fórmula de Euler
    resulta:
    \begin{align}
      \frac{2 e}{p} + \frac{2 e}{q} - e
	&= 2 \notag \\
      \frac{1}{p} + \frac{1}{q}
	&= \frac{1}{2} + \frac{1}{e}
	     \label{eq:Euler-solidos-regulares}
    \end{align}
    Por otro lado,
    debe ser \(p \ge 3\) y \(q \ge 3\),
    las caras deben tener al menos tres lados
    y deben encontrarse al menos tres aristas en un vértice.
    Si \(p\) y \(q\) fueran ambos mayores que tres,
    quedaría:
    \begin{equation*}
      \frac{1}{p} + \frac{1}{q}
	\le \frac{1}{4} + \frac{1}{4}
	= \frac{1}{2}
    \end{equation*}
    Esto contradice a~\eqref{eq:Euler-solidos-regulares},
    por tanto debe ser \(p = 3\) o \(q = 3\).
    Si \(p = 3\),
    de~\eqref{eq:Euler-solidos-regulares} resulta:
    \begin{equation}
      \label{eq:Euler-solidos-regulares-p=3}
      \frac{1}{q} - \frac{1}{6}
	= \frac{1}{e}
    \end{equation}
    Como~\eqref{eq:Euler-solidos-regulares-p=3} debe ser positivo,
    solo están las posibilidades \(q = 3, 4, 5\).
    Estas dan,
    respectivamente \(e = 6, 12, 30\).
    De la misma forma,
    \(q = 3\) da opciones \(p = 3, 4, 5\)
    que dan \(e = 30, 12, 6\).
    \begin{table}[htbp]
      \centering
      \begin{tabular}{|l|r|r|r|l|}
	\hline
	\multicolumn{1}{|c|}{\rule[-0.7ex]{0pt}{3ex}%
				\(\boldsymbol{\{p, q\}}\)} &
	  \multicolumn{1}{c|}{\(\boldsymbol{f}\)} &
	  \multicolumn{1}{c|}{\(\boldsymbol{e}\)} &
	  \multicolumn{1}{c|}{\(\boldsymbol{v}\)} &
	  \multicolumn{1}{c|}{\textbf{Nombre}} \\
	\hline
	\(\{3, 3\}\) &	4 &  6 &  4 & tetraedro	 \\
	\(\{4, 3\}\) &	6 & 12 &  8 & cubo	 \\
	\(\{3, 4\}\) &	8 & 12 &  6 & octaedro	 \\
	\(\{5, 3\)\} & 12 & 30 & 20 & dodecaedro \\
	\(\{3, 5\}\) & 20 & 30 & 12 & icosaedro	 \\
	\hline
      \end{tabular}
      \caption{Poliedros regulares}
      \label{tab:poliedros}
    \end{table}
    Esto da las cinco combinaciones
    listadas en el cuadro~\ref{tab:poliedros},
    que resultan ser todas posibles.
  \end{proof}

  Un problema famoso,
  planteado en~1852 por Francis Guthrie,
  es demostrar que bastan cuatro colores para pintar un mapa
  de forma que no hayan regiones del mismo color con fronteras
  (líneas, no simplemente puntos)
  en común.%
    \index{cuatro colores, teorema de}
  Esto es equivalente a demostrar
  que bastan cuatro colores para colorear todos los grafos planares:
  Considere cada región a colorear como un vértice del grafo,
  y conecte regiones vecinas con arcos.
  Este grafo claramente es planar,
  y un coloreo de él
  corresponde a una asignación de colores a las regiones.
  La primera demostración fue publicada por Appel y Haken~%
    \cite{appel77:_four_color_theorem_1,
	  appel77:_four_color_theorem_2}.
  La demostración parte de que todo posible contraejemplo
  contiene uno de \(1\,936\)~mapas irreductibles,
  y verificar que todos ellos pueden colorearse con cuatro colores.
  Esta tarea se hizo mediante un programa de computador,
  lo que produjo un motín entre los matemáticos
  (ver por ejemplo Swart~%
     \cite{swart80:_philos_impl_four_color_problem}),
  e intentos de una nueva manera de ver
  lo que significa ``demostración''
  (Tymoczko~\cite{tymoczko80:_comput_proof_mathem}).
  Fue el primer teorema importante
  en cuya demostración el computador resulta indispensable,
  es de suponer que es por esto que Petkovšek, Wilf y Zeilberger~%
    \cite{petkovsek96:_AeqB}
  tienen tanto cuidado de indicar
  que sus técnicas de demostración de identidades con sumatorias
  (que requieren el apoyo del computador,
   es común que deban revisar centenares de opciones)
  dan lugar a un ``certificado'',
  que permite verificar el resultado manualmente.

  Algunos detalles de la historia del problema
  y de las técnicas empleadas por Appel y Haken discute Thomas~%
    \cite{thomas98:_updates_four_color_theorem},
  desde entonces se han publicado demostraciones adicionales
  (todas apoyadas de una forma u otra por el computador,
   con lo que desafortunadamente no aplacan los recelos).

\section{Algoritmos de búsqueda en grafos}
\label{sec:busquedas}
\index{grafo!recorrido}
\index{grafo!busqueda@búsqueda|see{grafo!recorrido}}

  En muchas aplicaciones
  se requiere recorrer un grafo en forma completa
  (visitar cada uno de sus vértices),
  o al menos encontrar algún vértice
  que cumpla ciertas condiciones especiales.
  Esto lleva a considerar algoritmos de búsqueda en grafos
  (aunque tal vez un mejor nombre sería algoritmos de recorrido).

\subsection{Búsqueda en profundidad}
\label{sec:DFS}
\index{grafo!recorrido!profundidad}

  La idea de este algoritmo es partir de un vértice,
  visitar algún vecino de este y continuar de allí
  hasta llegar a un vértice ya visitado o no poder continuar,
  para luego retornar al vértice anterior
  y continuar con su siguiente vecino.
  Se le llama \emph{búsqueda en profundidad}
  porque la estrategia esbozada avanza a través del grafo
  todo lo que puede antes de considerar alternativas.
  En inglés
  se llama \emph{\foreignlanguage{english}{Depth First Search}},%
    \index{Depth First Search@\emph{\foreignlanguage{english}{Depth First Search}}|see{grafo!recorrido!profundidad}}
  abreviado \emph{DFS}.
  \begin{example}
    Búsqueda en profundidad

    Consideremos el grafo de la figura~\ref{fig:a-recorrer}.
    \begin{figure}[htbp]
      \centering
      \pgfimage{images/grafo-recorrer}
      \caption{Grafo para ejemplos de recorrido}
      \label{fig:a-recorrer}
    \end{figure}
    El grafo de la figura~\ref{fig:DFS}
    fue recorrido siguiendo el algoritmo búsqueda en profundidad.
    Se partió desde el vértice \(1\),
    y desde allí se recorrió y enumeró el resto de los vértices,
    eligiendo uno al azar como vecino a considerar luego.
    En rojo quedan registrados los arcos por los cuales se pasó.
    \begin{figure}
      \centering
      \pgfimage{images/grafo-recorrer-dfs}
      \caption{El grafo de la figura~\ref{fig:a-recorrer}
	       recorrido en profundidad}
      \label{fig:DFS}
    \end{figure}
  \end{example}

  Búsqueda en profundidad recorre un árbol recubridor del grafo.%
    \index{grafo!arbol recubridor@árbol recubridor}
  En general entrega un componente conexo del grafo,%
    \index{grafo!componente conexo}
  para encontrar todos los componentes conexos
  basta con repetir la búsqueda
  partiendo cada vez de un vértice aún no visitado
  hasta agotar los vértices.
  Ver el algoritmo~\ref{alg:DFS-recursivo}
  para una versión recursiva,
  que se invoca como \(DFS(v)\) para un vértice del grafo.
  Se usan marcas \emph{considerado}
  y \emph{visitado} en este algoritmo
  (y sus sucesores)
  para evitar caer en ciclos infinitos:
  Un vértice \emph{considerado} pero no \emph{visitado}
  está pendiente;
  si está \emph{visitado}
  ya fue procesado y no requiere más atención.
  \begin{algorithm}[htbp]
    \DontPrintSemicolon
    \SetKwFunction{DFS}{DFS}

    \KwProcedure \DFS{\(x\)} \;
    \BlankLine
    Marque \(x\) considerado \;
    \If{\(x\) no visitado}{
      Marque \(x\) visitado \;
      \ForEach{\(y\) adyacente a \(x\), no visitado y no considerado}{
	Marque \(y\) considerado \;
	\DFS{\(y\)} \;
      }
    }
    \caption{Búsqueda en profundidad,
	     versión recursiva}
    \label{alg:DFS-recursivo}
  \end{algorithm}
  El algoritmo~\ref{alg:DFS} es una versión
  que usa un \emph{\foreignlanguage{english}{stack}} explícitamente
  (resultado de eliminar la recursión).
  Debe tenerse cuidado de no incluir los vértices varias veces,
  por eso se marcan como considerados
  cuando se agregan al \emph{\foreignlanguage{english}{stack}}.
  \begin{algorithm}[htbp]
    \DontPrintSemicolon
    \SetKwFunction{DFS}{DFS}
    \SetKwFunction{Push}{push}
    \SetKwFunction{Pop}{pop}

    \KwProcedure \DFS{\(x\)} \;
    \BlankLine
    \KwVariables \(S\): Stack de vértices \;
    \BlankLine
    Marque \(x\) considerado \;
    \Push{\(S, \; x\)} \;
    \While{\(S\) no vacío}{
      \(x \leftarrow \Pop{S}\) \;
      \If{\(x\) no visitado}{
	Marque \(x\) visitado \;
	\ForEach{\(y\) adyacente a \(x\), no visitado y no considerado}{
	  Marque \(y\) considerado \;
	  \Push{\(S, \; y\)} \;
	}
      }
    }
    \caption{Búsqueda en profundidad,
	     versión no recursiva}
    \label{alg:DFS}
  \end{algorithm}

\subsection{Búsqueda a lo ancho}
\label{sec:BFS}
\index{grafo!recorrido!a lo ancho}

  Acá se visitan los vecinos del punto de partida,
  y recién cuando se han visitado todos ellos
  se avanza a los vecinos de estos.
  Por la forma en que avanza en el grafo
  se le llama \emph{búsqueda a lo ancho},
  en inglés \emph{\foreignlanguage{english}{Breadth First Search}},%
    \index{Breadth First Search@\emph{\foreignlanguage{english}{Breadth First Search}}|see{grafo!recorrido!a lo ancho}}
  que se abrevia \emph{BFS}.
  Una manera simple de manejar primero todos los vecinos,
  y recién después los vecinos de estos,
  es ingresar los nodos en una cola de espera
  (\emph{\foreignlanguage{english}{queue}} en inglés).

  \begin{example}
    Búsqueda a lo ancho

    Tomando el grafo del ejemplo de búsqueda en profundidad
    (ver figura~\ref{fig:a-recorrer}),
    ahora se enumeran los vértices y se marcan los arcos recorridos,
    pero ahora con el método de búsqueda a lo ancho.
    Véase la figura~\ref{fig:BFS}.
    \begin{figure}
      \centering
      \pgfimage{images/grafo-recorrer-bfs}
      \caption{El grafo de la figura~\ref{fig:a-recorrer}
	       recorrido a lo ancho}
      \label{fig:BFS}
    \end{figure}
  \end{example}
  Más formalmente el algoritmo
  para buscar a lo ancho es el~\ref{alg:BFS}.
  \begin{algorithm}[htbp]
    \DontPrintSemicolon
    \SetKwFunction{BFS}{BFS}
    \SetKwFunction{Enqueue}{enqueue}
    \SetKwFunction{Dequeue}{dequeue}

    \KwProcedure \BFS{\(x\)} \;
    \BlankLine
    \KwVariables \(Q\): Queue de vértices \;
    \BlankLine
    Marque \(x\) considerado \;
    \Enqueue{\(Q, \; x\)} \;
    \While{\(Q\) no vacío}{
      \(x \leftarrow \Dequeue{Q}\) \;
      \If{\(x\) no visitado}{
	Marque \(x\) visitado \;
	\ForEach{\(y\) adyacente a \(x\), no visitado y no considerado}{
	  Marque \(y\) considerado \;
	  \Enqueue{\(Q, \; y\)} \;
	}
      }
    }
    \caption{Búsqueda a lo ancho}
    \label{alg:BFS}
  \end{algorithm}
  Usamos marcas para asegurarnos de no agregar el mismo nodo varias veces.

\subsection{Búsqueda a lo ancho versus búsqueda en profundidad}
\label{sec:BFS+DFS}

  Si comparamos los dos procedimientos
  (algoritmos~\ref{alg:DFS} y~\ref{alg:BFS})%
    \index{grafo!recorrido}
  se ve que la única diferencia
  es el orden en que se procesan los vértices.
  En el peor caso,
  se exploran todos los vértices
  y se recorren todos los arcos,
  y la complejidad
  es simplemente \(O(\lvert V \rvert + \lvert E \rvert)\)
  para ambos.
  Podemos resumir las características de ambos algoritmos
  como sigue:
  \begin{itemize}
    \item
      Ambos llegan a todos los vértices del componente conexo.
    \item
      Los programas son similares,
      aunque búsqueda en profundidad (recursivo)
      es un tanto más simple de programar.
    \item
      Complejidad (equivalente a tiempo de ejecución) similares.
  \end{itemize}

  La pregunta entonces es cómo elegir entre los dos.
  \begin{itemize}
    \item
      Si interesa una solución cualquiera:
      Usar búsqueda en profundidad.
    \item
      Si interesa ``la mejor''
      (por lo general referida a cercanía de vértices):
      Usar búsqueda a lo ancho.
  \end{itemize}

  Pero un análisis más profundo de los dos algoritmos
  muestra que tienen una estructura común:
  Guardan vértices para considerarlos más adelante
  en alguna estructura,
  la única diferencia es si se procesan en orden de llegada
  (al buscar a lo ancho)
  o se atiende primero al último en llegar
  (búsqueda en profundidad).
  Claramente podemos usar algún otro criterio
  para elegir el siguiente vértice a considerar,
  no solamente el momento
  en que nos encontramos con él por primera vez.
  Esto da lugar a una gama de métodos de búsqueda heurísticos.

% Fixme (next):
%
% Aplicaciones de búsqueda en grafos
% - Componentes conexos
% - Fuertemente conexo
% - Puntos de articulación
% - ...

\section{Colorear vértices}
\label{sec:colorear-vertices}
\index{grafo!coloreo!vertices@vértices}

  Hay muchas situaciones
  en las cuales interesa encontrar una asignación
  que respeta restricciones dadas.
  Una forma de representar estas
  es modelando las entidades a recibir asignaciones
  como vértices,
  las asignaciones como colores de los vértices,
  las restricciones entre asignaciones
  como arcos entre los vértices,
  y buscamos asignar colores
  a los vértices de forma que vértices vecinos
  siempre tengan colores diferentes.

  \begin{example}
    Consideremos la situación
    en que deseamos programar seis charlas,
    cada una de una hora.
    Hay interesados en asistir a varias de ellas,
    en particular,
    hay interesados en las charlas \(1\) y \(2\),
    en las \(1\) y \(4\),
    en \(1\) y \(6\),
    en \(2\) y \(6\),
    en \(3\) y \(5\),
    en \(4\) y \(5\),
    en \(5\) y \(6\).
    Debemos programar las charlas en el mínimo número de horas.

    Podemos modelar esto mediante un grafo,
    conectando las charlas que tienen asistentes en común,
    véase la figura~\ref{fig:charlas}.
    \begin{figure}[htbp]
      \centering
      \pgfimage{images/charlas}
      \caption{Grafo representando charlas}
      \label{fig:charlas}
    \end{figure}
    Por ejemplo,
    podemos asignar
    las charlas \(v_1\) y \(v_3\) a la hora \(1\),
    \(v_2\) y \(v_4\) a la hora \(2\),
    \(v_5\) a la hora \(3\),
    y finalmente \(v_6\) a la hora \(4\).
    Si representamos la hora \(1\) con color azul,
    la \(2\) con rojo,
    la \(3\) con amarillo
    y la \(4\) con verde,
    se obtiene lo que muestra la figura~\ref{fig:charlas-coloreo1}.
    \begin{figure}[htbp]
      \centering
      \pgfimage{images/charlas-coloreo1}
      \caption{Asignación de horas a charlas como colores}
      \label{fig:charlas-coloreo1}
    \end{figure}

    En términos matemáticos,
    hemos particionado los vértices del grafo en cuatro,
    de forma que no hayan vértices adyacentes
    en ninguna de las partes.
    Una forma de representar esta situación usa la función
    \(c \colon V \rightarrow [4]\)
    que a cada vértice
    (charla)
    le asigna una hora.
    Generalmente hablamos de colores de los vértices,
    no de horas asignadas,
    aunque claramente la naturaleza exacta
    de los objetos \(\{1, 2, 3, 4\}\) es irrelevante.
    Podemos hablar de colores azul, rojo, amarillo, verde,
    o de hora \(1\), hora \(2\), hora \(3\), hora \(4\).
    Lo único que importa
    es que vértices vecinos tienen colores diferentes.
  \end{example}

  \begin{definition}
    \index{grafo!coloreo!vertices@vértices|textbfhy}
    Un \emph{coloreo de vértices} de un grafo \(G = (V, E)\)
    es una función \(c \colon V \rightarrow \mathbb{N}\)
    tal que \(c(x) \ne c(y)\)
    siempre que \(x y \in E\).
  \end{definition}
  \begin{definition}
    \index{grafo!numero cromatico@número cromático|textbfhy}
    \index{grafo!numero cromatico@número cromático|seealso{grafo!coloreo!vértices}}
    El \emph{número cromático} de un grafo \(G = (V, E)\),
    escrito \(\chi(G)\),
    es el mínimo entero \(k\)
    tal que el grafo tiene un coloreo de \(k\) colores.
  \end{definition}

  Volviendo a nuestro ejemplo,
  la asignación de horas de la figura~\ref{fig:charlas-coloreo1}
  corresponde a un coloreo con \(4\) colores.
  Pero probando un poco con \(3\) horas
  vemos que podemos asignar \(v_1\) y \(v_5\) a la hora \(1\),
  \(v_2\) y \(v_3\) a la hora \(2\),
  y dejar \(v_4\) y \(v_6\) para la hora \(3\).
  Ver la figura~\ref{fig:charlas-coloreo2}.
  \begin{figure}[htbp]
    \centering
    \pgfimage{images/charlas-coloreo2}
    \caption{Otra asignación de horas a charlas}
    \label{fig:charlas-coloreo2}
  \end{figure}
  Por lo demás,
  se requieren al menos \(3\) colores
  ya que \(v_1\), \(v_2\) y \(v_6\)
  están conectados entre sí.
  Hemos demostrado que el número cromático de este grafo es \(3\).

  En general,
  para demostrar que el número cromático de un grafo es \(k\)
  se requiere:
  \begin{enumerate}
  \item
    Encontrar un coloreo con \(k\) colores.
  \item
    Demostrar que es imposible hacerlo con menos de \(k\) colores.
  \end{enumerate}
  Este problema es \NP\nobreakdash-completo,
  uno de los \(21\)~problemas
  identificados inicialmente como intratables por Karp~%
    \cite{karp72:_reduc_among_combin_prob}.

  \begin{example}
    Números cromáticos

    Veamos cuántos colores se necesitan para colorear
    el grafo de la figura~\ref{fig:chromatic1}.
    \begin{figure}[htbp]
      \centering
      \pgfimage{images/chromatic1}
      \caption{Grafo a colorear}
      \label{fig:chromatic1}
    \end{figure}

    Como el grafo contiene ciclos de largo impar
    (por ejemplo el de la figura~\ref{fig:chromatic1-ciclo}),
    \begin{figure}[htbp]
      \centering
      \pgfimage{images/chromatic1-ciclo}
      \caption{Un ciclo de largo impar
	       en el grafo de la figura~\ref{fig:chromatic1}}
      \label{fig:chromatic1-ciclo}
    \end{figure}
    se desprende la necesidad de al menos 3 colores.
    La figura~\ref{fig:chromatic1-coloreo}
    \begin{figure}[htbp]
      \centering
      \pgfimage{images/chromatic1-coloreo}
      \caption{Un coloreo con tres colores
	       del grafo de la figura~\ref{fig:chromatic1}}
      \label{fig:chromatic1-coloreo}
    \end{figure}
    muestra un coloreo con \(3\)~colores,
    por lo que el número cromático del grafo es precisamente \(3\).
  \end{example}

  \begin{example}
    Números cromáticos: Otro caso.

    Consideremos el grafo de la figura~\ref{fig:chromatic2}.
    \begin{figure}
      \centering
      \pgfimage{images/chromatic2}
      \caption{Otro grafo a colorear}
      \label{fig:chromatic2}
    \end{figure}

    Dado que son \(6\) vértices,
    el número de colores es a lo más \(6\).
    En la figura~\ref{subfig:chromatic2-C5}
    \begin{figure}[htbp]
      \centering
      \subfloat[Subgrafo \(C_5\)]{
	\pgfimage{images/chromatic2-C5}
	\label{subfig:chromatic2-C5}
      }%
      \hspace{2.5em}%
      \subfloat[Subgrafo \(K_4\)]{
	\pgfimage{images/chromatic2-K4}
	\label{subfig:chromatic2-K4}
      }
      \caption{Subgrafos del grafo
	       de la figura~\ref{fig:chromatic2}}
      \label{fig:chromatic2-subgrafos}
% Fixme: Poca diferencia en BN
    \end{figure}
    se marca un ciclo impar en rojo,
    lo que indica que necesitan a lo menos \(3\) colores.
    Pero se ve en azul en la figura~\ref{subfig:chromatic2-K4}
    que el grafo tiene \(K_4\) como subgrafo,
    lo que indica que el número de colores
    tiene que ser a lo menos \(4\).
    \begin{figure}[htbp]
      \centering
      \pgfimage{images/chromatic2-coloreo}
      \caption{Coloreo con cuatro colores
	       del grafo de la figura~\ref{fig:chromatic2}}
      \label{fig:chromatic2-coloreo}
    \end{figure}
    La figura~\ref{fig:chromatic2-coloreo}
    muestra un coloreo con 4~colores,
    el número cromático es \(4\).
  \end{example}

  Como se comentó antes,
  muchos problemas de asignación de recursos
  pueden modelarse mediante coloreo de grafos.
  Por esta razón el encontrar soluciones óptimas
  (coloreo usando \(\chi(G)\) colores)
  o al menos buenas
  (pocos más que \(\chi(G)\) colores)
  tienen inmensa importancia práctica.
  Siendo un problema \NP\nobreakdash-completo%
    \index{NP-completo, problema@\NP-completo, problema}
  la investigación
  se concentra en hallar soluciones a casos especiales
  o hallar métodos aproximados.

  Por ejemplo,
  al compilar código para un programa%
    \index{generar codigo@generar código}
  interesa guardar los más valores posibles
  en los registros del procesador,
  de forma de que acceder a ellos sea más rápido,
  compárese con la sección~\ref{sec:Sethi-Ullman}.
  Una manera de hacer esto da Chaitin~%
    \cite{chaitin82:_regis_alloc_spill_via_graph_color}
  vía construir un grafo
  (el \emph{grafo de interferencia})
  cuyos vértices son los valores
  y hay un arco entre un par de valores si sus vidas
  (desde el instante de creación hasta el último uso)
  se traslapan.
  El mínimo número de registros requeridos
  es simplemente el número cromático
  del grafo de interferencia,
  y un coloreo mínimo
  es una asignación de valores a esos registros.
  Se obtiene un coloreo ordenando
  los vértices en orden de grado decreciente
  y aplicando el algoritmo voraz.
  Esto es rápido
  y da un coloreo suficientemente bueno en la práctica.

  Delahaye~%
    \cite{delahaye06:_sci_behind_sudoku}
  indica que el popular puzzle Sudoku puede interpretarse
  como completar un coloreo
  de un grafo de \(81\)~vértices con \(9\)~colores.%
    \index{Sudoku}

\subsection{El algoritmo voraz para colorear grafos}
\label{sec:coloreo-voraz}

  Encontrar el número cromático de un grafo
  es un problema \NP\nobreakdash-completo.%
    \index{NP-completo, problema@\NP-completo, problema}
  Sin embargo,
  hay una forma simple de construir un coloreo de vértices
  usando un número ``razonable'' de colores.
  La idea es ir asignando colores a los vértices
  de forma de usar siempre el primer color
  que no produce conflictos.%
    \index{algoritmo voraz}
  El algoritmo representa los colores asignados a los vértices
  mediante el arreglo \(c\)
  (índices son los vértices).
  Usamos el ``color'' \(0\)
  para indicar que el vértice aún no ha sido coloreado.
  El conjunto \(S\) recoge los colores de los vecinos del vértice
  bajo consideración.

  \begin{algorithm}[htbp]
    \DontPrintSemicolon
    \SetKwFunction{GreedyColoring}{GreedyColoring}
    % Note that this should not be changed to iterate over vertices
    % as sets, the ordering in which the vertices are considered is
    % relevant.

    \KwProcedure \GreedyColoring{\(G\)} \;
    \BlankLine
    \(n \leftarrow \text{Número de vértices de \(G\)}\) \;
    \For{\(i \leftarrow 1\) \KwTo \(n\)}{
      \(c[v_i] \leftarrow 0\) \;
    }
    \(c[v_1] \leftarrow 1\) \;
    \For{\(i \leftarrow 2\) \KwTo \(n\)}{
      \(S \leftarrow \varnothing\) \;
      \For{\(j \leftarrow 1\) \KwTo \(i - 1\)}{
	\If{\(v_j\) es adyacente a \(v_i\) y no ha sido coloreado}{
	  \(S \leftarrow S \cup \{c[v_j]\}\) \;
	}
      }
      \(c[v_j] \leftarrow\) primer color no en \(S\) \;
    }
    \caption{Coloreo voraz}
    \label{alg:coloreo-voraz}
  \end{algorithm}
  Por ejemplo,
  el coloreo de la figura~\ref{fig:charlas-coloreo1}
  resulta aplicando el algoritmo~\ref{alg:coloreo-voraz}
  a los vértices del grafo de la figura~\ref{fig:charlas}
  en orden del número del vértice.
  El coloreo de la figura~\ref{fig:chromatic1-coloreo}
  resulta de asignar colores azul, rojo, amarillo
  mediante el algoritmo voraz
  partiendo del vértice superior
  y avanzando en sentido contra reloj.
  El lector podrá entretenerse usando este algoritmo
  para colorear el grafo de la figura~\ref{fig:a-recorrer}
  siguiendo ya sea el orden de la figura~\ref{fig:DFS}
  o de la figura~\ref{fig:BFS},
  y de hallar el número cromático.

  El algoritmo~\ref{alg:coloreo-voraz} es corto de vista,
  por lo que el coloreo que entrega no necesariamente es bueno.
  Sin embargo,
  puede producir el coloreo con el mínimo número de colores,
  dependiendo del orden en que se consideren los vértices.
  Lo malo es que si hay \(n\)~vértices
  son \(n!\)~órdenes diferentes a considerar.
  A pesar de esto,
  el algoritmo es útil en teoría y en la práctica.
  Demostraremos un teorema usando esta estrategia.
  \begin{theorem}
    \label{theo:grafo-grados-cromatico}
    Si \(G\) es un grafo de grado máximo \(k\),
    entonces:
    \begin{enumerate}
    \item
      \(\chi(G) \le k + 1\).
    \item
      Si \(G\) no es regular,
      entonces \(\chi(G) \le k\).
    \end{enumerate}
  \end{theorem}
  \begin{proof}
    Como el coloreo de cada componente conexo
    es independiente del resto del grafo,
    basta discutir la situación de un grafo conexo.
    Demostramos cada aseveración por turno.
    \begin{enumerate}
    \item
      Sea \(v_1, v_2, \dotsc, v_n\)
      un orden de los vértices de \(G\).
      Si aplicamos el algoritmo de coloreo voraz,
      al considerar un vértice cualquiera
      tendrá a lo más \(k\) vecinos con colores ya asignados.
      Si contamos con \(k + 1\) colores,
      siempre habrá uno que podamos asignarle.
    \item
      Para este caso consideraremos los vértices
      en un orden especial.
      Como el grafo no es regular,
      habrá al menos un vértice cuyo grado es menor a \(k\).
      Llamémosle \(v_n\) a uno de ellos.
      Numeremos los vecinos de \(v_n\)
      como \(v_{n - 1}\), \(v_{n - 2}\), \(\dots\)
      (hay a lo más \(k - 1\) de estos).
      Una vez agotados los vecinos de \(v_n\),
      numeramos los vecinos de \(v_{n - 1}\) distintos de \(v_n\);
      nuevamente habrá a lo más \(k - 1\) de estos.
      Continuamos de esta forma,
      siempre dejando fuera de consideración
      los que ya tienen número asignado.
      Si ahora aplicamos el algoritmo voraz,
      siempre que considere un vértice
      tendrá a lo más \(k - 1\) vecinos ya coloreados,
      y se requerirán a lo más \(k\) colores.
    \qedhere
    \end{enumerate}
  \end{proof}
  La demostración del primer caso parece poco inteligente,
  pero considerar el grafo \(K_{k + 1}\),
  regular de grado \(k\),
  muestra que en realidad es lo mejor que puede hacerse.
  Por el otro lado,
  el grafo bipartito completo \(K_{n, n}\)
  es regular de grado \(n\),
  pero \(\chi(K_{n, n}) = 2\).

  Esta demostración sugiere la heurística
  de ordenar los vértices por grado decreciente
  y luego aplicar el algoritmo voraz
  como manera de obtener una buena coloración.

% Fixme: Ejemplos de números cromáticos
% Fixme: Polinomios cromáticos

\section{Colorear arcos}
\label{sec:colorear-arcos}
\index{grafo!colorear!arcos}

  Una situación análoga al coloreo de vértices
  se produce si coloreamos arcos
  de forma que no hayan arcos adyacentes
  (que comparten vértices)
  del mismo color.
  Formalmente:
  \begin{definition}
    Dado un grafo \(G= (V, E)\) un \emph{coloreo de arcos}
    es una función \(c \colon E \rightarrow \mathbb{N}\)
    tal que \(c(e_1) \neq c(e_2)\)
    si \(e_1 \ne e_2\) y \(e_1 \cap e_2 \neq \varnothing\).
  \end{definition}

  Acá tenemos una partición de los arcos
  \(E_1 \cup E_2 \cup \dotso \cup E_n\)
  tal que los arcos en \(E_i\) no tienen vértices en común.
  La nomenclatura difiere,
  hay autores que le llaman \emph{número cromático de arcos}%
    \index{grafo!numero cromatico de arcos@número cromático de arcos|see{grafo!índice cromático}}
  al mínimo número de colores en un coloreo de arcos,
  otros le llaman el \emph{índice cromático} del grafo.%
    \index{grafo!indice cromatico@índice cromático}
  Se anota \(\chi'(G)\) o \(\chi_1(G)\),
  aunque la noción
  (y la notación)
  es mucho menos usada que el número cromático.
  Nosotros le llamaremos índice cromático,
  y anotaremos \(\chi'(G)\).

  El índice cromático es \(2\) para un ciclo de orden par,
  y \(3\) para un ciclo de orden impar.
  Esto provee cotas para el índice cromático
  de grafos que contienen los anteriores.
  Obviamente el índice cromático
  es a lo menos el grado máximo de un vértice.

  Un bonito ejemplo es el coloreo de arcos del grafo de Frucht~%
    \cite{frucht39:_herst_graph_gruppe}%
    \index{Frucht, grafo de}
  dado en la figura~\ref{fig:Frucht}.
  Como hay vértices de grado \(3\)
  (en realidad es un grafo \(3\)\nobreakdash-regular),
  es imposible un coloreo con menos colores,
  y su índice cromático es \(3\).
  \begin{figure}[htbp]
    \centering
    \pgfimage{images/Frucht}
    \caption{Un coloreo de arcos del grafo de Frucht}
    \label{fig:Frucht}
  \end{figure}

  Aplicaciones de esto ocurren
  en muchos problemas de asignar recursos y tareas
  de manera que no interfieran en el tiempo
  (ver por ejemplo el texto de Skiena~%
    \cite{skiena08:_algor_desig_manual}).
  Al programar la primera ronda de un torneo de fútbol
  (donde compiten ``todos contra todos'')
  debe hallarse un calendario
  de forma que a nadie se le solicite
  estar jugando dos partidos a la vez,
  pero que también use el mínimo de tiempo
  (si hay \(N\) equipos en competencia,
   son \(N (N - 1) / 2\) los partidos a jugar,
   con lo que la solución obvia de programar un partido por semana
   tomaría demasiado tiempo).
  Esto se modela mediante un grafo
  en el que los vértices son los equipos participantes,
  mientras los arcos son partidos que deben jugarse.
  El coloreo de arcos entonces asigna fechas a los partidos,
  de forma que ningún equipo
  esté jugando dos partidos la misma fecha.
  Es de suponer que el problema real a resolver por la ANFP
  es mucho más complejo,
  al tratar de lograr que todos los fines de semana
  haya un partido entretenido.

\section{Grafos bipartitos}
\label{sec:grafos-bipartitos}
\index{grafo!bipartito|textbfhy}

  Un caso especial muy importante son los grafos
  para los cuales \(\chi(G) = 2\).
  En este caso,
  los vértices se dividen en dos grupos \(V_1\) y \(V_2\)
  que corresponden a los coloreados con los colores \(1\) y \(2\),
  respectivamente,
  y los arcos unen uno de cada grupo.
  En consecuencia,
  estos grafos se llaman \emph{bipartitos}.
  Aplicamos esta idea
  en el ejemplo~\ref{ex:queso} del ratón que come queso.
  Para representar un grafo bipartito escribiremos
  \(G = (V_1 \cup V_2, E)\),
  bajo el entendido que \(V_1\) y \(V_2\)
  son las particiones de los vértices según color.
  Hay muchas situaciones que se pueden modelar de esta forma,
  veremos algunos ejemplos pronto.
  Un caso importante es cuando cada vértice de \(V_1\)
  está conectado con todos los vértices de \(V_2\).
  Así obtenemos el \emph{grafo bipartito completo},%
    \index{grafo!bipartito completo|textbfhy}
  que se anota \(K_{m, n}\)
  si \(\lvert V_1 \rvert = m\) y \(\lvert V_2 \rvert = n\).
  Nótese que \(K_{2,2} \cong C_4\).

  Algunos ejemplos muestran las figuras~\ref{fig:K1n}
  y~\ref{fig:Kmn}.
  Por razones obvias
  a los grafos \(K_{1, n}\) se les suele llamar \emph{estrellas}.%
    \index{grafo!estrella|textbfhy}
  \begin{figure}[htbp]
    \setbox1=\hbox{\pgfimage{images/K11}}
    \setbox2=\hbox{\pgfimage{images/K12}}
    \setbox3=\hbox{\pgfimage{images/K13}}
    \setbox7=\hbox{\pgfimage{images/K17}}
    \centering
    \subfloat[\(K_{1, 1}\)]{
      \begin{minipage}[c]{1.0\wd3}
	\centering
	\makebox{\copy1}
      \end{minipage}
      \label{subfig:K11}
    }%
    \hspace*{4em}%
    \subfloat[\(K_{1, 2}\)]{
      \begin{minipage}[c]{1.0\wd2}
	\centering
	\makebox{\copy2}
      \end{minipage}
      \label{subfig:K12}
    }
    \\[1ex]
    \subfloat[\(K_{1, 3}\)]{
      \begin{minipage}[c][1.0\ht7][c]{1.0\wd3}
	\centering
	\makebox{\copy3}
      \end{minipage}
      \label{subfig:K13}
    }%
    \hspace*{4em}%
    \subfloat[\(K_{1, 7}\)]{
      \begin{minipage}[c]{1.0\wd2}
	\centering
	\makebox{\copy7}
      \end{minipage}
      \label{subfig:K17}
    }
    \caption{Algunas estrellas}
    \label{fig:K1n}
  \end{figure}
  \begin{figure}[htbp]
    \setbox1=\hbox{\pgfimage{images/K22}}
    \setbox2=\hbox{\pgfimage{images/K23}}
    \setbox3=\hbox{\pgfimage{images/K33}}
    \setbox4=\hbox{\pgfimage{images/K34}}
    \setbox5=\hbox{\pgfimage{images/K35}}
    \centering
    \subfloat[\(K_{2, 2}\)]{
      \raisebox{0.5\ht3-0.5\ht1}{\copy1}
      \label{subfig:K22}
    }%
    \hspace{3.5em}%
    \subfloat[\(K_{2, 3}\)]{
      \raisebox{0.5\ht3-0.5\ht2}{\copy2}
      \label{subfig:K23}
    }%
    \hspace{3.5em}%
    \subfloat[\(K_{3, 3}\)]{
      \raisebox{0.5\ht3-0.5\ht3}{\copy3}
      \label{subfig:K33}
    } \\[1ex]
    \subfloat[\(K_{3, 4}\)]{
      \raisebox{0.5\ht5-0.5\ht4}{\copy4}
      \label{subfig:K34}
    }%
    \hspace{4.5em}%
    \subfloat[\(K_{3, 5}\)]{
      \copy5
      \label{subfig:K35}
    }
    \caption{Algunos grafos bipartitos completos}
    \label{fig:Kmn}
  \end{figure}

  Para grafos bipartitos tenemos:
  \begin{theorem}
    Un grafo es bipartito
    si y solo si no tiene ciclos de largo impar.
  \end{theorem}
  \begin{proof}
    Demostramos implicancias en ambas direcciones.

    Si hay un ciclo de largo impar,
    se requieren tres colores solo para colorear ese ciclo,
    y \(\chi(G) \ge 3\).
    Por el otro lado,
    si no hay ciclos de largo impar,
    construiremos un ordenamiento de los vértices
    que produce un coloreo con dos colores.
    Elijamos un vértice cualquiera,
    llamémosle \(v_1\),
    y le asignamos el nivel \(0\).
    A los vecinos de \(v_1\) les llamamos
    \(v_2, \dotsc, v_r\),
    les asignamos el nivel \(1\).
    A los vecinos de los vértices de nivel \(1\)
    que no están ya numerados
    les asignamos el nivel \(2\),
    \ldots,
    a los vecinos no numerados de los vértices de nivel \(l - 1\)
    les asignamos el nivel \(l\).
    De esta forma completamos un componente conexo de \(G\),
    y procesamos a los demás componentes conexos de la misma forma.
    Lo crucial de este orden es que un vértice del nivel \(l\)
    solo tiene vecinos en los niveles \(l - 1\) y \(l + 1\).
    Para ver esto,
    supongamos que hay dos vértices conectados en el mismo nivel.
    Siguiendo sus conexiones hacia atrás
    a través de los distintos niveles,
    encontraremos caminos simples hacia un vértice común,
    que tendrán el mismo largo,
    ver la figura~\ref{fig:ciclo-impar}.
    \begin{figure}[htbp]
      \centering
      \pgfimage{images/ciclo-impar}
      \caption{Un ciclo de largo $2 l + 1$
	       si hay conexiones cruzadas}
      \label{fig:ciclo-impar}
    \end{figure}
    Pero estos forman un ciclo de largo impar
    junto con el arco entre los vértices del mismo nivel
    que supusimos conectados.
  \end{proof}

  Una manera tal vez más simple de entender lo anterior
  (y, a la pasada,
   dar un algoritmo para determinar si un grafo es bipartito
   y obtener las particiones)
  es tomar un vértice cualquiera y pintarlo de rojo,
  sus vecinos colorearlos de azul,
  y así sucesivamente
  ir pintando vértices vecinos aún no coloreados
  del color contrario
  (esto corresponde a búsqueda a lo ancho).
  Si esto termina con todos los vértices coloreados
  (recomenzando con otro aún no coloreado
   si se acabaran los vecinos)
  el grafo es bipartito,
  si encontramos conflictos
  (vecinos del mismo color)
  no lo es.
  Esto es una aplicación de los algoritmos de recorrido de grafos,
  y la complejidad
  es simplemente \(O(\lvert V \rvert + \lvert E \rvert)\).

  \begin{lemma}
    \label{lem:bipartito-grados}
    Sea \(G = (X \cup Y, E)\) un grafo bipartito
    con arcos entre \(X\) e \(Y\).
    Entonces:
    \begin{equation*}
      \sum_{x \in X} \delta(x) = \sum_{y \in Y} \delta(y)
	= \lvert E\rvert
    \end{equation*}
  \end{lemma}
  \begin{proof}
    Cada arco tiene un vértice en \(X\),
    y la primera suma
    es el número total de arcos vistos desde \(X\).
    Lo mismo respecto a la segunda suma e \(Y\).
  \end{proof}

  \begin{theorem}
    \label{theo:bipartito-cromatico}
    El índice cromático de un grafo bipartito%
      \index{grafo!bipartito!indice cromatico@índice cromático}
      \index{grafo!indice cromatico@índice cromático}
    es su grado máximo.
  \end{theorem}
  \begin{proof}
    Sea el grafo \(G = (V, E)\).
    Usamos inducción
    sobre el número de arcos \(m = \lvert E \rvert\).%
      \index{demostracion@demostración!induccion@inducción}
    \begin{description}
    \item[Base:]
      Cuando \(m = 1\),
      hay un único arco,
      y el grado máximo es \(1\).
      Claramente basta un color en este caso.
    \item[Inducción:]
      Supongamos ahora que es cierto
      para todos los grafos bipartitos
      de \(m\) arcos y grado máximo \(k\).
      Consideremos un grafo bipartito \(G = (X \cup Y, E)\)
      con \(m + 1\) arcos
      y grado máximo \(k\).
      Elegimos un arco \(x y \in E\),
      y consideramos el grafo \(G' = (X \cup Y, E')\) donde
      \(E' = E \smallsetminus x y\).
      El grafo \(G'\) tiene \(m\) arcos,
      y por inducción admite un coloreo de arcos con \(k\) colores.
      En \(G\),
      tanto \(x\) como \(y\) tienen grado a lo más \(k\);
      y al eliminar el arco \(x y\),
      en \(G'\) es \(\delta(x) \le k - 1\).
      De la misma forma \(\delta(y) \le k - 1\).
      Luego tanto \(x\) como \(y\)
      participan en a lo más \(k - 1\) arcos
      y por tanto están rodeados por a lo más \(k - 1\) colores.

      Ahora sea \(\alpha\) un color no adyacente a \(x\)
      y \(\beta\) un color no adyacente a \(y\)
      (los llamaremos \emph{libres} en \(x\) e \(y\),
       respectivamente).
      Estos colores deben existir por lo anterior.
      Se pueden presentar dos casos:
      \begin{description}
	\item[Caso simple:]
	  Podemos elegir \(\alpha = \beta\).
	  Tomamos ese color para el arco y asunto resuelto.
	\item[Caso complejo:]
	  No podemos elegir \(\alpha = \beta\).
	  Entonces hay un arco \(x y_1\) de color \(\beta\)
	  (de caso contrario \(\beta\) estaría libre en \(x\)
	   y podría elegir \(\alpha = \beta\)
	   como en el caso anterior).
	  Puede haber un arco \(x_1 y_1\) de color \(\alpha\),
	  un arco \(x_1 y_2\) de color \(\beta\) y así sucesivamente
	  (vamos de \(X\) a \(Y\) mediante arcos de color \(\beta\),
	   y de \(Y\) a \(X\)
	   a través de arcos de color \(\alpha\)).
	  Este proceso de crear un camino debe terminar,
	  ya que el grafo tiene un número finito de vértices,
	  y no puede formar un ciclo
	  ya que no hay arco de color \(\alpha\)
	  en \(x\).
% Fixme: Rehacer el ejemplo, usar maquinaria de grafos colorinches, ...
	  \begin{figure}[htbp]
	    \centering
	    \subfloat{\pgfimage{images/zigzag-1}}%
	    \hspace*{4em}%
	    \subfloat{\pgfimage{images/zigzag-2}}
	    \caption{Cómo operar
		     en el teorema~\ref{theo:bipartito-cromatico}}
	    \label{fig:zig-zag}
	  \end{figure}
	  Como ilustra la figura~\ref{fig:zig-zag},
	  luego de este proceso
	  podemos intercambiar los colores \(\alpha\) y \(\beta\),
	  cayendo en el caso simple.
      \end{description}
    \end{description}
    Por inducción,
    el resultado vale para todos los grafos bipartitos.
  \end{proof}
  Esta técnica de zig-zag
  e intercambio vale la pena tenerla presente,
  bastantes demostraciones se basan en ella.

\subsection{Matchings}
\label{sec:matchings}
\index{grafo!matching|textbfhy}
% Fixme: Encontrar término en castellano

  Supongamos la situación
  en que hay un conjunto \(X\) de personas
  y un conjunto \(Y\) de trabajos.
  Una pregunta con implicancias obvias
  es la siguiente:
  ¿Cómo asignamos personas a las tareas,
  de forma que el número máximo de personas queda asignada
  a una tarea para la que está calificada?
  Esta pregunta la traduciremos al lenguaje de grafos bipartitos.
  La relación ``estar calificado'' da un grafo bipartito
  \(G = (X \cup Y, E)\):
  El arco \(x y\) indica que la persona \(x\)
  está calificada para la tarea \(y\).
  Una asignación de tareas a personas
  corresponde a un \emph{\foreignlanguage{english}{matching}}
  en el sentido técnico que definiremos ahora.
  Otras aplicaciones de estas ideas
  ocurren en una gran variedad de áreas,
  llegando a la economía.
  Cabe hacer notar
  que los conjuntos \(X\) e \(Y\)
  no necesariamente son de la misma cardinalidad.
  Bajo nuestra interpretación
  tiene perfecto sentido considerar situaciones en que hay más
  (o menos)
  tareas a asignar que personas,
  tareas para las que no hay calificados,
  y personas que no están calificadas para ninguna de las tareas.

  \begin{definition}
    Sea \(G = (V, E)\) un grafo.
    Un \emph{\foreignlanguage{english}{matching}}
    es un subconjunto \(M \subseteq E\) de arcos
    tal que no hay vértices en común entre dos arcos.
    El \emph{tamaño} del \emph{\foreignlanguage{english}{matching}}
    es el número de arcos en él.
    Un \emph{\foreignlanguage{english}{matching}} de \(G\)
    se dice \emph{maximal}
    si no hay \emph{\foreignlanguage{english}{matchings}}
    de mayor tamaño
    en \(G\).
    Un \emph{\foreignlanguage{english}{matching}} \(M\)
    se dice que \emph{satura} a los vértices \(U \subseteq V\)
    si todos los vértices de \(U\) participan en \(M\).
  \end{definition}

  El caso más importante de lo anterior se da en grafos bipartitos,
  como se comentó antes.%
    \index{grafo!bipartito!matching@\emph{\foreignlanguage{english}{matching}}|textbfhy}
  Por ejemplo,
  la figura~\ref{fig:matchings}
  muestra dos \emph{\foreignlanguage{english}{matchings}}
  de un grafo,
  donde los arcos en \(M\) están marcados.
  \begin{figure}[htbp]
    \centering
    \subfloat[\(M_1\)]
	     {\pgfimage{images/match-3}\label{subfig:matching-1}}%
    \hspace*{4em}%
    \subfloat[\(M_2\)]
	     {\pgfimage{images/match-max}\label{subfig:matching-2}}
    \caption{Matchings en un grafo bipartito}
    \label{fig:matchings}
  \end{figure}
  El \emph{\foreignlanguage{english}{matching}} \(M_2\) es mayor,
  en el sentido que contiene más arcos.
  No puede haber uno mayor,
  ya que si consideramos el conjunto \(\{x_1, x_2, x_5\}\),
  en total solo están capacitados para \(\{y_3, y_4\}\),
  por lo que necesariamente quedará uno de los tres
  sin trabajo asignado.
  Esto motiva lo siguiente.
  \begin{definition}
    \index{grafo!bipartito!matching completo|textbfhy}
    Sea \(G = (X \cup Y, E)\) un grafo bipartito.
    Un \emph{\foreignlanguage{english}{matching}}
    es \emph{completo} si \(\lvert M \rvert = \lvert X \rvert\)
    (vale decir,
     satura a \(X\)).
  \end{definition}

  Analicemos primero las condiciones
  bajo las cuales
  hay \emph{\foreignlanguage{english}{matchings}} completos.
  Supongamos un grafo bipartito \(G = (X \cup Y, E)\),
  y para todo \(A \subseteq X\) definimos
  el conjunto de vértices vecinos
  (\emph{\foreignlanguage{english}{neighbors}} en inglés) como:
  \begin{equation*}
    N(A) = \{y \colon x y \in E \text{\ para algún\ } x \in A\}
  \end{equation*}
  En un \emph{\foreignlanguage{english}{matching}} completo
  el conjunto de tareas asignadas a los integrantes de \(A\)
  es un subconjunto de \(N(A)\),
  debe ser \(\lvert N(A) \rvert \ge \lvert A \rvert\)
  para todo \(A \subseteq X\).
  En nuestro ejemplo
  es \(N(\{x_1, x_2, x_5\}) = \{y_3, y_4\}\),
  y como
    \(\lvert N(\{x_1, x_2, x_5\}) \rvert
	< \lvert \{x_1, x_2, x_5\} \rvert\)
  no hay \emph{\foreignlanguage{english}{matching}} completo
  posible.

  Resulta que esta condición se cumpla
  para todo subconjunto de \(X\) es necesario y suficiente
  para la existencia
  de un \emph{\foreignlanguage{english}{matching}} completo,
  como demostraremos a continuación.

  \begin{theorem}[Hall]
    \index{Hall, teorema de}
    \label{theo:Hall}
    Sea \(G = (X \cup Y, E)\) un grafo bipartito.
    Entonces
    hay
    un \emph{\foreignlanguage{english}{matching}} completo de \(G\)
    si y solo si para todo \(A \subseteq X\)
    tenemos \(\lvert N(A) \rvert \ge \lvert A \rvert\).
  \end{theorem}
  \begin{proof}
    Demostramos implicancia en ambos sentidos.
    Para simplificar la discusión,
    seguiremos hablando de trabajos,
    calificación para los mismos y trabajos asignados.

    Si hay un \emph{\foreignlanguage{english}{matching}} completo,
    para cada subconjunto \(A \subseteq X\)
    tenemos en \(N(A)\)
    al menos los trabajos asignados a los integrantes de \(A\),
    o sea \(\lvert N(A) \rvert \ge \lvert A \rvert\).

    Al revés,
    supongamos que para todo \(A \subseteq X\)
    se cumple \(\lvert N(A) \rvert \ge \lvert A \rvert\),
    y consideremos
    un \emph{\foreignlanguage{english}{matching}} maximal \(M\)
    de \(G\).
    Demostraremos por contradicción que \(M\) es completo.
    Si \(M\) no es completo,
    demostraremos cómo construir
    un nuevo \emph{\foreignlanguage{english}{matching}} \(M'\)
    tal que \(\lvert M' \rvert = \lvert M \rvert + 1\),
    lo que contradice a que \(M\) era maximal.
    Llamaremos \(A_M(B)\) al conjunto de personas
    a las que el \emph{\foreignlanguage{english}{matching}} \(M\)
    asigna los trabajos en \(B \subseteq Y\).

    Como \(M\) no es completo,
    hay \(x_0 \in X\) que no participa en \(M\).
    Por hipótesis \(x_0\) está calificado al menos para un trabajo,
    el conjunto de tareas para las que está calificado \(x_0\)
    es \(N( \{ x_0 \} )\).
    Consideremos el conjunto \(X_0 = A_M(N( \{ x_0 \} ))\)
    (vale decir,
     las personas que tienen asignados los trabajos
     para los que está calificado \(x_0\)).
    Junto con \(x_0\) son \(\lvert X_0 \rvert + 1\) personas,
    que por hipótesis están calificadas en conjunto
    al menos para \(\lvert X_0 \rvert + 1\) trabajos.
    Esto significa que hay al menos una persona en \(X_0\)
    que está calificada para un trabajo
    que no está en \(N(\{ x_0 \})\),
    y que no tiene ese trabajo asignado.
    Si agregamos las personas
    (de haberlas)
    que tienen asignados esos trabajos a \(X_0\)
    construimos un conjunto \(X_1\).
    Aplicando el mismo proceso nuevamente a \(X_1\)
    construimos un conjunto \(X_2\),
    y así sucesivamente.
    Este proceso debe terminar,
    ya que los conjuntos \(X_i\)
    no pueden crecer en forma indefinida.
    Pero el proceso termina por hallar trabajos
    que no están asignados.
    Tomando uno de ellos y trazando el proceso hacia atrás
    tenemos un camino que parte de \(x_0\),
    va a \(Y\) por una tarea sin asignar,
    vuelve a \(X\) por una tarea asignada,
    \ldots,
    y finalmente pasa de \(X\) a \(Y\) por una tarea sin asignar.
    Este camino tiene un arco más fuera de \(M\) que en \(M\),
    intercambiando las tareas asignadas con las sin asignar en él
    da un \emph{\foreignlanguage{english}{matching}}
    mayor que \(M\).
    Pero habíamos supuesto que \(M\) es maximal,
    una contradicción.
    Por tanto,
    si \(M\) es maximal bajo la hipótesis dada,
    es completo.
  \end{proof}

  La demostración del teorema de Hall motiva la siguiente:
  \begin{definition}
    \index{grafo!bipartito!camino alternante|textbfhy}
    Sea \(G = (X \cup Y, E)\) un grafo bipartito,
    y \(M\) un \emph{\foreignlanguage{english}{matching}} de \(G\).
    Un camino en \(G\)
    se llama \emph{alternante para \(M\)}
    si alterna arcos de \(M\) con arcos que no están de \(M\).
    Un camino alternante se llama \emph{aumentante para \(M\)}
    si comienza y termina en vértices que no participan en \(M\)
    (el primer y el último arco del camino no están en \(M\)).
  \end{definition}

  Nuestra discusión previa
  indicaría que de \(A\)
  a lo más \(\lvert N(A) \rvert\) podrán encontrar trabajo.
  Esto lleva a:
  \begin{definition}
    Sea \(G = (X \cup Y, E)\) un grafo bipartito.
    La \emph{deficiencia} de \(G\) es:%
      \index{grafo!bipartito!deficiencia|textbfhy}
    \begin{equation*}
      d = \max_{A \subseteq X} \{\lvert A \rvert - \lvert N(A) \rvert\}
    \end{equation*}
  \end{definition}
  Siempre podemos tomar \(A = \varnothing\),
  y en tal caso \(\lvert A \rvert - \lvert N(A) \rvert = 0\),
  con lo que la deficiencia nunca es negativa.

  Con esto podemos demostrar:
  \begin{theorem}
    \label{theo:maximal-matching}
    \index{grafo!bipartito!matching maximal|textbfhy}
    Sea \(G = (X \cup Y, E)\)
    un grafo bipartito de deficiencia \(d\).
    Entonces
    el \emph{\foreignlanguage{english}{matching}} maximal \(M\)
    de \(G\)
    cumple \(\lvert M \rvert = \lvert X \rvert - d\).
  \end{theorem}
  \begin{proof}
    Primeramente,
    si \(A \subseteq X\) es un conjunto
    para el cual \(d = \lvert A \rvert - \lvert N(A) \rvert\),
    a lo menos \(d\) elementos de \(A\) quedarán sin \(Y\) asignado,
    y así ningún \emph{\foreignlanguage{english}{matching}}
    puede tener más que el tamaño indicado.
    Basta entonces demostrar
    que hay un \emph{\foreignlanguage{english}{matching}}
    de ese tamaño.

    Sea \(D\) un conjunto de vértices nuevos
    con \(\lvert D \rvert = d\).
    Definimos el grafo \(G^{*} = (X^{*} \cup Y^{*}, E^{*})\)
    mediante:
    \begin{align*}
      X^{*} &= X \\
      Y^{*} &= Y \cup D \\
      E^{*} &= E \cup \{ x y : x \in X \wedge y \in D \}
    \end{align*}
    Estamos agregando un nuevo conjunto de trabajos \(D\)
    para los que todos están calificados.
    Entonces \(G^{*}\)
    cumple con las condiciones del teorema de Hall
    y tiene
    un \emph{\foreignlanguage{english}{matching}} completo
    \(M^{*}\).
    Pero \(M^{*}\) incluye todos los vértices de \(D\),
    ya que si \(A\) es un conjunto
    para el cual \(d = \lvert A \rvert - \lvert N(A) \rvert\)
    es máximo,
    la única forma de parear todos los elementos de \(A\)
    es incluir los elementos de \(D\) en el pareo.
    Eliminando los arcos que incluyen vértices de \(D\) de \(M^{*}\)
    obtenemos un \emph{\foreignlanguage{english}{matching}}
    de tamaño \(\lvert X \rvert - d\).
  \end{proof}

  El teorema~\ref{theo:maximal-matching} no es particularmente útil
  para encontrar
  un \emph{\foreignlanguage{english}{matching}} maximal,
  ni ayuda a la hora de hallar su tamaño
  ya que considera
  analizar los \(2^{\lvert X \rvert}\) subconjuntos de \(X\)
  para determinar la deficiencia.
  Una forma práctica
  de encontrar \emph{\foreignlanguage{english}{matchings}} maximales
  las da el siguiente teorema.
  \begin{theorem}[Lema de Berge]
    \label{theo:matching-augmenting-path}
    \index{Berge, lema de}
    Sea \(G = (X \cup Y, E)\) un grafo bipartito,
    y \(M\) un \emph{\foreignlanguage{english}{matching}} de \(G\).
    Si \(M\) no es maximal,
    \(G\) contiene un camino aumentante para \(M\).
  \end{theorem}
  \begin{proof}
    Sea \(M^{*}\)
    un \emph{\foreignlanguage{english}{matching}} maximal de \(G\),
    y sea \(F = M^{*} \vartriangle M\)
    el conjunto de arcos en que están en \(M^{*}\) o \(M\),
    pero no en ambos.
    Los arcos en \(F\) y los vértices que contienen
    forman un grafo bipartito
    cuyos vértices tienen grado \(1\) o \(2\),
    por lo que sus componentes conexos son caminos y ciclos.
    Pero en todo camino o ciclo
    los arcos de \(M\) alternan con arcos no de \(M\),
    por lo que en todo ciclo el número de arcos en \(M\)
    debe ser igual al número de arcos no en \(M\)
    (al ser \(G\) bipartito,
     no tiene ciclos de largo impar).
    Como \(\lvert M^{*} \rvert > \lvert M \rvert\),
    hay más arcos de \(M^{*}\) que arcos de \(M\) en \(F\).
    Por lo tanto,
    hay al menos un componente conexo de \(F\)
    que es un camino con más arcos en \(M^{*}\)
    que en \(M\),
    y este es un camino aumentante para \(M\).
  \end{proof}

  Esto sugiere la siguiente estrategia
  para hallar un \emph{\foreignlanguage{english}{matching}} maximal:
  \begin{enumerate}
  \item \label{enum:step1}
    Comience
    con
    un \emph{\foreignlanguage{english}{matching}} \(M\) cualquiera
    (un arco por sí solo sirve).
  \item \label{enum:step2}
    Busque un camino aumentante para \(M\).
  \item \label{enum:step3}
    Si encontró un camino aumentante,
    construya un \emph{\foreignlanguage{english}{matching}} \(M'\)
    mejor intercambiando arcos
    que pertenecen
    al \emph{\foreignlanguage{english}{matching}} con los que no
    de la forma usual,
    y vuelva al paso (\ref{enum:step2}) con \(M'\) en vez de \(M\).
  \item \label{enum:step4}
    Si no hay camino aumentante,
    \(M\) es maximal.
  \end{enumerate}
  Para hallar un camino aumentante,
  podemos usar búsqueda a lo ancho.
  Comenzando
  con un vértice \(x_0\) que no tiene trabajo asignado
  construimos un árbol
  de caminos aumentantes ``parciales'' desde \(x_0\)
  como sigue:
  \begin{enumerate}
  \item
    En el nivel \(1\) inserte los vértices \(y_1, y_2, \dotsc, y_k\)
    adyacentes a \(x_0\).
    Si alguno de estos vértices
    no tiene \emph{\foreignlanguage{english}{match}},
    digamos \(y_i\),
    deténgase.
    En este caso \(\left\langle x_0, y_i \right\rangle\)
    es un camino aumentante.
  \item
    Si todos los vértices en el nivel \(1\)
    tienen \emph{\foreignlanguage{english}{match}},
    inserte vértices \(x_1, x_2, \dotsc, x_k\),
    los \emph{\foreignlanguage{english}{matches}}
    de \(y_1, y_2, \dotsc, y_k\),
    en el nivel \(2\).
  \item
    En el nivel \(3\),
    inserte los vértices adyacentes a los de nivel \(2\)
    que no tienen \emph{\foreignlanguage{english}{match}} con ellos.
    Si alguno de ellos
    no tiene \emph{\foreignlanguage{english}{match}},
    deténgase:
    El camino desde \(x_0\) hasta él es un camino aumentante.
  \item
    Si todos los vértices de nivel \(3\)
    tienen \emph{\foreignlanguage{english}{match}},
    inserte sus \emph{\foreignlanguage{english}{matches}}
    en el nivel \(4\),
    \ldots
  \end{enumerate}
  Claramente este proceso puede terminar
  porque no hay vértices a insertar en un nivel impar.
  En tal caso,
  no hemos hallado un camino aumentante,
  y habrá que intentar otro punto de partida.
  Si ninguno de los vértices
  sin \emph{\foreignlanguage{english}{match}}
  resulta en un camino aumentante,
  el \emph{\foreignlanguage{english}{match}} que tenemos es maximal.

  Consideremos el \emph{\foreignlanguage{english}{matching}}
  en el grafo bipartito
  de la figura~\ref{subfig:match-ini}.
  \begin{figure}[htbp]
    \setbox1=\hbox{\pgfimage{images/match-ini}}
    \setbox2=\hbox{\pgfimage{images/match-bca}}
    \centering
    \subfloat[Matching inicial]{
      \copy1
      \label{subfig:match-ini}
    }%
    \hspace{3em}%
    \subfloat[Búsqueda de un camino aumentante]{
      \raisebox{0.5\ht1-0.5\ht2}{\copy2}
      \label{subfig:match-bca}
    }
    \caption{Aumentando un matching}
    \label{fig:matching-1}
  \end{figure}
  \begin{figure}[htbp]
    \centering
    \pgfimage{images/match-fin}
    \caption{Matching resultante}
    \label{fig:match-fin}
  \end{figure}
  El vértice \(x_2\)
  no tiene \emph{\foreignlanguage{english}{match}};
  la figura~\ref{subfig:match-bca} muestra el ``árbol''
  construido a partir de ese vértice según la estrategia descrita,
  indicando un camino aumentante identificado en el proceso;
  y finalmente la figura~\ref{fig:match-fin}
  da el \emph{\foreignlanguage{english}{matching}}
  que resulta de intercambiar los arcos
  pertenecientes
  al \emph{\foreignlanguage{english}{matching}} inicial
  con los que no aparecen en él.
  Si todos los caminos desde el vértice elegido
  son como el camino
    \(\left\langle x_2\;y_2\;x_4\;y_6\;x_5 \right\rangle\),
  con el mismo número de arcos en \(M\) y fuera de él,
  quiere decir que desde ese vértice no hay camino aumentante.

  Un algoritmo mejor es el de Hopcroft-Karp~%
    \cite{hopcroft73:_alg_max_match_bipar_graph},%
    \index{Hopcroft-Karp, algoritmo de}
  que usa la misma estrategia básica,
  pero identifica todas las extensiones posibles en paralelo.

\subsection{Transversales de familias de conjuntos finitos}
\label{sec:transversals}
% Fixme: Encontrar término en castellano

  Una situación de interés se da
  cuando tenemos varios conjuntos que se intersectan,
  y buscamos encontrar un representante único
  para cada conjunto.

  \begin{example}
    En la Universidad de Miskatonic
    todo se resuelve en comités de sus académicos.
    Hay seis profesores que participan en los distintos comités,
    los profesores Atwood, Dexter, Ellery, Freeborn, Halsey y Upham.
    Están organizados en los siguientes comités:

    \begin{tabular}[c]{l}
      Académico: Atwood, Upham \\
      Investigación: Atwood, Dexter, Upham \\
      Administración: Dexter, Upham \\
      Estacionamientos: Ellery, Freeborn, Halsey
    \end{tabular}

    \noindent
    Se decide que cada comité envíe un representante
    al nuevo Comité de Comités de la Universidad,
    y cada uno puede representar solo a un comité.
    Si un integrante pertenece a varios comités,
    asiste como representante de uno de ellos solamente.
    En el ejemplo,
    Atwood puede representar al comité académico
    o al de investigación,
    pero no ambos.
    ¿Es posible crear este comité?

    Dados los miembros de los distintos comités,
    esto se puede lograr de diferentes formas.
    Por ejemplo,
    podemos elegir a Atwood, Dexter, Ellery y Freeborn.
    Sin embargo,
    si el comité de estacionamientos
    estuviera formado solo por Dexter y Ellery,
    no se puede formar el Comité de Comités.
  \end{example}

  La forma general de este problema se expresa más claramente
  usando la noción de \emph{familia de conjuntos}.%
    \index{conjunto!familia|textbfhy}
  Tenemos la familia
    \(\mathcal{F} = \{\mathcal{S}_i \colon i \in \mathcal{I}\}\)
  de conjuntos
  (usamos \(\mathcal{I}\) como el conjunto de los índices,
   básicamente los nombres de los conjuntos),
  no necesariamente diferentes,
  y buscamos un representante \(s_i\)
  para cada \(i \in \mathcal{I}\),
  tales que cada \(s_i \in \mathcal{S}_i\)
  y son todos diferentes.
  Tal conjunto de representantes distintos
  se llama \emph{transversal} de \(\mathcal{F}\).
  Nuestro problema es entonces hallar condiciones
  que aseguren que la familia \(\mathcal{F}\) tenga un transversal.
  Esto tiene perfecto sentido
  incluso cuando~\(\mathcal{I}\) es infinito,
  pero acá nos restringimos a familias finitas.
  En nuestro ejemplo,
  el conjunto índice \(\mathcal{I}\)
  no es más que los nombres de los comités,
  y \(\mathcal{S}_i\) son los integrantes del comité \(i\).

  En realidad,
  esto no es más que una forma disfrazada del problema
  de hallar un \emph{\foreignlanguage{english}{matching}}.
  Para ver esto,
  construimos un grafo bipartito
  cuyas partes corresponden a los conjuntos
  y a los elementos,
  y hay arcos que unen a los elementos
  con los conjuntos a los que pertenecen.
  La figura~\ref{fig:comites}
  muestra la situación
  de los comités de la Universidad de Miskatonic.
  \begin{figure}[htbp]
    \centering
    \pgfimage{images/comites}
    \caption{Comités de la Universidad de Miskatonic}
    \label{fig:comites}
  \end{figure}

  Formalmente,
  definimos \(G = (X \cup Y, E)\) mediante:
  \begin{equation*}
    \begin{array}{l@{{}={}}l@{\qquad}l}
       X & \mathcal{I}
	     & \text{(los nombres de los conjuntos)} \\
       Y & \bigcup_{i \in \mathcal{I}} \mathcal{S}_i
	     & \text{(todos los elementos de los conjuntos)}
    \end{array}	 \end{equation*}
  y el arco \(i y\) está en \(E\) si \(y \in \mathcal{S}_i\).
  Entonces un transversal de \(\mathcal{F}\)
  no es más que
  un \emph{\foreignlanguage{english}{matching}} completo de \(G\).
  En estos términos la condición de Hall es fácil de expresar.
  Un subconjunto \(\mathcal{H}\) de \(\mathcal{I}\)
  es una subfamilia de \(\mathcal{F}\),
  y \(N(\mathcal{H})\)
  es simplemente los miembros de todos esos conjuntos:
  \begin{equation*}
    N(\mathcal{H}) = \bigcup_{i \in \mathcal{H}} \mathcal{S}_i
  \end{equation*}
  Usando esta interpretación,
  el teorema de Hall~%
    \cite{hall35:_repres_subset}
  es:
  \begin{theorem}[Hall, versión original]
    \index{Hall, teorema de}
    La familia finita de conjuntos:
    \begin{equation*}
      \mathcal{F} = \{\mathcal{S}_i \colon i \in \mathcal{I}\}
    \end{equation*}
    tiene un transversal si y solo si:
    \begin{equation*}
      \biggl| \bigcup_{i \in \mathcal{H}} \mathcal{S}_i \biggr|
	\ge \bigl| \mathcal{H} \bigr|
	  \text{\ para todo\ } \mathcal{H} \subseteq \mathcal{I}
    \end{equation*}
  \end{theorem}
  Una forma simple de expresar esto es decir
  que cualquier unión de \(k\) de los conjuntos
  debe tener al menos \(k\) miembros en total.

  En realidad este es el teorema de Hall original,
  que también se conoce
  como ``\emph{\foreignlanguage{english}{Hall's Marriage Theorem}}'',
  por la interpretación siguiente:
  \(\mathcal{I}\) es un conjunto de mujeres,
  mientras \(\mathcal{S}_i\) corresponde al conjunto de hombres
  con los cuales \(i \in \mathcal{I}\) estaría dispuesta a casarse.
  Entonces hay forma de conseguirle pareja a todas las mujeres
  si y solo si para cada conjunto de mujeres
  el conjunto de hombres
  con los que estarían dispuestas a casarse en total
  no es menor a ese conjunto de mujeres.

% Fixme: Mencionar aplicaciones de esto

\section{Grafos rotulados}
\label{sec:grafos-rotulados}
\index{grafo!rotulado}

  En muchas aplicaciones
  los arcos tienen asociados ``pesos'' (costos),
  definimos entonces un \emph{grafo rotulado} como un grafo
  \(G = (V, E)\) y una función \(p \colon E \rightarrow C\)
  (típicamente \(C\) es \(\mathbb{R}\)),
  que asocia el rótulo
  (peso)
  \(p(e)\) al arco \(e\).

  Una situación similar se da con rótulos en los vértices
  (una función \(r \colon V \rightarrow C\)).
  Esto va más allá de la identidad del vértice,
  pueden haber varios vértices con el mismo rótulo.
  Hay aplicaciones en las cuales están rotulados los arcos,
  los vértices,
  o ambos.

\subsection{Árboles rotulados}
\label{sec:arboles-rotulados}
\index{grafo!rotulado!arbol@árbol}

  El resultado siguiente,
  debido a Cayley~%
    \cite{cayley89:_theo_trees},
  fue uno de los máximos triunfos de la combinatoria del siglo~XIX.
  \begin{theorem}[Cayley]
    \index{Cayley, teorema de}
    \label{theo:cayley}
    Hay \(n^{n - 2}\) árboles
    con \(n\) vértices rotulados.
  \end{theorem}
  \begin{proof}
    Sea \(\mathcal{T}\) la clase de árboles con vértices rotulados.%
      \index{metodo simbolico@método simbólico}
    La clase de árboles rotulados con raíz
    corresponde a marcar uno de los vértices,
    o sea es \(\mathcal{T}^\bullet\).
    Estos a su vez están conformados por el vértice raíz
    conectado a las raíces de una colección de árboles rotulados.
    Esto lleva a la ecuación simbólica:
    \begin{equation*}
      \mathcal{T}^\bullet
	= \mathcal{Z} \star \MSet(\mathcal{T}^\bullet)
    \end{equation*}
    Para la función generatriz exponencial correspondiente%
      \index{generatriz!exponencial}
    \(z \widehat{T}'(z)\)
    el teorema~\ref{theo:ms-EGF} da:
    \begin{equation*}
      z \widehat{T}'(z)
	= z \mathrm{e}^{z \widehat{T}'(z)}
    \end{equation*}
    Aplicar la fórmula de inversión de Lagrange,%
      \index{Lagrange, inversion de@Lagrange, inversión de}
    teorema~\ref{theo:LIF},
    da los coeficientes de \(z \widehat{T}'(z)\):
    \begin{align*}
      n \frac{t_n}{n!}
	&= \frac{1}{n} [ u^{n - 1} ] \mathrm{e}^{n u} \\
	&= \frac{1}{n} \cdot \frac{n^{n - 1}}{(n - 1)!} \\
      t_n
	&= n^{n - 2}
    \qedhere
    \end{align*}
  \end{proof}
  Acá lo derivamos de forma muy simple,
  las demostraciones tradicionales son complejas.
  Esta derivación
  es una clara demostración del poder del método simbólico
  (capítulo~\ref{cha:metodo-simbolico})
  en conjunto con herramientas analíticas
  como el teorema de inversión de Lagrange
  (teorema~\ref{theo:LIF}).

\subsection{Costo mínimo para viajar entre vértices}
\label{sec:costo-minimo-entre-vertices}

  En muchos casos interesa saber el costo
  (suma de los pesos de los arcos)
  para llegar a cada uno de los vértices de \(G\) partiendo desde
  el vértice \(v\).
  Hay varios algoritmos para resolver este importante problema.

\subsubsection{Algoritmo de Dijkstra}
\label{sec:Dijkstra}
\index{Dijkstra, algoritmo de}

  Una solución,
  debida a Dijkstra~%
   \cite{dijkstra59:_note_two_probl_connex_graph},
  es aplicar una variante de búsqueda a lo ancho.
  Supongamos que tenemos establecido que el camino más corto
  de \(v\) a \(x\) tiene largo \(l[x]\).
  Supongamos que tenemos un vecino \(y\),
  para el que tenemos la estimación \(l[y]\).
  La ruta más corta que tenemos de \(v\) a \(y\) pasando por \(x\)
  tiene costo \(l[x] + w(x y)\),
  y si nuestra estimación previa \(l[y] > l[x] + w(x y)\),
  debiéramos actualizar \(l[y]\).

  Informalmente,
  el algoritmo es el siguiente:
  Inicialmente sabemos que \(l[v] = 0\).
  Podemos partir con \(l[p] = \infty\)
  para todos los demás vértices \(p\),
  e ir actualizando los \(l[p]\)
  en búsqueda a lo ancho partiendo de \(v\)
  con nuestra mejor estimación hasta el momento
  del largo del camino de \(v\) a cada vértice.

  Una manera de entenderlo
  es considerar el grafo como una colección de hilos
  de los largos de los arcos
  y los vértices nudos entre ellos,
  y ponemos esto sobre la mesa.
  Tomamos el nudo que representa el vértice inicial,
  y lo levantamos hasta que un primer nudo se separa de la mesa.
  Este nudo es el que está más cerca del inicial,
  y registramos su distancia desde este.
  Continuamos de la misma forma,
  cada vez que un nudo se despega de la mesa
  es que hemos llegado a su distancia mínima del nudo inicial.
  En términos de trabajar con el grafo,
  significa mantener una colección de vértices
  a los cuales ya conocemos la distancia mínima
  (inicialmente solo el vértice inicial),
  y luego ir agregando aquel vértice no incluido aún en la colección
  que queda más cerca de alguno
  cuya distancia mínima ya es definitiva.

  Una versión más formal es el algoritmo~\ref{alg:Dijkstra}.
  Lo que hacemos acá es ir calculando distancias tentativas,
  y las dejamos definitivas
  una vez que esté claro que no cambiarán más.
% Fixme: Demostrar correctitud
  \begin{algorithm}[htbp]
    \DontPrintSemicolon
    \SetKwFunction{Dijkstra}{Dijkstra}

    \KwProcedure \Dijkstra{\(G, \; v\)} \;
    \BlankLine
    \KwVariables \(Q\): Conjunto de vértices \;
    \BlankLine
    \(Q \leftarrow V\) \;
    Marque todos los vértices \(x \in Q\) con \(l[x] = \infty\) \;
    \(l[v] \leftarrow 0\) \;
    \While{\(Q\) no vacío}{
      Elija \(x \in Q\) con \(l[x]\) mínimo \;
      \If(Los restantes no son alcanzables){\(l[x] = \infty\)}{
	\KwBreak \;
      }
      \(Q \leftarrow Q \smallsetminus \{x\}\) \;
      \ForEach{\(y\) vecino de \(x\)}{
	\(l[y] \leftarrow \min \{ l[y], l[x] + w(x y) \}\) \;
      }
    }
    \caption{Costos mínimos desde el vértice $v$ (Dijkstra)}
    \label{alg:Dijkstra}
  \end{algorithm}
  El ciclo externo se ejecuta para cada vértice,
  el ciclo interno considera cada arco una vez.
  El tiempo de ejecución de este algoritmo
  depende de la estructura de datos
  usada para representar el conjunto \(Q\),
  siendo:
  \begin{equation*}
    O(\lvert E \rvert
	  \cdot \left\langle
		  \text{\emph{disminuir clave en}\ \(Q\)}
		\right\rangle
	+ \lvert V \rvert
	    \cdot \left\langle
		    \text{\emph{extraer mínimo de}\ \(Q\)}
		  \right\rangle)
  \end{equation*}
  Si mantenemos los \(l[x]\) en un arreglo,
  disminuir la clave es simplemente dos accesos al arreglo,
  extraer el mínimo es recorrer el arreglo.
  Esto es:
  \begin{equation*}
    O(\lvert E \rvert + \lvert V \rvert^2)
      = O(\lvert V \rvert^2)
  \end{equation*}
  Usando una estructura más sofisticada,
  como un \emph{\foreignlanguage{english}{Fibonacci heap}}~%
     \cite{fredman87:_fibonacci_heaps},
  los costos amortizados son:
  \begin{align*}
    \left\langle
      \text{\emph{extraer mínimo de}\ \(Q\)}
    \right\rangle
      &= O(\log \lvert Q \rvert)
       = O(\log \lvert V \rvert) \\
    \left\langle
      \text{\emph{disminuir clave en}\ \(Q\)}
    \right\rangle
      &= O(\log \lvert Q \rvert)
       = O(\log \lvert V \rvert)
  \end{align*}
  y el costo resulta ser:
  \begin{equation*}
    O\left( (\lvert E \rvert
	      + \lvert V \rvert) \log \lvert V \rvert \right)
  \end{equation*}

\subsubsection{Algoritmo de Bellman-Ford}
\label{sec:Bellman-Ford}
\index{Bellman-Ford, algoritmo de}

  Otro algoritmo que resuelve el mismo problema,
  pero que tiene la ventaja
  de poder manejar arcos con costo negativo
  (claro que no ciclos de costo negativo,
   en cuyo caso la distancia mínima no está bien definida),
  es el de Bellman-Ford~%
    \cite{bellman58:_routin_probl}
  (algoritmo~\ref{alg:Bellman-Ford}).
% Fixme: Demostrar correctitud
  \begin{algorithm}[htbp]
    \DontPrintSemicolon
    \SetKwFunction{BellmanFord}{BellmanFord}

    \KwProcedure \BellmanFord{\(G, \; v\)} \;
    \BlankLine
    \(l[v] \leftarrow 0\) \;
    Marque todos los vértices \(x \in V \smallsetminus \{v\}\)
      con \(l[x] = \infty\) \;
    \For{\(i \leftarrow 1\) \KwTo \(\lvert V \rvert - 1\)}{
      \ForEach{\(x y \in E\)}{
	\(l[y] \leftarrow \min\{l[y], l[x] + w(x y)\}\) \;
      }
    }
    \BlankLine
    \If{hay un arco \(x y\) con \(l[x] + w(x y) < l[y]\)}{
      \tcc{Hay un ciclo de costo negativo}
    }
    \caption{Costos mínimos desde el vértice $v$ (Bellman-Ford)}
    \label{alg:Bellman-Ford}
  \end{algorithm}
  Que este algoritmo es correcto
  se demuestra por inducción%
    \index{demostracion@demostración!induccion@inducción}
  sobre las ejecuciones del ciclo \textbf{for} externo.
  \begin{theorem}
    Después de \(k\) repeticiones del ciclo,
    si para un vértice \(u\) tenemos \(l[u] < \infty\)
    entonces es el largo de algún camino desde \(v\) a \(u\).
    Si hay un camino de \(v\) a \(u\) de a lo más \(k\) arcos,
    entonces \(l[u]\) es a lo más el largo del camino más corto
    con a lo más \(k\) arcos de \(v\) a \(u\).
  \end{theorem}
  \begin{proof}
    Por inducción sobre el número de iteraciones.%
      \index{demostracion@demostración!induccion@inducción}
    \begin{description}
    \item[Base:]
      Cuando \(k = 0\),
      antes de ejecutar el ciclo por primera vez.
      En esta situación \(l[v] = 0\)
      y para los demás \(l[u] = \infty\),
      que corresponde a la situación descrita.
    \item[Inducción:]
      Primero lo primero.
      Al ajustar
	\(l[y] \leftarrow \min\{l[y], l[x] + w(x y)\}\),
      por inducción tenemos que \(l[x]\)
      es el largo de algún camino de \(v\) a \(x\),
      y \(l[x] + w(x y)\) es entonces el largo del camino
      que va de \(v\) a \(x\) y luego pasa por el arco \(x y\)
      para llegar a \(y\).
      También \(l[y]\) es el largo de algún camino de \(v\) a \(y\),
      y al ajustar
      estamos depositando en \(l[y]\)
      el largo de algún camino de \(v\) a \(y\).

      Para lo segundo,
      consideremos el camino más corto de \(v\) a \(y\)
      con a lo más \(k\) arcos.
      Sea \(z\) el último vértice antes de \(y\) en este camino.
      Por inducción,
      después de \(k - 1\) iteraciones tenemos que \(l[z]\)
      es a lo más el largo de un camino con a lo más \(k - 1\) arcos
      desde \(v\) a \(z\),
      y \(l[y]\) el largo de un camino de a lo más \(k - 1\) arcos
      desde \(v\) a \(y\).
      Agregando el arco \(z y\) al camino de \(v\) a \(z\)
      tenemos un camino de \(k\) arcos,
      si resulta más corto que el que tiene a lo más \(k - 1\) arcos
      actualizamos \(l[y]\).
      El resultado final,
      luego de considerar todos los arcos que llegan a \(y\),
      es que tenemos el largo
      del camino más corto con a lo más \(k\) arcos.
    \qedhere
    \end{description}
  \end{proof}

  Para completar la demostración
  de que el algoritmo~\ref{alg:Bellman-Ford} es correcto,
  basta observar que luego de \(\lvert V \rvert - 1\) iteraciones
  hemos calculado los costos mínimos
  de los caminos de largo a lo más \(\lvert V \rvert - 1\),
  que es el largo máximo posible de un camino en el grafo.
  Si aún pueden hacerse mejoras,
  es porque hay un ciclo de largo negativo.

  El tiempo de ejecución de este algoritmo es
  \(O(\lvert V \rvert \cdot \lvert E \rvert)\)).
  Esto es mayor que la complejidad del algoritmo de Dijkstra,
  pero como ya indicamos
  este algoritmo tiene la virtud de manejar arcos de largo negativo.
  Por lo demás,
  si en algún ciclo del \textbf{for} externo no hay cambios,
  ya no los habrá más y podemos terminar el algoritmo
  (básicamente hemos llegado al final del camino más largo),
  por lo que esta complejidad es pesimista.
  Hay una mejora debida a Yen~%
    \cite{yen70:_algor_findin_short_routes_all}
  que efectivamente disminuye a la mitad el número de pasadas
  requeridas.

% Fixme: Agregar un ejemplo paso a paso (también para los otros!)

  Una variante de este algoritmo se usa en redes de computadores
  (por ejemplo, es parte del protocolo RIP~\cite{RFC4822})%
    \index{red de computadores!protocolo RIP@protocolo \texttt{RIP}}
  para encontrar rutas óptimas,
  ya que los cómputos pueden distribuirse a los nodos:
  Cada nodo calcula los largos
  (y el primer paso del camino más corto)
  hacia todos los destinos en la red,
  recogiendo información de rutas y distancias óptimas estimadas
  desde sus vecinos,
  actualiza sus propias estimaciones de las mejores rutas
  y sus costos,
  y distribuye los resultados de vuelta a sus vecinos.

% Fixme: Análisis más detallado

\subsubsection{Algoritmo de Floyd-Warshall}
\label{sec:Floyd-Warshall}
\index{Floyd-Warshall, algoritmo de}

% Fixme: Monos paso a paso (ídem otros algoritmos)
% Sugerido por Paulina Silva <pasilva@alumnos.inf.utfsm.cl>

  A diferencia de los anteriores,
  este algoritmo~%
     \cite{floyd62:_algorithm97,warshall62:_theo_boolean_matrices}
  calcula los costos de los caminos mínimos
  entre todos los pares de vértices del grafo.
  Tomamos como los vértices del grafo
  los números \(\{1, 2, \dotsc, \lvert V \rvert\}\),
  y actualizamos un arreglo \(l[i, j]\)
  de forma que después de la iteración \(k\)
  el valor del elemento \(l[i, j]\)
  es el costo del camino más corto entre los vértices \(i\) y \(j\)
  que visita únicamente
  algunos de los vértices \(\{1, 2, \dotsc, k\}\) entremedio.
  Llamemos \(l^{(k)}[i, j]\) a este valor.
  Claramente \(l^{(0)}[i, j] = w(i j)\),
  el costo de ir directamente de \(i\) a \(j\),
  ya que no podemos pasar por ningún vértice intermedio.
  Para calcular el valor de \(l^{(k + 1)}[i, j]\),
  consideramos que el camino más corto
  que pasa a lo más por \(k + 1\)
  tiene dos posibilidades:
  Nunca pasa por \(k + 1\),
  solo llega hasta \(k\),
  el costo es \(l^{(k)}[i, j]\) en tal caso;
  o pasa por \(k + 1\),
  lo que significa que de \(i\) va a \(k + 1\)
  y luego de \(k + 1\) va a \(j\),
  en ambos casos pasando a lo más por \(\{1, 2, \dotsc, k\}\),
  con costo \(l^{(k)}[i, k + 1] + l^{(k)}[k + 1, j]\).
  Calculamos:
  \begin{equation*}
    l^{(k + 1)}[i, j]
      = \min \{l^{(k)}[i, j], l^{(k)}[i, k + 1]
		+ l^{(k)}[k + 1, j]\}
  \end{equation*}
  Una vez alcanzado \(k = \lvert V \rvert\),
  ya no hay restricciones sobre los vértices visitados en el camino
  y hemos calculado los costos mínimos.
  Posibles caminos de largo negativo
  no afectan al funcionamiento del algoritmo.
  Esto resulta en el algoritmo~\ref{alg:Floyd-Warshall},
  donde anotamos simplemente \(l[i, j]\) para \(l^{(k)}[i, j]\).
  Nótese que si no hay cambios entre dos iteraciones,
  ya no habrán más cambios y el algoritmo puede terminarse.
  No seguimos rigurosamente la descripción anterior,
  a veces usamos valores ya actualizados
  (en el fondo,
   de la iteración siguiente)
  de \(l^{(k)}[i, j]\).
  Esto no afecta la correctitud,
  pero ahorra espacio
  (no requiere guardar dos juegos de valores de \(l^{(k)}[i, j]\))
  y hace que el algoritmo sea algo más rápido al adelantar pasos
  si se detiene cuando ya no hay cambios.
% Fixme: Demostrar correctitud
  \begin{algorithm}[htbp]
    \DontPrintSemicolon
    \SetKwFunction{FloydWarshall}{FloydWarshall}

    \KwProcedure \FloydWarshall{\(G = (V, E)\)} \;
    \BlankLine
    \For{\(i \leftarrow 1\) \KwTo \(\lvert V \rvert\)}{
      \For{\(j \leftarrow 1\) \KwTo \(\lvert V \rvert\)}{
	\(l[i, j] \leftarrow w(i j)\) \;
      }
    }
    \For{\(k \leftarrow 1\) \KwTo \(\lvert V \rvert\)}{
      \For{\(i \leftarrow 1\) \KwTo \(\lvert V \rvert\)}{
	\For{\(j \leftarrow 1\) \KwTo \(\lvert V \rvert\)}{
	  \(l[i, j] \leftarrow
		      \min \{l[i, j], l[i, k] + l[k, j]\}\) \;
	}
      }
    }
    \caption{Costos mínimos entre todos los vértices
	     (Floyd-Warshall)}
    \label{alg:Floyd-Warshall}
  \end{algorithm}
  Lo notable de este algoritmo es que en un grafo con \(n\) vértices
  es que efectúa solo \(2 n^3\) comparaciones,
  cuando pueden haber hasta \(n - 1\) arcos en cada camino,
  y debemos considerar \(n (n -1) / 2 \) pares de vértices
  como inicio y fin.

% Fixme: Análisis más detallado
% Fixme: Aplicaciones (p.ej. ruteo): Stallings, Stevens, RFCs
% Fixme: Mostrar los algoritmos paso a paso

\subsection{Árbol recubridor mínimo}
\label{sec:MST}
\index{grafo!arbol recubridor minimo@árbol recubridor mínimo}

  En muchas aplicaciones interesa encontrar el grafo de costo mínimo
  (el costo es simplemente la suma de los pesos de los arcos
   en el subgrafo)
  que conecta a un conjunto dado de vértices.
  Claramente tal grafo será un árbol recubridor,
  el \emph{árbol recubridor mínimo}
  (\emph{\foreignlanguage{english}{Minimal Spanning Tree}}
   en inglés,
   abreviado \emph{MST}).
  Veremos varios algoritmos para construirlo.

\subsubsection{Algoritmo de Prim}
\label{sec:MST-Prim}
\index{Prim, algoritmo de}

  Una estrategia,
  debida a Prim~%
   \cite{prim57:_short_connection_networks}
   es agregar a un árbol parcial
  aquel arco que conecta a un vértice al árbol
  de forma que el costo sea mínimo.%
    \index{algoritmo voraz}

  Un poco más formalmente,
  eso sí abusando de la notación dando el nombre del grafo
  como su conjunto de vértices o arcos según corresponda:
  Inicialmente \(T\) es solo un vértice cualquiera del grafo.
  Sea \(T\) el árbol actual,
  elegimos el vértice \(v \in V \smallsetminus T\)
  tal que el costo de llegar a él
  desde un vértice \(x \in T\) es mínimo.
  Agregamos el arco \(v x\) a \(T\).
  Esto se repite hasta que no queden vértices sin cubrir.

  Aplicando el algoritmo al grafo de la figura~\ref{fig:MST},
  \begin{figure}
    \centering
    \pgfimage{images/grafo-mst}
    \caption{Ejemplo de grafo para árbol recubridor mínimo}
    \label{fig:MST}
  \end{figure}
  paso a paso se obtienen los árboles de la figura~\ref{fig:Prim}.
  \begin{figure}
    \centering
    \subfloat[Paso \(1\)]{
      \pgfimage{images/grafo-Prim-1}
    }%
    \hspace{2em}%
    \subfloat[Paso \(2\)]{
      \pgfimage{images/grafo-Prim-2}
    }

    \subfloat[Paso \(3\)]{
      \pgfimage{images/grafo-Prim-3}
    }%
    \hspace{2em}%
    \subfloat[Paso \(4\)]{
      \pgfimage{images/grafo-Prim-4}
    }

    \subfloat[Paso \(5\)]{
      \pgfimage{images/grafo-Prim-5}
    }
    \caption{El algoritmo de Prim
	     aplicado al grafo de la figura~\ref{fig:MST}}
    \label{fig:Prim}
  \end{figure}

  \begin{theorem}
    El algoritmo de Prim obtiene un árbol recubridor mínimo.
  \end{theorem}
  \begin{proof}
    Por contradicción.%
      \index{demostracion@demostración!contradiccion@contradicción}
    Sea \(w(G)\)  el costo total de los arcos en el grafo \(G\).
    Sea \(T\) el árbol recubridor construido por el algoritmo,
    con \(e_1, e_2, \dots, e_n\) los arcos
    en el orden en que los elige el algoritmo.
    Entonces:
    \begin{equation*}
      w(T) = w(e_1) + w(e_2) + \dotsb + w(e_n)
    \end{equation*}
    Sea \(U\) un árbol recubridor mínimo de \(G\),
    y supongamos que el árbol recubridor \(T\)
    que construye el algoritmo
    no es mínimo,
    vale decir \(w(U) < w(T)\).

    Sea \(e_k\) el primer arco en \(T\)
    (en el orden en que los elige el algoritmo)
    que no está en \(U\).
    Eliminando \(e_k\) de \(T\),
    por el teorema~\ref{theo:arbol-propiedades}
    (propiedad~\ref{T:quitar-arco})
    obtenemos dos componentes conexos que son árboles.
    Habrá algún arco \(e^{*} \in U\)
    que conecta estos dos componentes conexos.
    Si \(w(e^{*}) < w(e_k)\)
    el algoritmo habría elegido \(e^{*}\) en vez de \(e_k\),
    así que \(w(e^{*}) \ge w(e_k)\).

    Aplicando la misma idea al grafo obtenido
    al eliminar los arcos \(e_1\) a \(e_{k - 1}\)
    (y los vértices que contienen)
    del grafo
    vemos que nuestro algoritmo siempre habría elegido un arco
    de costo no mayor
    que el incluido en \(U\),
    con lo que \(w(T) \le w(U)\)
    y \(T\) es un árbol recubridor mínimo.
  \end{proof}

  En este caso,
  la estrategia voraz de elegir ``el mejor ahora''
  sin considerar consecuencias futuras tiene éxito.

  La complejidad del algoritmo depende de cómo se almacena el grafo
  y los respectivos costos por arco.
  Con la representación obvia de matriz de adyacencia
  en la que \(a[i, j]\) es el costo del arco \(i j\)
  y se busca en la matriz es \(O(\lvert V \rvert^2)\),
  almacenando el grafo en una lista de adyacencia
  y usando una estructura eficiente
  para ubicar el mínimo en cada paso
  esto se reduce
  a \(O(\lvert E \rvert + \lvert V \rvert \log \lvert V \rvert)\).
% Fixme: Discutir posibles estructuras de datos

\subsubsection{Algoritmo de Kruskal}
\label{sec:kruskal}
\index{Kruskal, algoritmo de}

  Otra idea,
  debida a Kruskal~%
    \cite{kruskal56:_short_spann_subtree},
  es ordenar los arcos en orden de costo creciente,
  y agregar sucesivamente el siguiente arco que no forma un ciclo.%
    \index{algoritmo voraz}
  Un ejemplo paso a paso para el grafo de la figura~\ref{fig:MST}
  se muestra en la figura~\ref{fig:Kruskal}.
  \begin{figure}
    \centering
    \subfloat[Paso \(1\)]{
      \pgfimage{images/grafo-Kruskal-1}
    }%
    \hspace{2em}%
    \subfloat[Paso \(2\)]{
      \pgfimage{images/grafo-Kruskal-2}
    }

    \subfloat[Paso \(3\)]{
      \pgfimage{images/grafo-Kruskal-3}
    }%
    \hspace{2em}%
    \subfloat[Paso \(4\)]{
      \pgfimage{images/grafo-Kruskal-4}
    }

    \subfloat[Paso \(5\)]{
      \pgfimage{images/grafo-Kruskal-5}
    }
    \caption{El algoritmo de Kruskal
	     aplicado al grafo de la figura~\ref{fig:MST}}
    \label{fig:Kruskal}
  \end{figure}

  Aplicando el algoritmo,
  vamos creando un bosque%
    \index{grafo!bosque|textbfhy}
  (una colección de árboles,
   en inglés ``\emph{\foreignlanguage{english}{forest}}'')
  Inicialmente el bosque es cada vértice por sí solo,
  luego en cada paso conectamos dos árboles.
  Se van agregando arcos con el menor costo posible
  que no generen un ciclo.
  Como el grafo resultante es conexo y no tiene ciclos,
  es un árbol recubridor del grafo.
  En adición,
  resulta un árbol recubridor mínimo,
  por un razonamiento similar al dado para el algoritmo de Prim.
  Nuevamente la estrategia voraz de tomar el mejor localmente
  tiene éxito.

  Para construir un programa eficiente para el algoritmo de Kruskal
  bastan estructuras simples.
  Sabemos de la sección~%
    \ref{sec:dividir-y-conquistar}
  que podemos ordenar los \(\lvert E \rvert\)
  arcos por orden de costo
  con \(O(\lvert E \rvert \log \lvert E \rvert)\) comparaciones.
  Luego consideramos cada arco por turno,
  y vemos si sus extremos están en árboles distintos del bosque
  que estamos construyendo.

  Requerimos seguir la pista a los árboles,
  fundamentalmente mantener conjuntos de vértices
  en cada uno de ellos,
  determinar rápidamente en qué conjunto está cada vértice,
  y unir dos conjuntos.
  Una forma sencilla
  (y suficientemente eficiente para los propósitos presentes)
  es representar cada conjunto mediante una lista,
  en la cual cada elemento
  mantiene un puntero al primer elemento de la lista.
  Para ver en qué lista está un vértice dado
  basta hacer referencia al primer elemento de su lista,
  lo que toma tiempo constante.
  Esta operación se efectúa \(2 \lvert E \rvert\) veces,
  para un total de \(O(\lvert E \rvert)\).
  Crear las \(\lvert V \rvert\) listas
  que representan los vértices individuales
  toma \(O(\lvert V \rvert)\)
  Para unir dos conjuntos
  se agregan los elementos de la lista más corta a la más larga,
  ajustando los punteros correspondientes.
  El costo total en que incurre el algoritmo
  uniendo conjuntos hasta tener uno solo
  podemos calcularlo
  a través de contar cuántas veces en el peor caso
  el puntero al primer elemento de la lista debe ajustarse
  para un vértice \(v\) cualquiera.
  Esto ocurrirá solo si \(v\) pertenece a la lista menor
  en una unión,
  y cada vez que esto ocurra la lista a la que pertenece \(v\)
  al menos se duplica,
  con lo que esto podrá ocurrir
  a lo más \(\log_2 \lvert V \rvert\) veces.
  Como hay un total de \(\lvert V \rvert\) vértices,
  el costo total de las uniones
  será \(O(\lvert V \rvert \log \lvert V \rvert)\).
  El costo total del algoritmo es entonces,
  dado que \(\lvert E \rvert = O(\lvert V \rvert^2)\):
  \begin{equation*}
    O(\lvert E \rvert \log \lvert E \rvert)
      + O(\lvert E \rvert)
      + O(\lvert V \rvert \log \lvert V \rvert)
      = O(\lvert E \rvert \log \lvert V \rvert)
  \end{equation*}

\subsubsection{Árboles recubridores en redes de computadores}
\label{sec:STP}
\index{grafo!arbol recubridor@árbol recubridor}
\index{red de computadores}

  Las redes de área local actualmente en uso
  (ver algún texto del área,
   como Stallings~%
     \cite{stallings10:_data_comput_commun,
	   tanenbaum10:_comput_networ})
  son redes de difusión,
  vale decir,
  lo que una de las estaciones transmite
  lo pueden recibir todas las demás conectadas
  al mismo medio físico.
  Suele ser de interés conectar entre sí redes de área local,
  lo que se hace a través de equipos
  denominados \emph{\foreignlanguage{english}{bridges}},%
    \index{red de computadores!bridge@\emph{\foreignlanguage{english}{bridge}}}
  que se encargan de retransmitir
  solo el tráfico de interés en la otra rama.
  Por razones de confiabilidad
  interesa tener conexiones redundantes,
  vale decir,
  varios caminos entre redes.
  Pero esto introduce la posibilidad de crear ciclos,
  y por tanto tráfico que se retransmite indefinidamente.
  \begin{figure}[htbp]
    \centering
    \pgfimage[width=0.85\textwidth]{images/red-esquema}
    \caption{Esquema de redes interconectadas por \emph{bridges}}
    \label{fig:red-esquema}
  \end{figure}
  Un ejemplo se muestra en la figura~\ref{fig:red-esquema}.
  Esta situación puede modelarse mediante un grafo,
  en el cual las redes son vértices
  y las conexiones entre redes son arcos.
  Para la red de la figura~\ref{fig:red-esquema}
  resulta el grafo de la figura~\ref{fig:red-grafo}.
  \begin{figure}[htbp]
    \centering
    \pgfimage{images/red-grafo}
    \caption{La red de la figura~\ref{fig:red-esquema} como grafo}
    \label{fig:red-grafo}
  \end{figure}
  En estos términos,
  lo que se busca es hallar un árbol recubridor del grafo
  (inhabilitando \emph{\foreignlanguage{english}{bridges}}
   que no participan en él).
  Para cumplir con redundancia y tolerancia a fallas
  (y evitar el siempre presente error humano)
  interesa que en caso de falla
  la red se reconfigure automáticamente.
  Los \emph{\foreignlanguage{english}{bridges}}
  originalmente contemplados
  (que conectan entre sí a dos redes)
  han sido desplazados
  por \emph{\foreignlanguage{english}{switches}},
  que cumplen la misma función
  pero pueden conectar varias redes entre sí.

  Para la tarea descrita se han estandarizado algoritmos
  (conocidos como STP,%
    \index{red de computadores!protodolo STP@protocolo \texttt{STP}}
   por \emph{\foreignlanguage{english}{Spanning Tree Protocol}}
   y RSTP,%
    \index{red de computadores!protodolo RSTP@protocolo \texttt{RSTP}}
   por \emph{\foreignlanguage{english}{Rapid Spanning Tree Protocol}},
   definidos por IEEE~\cite{ieee:_802.1d-2004})
  mediante los cuales
  los equipos cooperan para
  configurar un árbol recubridor automáticamente.
  Perlman~%
    \cite{Perlman:1985:ADC:318951.319004}
  describe cómo se hace.
  En resumen
  es que intercambian sus prioridades
  (fijadas por el administrador de la red)
  junto con sus identificadores
  (fijados de fábrica, únicos en el mundo).
  Aquel que tenga el mínimo par prioridad e identificador
  se elige de raíz,
  y todos los demás calculan sus distancias hacia la raíz
  a través de cada una de sus interfases.
  Se habilitan únicamente las interfases
  que dan el camino más corto hacia la raíz.
  Este proceso se repite periódicamente,
  de forma de reaccionar frente a fallas y reconfiguraciones.

%%% Local Variables:
%%% mode: latex
%%% TeX-master: "clases"
%%% End:


% digrafos.tex
%
% Copyright (c) 2009-2014 Horst H. von Brand
% Derechos reservados. Vea COPYRIGHT para detalles

\chapter{Digrafos, redes, flujos}
\label{cha:digrafos}
\index{digrafo|textbfhy}
\index{grafo!dirigido|see{digrafo}}

  En muchas situaciones que podrían modelarse mediante grafos
  las conexiones no son bidireccionales.
  Por ejemplo,
  están las calles de una sola vía.
  En un proyecto hay actividades que deben efectuarse en orden,
  interesa representar estas dependencias
  y organizarlas de forma de completar el proyecto lo antes posible.
  Si queremos analizar el flujo a través de una red de tuberías
  o el flujo de bienes transportados
  hay una dirección definida.
  En esto suele interesar la capacidad de transporte de la red.
  Estas situaciones se modelan por grafos dirigidos rotulados.
  Generalmente se tratan solo en un capítulo de textos
  que se concentran en grafos,
  pero se puede argüir
  que la importancia práctica de los grafos dirigidos
  (digrafos, para abreviar)
  es similar a la de los grafos.
  Trataremos algunos algoritmos importantes del área
  con análisis somero de sus rendimientos.

\section{Definiciones básicas}
\label{sec:definiciones-basicas}

  Nos interesa representar estructuras similares a grafos,
  solo que los arcos tienen una dirección definida.
  No está la simetría entre ambos extremos del arco como en grafos.
  Por ejemplo,
  si queremos representar las llamadas de funciones en un programa,
  interesa cuál de las dos es quien llama y cuál es llamada.
  \begin{definition}
    Un \emph{digrafo}
    (o \emph{grafo dirigido})
    consta de un conjunto finito \(V\) de \emph{vértices},%
      \index{digrafo!vertice@vértice|textbfhy}
    y un subconjunto \(A\) de \(V \times V\),
    cuyos miembros se llaman \emph{arcos}.%
      \index{digrafo!arco|textbfhy}
    Anotaremos \(D = (V, A)\)
    para el digrafo definido de esta forma.
  \end{definition}
  Un texto reciente,
  que cubre desde temas elementales
  hasta áreas de investigación activa,
  es el de Bang-Jensen y~Gutin~%
    \cite{bang-jensen09:_digraphs}.

  Los digrafos se pueden representar gráficamente
  de forma similar a los grafos,
  solo que en este caso un arco es un par ordenado \((x, y)\),
  mientras en el grafo es un par no ordenado \(\{x, y\}\).
  Si \((x, y)\) es un arco,
  lo indicamos mediante una flecha de \(x\) a \(y\),
  si hay un arco \((x, x)\) lo indicamos mediante un ciclo,
  si hay arcos \((x, y)\) e \((y, x)\)
  los indicamos por dos flechas.
  Para simplificar notación,
  similar al caso de grafos
  usaremos \(u v\) para indicar el arco \((u, v)\).
  La figura~\ref{fig:digrafos} muestra ejemplos.
  \begin{figure}[htbp]
    \setbox1=\hbox{\pgfimage{images/digrafo-1}}
    \setbox2=\hbox{\pgfimage{images/digrafo-2}}
    \centering
    \subfloat{\raisebox{0.5\ht2-0.5\ht1}{\copy1}}%
    \hspace{3em}%
    \subfloat{\copy2}
    \caption{Ejemplos de digrafos}
    \label{fig:digrafos}
  \end{figure}
  Nuestras representaciones de grafos para uso computacional
  se aplican con cambios obvios a este caso.%
    \index{digrafo!representacion@representación}
  Igualmente,
  podemos aplicar los algoritmos de recorrido acá.
  Los algoritmos de Dijkstra,
  algoritmo~\ref{alg:Dijkstra},%
    \index{Dijkstra, algoritmo de}
  de Floyd-Warshall,
  algoritmo~\ref{alg:Floyd-Warshall}%
    \index{Floyd-Warshall, algoritmo de}
  (que en este caso es llamado simplemente algoritmo de Floyd)%
    \index{Floyd, algoritmo de}
  y de Bellman-Ford~\ref{alg:Bellman-Ford}%
    \index{Bellman-Ford, algoritmo de}
  son aplicables con modificaciones obvias.

  Un digrafo
  es simplemente otra manera de representar una relación \(R\)%
    \index{relacion@relación}
  entre elementos del \emph{mismo} conjunto
  (un grafo bipartito,
  por otro lado,
  representa una relación
  entre elementos de conjuntos \emph{disjuntos}).
  Las propiedades de las relaciones pueden fácilmente traducirse
  en propiedades del digrafo.
  Por ejemplo,
  si la relación es simétrica los arcos aparecen en pares
  (salvo los bucles),
  \(x y\) es un arco exactamente cuando lo es \(y x\).

  Las definiciones de camino,
  camino simple,
  circuito y ciclo en un grafo dirigido
  son análogas a las para grafos.
  Un \emph{camino dirigido} en \(D = (V, A)\)%
    \index{digrafo!camino dirigido|textbfhy}
  es una secuencia de vértices \(v_1, v_2, \dotsc, v_k\)
  tal que \(v_i v_{i + 1} \in A\) para \(1 \le i \le k - 1\),
  un \emph{camino dirigido simple} es un camino dirigido
  en que todos los vértices son diferentes,
  un \emph{circuito dirigido}%
    \index{digrafo!circuito dirigido|textbfhy}
  es un camino cuyo inicio y fin coinciden,
  mientras un \emph{ciclo dirigido}%
    \index{digrafo!ciclo dirigido|textbfhy}
  es un camino dirigido
  en que todos los vértices son distintos,
  solo que el inicial y el final coinciden.
  Un caso particularmente importante
  lo ponen los \emph{grafos dirigidos acíclicos}%
    \index{digrafo!aciclico@acíclico}%
    \index{grafo dirigido aciclico@grafo dirigido acíclico|see{digrafo!acíclico}}
  (abreviados comúnmente \emph{DAG},%
    \index{DAG|see{digrafo!acíclico}}
   por la frase en inglés
   \emph{\foreignlanguage{english}{Directed Acyclic Graph}}),
  de alguna forma análogos a los árboles.
    \index{grafo!arbol@árbol}

\section{Orden topológico}
\label{sec:topological-sort}
\index{digrafo!orden topologico@orden topológico}

  Dado un digrafo \(G = (V, E)\),
  un \emph{orden topológico}
  (en inglés,
   \emph{\foreignlanguage{english}{topological sort}})
  de los vértices de \(G\)
  es un ordenamiento de \(V\) tal que para cada arco \(u v \in E\)
  el vértice \(u\) aparece antes que \(v\).
  Esto claramente solo puede existir si el digrafo es acíclico.

  La aplicación típica es definir un orden
  para efectuar un conjunto de tareas
  que dependen entre sí.
  Por ejemplo,
  al vestirse uno debe ponerse los calcetines
  antes que los zapatos,
  pero no hay precedencia entre la camisa y los calcetines.
  La figura~\ref{fig:dressing} ilustra las restricciones.
  \begin{figure}[ht]
    \centering
    \pgfimage{images/vestirse}
    \caption{Restricciones al vestirse}
    \label{fig:dressing}
  \end{figure}
  Otras aplicaciones aparecen en compiladores,
  al reordenar instrucciones;%
    \index{generar codigo@generar código}
  al definir el orden
  en que se calculan las celdas en planillas de cálculo;
  y lo usa \texttt{make(1)}%
    \index{make@\texttt{make(1)}}
  para organizar el orden en que se generan los archivos.
  Unix%
    \index{Unix}
  ofrece el comando \texttt{tsort(1)},%
    \index{tsort@\texttt{tsort(1)}}
  que toma líneas indicando dependencias
  y entrega un orden consistente con ellas.
  Aplicando este último a la tarea de vestirse,
  sugiere el orden
  calcetines,
  camisa,
  pantalón,
  jersey,
  zapatos,
  cinturón
  y finalmente chaqueta.
  Este orden no es único,
  se ve de la figura que podríamos haber comenzado por la camisa,
  o haber terminado con los zapatos.

  Algoritmos para hallar un orden topológico
  se basan en que en un digrafo acíclico
  habrán vértices que no tienen arcos de entrada
  (respectivamente de salida).
  Kahn~\cite{kahn62:_topol_sorting_large_net}
  publicó el algoritmo clásico~\ref{alg:tsort-Kahn}.%
    \index{Kahn, algoritmo de}
  \begin{algorithm}[htbp]
    \DontPrintSemicolon

    \(L \leftarrow \text{lista vacía}\) \;
    \(S \leftarrow \text{conjunto de nodos sin arcos entrantes}\) \;
    \While{\(S \ne \varnothing\)}{
      Extraiga \(n\) cualquiera de \(S\) \;
      Agregue \(n\) al final de \(L\) \;
      \ForEach{nodo \(m\) con arco \(e = (n, m)\)}{
	Elimine \(e\) del grafo \;
	\If{\(m\) no tiene más arcos entrantes}{
	  Inserte \(m\) en \(S\) \;
	}
      }
    }
    \uIf{quedan arcos en el grafo}{
      \Return{error} \tcc*{El digrafo tiene ciclos}
    }
    \Else{
      \Return{\(L\)}
    }
    \caption{Ordenamiento topológico de Kahn}
    \label{alg:tsort-Kahn}
  \end{algorithm}
  Se basa en la observación que en un digrafo acíclico
  deben haber vértices sin arcos entrantes,
  y que cualquiera de ellos puede tomar el primer lugar en el orden.

  El algoritmo~\ref{alg:tsort-Tarjan} es debido a Tarjan~%
    \cite{tarjan76:_edge_disjoin_spann_trees_depth_first_searc}.%
    \index{Tarjan, algoritmo de}
  \begin{algorithm}[htbp]
    \DontPrintSemicolon
    \SetKwFunction{Visit}{visit}

    \(L \leftarrow \text{lista vacía}\) \;
    \(S \leftarrow \text{conjunto de todos los nodos sin arcos salientes}\) \;
    \BlankLine
    \KwProcedure \Visit{\(n\)} \;
    \Begin{
      \If{\(n\) no ha sido visitado}{
	Marque \(n\) como visitado \;
	\ForEach{nodo \(m\) con \((m, n) \in E\)}{
	  \Visit{\(m\)} \;
	}
	Agregar \(n\) al final de \(L\) \;
      }
    }
    \BlankLine
    \ForEach{nodo \(n\) en \(S\)}{
      \Visit{\(n\)} \;
    }
    \Return{L} \;
    \caption{Ordenamiento topológico de Tarjan}
    \label{alg:tsort-Tarjan}
  \end{algorithm}
  Es una aplicación de búsqueda en profundidad
  (ver la sección~\ref{sec:DFS}).%
    \index{grafo!busqueda en profundidad@búsqueda en profundidad}
  Notar que procesa los vértices desde el final hacia el comienzo
  (orden inverso al algoritmo de Kahn).
  Cuidado,
  como el algoritmo~\ref{alg:tsort-Tarjan}
  como está escrito falla si el digrafo tiene ciclos.
% Fixme: Revisar TAoCP1

% Fixme: Agregar más material (p.ej. de tarea sobre digrafos)

\section{Redes y rutas críticas}
\label{sec:redes-rutas-criticas}
\index{red}

  Es común que se asocien costos o distancias a los arcos.
  Con esta idea en mente,
  llamaremos \emph{red} a un digrafo \(D = (V, A)\)
  junto con una función \(w \colon A \rightarrow \mathbb{R}\),%
    \index{red|see{digrafo}}
  que representa costos de algún tipo
  o capacidades,
  según la aplicación.

  Una aplicación típica de redes
  son las \emph{redes de actividades}.%
    \index{red de actividades}
  Suponiendo un gran proyecto,
  este se subdivide en actividades menores.
  Las actividades a su vez están relacionadas,
  en el sentido que algunas no pueden iniciarse
  antes que terminen otras.
  Al planificar un proyecto de este tipo
  se suele usar una red de actividades,
  con arcos representando actividades
  y los vértices representando ``eventos'',
  donde un evento es el fin de una actividad.
  Es claro que tal digrafo es acíclico,
  ninguna actividad puede depender directa o indirectamente
  de sí misma.
  El peso de un arco es la duración de la actividad,
  y se busca organizar las actividades
  de forma de minimizar el tiempo total del proyecto.
  Técnicas basadas en esta idea son
  CPM (\foreignlanguage{english}{Critical Path Method})~%
    \cite{kelley59:_CPM}%
    \index{Critical Path Method@\emph{\foreignlanguage{english}{Critical Path Method}}}%
    \index{CPM|see{\emph{\foreignlanguage{english}{Critical Path Method}}}}
  y PERT (\foreignlanguage{english}
	  {Project Management and Evaluation Technique})~%
    \cite{malcolm59:_PERT}.%
    \index{Project Management and Evaluation Technique@\emph{\foreignlanguage{english}{Project Management and Evaluation Technique}}}%
    \index{PERT|see{\emph{\foreignlanguage{english}{Project Management and Evaluation Technique}}}}

  Consideremos un ejemplo concreto.
  El cuadro~\ref{tab:actividades}
  da las duraciones de las actividades
  (en meses),
  y las dependencias entre ellas
  (qué actividades deben estar completas
   antes de comenzar la actividad indicada).
  \begin{table}[htbp]
    \centering
    \begin{tabular}[c]{|c|l|l|>{\(}c<{\)}|}
      \hline
      \multicolumn{1}{|c|}{\rule[-0.7ex]{0pt}{3ex}\textbf{Act}} &
	 \multicolumn{1}{c|}{\textbf{Descripción}} &
	 \multicolumn{1}{c|}{\textbf{Requisitos}} &
	 \multicolumn{1}{c|}{\textbf{Dur}} \\
      \hline
	\rule[-0.7ex]{0pt}{3ex}%
      \(A\) & Diseño del producto    &			   & 5 \\
      \(B\) & Análisis de mercado    &			   & 1 \\
      \(C\) & Análisis de producción & \(A\)		   & 2 \\
      \(D\) & Prototipo del producto & \(A\)		   & 3 \\
      \(E\) & Diseño de folleto	     & \(A\)		   & 2 \\
      \(F\) & Análisis de costos     & \(C\)		   & 3 \\
      \(G\) & Pruebas del producto   & \(D\)		   & 4 \\
      \(H\) & Entrenamiento ventas   & \(B\), \(E\)	   & 2 \\
      \(I\) & Definición de precios  & \(H\)		   & 1 \\
      \(J\) & Reporte del proyecto   & \(F\), \(G\), \(I\) & 1 \\
      \hline
    \end{tabular}
    \caption{Actividades y dependencias}
    \label{tab:actividades}
  \end{table}

  El primer paso es construir la red de actividades,
  ver figura~\ref{fig:actividades}.
  \begin{figure}[htbp]
    \centering
    \pgfimage{images/actividades}
    \vspace*{2ex}

    \begin{tabular}[c]{l@{\quad}
		       *{9}{>{\(}c<{\)}@{\hspace{0.75em}}}>{\(}c<{\)}}
      Actividad:
	  & A	   & B	    & C	& D
	  & E	   & F	    & G	& H
	  & I	   & J \\
      Arco:
	  & (1, 2) & (1, 3) & (2, 4) & (2, 5)
	  & (2, 3) & (4, 7) & (5, 7) & (3, 6)
	  & (6, 7) & (7, 8) \\
      Duración:
	  &	 5 &	  1 &	   2 &	    3
	  &	 2 &	  3 &	   4 &	    2
	  &	 1 &	  1
    \end{tabular}
    \caption{Una red de actividades}
    \label{fig:actividades}
  \end{figure}
  Los arcos representan actividades,
  y los vértices sus inicios y finales.
  Es claro que resulta un digrafo sin ciclos.
  Para cada evento calcularemos \(E(v)\),
  el instante más temprano en que ese evento puede tener lugar.
  Esto corresponde al momento más temprano
  en que todas las actividades previas a \(v\) han terminado.
  Iniciamos el proceso con \(E(1) = 0\).
  Luego está claro que \(E(2) = 5\),
  ya que la única actividad involucrada es \(A = (1, 2)\).
  Ahora,
  como el evento 3 involucra a \(B = (1, 3)\),
  pero también \(A = (1, 2)\) y \(E = (2, 3)\),
  \(E(3) = \max \{E(1) + 1, E(2) + 2\}
	 = \max \{0 + 1, 5 + 2\}
	 = 7\).
  Continuando de esta forma,
  obtenemos los instantes de término
  dados en el cuadro~\ref{tab:actividades-E},
  \begin{table}[htbp]
    \centering
    \begin{tabular}[c]{l*{9}{>{\(}c<{\)}}}
      \(v\):	  & 1 &	 2 &  3 &  4 &	5 &  6 &  7 &  8 \\
      \(E(v)\):	  & 0 &	 5 &  7 &  7 &	8 &  9 & 12 & 13 \\
    \end{tabular}
    \caption{Términos más tempranos por actividad
	     para la red de la figura~\ref{fig:actividades}}
    \label{tab:actividades-E}
  \end{table}
  con lo que el plazo mínimo
  para completar el proyecto es de \(13\) meses.

  Esto es básicamente usar el algoritmo de Dijkstra%
    \index{Dijkstra, algoritmo de}
  (ver sección~\ref{sec:Dijkstra}),
  solo que estamos calculando
  el camino más \emph{largo} a través de la red.
  Éste está perfectamente definido en este caso,
  ya que no hay ciclos.
  Basta un barrido a lo ancho,
  al no haber ciclos
  se obtiene el valor final directamente.
  En cada vértice visitado calculamos \(E(v)\)
  partiendo del evento inicial \(s\)
  (el comienzo del proyecto)
  mediante la regla:
  \begin{equation*}
    E(s) = 0 \qquad E(v) = \max_x \{E(x) + w(x, v)\}
  \end{equation*}
  donde el máximo es sobre los vértices \(x\) predecesores de \(v\).

  Esto es parte de la técnica
  que se conoce como \emph{análisis de camino crítico}.%
    \index{analisis de camino critico@análisis de camino crítico}
  El resto de la técnica continúa como sigue:
  Calculamos \(L(v)\),
  el último instante en que puede ocurrir el evento \(v\)
  sin retrasar el proyecto completo
  de forma similar a como se calcularon los \(E(v)\),
  pero comenzando del evento final \(t\) y trabajando en reversa:
  \begin{equation*}
    L(t) = E(t) \qquad L(v) = \min_x \{L(x) - w(v, x)\}
  \end{equation*}
  donde el mínimo es sobre los vértices \(x\) sucesores de \(v\).
  Aplicando esto al ejemplo de la figura~\ref{fig:actividades}
  da el cuadro~\ref{tab:actividades-L}.
  \begin{table}[htbp]
    \centering
    \begin{tabular}[c]{l*{9}{>{\(}c<{\)}}}
      \(v\):	& 1 &  2 &  3 &	 4 &  5 &  6 &	7 &  8 \\
      \(L(v)\): & 0 &  5 &  9 &	 9 &  8 & 11 & 12 & 13
    \end{tabular}
    \caption{Inicio más tardío por actividad
	     para la red~\ref{fig:actividades}}
    \label{tab:actividades-L}
  \end{table}

  Podemos además calcular la \emph{holgura} de cada actividad%
    \index{analisis de camino critico@análisis de camino crítico!holgura}
  \(u v\),
  que representa el máximo retraso
  que puede sufrir el comienzo de la actividad
  sin retrasar el fin del proyecto:
  \begin{equation*}
    F(u, v) = L(v) - E(u) - w(u, v)
  \end{equation*}
  Las actividades sin holgura se dice que son \emph{críticas},
    \index{analisis de camino critico@análisis de camino crítico!actividad critica@actividad crítica}
  cualquier retraso en éstas
  se traduce en un retraso del proyecto completo.
  Toda red de actividades
  tiene al menos un camino dirigido entre principio y fin
  formado únicamente por actividades críticas,
  tal camino se llama \emph{ruta crítica}.
    \index{analisis de camino critico@análisis de camino crítico!ruta critica@ruta crítica}
  Combinando los valores en los cuadros~\ref{tab:actividades-E}
  y~\ref{tab:actividades-L}
  obtenemos las holguras
  dadas en el cuadro~\ref{tab:actividades-F}.
  \begin{table}[htbp]
    \centering
    \begin{tabular}[c]{l@{\hspace{0.5em}}*{10}{@{\hspace{0.75em}}>{\(}c<{\)}}}
      Actividad:
	 & A	  & B	   & C	   & D
	 & E	  & F	   & G	   & H
	 & I	  & J \\
      Arco:
	 & (1, 2) & (1, 3) & (2, 4) & (2, 5)
	 & (2, 3) & (4, 7) & (5, 7) & (3, 6)
	 & (6, 7) & (7, 8)  \\
      Holgura:
	 &	0 &	 8 &	  2 &	   0
	 &	2 &	 2 &	  0 &	   2
	 &	2 &	 0
    \end{tabular}
    \caption{Holguras para las actividades
	     de la figura~\ref{fig:actividades}}
    \label{tab:actividades-F}
  \end{table}
  Son actividades críticas las que no tienen holgura,
  en nuestro caso
  \(A\), \(D\), \(G\) y \(J\);
  y se ve en la figura~\ref{fig:actividades}
  que forman un camino a través del grafo.

  El procedimiento
  es fundamentalmente un par de recorridos a lo ancho%
    \index{grafo!recorrido a lo ancho}
  del digrafo.
  Como se discutió para este algoritmo en el caso de grafos
  en la sección~\ref{sec:BFS+DFS},
  la complejidad es \(O(\lvert V \rvert + \lvert E \rvert)\).

% Fixme: Discuss PERT a bit, references. Text book!

\section{Redes y flujos}
\label{sec:redes-flujos}
\index{red!flujo}

  En lo que sigue interpretaremos los arcos como ``tuberías''
  por las que puede fluir alguna mercadería.
  Un  ejemplo claro es el de circuitos eléctricos,%
    \index{circuito electrico@circuito eléctrico}
  con corrientes en los distintos conductores.
  Nótese que a diferencia de la aplicación anterior
  a redes de actividades
  acá un ciclo dirigido es perfectamente posible
  (aunque probablemente ineficiente).
  Los pesos numéricos representan la capacidad del arco.
  Además,
  habrá un vértice \(s\) con la propiedad que todos los arcos
  que contienen a \(s\) se alejan de él,
  y otro vértice \(t\)
  con la propiedad de que todos los arcos que lo contienen
  se dirigen a él.
  Al primero se le llama \emph{fuente}%
    \index{red!fuente}
    (en inglés \emph{\foreignlanguage{english}{source}}),%
    \index{red!source@\emph{\foreignlanguage{english}{source}}|see{red!fuente}}
  al segundo \emph{sumidero}%
    \index{red!sumidero}
    (en inglés \emph{\foreignlanguage{english}{sink}}.%
    \index{red!sink@\emph{\foreignlanguage{english}{sink}}|see{red!sumidero}}
  En resumen,
  trataremos con redes que incluyen:
  \begin{enumerate}[label=(\roman{*})]
  \item
    Un digrafo \(D = (V, A)\).
  \item
    Una función de capacidad \(c \colon A \rightarrow \mathbb{R}\).
    Comúnmente haremos referencia a capacidades
    de enlaces inexistentes,
    usamos la convención que si \(x y \notin A\)
    entonces \(c(x y) = 0\).
  \item
    Una fuente \(s\) y un sumidero \(t\).
  \end{enumerate}
  La figura~\ref{fig:flow-network} muestra una red,
  con los arcos rotulados con sus capacidades.
  También describe un flujo en esta red.
  \begin{figure}[htbp]
    \centering
    \pgfimage{images/flow-network}
    \\[2ex]
    \begin{tabular}[c]{r*{9}{c}}
      \((x, y)\):  & \((s, a)\) & \((s, b)\) & \((s, c)\) & \((a, d)\)
		   & \((b, d)\) & \((c, d)\) & \((a, t)\) & \((c, t)\)
		   & \((d, t)\) \\
      \(c(x, y)\): &	      5 &	   4 &		3 &	     7
		   &	      2 &	   7 &		3 &	     5
		   &	      4 \\
      \(f(x, y)\): &	      3 &	   2 &		3 &	     1
		   &	      2 &	   1 &		2 &	     2
		   &	      4
    \end{tabular}
    \caption{Una red, sus capacidades y un flujo en la red}
    \label{fig:flow-network}
  \end{figure}
% Fixme: Notación: Usar u, v en vez de x, y?

  Supongamos que algún material fluye por la red,
  y sea \(f(x, y)\) el flujo a lo largo del arco \(x y\),
  de \(x\) a \(y\).
  Si \(x y\) ni \(y x\) son arcos,
  por convención \(f(x, y) = 0\).
  Insistiremos para todos los vértices,
  salvo para la fuente y el sumidero,
  en que el flujo que entra al vértice debe ser el flujo que sale
  (no hay acumulación de material en los vértices).

  \begin{definition}
    \index{red!flujo|textbfhy}
    Un \emph{flujo} en una red es una función
    que asigna un número \(f(x, y)\) a cada arco \(x y\),
    sujeto a las condiciones:
    \begin{description}
    \item[Viabilidad:]
      El flujo en cada enlace es a lo más su capacidad,
      \(f(x, y) \le c(x, y)\) para todo arco \(x y \in A\).
    \item[Simetría torcida:]
      En inglés \emph{\foreignlanguage{english}{skew symmetry}},
      una convención de notación:
      \(f(y, x) = -f(x, y)\) para todo arco \(x y \in A\).
    \item[Conservación:]
      Para todo vértice \(y \in V\)
      que no sea la fuente ni el sumidero de la red
      requerimos que el flujo neto que entra en él sea cero:
      \begin{equation*}
	\sum_{x \in V} f(x, y) = 0
      \end{equation*}
    \end{description}
    El \emph{valor} del flujo es el flujo total que entra a la red:
    \begin{equation*}
      \val(f) = \sum_{x \in V} f(s, x)
    \end{equation*}
  \end{definition}

  Exploremos un poco las propiedades de la definición.
  La viabilidad indica únicamente que el flujo
  no puede exceder la capacidad del enlace.
  Simetría torcida
  es simplemente una conveniencia notacional
  (básicamente,
   si vamos ``contra la corriente''
   contabilizamos el flujo como negativo).
  La conservación indica que el flujo neto
  que entra a un vértice es nulo,
  y por simetría el que sale también:
  \begin{equation*}
    \sum_{y \in V} f(x, y)
      = \sum_{y \in V} -f(y, x)
      = -\sum_{y \in V} f(y, x)
      = 0
  \end{equation*}
  En el caso de circuitos eléctricos,
  la conservación es lo que se conoce como la ley de Kirchhoff.%
    \index{Kirchhoff, leyes de}
  Si no hay arco \(x y\),
  no puede haber flujo entre ellos,
  y \(f(x, y) = - f(y, x) = 0\).
  Como no se permite acumulación en los vértices intermedios,
  está claro que el flujo que sale de \(s\)
  debiera ser el flujo que entra en \(t\):
  \begin{equation*}
    \val(f) = \sum_{x \in V} f(x, t)
  \end{equation*}
  Esto lo demostraremos formalmente más adelante.

  Un problema obvio
  es obtener el valor máximo del flujo en una red.%
    \index{red!flujo maximo@flujo máximo}
  Este problema nos ocupará de ahora en adelante.

  Para conveniencia,
  definimos los flujos de entrada y salida de un vértice:
  \begin{align*}
    \inflow(v)
      &= \sum_{\mathclap{\substack{
			   u \in V	\\
			   f(u, v) > 0
	      }}} f(u, v) \\
    \outflow(u)
      &= \sum_{\mathclap{\substack{
			   v \in V	 \\
			   f(u, v) > 0
	      }}} f(u, v)
  \end{align*}
  Algunos autores anotan \(v^{-}\)
  para lo que llamamos \(\inflow(v)\),
  y similarmente \(v^{+}\) para \(\outflow(v)\).

  En estos términos,
  conservación se expresa simplemente
  como el flujo que entra a un vértice es igual al que sale
  si el vértice no es la fuente ni el sumidero:
  \begin{equation*}
    \inflow(v) = \outflow(v)
  \end{equation*}

% Fixme: (Paulina Silva <pasilva@alumnos.inf.utfsm.cl>) Agregar monito
%	 con vértice + flujos,
%	 vértice con entrantes + vertice con salientes

  Lo indicado en la figura~\ref{fig:flow-network}
  debe cumplir las condiciones para ser un flujo.
  La figura no da flujos ``contracorriente'',
  estamos suponiendo implícitamente
  que por ejemplo \(f(d, b) = - f(b, d) = -2\).
  Las condiciones de capacidad se cumplen,
  ya que por ejemplo \(f(c, d) = 1\) mientras \(c(c, d) = 7\).
  También debemos verificar conservación,
  por ejemplo que para el vértice \(d\)
  la suma de los flujos se anula:
  \begin{align*}
    f(a, d) + f(b, d) + f(c, d) + f(t, d)
      &= 1 + 2 + 1 - 4 \\
      &= 0
  \end{align*}
  El valor de este flujo
  es la suma de los flujos que salen de la fuente,
  vale decir:
  \begin{align*}
    \val(f)
      &= f(s, a) + f(s, b) + f(s, c) \\
      &= 8
  \end{align*}
  Resulta también,
  como esperábamos,
  que el flujo hacia el sumidero
  es igual al flujo que sale de la fuente:
  \begin{equation*}
    f(a, t) + f(d, t) + f(c, t)
      = 8
  \end{equation*}
  No contabilizamos flujos entre \(s\) y los demás vértices
  (respectivamente entre los otros vértices y \(t\))
  ya que no hay conexiones directas entre ellos.

\subsection{Trabajando con flujos}
\label{sec:trabajando-flujos}

  Usaremos una convención de \emph{suma implícita},%
    \index{red!suma implicita@suma implícita}
  en que si mencionamos un conjunto de vértices
  como argumento a \(f\),
  estamos considerando la suma de los flujos sobre ese conjunto,
  y similarmente para \(c\).
  Por ejemplo,
  al anotar \(f(X, Y)\),
  donde \(X\) e \(Y\) son conjuntos de vértices,
  entenderemos:
  \begin{equation*}
    f(X, Y) = \sum_{\substack{
		      x \in X \\
		      y \in Y
		   }} f(x, y)
  \end{equation*}
  En estos términos,
  la condición de conservación
  se reduce a \(f(V, x) = 0\) para todo \(x \notin \{s, t\}\)
  (recuérdese que por convención
   \(s\) es la fuente y \(t\) el sumidero de la red).
  Además,
  omitiremos las llaves al restar conjuntos de un solo elemento.
  Así,
  en \(f(s, V - s) = f(s, V)\)
  la notación
  \(V - s\) significa el conjunto \(V \smallsetminus \{s\}\).
  Esto simplifica mucho las ecuaciones que involucran flujos.
  El lema siguiente recoge varias de las identidades más comunes.
  La demostración queda como ejercicio.
  \begin{lemma}
    \label{lem:identidades}
    Sea \(D = (V, A)\) una red,
    y sea \(f\) un flujo en \(D\).
    Entonces:
    \begin{enumerate}
    \item
      Para todo \(X \subseteq V\),
      se cumple \(f(X, X) = 0\).
    \item
      Para todo \(X, Y \subseteq V\)
      se cumple \(f(X, Y) = - f(Y, X)\).
    \item
      Para todo \(X, Y, Z \subseteq V\) siempre que
      \(X \cap Y = \varnothing\).
      se cumplen:
      \begin{align*}
	f(X \cup Y, Z)
	  &= f(X, Z) + f(Y, Z) \\
	f(Z, X \cup Y)
	  &= f(Z, X) + f(Z, Y)
      \end{align*}
    \end{enumerate}
  \end{lemma}

  Como un ejemplo de uso de la notación
  y del lema~\ref{lem:identidades},
  demostraremos que \(\val(f) = f(V, t)\).
  Intuitivamente,
  lo que entra a la red por la fuente debe salir por el sumidero,
  ya que no se permiten acumulaciones entremedio.
  Formalmente:
  \begin{align*}
    f(V, t)
      & = f(V, V) - f(V, V - t) \\
      & = -f(V, V - t) \\
      & = f(V - t, V) \\
      & = f(s, V) + f(V - s - t, V) \\
      & = f(s, V) \\
      &= \val(f)
  \end{align*}
  En esto usamos el hecho:
  \begin{align*}
    f(V - s - t, V)
      &= \sum_{x \in V \smallsetminus \{s, t\}} f(x, V) \\
      &= -\sum_{x \in V \smallsetminus \{s, t\}} f(V, x) \\
      &= 0
  \end{align*}
  que sigue de conservación,
  ya que cada término de la última suma se anula.

  Las capacidades son los flujos máximos en la dirección indicada.
  Si hay tuberías de capacidades \(5\) de \(u\) a \(v\)
  y \(3\) de \(v\) a \(u\),
  el máximo flujo de \(u\) a \(v\) es \(5\)
  y el máximo de \(v\) a \(u\) es \(3\).
  Esto se traduce en que al aplicar la notación a capacidades
  los términos negativos se omiten.

\subsection{Método de Ford-Fulkerson}
\label{sec:ford-fulkerson}
\index{Ford-Fulkerson, metodo de@Ford-Fulkerson, método de}

  Presentaremos ahora una manera de obtener el flujo de máximo valor
  en una red.
  No lo llamaremos ``algoritmo'',
  ya que la estrategia general~%
    \cite{ford56:_max_flow_networks}
  puede implementarse de varias formas,
  con características diferentes.
  \begin{algorithm}[htbp]
    \DontPrintSemicolon

    Inicialice \(f\) en \(0\) \;
    \While{hay un camino aumentable \(p\)}{
      Aumente el flujo \(f\) a lo largo de \(p\) \;
    }
    \caption{El método de Ford-Fulkerson}
    \label{alg:ford-fulkerson}
  \end{algorithm}
  En el proceso introduciremos varias ideas importantes
  en muchos problemas relacionados con flujos.
  Supondremos que las capacidades son enteras,
  de otra forma puede ser que los métodos planteados
  no terminen nunca
  (aunque converjan hacia la solución).

  El método de Ford-Fulkerson es iterativo.
  Comenzamos con \(f(x, y) = 0\) para todo arco \(x y\),
  con un valor inicial cero.
  En cada iteración aumentamos el valor del flujo
  a través de identificar
  un \emph{camino aumentable}%
    \index{red!camino aumentable}
  (\emph{\foreignlanguage{english}{augmenting path}} en inglés,
   un camino entre \(s\) y \(t\) que no está en su máxima capacidad)
  y aumentamos el flujo a lo largo de este camino.
  Continuamos hasta que no se pueda encontrar
  otro camino aumentable.
  Por el teorema \emph{\foreignlanguage{english}{Max-Flow Min-Cut}}
  (teorema~\ref{theo:max-flow=min-cut},
   que demostraremos más adelante)
  al finalizar el valor del flujo es máximo.
  Si las capacidades
  están dadas por números enteros,
  los flujos también serán enteros.
  Los flujos no pueden crecer indefinidamente,
  los algoritmos terminan.

\subsection{Redes residuales}
\label{sec:red-residual}
\index{red!residual|textbfhy}

  Dada una red y un flujo \(f\),
  habrán arcos que admiten flujo adicional,
  y estos arcos con sus capacidades sin usar
  a su vez constituyen una red.
  Intuitivamente,
  esta nos dice cuáles son las posibles mejoras del flujo,
  así interesa analizar la relación entre esta red y la original.

  Más formalmente,
  supongamos una red \(D = (V, A)\),
  con capacidades \(c \colon A \rightarrow \mathbb{R^+}\),
  fuente \(s\) y sumidero \(t\).
  Sea \(f\) un flujo en \(D\),
  y consideremos un par de vértices \(x\) e \(y\).
  El flujo adicional que podemos enviar de \(x\) a \(y\)
  antes de sobrepasar la capacidad de ese enlace
  es la \emph{capacidad residual} del enlace \(x y\),
  que se anota \(c_f(x, y)\).
  Por ejemplo,
  si \(c(x, y) = 10\) y \(f(x, y) = 7\),
  podemos enviar un flujo adicional de \(3\) de \(x\) a \(y\)
  sin sobrepasar la capacidad de ese enlace,
  o podemos disminuir ese flujo en \(7\).
  O sea,
  para un enlace \(x y\)
  con flujo positivo \(f(x, y)\)
  tenemos una capacidad residual
  de \(c_f(x, y) = c(x, y) - f(x, y)\)
  de \(x\) a \(y\),
  y una capacidad residual
  de \(c_f(y, x) = f(x, y)\) de \(y\) a \(x\).
  De forma similar,
  para un enlace \(x y\) con flujo negativo \(f(x, y)\),
  podemos aumentar el flujo de \(x\) a \(y\)
  disminuyendo el flujo de \(y\) a \(x\),
  el máximo aumento posible es dejarlo en cero.
  La \emph{red residual} \(D_f = (V, A_f)\) inducida por \(f\)
  es simplemente la red formada por los enlaces
  y sus capacidades residuales
  (capacidad libre con la corriente,
   flujo a través del enlace en contracorriente),
  o sea \(A_f = \{u v \in V \times V \colon c_f (u, v) > 0\}\).
  En \(D_f\) todos los arcos pueden admitir flujos mayores a cero,
  los arcos de \(D_f\) son ya sea arcos de \(D\) o sus reversos.

  Conviene definir la suma de flujos \(f_1\) y \(f_2\):
  \begin{equation*}
    (f_1 + f_2)(u, v) = f_1 (u, v) + f_2 (u, v)
  \end{equation*}
  Nótese que esto no siempre es un flujo,
  ya que no necesariamente
  cumple las restricciones de nuestra definición.

  El siguiente lema relaciona flujos en \(D\) con flujos en \(D_f\).
  \begin{lemma}
    \label{lem:suma-flujos}
    Sea \(D = (V, A)\) una red con fuente \(s\) y sumidero \(t\),
    y \(f\) un flujo en \(D\).
    Sea \(D_f = (V, A_f)\) la red residual inducida por \(f\),
    y \(f'\) un flujo en \(D_f\).
    Entonces la suma \(f + f'\) es un flujo en \(D\),
    con valor \(\val(f + f') = \val(f) + \val(f')\).
  \end{lemma}
  \begin{proof}
    Primero,
    para verificar que es un flujo,
    debe cumplir las condiciones de la definición.
    Para simetría torcida tenemos:
    \begin{align*}
      (f + f')(x, y)
	&= f(x, y) + f' (x, y) \\
	&= -f(y, x) - f' (y, x) \\
	&= - (f + f')(y, x)
    \end{align*}
    Para las restricciones de capacidad,
    note que \(f' (x, y) \le c_f (x, y)\) para todo \(x, y \in V\)
    (incluso cuando \(f'(x, y) < 0\),
     o sea,
     el arco \(x y\) es contracorriente).
    Por tanto:
    \begin{align*}
      (f + f')(x, y)
	&= f(x, y) + f' (x, y) \\
	&\le f(x, y) + (c(x, y) - f(x, y)) \\
	&= c(x, y)
    \end{align*}
    Para conservación,
    al ser \(f\) y \(f'\) flujos,
    con \(x \ne s, t\) es \(f(x, V) = f'(x, V) = 0\),
    y \((f + f')(x, V) = 0\).

    Finalmente:
    \begin{align*}
      \val(f + f')
	&= (f + f')(s, V) \\
	&= f(s, V) + f' (s, V) \\
	&= \val(f) + \val(f')
    \end{align*}
    Como cumple con nuestra definición y nuestras convenciones,
    queda demostrado lo que perseguíamos.
  \end{proof}

\subsection{Caminos aumentables}
\label{sec:caminos-aumentables}
\index{red!camino aumentable|textbfhy}

  Dada una red \(D = (V, A)\) y un flujo \(f\),
  un \emph{camino aumentable}
  (en inglés \emph{\foreignlanguage{english}{augmenting path}})
  \(p\)
  es un camino dirigido entre \(s\) y \(t\)
  en la red residual \(D_f\).
  Por la definición de red residual,
  cada arco \(x y\) a lo largo de \(p\)
  admite flujo positivo de \(x\) a \(y\)
  sin violar la restricción de capacidad.
  Podemos aumentar el flujo a lo largo de \(p\) en:
  \begin{equation*}
    c_f(p) = \min_{(x, y) \in p} \{c_f (x, y)\}
  \end{equation*}
  sin sobrepasar la capacidad de ningún enlace.
  A \(c_f(p)\) se le llama la \emph{capacidad residual} de \(p\).
  Hemos demostrado:
  \begin{lemma}
    \label{lem:f_p}
    Sea \(D = (V, A)\) una red,
    sea \(f\) un flujo en \(D\),
    y sea \(p\) un camino aumentable en \(D_f\).
    Defina la función
    \(f_p \colon V \times V \rightarrow \mathbb{R}\) mediante:
    \begin{equation*}
      f_p (u, v) =
      \begin{cases}
	c_f (p)	 & \text{si \(x y\) está en \(p\)} \\
	-c_f (p) & \text{si \(y x\) está en \(p\)} \\
	0	 & \text{caso contrario}
      \end{cases}
    \end{equation*}
    Entonces \(f_p\) es un flujo en \(D_f\)
    con valor\/ \(\val(f_p) = c_f(p) > 0\).
  \end{lemma}

  De lo anterior tenemos:
  \begin{corollary}
    \label{cor:f.prime}
    Sea \(D = (V, A)\) una red,
    \(f\) un flujo en \(D\)
    y \(p\) un camino aumentable en \(D_f\).
    Sea \(f_p\) como definido en el lema~\ref{lem:f_p},
    y defina \(f' \colon V \times V \rightarrow \mathbb{R}\)
    como \(f' = f + f_p\).
    Entonces \(f'\) es un flujo en \(D\),
    y su valor es:
    \begin{equation*}
      \val(f')
	= \val(f) + \val(f_p)
	> \val(f)
    \end{equation*}
  \end{corollary}
  \begin{proof}
    Inmediata por los lemas~\ref{lem:suma-flujos} y~\ref{lem:f_p}.
  \end{proof}

  Para clarificar estas ideas,
  véanse las figuras~\ref{fig:red} y~\ref{fig:red-flujos}.
  La figura~\ref{fig:red} muestra una red,
  \begin{figure}[htbp]
    \centering
    \pgfimage{images/red}
    \caption{Una red}
    \label{fig:red}
  \end{figure}
  \begin{figure}[htbp]
    \centering
    \subfloat[Un flujo]{
      \pgfimage{images/red-flujo}
      \label{subfig:red-flujo}
    }

    \subfloat[La red residual y un camino aumentable]{
      \pgfimage{images/red-residual}
      \label{subfig:red-residual}
    }
    \caption{Flujo y red residual
	     en la red de la figura~\ref{fig:red}}
    \label{fig:red-flujos}
  \end{figure}
  la figura~\ref{subfig:red-flujo}
  muestra un flujo en la red de la figura~\ref{fig:red}.
  La figura~\ref{subfig:red-residual}
  muestra la red residual con ese flujo
  con un camino aumentable marcado.
  Hay otros caminos aumentables,
  como \((s, a, b, c, t)\).
  Nótese que en la red residual tenemos enlaces contracorriente
  cuyas capacidades son el flujo actual.
  Esto significa que podemos enviar hasta ese flujo contracorriente
  (disminuyendo el flujo actual)
  a través de ese enlace.
  \begin{figure}[htbp]
    \centering
    \pgfimage{images/red-flujo-2}
    \caption{El flujo aumentado según el camino aumentable
	     de la figura~\ref{subfig:red-residual}}
    \label{fig:red-flujo-aumentado}
  \end{figure}
  Este camino aumentable tiene capacidad residual~\(4\),
  y la figura~\ref{fig:red-flujo-aumentado}
  muestra el resultado de aumentar el flujo.

\subsection{Cortes}
\label{sec:cortes}
\index{red!corte}

  La idea del método de Ford-Fulkerson
  es hallar sucesivamente un camino aumentable
  y aumentar el flujo a lo largo de él.
  Cuando este proceso termina
  por no hallar un camino aumentable,
  tenemos un flujo de valor máximo.
  Al discutir estos métodos
  se restringen las capacidades a números naturales.
  De partida,
  capacidades negativas no tienen sentido,
  y resulta que si las capacidades son irracionales
  hay casos en que los algoritmos no terminan nunca
  (convergen hacia la solución,
   pero nunca la alcanzan).

  La correctitud de estos métodos
  la garantiza
  el teorema \foreignlanguage{english}{Max-Flow Min-Cut},%
    \index{max-flow min-cut, teorema}
  que es nuestro próximo objetivo.
  Primeramente consideraremos el concepto de un \emph{corte}
  (en inglés \emph{\foreignlanguage{english}{cut}})
  en una red.
  Un corte \((S, T)\) corresponde simplemente
  a una partición de los vértices de la red
  en conjuntos \(S\) y \(T = V \smallsetminus S\)
  tal que \(s \in S\) y \(t \in T\),
  véase por ejemplo la figura~\ref{fig:red-cut}.
  \begin{figure}[htbp]
    \centering
    \pgfimage{images/red-cut}
    \caption{Un corte en la red de la figura~\ref{fig:red}
	     con el flujo de la figura~\ref{subfig:red-flujo}}
  \label{fig:red-cut}
  \end{figure}
  Si \(f\) es un flujo,
  el \emph{flujo neto} a través del corte \((S, T)\)
  se define como \(f(S, T)\).
  En nuestro caso (figura~\ref{fig:red-cut}) el flujo neto
  es \(9 - 4 + 7 = 12\).
  La \emph{capacidad} del corte \((S, T)\) es \(c(S, T)\),
  que en la misma figura correspondería a \(12 + 14 = 26\).
  También se define un \emph{corte mínimo}
  (en inglés \emph{\foreignlanguage{english}{minimum cut}})
  como un corte de capacidad mínima.
  El flujo neto que cruza el corte incluye flujos de \(S\) a \(T\)
  (aportes positivos)
  y flujos de \(T\) a \(S\)
  (aportes negativos).
  Por otro lado,
  en las capacidades se incluyen solo las de arcos de \(S\) a \(T\).
  El lema siguiente
  relaciona los flujos
  con las capacidades a través de cortes de la red.
  \begin{lemma}
    \label{lem:flujos-cortes}
    Sea \(f\) un flujo en la red \(D = (V, A)\)
    con fuente \(s\) y sumidero \(t\),
    y sea \((S, T)\) un corte de la red.
    Entonces el flujo neto a través del corte es el valor del flujo.
  \end{lemma}
  \begin{proof}
    Por conservación de flujo \(f(S - s, V) = 0\),
    y aplicando el lema~\ref{lem:identidades},
    tenemos:
    \begin{align*}
      f(S, T)
	&= f(S, V) - f(S, S) \\
	&= f(S, V) \\
	&= f(s, V) + f(S - s, V) \\
	&= f(s, V) \\
	&= \val(f)
      \qedhere
    \end{align*}
  \end{proof}
  Un resultado inmediato del lema~\ref{lem:flujos-cortes}
  es el resultado que demostramos antes,
  que el flujo al sumidero es el valor del flujo en la red:
  Basta tomar el corte \((V - t, \{t\})\) para ello.

  Pero también podemos deducir:
  \begin{corollary}
    \label{cor:flow<cut}
    En una red \(D\) con un corte \((S, T)\),
    el valor de cualquier flujo
    está acotado por la capacidad del corte.
  \end{corollary}
  \begin{proof}
    Sea \((S, T)\) un corte cualquiera
    de \(D\) y sea \(f\) un flujo.
    Por el lema~\ref{lem:flujos-cortes}
    y las restricciones de capacidad:
    \begin{align*}
      \val(f)
	& =   f(S, T) \\
	& =   \sum_{x \in S} \sum_{y \in T} f(x, y) \\
	& \le \sum_{x \in S} \sum_{y \in T} c(x, y) \\
	& =   c(S, T)
      \qedhere
    \end{align*}
  \end{proof}
  Una consecuencia inmediata del corolario~\ref{cor:flow<cut}
  es que el valor de un flujo está acotado
  por la capacidad de un corte mínimo.
  El teorema siguiente
  nos dice que el flujo máximo en realidad es esta cota.

  \begin{theorem}[Max-Flow Min-Cut]
    \index{max-flow min-cut, teorema|textbfhy}
    \label{theo:max-flow=min-cut}
    Si \(f\) es un flujo en la red \(D = (V, A)\)
    con fuente \(s\) y sumidero \(t\),
    entonces las siguientes son equivalentes:
    \begin{enumerate}[label=(\arabic{*})]
    \item\label{item:mfmc:f-max}
      \(f\) es un flujo máximo en \(D\).
    \item\label{item:mfmc:res}
      La red residual \(D_f\) no contiene
      caminos aumentables.
    \item\label{item:mfmc:val=cap}
      \(\val(f) = c(S, T)\) para algún corte \((S, T)\) de \(D\).
    \end{enumerate}
  \end{theorem}
  \begin{proof}
    Demostramos la equivalencia
    a través de un ciclo de implicancias.%
      \index{demostracion@demostración!ciclo de implicancias}
    \begin{description}
    \item[\boldmath
	  \ref{item:mfmc:f-max}~\(\implies\)~\ref{item:mfmc:res}:
	  \unboldmath]
      Por contradicción.
      Supongamos en contrario que \(f\) es máximo en \(D\),
      pero que \(D_f\)
      tiene un camino aumentable \(p\).
      Por el corolario~\ref{cor:f.prime}
      sabemos que \(f + f_p\) es un flujo,
      cuyo valor es mayor que \(\val(f)\),
      lo que contradice la suposición de que \(f\) es máximo.
    \item[\boldmath
	  \ref{item:mfmc:res}~\(\implies\)~\ref{item:mfmc:val=cap}:
	  \unboldmath]
      Supongamos que \(D_f\)
      no tiene camino aumentable,
      es decir,
      no hay camino dirigido de \(s\) a \(t\) en \(D_f\).
      Definamos:
      \begin{align*}
	S & = \{x \in V \colon
		  \text{\ hay un camino de \(s\) a \(x\)
			en \(D_f\)\}} \\
	T & = V \smallsetminus S
      \end{align*}
      Entonces \((S, T)\) es un corte de \(D\),
      ya que obviamente \(s \in S\) y \(t \notin S\)
      ya que no hay camino de \(s\) a \(t\) en \(D_f\)
      por suposición.
      Para cada par de vértices \(x \in S\) e \(y \in T\) tenemos
      \(f(x, y) = c(x, y)\),
      dado que de lo contrario \(x y \in A_f\) y habría un camino
      \(s \rightsquigarrow x \rightarrow y\)
      y así \(y\) estaría en \(S\).
      Por el lema~\ref{lem:flujos-cortes} es:
      \begin{equation*}
	\val(f)
	  = f(S, T)
	  = c(S, T)
      \end{equation*}
    \item[\boldmath
	  \ref{item:mfmc:val=cap}~%
	     \(\implies\)~\ref{item:mfmc:f-max}:
	  \unboldmath]
      Por el corolario~\ref{cor:flow<cut},
      \(\val(f) \le c(S, T)\) para todo corte \((S, T)\).
      La condición \(\val(f) = c(S, T)\)
      entonces asegura que el flujo es máximo.
    \qedhere
    \end{description}
  \end{proof}

  Este teorema sirve para demostrar
  la validez del método de Ford-Fulkerson:%
    \index{Ford-Fulkerson, metodo de@Ford-Fulkerson, método de}
  Si en una iteración
  no hay un camino aumentable,
  quiere decir que el flujo actual es máximo.
  Una manera razonable
  de buscar un camino aumentable
  es usar búsqueda a lo ancho en la red residual.
  A esta forma de implementar el método de Ford-Fulkerson
  se le conoce como el algoritmo de Edmonds-Karp~%
    \cite{edmonds72:_theor_improv_algor_effic_networ_flow_probl}.%
    \index{Edmonds-Karp, algorithmo de}
  Una discusión detallada del problema,
  incluyendo historia de los algoritmos,
  presenta Wilf~\cite[capítulo~3]{wilf03:_algor_compl}.

% Fixme: Agregar un ejemplo completo (p.ej. de ayudantía)
% Sugerido por Franco Castro <fcastro@alumnos.inf.utfsm.cl>

%%% Local Variables:
%%% mode: latex
%%% TeX-master: "clases"
%%% End:


% permutaciones.tex
%
% Copyright (c) 2009-2014 Horst H. von Brand
% Derechos reservados. Vea COPYRIGHT para detalles

\chapter{Permutaciones}
\label{cha:permutaciones}

  Informalmente,
  una permutación es un reordenamiento de un conjunto de objetos.
  Las permutaciones aparecen,
  en forma más o menos prominente,
  en casi todos los ámbitos de las matemáticas.
  Al tratar con conjuntos finitos es común estar interesados
  en las distintas formas de ordenarlos,
  posiblemente simplemente para considerar equivalentes
  los distintos órdenes.
  En el análisis de muchos algoritmos,
  particularmente los de ordenamiento,
  son características de las permutaciones
  lo que determina su rendimiento.

\section{Definiciones básicas}
\label{sec:permutaciones-def}

% Fixme: Revisar el material este en Shoup, GKP, TAoCP, Biggs.
% ¡Citarlos!
% Fixme: Aplicar teoría a análisis de algoritmos de ordenamiento
% simples

  Al reordenar elementos de un conjunto estamos definiendo
  una biyección entre la posición original y la nueva.
  Nos interesa estudiar estas biyecciones.
  \begin{definition}
    \index{permutacion@permutación|textbfhy}
    Una \emph{permutación} de un conjunto finito \(\mathcal{X}\)
    es una biyección de \(\mathcal{X}\) a \(\mathcal{X}\).
  \end{definition}
  Comúnmente usaremos \(\mathcal{X} = \{1, 2, \dotsc, n\}\)
  para concretar.
  Por ejemplo,
  una permutación típica de \(\{1, 2, \dotsc, 5\}\) es
  la función \(\alpha\) definida por la tabla:

  \begin{center}
    \begin{tabular}[c]{>{\(}l<{\)}@{\hspace{0.7em}}
		       *{4}{>{\(}c<{\)}@{\hspace{0.5em}}}>{\(}c<{\)}}
      & 1 & 2 & 3 & 4 & 5 \\
      \alpha & \downarrow & \downarrow & \downarrow & \downarrow
	     & \downarrow \\
      & 2 & 4 & 5 & 1 & 3
    \end{tabular}
  \end{center}
  Para abreviar,
  anotaremos una permutación dando
  el elemento al que va el de la posición indicada,
  o sea en este caso particular:
  \begin{equation*}
    \alpha
      = (2\;4\;5\;1\;3)
  \end{equation*}
  Viene a ser simplemente la última línea en lo anterior.

  Hay \(n!\)~permutaciones
  de un conjunto de \(n\) elementos%
    \index{permutacion@permutación!numero@número}
  (para crear la permutación \(\pi\)
   el valor de \(\pi(1)\) puede elegirse de \(n\) maneras,
   una vez elegido este
   quedan solo \(n - 1\) opciones para \(\pi(2)\),
   y así sucesivamente,
   y finalmente queda solo una opción para \(\pi(n)\)).

  Tomemos la anterior permutación \(\alpha\)
  y la permutación \(\beta = (3\;5\;1\;4\;2)\).
  La composición \(\beta \alpha\) se obtiene de aplicar \(\alpha\),
  luego \(\beta\),
  y resulta:
  \begin{equation*}
    \beta \alpha
      = (5\;4\;2\;3\;1)
  \end{equation*}
  En general,
  la composición no es conmutativa,
  en nuestro caso:
  \begin{equation*}
    \alpha \beta
      = (5\;3\;2\;1\;4)
  \end{equation*}
  y claramente \(\alpha \beta \ne \beta \alpha\).
  Cuidado,
  la convención general
  al operar con permutaciones es que las operaciones
  se efectúan de izquierda a derecha,
  no de derecha a izquierda como correspondería
  dado que la operación es composición de funciones.
  O sea,
  \(\alpha \beta \gamma\)
  se interpreta como \((\alpha \beta) \gamma\).
  En todo caso,
  dado que esto es un grupo,
  la operación es asociativa.%
    \index{operacion@operación!asociativa}

  \begin{theorem}
    \label{theo:permutaciones-grupo}
    \index{grupo}
    \index{permutacion@permutación!grupo}
    Las permutaciones cumplen las siguientes:
    \begin{enumerate}[label = (\roman*), ref = (\roman*)]
    \item
      \label{en:pg-1}
      Si \(\pi\) y \(\sigma\)
      son permutaciones de un conjunto de \(n\) elementos,
      lo es también \(\pi \sigma\).
    \item
      \label{en:pg-2}
      Para todas permutaciones \(\pi\), \(\sigma\), \(\tau\)
      de un conjunto,
      se cumple:
      \begin{equation*}
	(\pi \sigma) \tau = \pi (\sigma \tau)
      \end{equation*}
    \item
      \label{en:pg-3}
      La función identidad,
      que llamaremos \(\iota\),
      definida por \(\iota(r) = r\),
      es una permutación,
      y tenemos para toda permutación \(\sigma\):
      \begin{equation*}
	\iota \sigma = \sigma \iota = \sigma
      \end{equation*}
    \item
      \label{en:pg-4}
      Para toda permutación \(\pi\) de un conjunto dado
      hay una permutación inversa
      \(\pi^{-1}\) tal que:
      \begin{equation*}
	\pi \pi^{-1} = \pi^{-1} \pi = \iota
      \end{equation*}
    \end{enumerate}
  \end{theorem}
  \begin{proof}
    La propiedad~\ref{en:pg-1}
    sigue directamente de que la composición de biyecciones
    es una biyección,~%
    \ref{en:pg-2}
    es una propiedad básica de la composición de funciones,~%
    \ref{en:pg-3} es obvio,
    y~\ref{en:pg-4}
    es simplemente que toda biyección tiene una inversa.
  \end{proof}

  El teorema~\ref{theo:permutaciones-grupo}
  equivale a decir que las permutaciones forman un grupo
  con la operación de composición.
  Al grupo de permutaciones de \(n\) elementos
  se le llama el \emph{grupo simétrico de orden \(n\)},%
    \index{grupo!simetrico@simétrico|textbfhy}
  que se anota \(\mathtt{S}_n\).
  Note eso sí que el orden de \(\mathtt{S}_n\) es \(n!\),%
    \index{grupo!orden}
  a diferencia de lo que parece indicar su nombre.

  Es conveniente una notación compacta
  para permutaciones individuales.
  Consideremos nuevamente la permutación \(\alpha\).
  Es \(\alpha(1) = 2\), \(\alpha(2) = 4\), \(\alpha(4) = 1\).
  Esto forma un \emph{ciclo}
  (\(1 \rightarrow 2 \rightarrow 4 \rightarrow 1\))
  de largo \(3\).
  Similarmente,
  \(3\) y \(5\) forman un ciclo de largo \(2\),
  y podemos escribir \(\alpha\) en \emph{notación de ciclos}
  como \(\alpha = (1\;2\;4) (3\;5)\).

  Toda permutación \(\pi\) se puede escribir
  en notación de ciclos mediante el siguiente algoritmo:%
    \index{permutacion@permutación!notacion ciclo@notación ciclo|textbfhy}
  \begin{itemize}
  \item
    Comience con algún símbolo
    (digamos \(1\)),
    y trace el efecto de \(\pi\) sobre él y sus sucesores
    hasta que nuevamente encontremos \(1\),
    así tenemos un ciclo.
  \item
    Tome un símbolo cualquiera que no haya sido considerado aún,
    y construya el ciclo que lo contiene de la misma forma.
  \item
    Continúe de la misma forma
    hasta haber dado cuenta de todos los símbolos.
  \end{itemize}
  Podemos elegir cualquiera de los elementos
  de cada ciclo como primero --
  por ejemplo,
  \((7\;8\;2\;1\;3)\) es lo mismo que \((1\;3\;7\;8\;2)\).
  Por otro lado,
  podemos reordenar los ciclos --
  por ejemplo,
  \((1\;2\;4)(3\;5)\) y \((3\;5) (1\;2\;4)\)
  corresponden a la misma permutación,
  ya que al operar los ciclos en esta representación
  sobre elementos diferentes los ciclos conmutan.
  Lo importante son el número de ciclos,
  sus largos,
  y la disposición de sus elementos.
  Se adopta la convención
  de comenzar cada ciclo con su mínimo elemento,
  y luego ordenar los ciclos en orden de elemento mínimo.

  Para ver que la representación es única,
  debemos mostrar una biyección entre permutaciones y ciclos.
  Si anotamos los ciclos siempre con el máximo elemento al comienzo,
  y listamos los ciclos en orden de máximo elemento creciente,
  tomamos una permutación cualquiera
  podemos interpretarla como escrita en notación de ciclos:
  Los comienzos de cada ciclo son los máximos hasta ese punto.
  Esto constituye una biyección,%
    \index{biyeccion@biyección}
  y toda permutación puede escribirse en ciclos de una única forma.

  Por ejemplo,
  para la permutación \(\beta\) dada anteriormente,
  la notación de ciclos es \(\beta = (1\;3) (2\;5) (4)\).
  Acá 4 forma un ciclo ``degenerado'' por sí mismo,
  ya que \(\beta(4) = 4\).
  Simplemente omitiremos estos ciclos de largo \(1\),
  ya que corresponden a símbolos
  que no son afectados por la permutación.
  En todo caso,
  es mejor no omitirlos
  hasta que uno se haya familiarizado con la notación.
  Puede considerarse la notación de ciclos
  como el producto de las permutaciones con esos ciclos
  (los demás elementos permanecen fijos).
  O sea, en nuestro caso:
  \begin{equation*}
    \beta
      = (1\;3) (2) (4) (5) \cdot (1) (2\;5) (3) (4)
  \end{equation*}
  Esto es consistente
  con la convención de omitir ciclos de largo uno.

  \begin{example}
    Se tienen cartas numeradas \(1\) a \(12\),
    que se reparten en filas
    como muestra la figura~\ref{subfig:cartas-1}.
    \begin{figure}[htbp]
      \centering
      \subfloat[Original]{
	\begin{tabular}[c]{*{3}{>{\(}r<{\)}}}
	   1 &	2 &  3 \\
	   4 &	5 &  6 \\
	   7 &	8 &  9 \\
	  10 & 11 & 12
	\end{tabular}
	\label{subfig:cartas-1}
      }%
      \qquad%
      \subfloat[Columnas]{
	\begin{tabular}[c]{*{3}{>{\(}r<{\)}}}
	   1 &	4 &  7 \\
	  10 &	2 &  5 \\
	   8 & 11 &  3 \\
	   6 &	9 & 12
	\end{tabular}
	\label{subfig:cartas-2}
      }
      \caption{Ordenamiento de cartas}
      \label{fig:cartas}
    \end{figure}
    Luego se toman las cartas por filas
    y se distribuyen en columnas,
    quedando como lo muestra la figura~\ref{subfig:cartas-2}.
    ¿Cuántas veces hay que repetir esta operación
    hasta obtener nuevamente el orden original?

    Sea \(\pi\) la permutación que corresponde a esta operación,
    lo que buscamos es el orden de \(\pi\) en \(\mathtt{S}_{12}\).
    Esta manera de verlo
    nos dice que el problema tiene solución,
    cosa que no es obvia:
    Perfectamente podría ocurrir
    que saliendo de la configuración inicial
    ya no haya manera de volver a ella.
    En notación de ciclos
    tenemos \(\pi = (1) (2\;4\;10\;6\;5) (3\;7\;8\;11\;9) (12)\).
    Las cartas \(1\) y \(12\) no cambian de posición,
    las demás forman dos ciclos de largo \(5\).
    Al repetir esta operación \(5\) veces
    todas las cartas quedan en sus posiciones originales.
  \end{example}

  \begin{example}
    En la Prisión de Dunwich hay \(100\)~prisioneros.
    En su aburrimiento al alcaide se le ocurre un jueguito:
    Toma \(100\)~cajas,
    las rotula con los números de los prisioneros
    y en cada una coloca un número de prisionero al azar.
    Luego ofrece a los prisioneros lo siguiente:
    Cada uno puede elegir \(50\)~de las cajas,
    si alguno no encuentra su número
    los llevará a todos a un paseo a las colinas de Vermont.

    Si se considera que cada prisionero abre \(50\)~cajas al azar,
    es claro que hallará su número la mitad de las veces,
    y por tanto la probabilidad que todos encuentren los suyos
    es \(2^{-100}\),
    verdaderamente microscópico.

    Sin embargo,
    hay un algoritmo que asegura una razonable probabilidad
    de evitar una suerte horrorosa:
    Cada cual abre la caja con su número,
    luego la caja con el número que halló en la primera,
    y así sucesivamente hasta agotar las~\(50\) o hallar el suyo.
    Lo que está haciendo cada prisionero
    es trazar un ciclo de la permutación,
    interesa entonces saber
    cuántas permutaciones de \(100\)~elementos
    tienen ciclos de largo mayor a~\(50\).
    Como son \(100\)~prisioneros,
    una permutación
    puede tener a lo más un ciclo de los largos de interés,
    por lo que contar el número de permutaciones con estos ciclos
    es simplemente contar el número total de tales ciclos
    en permutaciones de \(100\) elementos.
    Si consideramos un ciclo de \(r\) elementos,
    estos podemos elegirlos
    de \(\binom{100}{r}\)~maneras entre los 100,
    y podemos ordenarlos de \((r - 1)!\)~formas en un ciclo.
    Los \(100 - r\)~elementos restantes
    se pueden organizar de \((100 - r)!\)~maneras,
    con lo que el número de permutaciones
    con un ciclo de largo \(r\) son:
    \begin{equation*}
      \binom{100}{r} (r - 1)! (100 - r)!
	= \frac{100!}{r}
    \end{equation*}
    Vale decir,
    exactamente \(1\) en \(r\)
    permutaciones tienen un ciclo de largo \(r\)
    cuando \(r > 50\).
    Nos interesa entonces la proporción:
    \begin{equation*}
      \sum_{51 \le r \le 100} \frac{1}{r}
	= H_{100} - H_{50}
	\approx 0,6882
    \end{equation*}
    Tienen un poco más de \(31\)\% de probabilidades
    de salvarse.
  \end{example}

  Si un ciclo se compone consigo mismo,
  el primer elemento del ciclo termina en el tercer lugar,
  y así sucesivamente.
  Para que el primer elemento de un ciclo de largo \(k\)
  vuelva a su posición original
  debe elevarse a la potencia \(k\).
  Para que todos los elementos de una permutación
  de tipo \([1^{\alpha_1} 2^{\alpha_2} \dotso n^{\alpha_n}]\)
  vuelvan por primera vez a sus posiciones originales
  (lo que determina el orden de la permutación)%
    \index{permutacion@permutación!orden}
  deben volver a sus posiciones originales
  todos los elementos de todos los ciclos.
  Esto es el mínimo común múltiplo de los largos de los ciclos.
  No importa cuántos ciclos de cada largo hay,
  solo si de los largos respectivos hay o no ciclos.

  \begin{definition}
    \index{permutacion@permutación!involucion@involución}
    \index{involucion@involución|see{permutación!involución}}
    Una permutación \(\tau\)
    que es su propio inverso se llama \emph{involución}.
    \index{involucion@involución|textbfhy}
  \end{definition}

  Las involuciones son aquellas permutaciones que solo tienen ciclos
  de largo \(1\) y \(2\),
  quedan representadas
  por \(\MSet(\Cyc_{\le 2}(\mathcal{Z}))\).%
    \index{metodo simbolico@método simbólico}
  Como la función generatriz exponencial de un ciclo de largo \(r\)
  es simplemente \(z^r / r\),
  la función generatriz exponencial \(\widehat{I}(z)\)
  para el número de involuciones de \(n\) elementos es:
  \begin{equation}
    \label{eq:involution-egf}
    \index{involucion@involución!funcion generatriz@función generatriz}
    \widehat{I}(z)
      = \exp\left(
	      z + z^2 / 2
	    \right)
  \end{equation}
  Prácticamente indoloro.

  Un desarreglo%
    \index{desarreglo}
  (ver sección~\ref{sec:desarreglos}
   y también el capítulo~\ref{cha:pie})
  es simplemente una permutación
  que no tiene ciclos de largo uno.
  La expresión simbólica correspondiente%
    \index{metodo simbolico@método simbólico}
  es \(\MSet(\Cyc_{\ge 2}(\mathcal{Z}))\),
  que podemos expresar
    \(\MSet(\Cyc(\mathcal{Z}) - \Cyc_{= 1}(\mathcal{Z}))\),
  la función generatriz exponencial que resulta es:%
    \index{generatriz!exponencial}
  \begin{equation*}
    \widehat{D}(z)
      = \exp ( -\ln (1 - z) - z )
      = \frac{\mathrm{e}^{-z}}{1 - z}
  \end{equation*}
  De acá nuevamente obtenemos
  el número de desarreglos de \(n\) elementos
  como \(D_n = n! \exp\lvert_n (-1)\).

  En el ejemplo de la prisión de Dunwich
  buscábamos el número de permutaciones
  con ciclos de largo mayor a \(50\).
  Este tipo de problemas puede atacarse
  marcando las subestructuras de interés.%
    \index{metodo simbolico@método simbólico}
  Usamos la clase \(\mathcal{U}\) para marcar,
  con un único elemento \(\upsilon\) de tamaño uno,
  y usamos la letra \(u\) en funciones generatrices.
  Si queremos saber cuántos ciclos de largo \(r\) hay en total
  en permutaciones de \(n\) elementos,
  consideraremos:
  \begin{equation*}
    \MSet(\Cyc_{\ne r}(\mathcal{Z})
	      + \mathcal{U} \times \Cyc_{= r}(\mathcal{Z}))
  \end{equation*}
  Abusando de la notación,
  escribimos:
  \begin{equation*}
    \MSet(\Cyc_{\ne r}(\mathcal{Z})
	      + u \Cyc_{= r}(\mathcal{Z}))
      = \MSet(\Cyc(\mathcal{Z}) + (u - 1) \Cyc_{= r}(\mathcal{Z}))
  \end{equation*}
  Tenemos directamente:
  \begin{equation*}
    C_r(z, u)
      = \exp \left( - \ln (1 - z) + (u - 1) \, \frac{z^r}{r}\right)
      = \frac{\mathrm{e}^{(u - 1) z^r / r}}{1 - z}
  \end{equation*}
  El número de ciclos de largo \(r\) es el exponente de \(u\),
  que podemos extraer derivando y haciendo \(u = 1\),
  lo que da la función generatriz exponencial
  para el número total de ciclos de largo \(r\)
  en permutaciones de \(n\) elementos:
  \begin{equation*}
    C_r(z)
      = \left.
	  \frac{\partial}{\partial u} \, C_r(z, u)
	\right\rvert_{u = 1}
      = \frac{z^r}{r} \, \frac{1}{1 - z}
  \end{equation*}
  De acá podemos obtener el número promedio de ciclos de largo \(r\)
  en las permutaciones de \(n\)~elementos,
  al haber \(n!\)~permutaciones
  y ser una función generatriz exponencial
  es directamente el coeficiente de \(z^n\):
  \begin{equation*}
    [ z^n ] \frac{z^r}{r} \, \frac{1}{1 - z}
      = \frac{1}{r}
  \end{equation*}
  Esto ya lo habíamos obtenido en el ejemplo,
  y al discutir desarreglos en el capítulo~\ref{cha:pie}%
    \index{desarreglo}
  vimos que para \(r = 1\) es \(1\),
  pero esta técnica es mucho más general.
% Fixme: ¿Análisis p.ej.de bubblesort, selección?

  A cada permutación \(\pi\) en \(\mathtt{S}_n\)
  corresponde una partición de los \(n\) elementos
  en los elementos de sus ciclos.
  Al número de ciclos de cada largo de la permutación
  le llamaremos su \emph{tipo}.%
    \index{permutacion@permutación!tipo|textbfhy}
  O sea,
  si \(\pi\) tiene \(\alpha_i\) ciclos de largo \(i\)
  (\(1 \le i \le n\)),
  el tipo de \(\pi\) lo anotamos
  \([1^{\alpha_1} 2^{\alpha_2} \dotso n^{\alpha_n}]\).
  Generalmente omitiremos los factores con \(\alpha_i = 0\)
  (largos para los cuales no hay ciclos).

  \begin{definition}
    \label{def:permutacion-conjugada}
    \index{permutacion@permutación!conjugada|textbfhy}
    Si hay una permutación \(\sigma\)
    tal que \(\sigma \alpha \sigma^{-1} = \beta\)
    se dice que \(\beta\) es \emph{conjugada} de \(\alpha\).
  \end{definition}
  El teorema siguiente es básico
  en la teoría algebraica de permutaciones.
  \begin{theorem}
    Dos permutaciones \(\alpha\) y \(\beta\) son conjugadas
    si y solo si tienen el mismo tipo.
  \end{theorem}
  \begin{proof}
    Demostramos implicancias en ambas direcciones.

    Si \(\alpha\) y \(\beta\) son permutaciones conjugadas,
    hay \(\sigma\)
    tal que podemos escribir \(\sigma \alpha \sigma^{-1} = \beta\).
    Tomemos un ciclo \((x_1, x_2, \dotsc, x_r)\) de \(\alpha\),
    vale decir,
    \(\alpha(x_1) = x_2\),
    \(\alpha(x_2) = x_3\), \ldots,
    \(\alpha(x_{r - 1}) = x_r\), \(\alpha(x_r) = x_1\).
    Definamos \(y_i = \sigma(x_i)\) para \(1 \le i \le r\).
    Tenemos,
    tomando los índices módulo \(r\):
    \begin{equation*}
      \beta(y_i)
	= \sigma \alpha \sigma^{-1}(\sigma(x_i))
	= \sigma \alpha (x_i)
	= \sigma(x_{i + 1})
	= y_{i + 1}
    \end{equation*}
    Esto es una biyección entre el ciclo de \(\alpha\)
    que contiene a \(x_1\)
    y el ciclo de \(\beta\) que contiene a \(y_1\),
    y habrán biyecciones similares
    entre los demás ciclos de \(\alpha\) y \(\beta\),
    que así tienen el mismo tipo.

    Por otro lado,
    supongamos que \(\alpha\) y \(\beta\) tienen el mismo tipo.
    Como tienen el mismo número de ciclos de cada uno de los largos,
    podemos construir una biyección
    entre ciclos del mismo largo de \(\alpha\) y \(\beta\),
    digamos \((x_1, x_2, \dotsc, x_r)\) en \(\alpha\)
    corresponde a \((y_1, y_2, \dotsc, y_r)\) en \(\beta\).
    Definiendo \(\sigma(x_i) = y_i\) para este ciclo,
    y de forma similar para los demás ciclos,
    tenemos \(\sigma \alpha \sigma^{-1} = \beta\) ya que
    (considerando los índices módulo el largo del ciclo)
    tenemos:
    \begin{equation*}
      \sigma \alpha \sigma^{-1}(y_i)
	= \sigma \alpha (x_i)
	= \sigma (x_{i + 1})
	= y_{i + 1}
	= \beta(y_i)
      \qedhere
    \end{equation*}
  \end{proof}

  Una manera fundamental de clasificar permutaciones%
    \index{permutacion@permutación!clasificacion@clasificación}
  viene de considerarlas como composición de \emph{transposiciones},
  que son permutaciones que solo intercambian dos elementos.%
    \index{permutacion@permutación!transposicion@transposición|textbfhy}
    \index{transposicion@transposición|see{permutación!transposición}}%
    \glossary{Transposición}
	     {Una permutación que solo intercambia dos elementos}

  Una permutación es una transposición
  si es del tipo \([1^{n - 2} 2^1]\)
  (tiene \(n - 2\) ciclos de largo \(1\) y un ciclo de largo \(2\)).
  Ahora bien,
  el ciclo \((x_1 \, x_2 \dotso x_r)\),
  transforma \((x_1 \, x_2 \dotso x_r)\)
  en \((x_2 \, x_3 \dotso x_r \, x_1)\),
  y este mismo efecto se logra intercambiando \(x_1\) con \(x_2\),
  luego \(x_2\)
  (que ahora está en la posición \(x_1\))
  con \(x_3\),
  y así sucesivamente.
  Así podemos escribir
  \((x_1 \, x_2 \dotso x_r)
      = (x_1 \, x_r) \dotso (x_1 \, x_3) (x_1 \, x_2)\).
  Como toda permutación se puede descomponer en ciclos,
  toda permutación puede expresarse en términos de transposiciones.
  Por ejemplo,
  aplicando la idea indicada arriba a cada uno de los ciclos:
  \begin{equation*}
    (1\;3\;6)(2\;4\;5\;7) = (1\;6) (1\;3) (2\;7) (2\;5) (2\;4)
  \end{equation*}
  Las transposiciones se pueden traslapar,
  un elemento puede moverse más de una vez.
  Esta representación no es única,
  además de reordenar las transposiciones
  podemos usar un conjunto completamente diferente de estas,
  como:
  \begin{equation*}
    (1\;3\;6)(2\;4\;5\;7)
      = (1\;5) (3\;5) (3\;6) (5\;7) (1\;4) (2\;7) (1\;2)
  \end{equation*}
  Sin embargo,
  hay una característica común entre estas descomposiciones.
  Sea \(c(\pi)\) al número total de ciclos de \(\pi\).
  Si \(\pi\)
  es de tipo \([1^{\alpha_1} 2^{\alpha_2} \dotso n^{\alpha_n}]\),
  entonces \(c(\pi) = \alpha_1 + \alpha_2 + \dotsb + \alpha_n\).
  Si combinamos \(\pi\) con una transposición \(\tau\),
  dando \(\tau \pi\),
  interesa determinar
  la relación entre \(c(\pi)\) y \(c(\tau \pi)\).
  Si \(\tau\) intercambia \(a\) con  \(b\),
  tenemos \(\tau(a) = b\), \(\tau(b) = a\)
  y \(\tau(k) = k\) si \(k \ne a, b\).
  Cuando \(a\) y \(b\) están en el mismo ciclo de \(\pi\),
  podemos escribir:
  \begin{equation*}
    \pi = (a \, x \dotso y \, b \dotso z) \text{\ y otros ciclos}
  \end{equation*}
  Veamos \(\tau \pi\).
  Como \(\tau\) intercambia \(a\) con \(b\),
  los ciclos de \(\pi\) que no los incluyen no cambian.
  Sean \(\pi(a) = x\), \(\pi(y) = b\), \(\pi(z) = a\);
  con lo que
  \(\tau \pi(a) = \tau(x) = x\) y \(\tau \pi(y) = \tau(b) = a\).
  De la misma forma \(\tau \pi(b) = \pi(b)\), \ldots,
		\(\tau \pi(z) = \tau(a) = b\).
  Se corta el ciclo y resulta:
  \begin{equation*}
    \tau \pi
      = (a \, x \dotso y) (b \dotso z) \text{\ y los otros ciclos}
  \end{equation*}
  con lo que \(c(\tau \pi) = c(\pi) + 1\).
  Por otro lado,
  si \(a\) y \(b\) pertenecen a ciclos distintos,
  o sea podemos escribir:
  \begin{equation*}
    \pi
      = (a \, x \dotso y) (b \dotso z) \text{\ y otros ciclos}
  \end{equation*}
  vemos que
    \(\tau \pi(y) = \tau(a) = b\) y \(\tau \pi(z) = \tau(b) = a\),
  los ciclos se funden:
  \begin{equation*}
    \tau \pi
      = (a \, x \dotso y \, b \dotso z) \text{\ y los otros ciclos}
  \end{equation*}
  con lo que \(c(\tau \pi) = c(\pi) - 1\).
  En resumen,
  seguir una permutación por una transposición
  cambia el número de ciclos en \(1\),
  y tenemos:
  \begin{theorem}
    \label{theo:permutations-transposiciones}
    Supóngase que la permutación \(\pi\) de \(\mathtt{S}_n\)
    puede escribirse como la composición de \(r\) transposiciones
    y también como la composición de \(r'\) transposiciones.
    Entonces \(r\) y \(r'\) son ambos pares o ambos impares.
  \end{theorem}
  \begin{proof}
    Sea \(\pi = \tau_r \tau_{r - 1} \dotso \tau_1\),
    con \(\tau_i\) (\(1 \le i \le r\)) una transposición.
    Como \(\tau_1\) es un ciclo de largo \(2\)
    y \(n - 2\) ciclos de largo \(1\),
    es \(c(\tau_1) = 1 + (n - 2) = n - 1\).
    Combinando \(\tau_1\)
    con \(\tau_2\), \(\tau_3\), \ldots, \(\tau_r\)
    finalmente se obtiene \(\pi\).
    En cada paso \(c\) aumenta o disminuye en \(1\).
    Sea \(g\) el número de veces que aumenta
    y \(h\) el número de veces que disminuye,
    El número final de ciclos será \(c(\pi) = (n - 1) + g - h\).
    Pero tenemos \(g + h = r - 1\),
    con lo que:
    \begin{align*}
      r &= 1 + g + h \\
	&= 1 + g + (n - 1 + g - c(\pi)) \\
	&= n - c(\pi) + 2 g
    \end{align*}
    De la misma forma,
    para cualquier otra manera de expresar \(\pi\)
    como producto de \(r'\) transposiciones
    hay un entero \(g'\) tal que \(r' = n - c(\pi) + 2 g'\),
    y \(r - r' = 2 (g - g')\),
    que es par.
  \end{proof}
  En vista del teorema~\ref{theo:permutations-transposiciones}
  podemos clasificar las permutaciones
  en \emph{pares} o \emph{impares}%
    \index{permutacion@permutación!par}%
    \index{permutacion@permutación!impar}
  dependiendo del número de transposiciones que las conforman.
  Definimos el \emph{signo} de la permutación \(\rho\),%
    \index{permutacion@permutación!signo}
  escrito \(\sgn \rho\),
  como	\(+1\) si \(\rho\) es par,
  y \(-1\) si es impar:
  \begin{equation*}
    \sgn \rho = (-1)^r
  \end{equation*}
  donde \(\rho\) es la composición de \(r\) transposiciones.
  En particular,
  \(\sgn \iota = (-1)^0 = +1\).
  Claramente,
  si \(\rho\) se puede descomponer en \(r\) transposiciones
  y \(\sigma\) se puede descomponer en \(s\) transposiciones,
  entonces \(\rho \sigma\)
  se puede descomponer en \(r + s\) transposiciones:
  \begin{align*}
    \sgn (\rho \sigma)
       &= (-1)^{r + s} \\
       &= (-1)^r \cdot (-1)^s \\
       &= \sgn \rho \cdot \sgn \sigma
  \end{align*}
  Como \(\rho \rho^{-1} = \iota\),
  resulta \(\sgn \rho = \sgn \rho^{-1}\).

  Recordando nuestra técnica para escribir
  un ciclo de largo \(k\) como \(k - 1\) transposiciones,
  tenemos un algoritmo simple
  para determinar el signo de una permutación:
  Descompóngala en ciclos,
  la paridad de la permutación
  es simplemente la del número de ciclos de largo par.%
    \index{permutacion@permutación!signo!determinar}

  \begin{theorem}
    Para todo entero \(n \ge 2\)
    exactamente la mitad
    de las permutaciones de \(\mathtt{S}_n\) son pares
    y la otra mitad impares.
  \end{theorem}
  \begin{proof}
    Sea \(\pi_1\), \(\pi_2\), \ldots, \(\pi_k\)
    la lista de permutaciones pares
    en \(\mathtt{S}_n\).
    Esta lista no es vacía,
    ya que \(\iota\) es par.

    Sea \(\tau\) una transposición cualquiera en \(S_n\),
    por ejemplo \(\tau = (1\;2)\).
    Entonces \(\tau \pi_1\), \(\tau \pi_2\), \ldots, \(\tau \pi_k\)
    son todas distintas,
    ya que si \(\tau \pi_i = \tau \pi_j\)
    por asociatividad e inverso en el grupo:
    \begin{equation*}
      \pi_i = (\tau^{-1} \tau) \pi_i
	    = \tau^{-1} (\tau \pi_i)
	    = \tau^{-1} (\tau \pi_j)
	    = (\tau^{-1} \tau) \pi_j
	    = \pi_j
    \end{equation*}
    Aún más,
    todas estas permutaciones son impares,
    ya que:
    \begin{equation*}
      \sgn (\tau \pi_i)
	 = \sgn \tau \cdot \sgn \pi_i
	 = (-1) \cdot (+1)
	 = -1
    \end{equation*}
    Finalmente,
    toda permutación impar \(\rho\)
    es de una de las \(\tau \pi_i\) (\(1 \le i \le k\)),
    ya que:
    \begin{equation*}
      \sgn (\tau^{-1} \rho)
	 = \sgn \tau^{-1} \cdot \sgn \rho
	 = (-1) \cdot (-1)
	 = +1
    \end{equation*}
    con lo que \(\tau^{-1} \rho\) es par,
    y debe ser \(\pi_i\)
    para algún \(i\) en el rango \(1 \le i \le k\),
    y así \(\rho = \tau \pi_i\).
    Con esto tenemos una biyección
    entre las permutaciones pares e impares.
  \end{proof}
  El conjunto de permutaciones pares
  es un subgrupo de \(\mathtt{S}_n\)
  (\(\mathtt{S}_n\) es finito,
   y este subconjunto es cerrado respecto de la composición),
  se le llama el \emph{grupo alternante}
  y se anota \(\mathtt{A}_n\).%
    \index{grupo!alternante|textbfhy}

  Esto tiene aplicación en muchas áreas,
  por ejemplo en ``matemáticas recreativas''.
  \begin{example}
    Ocho bloques marcados \(1\) a \(8\) se disponen
    en un marco cuadrado
    como se indica en la figura~\ref{subfig:bloques-1}.
    \begin{figure}[htbp]
      \centering
      \subfloat[Inicial]{
	\setlength{\tabcolsep}{3pt}
	\begin{tabular}{|*{3}{>{\(}c<{\)}|}}
	  \hline
	    \rule[-0.7ex]{0pt}{3ex}%
	  1 & 2 & 3 \\
	  \hline
	    \rule[-0.7ex]{0pt}{3ex}%
	  4 & 5 & 6 \\
	  \hline
	    \rule[-0.7ex]{0pt}{3ex}%
	  7 & 8 &   \\
	  \hline
	\end{tabular}
	\label{subfig:bloques-1}
      }%
      \qquad%
      \subfloat[Final]{
	\setlength{\tabcolsep}{3pt}
	\begin{tabular}{|*{3}{>{\(}c<{\)}|}}
	  \hline
	    \rule[-0.7ex]{0pt}{3ex}%
	  8 & 5 & 2 \\
	  \hline
	    \rule[-0.7ex]{0pt}{3ex}%
	  7 & 4 & 1 \\
	  \hline
	    \rule[-0.7ex]{0pt}{3ex}%
	  6 & 3 &   \\
	  \hline
	\end{tabular}
	\label{subfig:bloques-2}
      }
      \caption{¿Puede hacerse?}
      \label{fig:bloques}
    \end{figure}
    Una movida legal
    corresponde a deslizar un bloque al espacio vacío.
    Determine si es posible
    lograr la configuración de~\ref{subfig:bloques-2}.

    Si anotamos \(\openbox\) para el espacio,
    la configuración inicial
    es \(1\;2\;3\;4\;5\;6\;7\;8\;\openbox\),
    y la final solicitada es \(8\;5\;2\;7\;4\;1\;6\;3\;\openbox\).
    Claramente toda movida legal es una transposición.
    Como el espacio solo puede moverse en vertical u horizontal,
    para regresar a su posición original
    debe hacer un número par de movidas en cada dirección,
    y el número total de movidas es par.
    Luego la permutación que lleva de la configuración inicial
    a otra configuración posible bajo las reglas
    en la que \(\openbox\)
    está nuevamente en la esquina inferior derecha
    está conformada por un número par de transposiciones,
    o sea es par.
    La permutación entre la configuración final solicitada
    y la inicial es \((1\;8\;3\;2\;5\;4\;7\;6)(\openbox)\);
    el ciclo de largo \(8\) que aparece acá es impar,
    por lo que esto no puede hacerse.

    Nótese que esto sirve
    para demostrar lo que \emph{no} puede hacerse,
    no significa que todas las permutaciones pares puedan lograrse
    con estas restricciones.
  \end{example}

  Es simple contabilizar las permutaciones de un tipo particular.%
    \index{permutacion@permutación!tipo!numero@número}
  Supongamos,
  por ejemplo,
  que nos interesa saber cuántos elementos de \(\mathtt{S}_{14}\)
  tienen tipo \([2^2 3^2 4^1]\).
  Esto corresponde a poner
  los símbolos \(1\), \(2\), \ldots, \(14\) en el patrón:
  \begin{equation*}
    (.\;.) (.\;.) (.\;.\;.) (.\;.\;.) (.\;.\;.\;.)
  \end{equation*}
  y hay \(14!\)~formas
  de distribuir los \(14\) símbolos en las \(14\) posiciones.
  Sin embargo,
  muchas de estas dan la misma permutación.
  Considerando cada ciclo,
  podemos elegir el primer elemento de él
  y el resto quedará determinado por la permutación.
  Así hay \(2\)~maneras de obtener cada \(2\)\nobreakdash-ciclo,
  \(3\)~maneras de obtener cada \(3\)\nobreakdash-ciclo,
  y \(4\)~maneras de obtener cada \(4\)\nobreakdash-ciclo.
  El ordenamiento interno de cada ciclo se puede llevar a cabo
  de \(2^2 \cdot 3^2 \cdot 4\) formas en este caso.
  En general,
  si el tipo es \([1^{\alpha_1} 2^{\alpha_2} \dotso n^{\alpha_n}]\),
  habrán \(1^{\alpha_1} \cdot 2^{\alpha_2}
	     \cdot \dotsm \cdot n^{\alpha_n}\)~%
  ordenamientos internos equivalentes.
  Por otro lado,
  el orden de los ciclos del mismo largo es arbitrario,
  por lo que el número de reordenamientos en el caso general será~%
  \(\alpha_1 ! \, \alpha_2 ! \dotso \alpha_n !\),
  y el número de permutaciones
  de tipo \([1^{\alpha_1} 2^{\alpha_2} \dotso n^{\alpha_n}]\) es:
  \begin{equation*}
    \frac{n!}{1^{\alpha_1} 2^{\alpha_2} \dotso n^{\alpha_n} \,
	      \alpha_1 ! \, \alpha_2 ! \dotso \alpha_n !}
  \end{equation*}
  Esto se ve seductoramente similar a un término multinomial,
  pero debe recordarse que acá la condición es:
  \begin{equation*}
    \sum_k k \alpha_k = n
  \end{equation*}

% Fixme: Paulina Silva <pasilva@alumnos.inf.utfsm.cl>
%	 pide un ejemplo más aplicado

% Fixme: Análisis somero de bubblesort/inserción

%%% Local Variables:
%%% mode: latex
%%% TeX-master: "clases"
%%% End:


% coloreo-Polya.tex
%
% Copyright (c) 2011-2014 Horst H. von Brand
% Derechos reservados. Vea COPYRIGHT para detalles

\chapter{Teoría de coloreos de Pólya}
\label{cha:coloreo-Pólya}
\index{Polya, teoria de enumeracion de@Pólya, teoría de enumeración de}

  Los problemas combinatorios que hemos enfrentado hasta acá
  han sido relativamente sencillos.
  Sólo al considerar la construcción ciclo de objetos no rotulados
  (capítulo~\ref{cha:metodo-simbolico})%
    \index{metodo simbolico@método simbólico}
  tuvimos que considerar simetrías de los objetos bajo estudio.
  El teorema de enumeración de Pólya
  (abreviado \emph{PET},
   por \emph{\foreignlanguage{english}{Pólya Enumeration Theorem}})
  en realidad fue publicado en 1927 por John Howard Redfield,
  en 1937 George Pólya lo redescubrió independientemente
  y lo popularizó aplicándolo a muchos problemas de conteo,
  en particular de compuestos químicos~%
    \cite{polya37:_kombin_anzah_grupp_graph_verbin,
	  polya87:_combin_enumer_group_graph_chemic_compoun}.

\section{Grupos de permutaciones}
\label{sec:grupos-permutaciones}

  Desarrollaremos la teoría para la situación simple
  en la cual solo contamos coloreos.
  Una derivación más intuitiva
  (que puede ser útil para motivar el desarrollo presente)
  ofrece Tucker~\cite{tucker74:_Polya_enum_example}.
  \begin{definition}
    \index{grupo!permutaciones}
    Sea \(G\) un conjunto de permutaciones
    del conjunto finito \(\mathcal{X}\).
    Si \(G\) es un grupo
    (con la composición de permutaciones)
    decimos que
      \emph{\(G\) es un grupo de permutaciones de \(\mathcal{X}\)}.
  \end{definition}
  Si tomamos \(\mathcal{X} = \{1, 2, \dotsc, n\}\),
  entonces un grupo de permutaciones
  es simplemente un subgrupo de \(\mathtt{S}_n\).%
    \index{grupo!simetrico@simétrico}%
    \index{grupo!subgrupo}
  El cuadro~\ref{tab:subgrupos-S3}
  lista todos los subgrupos de \(\mathtt{S}_3\).
  \begin{table}[htbp]
    \centering
    \begin{tabular}{*{3}{>{\(}l<{\)}}}
      H_1 = \{\iota\}			       &
	 H_2 = \{\iota, (1\;2)\}	       &
	 H_3 = \{\iota, (1\;3)\}		   \\
      H_4 = \{\iota, (2\;3)\}		       &
	 H_5 = \{\iota, (1\;2\;3), (1\;3\;2)\} &
	 H_6 = \mathtt{S}_3
    \end{tabular}
    \caption{Los subgrupos de $\mathtt{S}_3$}
    \label{tab:subgrupos-S3}
  \end{table}
  El grupo alternante \(\mathtt{A}_3\)%
    \index{grupo!alternante}
  aparece como \(H_5\) en el cuadro.
  Para ver si un subconjunto de un grupo finito es un subgrupo,
  basta verificar si es cerrado
  por lo demostrado anteriormente para grupos.

  Otros ejemplos se obtienen
  como grupos de simetría de objetos geométricos.
  Por ejemplo,
  si rotulamos los vértices de un cuadrado
  en orden contrario a las manecillas del reloj
  (ver la figura~\ref{fig:cuadrado}),
  \begin{figure}[htbp]
    \centering
    \pgfimage{images/cuadrado}
    \\[3ex]
    \begin{tabular}[c]{l>{\(}l<{\)}}
      \hline
	\rule[-0.7ex]{0pt}{3ex}%
      Identidad					    & (1) (2) (3) (4) \\
      Rotación en \(\pi / 2\)			    & (1\;2\;3\;4)    \\
      Rotación en \(\pi\)			    & (1\;3) (2\;4)   \\
      Rotación en \(3 \pi / 2\)			    & (1\;4\;3\;2)    \\
      Reflexión en diagonal 1\;3		    & (2\;4)	      \\
      Reflexión en diagonal 2\;4		    & (1\;3)	      \\
      Reflexión en bisector perpendicular de 1\;2   & (1\;2) (3\;4)   \\
      Reflexión en bisector perpendicular de 1\;4   & (1\;4) (2\;3)   \\
      \hline
    \end{tabular}
    \caption{Un cuadrado y sus simetrías}
    \label{fig:cuadrado}
  \end{figure}
  cada simetría
  induce una permutación del conjunto \(\{1, 2, 3, 4\}\),
  y obtenemos las simetrías indicadas.
  Estas \(8\)~permutaciones
  forman el llamado \emph{grupo dihedral de orden \(8\)},%
    \index{grupo!dihedral}
  \(\mathtt{D}_8\).
  En general,
  las simetrías de un \(n\)\nobreakdash-ágono regular%
    \index{poligono@polígono}
  forman un grupo de \(2 n\)~elementos,
  el \emph{grupo dihedral de orden \(2 n\)}
  que se anota~\(\mathtt{D}_{2 n}\).
  Debe tenerse cuidado:
  Esta es la notación que se usa en álgebra,
  en geometría este mismo grupo se anota~\(\mathtt{D}_n\).
  Si solo se consideran las rotaciones
  de un polígono regular de \(n\) lados,
  tenemos el grupo cíclico de orden \(n\),%
    \index{grupo!ciclico@cíclico}
  anotado \(\mathtt{C}_n\),
  isomorfo a \(\mathbb{Z}_n\) con la suma.

  Una situación similar se produce cuando estudiamos grafos
  en vez de figuras geométricas,
  en donde las ``simetrías'' son permutaciones de los vértices
  que transforman arcos en arcos.
  A una permutación de este tipo
  se le llama \emph{automorfismo} del grafo%
    \index{grafo!automorfismo}
  (viene a ser un isomorfismo del grafo consigo mismo).%
    \index{grafo!isomorfismo}

  Un ejemplo de grafo es la figura~\ref{fig:automorfismos},
  interesa saber cuántos automorfismos tiene.
  \begin{figure}[htbp]
    \centering
    \pgfimage{images/automorfismo}
    \\[2ex]
    \begin{tabular}[c]{>{\(}l<{\)}@{ se extiende a }>{\(}l<{\)}@{\qquad}
		       >{\(}l<{\)}@{ se extiende a }>{\(}l<{\)}}
       \iota	 & \iota	       &
	  (1\;3) & (1\;3) (4\;6) \\
       (1\;3\;5) & (1\;3\;5) (2\;4\;6) &
	  (1\;5) & (1\;5) (2\;4) \\
       (1\;5\;3) & (1\;5\;3) (2\;6\;4) &
	  (3\;5) & (3\;5) (2\;6)
    \end{tabular}
    \caption{Un grafo de seis vértices y sus automorfismos}
    \label{fig:automorfismos}
  \end{figure}
  Primero observamos que los vértices del grafo
  caen naturalmente en dos grupos:
  Los vértices \(\{1, 3, 5\}\) son de grado \(4\),
  mientras \(\{2, 4, 6\}\) son de grado \(2\).
  Ningún automorfismo puede transformar
  un vértice del primer grupo en uno del segundo.
  Por otro lado,
  está claro
  que podemos tomar \emph{cualquier} permutación de \(\{1, 2, 3\}\)
  y extenderla a un automorfismo del grafo.
  Por ejemplo,
  si \((1\;3\;5)\) es parte de un automorfismo \(\alpha\),
  entonces \(\alpha\) tiene que transformar \(2\) en \(4\),
  ya que \(2\) es el único vértice adyacente a \(1\) y \(3\),
  y \(4\) es el único vértice adyacente
  a sus imágenes \(3\) y \(5\).
  De la misma forma,
  \(\alpha\) lleva \(4\) en \(6\) y \(6\) en \(2\),
  por lo que \(\alpha = (1\;3\;5) (2\;4\;6)\).
  En forma análoga,
  cada una de las seis permutaciones de \(\{1, 2, 3\}\)
  puede extenderse de forma única a un automorfismo del grafo,
  como muestra la misma figura~\ref{fig:automorfismos}.
  Hay exactamente seis automorfismos,
  que son las permutaciones listadas arriba.

\section{Órbitas y estabilizadores}
\label{sec:orbitas-estabilizadores}
\index{grupo!permutaciones!orbita@órbita}
\index{grupo!permutaciones!estabilizador}

  Sea \(G\) un grupo de permutaciones
  de un conjunto \(\mathcal{X}\).
  Veremos que la estructura del grupo
  lleva naturalmente a una partición de \(\mathcal{X}\).
  Definamos la relación \(\sim\) sobre \(\mathcal{X}\) mediante
  \(x \sim y\) siempre que para algún \(\gamma \in G\)
  tenemos \(\gamma(x) = y\).
  Verificamos que \(\sim\)
  es una relación de equivalencia de la forma usual:%
    \index{relacion@relación!equivalencia}
  \begin{description}
  \item[Reflexiva:]
    Como \(\iota\) es parte de todo grupo,
    y \(\iota(x) = x\) para todo \(x \in \mathcal{X}\),
    tenemos \(x \sim x\).
  \item[Simétrica:]
    Supongamos \(x \sim y\),
    o sea \(\gamma(x) = y\) para algún \(\gamma \in G\).
    Como \(G\) es un grupo,
    \(\gamma^{-1} \in G\),
    y como \(\gamma^{-1}(y) = x\),
    tenemos \(y \sim x\).
  \item[Transitiva:]
    Si \(x \sim y\) y \(y \sim z\)
    debe ser \(\gamma_1(x) = y\) y \(\gamma_2(y) = z\)
    para \(\gamma_1, \gamma_2 \in G\),
    y como \(G\) es un grupo,
    \(\gamma_2 \gamma_1 \in G\),
    con lo que \(\gamma_2 \gamma_1 (x) = z\)
    y \(x \sim z\).
  \end{description}
  Como \(\sim\) es relación de equivalencia,
  define una partición de \(\mathcal{X}\);
  \(x\) e \(y\) pertenecen a la misma clase
  si y solo si hay una permutación en \(G\)
  que transforma \(x\) en \(y\).
  A las clases de equivalencia
  se les conoce como las \emph{órbitas} de \(G\) en \(\mathcal{X}\).
  La \emph{órbita de \(x\)} es la clase que contiene a \(x\):
  \begin{equation*}
    G x
      = \{y \in \mathcal{X} \colon y = \gamma(x)
	       \text{\ para algún\ } \gamma \in G\}
  \end{equation*}

  Intuitivamente,
  la órbita \(G x\) son los elementos de \(\mathcal{X}\)
  que no se distinguen de \(x\) bajo operaciones de \(G\).
  En el caso de la figura~\ref{fig:automorfismos}
  los conjuntos de vértices \(\{1, 3, 5\}\) y \(\{2, 4, 6\}\)
  son órbitas del grupo.
  El grafo de la figura~\ref{fig:ejemplo-orbitas}
  tiene un grupo más complejo.
  \begin{figure}[htbp]
    \centering
    \pgfimage{images/orbitas}
    \vspace{2\baselineskip}

    \begin{tabular}[c]{>{\(}l<{\)}*{2}{@{\quad}>{\(}l<{\)}}}
      \iota  & \iota  & \iota		    \\
      (a\;b) & (d\;f) & (h\;i) (k\;l)	    \\
	     &	      & (h\;j) (k\;m)	    \\
	     &	      & (i\;j) (l\;m)	    \\
	     &	      & (h\;i\;j) (k\;l\;m) \\
	     &	      & (h\;j\;i) (k\;m\;l)
    \end{tabular}
    \caption{Un ejemplo de grafo
	     y los generadores de su grupo de automorfismos}
    \label{fig:ejemplo-orbitas}
  \end{figure}
  Acá los automorfismos
  se obtienen combinando las permutaciones
  de la figura~\ref{fig:ejemplo-orbitas}.
  Hay un total de \(2 \cdot 2 \cdot 6 = 24\) permutaciones
  en este grupo.
  Son órbitas
  \(\{a, b\}\), \(\{c\}\), \(\{d, f\}\), \(\{e\}\), \(\{g\}\),
  \(\{h, i, j\}\), \(\{k, l, m\}\),
  y se ve que ``son parecidos'' los elementos de cada una de ellas,
  en que las operaciones del grupo los intercambian.

  Las órbitas presentan un par de problemas numéricos obvios:
  ¿Cuántas órbitas hay?
  ¿Qué tamaños tienen?

  Si \(G\) es un grupo de permutaciones,
  llamaremos \(G(x \rightarrow y)\)
  al conjunto de permutaciones que llevan \(x\) a \(y\),
  o sea:
  \begin{equation*}
    G(x \rightarrow y) = \{\gamma \in G \colon g(x) = y\}
  \end{equation*}
  En particular,
  \(G(x \rightarrow x)\)
  es el conjunto de permutaciones
  que tienen a \(x\) como punto fijo.
  Este conjunto se llama el \emph{estabilizador} de \(x\),%
    \index{grupo!permutaciones!estabilizador}
  y se anota \(G_x\).
  Si \(\gamma_1\) y \(\gamma_2\) están en \(G_x\):
  \begin{equation*}
    \gamma_2 \gamma_1(x) = \gamma_2(x) = x
  \end{equation*}
  por lo que \(\gamma_2 \gamma_1 \in G_x\),
  y \(G_x\) es un subgrupo de \(G\).
  También tenemos:
  \begin{theorem}
    \label{theo:stabilizer-coset-l}
    Sea \(G\) un grupo de permutaciones,
    y sea \(\gamma \in G(x \rightarrow y)\).
    Entonces:
    \begin{equation*}
      G(x \rightarrow y) = \gamma G_x
    \end{equation*}
    el coset izquierdo de \(G_x\) respecto a \(\gamma\).
  \end{theorem}
  \begin{proof}
    Demostraremos que todo elemento de \(\gamma G_x\)
    pertenece a \(G(x \rightarrow y)\) y viceversa,
    con lo que ambos conjuntos son iguales.

    Si \(\alpha\) pertenece a \(\gamma G_x\),
    es \(\alpha = \gamma \beta\) para algún \(\beta \in G_x\).
    O sea,
    \(\alpha(x) = \gamma \beta(x) = \gamma(x) = y\),
    con lo que \(\alpha\) pertenece a \(G(x \rightarrow y)\).
    Por el otro lado,
    si \(\pi \in G(x \rightarrow y)\),
    entonces \(\gamma^{-1} \pi(x) = \gamma^{-1}(y) = x\),
    de manera que \(\gamma^{-1} \pi = \delta\),
    donde \(\delta \in G_x\),
    y así \(\pi = \gamma \delta \in \gamma G_x\).
    Ambos conjuntos son iguales.
  \end{proof}
  De forma muy similar al teorema~\ref{theo:stabilizer-coset-l}
  se demuestra lo siguiente:
  \begin{theorem}
    \label{theo:stabilizer-coset-r}
    Sea \(G\) un grupo de permutaciones de \(\mathcal{X}\),
    y sea \(\gamma \in G(x \rightarrow y)\).
    Entonces:
    \begin{equation*}
      G(x \rightarrow y) = G_y \gamma
    \end{equation*}
    el coset derecho de \(G_y\) respecto a \(\gamma\).
  \end{theorem}
  \begin{proof}
    Si \(\alpha\) pertenece a \(G_y \gamma\),
    es \(\alpha = \beta \gamma\) para algún \(\beta \in G_y\),
    vale decir
    \(
      \alpha(y)
	= \beta \gamma(y)
	= \beta(x)
	= y
    \)
    o sea \(\beta \in G(x \rightarrow y)\).
    Al revés,
    supongamos \(\pi \in G(x \rightarrow y)\),
    y consideremos
    \(
      \pi \gamma^{-1}(y)
	= \pi(x)
	= y
    \)
    y por tanto \(\pi \gamma^{-1} \in G_y\),
    o \(\pi \in G_y \gamma\),
    y sigue el resultado.
  \end{proof}

  De los anteriores teoremas obtenemos:
  \begin{corollary}
    \label{cor:G_x=G_y}
    Sea \(G\) un grupo de permutaciones de \(\mathcal{X}\),
    sea \(x \in \mathcal{X}\) un elemento cualquiera,
    e \(y\) un elemento en la órbita de \(x\).
    Entonces \(\lvert G_x \rvert = \lvert G_y \rvert\).
  \end{corollary}
  \begin{proof}
    Inmediato,
    ya que el tamaño de un coset es el orden del subgrupo
    (lo demostramos para el teorema de Lagrange);%
      \index{Lagrange, teorema de}
% Wreckme: Referencia al teorema cuando se unan ambos apuntes
    y por los teoremas~\ref{theo:stabilizer-coset-l}
    y~\ref{theo:stabilizer-coset-r} tenemos
    \(
      \lvert G_y \rvert
	= \lvert G(x \rightarrow y) \rvert
	= \lvert G_x \rvert
    \).
  \end{proof}

  \begin{theorem}
    \label{theo:GxG_x=G}
    Sea \(G\) un grupo de permutaciones de \(\mathcal{X}\),
    y sea \(x\) un elemento de \(\mathcal{X}\).
    Entonces:
    \begin{equation*}
      \lvert G x \rvert \cdot \lvert G_x \rvert
	= \lvert G \rvert
    \end{equation*}
  \end{theorem}
  \begin{proof}
    Usamos la idea de contar por filas y columnas.
    Para un elemento \(x \in \mathcal{X}\)
    el conjunto de pares
      \(S_x = \{(\gamma, y) \colon \gamma(x) = y\}\)
    puede describirse
    mediante una tabla como la del cuadro~\ref{tab:gamma-y}.
    \begin{table}[htbp]
      \centering
      \begin{tabular}[c]{>{\(}l<{\)}|>{\(}r<{\)}>{\(}c<{\)}
			 >{\(}l<{\)}|>{\(}l<{\)}}
		   & \cdots & y & \cdots		 &
			\\
	\hline
	\vdots	   &	    &	&			 &
			\\
	\gamma & \phantom{\text{si \(\gamma(x) = y\)}} &
	  \multicolumn{2}{l|}{\;\;\;\checkmark\
			      si \(\gamma(x) = y\)
			      \hspace*{1.5em}} &
	  r_\gamma(S_x) \\
	\vdots	   &	    &	&			 &
			\\
	\hline
		   &	     & c_y(S_x) &		 &
      \end{tabular}
      \caption{Pares $(\gamma, y)$
	       para demostración del teorema~\ref{theo:GxG_x=G}}
      \label{tab:gamma-y}
    \end{table}
    Dado que \(\gamma\) es una permutación,
    hay un único \(y\)
    tal que \(\gamma(x) = y\) para cada \(\gamma\),
    con lo que \(r_\gamma(S_x) = 1\).
    El total por columna \(c_y(S_x)\)
    es el número de permutaciones \(\gamma\)
    tales que \(\gamma(x) = y\),
    vale decir
    \(\lvert G(x \rightarrow y) \rvert\).
    Si \(y\) está en la órbita \(G x\),
    por el teorema~\ref{theo:stabilizer-coset-l}
    y el hecho que el coset de un subgrupo%
      \index{coset}
    tiene el tamaño del subgrupo,
    tenemos
    \(\lvert G(x \rightarrow y) \rvert = \lvert G_x \rvert\).
    Por otro lado,
    si \(y\) no está en la órbita \(G x\),
    \(\lvert G(x \rightarrow y) \rvert = 0\).
    Las dos formas de contar los elementos de \(S_x\) dan:
    \begin{equation*}
      \sum_{y \in \mathcal{X}} c_y (S_x)
	= \sum_{\gamma \in G} r_\gamma(S_x)
    \end{equation*}
    Al lado izquierdo hay \(\lvert G x \rvert\) términos
    que valen \(\lvert G_x \rvert\) cada uno,
    los demás valen \(0\);
    al lado derecho hay \(\lvert G \rvert\) términos
    que valen \(1\) cada uno.
    Así tenemos el resultado prometido.
  \end{proof}

  Este teorema permite calcular el tamaño de un grupo
  si se conoce el tamaño de una órbita
  y el estabilizador respectivo.
  Consideremos por ejemplo el grupo \(\mathtt{T}\)
  de rotaciones en el espacio de un tetraedro,%
    \index{poliedro!regular}%
    \index{tetraedro}
  ver la figura~\ref{fig:rotaciones-tetraedro-1}.
  \begin{figure}[htbp]
    \centering
    \pgfimage{images/tetraedro-vertice}
    \caption{Rotaciones de un tetraedro}
    \label{fig:rotaciones-tetraedro-1}
  \end{figure}
  Las rotaciones alrededor del eje marcado
  son las que mantienen fijo el vértice \(1\),
  y hay \(3\) de estas,
  \(\lvert \mathtt{T}_d \rvert = 3\).
  Por otro lado,
  girando el tetraedro en el espacio
  se puede colocar en la posición \(1\)
  cualquiera de los \(4\) vértices,
  y tenemos \(\lvert \mathtt{T} d \rvert = 4\).
  En consecuencia,
  el grupo de rotaciones en el espacio de un tetraedro es
  de orden
  \(\lvert \mathtt{T} \rvert
      = \lvert \mathtt{T}_d \rvert
	  \cdot \lvert \mathtt{T} d \rvert
      = 3 \cdot 4
      = 12\).
  Resulta que \(\mathtt{T}\)
  no es más que el grupo alternante \(\mathtt{A}_4\).%
    \index{grupo!alternante}

  Otro ejemplo lo da el icosaedro trunco,%
    \index{poliedro}%
    \index{icosaedro trunco}
  la forma básica de la pelota de fútbol tradicional,
  ver la figura~\ref{fig:icosaedro-trunco}.
  \begin{figure}[htbp]
    \centering
   \pgfimage[width=0.35\textwidth]
	     {images/20070219213318!Truncated_icosahedron}
    % http://upload.wikimedia.org/wikipedia/commons/archive/c/c3/20070219213318%21Truncated_icosahedron.png
    % Public domain
    \caption{Icosaedro trunco}
    \label{fig:icosaedro-trunco}
  \end{figure}
  Este es el sólido arquimedeano%
    \index{solido arquimedeano@sólido arquimedeano|see{poliedro!arquimedeano}}%
    \index{poliedro!arquimedeano}
  limitado por \(12\)~hexágonos y \(20\)~pentágonos%
    \index{poligono@polígono}
  (un total de \(32\)~caras),
  \(90\)~aristas y \(60\)~vértices.
  Si consideramos rotaciones en el espacio de este sólido,
  como en cada vértice confluyen un pentágono y dos hexágonos
  la única simetría
  que mantiene fijo un vértice es \(\iota\).
  Vía rotaciones podemos hacer coincidir ese vértice con cualquiera,
  por lo que tenemos
  que \(\lvert G \rvert
	  = \lvert G_x \rvert \cdot \lvert G x \rvert
	  = 1 \cdot 60
	  = 60\).
  Obtener el orden de este grupo manipulando el sólido
  sería mucho más complicado.

\section{Número de órbitas}
\label{sec:numero-orbitas}

  Vamos ahora a contar el número de órbitas de un grupo \(G\)
  de permutaciones de \(\mathcal{X}\).
  Cada órbita es un subconjunto de elementos indistinguibles
  bajo las operaciones de \(G\),
  y el número de órbitas
  dice cuántos tipos de elementos distinguibles hay.

  Supóngase que se quieren fabricar tarjetas de identidad cuadradas,
  divididas en nueve cuadrados de los cuales se perforan dos.
  Véase la figura~\ref{fig:tarjetas} para algunos ejemplos.
  \begin{figure}[htbp]
    \centering
    \subfloat[]{
      \pgfimage{images/badge-1}
      \label{subfig:badge-1}
    }%
    \hspace{2em}%
    \subfloat[]{
      \pgfimage{images/badge-2}
      \label{subfig:badge-2}
    }%
    \hspace{2em}%
    \subfloat[]{
      \pgfimage{images/badge-3}
      \label{subfig:badge-3}
    }
    \caption{Ejemplos de tarjetas de identidad}
    \label{fig:tarjetas}
  \end{figure}
  Las primeras dos no se pueden distinguir,
  ya que se obtiene la de la figura~\ref{subfig:badge-2}
  rotando la de~\ref{subfig:badge-1};
  en cambio,
  la de~\ref{subfig:badge-3} claramente es diferente de las otras,
  independiente de si se gira o se da vuelta.

  El grupo que está actuando acá
  es el grupo \(\mathtt{D}_8\) de ocho simetrías de un cuadrado,%
    \index{grupo!dihedral}
  pero interesa el efecto que tiene
  sobre las \(\binom{9}{2} = 36\) configuraciones de dos agujeros
  en un cuadrado de \(3 \times 3\),
  no solo su acción sobre los cuatro vértices.
  El número de órbitas es el número de tarjetas distinguibles.
  Hacer esto por la vía de dibujar las \(36\) configuraciones,
  y analizar lo que ocurre
  con cada una de ellas con las \(8\) simetrías
  es bastante trabajo.
  Por suerte hay maneras mejores.

  Dado un elemento \(\gamma\) del grupo de permutaciones \(G\)
  definimos:
  \begin{equation*}
    F(\gamma) = \{x \in \mathcal{X} \colon \gamma(x) = x\}
  \end{equation*}
  Vale decir,
  \(F(\gamma)\) es el número de puntos fijos de \(\gamma\).
  Nuestro teorema siguiente relaciona esto con el número de órbitas.
  Este resultado se conoce bajo el nombre de Burnside,
  de Cauchy-Frobenius y de Pólya.
  Burnside lo popularizó en su libro~%
    \cite{burnside97:_theor_group_finit_order},
  atribuyéndolo a Frobenius,
  aunque el resultado lo conocía Cauchy antes.
  Por esta enredada historia
  a veces se le llama ``el lema que no es de Burnside''.
  \begin{theorem}[Lema de Burnside]
    \label{theo:Burnside}
    \index{Burnside, lema de}
    El número de órbitas de \(G\)
    sobre \(\mathcal{X}\) está dado por:
    \begin{equation*}
      \frac{1}{\lvert G \rvert} \,
	\sum_{\gamma \in G} \lvert F(\gamma) \rvert
    \end{equation*}
  \end{theorem}
  \begin{proof}
    Nuevamente,
    contar por filas y columnas.
    Sea:
    \begin{equation*}
      E = \{(\gamma, x) \colon \gamma(x) = x\}
    \end{equation*}
    Entonces el total por fila \(r_\gamma(E)\)
    es el número de \(x\) fijados por \(\gamma\),
    o sea \(\lvert F(\gamma) \rvert\).
    El total por columna \(c_x(E)\)
    es el número de \(\gamma\) que tienen \(x\) como punto fijo,
    \(\lvert G_x \rvert\).
    Contabilizando \(E\) de ambas formas da:
    \begin{equation*}
      \sum_{\gamma \in G} \lvert F(\gamma) \rvert
	= \sum_{x \in \mathcal{X}} \lvert G_x \rvert
    \end{equation*}
    Supongamos que hay \(t\) órbitas,
    y elijamos \(z \in \mathcal{X}\).
    Por el teorema~\ref{theo:stabilizer-coset-r},
    si \(x\) pertenece a la órbita \(G z\)
    entonces \(\lvert G_x \rvert = \lvert G_z \rvert\).
    Cada órbita contribuye
    al lado derecho \(\lvert G z \rvert\) términos,
    todos \(\lvert G_z \rvert\);
    la contribución total de la órbita
    es \(\lvert G z \rvert \cdot \lvert G_z \rvert
	   = \lvert G \rvert\)
    por el teorema~\ref{theo:GxG_x=G},
    lo que lleva a:
    \begin{equation*}
      \sum_{\gamma \in G} \lvert F(\gamma) \rvert
	= t \cdot \lvert G \rvert
    \end{equation*}
    que es equivalente a lo que queríamos demostrar.
  \end{proof}

  Ahora podemos resolver nuestro problema de tarjetas de identidad.
  Necesitamos calcular el número de configuraciones fijas
  bajo cada una de las ocho permutaciones.
  Por ejemplo,
  cuando \(\gamma\) es la rotación en \(\pi\),
  hay cuatro configuraciones fijas
  (ver la figura~\ref{fig:tarjetas-fijas-180}).
  No hay configuraciones fijas bajo rotaciones de \(\pi / 2\)
  ni de \(3 \pi / 2\).
  \begin{figure}[htbp]
    \centering
    \subfloat{
      \pgfimage{images/badge-r2-a}
      \label{subfig:badge-r2-a}
    }%
    \hspace{2em}%
    \subfloat{
      \pgfimage{images/badge-r2-b}
      \label{subfig:badge-r2-b}
    }%
    \hspace{2em}%
    \subfloat{
      \pgfimage{images/badge-r2-c}
      \label{subfig:badge-r2-c}
    }%
    \hspace{2em}%
    \subfloat{
      \pgfimage{images/badge-r2-d}
      \label{subfig:badge-r2-d}
    }
    \caption{Configuraciones fijas bajo rotación en $\pi$}
    \label{fig:tarjetas-fijas-180}
  \end{figure}
  Podemos de la misma forma enumerar las configuraciones fijas
  bajo una reflexión en la vertical,
  ver la figura~\ref{fig:tarjetas-fijas-vertical}.
  Para la reflexión en la horizontal por simetría
  también tenemos seis configuraciones fijas.
  \begin{figure}[htbp]
    \centering
    \subfloat{
      \pgfimage{images/badge-v-a}
      \label{subfig:badge-v-a}
    }%
    \hspace{2em}%
    \subfloat{
      \pgfimage{images/badge-v-b}
      \label{subfig:badge-v-b}
    }%
    \hspace{2em}%
    \subfloat{
      \pgfimage{images/badge-v-c}
      \label{subfig:badge-v-c}
    }

    \subfloat{
      \pgfimage{images/badge-v-d}
      \label{subfig:badge-v-d}
    }%
    \hspace{2em}%
    \subfloat{
      \pgfimage{images/badge-v-e}
      \label{subfig:badge-v-e}
    }%
    \hspace{2em}%
    \subfloat{
      \pgfimage{images/badge-v-f}
      \label{subfig:badge-v-f}
    }
    \caption{Configuraciones fijas bajo reflexión en la vertical}
    \label{fig:tarjetas-fijas-vertical}
  \end{figure}
  Al enumerar las configuraciones fijas
  bajo una reflexión en la diagonal de la esquina inferior izquierda
  a la superior derecha también resultan seis configuraciones
  (figura~\ref{fig:tarjetas-fijas-diagonal}),
  y obtenemos otras seis para la reflexión en la otra diagonal.
  \begin{figure}[htbp]
    \centering
    \subfloat{
      \pgfimage{images/badge-d-a}
      \label{subfig:badge-d-a}
    }%
    \hspace{2em}%
    \subfloat{
      \pgfimage{images/badge-d-b}
      \label{subfig:badge-d-b}
    }%
    \hspace{2em}%
    \subfloat{
      \pgfimage{images/badge-d-c}
      \label{subfig:badge-d-c}
    }

    \subfloat{
      \pgfimage{images/badge-d-d}
      \label{subfig:badge-d-d}
    }%
    \hspace{2em}%
    \subfloat{
      \pgfimage{images/badge-d-e}
      \label{subfig:badge-d-e}
    }%
    \hspace{2em}%
    \subfloat{
      \pgfimage{images/badge-d-f}
      \label{subfig:badge-d-f}
    }
    \caption[Configuraciones fijas bajo reflexión en la diagonal]
	    {Configuraciones fijas bajo reflexión en la diagonal
	     de izquierda inferior a derecha superior}
    \label{fig:tarjetas-fijas-diagonal}
  \end{figure}
  El cuadro~\ref{tab:simetrias-tarjetas}
  resume los valores anteriores.
  \begin{table}[htbp]
    \centering
    \begin{tabular}[c]{|l|>{\(}r<{\)}|}
      \hline
      \multicolumn{1}{|c|}{\rule[-0.7ex]{0pt}{3ex}\textbf{Operación}} &
	\multicolumn{1}{c|}{\textbf{Fijos}} \\
      \hline
	\rule[-0.7ex]{0pt}{3ex}%
      Identidad				     & 36 \\
      Rotación en \(\pi / 2\)		     &	0 \\
      Rotación en \(\pi\)		     &	4 \\
      Rotación en \(3 \pi / 2\)		     &	0 \\
      Reflexión en diagonal \(1\;3\)	     &	6 \\
      Reflexión en diagonal \(2\;4\)	     &	6 \\
      Reflexión en perpendicular a \(1\;2\)  &	6 \\
      Reflexión en perpendicular a \(1\;4\)  &	6 \\
      \hline
    \end{tabular}
    \caption{Número de configuraciones de tarjetas
	     respetadas por cada simetría del cuadrado}
    \label{tab:simetrias-tarjetas}
  \end{table}
  Con estos valores tenemos el lema de Burnside
  que el número de órbitas es:
  \begin{equation*}
    \frac{1}{8} \, (36 + 0 + 4 + 0 + 6 + 6 + 6 + 6) = 8
  \end{equation*}
  En este caso es sencillo
  listar las ocho configuraciones por prueba y error,
  máxime sabiendo que son ocho
  (ver la figura~\ref{fig:tarjetas-distinguibles}),
  \begin{figure}[htbp]
    \centering
   \subfloat{
      \pgfimage{images/badge-a}
      \label{subfig:badge-a}
    }%
    \hspace{2em}%
    \centering
    \subfloat{
      \pgfimage{images/badge-b}
      \label{subfig:badge-b}
    }%
    \hspace{2em}%
    \subfloat{
      \pgfimage{images/badge-c}
      \label{subfig:badge-c}
    }

    \subfloat{
      \pgfimage{images/badge-d}
      \label{subfig:badge-d}
    }%
    \hspace{2em}%
    \subfloat{
      \pgfimage{images/badge-e}
      \label{subfig:badge-e}
    }

    \subfloat{
      \pgfimage{images/badge-f}
      \label{subfig:badge-f}
    }%
    \hspace{2em}%
    \subfloat{
      \pgfimage{images/badge-g}
      \label{subfig:badge-g}
    }%
    \hspace{2em}%
    \subfloat{
      \pgfimage{images/badge-h}
      \label{subfig:badge-h}
    }
    \caption{Las ocho tarjetas distinguibles}
    \label{fig:tarjetas-distinguibles}
  \end{figure}
  pero el resultado es aplicable en forma mucho más general.

\section{Índice de ciclos}
\label{sec:indice-ciclos}
\index{permutacion@permutación!indice de ciclos@índice de ciclos}

  Definimos el tipo de una permutación%
    \index{permutacion@permutación!tipo}
  como
    \(\left[ 1^{\alpha_1} \, 2^{\alpha_2}
	\, \dotso \, n^{\alpha_n} \right]\)
  si tiene \(\alpha_k\) ciclos de largo \(k\)
  para \(1 \le k \le n\).
  Una expresión afín asociada a la permutación \(\gamma\) es:
  \begin{equation*}
    \zeta_\gamma (x_1, x_2, \dotsc, x_n)
      = x_1^{\alpha_1} x_2^{\alpha_2} \dotso x_n^{\alpha_n}
  \end{equation*}
  Para un grupo \(G\) de permutaciones
  definimos el \emph{índice de ciclos}:
  \begin{equation*}
    \zeta_G (x_1, x_2, \dotsc, x_n)
      = \frac{1}{\lvert G \rvert} \,
	  \sum_{\gamma \in G} \zeta_\gamma (x_1, x_2, \dotsc, x_n)
  \end{equation*}
  Esto es esencialmente una función generatriz%
    \index{generatriz}
  en la que \(x_l\) marca los ciclos de largo \(l\).
  Esta función tiene muchos usos,
  algunos los veremos más adelante.

  Interesa calcular el índice de ciclos para diversos grupos
  de manera de tenerlos a mano más adelante.
  Consideremos primero los grupos \(\mathtt{C}_n\),%
    \index{grupo!ciclico@cíclico}
  que sabemos isomorfos con \(\mathbb{Z}_n\) y la suma.
  Si consideramos \(a \in \mathbb{Z}_n\),
  su orden determina el largo de los ciclos,
  y el número de ciclos es simplemente \(n / \ord(a)\).
  El orden es el mínimo \(b > 0\)
  tal que \(a \cdot b \equiv 0 \pmod{n}\).
  Si \(\gcd(a, n) = 1\),
  es \(b = n\) y
  hay \(\phi(n)\) de tales \(a\)
  que dan \(n / n = 1\) ciclo de largo \(n\).
  En general,
  si \(\gcd(a, n) = c\),
  las posibilidades de \(a\) esencialmente diferentes
  se restringen a \(n / c\) elementos,
  y de estos dan orden \(n / c\)
  exactamente los que tienen \(\gcd(a, n / c) = 1\).
  Expresarlo de esta forma es incómodo,
  llamemos \(d = n / c\).
  Entonces para \(d \mid n\) hay \(\phi(d)\) elementos
  de orden \(d\),
  los cuales forman \(n / d\) ciclos de largo \(d\):
  \begin{equation*}
    \zeta_{\mathtt{C}_n} (x_1, \dotsc, x_n)
      = \frac{1}{n} \,
	  \sum_{d \mid n} \phi(d) \, x_d^{n / d}
  \end{equation*}
% Fixme: Relacionar con operador \Cyc() de método simbólico sin rotular

  Veamos ahora el caso \(\mathtt{D}_{2 n}\).%
    \index{grupo!dihedral}
  A las simetrías anteriores se añaden \(n\) reflexiones.
  Si \(n\) es par,
  hay \(n / 2\) reflexiones a través de vértices opuestos,
  son \(2\) ciclos de largo \(1\) y \(n - 2\) ciclos de largo \(2\)
  que aportan \(n x_1^2 x_2^{(n - 2) / 2} / 2\);
  y \(n / 2\) reflexiones a través de lados opuestos,
  son \(n / 2\) ciclos de largo \(2\)
  que aportan \(n x_2^{n / 2} / 2\).
  Si \(n\) es impar,
  hay \(n\) reflexiones a través de un vértice y el lado opuesto,
  o sea un ciclo de largo \(1\)
  y \((n - 1) / 2\) ciclos de largo \(2\),
  que aportan \(n x_1 x_2^{(n - 1) / 2}\).
  En resumen,
  como en el índice de ciclos
  aparecen divididos por el orden del grupo,
  que se duplica a \(2 n\)
  entre \(\mathtt{C}_n\) y \(\mathtt{D}_{2 n}\):
  \begin{equation*}
    \zeta_{\mathtt{D}_{2 n}}
      = \frac{1}{2} \, \zeta_{\mathtt{C}_n}(x_1, \dotsc, x_n) +
	   \begin{cases}
	     \frac{1}{4} \, (x_1^2 x_2^{(n - 2) / 2} + x_2^{n / 2})
		 & \text{si \(n\) es par} \\[1ex]
	     \frac{1}{2} \, x_1 x_2^{(n - 1) / 2}
		 & \text{si \(n\) es impar}
	   \end{cases}
  \end{equation*}
  Así tenemos por ejemplo para el cuadrado:
  \begin{align*}
    \zeta_{\mathtt{C}_4} (x_1, x_2, x_3, x_4)
      &= \frac{1}{4} \, \sum_{d \mid 4} \phi(d) \, x_d^{4 / d} \\
      &= \frac{1}{4} \, \left(
			  x_1^4 + x_2^2 + 2 x_4
			\right) \\
    \zeta_{\mathtt{D}_8} (x_1, x_2, x_3, x_4)
      &= \frac{1}{2} \, \zeta_{\mathtt{C}_4} (x_1, x_2, x_3, x_4)
	   + \frac{1}{4} \, \left(
			       x_1^2 x_2 + x_2^2
			    \right) \\
      &= \frac{1}{8} \, \left(
			   x_1^4 + 2 x_1^2 x_2 + 3 x_2^2 + 2 x_4
			\right)
  \end{align*}
% Fixme: \Cyc() tal vez?

\section{Número de coloreos distinguibles}
\label{sec:coloreos-distinguibles}

% Fixme: Discutir grupo de permutaciones de arcos en K_n

% Fixme: Esta discusión es más complicada de lo que se requiere...
%	 Revisar/clarificar
%	 Agregar ejemplos de Richard P. Stanley "Topics in Algebraic
%	 Combinatorics"
%	 Más ejemplos útiles en William May "Introduction to Pólya
%	 Enumeration Theory"

  Supongamos un grupo \(G\) de permutaciones
  de un conjunto \(\mathcal{X}\) de \(n\) elementos,
  y a cada elemento se le puede asignar uno de \(r\) colores.
  Si el conjunto de colores es \(\mathcal{K}\),
  un \emph{coloreo}
  es una función
    \(\omega \colon \mathcal{X} \rightarrow \mathcal{K}\).
  El número total de coloreos es \(r^n\),
  a este conjunto le llamaremos \(\Omega\).
  Ahora bien,
  cada permutación \(g\) en \(G\)
  induce una permutación \(\widehat{g}\) de \(\Omega\):
  Para \(\omega\) definimos \(\widehat{g}(\omega)\)
  como el coloreo en el cual el color asignado a \(x\)
  es el que \(\omega\) asigna a \(g(x)\),
  vale decir:
  \begin{equation*}
    (\widehat{g}(\omega))(x) = \omega g^{-1}(x)
  \end{equation*}
  La inversa aparece porque al aplicar la permutación al coloreo
  estamos asignando a \(x\)
  el color que tiene su predecesor vía \(g\).
  La figura~\ref{fig:g-hat} muestra un ejemplo.
  \begin{figure}[htbp]
    \centering
    \subfloat{
      \pgfimage{images/g-hat-a}
    }%
    \hspace{3em}%
    \subfloat{
      \pgfimage{images/g-hat-b}
    }
    \caption{Efecto de la permutación $\widehat{g}$
	     sobre un coloreo $\omega$}
    \label{fig:g-hat}
  \end{figure}
% Fixme: Tomar ejemplo completo p.ej. de R. Stanley
  La función que lleva \(g\) a \(\widehat{g}\)
  es una representación del grupo \(G\)
  en un grupo \(\widehat{G}\) de permutaciones de \(\Omega\).
  Dos coloreos son indistinguibles
  si uno puede transformarse en el otro
  mediante una permutación \(\widehat{g}\);
  vale decir,
  si ambas pertenecen a la misma órbita
  de \(\widehat{G}\) en \(\Omega\).
  El número de coloreos distinguibles
  (\emph{inequivalentes})
  entonces es el número de órbitas de \(\widehat{G}\).
  Antes de aplicar nuestro teorema para el número de órbitas
  debemos relacionar \(G\) y \(\widehat{G}\).
  Supongamos que para dos permutaciones \(g_1\) y \(g_2\)
  tenemos \(\widehat{g}_1 = \widehat{g}_2\),
  de forma que
  \begin{equation*}
    (\widehat{g}_1(\omega))(x) = (\widehat{g}_2(\omega))(x)
  \end{equation*}
  y en consecuencia para todo \(\omega \in \Omega\)
  y todo \(x \in \mathcal{X}\) debe ser
  \begin{equation*}
    \omega(g_1^{-1}(x)) = \omega(g_2^{-1}(x))
  \end{equation*}
  Como esto es válido para todo \(\omega\),
  en particular vale para el coloreo
  que asigna el color especificado a \(g_1^{-1}(x)\)
  y otro color a todos los demás miembros de \(\mathcal{X}\).
  En este caso particular la ecuación
  dice que \(g_1^{-1}(x) = g_2^{-1}(x)\),
  con lo que \(g_1 = g_2\),
  y el grupo de permutaciones \(G\) de \(\mathcal{X}\)
  y el grupo de permutaciones \(\widehat{G}\)
  de los coloreos \(\Omega\) son isomorfos.

  Otra manera de entender esta situación
  es considerar un coloreo que le asigna un color diferente
  a cada elemento de \(\mathcal{X}\).
  Una permutación de ese coloreo
  no es más que una permutación
  de nuevos ``nombres'' de los elementos de \(\mathcal{X}\),
  con lo que está claro que ambos grupos de permutaciones
  están muy relacionados.

  \begin{theorem}
    \label{theo:coloreos}
    Si \(G\) es un grupo de permutaciones de \(\mathcal{X}\),
    y \(\zeta_G(x_1, \dotsc, x_n)\) es su índice de ciclos,
    el número de coloreos inequivalentes
    de \(\mathcal{X}\) con \(r\) colores
    es \(\zeta_G(r, \dotsc, r)\),
    donde un coloreo de \(\mathcal{X}\)
    es una función
      \(\omega \colon \mathcal{X} \rightarrow \mathcal{K}\).
  \end{theorem}
  \begin{proof}
    Interesa el número de órbitas del grupo \(G\)
    operando sobre coloreos.
    Hemos demostrado
    que la representación \(g \rightarrow \widehat{g}\)
    es una biyección,
    de forma que \(\lvert G \rvert = \lvert \widehat{G} \rvert\).
    Además,
    por el teorema de Burnside
    el número de órbitas de \(\widehat{G}\) en \(\Omega\) es
    \begin{equation*}
      \frac{1}{\lvert \widehat{G} \rvert} \,
	  \sum_{\widehat{g} \in \widehat{G}}
	    \lvert F(\widehat{g}) \rvert
	= \frac{1}{\lvert G \rvert} \,
	      \sum_{g \in G} \lvert F(\widehat{g}) \rvert
    \end{equation*}
    donde \(F(\widehat{g})\)
    es el conjunto de coloreos fijados por \(\widehat{g}\).
    Supongamos ahora que \(\omega\)
    es un coloreo fijado por \(\widehat{g}\),
    de forma que \(\widehat{g}(\omega) = \omega\),
    y sea \((x \; y \; z \; \dotso)\) un ciclo cualquiera de \(g\).
    Tenemos:
    \begin{equation*}
      \omega(x)
	= \omega(g(y))
	= (\widehat{g}(\omega))(y)
	= \omega(y)
    \end{equation*}
    de forma que \(\omega\) asigna el mismo color a \(x\) e \(y\).
    Aplicando el mismo razonamiento,
    este es el color asignado a todo el ciclo.
    Esto ocurre con cada uno de los ciclos de \(g\).
    Si \(g\) tiene \(k\) ciclos en total,
    el número de coloreos posibles es \(r^k\),
    ya que podemos asignar independientemente
    cualquiera de los \(r\) colores a cada uno de los \(k\) ciclos.
    De esta forma,
    si \(g\) tiene \(\alpha_i\) ciclos de largo \(i\)
    para (\(1 \le i \le n)\),
    tenemos \(\alpha_1 + \alpha_2 + \dotsb + \alpha_n = k\) y
    \begin{equation*}
      \lvert F(\widehat{g}) \rvert
	= r^k
	= r^{\alpha_1 + \alpha_2 + \dotsb + \alpha_n}
	= \zeta_g(r, r, \dotsc, r)
    \end{equation*}
    y el resultado sigue de sumar esto.
  \end{proof}

  \begin{example}
    Una tribu de hippies artesanos fabrica pulseras
    formadas alternadamente por tres arcos y tres cuentas,
    y tienen arcos y cuentas de cinco colores.
    Para efectos de simetría
    pueden considerarse las pulseras como triángulos equiláteros
    en los cuales se colorean los vértices y las aristas.
    Por razones que solo ellos entienden las pulseras
    deben siempre usar tres colores.
    Interesa saber cuántas pulseras diferentes pueden crear.

    Esta es una aplicación típica
    del principio de inclusión y exclusión%
      \index{inclusion y exclusion, principio de@inclusión y exclusión, principio de}
    (capítulo~\ref{cha:pie}),
    el teorema~\ref{theo:coloreos} da el número de coloreos
    con \emph{a lo más} el número de colores dado,
    pero nos interesan los coloreos
    con \emph{exactamente} tres colores.

    \begin{table}[ht]
      \centering
      \begin{tabular}{|l|>{\(}c<{\)}|>{\(}l<{\)}|}
	\hline
	\multicolumn{1}{|c|}{\rule[-0.7ex]{0pt}{3ex}\textbf{Operación}} &
	  \multicolumn{1}{c|}{\textbf{Nº}} &
	  \multicolumn{1}{c|}{\textbf{Término}}				\\
	\hline
	\rule[-0.7ex]{0pt}{3ex}%
	Identidad				      & 1 & x_1^6	\\
	Rotaciones (en \(2 \pi / 3\) y \(4 \pi / 3\)) & 2 & x_3^2	\\
	Reflexiones en cada eje			      & 3 & x_1^2 x_2^2 \\
	\hline
      \end{tabular}
      \caption{Elementos del grupo para pulseras}
      \label{tab:tabla-hippies}
    \end{table}
    El cuadro~\ref{tab:tabla-hippies}
    da los elementos del grupo relevante.
    Este grupo es de orden \(6\),
    así que su índice de ciclos es:
    \begin{equation*}
      \zeta_G(x_1, x_2, x_3, x_4, x_5, x_6)
	= \frac{1}{6} \,
	    \left(
	      x_1^6
		+ 3 x_1^2 x_2^2
		+ 2 x_3^2
	    \right)
    \end{equation*}

    Luego aplicamos nuestra receta
    del principio de inclusión y exclusión.
    \begin{enumerate}
    \item
      El universo \(\Omega\)
      es el conjunto de coloreos con \(5\) colores.
      Un coloreo tiene la propiedad \(i\)
      si el color \(i\) no está presente,
      e interesa el número
      de los que tienen exactamente \(2\) propiedades
      (están presentes los otros \(3\) colores).
    \item
      Acá \(N(\supseteq S)\) es el número de coloreos
      que no consideran los colores en \(S\),
      vale decir son coloreos
      tomando a lo más \(5 - \lvert S \rvert\) colores.
      Por el teorema~\ref{theo:coloreos}:
      \begin{equation*}
	N(\supseteq S)
	  = \zeta_G(5 - \lvert S \rvert,\,
		    5 - \lvert S \rvert,\,
		    5 - \lvert S \rvert,\,
		    5 - \lvert S \rvert,\,
		    5 - \lvert S \rvert,\,
		    5 - \lvert S \rvert)
      \end{equation*}
    \item
      Como los \(r\) colores a excluir se eligen de entre los \(5\),
      y en el número de posibilidades
      solo influye el número de colores restantes
      con los que se colorea:
      \begin{equation*}
	N_r
	  = \binom{5}{r} \,
	     \zeta_G(5 - r,\,
		     5 - r,\,
		     5 - r,\,
		     5 - r,\,
		     5 - r,\,
		     5 - r)
      \end{equation*}
      En este caso tenemos:
      \begin{align*}
	N_0
	  &= \binom{5}{0}
	       \, \zeta_G(5, 5, 5, 5, 5, 5) =		2\,925 \\
	N_1
	  &= \binom{5}{1}
	       \, \zeta_G(4, 4, 4, 4, 4, 4) =		4\,080 \\
	N_2
	  &= \binom{5}{2}
	       \, \zeta_G(3, 3, 3, 3, 3, 3) =		1\,650 \\
	N_3
	  &= \binom{5}{3}
	       \, \zeta_G(2, 2, 2, 2, 2, 2) = \phantom{0}\,200 \\
	N_4
	  &= \binom{5}{2}
	       \, \zeta_G(1, 1, 1, 1, 1, 1) = \phantom{000}\,5 \\
	N_5
	  &= \binom{5}{5}
	       \, \zeta_G(0, 0, 0, 0, 0, 0) = \phantom{000}\,0
      \end{align*}
      La función generatriz es
      \begin{equation*}
	N(z)
	  = 5 z^4 + 200 z^3 + 1\,650 z^2 + 4\,080 z + 2\,925
      \end{equation*}
    \item
      Nos interesa \(e_2\),
      que se obtiene de la función generatriz de los \(e_t\),
      que es \(E(z) = N(z - 1)\):
      \begin{equation*}
	E(z) = 5 z^4 + 180 z^3 + 1\,080 z^2 + 1\,360 z + 300
      \end{equation*}
      Se pueden formar \(1\,080\) brazaletes de tres colores.
    \end{enumerate}
  \end{example}

  Pero podemos hacer algo más.
  Si hay \(r\) colores,
  podemos definir variables \(z_i\) para \(1 \le i \le r\)
  representando los distintos colores.
  Entonces la función generatriz
  de los números de nodos de cada color
  que se pueden asignar a un ciclo de largo \(k\) es simplemente:
  \begin{equation*}
    z_1^k + z_2^k + \dotsb + z_r^k
  \end{equation*}
  ya que serían \(k\) nodos,
  todos del mismo color.
  Si hay \(\alpha_k\) ciclos de largo \(k\),
  entonces corresponde el factor:
  \begin{equation*}
    (z_1^k + z_2^k + \dotsb + z_r^k)^{\alpha_k}
  \end{equation*}
  La anterior discusión demuestra el siguiente resultado.
  \begin{theorem}[Enumeración de Pólya]
    \label{theo:Pólya}
    Sea \(G\) un grupo de permutaciones de \(\mathcal{X}\),
    y \(\zeta_G(x_1, \dotsc, x_n)\) su índice de ciclos.
    La función generatriz
    del número de coloreos inequivalentes de \(\mathcal{X}\)
    en que hay \(n_i\) nodos de color \(i\) para \(1 \le i \le r\),
    llamémosle \(u_{n_1, n_2, \dotsc, n_r}\),
    es:
    \begin{align*}
      U(z_1, z_2, \dotsc, z_r)
	&= \sum_{n_1, n_2, \dotsc, n_r}
	     u_{n_1, n_2, \dotsc, n_r} z_1^{n_1} z_2^{n_2}
		\dotsm z_r^{n_r} \\
	&= \zeta_G(z_1 + z_2 + \dotsb + z_r,
		   z_1^2 + z_2^2 + \dotsb + z_r^2,
		   \dotsc,
		   z_1^n + z_2^n + \dotsb + z_r^n)
    \end{align*}
  \end{theorem}

  \begin{table}[htbp]
    \centering
    \begin{tabular}{|l|>{\(}c<{\)}|>{\(}l<{\)}|}
      \hline
      \multicolumn{1}{|c|}{\rule[-0.7ex]{0pt}{3ex}\textbf{Operación}} &
	\multicolumn{1}{c|}{\textbf{Nº}} &
	\multicolumn{1}{c|}{\textbf{Término}} \\
      \hline
	\rule[-0.7ex]{0pt}{3ex}%
      Identidad					  & 1 & x_1^9	    \\
      Giro en \(\pi / 2\), \(3 \pi / 2\)	  & 2 & x_1 x_4^2   \\
      Giro en \(\pi\)				  & 1 & x_1 x_2^4   \\
      Reflexión horizontal, vertical		  & 2 & x_1^3 x_2^3 \\
      Reflexión diagonal			  & 2 & x_1^3 x_2^3 \\
      \hline
    \end{tabular}
    \caption{Las operaciones sobre tarjetas y sus tipos}
    \label{tab:tarjetas-tipos}
  \end{table}
  Volviendo a nuestro ejemplo de las tarjetas de identidad,
  el grupo subyacente es \(\mathtt{D}_8\),%
    \index{grupo!dihedral}
  las operaciones y sus tipos
  (operando sobre los nueve cuadraditos)
  se resumen en el cuadro~\ref{tab:tarjetas-tipos}.
  El índice de ciclos del grupo que interesa es:
  \begin{equation*}
    \zeta_t (x_1, x_2, x_3, x_4, x_5, x_6, x_7, x_8, x_9)
      = \frac{1}{8} \,
	  \left(
	    x_1^9 + 2 x_1 x_4^2 + x_1 x_2^4 + 4 x_1^3 x_2^3
	  \right)
  \end{equation*}
  El número total de tarjetas distinguibles con dos colores
  (agujero o no)
  en cada cuadradito es:
  \begin{equation*}
    \zeta_t (2, 2, 2, 2, 2, 2, 2, 2, 2)
      = 102
  \end{equation*}
  Para obtener el número de tarjetas con dos agujeros
  calculamos:
  \begin{equation*}
    \left[ z^2 \right]
       \zeta_t (1 + z, 1 + z^2, 1 + z^3, 1 + z^4, 1 + z^5, 1 + z^6,
		1 + z^7, 1 + z^8, 1 + z^9)
      = 8
  \end{equation*}
  Esto ya lo habíamos calculado antes,
  aunque de forma más trabajosa.
  Queda de ejercicio calcular del número de collares diferentes
  que se pueden crear
  de \(16\)~cuentas con \(3\)~negras de la misma forma.

% Fixme: Completar \Cyc() de método simbólico:
% A = \Cyc(B) ==>
%   A(z) = \sum_{k \ge 1} \frac{\phi(k)}{k} \cdot \ln \frac{1}{1 - B(z^k)}

  La teoría de enumeración de Pólya%
    \index{Polya, teoria de enumeracion de@Pólya, teoría de enumeración de}
  fue desarrollada en parte para aplicación a la química.
  Algunos ejemplos de fórmulas químicas
  se dan en la figura~\ref{fig:aromaticos}.
  \begin{figure}[htbp]
    \centering
    \subfloat[Benceno]{
      \pgfimage[height=0.25\textwidth]{images/benceno}
      \label{subfig:benceno}
    }%
    \hspace*{3em}%
    \subfloat[Xyleno]{
      \pgfimage[height=0.25\textwidth]{images/o-xyleno}
      \label{subfig:xyleno}
    }%
    \hspace*{3em}%
    \subfloat[Clorotolueno]{
      \pgfimage[height=0.25\textwidth]{images/clorotolueno}
      \label{subfig:clorotolueno}
    }
    \caption{Algunos compuestos aromáticos}
    \label{fig:aromaticos}
  \end{figure}
  Este tipo de compuestos,
  derivados del benceno (figura~\ref{subfig:benceno})
  son hexágonos que pueden girar en el espacio.
  Hay muchas posibilidades de grupos de átomos
  que pueden reemplazar los hidrógenos (\(\text{H}\)),
  como es radicales metilo (\(\text{CH}_3\))
  o átomos de cloro (\(\text{Cl}\)).
  Una pregunta obvia entonces
  es cuántos compuestos distintos pueden crearse
  con un conjunto de radicales,
  o cuántos son posibles con un número particular
  de cada uno de un conjunto de radicales dados.
  Por ejemplo,
  hay tres isómeros del xyleno
  (un anillo de benceno en el cual dos de los hidrógenos
   se substituyen por metilos),
  como muestra la figura~\ref{fig:isomeros-xyleno}.
  \begin{figure}[htbp]
    \centering
    \subfloat[Orto-xyleno]{
      \pgfimage[height=0.25\textwidth]{images/o-xyleno}
      \label{subfig:o-xyleno}
    }%
    \hspace*{3em}%
    \subfloat[Para-xyleno]{
      \pgfimage[height=0.25\textwidth]{images/p-xyleno}
      \label{subfig:p-xyleno}
    }%
    \hspace*{3em}%
    \subfloat[Meta-xyleno]{
      \pgfimage[height=0.25\textwidth]{images/m-xyleno}
      \label{subfig:m-xyleno}
    }
    \caption{Los tres isómeros del xyleno}
    \label{fig:isomeros-xyleno}
  \end{figure}

  El grupo de simetría relevante es \(\mathtt{D}_{12}\),
  cuyo índice de ciclos podemos calcular como antes:
  \begin{align*}
    \zeta_{\mathtt{C}_6} (x_1, x_2, x_3, x_4, x_5, x_6)
      &= \frac{1}{6} \, \sum_{d \mid 6} \phi(d) x_d^{6 / d} \\
      &= \frac{1}{6} \, \left(
			  x_1^6 + x_2^3 + 2 x_3^2 + 2 x_6
			\right) \\
    \zeta_{\mathtt{D}_{12}} (x_1, x_2, x_3, x_4, x_5, x_6)
      &= \frac{1}{2} \, \zeta_{\mathtt{C}_6}
			  (x_1, x_2, x_3, x_4, x_5, x_6)
	   + \frac{1}{4} \, \left(
			      x_1^2 x_2^2 + x_2^3
			    \right) \\
      &= \frac{1}{12} \, \left(
			   x_1^6 + 3 x_1^2 x_2^2
			     + 4 x_2^3 + 2 x_3^2 + 2 x_6
			 \right)
  \end{align*}
  Hecho el trabajo duro,
  determinar cuántos compuestos pueden crearse
  con radicales hidrógeno (\(\text{H}\))
  y metilo (\(\text{CH}_3\))
  es fácil:
  Es colorear los vértices con dos colores,
  lo que da:
  \begin{align*}
    \zeta_{\mathtt{D}_{12}}(2, 2, 2, 2, 2, 2)
      & = \frac{1}{12} \,
	     (2^6 + 3 \cdot 2^2 \cdot 2^2
		  + 4 \cdot 2^3 + 2 \cdot 2) \\
      & = 13
  \end{align*}

  Para comprobar cuántos isómeros del xyleno hay,
  consideramos coloreo de los vértices del hexágono con dos colores
  (hidrógenos y metilos),
  y de los últimos hay exactamente dos.
  Si consideramos que \(z\) marca metilo,
  la función generatriz que corresponde a un ciclo de largo \(l\)
  es simplemente \(1 + z^l\)
  (hay 1 forma de tener 0 metilos en él,
   lo que aporta \(1 \cdot z^0\),
   y 1 forma de tener l metilo,
   lo que aporta \(1 \cdot z^l\)),
  y al substituir \(x_l = 1 + z^l\) obtenemos:
  \begin{equation*}
    \zeta_{\mathtt{D}_{12}}(1 + z, 1 + z^2, 1 + z^3, 1 + z^4,
			    1 + z^5, 1 + z^6)
      = z^6 + z^5 + 3 z^4 + 3 z^3 + 3 z^2 + z + 1
  \end{equation*}
  Esto confirma que hay tres isómeros en su coeficiente de \(z^2\).

  Si interesa determinar cuántos compuestos distintos tienen
  2 radicales cloro (\(\text{Cl}\)),
  2 metilos (\(\text{CH}_3\))
  y 2 hidrógenos  (\(\text{H}\)),
  usamos las variables \(u\), \(v\) y \(w\)
  para estas tres opciones,
  y el valor buscado es simplemente:
  \begin{equation*}
    \left[ u^2 v^2 w^2 \right]
      \zeta_{\mathtt{D}_6} (u + v + w,
			    u^2 + v^2 + w^2,
			    \dotsc,
			    u^6 + v^6 + w^6)
	 = 11
  \end{equation*}

  Por otro lado,
  un átomo de carbono puede unirse con cuatro otros átomos,
  dispuestos en los vértices de un tetraedro.%
    \index{poliedro!regular}%
    \index{tetraedro}
  \begin{figure}[htbp]
    \centering
    \subfloat[Eje a través de un vértice]{
      \pgfimage{images/tetraedro-vertice}
      \label{subfig:tetraedro-vertice}
    }
    \vspace*{1em}
    \subfloat[Eje en el punto medio de aristas]{
      \pgfimage{images/tetraedro-arista}
      \label{subfig:tetraedro-arista}
    }
    \caption{Operaciones de simetría (rotaciones) de un tetraedro}
    \label{fig:rotaciones-tetraedro-2}
  \end{figure}
  Las operaciones de simetría de un tetraedro en el espacio
  (solo rotaciones, no reflexiones)
  son giros alrededor de un eje que pasa por un vértice
  y el centroide de la cara opuesta
  (ver la figura~\ref{subfig:tetraedro-vertice})
  y giros alrededor de un eje
  que pasa por el punto medio de una arista
  y el punto medio de la arista opuesta
  (ver la figura~\ref{subfig:tetraedro-arista}).
  \begin{table}[htbp]
    \centering
    \begin{tabular}{|l|*{4}{>{\(}l<{\)}|}}
      \hline
      \multicolumn{1}{|c|}{\rule[-0.7ex]{0pt}{3ex}\textbf{Operación}} &
	\multicolumn{1}{c|}{\textbf{Ciclos}}  &
	\multicolumn{1}{c|}{\textbf{Tipo}}    &
	\multicolumn{1}{c|}{\textbf{Nº}}      &
	\multicolumn{1}{c|}{\textbf{Término}} \\
      \hline
	 \rule[-0.7ex]{0pt}{3ex}%
      Identidad			 &
	(1)(2)(3)(4) & [1^4]	  & 1 & x_1^4	  \\
      Giro en vértice 4 en 1/3	 &
	(1\;2\;3)(4) & [1 \, 3^1] & 4 & x_1^3 x_3 \\
      Giro en vértice 4 en 2/3	 &
	(1\;3\;2)(4) & [1 \, 3^1] & 4 & x_1^3 x_3 \\
      Giro en arista 1\;2 en 1/2 &
	(1\;2)(3\;4) & [2^2]	  & 3 & x_2^2	  \\
      \hline
    \end{tabular}
    \caption{Rotaciones de un tetraedro}
    \label{tab:rotaciones-tetraedro}
  \end{table}
  Las simetrías son de los tipos dados
  en el cuadro~\ref{tab:rotaciones-tetraedro}
  (resulta que esto no es más
   que el grupo alternante \(\mathtt{A}_4\)),%
    \index{grupo!alternante}
  y en consecuencia el índice de ciclos del grupo es
  \begin{equation*}
    \zeta_{\mathtt{A}_4}(x_1, x_2, x_3, x_4)
      = \frac{1}{12}(x_1^4 + 8 x_1 x_3 + 3 x_2^2)
  \end{equation*}
  Así,
  para dos radicales diferentes
  hay \(\zeta_{\mathtt{A}_4}(2, 2, 2, 2) = 5\) compuestos posibles,
  y para cuatro radicales
  hay \(\zeta_{\mathtt{A}_4}(4, 4, 4, 4) = 36\).
  Si hay dos tipos de radicales,
  la función generatriz es:
  \begin{equation*}
    \zeta_{\mathtt{A}_4}(u + v,
		u^2 + v^2,
		u^3 + v^3,
		u^4 + v^4)
      = u^4 + u^3 v + u^2 v^2 + u v^3 + v^4
  \end{equation*}
  Vale decir,
  hay un solo compuesto
  de cada una de las cinco composiciones posibles.

  Considere árboles binarios completos de altura 2,
  como en la figura~\ref{fig:arbol-binario-simetria},
  \begin{figure}[htbp]
    \centering
    \pgfimage{images/arbol-simetria}
    \caption{Un árbol binario completo}
    \label{fig:arbol-binario-simetria}
  \end{figure}
  que se consideran iguales al intercambiar izquierda y derecha
  (como 3 con 4;
  pero también 2 con 5,
  que lleva consigo intercambiar 3 con 6 y 4 con 7).
  Interesa determinar cuántos árboles hay con 3 nodos azules,
  si los nodos se pintan de azul,
  rojo y amarillo.

  Antes de entrar en el tema,
  es útil obtener información sobre el grupo.
  \begin{table}[htbp]
    \centering
    \begin{tabular}{|>{\(}l<{\)}|>{\(}l<{\)}|}
      \hline
      \multicolumn{1}{|c|}{\rule[-0.7ex]{0pt}{3ex}\textbf{Operación}} &
	\multicolumn{1}{c|}{\textbf{Término}}	\\
      \hline
	 \rule[-0.7ex]{0pt}{3ex}%
      \text{Identidad}	   & x_1^7	  \\
      (3\;4)		   & x_1^5 x_2	  \\
      (6\;7)		   & x_1^5 x_2	  \\
      (3\;4) (6\;7)	   & x_1^3 x_2^2  \\
      (2\;5) (3\;6) (4\;7) & x_1 x_2^3	  \\
      (2\;5) (3\;7) (4\;6) & x_1 x_2^3	  \\
      (2\;5) (3\;6\;4\;7)  & x_1 x_2 x_4  \\
      (2\;5) (3\;7\;4\;6)  & x_1 x_2 x_4  \\
      \hline
    \end{tabular}
    \caption{El grupo de operaciones del árbol}
    \label{tab:grupo-arbol}
  \end{table}
  Para determinar el orden del grupo,
  tomamos algún elemento y analizamos su órbita y estabilizador.
  Tomando 3,
  su órbita es \(G3 = \{3, 4, 6, 7\}\),
  mientras su estabilizador es \(G_3 = \{\iota, (6\;7)\}\),
  con lo que
  \(\lvert G \rvert
      = \lvert G3 \rvert \cdot \lvert G_3 \rvert
      = 4 \cdot 2
      = 8\).
  Los elementos del grupo los da el cuadro~\ref{tab:grupo-arbol},
  el índice de ciclos del grupo resulta ser:
  \begin{equation*}
    \zeta_G(x_1, x_2, x_3, x_4, x_5, x_6, x_7)
      = \frac{1}{8} \,
	  \left(
	    x_1^7
	      + 2 x_1^5 x_2
	      + x_1^3 x_2^2
	      + 2 x_1 x_2^3
	      + x_1 x_2 x_4
	  \right)
  \end{equation*}
  La manera más simple de obtener el resultado buscado
  es reconocer que la función generatriz%
    \index{generatriz}
  para el número de maneras de formar órbitas de \(l\) nodos
  donde \(u\) marca el número de nodos azules
  es simplemente \(2 + u^l\)
  (dos formas de ningún azul,
   vale decir solo rojos o solo amarillos;
   y una forma de \(l\) azules),
  y para aplicar el teorema de Pólya interesa:
  \begin{align*}
    \left[ u^3 \right] \zeta_G(2 + u,
	       &  2 + u^2,
		  2 + u^3,
		  2 + u^4,
		  2 + u^5,
		  2 + u^6,
		  2 + u^7) \\
      &= \left[ u^3 \right] \left(
		 u^7
		   +   6 u^6
		   +  25 u^5
		   +  68 u^4
		   + 120 u^3
		   + 146 u^2
		   + 105 u
		   +  42
		 \right) \\
       &= 120
  \end{align*}
  Obtener esto por prueba y error sería impensable.
  Nuevamente agradecemos el apoyo algebraico
  de \texttt{maxima}~%
    \cite{maxima14b:_computer_algebra}.%
    \index{maxima@\texttt{maxima}}

  Un dado es un cubo,%
    \index{poliedro!regular}%
    \index{cubo}
  cuyas caras están numeradas de 1 a 6.
  Interesa saber de cuántas maneras distintas
  se pueden distribuir los seis números sobre las caras.
  En este caso,
  interesan las simetrías rotacionales del cubo en el espacio
  (las reflexiones corresponden
   a operaciones imposibles con un sólido).
  Primeramente calculamos el orden del grupo que nos interesa,
  que resulta ser el grupo de rotaciones
  en el espacio de un octaedro
  y se denomina \(\mathtt{O}\).
  Sabemos que
    \(\lvert \mathtt{O} \rvert
	= \lvert \mathtt{O}_x \rvert
	    \cdot \lvert \mathtt{O} x \rvert\).
  Si fijamos una de las caras del cubo,
  hay 4 operaciones que la mantienen fija
  (rotaciones alrededor del centroide
   de esa cara en múltiplos de \(\pi / 2\)),
  y esta cara puede ocupar cualquiera de las 6 posiciones.
  Luego \(\lvert \mathtt{O} \rvert = 4 \cdot 6 = 24\).
  Las operaciones y sus tipos las resume el cuadro~\ref{tab:dado}.
  \begin{table}[htbp]
    \centering
    \begin{tabular}{|l|c|>{\(}l<{\)}|}
      \hline
      \multicolumn{1}{|c|}{\rule[-0.7ex]{0pt}{3ex}\textbf{Operación}} &
	\multicolumn{1}{c|}{\textbf{Nº}} &
	\multicolumn{1}{c|}{\textbf{Término}} \\
      \hline
	\rule[-0.7ex]{0pt}{3ex}%
      Identidad						      & 1 &
	 x_1^6 \\
      Giro alrededor de centro de una cara en \(\pi / 2\)     & 3 &
	 x_1^2 x_4 \\
      Giro alrededor de centro de una cara en \(\pi\)	      & 3 &
	 x_1^2 x_2^2 \\
      Giro alrededor de centro de una cara en \(3 \pi / 2\)   & 3 &
	 x_1^2 x_4 \\
      Giro alrededor del punto medio de una arista en \(\pi\) & 6 &
	 x_2^3 \\
      Giro alrededor de un vértice en \(2 \pi / 3\)	      & 4 &
	 x_3^2 \\
      Giro alrededor de un vértice en \(4 \pi / 3\)	      & 4 &
	 x_3^2 \\
      \hline
    \end{tabular}
    \caption{Operaciones de simetría rotacional de caras de un cubo}
    \label{tab:dado}
  \end{table}
  El índice de ciclos del grupo \(\mathtt{O}\) es
  \begin{equation*}
    \zeta_{\mathtt{O}}(x_1, x_2, x_3, x_4, x_5, x_6)
      = \frac{1}{24} \,
	  \left(
	    x_1^6 + 6 x_1^2 x_4 + 3 x_1^2 x_2^2 + 6 x_2^3 + 8 x_3^2
	  \right)
  \end{equation*}
  Como interesa saber de cuántas maneras
  se pueden distribuir los 6~números sobre las 6~caras:
  \begin{equation*}
    \left[ z_1 z_2 z_3 z_4 z_5 z_6 \right]
      \zeta_{\mathtt{O}}(z_1 + \dotsb + z_6,
	      z_1^2 + \dotsb + z_6^2,
	      z_1^3 + \dotsb + z_6^3,
	      z_1^4 + \dotsb + z_6^4,
	      z_1^5 + \dotsb + z_6^5,
	      z_1^6 + \dotsb + z_6^6)
  \end{equation*}
  Siquiera encontrar el término que interesa en esta expresión
  ya es toda una tarea.
  Pero si observamos que la única forma
  de obtener términos en los que los \(z_i\)
  entran en la primera potencia vienen de aquellos términos
  en que solo participa \(x_1\),
  nuestro problema se reduce a calcular:
  \begin{align*}
    \left[ z_1 z_2 z_3 z_4 z_5 z_6 \right]
	\frac{1}{24} \, (z_1 + z_2 + z_3 + z_4 + z_5 + z_6)^6
      &= \frac{1}{24} \, \binom{6}{1\;1\;1\;1\;1\;1} \\
      &= 30
  \end{align*}

  Si nos interesa contar el número de maneras
  de numerar las caras del dado
  respetando la restricción que caras opuestas sumen 7,
  la situación relevante
  es considerar las tres caras numeradas 1, 2 y 3.
  Estas caras serán adyacentes,
  y por tanto la situación
  es la que indica la figura~\ref{fig:cubo-vertice},
  \begin{figure}[htbp]
    \centering
    \pgfimage{images/cubo-vertice}
    \caption{Un cubo visto desde un vértice}
    \label{fig:cubo-vertice}
  \end{figure}
  que muestra las tres caras vistas desde un vértice.
  Está claro que la simetría es \(\mathtt{C}_3\),
  un triángulo equilátero rotando en el plano
  (consideramos los vértices del triángulo
   como las caras a ser numeradas).
  Para este grupo el índice de ciclos es:
  \begin{equation*}
    \zeta_{\mathtt{C}_3}(x_1, x_2, x_3)
      = \frac{1}{3} \,
	  \left(
	    x_1^3 + 2 x_3
	  \right)
  \end{equation*}
  Nos interesa colorear con tres colores,
  y que cada uno aparezca exactamente una vez,
  por Pólya:
  \begin{equation*}
    \left[ z_1 z_2 z_3 \right] \zeta_{\mathtt{C}_3}(z_1 + z_2 + z_3,
			      z_1^2 + z_2^2 + z_3^2,
			      z_1^3 + z_2^3 + z_3^3)
  \end{equation*}
  El único término en que entran los \(z_i\) en la primera potencia
  es el término de la identidad,
  y en él nos interesa cada uno en la primera potencia:
  \begin{align*}
    \left[ z_1 z_2 z_3 \right] \zeta_{\mathtt{C}_3}(z_1 + z_2 + z_3,
			      z_1^2 + z_2^2 + z_3^2,
			      z_1^3 + z_2^3 + z_3^3)
      &= \frac{1}{3} \, \left[
			  z_1 z_2 z_3
			\right] (z_1 + z_2 + z_3)^3 \\
      &= \frac{1}{3} \, \binom{3}{1\;1\;1} \\
      &= 2
  \end{align*}
  Hay dos maneras diferentes de numerar las caras de un cubo
  con los números uno a seis tal que caras opuestas sumen siete.

%%% Local Variables:
%%% mode: latex
%%% TeX-master: "clases"
%%% End:


% analisis-complejo.tex
%
% Copyright (c) 2013-2015 Horst H. von Brand
% Derechos reservados. Vea COPYRIGHT para detalles

\chapter{Introducción al análisis complejo}
\label{cha:analisis-complejo}
\index{analisis complejo@análisis complejo|textbfhy}
\index{analisis complejo@análisis complejo|seealso{\(\mathbb{C}\) (números complejos)}}

  Nahin~\cite{nahin10:_imaginary_tale}
  narra la larga y variada historia de los números complejos.
  En forma similar al análisis con los reales
  se puede desarrollar análisis en el ámbito complejo.
  Muchos resultados son simples de obtener para los complejos,
  y algunos fenómenos misteriosos en el análisis real
  se explican al observar desde esta óptica.
  La teoría tiene su propio encanto.

  Algunas de nuestras maniobras
  son en extremo engorrosas
  usando solo las técnicas del análisis real.
  Daremos acá los resultados más importantes del análisis complejo,
  que ayuda enormemente al simplificar integrales definidas
  y sumas.
  Entrega además herramientas útiles
  para construir aproximaciones asintóticas
  a muchos valores de interés.
  Textos introductorios más completos son por ejemplo los de
  Ash y Novinger~\cite{ash07:_complex_variables},
  Beck, Marchesi, Pixton y Sabalka~%
    \cite{beck12:_first_course_compl_analysis},
  Cain~\cite{cain01:_compl_analy}
  y~Chen~\cite{chen08:_intro_complex_anal}.
  Una visión detallada y bastante completa dan Stein y~Shakarchi~%
    \cite{stein10:_compl_analy}.

\section{Aritmética}
\label{sec:aritmetica-complejos}
\index{C (numeros complejos)@\(\mathbb{C}\) (números complejos)!operaciones}

  Sean \(w = u + \mathrm{i} v\) y \(z = x + \mathrm{i} y\) complejos
  (con la \emph{unidad imaginaria}\, \(\mathrm{i} = \sqrt{-1}\)%
    \index{C (numeros complejos)@\(\mathbb{C}\) (números complejos)!unidad imaginaria}
   y \(u, v, x, y \in \mathbb{R}\) ).
  La \emph{parte real} de \(z\) es \(\Re z = x\),%
    \index{C (numeros complejos)@\(\mathbb{C}\) (números complejos)!parte real}
  la \emph{parte imaginaria} de \(z\) es \(\Im z = y\).%
    \index{C (numeros complejos)@\(\mathbb{C}\) (números complejos)!parte imaginaria}
  La suma y multiplicación
  se calculan como polinomios en \(\mathbb{R}\)
  en la variable \(\mathrm{i}\),
  para luego considerar \(\mathrm{i}^2 = -1\).
  Vale decir:
  \begin{align*}
    w + z
      &= (u + x) + \mathrm{i} (v + y) \\
    w \cdot z
      &= (u x - v y) + \mathrm{i} (v x + u y)
  \end{align*}
  Con estas operaciones los números complejos
  son un campo,%
    \index{campo (algebra)@campo (álgebra)}
  que anotamos \(\mathbb{C}\).

  Podemos identificar los complejos con parte imaginaria \(0\)
  con los reales,
  y representar el complejo \(z = x + \mathrm{i} y\)
  por el punto \((x, y)\) del plano \(\mathbb{R}^2\).
  A esta forma de representarlos
  se le llama \emph{forma rectangular}%
    \index{C (numeros complejos)@\(\mathbb{C}\) (números complejos)!forma rectangular|see{\(\mathbb{C}\) (números complejos)!forma cartesiana}}
  o \emph{forma cartesiana}.%
    \index{C (numeros complejos)@\(\mathbb{C}\) (números complejos)!forma cartesiana}
  Llamamos \emph{eje real} al eje~\(X\)
  y \emph{eje imaginario} al eje~\(Y\).
  La suma de complejos
  es simplemente la suma de los vectores correspondientes.
  \begin{figure}[ht]
    \centering
    \subfloat[Suma]{
      \pgfimage{images/complex-sum}
      \label{subfig:complex-sum}
    }%
    \hspace*{3em}%
    \subfloat[Producto]{
      \pgfimage{images/complex-product}
      \label{subfig:complex-product}
    }
    \caption{Operaciones entre complejos}
    \label{fig:complex-operations}
  \end{figure}
  Vea la figura~\ref{subfig:complex-sum}.

  Como alternativa a dar las coordenadas del vector,
  podemos describirlo mediante su largo
  y el ángulo que forma con el eje~\(X\).
  Para \(z = x + \mathrm{i} y\)
  el \emph{valor absoluto}
  (también \emph{módulo}) \(r = \lvert z \rvert\) se define como:%
    \index{C (numeros complejos)@\(\mathbb{C}\) (números complejos)!modulo@módulo}
  \begin{equation}
    \label{eq:complex-modulus}
    \lvert z \rvert
      = \sqrt{x^2 + y^2}
  \end{equation}
  Un \emph{argumento} de \(z\),%
    \index{C (numeros complejos)@\(\mathbb{C}\) (números complejos)!argumento}
  anotado \(\arg z\),
  es un número real \(\phi\) tal que
  \begin{equation}
    \label{eq:complex-argument}
    x = r \cos \phi \qquad y = r \sin \phi
  \end{equation}
  Nótese que todo número complejo tiene infinitos argumentos.
  En el caso excepcional \(z = 0\)
  el módulo es \(0\) y cualquier ángulo sirve de argumento.
  En caso \(z \ne 0\)
  vemos que si \(\phi\) es un argumento de \(z\),
  también lo es \(\phi + 2 \pi k\),
  para todo \(k \in \mathbb{Z}\).
  La representación del número complejo como módulo y argumento
  se conoce como \emph{representación polar}.%
    \index{C (numeros complejos)@\(\mathbb{C}\) (números complejos)!forma polar}
  El \emph{valor principal} del argumento se anota \(\Arg z\),%
    \index{C (numeros complejos)@\(\mathbb{C}\) (números complejos)!valor principal}
  es el ángulo restringido al rango \(-\pi < \phi \le \pi\).
  A veces resulta útil restringir el ángulo a otro rango,
  usaremos \(\arg_\alpha z\)
  para el ángulo en el rango \(\alpha \le \arg_\alpha z < \alpha + 2 \pi\).

  La representación polar
  da una bonita interpretación de la multiplicación.
  Sean números complejos \(z_1 = x_1 + \mathrm{i} y_1\)
  y \(z_2 = x_2 + \mathrm{i} y_2\),
  respectivamente con módulos \(r_1\) y \(r_2\)
  y argumentos \(\phi_1\) y \(\phi_2\).
  Entonces:
  \begin{align*}
    (x_1 + \mathrm{i} y_1) \cdot (x_2 + \mathrm{i} y_2)
      &= (r_1 \cos \phi_1 + \mathrm{i} r_1 \sin \phi_1)
	   \cdot (r_2 \cos \phi_2 + \mathrm{i} r_2 \sin \phi_2) \\
      &= r_1 r_2
	   \left(
	     (\cos \phi_1 \cos \phi_2 - \sin \phi_1 \sin \phi_2)
	       + \mathrm{i}
		   (\sin \phi_1 \cos \phi_2 + \cos \phi_1 \sin \phi_2)
	   \right) \\
      &= r_1 r_2 (\cos (\phi_1 + \phi_2)
		    + \mathrm{i} \, \sin(\phi_1 + \phi_2))
  \end{align*}
  Vea la figura~\ref{subfig:complex-product} para un ejemplo.

  Deberemos manipular expresiones de la forma
    \(\cos \phi + \mathrm{i} \, \sin \phi\)
  con bastante frecuencia,
  rindiéndonos a la flojera
  (con la excusa de ahorrar papel,
   tinta,
   etc.)
  escribimos:
  \begin{equation}
    \index{C (numeros complejos)@\(\mathbb{C}\) (números complejos)!exponencial}
    \label{eq:imaginary-exponential}
    \exp(\mathrm{i} \, \phi)
      = \mathrm{e}^{\mathrm{i} \, \phi}
      = \cos \phi + \mathrm{i} \, \sin \phi
  \end{equation}
  Por ahora~\eqref{eq:imaginary-exponential}
  es simplemente una abreviatura cómoda,
  más adelante demostraremos que es consistente
  con la función exponencial
  del cálculo real.
  También es común la notación:
  \begin{equation*}
    \operatorname{cis} \phi
      = \cos \phi + \mathrm{i} \, \sin \phi
  \end{equation*}

  El siguiente lema recoge algunas propiedades salientes,
  alentamos al lector interesado demostrarlas.
  \begin{lemma}
    \label{lem:imaginary-exponential}
    Para cualquier \(\phi, \phi_1, \phi_2 \in \mathbb{R}\)
    y todo \(k \in \mathbb{Z}\):
    \begin{enumerate}[label=(\roman*), ref=(\roman*)]
    \item
      \(\lvert \mathrm{e}^{\mathrm{i} \, \phi} \rvert
	  = 1\)
    \item
      \(\mathrm{e}^{\mathrm{i} \, \phi_1}
	\mathrm{e}^{\mathrm{i} \, \phi_2}
	  = \mathrm{e}^{\mathrm{i} (\phi_1 + \phi_2)}\)
    \item
      \(1 / \mathrm{e}^{\mathrm{i} \, \phi}
	  = \mathrm{e}^{- \mathrm{i} \, \phi}\)
    \item
      \(\mathrm{e}^{\mathrm{i} (\phi + 2 k \pi)}
	  = \mathrm{e}^{\mathrm{i} \, \phi}\)
    \end{enumerate}
  \end{lemma}
  Con esta notación,
  el número complejo de módulo \(r\) y argumento \(\phi\)
  puede escribirse:
  \begin{equation}
    \label{eq:complex-exponential-notation}
    x + \mathrm{i} y
      = r \mathrm{e}^{\mathrm{i} \, \phi}
  \end{equation}

  El cuadrado del valor absoluto de \(z\) tiene la bonita propiedad:
  \begin{equation}
    \label{eq:complex-modulus-conjugates}
    \lvert z \rvert^2
      = x^2 + y^2
      = (x + \mathrm{i} y) \cdot (x - \mathrm{i} y)
  \end{equation}
  Esto hace útil la operación de \emph{conjugado}:%
    \index{C (numeros complejos)@\(\mathbb{C}\) (números complejos)!conjugado}
  \begin{equation}
    \label{eq:complex-conjugate}
    \overline{x + \mathrm{i} y}
      = x - \mathrm{i} y
  \end{equation}
  En el plano cartesiano
  corresponde a reflejar el vector en el eje~\(X\).
  Tenemos algunas propiedades,
  que nuevamente animamos a demostrar.
  \begin{lemma}
    \label{lem:complex-conjugate}
    Para todo \(z, z_1, z_2 \in \mathbb{C}\),
    y para todo \(\phi \in \mathbb{R}\):
    \begin{enumerate}[label=(\roman*), ref=(\roman*)]
    \item
      \(\overline{z_1 \pm z_2} = \overline{z_1} \pm \overline{z_2}\)
    \item
      \(\overline{z_1 \cdot z_2}
	   = \overline{z_1} \cdot \overline{z_2}\)
    \item
      \(\overline{z_1 / z_2} = \overline{z_1} / \overline{z_2}\)
    \item
      \(\overline{\overline{z}} = z\)
    \item
      \(\lvert \overline{z} \rvert = \lvert z \rvert\)
    \item
      \label{lem:complex-conjugate:part-f}
      \(\lvert z \rvert^2 = z \overline{z}\)
    \item
      \(\Re z = \frac{1}{2} \, ( z + \overline{z})\)
      \qquad
      \(\Im z = \frac{1}{2 \mathrm{i}} \, ( z - \overline{z})\)
    \item
      \(\overline{\mathrm{e}^{\mathrm{i} \, \phi}}
	  = \mathrm{e}^{- \mathrm{i} \, \phi}\)
    \end{enumerate}
  \end{lemma}
  La parte~\ref{lem:complex-conjugate:part-f}
  del lema~\ref{lem:complex-conjugate}
  da una fórmula limpia para el recíproco de un complejo no cero:
  \begin{equation}
    \index{C (numeros complejos)@\(\mathbb{C}\) (números complejos)!reciproco@recíproco}
    \label{eq:complex-inverse}
    \frac{1}{z}
      = \frac{\overline{z}}{\lvert z \rvert^2}
  \end{equation}

\section{Un poquito de topología del plano}
\label{sec:topologia}
\index{topologia@topología}

  Al considerar funciones en \(\mathbb{R}\)
  la situación es bastante simple,
  basta hablar de intervalos.
  Incluso en \(\mathbb{R}^n\)
  el tratamiento puede seguir esencialmente
  variable a variable
  y restringirse a intervalos adecuados.
  En \(\mathbb{C}\) esto no es satisfactorio,
  áreas del plano
  pueden tener relaciones mucho más complicadas entre sí
  que los simples intervalos.
  Las definiciones siguientes
  serán usadas con mucha frecuencia en lo que sigue,
  es importante familiarizarse con ellas.

  Requeriremos alguna terminología
  para tratar con subconjuntos de \(\mathbb{C}\).
  Si \(w, z \in \mathbb{C}\),
  entonces \(\lvert z - w \rvert\) es la distancia en el plano
  entre esos puntos.
  Si fijamos un número complejo \(a\) y un real positivo \(r\),
  el conjunto \(\lvert z - a \rvert = r\)
  es la circunferencia de radio \(r\) alrededor de \(a\).
  Al interior del círculo
  se le llama el \emph{disco abierto} de radio \(r\)%
    \index{disco abierto}
  alrededor de \(a\),
  que anotaremos \(D_r(a)\).
  Más precisamente,
  \(D_r(a) = \{z \in \mathbb{C} \colon \lvert z - a \rvert < r\}\).
  Nótese que esto no incluye la circunferencia.

  \begin{definition}
    Sea \(E\) un subconjunto cualquiera de \(\mathbb{C}\).
    \begin{enumerate}[label=(\roman*), ref=(\roman*)]
    \item
      \index{punto interior}
      Un punto \(a\) es un \emph{punto interior} de \(E\)
      si hay algún disco abierto \(D_r(a)\)
      que está completamente en \(E\)
    \item
      \index{punto frontera}
      Un punto \(b\) es un \emph{punto frontera} de \(E\)
      si todo disco abierto \(D_r(b)\) contiene un punto de \(E\)
      y un punto que no pertenece a \(E\).
    \item
      \index{punto de acumulacion@punto de acumulación}
      Un punto \(c\) es un \emph{punto de acumulación} de \(E\)
      si todo disco abierto \(D_r(c)\) contiene un punto de \(E\)
      diferente de \(c\)
    \item
      \index{punto aislado}
      Un punto \(d\) es un \emph{punto aislado}	 de \(E\)
      si pertenece a \(E\) y algún disco abierto \(D_r(d)\)
      no contiene ningún punto de \(E\) excepto \(d\)
    \end{enumerate}
  \end{definition}
  En lo anterior \(a\) pertenece a \(E\),
  pero \(b\) y \(c\) no necesariamente pertenecen a \(E\).
  Desde un punto interior
  podemos movernos un poco en cualquier dirección
  sin salir de \(E\),
  de un punto frontera moviéndonos un poco quedamos dentro de \(E\),
  pero otros movimientos arbitrariamente pequeños
  nos dejan al exterior.
  Como el nombre indica,
  un punto aislado está desconectado del resto del conjunto,
  hay un entorno de él que no contiene otros puntos del conjunto.
  \begin{definition}
    \index{conjunto!abierto|see{topología}}
    \index{conjunto!cerrado|see{topología}}
    Un conjunto es \emph{abierto} si todos sus puntos son internos,
    y es \emph{cerrado} si incluye todos sus puntos frontera.
  \end{definition}
  Como	ejemplos,
  para \(r > 0\) y \(z_0 \in \mathbb{C}\)
  los conjuntos
    \(\{z \in \mathbb{C} \colon \lvert z - z_0 \rvert < r\}\),
  \(\{z \in \mathbb{C} \colon \lvert z - z_0 \rvert > r\}\)
  y \(\{x + \mathrm{i} y \in \mathbb{C} \colon -1 < x < 1\}\)
  son abiertos.
  El conjunto
    \(\{ x + \mathrm{i} y \in \mathbb{C}
	   \colon -1 \le x \le 1 \wedge -5 \le y \le 5\}\)
  es cerrado.
  Los conjuntos \(\varnothing\) y \(\mathbb{C}\) son abiertos,
  pero también son cerrados
  (no tienen puntos frontera,
   con lo que vacuamente incluyen sus fronteras).
  El conjunto
    \(\{ x + \mathrm{i} y \colon 0 \le x \le 1 \wedge 0 < y < 1 \}\)
  no es abierto ni cerrado.
  \begin{definition}
    La \emph{frontera} de \(E\),%
      \index{frontera|see{topología}}
    anotada \(\partial E\),
    es el conjunto de los puntos frontera de \(E\).
    La \emph{clausura} de \(E\),%
      \index{clausura|see{topología}}
    anotada \(\overline{E}\),
    es el conjunto \(E\) junto con su frontera.
  \end{definition}
  Para el disco abierto
    \(D_r(z_0)
	= \{ z \in \mathbb{C} \colon\lvert z - z_0 \rvert < r \}\)
  la frontera es
    \(\partial D_r(z_0)
	= \{ z \in \mathbb{C} \colon \lvert z - z_0 \rvert = r \}\),
  y su clausura es
    \(\overline{D}_r(z_0)
	= \{ z \in \mathbb{C}
	     \colon \lvert z - z_0 \rvert \le r \}\).
  Un tema un tanto sutil en los complejos
  es la idea de \emph{conectividad}.%
      \index{conectividad|see{topología}}
  Intuitivamente,
  un conjunto es conexo si es ``una sola pieza''.
  En los reales un conjunto es conexo
  exactamente si es un único intervalo,
  lo que no tiene mucho interés.
  En un plano hay gran variedad de conjuntos conexos,
  y se requiere una definición precisa.
  \begin{definition}
    Dos conjuntos \(X, Y \subseteq \mathbb{C}\)
    se dicen \emph{separados}
    si hay conjuntos abiertos \(A\) y \(B\) disjuntos
    tales que \(X \subseteq A\) y \(Y \subseteq B\).
    El conjunto \(D \subseteq \mathbb{C}\) es \emph{conexo}
    si es imposible hallar conjuntos abiertos disjuntos
    tales que \(D\) es su unión.
    Una \emph{región} es un conjunto conexo abierto.
  \end{definition}
  Por ejemplo,
  los intervalos \([0, 1)\) y \((1, 2]\) en el eje real
  están separados
  (hay infinitas posibilidades para \(X\) e \(Y\) de la definición,
   por ejemplo \(X = D_1(0)\) e \(Y = D_1(2)\)).

  Un tipo de conjunto conexo
  que usaremos con frecuencia es la curva.
  \begin{definition}
    \index{camino|see{topología}}
    \index{curva|see{topología}}
    Un \emph{camino} o \emph{curva} en \(\mathbb{C}\)
    es la imagen de una función continua
      \(\gamma \colon [a, b] \rightarrow \mathbb{C}\),
    donde \([a, b]\) es un intervalo cerrado en \(\mathbb{R}\).
    Acá la continuidad
    se refiere a que \(t \mapsto x(t) + \mathrm{i} y(t)\),
    y que tanto \(x\) como \(y\) son continuas.
    La curva se dice \emph{suave}
    si ambas componentes son diferenciables.
  \end{definition}
  Decimos que la curva está \emph{parametrizada} por \(\gamma\),
  y en un abuso común de la notación usaremos \(\gamma\)
  para referirnos a la curva.
  La curva se dice \emph{cerrada} si \(\gamma(a) = \gamma(b)\),
  y es una \emph{curva simple cerrada} si \(\gamma(a) = \gamma(b)\)%
    \index{curva simple cerrada}
  y \(\gamma(s) = \gamma(t)\) solo si \(s = t\),
  \(s = a\) y \(t = b\),
  o \(s = b\) y \(t = a\).
  Vale decir,
  la curva no se intersecta a sí misma,
  solo coinciden los puntos inicial y final.

  Lo siguiente es intuitivamente obvio,
  pero requiere ahondar bastante para demostrarse:
  \begin{theorem}
    \label{theo:curve=connected}
    Toda curva en \(\mathbb{C}\) es conexa.
  \end{theorem}
  Es claro que \(\gamma \colon [0, 1] \rightarrow \mathbb{C}\)
  con \(\gamma(t) = z_0 + t (z_1 - z_0)\)
  define un segmento de una recta en \(\mathbb{C}\)
  que va de \(z_0\) a \(z_1\).
  Al segmento de la recta \(z_0 z_1\) así parametrizada
  la anotaremos \([z_0, z_1]\).
  Podemos definir una curva formada
  por los segmentos \(z_0 z_1\),
  \(z_1 z_2\),
  \ldots,
  \(z_{n - 1} z_n\),
  que anotaremos \([z_0, z_1, \dotsc, z_n]\).
  A tales curvas las llamaremos \emph{poligonales}.
  La parametrización quedará a cargo del amable lector.
  Un teorema intuitivamente obvio,
  pero cuya demostración tiene sus sutilezas,
  es el siguiente:
  \begin{theorem}
    \label{theo:connected=curve-inside}
    Si \(D\) es un subconjunto de \(\mathbb{C}\)
    tal que cualquier par de puntos en \(D\)
    pueden conectarse mediante una curva en \(D\)
    entonces \(D\) es conexo.
    Por el otro lado,
    si \(D\) es un subconjunto abierto conexo de \(\mathbb{C}\)
    entonces cualquier par de puntos de \(D\)
    pueden conectarse mediante una curva en \(D\),
    incluso es posible conectarlos mediante una curva poligonal.
  \end{theorem}
  Un teorema central,
  bastante difícil de demostrar en su generalidad
  (ver Jordan~%
    \cite[páginas~587-594]{jordan87:_cours_analyse}
   para la demostración original)
  es el siguiente:
  \begin{theorem}[Jordan]
    \label{theo:Jordan}
    \index{Jordan, teorema de}
    Sea \(\gamma\) una curva suave simple cerrada.
    Entonces el complemento de \(\gamma\)
    consiste exactamente de dos componentes conexos.
    Uno de estos componentes es acotado
    (el \emph{interior} de \(\gamma\)),
    el otro no es acotado
    (el \emph{exterior} de \(\gamma\)).
  \end{theorem}
  Por este teorema comúnmente se les llama \emph{curvas de Jordan}%
    \index{Jordan, curva de|see{curva simple cerrada}}
  a las curvas simples cerradas.

\section{Límites y derivadas}
\label{sec:limites-derivadas}

  Si \(z\) es una variable compleja,
  y \(f(z)\) alguna función de la misma,
  el límite se define igual que en los reales.%
    \index{C (numeros complejos)@\(\mathbb{C}\) (números complejos)!limite@límite}
  Decimos que
  \begin{equation*}
    \lim_{z \rightarrow z_0} f(z) = \omega
  \end{equation*}
  si para todo \(\epsilon > 0\) existe \(\delta > 0\) tal que:
  \begin{equation}
    \label{eq:complex-limit}
    0 < \lvert z - z_0 \rvert < \delta
      \implies \lvert f(z) - \omega \rvert < \epsilon
  \end{equation}
  Formalmente es idéntica a la definición para los reales,
  pero debe tenerse presente
  que acá \(\lvert z - z_0 \rvert < \delta\)
  representa un círculo alrededor de \(z_0\).
  Esto suele describirse diciendo que el límite debe ser el mismo,
  independiente del camino que siga \(z = x + \mathrm{i} y\)
  para acercarse a \(z_0 = x_0 + \mathrm{i} y_0\).
  Definimos que \(f(z)\) es \emph{continua} en \(z_0\) si
  \begin{equation*}
    \lim_{z \rightarrow z_0} f(z)
      = f(z_0)
  \end{equation*}
  Si \(f\) es continua en todos los puntos en que está definida
  decimos simplemente que es continua.
  Si \(z = x + \mathrm{i} y\),
  \(z_0 = x_0 + \mathrm{i} y_0\)
  y \(f(z) = u(x, y) + \mathrm{i} v(x, y)\)
  (como por ejemplo \(f(z) = z^2 = x^2 - y^2 + 2 x y \mathrm{i}\)) ,
  es fácil ver que \(f(z)\) es continua si y solo si
  lo son \(u(x, y)\) y \(v(x, y)\).
  \begin{lemma}
    \label{lem:complex-limits}
    Si \(\lim_{z \rightarrow z_0} f(z)\)
    y \(\lim_{z \rightarrow z_0} g(z)\)
    existen,
    tenemos las siguientes propiedades de los límites:
    \begin{enumerate}[label=(\roman*), ref=(\roman*)]
    \item
      \(\displaystyle \lim_{z \rightarrow z_0} c f(z)
	= c \lim_{z \rightarrow z_0} f(z)\)
    \item
      \(\displaystyle \lim_{z \rightarrow z_0} (f(z) + g(z))
	= \lim_{z \rightarrow z_0} f(z)
	    + \lim_{z \rightarrow z_0} g(z)\)
    \item
      \(\displaystyle \lim_{z \rightarrow z_0} (f(z) \cdot g(z))
	= \lim_{z \rightarrow z_0} f(z)
	    \cdot \lim_{z \rightarrow z_0} g(z)\)
    \item
      Siempre que \(\lim_{z \rightarrow z_0} g(z) \ne 0\)
      es:\\[0.3ex]
      \(\displaystyle \lim_{z \rightarrow z_0} \frac{f(z)}{g(z)}
	= \frac{\lim_{z \rightarrow z_0} f(z)}
	       {\lim_{z \rightarrow z_0} g(z)}\)
    \end{enumerate}
  \end{lemma}
  La demostración es simple,
  y quedará de ejercicio.
  De acá es inmediato que la suma, diferencia, producto y cociente
  de funciones continuas son continuas
  (siempre que no tengamos un denominador cero,
   claro está).

  Dada una función compleja \(f(z)\)
  definimos su \emph{derivada} en \(z = z_0\)%
    \index{C (numeros complejos)@\(\mathbb{C}\) (números complejos)!derivada}
  como el siguiente límite,
  si existe:
  \begin{equation}
    \label{eq:complex-derivative}
    f'(z_0)
      = \lim_{h \rightarrow 0} \frac{f(z_0 + h) - f(z_0)}{h}
  \end{equation}
  Una notación alternativa común es:
  \begin{equation*}
    \frac{\mathrm{d} f}{\mathrm{d} z}
      = f'(z)
  \end{equation*}
  La condición que el límite sea el mismo,
  independiente del camino seguido por \(h\)
  para aproximarse a cero,
  hace que para
    \(f(x + \mathrm{i} y) = u(x, y) + \mathrm{i} v(x, y)\)
  con \(\Delta x\) y \(\Delta y\) reales deba ser:
  \begin{align}
    \lim_{\Delta x \rightarrow 0}
       \frac{f(z_0 + \Delta x) - f(z_0)}{\Delta x}
      &= \lim_{\Delta y \rightarrow 0}
	   \frac{f(z_0 + \mathrm{i} \, \Delta y) - f(z_0)}
		{\mathrm{i} \, \Delta y}
	     \notag \\
    \frac{\partial u}{\partial x}
      + \mathrm{i} \, \frac{\partial v}{\partial x}
      &= - \mathrm{i} \, \frac{\partial u}{\partial y}
	     + \frac{\partial v}{\partial y}
	     \label{eq:complex-derivative-uv}
  \end{align}
  Las expresiones~\eqref{eq:complex-derivative-uv}
  expresan la derivada compleja en términos de las coordenadas.

  Igualando partes reales y complejas,
  resultan las \emph{ecuaciones de Cauchy-Riemann}:
  \begin{equation}
    \label{eq:Cauchy-Riemann}
    \index{Cauchy-Riemann, ecuaciones de|textbfhy}
    \begin{split}
      \frac{\partial u}{\partial x}
	&= \frac{\partial v}{\partial y} \\
      \frac{\partial u}{\partial y}
	&= - \frac{\partial v}{\partial x}
    \end{split}
  \end{equation}
  Esto ya demuestra que hay condiciones fuertes
  para que una función tenga derivada en un punto.
  Resulta que las ecuaciones~\eqref{eq:Cauchy-Riemann}
  junto con continuidad de las derivadas
  son condiciones necesarias y suficientes
  para que \(f(x + \mathrm{i} y) = u(x, y) + \mathrm{i} v(x, y)\)
  tenga derivada en \(z_0 = x_0 + \mathrm{i} y_0\).

  Si la función \(f\) tiene derivada en \(z_0\),
  se dice que es \emph{diferenciable} en \(z_0\).
  A una función diferenciable en todo punto en una región abierta
  se le llama \emph{holomorfa}%
    \index{C (numeros complejos)@\(\mathbb{C}\) (números complejos)!funcion holomorfa@función holomorfa|textbfhy}
  (mucha literatura erróneamente se refiere a ellas
   como \emph{funciones analíticas},
   un concepto relacionado).
  Una función holomorfa sobre todo \(\mathbb{C}\)
  se llama \emph{entera}.%
    \index{C (numeros complejos)@\(\mathbb{C}\) (números complejos)!funcion entera@función entera|textbfhy}

  Las siguientes propiedades de la derivada
  se demuestran básicamente
  cambiando \(x\) por \(z\)
  en las demostraciones respectivas para los reales:
  \begin{lemma}
    \label{lem:complex-derivative-rules}
    Sean \(f\) y \(g\) diferenciables en \(z \in \mathbb{C}\),
    sea \(c \in \mathbb{C}\),
    sea \(n \in \mathbb{Z}\),
    y sea \(h\) diferenciable en \(g(z)\).
    Entonces:
    \begin{enumerate}[label=(\roman*), ref=(\roman*)]
    \item
      \((c f(z))'
	  = c f'(z)\)
    \item
      \((f(z) + g(z))'
	  = f'(z) + g'(z)\)
    \item
      \((f(z) \cdot g(z))'
	  = f'(z) \cdot g(z) + f(z) \cdot g'(z)\)
    \item
      Siempre que \(g(z) \ne 0\) es
      \((f(z) / g(z))'
	  = (f'(z) \cdot g(z) - f(z) \cdot g'(z)) / g^2 (z)\)
    \item
      \((z^n)'
	  = n z^{n - 1}\)
    \item
      \((h(g(z)))'
	  = h'(g(z)) \cdot g'(z)\)
    \end{enumerate}
  \end{lemma}
  Un último resultado se refiere a funciones inversas.
  \begin{lemma}
    \label{lem:complex-derivative-inverse}
    Sean \(G\) y \(H\) conjuntos abiertos en \(\mathbb{C}\),
    \(f \colon G \rightarrow H\) una biyección
    con inversa \(g \colon H \rightarrow G\),
    y sea \(z_0 \in H\).
    Si \(f\) es diferenciable en \(g(z_0)\)
    con \(f'(g(z_0)) \ne 0\),
    y \(g\) es continua en \(z_0\),
    entonces \(g\) es diferenciable en \(z_0\),
    y:
    \begin{equation*}
      g'(z_0)
	= \frac{1}{f'(g(z_0))}
    \end{equation*}
  \end{lemma}
  \begin{proof}
    Por definición:
    \begin{equation*}
      g'(z_0)
	= \lim_{z \rightarrow z_0} \frac{g(z) - g(z_0)}{z - z_0}
	= \lim_{z \rightarrow z_0}
	    \frac{g(z) - g(z_0)}{f(g(z)) - f(g(z_0))}
	= \lim_{z \rightarrow z_0}
	    \frac{1}{\frac{f(g(z)) - f(g(z_0))}{g(z) - g(z_0)}}
    \end{equation*}
    Dado que \(g\) es continua en \(z_0\),
    \(g(z) \rightarrow g(z_0)\) cuando \(z \rightarrow z_0\),
    lo que da:
    \begin{equation*}
      g'(z_0)
	= \lim_{w \rightarrow g(z_0)}
	    \frac{1}{\frac{f(w) - f(g(z_0))}{w - g(z_0)}}
    \end{equation*}
    El denominador es continuo y diferente de cero,
    por el lema~\ref{lem:complex-limits} tenemos lo prometido.
  \end{proof}
  Un resultado importante es:
  \begin{theorem}
    \label{theo:complex-zero-derivative}
    Si \(f'(z) = 0\) para todo \(z\) en una región \(D\),
    entonces \(f(z)\) es constante en \(D\).
  \end{theorem}
  Elegimos un punto fijo \(z_0 \in D\),
  y conectaremos un punto arbitrario \(z \in D\) con \(z_0\)
  mediante una poligonal,
  y demostraremos que \(f\) es constante sobre esa poligonal.
  Como \(z\) es arbitrario,
  \(f\) es constante en \(D\).
  \begin{proof}
    Sean \(z_0, z \in D\).
    Por el teorema~\ref{theo:connected=curve-inside}
    hay una poligonal \([z_0, z_1, \dotsc, z]\) en \(D\)
    que los conecta.
    Sea \(f(z) = u + \mathrm{i} v\),
    si \(f'(z) = 0\) de las ecuaciones de Cauchy-Riemann
    es:
    \begin{equation*}
      \frac{\partial u}{\partial x}
	= \frac{\partial u}{\partial y}
	= \frac{\partial v}{\partial x}
	= \frac{\partial v}{\partial y}
	= 0
    \end{equation*}
    Considerando uno de los tramos,
    digamos \([z_k, z_{k + 1}]\)
    vemos que esta recta
    queda descrita por una parametrización de la forma
    \(z_k + (t - t_k) (z_{k + 1} - z_k) / (t_{k + 1} - t_k)
       = \alpha t + \beta\).
    Resulta que la derivada de \(f\) respecto a \(t\)
    a lo largo de \([z_k, z_{k + 1}]\)
    se anula:
    \begin{equation*}
      \frac{f(\alpha (t + h) + \beta)
	     - f(\alpha t + \beta)}
	   {h}
	= \frac{\partial u}{\partial x}
	     \frac{\mathrm{d} x}{\mathrm{d} t}
	     + \mathrm{i} \,
		 \frac{\partial v}{\partial x}
		 \frac{\mathrm{d} y}{\mathrm{d} t}
	= 0
    \end{equation*}
    Acá usamos la fórmula~\eqref{eq:complex-derivative-uv}
    para la derivada compleja.
    Podemos aplicar el teorema del valor medio para funciones reales
    (componente a componente)
    a \(f\) como función de \(t\).
    Como la derivada se anula,
    esto nos dice que no hay cambios:
    \begin{equation*}
      f(z_{k + 1}) - f(z_k)
	= 0
    \end{equation*}
    Esto es lo que queríamos demostrar.
  \end{proof}

\section{Funciones elementales}
\label{sec:complex-elementary-functions}
\index{C (numeros complejos)@\(\mathbb{C}\) (números complejos)!funciones elementales}

  Es claro que los polinomios son funciones enteras,
  y si se restringen a argumentos reales
  son simplemente las funciones conocidas.
  Un poquito más delicado es el caso de funciones racionales,
  como:
  \begin{equation*}
    \frac{z^3 - 3 z + 2}{z^2 + 1}
  \end{equation*}
  Esta función es holomorfa
  salvo en los puntos \(z = \pm \mathrm{i}\),
  donde el denominador se anula.
  Como función real es continua y tiene derivada en todas partes.

  Veamos la función exponencial.
  Parece razonable extender la propiedad básica
    \(\mathrm{e}^{x_1} \cdot \mathrm{e}^{x_2}
	= \mathrm{e}^{x_1 + x_2}\)
  a argumentos complejos,
  lo que para \(x, y \in \mathbb{R}\) da:
  \begin{equation*}
    \mathrm{e}^{x + \mathrm{i} y}
      = \mathrm{e}^x \cdot \mathrm{e}^{\mathrm{i} y}
  \end{equation*}
  Esto,
  con la convención~\eqref{eq:imaginary-exponential},
  resulta en:
  \begin{definition}
    Para \(x, y \in \mathbb{R}\) definimos:
    \begin{equation}
      \index{C (numeros complejos)@\(\mathbb{C}\) (números complejos)!exponencial|textbfhy}
      \label{eq:complex-exponential}
      \mathrm{e}^{x + \mathrm{i} y}
	= \mathrm{e}^x ( \cos y + \mathrm{i} \, \sin y )
    \end{equation}
  \end{definition}
  Escribiendo:
  \begin{equation*}
    \mathrm{e}^z
      = u(x, y) + \mathrm{i} v(x, y)
  \end{equation*}
  vemos que se satisfacen
  las ecuaciones de Cauchy-Riemann~\eqref{eq:Cauchy-Riemann}%
    \index{Cauchy-Riemann, ecuaciones de}
  y que las derivadas son continuas para todo \(z \in \mathbb{C}\).
  Esta función es entera.
  Vemos también que:
  \begin{equation}
    \label{eq:complex-exponential-derivative}
    \frac{\mathrm{d}}{\mathrm{d} z} \mathrm{e}^z
      = \frac{\partial u}{\partial x}
	  + \mathrm{i} \, \frac{\partial v}{\partial x}
      = \mathrm{e}^x \cos y + \mathrm{i} \, \mathrm{e}^x \sin y
      = \mathrm{e}^z
  \end{equation}
  Ya vimos que
    \(\mathrm{e}^{\mathrm{i} y_1} \cdot \mathrm{e}^{\mathrm{i} y_2}
	= \mathrm{e}^{\mathrm{i} (y_1 + y_2)}\),
  y
    \(\mathrm{e}^{x_1} \cdot \mathrm{e}^{x_2}
	= \mathrm{e}^{x_1 + x_2}\),
  que en conjunto hacen que
    \(\mathrm{e}^{z_1} \cdot \mathrm{e}^{z_2}
	= \mathrm{e}^{z_1 + z_2}\),
  como buscábamos.
  Notamos que:
  \begin{equation}
    \label{eq:complex-exponential-modulus}
    \left\lvert \mathrm{e}^z \right\rvert
      = \left\lvert \mathrm{e}^x \right\rvert \cdot
	  \left\lvert \cos y + \mathrm{i} \, \sin y \right\rvert
      = \left\lvert \mathrm{e}^x \right\rvert \cdot
	  \left( \cos^2 y + \sin^2 y \right)^{1/2}
      = \mathrm{e}^x
  \end{equation}
  Como \(\mathrm{e}^x\) nunca se anula para \(x \in \mathbb{R}\),
  y \(\cos y\) y \(\sin y\) no se anulan juntas,
  \(\mathrm{e}^z \ne 0\) para todo \(z \in \mathbb{C}\).

  En nuestra lista siguen las funciones trigonométricas.%
    \index{C (numeros complejos)@\(\mathbb{C}\) (números complejos)!funciones trigonometricas@funciones trigonométricas}
  La convención~\eqref{eq:imaginary-exponential}
  para \(\pm \phi\) da:
  \begin{equation*}
    \mathrm{e}^{\mathrm{i} \, \phi}
      = \cos \phi + \mathrm{i} \, \sin \phi
    \qquad
    \mathrm{e}^{- \mathrm{i} \, \phi}
      = \cos \phi - \mathrm{i} \, \sin \phi
  \end{equation*}
  De este sistema de ecuaciones:
  \begin{equation*}
    \cos \phi
      = \frac{\mathrm{e}^{\mathrm{i} \, \phi}
		+ \mathrm{e}^{- \mathrm{i} \, \phi}}
	     {2}
    \qquad
    \sin \phi
      = \frac{\mathrm{e}^{\mathrm{i} \, \phi}
		- \mathrm{e}^{- \mathrm{i} \, \phi}}
	     {2 \mathrm{i}}
  \end{equation*}
  lo que sugiere definir:
  \begin{align}
    \cos z
      &= \frac{\mathrm{e}^{\mathrm{i} z} + \mathrm{e}^{- \mathrm{i} z}}{2}
	    \label{eq:complex-cos} \\
    \sin z
      &= \frac{\mathrm{e}^{\mathrm{i} z} - \mathrm{e}^{- \mathrm{i} z}}
	      {2 \mathrm{i}}
	    \label{eq:complex-sin}
  \end{align}
  Es claro que estas funciones son enteras,
  y cumplen las identidades trigonométricas conocidas
  para los reales.
  Cuidado,
  estas funciones no son acotadas en \(\mathbb{C}\).
  El lector escéptico podrá entretenerse
  demostrando algunas identidades,
  como \(\cos^2 z + \sin^2 z = 1\)
  o las fórmulas para sumas de ángulos.

  De las relaciones~\eqref{eq:complex-cos} y~\eqref{eq:complex-sin}
  vemos que para las funciones hiperbólicas:%
    \index{C (numeros complejos)@\(\mathbb{C}\) (números complejos)!funciones hiperbolicas@funciones hiperbólicas}
  \begin{align}
    \cosh z
      &= \frac{\mathrm{e}^z + \mathrm{e}^{- z}}{2}
       = \cos \mathrm{i} z
	    \label{eq:complex-cosh} \\
    \sinh z
      &= \frac{\mathrm{e}^z  - \mathrm{e}^{- z}}{2}
       = - \mathrm{i} \, \sin \mathrm{i} z
	    \label{eq:complex-sinh}
  \end{align}
  Estas funciones también son enteras.
  Podemos definir las demás funciones trigonométricas
  e hiperbólicas usando las mismas definiciones que para los reales.

\section{Logaritmos y potencias}
\label{sec:complex-logarithm}

  Entre los reales,
  el logaritmo es simplemente el inverso de la exponencial.
  Esto está perfectamente bien definido en ese caso.
  Entre los complejos,
  sin embargo,
  la exponencial es una función periódica
  (el período es \(2 \pi \mathrm{i}\)).
  Como hay infinitas soluciones a la ecuación \(\mathrm{e}^z = w\)
  siempre que \(w \ne 0\),
  no podemos esperar definir una función análoga,
  deberemos dar algunos rodeos.
  En particular,
  para \(x \in \mathbb{R}\) con \(x > 0\)
  realmente es \(\log x = \ln x + 2 k \pi \mathrm{i}\),%
    \index{C (numeros complejos)@\(\mathbb{C}\) (números complejos)!logaritmo}
  ya que debemos considerar los posibles argumentos.
  Acá \(\ln x\) es el familiar logaritmo natural de los reales.
  En los reales
  simplemente dejamos de lado la componente imaginaria.

  Para \(z \ne 0\) definimos:
  \begin{equation}
    \label{eq:log-definition}
    \log z
      = \ln \lvert z \rvert + \mathrm{i} \, \arg z
  \end{equation}
  Con esto tenemos el caso emblemático:
  \begin{equation*}
    \log(-1)
      = \ln 1 + \mathrm{i} \, \arg(-1)
      = (2 k + 1) \pi \mathrm{i}
  \end{equation*}
  Esto cumple el familiar:
  \begin{equation*}
    \mathrm{e}^{\log z}
      = \mathrm{e}^{\ln \lvert z \rvert + \mathrm{i} \, \arg z}
      = \mathrm{e}^{\ln \lvert z \rvert}
	  \cdot \mathrm{e}^{\mathrm{i} \, \arg z}
      = z
  \end{equation*}
  Pero aparece una complicación.
  Con el ya tradicional \(z = x + \mathrm{i} y\) tenemos:
  \begin{equation*}
    \log \left( \mathrm{e}^z \right)
      = \ln \mathrm{e}^x + \mathrm{i} \, \arg \mathrm{e}^z
      = x + (y + 2 k \pi) \mathrm{i}
      = z + 2 k \pi \mathrm{i}
  \end{equation*}
  Acá \(k\) es un entero cualquiera.
  De la misma manera,
  para un entero cualquiera \(k\):
  \begin{align*}
    \log (w z)
      &= \ln (\lvert w \rvert \cdot \lvert z \rvert)
	   + \mathrm{i} \, \arg (w z)
       = \ln \lvert w \rvert + \mathrm{i} \, \arg w
	   + \ln \lvert z \rvert + \mathrm{i} \, \arg z
	   + 2 k \pi \mathrm{i} \\
      &= \log w + \log z + 2 k \pi \mathrm{i}
  \end{align*}
  Definimos la \emph{rama principal} del logaritmo mediante:%
    \index{C (numeros complejos)@\(\mathbb{C}\) (números complejos)!logaritmo!rama principal}
  \begin{equation}
    \label{eq:log-principal-branch}
    \Log z
      = \ln \lvert z \rvert + \mathrm{i} \, \Arg z
  \end{equation}
  Si \(z = x\),
  un real positivo,
  es:
  \begin{equation*}
    \Log x
      = \ln x + \mathrm{i} \, \Arg x
      = \ln x
  \end{equation*}
  Vemos que la nueva función es una extensión del logaritmo real.
  La definición del argumento principal con este rango,
  en vez del mucho más natural \(0 \le \phi < 2 \pi\),
  es precisamente para asegurar esta coincidencia
  sin estar equilibrándonos en el borde del abismo
  a lo largo de la línea real.

  La función \(\Log\) es holomorfa en muchas partes.
  No está definida para \(z = 0\),
  y tiene un corte en la línea real negativa.
  Sea \(z_0 = x_0 + \mathrm{i} y_0\)
  tal que \(\Log z_0\) esté definido,
  y veamos su derivada:
  \begin{equation*}
    \lim_{z \rightarrow z_0} \frac{\Log z - \Log z_0}{z - z_0}
      = \lim_{z \rightarrow z_0}
	  \frac{\Log z - \Log z_0}
	       {\mathrm{e}^{\Log z} - \mathrm{e}^{\Log z_0}}
  \end{equation*}
  Deberemos restringirnos a trabajar en una región
  que no incluya los puntos conflictivos mencionados.
  Con \(w = \Log z\) y \(w_0 = \Log z_0\),
  notando que \(w \rightarrow w_0\)
  cuando \(z \rightarrow z_0\),
  esto es:
  \begin{align*}
    \lim_{z \rightarrow z_0} \frac{\Log z - \Log z_0}{z - z_0}
      &= \lim_{w \rightarrow w_0}
	   \frac{w - w_0}{\mathrm{e}^w - \mathrm{e}^{w_0}}
       = \frac{1}{e^{w_0}} \\
      &= \frac{1}{z_0}
  \end{align*}

  \begin{figure}[ht]
    \centering
    \pgfimage{images/log-domain}
    \caption{Dominio alternativo de $\log z$}
    \label{fig:log-domain}
  \end{figure}
  Nótese que la restricción del argumento es un tanto arbitraria,
  si nos interesa trabajar en la región \(D\)
  de la figura~\ref{fig:log-domain},
  podemos restringir el argumento
  al rango \([\pi / 4,	9 \pi / 4)\).

  Ahora estamos en condiciones
  de definir potencias arbitrarias de \(z\).%
    \index{C (numeros complejos)@\(\mathbb{C}\) (números complejos)!potencia}
  La definición obvia es:
  \begin{equation*}
    z^c
      = \mathrm{e}^{c \log z}
  \end{equation*}
  Hay muchos valores de \(\log z\),
  con lo que pueden haber muchos valores de \(z^c\).
  Al lector atento no le extrañará
  que se le llame el \emph{valor principal} de \(z^c\)
  a \(\mathrm{e}^{c \Log z}\).%
    \index{C (numeros complejos)@\(\mathbb{C}\) (números complejos)!potencia!rama principal}
  En caso que \(c = n\),
  un entero,
  la definición da:
  \begin{align*}
    z^n
      &= \mathrm{e}^{n \log z}
       = \mathrm{e}^{n (\Log z + 2 k \pi \mathrm{i})}
       = \mathrm{e}^{n \Log z}
	   \cdot \mathrm{e}^{2 n k \pi \mathrm{i}} \\
      &= \mathrm{e}^{n \Log z} \\
      &= \lvert z \rvert^n \cdot \mathrm{e}^{\mathrm{i} n \Arg z}
  \end{align*}
  Exactamente como debiera ser.
  El lector interesado verificará que si \(c\) es un racional,
  la fórmula da todos los valores esperados.

  Esto introduce un nuevo problema.
  Tenemos una definición de potencias que aplicada a \(z = e\)
  puede dar infinitos valores para \(\mathrm{e}^c\).
  Hasta acá hemos asumido simplemente
  que para \(z = x + \mathrm{i} y\)
  es:
  \begin{equation*}
    \mathrm{e}^z
      = \exp(z)
      = \mathrm{e}^x (\cos y + \mathrm{i} \, \sin y)
  \end{equation*}
  Equivalentemente,
  \(\mathrm{e}^z\) se refiere al valor principal de esta expresión.
  Esta es la convención que adoptaremos,
  que por lo demás es universal.

\section{Integrales}
\label{sec:complex-integrals}
\index{C (numeros complejos)@\(\mathbb{C}\) (números complejos)!integral}

  La integral en los reales
  es simplemente a lo largo de la línea real.
  Al integrar en los complejos
  hay muchos caminos distintos que podemos seguir.
  De todas formas,
  una definición natural para la integral de la función \(f\)
  a lo largo del camino suave
    \(\gamma \colon [a, b] \rightarrow \mathbb{C}\)
  es seguir la definición de la integral de Riemann en los reales.
  Imaginemos una subdivisión del rango \([a, b]\) en \(n\) tramos
  \([t_{k - 1}, t_k]\),
  donde \(1 \le k \le n\).
  Vemos que al tramo \([t_{k - 1}, t_k]\)
  corresponde un arco \([z_{k - 1}, z_k]\) del camino \(\gamma\),
  un punto \(t_k^*\) en el tramo \([t_{k - 1}, t_k]\)
  da un punto \(z_k^*\) en el arco correspondiente.
  Supongamos ahora dado \(\epsilon > 0\) cualquiera.
  Dada una partición \(P\) tal que se cumple para todo tramo
  que \(\lvert z_k - z_{k - 1} \rvert < \epsilon\),
  eligiendo puntos \(t_k^*\) en cada tramo
  calculamos la suma:
  \begin{equation*}
    S(P)
      = \sum_{1 \le k \le n} f(z_k^*) (z_k - z_{k - 1})
  \end{equation*}
  Si estas sumas tienden al valor \(L\)
  cuando \(\epsilon \rightarrow 0\),
  llamamos a este límite el valor de la integral,
  y anotamos:
  \begin{equation*}
    \int_\gamma f(z) \, \mathrm{d} z
      = L
  \end{equation*}
  Podemos expresar la suma en términos de \(t\):
  \begin{align*}
    S(p)
      &= \sum_{1 \le k \le n}
	   f(\gamma(t_k^*)) (\gamma(t_k^*) - \gamma(t_{k - 1}^*)) \\
      &= \sum_{1 \le k \le n}
	   f(\gamma(t_k^*)) (\gamma(t_k^*) - \gamma(t_{k - 1}^*)) \\
      &= \sum_{1 \le k \le n}
	   f(\gamma(t_k^*)) \,
	     \frac{\gamma(t_k^*) - \gamma(t_{k - 1}^*)}
		  {t_k^* - t_{k - 1}^*}
	     \cdot (t_k^* - t_{k - 1}^*)
  \end{align*}
  Si \(\epsilon \rightarrow 0\),
  vemos que esto tiende a:
  \begin{equation}
    \label{eq:complex-integral-parametrized}
    \int_\gamma f(z) \, \mathrm{d} z
      = \int_a^b f(\gamma(t)) \gamma'(t) \, \mathrm{d} t
  \end{equation}
  Esta misma fórmula indica que el valor de la integral
  es independiente de la parametrización de la curva.

  Una cota que usaremos frecuentemente es la siguiente.
  \begin{lemma}[Cota para integrales complejas]
    \label{lem:complex-integral-bound}
    Supóngase que hay un número \(M\) tal que \(\lvert f(z) \rvert \le M\)
    para todo \(z\) en la curva suave
      \(\gamma \colon [a, b] \rightarrow \mathbb{C}\),
    y sea \(l_\gamma\) el largo de la curva \(\gamma\).
    Entonces:
    \begin{equation}
      \label{eq:complex-integral-bound}
      \left\lvert \int_\gamma f(z) \, \mathrm{d} z \right\rvert
	\le M l_\gamma
    \end{equation}
  \end{lemma}
  \begin{proof}
    Usando la desigualdad triangular
    sobre la definición de la integral,%
      \index{desigualdad triangular}
    vemos que:
    \begin{equation*}
      \left\lvert \int_\gamma f(z) \, \mathrm{d} z \right\rvert
	= \left\lvert \int_a^b f(z) \gamma'(t)
	    \, \mathrm{d} t \right\rvert
	\le \int_a^b
	      \lvert f(z) \rvert
		\cdot	\lvert \gamma'(t) \rvert \, \mathrm{d} t
	\le M \int_a^b \lvert \gamma'(t) \rvert \, \mathrm{d} t
    \end{equation*}
    Si describimos \(\gamma(t) = x(t) + \mathrm{i} y(t)\) tenemos:
    \begin{equation*}
      \int_a^b \lvert \gamma'(t) \rvert \, \mathrm{d} t
	= \int_a^b
	    \left(
	      \left( x'(t) \right)^2 + \left( y'(t) \right)^2
	    \right)^{1/2} \, \mathrm{d} t
    \end{equation*}
    que reconocemos como el largo \(l_\gamma\) de la curva.
  \end{proof}

\subsection{Integrales y antiderivadas}
\label{sec:complex-integral-antiderivative}

  Supongamos nuevamente
  un camino suave \(\gamma\) entre \(a\) y \(b\),
  una función \(g(z)\) diferenciable en \(\gamma\),
  y consideremos \(t \in [a, b]\).
  Veamos cuál es la derivada de \(g(\gamma(t))\).
  Resulta ser exactamente como nos imaginamos.
  Primero,
  con \(g(x + \mathrm{i} y) = u(x, y) + \mathrm{i} v(x, y)\)
  y \(\gamma(t) = x(t) + \mathrm{i} y(t)\),
  es:
  \begin{equation*}
    g(\gamma(t))
      = u(x(t), y(t)) + \mathrm{i} v(x(t), y(t))
  \end{equation*}
  Enseguida:
  \begin{align*}
    \frac{\mathrm{d}}{\mathrm{d} t} \, g(\gamma(t))
      &= \frac{\partial u}{\partial x} \,
	    \frac{\mathrm{d} x}{\mathrm{d} t}
	    + \frac{\partial u}{\partial y} \,
		\frac{\mathrm{d} y}{\mathrm{d} t}
	  + \mathrm{i}
	      \left(
		\frac{\partial v}{\partial x} \,
		  \frac{\mathrm{d} x}{\mathrm{d} t}
		  + \frac{\partial v}{\partial y} \,
		      \frac{\mathrm{d} y}{\mathrm{d} t}
	      \right) \\
  \intertext{Usando las ecuaciones de Cauchy-Riemann:%
	       \index{Cauchy-Riemann, ecuaciones de}}
    \frac{\mathrm{d}}{\mathrm{d} t} \, g(\gamma(t))
      &= \frac{\partial u}{\partial x} \,
	   \frac{\mathrm{d} x}{\mathrm{d} t}
	     - \frac{\partial v}{\partial x} \,
		 \frac{\mathrm{d} y}{\mathrm{d} t}
	   + \mathrm{i} \,
	       \left(
		\frac{\partial v}{\partial x} \,
		  \frac{\mathrm{d} x}{\mathrm{d} t}
		  + \frac{\partial u}{\partial x} \,
		      \frac{\mathrm{d} y}{\mathrm{d} t}
	       \right) \\
      &= \left(
	   \frac{\partial u}{\partial x}
	     + \mathrm{i} \, \frac{\partial v}{\partial x}
	 \right)
	   \cdot \left(
		   \frac{\mathrm{d} x}{\mathrm{d} t}
		     + \mathrm{i} \, \frac{\mathrm{d} y}
					  {\mathrm{d} t}
		 \right) \\
      &= g'(\gamma(t)) \gamma'(t)
  \end{align*}

  Volvamos a las integrales ahora.
  Sea una región \(D\)
  y una función \(F \colon D \rightarrow \mathbb{C}\)
  tal que en \(D\) tenemos \(F'(z) = f(z)\).
  Supongamos un camino suave \(\gamma \colon [a, b] \rightarrow D\).
  Sabemos de arriba que:
  \begin{equation*}
    \frac{\mathrm{d}}{\mathrm{d} t} \, F(\gamma(t))
      = F'(\gamma(t)) \gamma'(t)
      = f(\gamma(t)) \gamma'(t)
  \end{equation*}
  Entonces:
  \begin{equation}
    \label{eq:complex-integral-antiderivative}
    \int_\gamma f(\zeta) \, \mathrm{d} \zeta
      = \int_a^b f(\gamma(t)) \gamma'(t) \, \mathrm{d} t
      = \int_a^b \frac{\mathrm{d}}{\mathrm{d} t} \, F(\gamma(t))
	  \, \mathrm{d} t
      = F(\gamma(b)) - F(\gamma(a))
  \end{equation}
  La última relación
  resulta del teorema fundamental del cálculo integral.

  Muy agradable,
  la integral depende únicamente
  de los puntos inicial y final del camino.
  En particular,
  si el camino es cerrado,
  la integral es cero.

  El recíproco ahora.
  Supongamos que la integral de la función continua \(f\)
  no depende del camino,
  vale decir podemos tomar un punto \(z_0 \in D\)
  y definir una función:
  \begin{equation*}
    F(z)
      = \int_{\gamma_z} f(\zeta) \, \mathrm{d} \zeta
  \end{equation*}
  donde el camino \(\gamma_z\)
  comienza en \(z_0\) y termina en \(z\),
  sin salir de \(D\).
  Sabemos que de existir tales caminos para cada elección de \(z\)
  la región \(D\) debe ser conexa.
  Evaluemos la derivada de \(F\):
  \begin{equation*}
    \lim_{h \rightarrow 0}
      \frac{F(z + h) - F(z)}{h}
      = \lim_{h \rightarrow 0}
	  \frac{1}{h} \int_{\gamma_h} f(\zeta) \, \mathrm{d} \zeta
  \end{equation*}
  Acá \(\gamma_h\)
  es un camino que comienza en \(z\) y termina en \(z + h\).
  Vemos también que:
  \begin{align*}
    \int_{\gamma_h} \, \mathrm{d} \zeta
      &= h \\
    \int_{\gamma_h} f(z) \, \mathrm{d} \zeta
      &= h f(z)
  \end{align*}
  Con esto:
  \begin{equation*}
    \lim_{h \rightarrow 0}
      \frac{F(z + h) - F(z)}{h}
	- f(z)
      = \lim_{h \rightarrow 0}
	  \frac{1}{h} \, \int_{\gamma_h} (f(\zeta) - f(z))
	  \, \mathrm{d} \zeta
  \end{equation*}
  Ahora bien,
  como las integrales no dependen del camino
  podemos calcularla sobre la recta de \(z\) a \(z + h\):
  \begin{equation*}
    \left\lvert
      \frac{1}{h} \, \int_{\gamma_h} (f(\zeta) - f(z))
	\, \textrm{d} \zeta
    \right\rvert
      \le \left\lvert \frac{1}{h} \right\rvert
	    \cdot \lvert h \rvert
	    \cdot \max \left\{ \lvert f(\zeta) - f(z) \rvert
			   \colon \zeta \in [z, z + h] \right\}
  \end{equation*}
  Como \(f\) es continua,
  cuando \(h \rightarrow 0\) esto tiende a cero,
  y \(F'(z) = f(z)\),
  como esperábamos.

  En resumen:
  \begin{theorem}
    \label{theo:complex-integral=antiderivative}
    Sea \(D\) una región conexa,
    y sea \(f \colon D \rightarrow \mathbb{C}\) continua.
    Entonces \(f\) tiene antiderivada en \(D\) si y solo si
    la integral entre dos puntos de \(D\)
    es independiente del camino.
    El valor de la integral es la diferencia
    entre los valores de la antiderivada.
  \end{theorem}
  Pero también hemos demostrado:
  \begin{theorem}[Morera]
    \label{theo:Morera}
    Sea \(f\) continua en \(D\) tal que
    para toda curva suave cerrada simple \(\gamma \subset D\):
    \begin{equation*}
      \int_\gamma f(z) \, \mathrm{d} z
	= 0
    \end{equation*}
    Entonces \(f\) es holomorfa en \(D\).
  \end{theorem}

\subsection{El teorema de Cauchy}
\label{sec:Cauchy-theorem}

  Nos interesa evaluar integrales sobre caminos cerrados,
  en particular demostrar que tales integrales
  no dependen del detalle del camino.
  Para ello requeriremos algunas herramientas adicionales.
  \begin{definition}
    \index{curva cerrada!homotopica@homotópica|textbfhy}
    Sean \(\gamma_0\) y \(\gamma_1\) curvas cerradas
    en el conjunto abierto \(D \subseteq \mathbb{C}\),
    parametrizadas por \(\gamma_0 \colon [0, 1] \rightarrow D\)
    y \(\gamma_1 \colon [0, 1] \rightarrow D\),
    respectivamente.
    Decimos que \(\gamma_0\)
    es \emph{\(D\)-homotópica} a \(\gamma_1\),
    en símbolos \(\gamma_0 \sim_D \gamma_1\),
    si hay una función continua \(h \colon [0, 1]^2 \rightarrow D\)
    tal que:
    \begin{align*}
      h(t, 0)
	&= \gamma_0(t) \\
      h(t, 1)
	&= \gamma_1(t) \\
      h(0, s)
	&= h(1, s)
    \end{align*}
    Si \(D\) es conexa
    y tal que toda curva cerrada simple es homotópica
    a un punto,
    se dice que \(D\) es \emph{conexa simple}.%
      \index{C (numeros complejos)@\(\mathbb{C}\) (números complejos)!region conexa simple@región conexa simple}
  \end{definition}
  \begin{figure}[ht]
    \centering
    \pgfimage{images/homotopy}
    \caption{Ejemplos de homotopía}
    \label{fig:homotopy}
  \end{figure}
  La idea es que \(h(t, s)\) es una curva en \(D\),
  la última condición asegura que sea siempre cerrada.
  Cambiando \(s\) cambia la curva,
  que va en forma continua de \(\gamma_0\) a \(\gamma_1\).
  Nótese también que las curvas se recorren todas
  en dirección de \(t\) creciente,
  arbitrariamente definimos la dirección positiva
  como aquella en que el interior encerrado por la curva
  queda a su izquierda.
  Es simple ver que la homotopía
  es una relación de equivalencia entre curvas.
  Frecuentemente consideraremos un punto aislado como una ``curva''
  de largo cero.
  Una región conexa simple no tiene ``agujeros''.

  Mucho de lo que viene a continuación
  se basa en el siguiente teorema.
  La demostración es de Beck, Marchesi, Pixton y Sabalka~%
    \cite{beck12:_first_course_compl_analysis}.
  \begin{theorem}[Cauchy]
    \index{Cauchy, teorema de}
    Sea \(D \subseteq \mathbb{C}\) una región abierta,
    \(f\) holomorfa en \(D\),
    y \(\gamma_0 \sim_D \gamma_1\) vía una homotopía
    con segundas derivadas continuas
    y que coinciden para \(s = 0\) y \(s = 1\).
    Entonces:
    \begin{equation*}
      \int_{\gamma_0} f(z) \, \mathrm{d} z
	= \int_{\gamma_1} f(z) \, \mathrm{d} z
    \end{equation*}
  \end{theorem}
  La condición de suavidad de la homotopía puede relajarse bastante,
  pero la demostración se hace muy compleja.
  Para las aplicaciones de nuestro interés
  esta condición se cumple.
  \begin{proof}
    Sea \(h(t, s)\) la homotopía de \(\gamma_0\) a \(\gamma_1\),
    y definamos \(\gamma_s\) como la curva definida por \(h\)
    para \(s\).
    Consideremos la función:
    \begin{equation*}
      I(s)
	= \int_{\gamma_s} f(z) \, \mathrm{d} z
	= \int_0^1 f(h(t, s)) \frac{\partial h}{\partial t}
	    \, \mathrm{d} t
    \end{equation*}
    Esta expresión
    resulta de~\eqref{eq:complex-integral-parametrized}.
    Demostraremos que \(I(s)\) es constante,
    con lo que se cumple lo prometido como \(I(0) = I(1)\).
    Por la regla de Leibnitz:%
      \index{Leibnitz, regla de}
    \begin{equation*}
      \frac{\mathrm{d}}{\mathrm{d} s} \, I(s)
	= \frac{\mathrm{d}}{\mathrm{d} s}
	    \int_0^1 f(h(t, s)) \,
	      \frac{\partial h}{\partial t} \, \mathrm{d} t
	= \int_0^1 \frac{\partial}{\partial s} \,
	    \left(
	      f(h(t, s)) \frac{\partial h}{\partial t}
	    \right)
	    \, \mathrm{d} t
    \end{equation*}
    Usando el menú completo
    de propiedades de las derivadas parciales:
    \begin{align*}
      \frac{\mathrm{d}}{\mathrm{d} s} I(s)
	&= \int_0^1
	     \left(
	       f'(h(t, s))
		 \frac{\partial h}{\partial s}
		     \frac{\partial h}{\partial t}
		 + f(h(t, s))
		     \frac{\partial^2 h}{\partial s \partial t}
	     \right)
	     \, \mathrm{d} t \\
	&= \int_0^1
	     \left(
	       f'(h(t, s))
		 \frac{\partial h}{\partial t}
		    \frac{\partial h}{\partial s}
		 + f(h(t, s))
		    \frac{\partial^2 h}{\partial t \partial s}
	     \right)
	     \, \mathrm{d} t \\
	&= \int_0^1 \frac{\partial}{\partial t}
	     \left(
	       f(h(t, s)) \frac{\partial h}{\partial s}
	     \right)
	     \, \mathrm{d} t
    \end{align*}
    Aplicando el teorema fundamental del cálculo integral%
      \index{calculo integral, teorema fundamental del@cálculo integral, teorema fundamental del}
    por separado a las componentes real e imaginaria,
    y recordando la condición \(h(0, s) = h(1, s)\)
    y que las respectivas derivadas coinciden:
    \begin{equation*}
      \frac{\mathrm{d}}{\mathrm{d} s} I(s)
	= f(h(1, s)) \, \frac{\partial h}{\partial s} (1, s)
	   - f(h(0, s)) \, \frac{\partial h}{\partial s} (0, s)
	= 0
    \end{equation*}
    Si la derivada compleja es cero,
    lo son las derivadas parciales
    de las componentes real e imaginaria,
    y así la función es constante.
  \end{proof}
  Una consecuencia inmediata es que si \(D\)
  es una región conexa simple
  en la cual la función \(f\) es holomorfa
  entonces para todo camino suave cerrado \(\gamma \subset D\):
  \begin{equation*}
    \int_\gamma f(z) \, \mathrm{d} z
      = 0
  \end{equation*}
  Esto porque en este caso
  cualquier curva \(\gamma \subset D\) es homotópica con un punto,
  y claramente la integral para un punto es cero.
  Por el teorema~\ref{theo:complex-integral=antiderivative}
  en tales regiones la integral es independiente del camino
  y existe una antiderivada.

\subsection{La fórmula integral de Cauchy}
\label{sec:Cauchy-integral-formula}

  Tenemos el siguiente resultado notable:
  \begin{theorem}[Fórmula integral de Cauchy]
    \index{Cauchy, formula integral de@Cauchy, fórmula integral de|textbfhy}
    \label{theo:Cauchy-integral-formula}
    Sea \(f\) holomorfa en la región \(D\)
    que contiene el camino cerrado simple \(\gamma\),
    con la orientación habitual que el interior está a la izquierda,
    y suponga que \(z_0\) está al interior de \(\gamma\).
    Entonces:
    \begin{equation*}
      f(z_0)
	= \frac{1}{2 \pi \mathrm{i}} \,
	    \int_\gamma	 \frac{f(z)}{z - z_0} \, \mathrm{d} z
    \end{equation*}
  \end{theorem}
  \begin{proof}
    Sea \(\epsilon > 0\) arbitrario.
    Sabemos que \(f\) es continua en \(z_0\),
    por lo que existe \(\delta > 0\) tal que
    \(\lvert f(z) - f(z_0) \rvert < \epsilon\)
    siempre que \(\lvert z - z_0 \rvert < \delta\).
    Sea ahora \(r > 0\) tal que \(r < \delta\)
    y la circunferencia
      \(C_0 = \{ z \colon \lvert z - z_0 \rvert = r\}\)
    está dentro de \(\gamma\).
    Entonces \(f(z) / (z - z_0)\) es holomorfa
    en la región entre \(\gamma\) y \(C_0\),
    por lo que del teorema de Cauchy:
    \begin{equation*}
      \int_\gamma \frac{f(z)}{z - z_0} \, \mathrm{d} z
	= \int_{C_0} \frac{f(z)}{z - z_0} \, \mathrm{d} z
    \end{equation*}
    La integral siguiente es fácil de evaluar
    si parametrizamos \(C_0\)
    como \(z_0 + r \mathrm{e}^{\mathrm{i} t}\):
    \begin{equation*}
      \int_{C_0} \frac{1}{z - z_0} \, \mathrm{d} z
	= \int_0^{2 \pi} \frac{1}{r}
			   \cdot r \mathrm{i}
			     \, \mathrm{e}^{\mathrm{i} t}
			   \, \mathrm{d} t
	= 2 \pi \mathrm{i}
    \end{equation*}
    Con esto:
    \begin{equation*}
      \int_{C_0} \frac{f(z)}{z - z_0} \, \mathrm{d} z
	- 2 \pi \mathrm{i} f(z_0)
	= \int_{C_0} \frac{f(z)}{z - z_0} \, \mathrm{d} z
	    - \int_{C_0} \frac{f(z_0)}{z - z_0} \, \mathrm{d} z
	= \int_{C_0} \frac{f(z) - f(z_0)}{z - z_0} \, \mathrm{d} z
    \end{equation*}
    Sobre \(C_0\) tenemos:
    \begin{equation*}
      \left\lvert \frac{f(z) - f(z_0)}{z - z_0} \right\rvert
	= \frac{\lvert f(z) - f(z_0) \rvert}{\lvert z - z_0 \rvert}
	\le \frac{\epsilon}{r}
    \end{equation*}
    De nuestra cota~\eqref{eq:complex-integral-bound}
    para integrales:
    \begin{equation*}
      \int_{C_0} \frac{f(z) - f(z_0)}{z - z_0} \, \mathrm{d} z
	\le \frac{\epsilon}{r} \cdot 2 \pi r
	= 2 \pi \epsilon
    \end{equation*}
    Pero \(\epsilon\) es un número positivo arbitrario,
    la integral debe ser cero.
  \end{proof}
  Esto es realmente notable:
  Si \(f\) es holomorfa
  al interior del camino cerrado simple \(\gamma\)
  y conocemos los valores de \(f\) sobre \(\gamma\),
  los conocemos en todo su interior.

  Incluso da una manera sencilla de evaluar ciertas integrales.
  Por ejemplo,
  evaluemos la integral:
  \begin{equation*}
    \int_0^\infty \frac{1}{x^2 + 1} \, \mathrm{d} x
  \end{equation*}
  Primero,
  el integrando es par,
  con lo que:
  \begin{equation*}
    \int_0^\infty \frac{1}{x^2 + 1} \, \mathrm{d} x
      = \frac{1}{2} \,
	  \int_{- \infty}^\infty \frac{1}{x^2 + 1} \, \mathrm{d} x
  \end{equation*}
  El integrando tiene problemas en \(\pm \mathrm{i}\),
  es holomorfo
  en \(\mathbb{C} \smallsetminus \{- \mathrm{i}, \mathrm{i} \}\).
  \begin{figure}[ht]
    \centering
    \pgfimage{images/example-integral-contour}
    \caption{Curva para integral ejemplo}
    \label{fig:example-integral-contour}
  \end{figure}
  La curva \(\gamma\)
  de la figura~\ref{fig:example-integral-contour}
  se descompone en un arco \(A\) de radio \(R\)
  y la línea \(L\) de \(-R\) a \(R\) a lo largo del eje~\(X\).
  Si hacemos tender \(R \rightarrow \infty\),
  la integral sobre \(L\) es el resultado que nos interesa.
  Debemos evaluar la integral sobre el arco.
  Como nos interesa \(R \rightarrow \infty\),
  perfectamente podemos concentrarnos en \(R > 1\).
  Tenemos:
  \begin{equation*}
    \left\lvert
      \int_A \frac{1}{z^2 + 1} \, \mathrm{d} z
    \right\rvert
      \le \int_A \frac{1}{\lvert z^2 + 1 \rvert} \, \mathrm{d} z
      \le \int_A \frac{1}{\lvert z \rvert^2 - 1} \, \mathrm{d} z
      =	  \frac{1}{R^2 - 1} \cdot \pi R
  \end{equation*}
  Esto tiende a cero cuando \(R \rightarrow \infty\),
  como esperábamos.
  Acá usamos:
  \begin{equation*}
    \lvert z \rvert^2
      =	  \lvert (z^2 + 1) - 1 \rvert
      \le \lvert z^2 + 1 \rvert + 1
  \end{equation*}
  Por otro lado,
  por la fórmula de Cauchy%
    \index{Cauchy, formula integral de@Cauchy, fórmula integral de}
  podemos escribir para \(z_0 = \mathrm{i}\):
  \begin{equation*}
    \int_\gamma \frac{1 / (z + \mathrm{i})}{z - \mathrm{i}}
	\, \mathrm{d} z
      = \left.
	  2 \pi \mathrm{i} \, \frac{1}{z + \mathrm{i}}
	\right\rvert_{z = \mathrm{i}}
      = \pi
  \end{equation*}
  y nuestra integral original es:
  \begin{equation*}
    \int_0^\infty \frac{1}{x^2 + 1} \, \mathrm{d} x
      = \frac{\pi}{2}
  \end{equation*}
  Esta integral es simple de evaluar en forma tradicional,
  pero esta técnica es aplicable en forma mucho más amplia.

  \begin{theorem}
    \label{theo:Cauchy-formula-f-prime}
    Sea \(f\) holomorfa en la región \(D\).
    Entonces \(f'\) es holomorfa en \(D\).
  \end{theorem}
  \begin{proof}
    Sea \(C\) una circunferencia centrada en \(z\) de radio \(r\)
    dentro de \(D\),
    y \(z + h\) un punto dentro de \(C\).
    Es rutina verificar que:
    \begin{equation*}
      \frac{1}{h} \,
	\left(
	  \frac{1}{\zeta - z - h} - \frac{1}{\zeta - z}
	\right)
	= \frac{1}{(\zeta - z)^2}
	    + \frac{h}{(\zeta - z)^2 (\zeta - z - h)}
    \end{equation*}
    Calculamos:
    \begin{align*}
      \frac{f(z + h) - f(z)}{h}
	&= \frac{1}{2 \pi h \mathrm{i}} \,
	     \int_C \frac{f(\zeta)}{\zeta - z - h}
		 \, \mathrm{d} \zeta
	     - \frac{1}{2 \pi h \mathrm{i}}
		 \int_C \frac{f(\zeta)}{\zeta - z}
		    \, \mathrm{d} \zeta \\
	&= \frac{1}{2 \pi \mathrm{i}} \,
	     \int_C \frac{f(\zeta)}{(\zeta - z)^2}
		 \, \mathrm{d} \zeta
	       + \frac{h}{2 \pi \mathrm{i}} \,
		   \int_C \frac{f(\zeta)}
			       {(\zeta - z)^2 (\zeta - z - h)}
		     \, \mathrm{d} \zeta
    \end{align*}
    Si \(\lvert h \rvert < r / 2\),
    por la desigualdad triangular
    (teorema~\ref{theo:desigualdad-triangular})%
      \index{desigualdad triangular}
    para todo \(\zeta \in C\) es:
    \begin{equation*}
      \lvert \zeta - z - h \rvert
	\ge \lvert \zeta - z \rvert - \lvert h \rvert
	> r - \frac{r}{2}
	= \frac{r}{2}
    \end{equation*}
    Por el otro lado,
    \(f\) es continua sobre \(C\),
    por lo que hay \(M\) tal que \(\lvert f(\zeta) \rvert \le M\)
    para \(\zeta \in C\).
    De la estimación~\eqref{eq:complex-integral-bound} tenemos:
    \begin{equation*}
      \left\lvert
	\frac{h}{2 \pi \mathrm{i}} \,
	  \int_C \frac{f(\zeta)}{(\zeta - z)^2 (\zeta - z - h)}
		   \, \mathrm{d} \zeta
      \right\rvert
	\le \frac{\lvert h \rvert}{2 \pi}
	       \, \frac{2 M}{r^3} \, 2 \pi r
	= \frac{2 M \lvert h \rvert}{r^2}
    \end{equation*}
    Cuando \(h \rightarrow 0\) esto tiende a cero.
    Esto incluso da una fórmula explícita para \(f'(z)\):
    \begin{equation*}
      f'(z)
	= \frac{1}{2 \pi \mathrm{i}}
	    \int_C \frac{f(\zeta)}{(\zeta - z)^2}
	       \, \mathrm{d} \zeta
    \end{equation*}
  \end{proof}
  Aplicando el mismo argumento,
  obtenemos \(f''(z)\),
  con lo que \(f'\) es holomorfa.%
    \index{C (numeros complejos)@\(\mathbb{C}\) (números complejos)!funcion holomorfa@función holomorfa}
  Continuando tenemos derivadas de todos los órdenes.
  Hemos demostrado:
  \begin{theorem}
    \label{theo:holomorphic=>f-(n)}
    Sea \(f\) holomorfa en la región \(D\).
    Entonces para todo \(n \in \mathbb{N}\)
    la función \(f^{(n)}\) es holomorfa en \(D\).
  \end{theorem}
  Esto es notable,
  en los reales la existencia de la primera derivada
  nada dice de las derivadas superiores,
  acá la existencia de la primera derivada
  asegura que hay derivadas de todos los órdenes.
  Aún más:
  \begin{theorem}[Fórmula integral de Cauchy generalizada]
    \label{theo:Cauchy-formula-f-(n)}
    Sea \(f\) holomorfa en la región \(D\),
    y \(\gamma\) un camino cerrado simple al interior de \(D\).
    Entonces:
    \begin{equation}
      \label{eq:Cauchy-formula-f-(n)}
      f^{(n)}(z)
	= \frac{n!}{2 \pi \mathrm{i}} \,
	    \int_\gamma \frac{f(\zeta)}{(\zeta - z)^{n + 1}}
	      \, \mathrm{d} \zeta
    \end{equation}
  \end{theorem}
  \begin{proof}
    Por el teorema~\ref{theo:holomorphic=>f-(n)}
    sabemos que \(f^{(n)}\)
    es holomorfa en \(D\).
    La fórmula integral de Cauchy permite escribir:%
      \index{Cauchy, formula integral de@Cauchy, fórmula integral de}
    \begin{equation*}
      f^{(n)}(z)
	= \frac{1}{2 \pi \mathrm{i}} \,
	    \int_\gamma \frac{f^{(n)}(\zeta)}{\zeta - z}
	      \, \mathrm{d} \zeta
    \end{equation*}
    Integrando por partes \(n\) veces entrega lo prometido.
  \end{proof}
  Hay más consecuencias de interés.
  \begin{theorem}[Liouville]
    \index{Liouville, teorema de}
    \label{theo:Liouville}
    Si una función entera es acotada en valor absoluto,
    es constante.
  \end{theorem}
  \begin{proof}
    Por hipótesis hay una constante \(M\)
    tal que \(\lvert f(z) \rvert  \le M\)
    para todo \(z \in \mathbb{C}\).
    Demostramos por contradicción
    que \(f'(z) = 0\) en todo \(\mathbb{C}\),
    con lo que por el teorema~\ref{theo:complex-zero-derivative}
    \(f\) es constante.

    Supongamos que para algún \(z\) es \(f'(z) \ne 0\).
    Elija \(R\) de manera que \(M / R < \lvert f'(z) \rvert\).
    Sea \(C\) la circunferencia de radio \(R\) alrededor de \(z\).
    Entonces:
    \begin{equation*}
      \frac{M}{R}
	<   \lvert f'(z) \rvert
	=   \left\lvert
	      \frac{1}{2 \pi \mathrm{i}} \,
		\int_C \frac{f(\zeta)}{(\zeta - w)^2}
		  \, \mathrm{d} \zeta
	    \right\rvert
	\le \frac{1}{2 \pi} \, \frac{M}{R^2} \, 2 \pi R
	= \frac{M}{R}
    \end{equation*}
    Esta contradicción muestra que tal \(z\) no existe.
  \end{proof}
  Podemos aprovechar esto inmediatamente,
  básicamente como lo hizo Gauß en su disertación
  (aunque Liouville es bastante posterior).
  Pese a su nombre,
  poco tiene que ver con el álgebra actual
  y no es particularmente fundamental.
  \begin{theorem}[Teorema fundamental del álgebra]
    \index{algebra, teorema fundamental del@álgebra, teorema fundamental del}
    \index{polinomio!cero}
    \label{theo:fundamental-algebra}
    Todo polinomio no constante con coeficientes complejos
    tiene un cero complejo.
  \end{theorem}
  \begin{proof}
    Por contradicción.
    Sea \(p(z)\) un polinomio no constante sin ceros complejos.
    Entonces \(1 / p(z)\) es entera y acotada
    (cuando \(z \rightarrow \infty\)
     también \(\lvert p(z) \rvert \rightarrow \infty\),
     y la función \(1 / p(z)\) es acotada en todo \(\mathbb{C}\)),
    y por el teorema de Liouville es constante.
    Esto contradice el que \(p\) no es constante.
  \end{proof}
  La manera tradicional
  de expresar el teorema~\ref{theo:fundamental-algebra}
  es diciendo que todo polinomio no constante
  de coeficientes reales
  se puede factorizar en factores lineales
  o cuadráticos sin ceros reales,
  o que si tiene grado \(n\)
  tiene \(n\) ceros reales o complejos conjugados
  (contando multiplicidades).

\section{Secuencias y series}
\label{sec:complex-sequences-series}

  Las definiciones de secuencias y series complejas
  son esencialmente las mismas que para los reales.
  Anotamos \(\langle a_n \rangle_{n \ge 0}\)
  para la secuencia de los \(a_n\)
  (formalmente,
   es una función
     \(a \colon \mathbb{N}_0 \rightarrow \mathbb{C}\)).%
     \index{C (numeros complejos)@\(\mathbb{C}\) (números complejos)!secuencia}
  El número \(L\) se llama el \emph{límite} de la secuencia
  si para cualquier \(\epsilon > 0\) que se elija
  hay un entero \(n_\epsilon\),
  dependiente de \(\epsilon\),
  tal que siempre que \(n \ge n_\epsilon\)
  es \(\lvert L - a_n \rvert < \epsilon\).
  Esto lo anotamos \(\lim a_n = L\).%
     \index{C (numeros complejos)@\(\mathbb{C}\) (números complejos)!secuencia!limite@límite}
  Es fácil ver que si \(a_n = u_n + \mathrm{i} v_n\)
  y la secuencia \(\langle a_n \rangle_{n \ge 0}\) converge a \(L\),
  tenemos \(\lim u_n = \Re L\) y \(\lim v_n = \Im L\).
  Al revés,
  si las secuencias reales \(\langle u_n \rangle_{n \ge 0}\)
  y \(\langle v_n \rangle_{n \ge 0}\) convergen,
  converge la secuencia compleja
    \(\langle u_n + \mathrm{i} v_n \rangle_{n \ge 0}\).
  Se desprenden todas las familiares propiedades de los límites.
  Una condición necesaria y suficiente
  para la convergencia
  de la secuencia \(\langle a_n \rangle_{n \ge 0}\)
  es el \emph{criterio de Cauchy}:%
    \index{Cauchy, criterio de}
  Dado \(\epsilon > 0\)
  hay un entero \(n_\epsilon\)
  tal que \(\lvert a_m - a_n \rvert < \epsilon\)
  siempre que \(m, n \ge n_\epsilon\).

  Es obvio considerar secuencias de funciones en una región \(D\).
  Para cada \(z \in D\) tenemos una secuencia ordinaria
  \(\langle f_n(z) \rangle_{n \ge 0}\).
  Si estas secuencias convergen,
  la secuencia \emph{converge punto a punto}%
     \index{C (numeros complejos)@\(\mathbb{C}\) (números complejos)!secuencia!convergencia}
  a la función \(f(z) = \lim f_n(z)\).
  Se dice que la secuencia de funciones
  converge \emph{uniformemente}%
    \index{convergencia uniforme}
  sobre el conjunto \(S\) si dado un \(\epsilon > 0\)
  hay un entero \(n_\epsilon\)
  tal que \(\lvert f(z) - f_n(z) \rvert < \epsilon\)
  para todo \(n \ge n_\epsilon\) y todo \(z \in S\).
  El punto de la convergencia uniforme
  es que el mismo \(n_\epsilon\) sirve para todos los \(z \in S\).

  \begin{figure}[ht]
    \centering
    \pgfimage{images/sequence-continuous-limit-discontinuous}
    \caption{Secuencia de funciones continuas con límite discontinuo}
    \label{fig:sequence-continuous-limit-discontinuous}
  \end{figure}
  Note que una secuencia de funciones continuas
  puede converger a una función discontinua.
  Considere por ejemplo la secuencia de funciones
  definidas para \(n \ge 1\) por:
  \begin{equation}
    \label{eq:sequence-continuous-functions}
    f_n(x)
      = \begin{cases}
	  0	    & x \le - 1 / n \\
	  1 + n x   & -1 / n < x \le 0 \\
	  1 - n x   & 0 < x \le 1 / n \\
	  0	    & 1 / n < x
	\end{cases}
  \end{equation}
  La figura~\ref{fig:sequence-continuous-limit-discontinuous}
  grafica algunas de las funciones~%
    \eqref{eq:sequence-continuous-functions}.
  Es claro que:
  \begin{equation}
    \label{eq:continuous-functions-discontinuous-limit}
    \lim_{n \rightarrow \infty} f_n(x)
      = \begin{cases}
	  0  & x \ne 0 \\
	  1  & x = 1
	\end{cases}
  \end{equation}
  Este comportamiento es imposible si la convergencia es uniforme.
  Porque suponga que \(\langle f_n(z) \rangle_{n \ge 0}\)
  converge uniformemente a \(f\) en la región \(D\),
  sea \(z_0 \in D\) y \(\epsilon > 0\).
  Demostraremos que hay \(\delta\)
  tal que \(\lvert f(z_0) - f(z) \rvert < \epsilon\)
  siempre que \(\lvert z_0 - z \rvert < \delta\).
  Elija \(n_\epsilon\)
  tal que \(\lvert f_{n_\epsilon}(z) - f(z) \rvert < \epsilon / 3\).
  Por convergencia uniforme
  también es
     \(\lvert f_{n_\epsilon}(z_0) - f(z_0) \rvert < \epsilon / 3\).
  Ahora elija \(\delta\)
  de forma que
    \(\lvert f_{n_\epsilon}(z_0) - f_{n_\epsilon}(z) \rvert
	< \epsilon / 3\)
  siempre que \(\lvert z_0 - z \rvert < \delta\).
  Esto es posible ya que \(f_{n_\epsilon}\) es continua.
  Si \(\lvert z_0 - z \rvert < \delta\) resulta
  para todo \(n > n_\epsilon\):
  \begin{align*}
    \lvert f(z_0) - f(z) \rvert
      &=   \lvert f(z_0) - f_n(z_0)
		    + f_n(z_0) - f_n(z)
		    + f_n(z) - f(z) \rvert \\
      &\le \lvert f(z_0) - f_n(z_0) \rvert
	     + \lvert f_n(z_0) - f_n(z) \rvert
	     + \lvert f_n(z) - f(z) \rvert \\
      &<   \frac{\epsilon}{3}
	     + \frac{\epsilon}{3}
	     + \frac{\epsilon}{3} \\
      &=   \epsilon
  \end{align*}

  En los reales hay secuencias de funciones diferenciables
  que convergen uniformemente a funciones que no son diferenciables.
  La función símbolo que no es diferenciable en \(0\)
  es \(\lvert x \rvert\),
  interesa construir una secuencia de funciones
  que se parecen a las ramas del valor absoluto,
  ``suavizando'' la esquina
  por ejemplo con una parábola \(y = a x^2\)
  entre \(\pm 1 / n\).
  Las ramas serán rectas de pendiente \(\pm 1\),
  digamos \(y = \pm x + b\);
  queremos que los valores
  y las derivadas coincidan en \(\pm 1 / n\):
  \begin{equation*}
    f_n	 \left( \pm \frac{1}{n} \right)
      = \frac{1}{n} + b
    \hspace{4em}
    f_n' \left( \pm \frac{1}{n} \right)
      = \pm 2 a \frac{1}{n}
      = \pm 1
  \end{equation*}
  De acá:
  \begin{equation*}
    a = \frac{n}{2}
    \hspace{3em}
    b = - \frac{1}{2 n}
  \end{equation*}
  Nuestra función es:
  \begin{equation*}
    f_n(x)
      = \begin{cases}
	  - x - 1 / 2 n	      & -1 \le x < 1 / n \\
	  n x^2 / 2	      & - 1 / n < x \le 1 / n \\
	  x - 1 / 2 n	      & 1 / n < x \le 1
	\end{cases}
  \end{equation*}
  Por la forma que la construimos,
  \(f_n\) es diferenciable en \([-1, 1]\).
  Difiere de \(\lvert x \rvert\) a lo más en \(1 / 2 n\),
  con lo que la convergencia es uniforme.
  Pero \(\lim f_n(x) = \lvert x \rvert\),
  que no es diferenciable en \(x = 0\).

  Pero también:
  \begin{theorem}
    \label{theo:int_f_n->int_f}
    Sea \(\gamma\) una curva suave,
    sobre la cual las funciones \(f_n\) son continuas
    y convergen uniformemente a \(f\).
    Entonces:
    \begin{equation}
      \label{eq:int_f_n->int_f}
      \lim_{n \rightarrow \infty} \int_\gamma f_n(z) \, \mathrm{d} z
	= \int_\gamma f(z) \, \mathrm{d} z
    \end{equation}
  \end{theorem}
  Este resultado tiene múltiples consecuencias,
  que veremos más adelante.
  La demostración es rutina:
  \begin{proof}
    Podemos acotar:
    \begin{equation*}
      \left\lvert
	\int_\gamma f_n(z) \, \mathrm{d} z
	  - \int_\gamma f(z) \, \mathrm{d} z
      \right\rvert
	= \left\lvert
	    \int_\gamma (f_n(z) - f(z)) \, \mathrm{d} z
	  \right\rvert
	\le \max_{z \in \gamma} \lvert f_n(z) - f(z) \rvert
	      \cdot l_\gamma
    \end{equation*}
    Por convergencia uniforme
    podemos hacer el primer factor de la cota
    tan pequeño como deseemos.
  \end{proof}
  Incluso más:
  \begin{theorem}
    \label{theo:holomorphic_f_n->holomorphic_f}
    Sea una secuencia de funciones holomorfas
    \(\langle f_n(z) \rangle_{n \ge 0}\) que en \(D\)
    convergen uniformemente a \(f(z)\).
    Entonces \(f\) es holomorfa en \(D\).
  \end{theorem}
  Nótese que el resultado no se cumple para reales.
  \begin{proof}
    Sea \(\gamma \subset D\) una curva cerrada simple.
    Del teorema de Cauchy sabemos:%
      \index{Cauchy, teorema integral de}
    \begin{equation*}
      \int_\gamma f_n(z) \, \mathrm{d} z
	= 0
    \end{equation*}
    Por convergencia uniforme:
    \begin{equation*}
      \int_\gamma f(z) \, \mathrm{d} z
	= 0
    \end{equation*}
    El teorema de Morera,
    teorema~\ref{theo:Morera},
    nos dice que
    la función \(f\) es holomorfa en \(D\).
  \end{proof}

\subsection{Series}
\label{sec:complex-series}

  Una serie es simplemente
  la secuencia \(\langle s_n \rangle_{n \ge 0}\)
  resultante de sumar los elementos
  de una secuencia \(\langle a_n \rangle_{n \ge 0}\),
  vale decir,
  \(s_n = a_0 + a_1 + \dotsb + a_n\).
  Si la serie converge,
  debe ser \(\lim a_n = 0\).
  Para el límite de la serie
  anotamos según nuestra convención sobre sumas:
  \begin{equation*}
    \sum_{n \ge 0} a_n
  \end{equation*}
  o el más familiar:
  \begin{equation*}
    \sum_{n = 0}^\infty a_n
  \end{equation*}

  Igual que en el caso de series en los reales,
  es útil distinguir series
  que \emph{convergen en valor absoluto},%
    \index{serie!convergencia absoluta}
  vale decir la secuencia:
  \begin{equation*}
    \sum_{0 \le k \le n} \lvert a_k \rvert
  \end{equation*}
  converge.
  Es simple demostrar que si la serie converge en valor absoluto,
  converge la serie original;
  pero la convergencia de la serie original
  no asegura convergencia en valor absoluto.
  Por ejemplo,
  tenemos la serie harmónica alternante
  (el valor lo justificaremos más adelante):
  \begin{equation}
    \label{eq:alternating-harmonic-series}
    \sum_{k \ge 1} \frac{(-1)^{k + 1}}{k}
      = \ln 2
  \end{equation}
  pero la contraparte de valores absolutos es la serie harmónica,
  que no converge.
  Incluso más:
  \begin{theorem}[Reordenamiento de Riemann]
    \index{Riemann, teorema de reordenamiento de}
    \label{theo:Riemann-rearrangement}
    Sea una serie real que converge pero no absolutamente.
    Entonces sus términos
    pueden reordenarse para dar cualquier suma,
    e incluso diverger a \(\pm \infty\)
    o no tener límite.
  \end{theorem}
  \begin{proof}
    Si la serie converge,
    pero no absolutamente,
    tiene infinitos términos positivos cuya suma diverge
    y de la misma forma tiene infinitos términos negativos
    cuya suma diverge.
    Fijemos un valor \(L\) cualquiera;
    consideraremos el caso en que \(L \ge 0\),
    el caso \(L < 0\) es similar.
    Ordenamos los términos como sigue:
    \begin{itemize}
    \item
      Elegimos términos positivos
      hasta que la suma sobrepase a \(L\).
      Como la suma de los términos positivos diverge,
      esto puede hacerse.
    \item
      Elegimos luego términos negativos
      hasta que la suma sea menor a \(L\).
      Nuevamente,
      como los términos negativos divergen esto puede hacerse.
    \end{itemize}
    Repitiendo este proceso obtenemos un ordenamiento
    de los términos de la serie que converge a \(L\).
    La diferencia entre la suma y el valor elegido va disminuyendo,
    como la serie original converge
    sabemos que los términos de la serie reordenada
    disminuyen en valor absoluto.

    Siendo suficientemente tacaños con los términos negativos
    (respectivamente positivos)
    logramos que diverja;
    tomando dos valores podemos hacer oscilar los valores de la suma
    alrededor de ellos.
  \end{proof}
  De tales series se dice que \emph{convergen condicionalmente}.%
    \index{serie!convergencia condicional}
  Un ejemplo de este fenómeno
  es escribir la serie harmónica alternante~%
    \eqref{eq:alternating-harmonic-series}
  como:
  \begin{equation}
    \label{eq:alternating-harmonic-series-reordered-value}
    \sum_{k \ge 1}
      \left(
	\frac{1}{2 k - 1} - \frac{1}{2 (2 k - 1)} - \frac{1}{4 k}
      \right)
      = \sum_{k \ge 1}
	  \left(
	    \frac{1}{2 (2 k - 1)} - \frac{1}{2 \cdot 2 k}
	  \right)
      = \frac{1}{2}
	  \sum_{k \ge 1} \frac{(-1)^{k + 1}}{k}
      = \frac{1}{2} \ln 2
  \end{equation}
  Este es un reordenamiento correcto,
  aparecen los recíprocos de todos los impares con signo positivo
  y los recíprocos de todos los pares con signo negativo.
  La mitad
  del valor original~\eqref{eq:alternating-harmonic-series}.

  En forma análoga podemos considerar series de funciones:
  \begin{equation*}
    \sum_{k \ge 0} f_k(z)
  \end{equation*}
  Tales series pueden converger para ciertos valores de \(z\)
  y no para otros.
  Un criterio útil de convergencia es el siguiente:
  \begin{theorem}[Prueba \(M\) de Weierstraß]
    \index{Weierstrass, prueba \(M\) de@Weierstraß, prueba \(M\) de}
    \label{theo:Weierstrass-M}
    Sea \(\langle M_k \rangle_{k \ge 0}\)
    una secuencia de números reales,
    que hay \(K\) tal que \(M_k \ge 0\) para todo \(k > K\),
    y suponga que la secuencia
    \begin{equation*}
      \left\langle \sum_{0 \le k \le n} M_k \right\rangle_{n \ge 0}
    \end{equation*}
    converge.
    Si para todo \(z \in D\) es
    \(\lvert f_k(z) \rvert \le M_k\) para \(k \ge K\),
    la serie
    \begin{equation*}
      \sum_{k \ge 0} f_k(z)
    \end{equation*}
    converge uniformemente en valor absoluto en \(D\).
  \end{theorem}
  \begin{proof}
    Sea \(\epsilon > 0\) cualquiera,
    y elegimos \(N > K\) tal que:
    \begin{equation*}
      \sum_{m \le k \le n} M_k
	< \epsilon
    \end{equation*}
    para todo \(m, n > N\)
    (esto resulta del criterio de Cauchy).
    Por la desigualdad triangular,
    teorema~\ref{theo:desigualdad-triangular},
    es:
    \begin{equation*}
      \left\lvert
	\sum_{m \le k \le n} f_k(z)
      \right\rvert
	\le \sum_{m \le k \le n} \lvert f_k(z) \rvert
	\le \sum_{m \le k \le n} M_k
	< \epsilon
    \end{equation*}
    La serie converge.
    Para convergencia uniforme,
    observe que para todo \(z \in D\) y \(n > m > N\):
    \begin{equation*}
      \left\lvert
	\sum_{m \le k \le n} f_k(z)
      \right\rvert
	= \left\lvert
	    \sum_{0 \le k \le n} f_k(z)
	      - \sum_{0 \le k \le m - 1} f_k(z)
	  \right\rvert
	< \epsilon
    \end{equation*}
    En consecuencia:
    \begin{equation*}
      \lim_{n \rightarrow \infty}
	\left\lvert
	  \sum_{m \le k \le n} f_k(z)
	\right\rvert
	= \left\lvert
	    \sum_{k \ge 0} f_k(z)
	      - \sum_{0 \le k \le m - 1} f_k(z)
	  \right\rvert
	\le \epsilon
    \end{equation*}
    y la convergencia es uniforme y en valor absoluto.
  \end{proof}

  El caso más interesante es el de series de potencias:
  \begin{equation*}
    s_n(z)
      = \sum_{0 \le k \le n} c_k (z - z_0)^k
  \end{equation*}
  Una serie de potencias
  podrá tener un límite para ciertos valores de \(z\)
  y no para otros.
  Claramente siempre tiene límite si \(z = z_0\).
  \begin{theorem}[Cauchy-Hadamard]
    \index{Cauchy-Hadamard, teorema de}
    \index{serie de potencias!radio de convergencia}
    \label{theo:convergence-root}
    Sea la serie:
    \begin{equation*}
      \sum_{0 \le k \le n} c_k (z - z_0)^k
    \end{equation*}
    Sea:
    \begin{equation}
      \label{eq:convergence-radius-root}
      \lambda
	= \limsup_{k \rightarrow \infty} \sqrt[k]{\lvert c_k \rvert}
    \end{equation}
    Sea \(R = \lambda^{-1}\),
    donde diremos que \(R = \infty\) si \(\lambda = 0\)
    y que \(R = 0\) si \(\lambda = \infty\).
    Entonces la serie converge uniformemente en valor absoluto
    para todo \(\lvert z - z_0 \rvert < r < R\)
    y diverge para todo \(\lvert z - z_0 \rvert > R\).
  \end{theorem}
  \begin{proof}
    Primero demostramos que la serie no converge
    para \(\lvert z - z_0 \rvert > R\).
    Sea \(L\) tal que:
    \begin{equation*}
      \frac{1}{\lvert z - z_0 \rvert}
	< L
	< \frac{1}{R}
	= \lambda
    \end{equation*}
    Hay un número infinito de \(c_k\)
    tales que \(\sqrt[k]{\lvert c_k \rvert} > L\),
    ya que de lo contrario el límite superior sería menor a \(L\).
    Para cada uno de ellos tenemos:
    \begin{equation*}
      \lvert c_k (z - z_0)^k \rvert
	= \left(
	    \sqrt[k]{\lvert c_k \rvert} \cdot \lvert z - z_0 \rvert
	  \right)^k
	> \left( L \lvert z - z_0 \rvert \right)^k
	> 1
    \end{equation*}
    y la serie no puede converger.

    Enseguida demostramos que converge uniformemente
    para todo \(\lvert z - z_0 \rvert < r < R\).
    Sea \(L\) tal que:
    \begin{equation*}
      \lambda
	= \frac{1}{R}
	< L
	< \frac{1}{r}
    \end{equation*}
    Para \(k\) suficientemente grande
    es \(\sqrt[k]{\lvert c_k \rvert} < L\),
    con lo que si \(\lvert z - z_0 \rvert \le r\):
    \begin{equation*}
      \lvert c_k (z - z_0)^k \rvert
	= \left(
	    \sqrt[k]{\lvert c_k \rvert} \cdot \lvert z - z_0 \rvert
	  \right)^k
	< (L \lvert z - z_0 \rvert)^k
	< (L r)^k
    \end{equation*}
    La serie geométrica de los \((L r)^k\) converge,
    y la prueba de \(M\) da convergencia uniforme en valor absoluto.
  \end{proof}
  Nótese que hemos demostrado que toda serie de potencias
  converge uniformemente en valor absoluto
  en el disco abierto \(D_R(z_0)\),
  su \emph{región de convergencia};
  a \(R\) se le llama el \emph{radio de convergencia} de la serie.

  Ya que estamos en eso,
  explicitemos lo que el teorema~\ref{theo:int_f_n->int_f}
  dice para series de potencias.
  La función \(g\) que introduciremos nos vendrá bien más adelante.
  \begin{corollary}
    \label{cor:power-series-integrate-termwise}
    Suponga una serie de potencias
      \(\sum_{k \ge 0} c_k (z - z_0)^k\)
    con radio de convergencia \(R\),
    y una curva suave \(\gamma \subset D_R(z_0)\),
    y sea \(g(z)\) continua
    sobre \(\gamma\).
    Entonces:
    \begin{equation*}
      \int_\gamma g(z) \sum_{k \ge 0} c_k (z - z_0)^k
	  \, \mathrm{d} z
	= \sum_{k \ge 0} c_k \int_\gamma g(z) (z - z_0)^k
	    \, \mathrm{d} z
    \end{equation*}
    En particular,
    si \(\gamma\) es cerrada:
    \begin{equation*}
      \int_\gamma \sum_{k \ge 0} g(z) c_k (z - z_0)^k
	  \, \mathrm{d} z
	= 0
    \end{equation*}
  \end{corollary}
  \begin{proof}
    Sea \(\epsilon > 0\),
    y sea \(M\) el máximo de \(g\) sobre \(\gamma\),
    y \(l_\gamma\) el largo de la curva.
    Entonces hay un entero \(N\) tal que para \(n > N\):
    \begin{equation*}
      \left\lvert
	\sum_{k \ge n} c_k (z - z_0)^k
      \right\rvert
	< \frac{\epsilon}{M l_\gamma}
    \end{equation*}
    de donde:
    \begin{equation*}
      \left\lvert
	\int_\gamma g(z) \sum_{k \ge n} c_k (z - z_0)^k
	  \, \mathrm{d} z
      \right\rvert
	< M l_\gamma \, \frac{\epsilon}{M l_\gamma}
	= \epsilon
    \end{equation*}
    Con esto:
    \begin{equation*}
      \left\lvert
	\int_\gamma g(z) \sum_{k \ge n} c_k (z - z_0)^k
	  \, \mathrm{d} z
	  - \sum_{0 \le k \le n - 1}
	      c_k \int_\gamma g(z) (z - z_0)^k \, \mathrm{d} z
      \right\rvert
	= \left\lvert
	    \int_\gamma g(z) \sum_{k \ge n} c_k (z - z_0)^k
	      \, \mathrm{d} z
	  \right\rvert
	< \epsilon
    \end{equation*}
    Esto es lo que prometimos.
  \end{proof}
  Si elegimos \(g(z) = 1\),
  tenemos que
  podemos integrar término a término
  dentro del radio de convergencia,
  y la serie define una función holomorfa dentro de \(D_R(z_0)\).
  Si analizamos los términos de la serie,
  tenemos el siguiente:
  \begin{corollary}
    \label{cor:ratio-test-1}
    Sean \(c_k\) los coeficientes de una serie de potencias
    con radio de convergencia \(R\).
    Entonces para \(r < R\):
    \begin{equation*}
      \lim_{k \rightarrow \infty} \lvert c_k \rvert r^k
	= 0
    \end{equation*}
    mientras para \(r > R\):
    \begin{equation*}
      \lim_{k \rightarrow \infty} \lvert c_k \rvert r^k
	= \infty
    \end{equation*}
  \end{corollary}
  Nótese que esto nada dice sobre la convergencia o divergencia
  en la frontera de la región de convergencia.
  La manoseada serie:
  \begin{equation}
    \label{eq:logarithmic-series}
    \sum_{k \ge 1} \frac{z^k}{k}
  \end{equation}
  tiene radio de convergencia \(1\).
  Si partimos con:
  \begin{equation}
    \label{eq:geometric-series}
    \frac{1}{1 - z}
      = \sum_{k \ge 0} z^k
  \end{equation}
  cuyo radio de convergencia es \(1\),
  obtenemos que su integral
  (vale decir,
   \eqref{eq:logarithmic-series})
  tiene el mismo radio de convergencia.
  Dentro del radio de convergencia converge a \(\Log (1 - z)\);
  como converge para \(z = -1\),
  por continuidad converge a \(\ln 2\) para \(z = -1\).
  Diverge para \(z = 1\),
  donde resulta la serie harmónica.

  Una forma más sencilla
  de usar del corolario~\ref{cor:ratio-test-1}
  es la siguiente:
  \begin{corollary}
    \label{cor:ratio-test}
    Sean \(c_k\) los coeficientes de una serie de potencias,
    con \(c_k \ne 0\) para \(k\) suficientemente grande.
    Entonces su radio de convergencia es:
    \begin{equation}
      \label{eq:convergence-radius-ratio}
      R
	= \lim_{k \rightarrow \infty}
	    \frac{\lvert c_{k + 1} \rvert}{\lvert c_k \rvert}
    \end{equation}
  \end{corollary}

  Vimos
  (teorema~\ref{theo:holomorphic_f_n->holomorphic_f})
  que si una secuencia de funciones holomorfas
  converge uniformemente
  en una región,
  su límite es holomorfo.
  Aplicando esto a series de potencias:
  \begin{corollary}
    \label{cor:holomorphic=>analytic}
    Suponga \(f(z) = \sum_{k \ge 0} c_k (z - z_0)^k\)
    tiene radio de convergencia positivo \(R\).
    Entonces \(f\) es holomorfa en \(D_R(z_0)\).
  \end{corollary}
  Series de potencias con radio de convergencia infinito
  representan funciones enteras.
  Podemos derivar término a término:
  \begin{theorem}
    \label{theo:complex-series-differentiate-termwise}
    Suponga \(f(z) = \sum_{k \ge 0} c_k (z - z_0)^k\)
    tiene radio de convergencia positivo \(R\).
    En \(D_R(z_0)\) tenemos:
    \begin{equation*}
      f'(z)
	= \sum_{k \ge 0} k c_k (z - z_0)^{k - 1}
    \end{equation*}
  \end{theorem}
  \begin{proof}
    Sea \(z\) un punto dentro de la región de convergencia,
    y sea \(C\) una circunferencia orientada positivamente
    centrada en \(z\) al interior de la región de convergencia.
    Defina:
    \begin{equation*}
      g(\zeta)
	= \frac{1}{2 \pi \mathrm{i} (\zeta - z)^2}
    \end{equation*}
    Aplicamos el corolario~\ref{cor:power-series-integrate-termwise}
    para concluir:
    \begin{align*}
      \int_C g(\zeta) f(\zeta) \, \mathrm{d} \zeta
	&= \sum_{k \ge 0}
	     c_k \int_C g(\zeta) (\zeta - z_0)^k
	       \, \mathrm{d} \zeta \\
      \frac{1}{2 \pi \mathrm{i}} \,
	\int_C \frac{f(\zeta)}{(\zeta - z)^2} \, \mathrm{d} \zeta
	&= \sum_{k \ge 0} c_k \frac{1}{2 \pi \mathrm{i}} \,
	     \int_C \frac{(\zeta - z_0)^k}{(\zeta - z)^2}
	       \, \mathrm{d} \zeta \\
    \intertext{La fórmula integral de Cauchy,
	       teorema~\ref{theo:Cauchy-formula-f-prime},
	       da:}
      f'(z)
	&= \sum_{k \ge 0} k c_k (z - z_0)^k
    \end{align*}
    La expansión es válida
    dentro de la región de convergencia original.
  \end{proof}

\section{Series de Taylor y Laurent}
\label{sec:Taylor-Laurent-series}

  Veamos ahora cómo construir series
  para una función holomorfa dada.

\subsection{Serie de Taylor}
\label{sec:Taylor-series}
\index{Taylor, serie de}

  \begin{theorem}[Serie de Taylor]
    \label{theo:Taylor}
    Suponga \(f\) holomorfa en el disco abierto \(D_R(z_0)\).
    Entonces:
    \begin{equation}
      \label{eq:Taylor-series}
      f(z)
	= \sum_{k \ge 0}
	    \frac{f^{(k)}(z_0)}{k!} (z - z_0)^k
    \end{equation}
    Esta serie converge en \(D_R(z_0)\).
  \end{theorem}
  \begin{proof}
    Sea \(z \in D_R(z_0)\),
    y sea \(C\)
    una circunferencia de radio \(r\) alrededor de \(z\)
    dentro del disco abierto.
    Entonces \(0 < r < R\).
    Para \(\zeta \in C\) podemos escribir:
    \begin{equation*}
      \frac{1}{\zeta - z}
	= \frac{1}{(\zeta - z_0) - (z - z_0)}
	= \frac{1}
	       {\zeta - z_0}
		 \cdot \frac{1}{1 - \frac{z - z_0}{\zeta - z_0}}
	= \sum_{k \ge 0} \frac{(z - z_0)^k}{(\zeta - z_0)^{k + 1}}
    \end{equation*}
    Esto es válido ya que \((z - z_0) / (\zeta - z_0) < 1\),
    y la convergencia es uniforme.
    Podemos integrar:
    \begin{align*}
      \int_C \frac{f(\zeta)}{\zeta - z_0} \, \mathrm{d} \zeta
	&= \sum_{k \ge 0}
	     \left(
	       \int_C \frac{f(\zeta)}{(\zeta - z_0)^{k + 1}}
		 \, \mathrm{d} \zeta
	     \right) \, (z - z_0)^k \\
      f(z)
	&= \frac{1}{2 \pi \mathrm{i}} \,
	     \int_C \frac{f(\zeta)}{\zeta - z} \, \mathrm{d} \zeta
	 = \sum_{k \ge 0}
	     \left(
	       \frac{1}{2 \pi \mathrm{i}} \,
		 \int_C \frac{f(\zeta)}{(\zeta - z_0)^{k + 1}}
		   \, \mathrm{d} \zeta
	     \right) \, (z - z_0)^k \\
	&= \sum_{k \ge 0} \frac{f^{(k)}(z_0)}{k!} \, (z - z_0)^k
      \qedhere
    \end{align*}
  \end{proof}
  A una función que puede representarse localmente
  (en un disco abierto centrado en \(z_0\))
  mediante una serie de potencias convergente
  se le llama \emph{analítica}
  (en \(z_0\)).
  Vemos que es muy similar al concepto de holomorfismo
  en los complejos,
  lo que explica que comúnmente se usen intercambiablemente.

  Vale la pena tomar un poco de distancia
  y recapitular lo que hemos demostrado.
  \begin{itemize}
  \item
    Si la función \(f\) es derivable
    en una región \(D\) de \(\mathbb{C}\),
    tiene derivadas de todos los órdenes en \(D\).
    Igualmente tiene antiderivadas de todos los órdenes en \(D\).
  \item
    La integral de \(f\) es independiente del camino
    al interior de la región simple \(D\),
    y el valor de la integral
    es la diferencia de los valores de una antiderivada
    en los puntos inicial y final.
  \item
    El valor de la función en un punto interior de \(D\)
    queda determinado por los valores en la frontera de \(D\).
  \item
    Podemos representar la función \(f\) mediante
    su serie de Taylor~\eqref{eq:Taylor-series}
    sobre un disco abierto
      \(\{z \colon \lvert z - z_0 \rvert < R\}\).
  \item
    La serie de Taylor de \(f\)
    converge uniformemente en valor absoluto
    a \(f\) para \(\{z \colon \lvert z - z_0 \rvert \le r < R\}\),
    donde el radio de convergencia \(R\)
    está dado por~\eqref{eq:convergence-radius-root}
    o~\eqref{eq:convergence-radius-ratio}.
  \item
    Una serie de potencias
    puede derivarse e integrarse término a término,
    resultando series con el mismo radio de convergencia.
    Las series resultantes
    igualmente convergen uniformemente en valor absoluto.
  \item
    La serie de potencias para \(f\) es única
    (si tenemos dos series de potencias
     que representan la misma función
     centradas en el mismo punto,
     tienen los mismo coeficientes).
  \end{itemize}

\subsection{Singularidades}
\label{sec:singularities}
\index{C (numeros complejos)@\(\mathbb{C}\) (números complejos)!singularidad|textbfhy}
\index{singularidad}

  Requeriremos un poco de jerga adicional.
  Si la función \(f\) es tal que:
  \begin{equation*}
    f(z_0)
      = 0
  \end{equation*}
  decimos que \(z_0\) es un \emph{cero} de \(f\).%
    \index{cero|textbfhy}
  Si \(m \in \mathbb{N}\) es tal que:
  \begin{equation*}
    \lim_{z \rightarrow z_0} \frac{f(z)}{(z - z_0)^{m - 1}}
      = 0
    \text{\ y\ }
    \lim_{z \rightarrow z_0} \frac{f(z)}{(z - z_0)^m}
      \ne 0
  \end{equation*}
  decimos que es un \emph{cero de multiplicidad} \(m\).
  En el caso \(m = 1\) se habla de \emph{cero simple}.

  Veamos cómo extender la representación de funciones
  para abarcar puntos en los que dejan de ser holomorfas,
  \emph{singularidades} en las que la función no está definida.
  Estamos interesados en \emph{singularidades aisladas},
  vale decir,
  hay un disco abierto perforado
  (en inglés, \emph{\foreignlanguage{english}{punctured disk}})
  \(\{ z \in \mathbb{C} \colon 0 < \lvert z - z_0 \rvert < R \}\)
  sobre el que la función \(f\) es holomorfa.
  Tenemos los siguientes casos,
  que detallaremos en lo que sigue:
  \begin{description}
  \item[Singularidad removible:]
    Si existe:
    \begin{equation*}
      \lim_{z \rightarrow z_0} f(z)
    \end{equation*}
    la singularidad es removible.
    Basta definir \(f(z_0)\) como el límite del caso,
    y asunto resuelto.%
      \index{singularidad!removible}
    Véase el teorema~\ref{theo:Riemann-removable-singularity}.
  \item[Polo:]\index{C (numeros complejos)@\(\mathbb{C}\) (números complejos)!polo}
    Si hay \(m \in \mathbb{N}\)
    tal que la función \(g\) definida por:
    \begin{equation*}
      g(z)
	= (z - z_0)^m f(z)
    \end{equation*}
    cumple
    \begin{equation*}
      \lim_{z \rightarrow z_0} g(z) \ne 0
    \end{equation*}
    y \(g\) es holomorfa en un entorno de \(z_0\),
    decimos que \(f\)
    tiene un \emph{polo} de orden \(m\) en \(z_0\).%
      \index{singularidad!polo}
    Si \(m = 1\),
    también se le llama \emph{polo simple}.

    Es claro que si \(f\) tiene un cero de multiplicidad \(m\)
    en \(z_0\),
    entonces \(1 / f(z)\) tiene un polo de multiplicidad \(m\)
    en ese punto.
  \item[Singularidad esencial:]
    Si \(f\) tiene una singularidad en \(z_0\)
    que no es removible ni es un polo,
    es una \emph{singularidad esencial}.%
      \index{singularidad!esencial}
  \end{description}
  Una definición que será útil más adelante es la siguiente:
  \begin{definition}
    A una función \(f\) que es holomorfa sobre una región \(D\)
    salvo polos se le llama \emph{meromorfa}.%
      \index{C (numeros complejos)@\(\mathbb{C}\) (números complejos)!funcion meromorfa@función meromorfa}
  \end{definition}

  Nótese que al conocer todas las derivadas de una función
  en un punto cualquiera,
  podemos expandirla en serie de Taylor alrededor de ese punto,
  y esta serie converge en el disco abierto centrado en ese punto
  y que llega a la singularidad más cercana.
  De esta forma,
  podemos \emph{extender analíticamente} la función%
      \index{C (numeros complejos)@\(\mathbb{C}\) (números complejos)!extension analitica@extensión analítica}
  a nuevas áreas del plano,
  salvo que las singularidades formen un conjunto denso
  en alguna curva cerrada.

  Algunos resultados al respecto son los siguientes:
  \begin{theorem}[De Riemann sobre singularidades removibles]
    \label{theo:Riemann-removable-singularity}
    Suponga que \(f\) es holomorfa sobre el disco abierto perforado
    \(\{ z \colon 0 < \lvert z - z_0 \rvert < R \}\),
    y que
    \begin{equation*}
      \lim_{z \rightarrow z_0} (z - z_0) f(z)
	= 0
    \end{equation*}
    Entonces \(f\) tiene una singularidad removible en \(z_0\).
  \end{theorem}
  \begin{proof}
    Sea \(z\) un punto en el disco perforado
    \(\{ z \colon 0 < \lvert z - z_0 \rvert < R \}\),
    y sean \(r_1\) y \(r_2\) tales que
    \(0 < r_1 < \lvert z \rvert < r_2\),
    y sean \(C_1\) y \(C_2\) circunferencias
    de radios \(r_1\) y \(r_2\) respectivamente
    centradas en \(z_0\),
    orientadas ambas en sentido positivo.
    La función:
    \begin{equation*}
      g(\zeta)
       = \begin{cases}
	   \displaystyle f'(z)
				   & \zeta =   z \\
	   \\
	   \displaystyle \frac{f(\zeta) - f(z)}{\zeta - z}
				   & \zeta \ne z
	 \end{cases}
    \end{equation*}
    claramente es holomorfa
    en \(\{ \zeta \colon \lvert \zeta - z_0 \rvert < R \}\).
    Por el teorema de Cauchy:
    \begin{equation*}
      \int_{C_1} g(\zeta) \, \mathrm{d} \zeta
	= \int_{C_2} g(\zeta) \, \mathrm{d} \zeta
    \end{equation*}
    De la definición de \(g\) esto significa:
    \begin{equation*}
      \int_{C_1} \frac{f(\zeta)}{\zeta - z} \, \mathrm{d} \zeta
	-  f(z) \int_{C_1} \frac{1}{\zeta - z} \, \mathrm{d} \zeta
	= \int_{C_2} \frac{f(\zeta)}{\zeta - z} \, \mathrm{d} \zeta
	    -  f(z) \int_{C_2} \frac{1}{\zeta - z}
		      \, \mathrm{d} \zeta
    \end{equation*}
    En la región
      \(\{ \zeta \colon \lvert \zeta - z_0 \rvert
	   < \lvert z - z_0 \rvert \}\)
    (que incluye la curva \(C_1\) y su interior,
     pero excluye \(z\))
    es holomorfa la función:
    \begin{equation*}
      \frac{1}{\zeta - z}
    \end{equation*}
    y por lo tanto:
    \begin{equation*}
      \int_{C_1} \frac{1}{\zeta - z} \, \mathrm{d} \zeta
	= 0
    \end{equation*}
    Por el otro lado,
    la fórmula integral de Cauchy
    aplicada a la función \(1 = 1(z)\) da:%
      \index{Cauchy, formula integral de@Cauchy, fórmula integral de}
    \begin{equation*}
      \int_{C_2} \frac{1}{\zeta - z} \, \mathrm{d} \zeta
	= 2 \pi \mathrm{i}
    \end{equation*}
    Por hipótesis,
    dado \(\epsilon > 0\) hay \(\delta > 0\)
    tal que si \(0 < \lvert \zeta - z_0 \rvert < \delta\)
    entonces \(\lvert (\zeta - z_0) f(\zeta) \rvert < \epsilon\).
    Sin perder generalidad podemos suponer:
    \begin{equation*}
      \delta < \frac{1}{2} \, \lvert z - z_0 \rvert
    \end{equation*}
    Si elegimos \(r_1 = \delta\),
    por~\eqref{eq:complex-integral-bound}:
    \begin{equation*}
      \left\lvert
	\int_{C_1} \frac{f(\zeta)}{\zeta - z} \, \mathrm{d} \zeta
      \right\rvert
	\le \left\lvert
	      \int_{C_1} \frac{(\zeta - z_0) f(\zeta)}
			      {(\zeta - z_0) \, (\zeta - z)}
			 \, \mathrm{d} \zeta
	    \right\rvert
	\le \frac{\epsilon}
		 {\delta ( \lvert z - z_0 \rvert - \delta )}
	      \cdot 2 \pi \delta
	\le \frac{4 \pi \epsilon}{\lvert z - z_0 \rvert}
    \end{equation*}
    Como \(\epsilon\) es arbitrario,
    la integral se anula.
    Concluimos:
    \begin{equation*}
      f(z)
	= \frac{1}{2 \pi \mathrm{i}} \,
	    \int_{C_2} \frac{f(\zeta)}{\zeta - z}
	      \, \mathrm{d} \zeta
    \end{equation*}
    Esta definición vale sobre el disco perforado
    \(\{ z  \colon 0 < \lvert z - z_0 \rvert < R \}\),
    pero la integral define una función holomorfa
    en \(\{z \colon \lvert z - z_0 \rvert < R \}\),
    con lo que definiendo \(f(z_0)\) mediante:
    \begin{equation*}
      f(z_0)
	= \frac{1}{2 \pi \mathrm{i}} \,
	    \int_{C_2} \frac{f(\zeta)}{\zeta - z_0}
	      \, \mathrm{d} \zeta
    \end{equation*}
    la función \(f\) es holomorfa
    en \(\{z \colon \lvert z - z_0 \rvert < R \}\),
    como prometimos.
  \end{proof}
  La condición del teorema~\ref{theo:Riemann-removable-singularity}
  se cumple si \(\lvert f(z) \rvert\) es acotada.
  Como una función holomorfa es continua,
  si la singularidad es removible basta definir:
  \begin{equation*}
    f(z_0)
      = \lim_{z \rightarrow z_0} f(z)
  \end{equation*}

  Cerca de singularidades esenciales las funciones
  se comportan en forma descontrolada.
  El gran teorema de Picard~%
    \cite{picard79:_sur_propr_fonct_entier}
  (para demostración ver textos de análisis complejo,
   como el de Conway\cite{conway78:_funct_one_compl_variab_i})
  muestra que en todo entorno de la singularidad
  la función toma todos los valores posibles,%
    \index{Picard, teorema de}%
    \index{Picard, Charles Emile@Picard, Charles Émile}
  salvo posiblemente uno.
  El resultado siguiente es mucho más restringido,
  pero también más fácil de demostrar:
  \begin{theorem}[Casorati-Weierstraß]
    \index{Casorati-Weierstrass, teorema de@Casorati-Weierstraß, teorema de}
    \label{theo:Casorati-Weierstrass}
    Sea \(f\) holomorfa en el disco perforado
    \(\{ z \colon 0 < \lvert z - z_0 \rvert < R \}\),
    con una singularidad esencial en \(z_0\).
    Entonces dado cualquier \(w \in \mathbb{C}\)
    y números reales arbitrarios
    \(\epsilon > 0\) y \(\delta > 0\)
    hay \(z\) en el disco perforado tal que
    \begin{equation*}
      0 < \lvert z - z_0 \rvert < \delta
      \text{\ y\ }
      \lvert f(z) - w \rvert < \epsilon
    \end{equation*}
  \end{theorem}
  \begin{proof}
    Por contradicción.%
      \index{demostracion@demostración!contradiccion@contradicción}
    Suponga que hay \(w \in \mathbb{C}\),
    \(\epsilon > 0\) y	\(\delta > 0\) tales que
    \(\lvert f(z) - w \lvert \ge \epsilon\)
    siempre que \(0 < \lvert z - z_0 \rvert < \delta\).
    Entonces la función
    \begin{equation*}
      g(z)
	= \frac{1}{f(z) - w}
    \end{equation*}
    es holomorfa y acotada en el disco perforado
    \(\{ z \colon 0 < \lvert z - z_0 \rvert < \delta \}\),
    con una singularidad removible en \(z_0\)
    (teorema~\ref{theo:Riemann-removable-singularity}).
    Definiendo \(g(z_0)\) apropiadamente,
    \(g(z)\) es holomorfa
    en \(\{ z \colon \lvert z - z_0 \rvert < \delta \}\),
    y no es idénticamente cero en este disco.
    Note que:
    \begin{equation*}
      f(z)
	= w + \frac{1}{g(z)}
    \end{equation*}
    con lo que si \(g(z_0) \ne 0\),
    \(f\) es derivable en \(z_0\);
    si \(g(z_0) = 0\) entonces \(f\) tiene un polo en \(z_0\).
    Esto contradice la hipótesis
    de que \(f\) tiene una singularidad esencial
    en \(z_0\).
  \end{proof}

  Un par de ejemplos servirán para clarificar el tema.
  La función \(\mathrm{e}^{1 / z}\)
  es holomorfa en todo \(\mathbb{C}\)
  salvo en \(z = 0\),
  que es una singularidad aislada.
  Si nos restringimos al eje real,
  vemos que:
  \begin{equation*}
    \lim_{x \rightarrow 0^+} \mathrm{e}^{1 / x}
      = \lim_{u \rightarrow \infty} \mathrm{e}^u
      = \infty
  \end{equation*}
  y la singularidad no es removible.
  Por otro lado,
  para \(n \in \mathbb{N}\):
  \begin{equation*}
    \lim_{x \rightarrow 0^+} x^n \mathrm{e}^{1 / x}
      = \lim_{u \rightarrow \infty} \frac{\mathrm{e}^u}{u^n}
      = \infty
  \end{equation*}
  y tampoco es un polo.
  En consecuencia,
  es una singularidad esencial.

  Analicemos ahora la función:
  \begin{equation*}
    f(z)
      = \frac{\mathrm{e}^z - 1}{z (z - 1)}
  \end{equation*}
  Es claro que tiene singularidades aisladas
  en \(z = 0\) y \(z = 1\).
  En \(z = 1\) tiene un polo simple,
  ya que:
  \begin{equation*}
    g(z)
      = (z - 1) f(z)
      = \frac{\mathrm{e}^z - 1}{z}
  \end{equation*}
  y \(g(1) = \mathrm{e} - 1 \ne 0\).
  La singularidad en \(z = 0\) es removible,
  ya que por la regla de l'Hôpital:%
    \index{Hopital, regla de@l'Hôpital, regla de}
  \begin{equation*}
    \lim_{z \rightarrow 0} \frac{\mathrm{e}^z - 1}{z (z - 1)}
      = \lim_{z \rightarrow 0} \frac{\mathrm{e}^z}{2 z - 1}
      = -1
  \end{equation*}
  Podemos analizar lo que ocurre en \(\infty\)
  considerando \(f(1 / z)\),
  que tiene una singularidad aislada en \(0\).
  Según el comportamiento,
  diremos que \(f\) tiene una singularidad removible,
  un polo o una singularidad esencial en \(\infty\).
  Primeramente,
  el límite:
  \begin{equation*}
    \lim_{\lvert z \rvert \rightarrow \infty}
      \frac{\mathrm{e}^z - 1}{z (z - 1)}
  \end{equation*}
  no existe,
  ya que:
  \begin{equation*}
    \lim_{x \rightarrow \infty} \frac{\mathrm{e}^x - 1}{x (x - 1)}
      = \infty
    \qquad
    \lim_{x \rightarrow -\infty} \frac{\mathrm{e}^x - 1}{x (x - 1)}
      = 0
  \end{equation*}
  O sea,
  la singularidad no es removible.
  Enseguida,
  para \(n \in \mathbb{N}\) cualquiera,
  no existe el límite:
  \begin{equation*}
    \lim_{\lvert z \rvert \rightarrow \infty}
      \frac{\mathrm{e}^z - 1}{z^{n + 1} (z - 1)}
  \end{equation*}
  porque:
  \begin{equation*}
    \lim_{x \rightarrow \infty}
	\frac{\mathrm{e}^x - 1}{x^{n + 1} (x - 1)}
      = \infty
    \qquad
    \lim_{x \rightarrow -\infty}
	\frac{\mathrm{e}^x - 1}{x^{n + 1} (x - 1)}
      = 0
  \end{equation*}
  Por tanto,
  al no ser removible ni un polo,
  es una singularidad esencial.

  Es simple demostrar que si
  \(\lim_{\lvert z \rvert \rightarrow \infty}
	\lvert f(z) \rvert = \infty\),
  entonces \(f\) es entera si y solo si es un polinomio.
  O sea,
  una función entera que no es un polinomio
  tiene una singularidad esencial en \(\infty\).
  Supongamos que \(f\) tiene un polo de orden \(m\) en \(\infty\).
  Restando un polinomio \(p\) de \(f\),
  de ser necesario,
  podemos arreglar que \(f(z) - p(z)\)
  tenga un cero de orden \(m\) en \(0\).
  Claramente \(p\) tiene grado a lo más \(m\).
  La función:
  \begin{equation*}
    g(z)
      = \frac{f(z) - p(z)}{z^m}
  \end{equation*}
  es entera y acotada,
  con lo que por el teorema de Liouville
  (teorema~\ref{theo:Liouville})
  es constante.
  Entonces \(f\) es un polinomio.

  La función tangente se define como:
  \begin{equation*}
    \tan z
      = \frac{\sin z}{\cos z}
  \end{equation*}
  Esta función tiene singularidades no removibles
  en los ceros de \(\cos z\),
  que son \((k + 1 / 2) \pi\) para \(k \in \mathbb{Z}\).
  Por l'Hôpital vemos que:%
    \index{Hopital, regla de@l'Hôpital, regla de}
  \begin{equation*}
    \lim_{z \rightarrow \left( k + \frac{1}{2} \right) \, \pi}
       \frac{\cos z}{z - \left( k + \frac{1}{2} \right) \, \pi}
      = \lim_{z \rightarrow
	    \left( k + \frac{1}{2} \right) \, \pi} - \sin z
      = -1
  \end{equation*}
  Como los ceros de \(\cos z\) son simples,
  las singularidades mencionadas son polos simples.
  La singularidad en \(\infty\) no es aislada,
  ya que todo entorno de \(z = 0\) de \(\tan 1 / z\)
  contiene infinitas singularidades
  (no hay \(R\) tal que \(\tan 1 / z\) es holomorfa
   en \(\{ z \colon R < \lvert z \rvert < \infty \}\)).

\subsection{Series de Laurent}
\label{sec:Laurent-series}

  Supongamos la función \(f\) holomorfa en el disco perforado
  \(\{ z \colon 0 < \lvert z - z_0 \rvert < R \}\),
  con un polo de orden \(m\) en \(z_0\).
  De la discusión precedente esto significa que la función siguiente
  es holomorfa
  en \(\{ z \colon \lvert z - z_0 \rvert < R \}\):
  \begin{equation*}
    g(z)
      = (z - z_0)^m f(z)
  \end{equation*}
  Como es holomorfa,
  tiene una serie de Taylor,
  lo que significa que podemos escribir:
  \begin{equation}
    \label{eq:Laurent-series-1}
    f(z)
      = \sum_{k \ge 0} \frac{g^{(k)}(z_0)}{k!} (z - z_0)^{k - m}
  \end{equation}
  A la expresión:
  \begin{equation}
    \label{eq:Laurent-series-pp}
    \frac{g(z_0)}{(z - z_0)^m}
      + \frac{g'(z_0)}{(z - z_0)^{m - 1}}
      + \frac{g''(z_0)}{2! (z - z_0)^{m - 1}}
      + \dotsb
      + \frac{g^{(m - 1)}(z_0)}{(m - 1)! (z - z_0)}
  \end{equation}
  se le llama la \emph{parte principal}%
    \index{parte principal}
  de la serie~\eqref{eq:Laurent-series-1},
  que se anota \(\pp(f; z_0)\).

  Nos interesa formalizar y extender lo anterior.
  \begin{theorem}
    \label{theo:complex-f1+f2-singularity}
    Sea \(f\) una función holomorfa
    en el disco perforado
      \(\{ z \colon 0 < \lvert z - z_0 \rvert < R \}\),
    con una singularidad aislada en \(z_0\).
    Entonces existen funciones únicas \(f_1\) y \(f_2\) tales que:
    \begin{enumerate}[label=(\alph*), ref=(\alph*)]
    \item
      \label{theo:complex-f1+f2-singularity:a}
      \(f(z) = f_1(z) + f_2(z)\)
      en \(\{ z \colon 0 < \lvert z - z_0 \rvert < R \}\)
    \item
      \label{theo:complex-f1+f2-singularity:b}
      \(f_1\) es holomorfa en \(\mathbb{C}\),
      excepto posiblemente en \(z_0\)
    \item
      \label{theo:complex-f1+f2-singularity:c}
      \(f_1(z) \rightarrow 0\)
	 cuando \(\lvert z \rvert \rightarrow \infty\)
    \item
      \label{theo:complex-f1+f2-singularity:d}
      \(f_2\) es holomorfa
	 en \(\{ z \colon \lvert z - z_0 \rvert < R \}\)
    \end{enumerate}
  \end{theorem}
  \begin{proof}
    Comenzamos como la demostración
    del teorema~\ref{theo:Riemann-removable-singularity}.
    Sea \(z\) un punto en el disco perforado
    \(\{ z \colon 0 < \lvert z - z_0 \rvert < R \}\),
    y sean \(0 < r_1 < \lvert z - z_0 \rvert < r_2 < R\).
    Sean además \(C_1\) y \(C_2\)
    las circunferencias
    de radios \(r_1\) y \(r_2\) respectivamente
    alrededor de \(z_0\).
    Obtenemos:
    \begin{equation*}
      f(z)
	= \frac{1}{2 \pi \mathrm{i}} \,
	    \int_{C_2} \frac{f(\zeta)}{\zeta - z}
	      \, \mathrm{d} \zeta
	      - \frac{1}{2 \pi \mathrm{i}} \,
		  \int_{C_1} \frac{f(\zeta)}{\zeta - z}
		    \, \mathrm{d} \zeta
    \end{equation*}
    Defina entonces:
    \begin{equation*}
      f_1(z)
	= - \frac{1}{2 \pi \mathrm{i}} \,
	      \int_{C_1} \frac{f(\zeta)}{\zeta - z}
		\, \mathrm{d} \zeta
     \qquad
      f_2(z)
	= \frac{1}{2 \pi \mathrm{i}} \,
	    \int_{C_2} \frac{f(\zeta)}{\zeta - z}
	      \, \mathrm{d} \zeta
    \end{equation*}
    La parte~\ref{theo:complex-f1+f2-singularity:a} es inmediata.
    Para la parte~\ref{theo:complex-f1+f2-singularity:d},
    la función \(f_2\) definida por la integral es holomorfa
    en el disco \(\{ z \colon \lvert z - z_0 \rvert < r_2 \}\)
    (como en el teorema~\ref{theo:Riemann-removable-singularity}).
    Para~\ref{theo:complex-f1+f2-singularity:b},
    la función \(f_2\) es holomorfa en el anillo
    \(\{ z \colon r_1 < \lvert z - z_0 \rvert < r_2 \}\).
    Como \(f\) y \(f_2\) no dependen de \(r_1\),
    tampoco depende de \(r_1\)
    la función \(f_1(z) = f(z) - f_2(z)\).
    Tampoco depende de \(r_2\) la función \(f_2\),
    por un razonamiento similar.
    Se ve que:
    \begin{equation*}
      \lim_{\lvert z \rvert \rightarrow \infty}
	\int_{C_1} \frac{f(\zeta)}{\zeta - z} \, \mathrm{d} \zeta
	= 0
    \end{equation*}
    que es la parte~\ref{theo:complex-f1+f2-singularity:c}.
    Para demostrar que \(f_1\) y \(f_2\) son únicas,
    suponga funciones \(g_1\) y \(g_2\) con las mismas propiedades
    en el disco perforado
      \(\{ z \colon 0 < \lvert z - z_0 \rvert < R \}\),
    con lo que allí:
    \begin{equation*}
      f_1(z) - g_1(z)
	= g_2(z) - f_2(z)
    \end{equation*}
    Defina:
    \begin{equation*}
      F(z)
	= \begin{cases}
	    g_2(z) - f_2(z) & \lvert z - z_0 \rvert < R \\
	    g_1(z) - f_1(z) & \lvert z - z_0 \rvert \ge R
	  \end{cases}
    \end{equation*}
    Entonces \(F\) es entera.
    Pero por la parte~\ref{theo:complex-f1+f2-singularity:c}
    \(F(z) \rightarrow 0\)
    cuando \(\lvert z \rvert \rightarrow \infty\),
    con lo que \(F\) es acotada.
    Por el teorema de Liouville,%
      \index{Liouville, teorema de}
    \(F(z)\) es constante,
    igual al límite mencionado antes;
    con lo que \(F(z) = 0\) para todo \(z \in \mathbb{C}\),
    y las funciones coinciden.
  \end{proof}
  Estamos en condiciones de completar nuestro ejemplo inicial.
  \begin{theorem}[Serie de Laurent]
    \label{theo:Laurent-series}
    Sea \(f\) holomorfa en el disco perforado
    \(\{ z \colon 0 < \lvert z - z_0 \rvert < R \}\),
    con una singularidad aislada en \(z_0\).
    Sea \(C\) una circunferencia de radio \(r\),
    donde \(0 < r < R\),
    y para cada \(n \in \mathbb{Z}\) sea
    \begin{equation}
      \label{eq:Laurent-series-coefficient}
      a_n
	= \frac{1}{2 \pi \mathrm{i}} \,
	    \int_C \frac{f(z)}{(z - z_0)^{n + 1}} \, \mathrm{d} z
    \end{equation}
    Entonces la serie
    \begin{equation}
      \label{eq:Laurent-series}
      f(z)
	= \sum_n a_n (z - z_0)^n
    \end{equation}
    converge en el disco perforado
      \(\{ z \colon 0 < \lvert z - z_0 \rvert < R \}\).
    Más aún,
    la convergencia es uniforme en todo anillo
      \(\{ z \colon r_1 < \lvert z - z_0 \rvert < r_2 \}\),
    donde \(0 < r_1 < r_2 < R\).
  \end{theorem}
  Nótese que la convergencia uniforme de la serie en el anillo
    \(\{ z \colon r_1 < \lvert z - z_0 \rvert < r_2 \}\)
  significa que para todo \(\epsilon > 0\)
  existe \(N_0 = N_0(\epsilon, r_1, r_2)\),
  independiente de \(z\),
  tal que si \(N_1 \ge N_0\) y \(N_2 \ge N_0\):
  \begin{equation*}
    \left\lvert
      f(z)
	- \sum_{-N_1 \le n \le N_2} a_n (z - z_0)^n
    \right\rvert
      < \epsilon
  \end{equation*}
  A la serie~\eqref{eq:Laurent-series} se la llama
  la \emph{serie de Laurent} de \(f\) alrededor de \(z_0\).%
    \index{Laurent, serie de|textbfhy}
  \begin{proof}
    El primer paso es demostrar
    que la serie~\eqref{eq:Laurent-series}
    converge uniformemente a \(f(z)\)
    sobre la circunferencia \(C\)
    centrada en \(z_0\) de radio \(r\),
    donde \(0 < r < R\),
    y los coeficientes
    están dados por~\eqref{eq:Laurent-series-coefficient}.
    Suponga \(n \in \mathbb{Z}\) dado y fijo.
    Para cualquier \(\epsilon > 0\)
    podemos elegir \(N_1\) y \(N_2\)
    suficientemente grandes para que
    \(-N_1 \le n \le N_2\)
    y tal que para todo \(z \in C\):
    \begin{equation*}
      \left\lvert
	f(z)
	  - \sum_{-N_1 \le k \le N_2} a_k (z - z_0)^k
      \right\rvert
	< \epsilon
    \end{equation*}
    Por la cota~\eqref{eq:complex-integral-bound} es:
    \begin{equation*}
      \left\lvert
	\frac{1}{2 \pi \mathrm{i}} \,
	  \int_C
	    \left(
	      f(z)
	      - \sum_{-N_1 \le k \le N_2} a_k (z - z_0)^k
	    \right) \,
	      \frac{1}{(z - z_0)^n}
	      \, \mathrm{d} z
      \right\rvert
	<\frac{\epsilon}{r^n}
    \end{equation*}
    Como:
    \begin{equation*}
      \frac{1}{2 \pi \mathrm{i}}
	\, \int_C (z - z_0)^k \, \mathrm{d} z
	= [k = -1]
    \end{equation*}
    resulta:
    \begin{equation*}
      \frac{1}{2 \pi \mathrm{i}} \,
	\int_C
	  \left(
	    \sum_{-N_1 \le k \le N_2} a_k (z - z_0)^k
	  \right) \,
	    \frac{1}{(z - z_0)^{n + 1}}
	    \, \mathrm{d} z
       = a_n
    \end{equation*}
    Esto permite simplificar:
    \begin{equation*}
      \left\lvert
	\frac{1}{2 \pi \mathrm{i}} \,
	  \int_C \frac{f(z)}{(z - z_0)^{n + 1}} \, \mathrm{d} z
	  - a_n
      \right\rvert
	<\frac{\epsilon}{r^n}
    \end{equation*}
    Como \(\epsilon\) es arbitrario,
    resulta~\eqref{eq:Laurent-series-coefficient}.

    Queda por demostrar
    que la representación~\eqref{eq:Laurent-series}
    vale en el disco perforado
    \(\{ z \colon 0 < \lvert z - z_0 \rvert < R \}\),
    y que la convergencia es uniforme en todo anillo
    \(\{ z \colon r_1 < \lvert z - z_0 \rvert < r_2 \}\)
    con \(0 < r_1 < r_2 < R\).
    Suponga \(r_1 < r < r_2\).
    Por el teorema~\ref{theo:complex-f1+f2-singularity}
    podemos escribir \(f(z) = f_1(z) + f_2(z)\)
    donde \(f_1\) y \(f_2\) son únicas
    y cumplen las condiciones~\ref{theo:complex-f1+f2-singularity:a}
    a~\ref{theo:complex-f1+f2-singularity:d}.
    Como \(f_2\) es holomorfa en el disco
      \(\{ z \colon \lvert z - z_0 \rvert < R \}\),
    su serie de Taylor:
    \begin{equation*}
      f_2(z)
	= \sum_{n \ge 0} A_n (z - z_0)^n
    \end{equation*}
    converge
    en el disco \(\{ z \colon \lvert z - z_0 \rvert < R \}\),
    uniformemente
    en el disco \(\{ z \colon \lvert z - z_0 \rvert < r_2 \}\).
    Para estudiar \(f_1\),
    efectuamos el cambio de variables:
    \begin{equation*}
      w = \frac{1}{z - z_0}
      \qquad
      z = z_0 + \frac{1}{w}
    \end{equation*}
    La función:
    \begin{equation*}
      f_1(z)
	= f_1 \left( \frac{1}{w} + z_0 \right)
    \end{equation*}
    es entera en \(w\),
    con lo que la serie de Taylor:%
      \index{Taylor, serie de}
    \begin{equation*}
      f_1 \left( \frac{1}{w} + z_0 \right)
	= \sum_{m \ge 1} B_m w^m
    \end{equation*}
    converge en \(\mathbb{C}\),
    uniformemente en el disco cerrado
      \(\{ w \colon \lvert w \rvert \le 1 / r_1 \}\).
    El coeficiente \(B_0\) se anula.
    Es el valor de \(f_1\) en \(w = 0\),
    vale decir \(z = \infty\);
    y por la parte~\ref{theo:complex-f1+f2-singularity:c}
    del teorema~\ref{theo:complex-f1+f2-singularity}
    esto se anula.
    Combinando las series para \(f_1\) en términos de \(z - z_0\)
    y para \(f_2\)
    resulta lo prometido.
  \end{proof}
  El tipo de singularidad aislada
  es sencillo de ver de los coeficientes
  de la serie de Laurent~\eqref{eq:Laurent-series}:
  \begin{corollary}
    \label{cor:Laurent-singularity}
    Sea \(f\) holomorfa en el disco perforado
      \(\{ z \colon 0 < \lvert z - z_0 \rvert < R \}\),
    y serie de Laurent dada por~\eqref{eq:Laurent-series}
    con coeficientes \(a_n\)
    como en~\eqref{eq:Laurent-series-coefficient}.
    Entonces:
    \begin{enumerate}[label=(\alph*), ref=(\alph*)]
    \item
      \label{cor:Laurent-singularity:a}
      La función \(f\) es diferenciable en \(z_0\)
      o tiene una singularidad removible si y solo si
      \(a_n = 0\) para todo \(n < 0\)
    \item
      \label{cor:Laurent-singularity:b}
      La función \(f\) tiene un polo
      si y solo si un número finito pero no nulo
      de coeficientes \(a_n\) con \(n\) negativo
      son diferentes de cero
    \item
      \label{cor:Laurent-singularity:c}
      La función \(f\) tiene una singularidad esencial
      si y solo si un número infinito de coeficientes \(a_n\)
      con \(n\) negativo son diferentes de cero
    \end{enumerate}
  \end{corollary}
  \begin{proof}
    Para la parte~\ref{cor:Laurent-singularity:a},
    sabemos que si \(f\)
    tiene una singularidad removible en \(z_0\),
    podemos hacerla holomorfa definiendo \(f(z_0)\) adecuadamente.
    Y la función holomorfa tiene serie de Taylor,
    vale decir,
    sin términos de índice negativo.

    Para la parte~\ref{cor:Laurent-singularity:b},
    si solo un número finito de coeficientes de índice negativo
    no se anulan,
    habrá \(m > 0\) tal que \(a_{-m} \ne 0\)
    pero \(a_n = 0\) para todo \(n < -m\).
    En tal caso:
    \begin{equation*}
      f(z)
	= \sum_{n  \ge -m} a_n (z - z_0)^n
    \end{equation*}
    de forma que:
    \begin{equation*}
      g(z)
	= (z - z_0)^m f(z)
    \end{equation*}
    es holomorfa en un entorno de \(z_0\),
    y \(g(z_0) = a_{-m} \ne 0\),
    y \(f\) tiene un polo de orden \(m\) en \(z_0\).

    La parte~\ref{cor:Laurent-singularity:c}
    resulta por no ser~\ref{cor:Laurent-singularity:a}
    ni ~\ref{cor:Laurent-singularity:b}.
  \end{proof}
  Como la serie de Laurent es única,
  podemos usar técnicas alternativas
  a la fórmula~\eqref{eq:Laurent-series-coefficient}
  para obtener los coeficientes.
  Por ejemplo,
  de la substitución \(w = 1 / z\) en la serie para \(\mathrm{e}^w\)
  tenemos directamente:
  \begin{equation*}
    \mathrm{e}^{1/z}
      = \sum_{n \ge 0} \frac{1}{n! z^n}
  \end{equation*}
  Nuevamente concluimos
  que la singularidad en \(z = 0\) de esta función
  es esencial.

\subsection{Residuos}
\label{sec:residues}

  Al calcular la integral sobre una curva cerrada simple
  que encierra una región en la que el integrando no es holomorfo,
  el resultado no siempre es cero.
  Primeramente tenemos:
  \begin{lemma}
    \label{lem:integral-residue}
    Sea \(f\) una función holomorfa
    en la región conexa simple \(D\),
    excepto una singularidad aislada en \(z_0\),
    y sea:
    \begin{equation*}
      f_1(z)
	= \sum_{n \le -1} a_n (z - z_0)^n
    \end{equation*}
    la parte principal de la serie de Laurent de \(f\) en \(z_0\).
    Sea también \(\gamma \subset D\) una curva cerrada simple suave
    que se sigue en la dirección positiva,
    que no pasa por \(z_0\).
    Entonces:
    \begin{equation*}
      \frac{1}{2 \pi \mathrm{i}}
	\, \int_\gamma f(\zeta) \, \mathrm{d} \zeta
	= \begin{cases}
	    a_{-1}
	       & \text{si \(z_0\) está en el interior
		       de \(\gamma\)} \\
	    0
	       & \text{si \(z_0\) está en el exterior de \(\gamma\)}
	  \end{cases}
   \end{equation*}
  \end{lemma}
  \begin{proof}
    Si \(z_0\) está al exterior de \(\gamma\),
    vemos que \(\gamma\) es holomorfa a un punto en \(D\),
    y la integral es cero.

    En caso que \(z_0\) está al interior de \(\gamma\),
    la integral no es más que el coeficiente \(a_{-1}\)
    de la serie de Laurent
    según~\eqref{eq:Laurent-series-coefficient}.
  \end{proof}
  Al coeficiente \(a_{-1}\)
  se le llama el \emph{residuo} de \(f\) en \(z_0\),
  y se anota \(a_{-1} = \res(f, z_0)\).

  Con esto podemos demostrar una forma simple
  del teorema de residuos de Cauchy.
  \begin{theorem}[De residuos de Cauchy]
    \index{Cauchy, teorema de residuos de|textbfhy}
    \label{theo:Cauchy-residues}
    Sea \(f\) holomorfa,
    salvo singularidades aisladas \(z_1, \dotsc, z_n\),
    en la región simple conexa \(D\),
    y \(\gamma \subset D\) una curva cerrada simple
    trazada en dirección positiva.
    Entonces:
    \begin{equation}
      \label{eq:Cauchy-integral-residues}
      \int_\gamma f(z) \, \mathrm{d} z
	= \sum_{\substack{1 \le k \le n \\
			  \text{\(z_k\) interior a \(\gamma\)}}}
	    \res(f, z_k)
    \end{equation}
  \end{theorem}
  \begin{proof}
    Sea \(f_k(z)\) la parte principal de \(f(z)\) en \(z_k\).
    Por el teorema~\ref{theo:complex-f1+f2-singularity}
    sabemos que \(f_k\) es holomorfa excepto en \(z_k\),
    por lo que la función:
    \begin{equation*}
      g(z)
	= f(z) - \sum_{1 \le k \le n} f_k(z)
    \end{equation*}
    es holomorfa en \(D\).
    Aplicando el lema~\ref{lem:integral-residue} resulta:
    \begin{equation*}
      \int_\gamma f(z) \, \mathrm{d} z
	= \int_\gamma g(z) \, \mathrm{d} z
	    + \sum_{1 \le k \le n}
		\int_\gamma f_k(z) \, \mathrm{d} z
	= 0 + \sum_{\substack{1 \le k \le n \\
			  \text{\(z_k\) interior a \(\gamma\)}}}
		\res(f_k, z_k)
	= \sum_{\substack{1 \le k \le n \\
			  \text{\(z_k\) interior a \(\gamma\)}}}
	    \res(f, z_k)
    \qedhere
    \end{equation*}
  \end{proof}
  Para que esto resulte útil,
  requerimos formas de calcular el residuo de \(f\)
  en una singularidad \(z_0\).

  Si la singularidad es removible,
  \(f\) tiene una serie de Taylor alrededor de \(z_0\),%
    \index{Taylor, serie de}
  y el residuo es cero.

  Enseguida,
  si \(z_0\) es un polo simple de \(f\),
  entonces tiene serie de Laurent:%
    \index{Laurent, serie de}
  \begin{equation*}
    f(z)
      = \frac{a_{-1}}{z - z_0} + g(z)
  \end{equation*}
  La función \(g(z)\) está representada por una serie de Taylor,
  con lo que es holomorfa en un disco centrado en \(z_0\),
  y \(\lim_{z \rightarrow z_0} (z - z_0) g(z) = 0\).
  O sea:
  \begin{equation*}
    a_{-1}
      = \lim_{z \rightarrow z_0} (z - z_0) f(z)
  \end{equation*}
  Una forma alternativa útil es la siguiente:
  Suponga que \(f(z) = g(z) / h(z)\),
  donde \(g\) y \(h\) son holomorfas,
  \(g(z_0) \ne 0\)
  y \(h\) tiene un cero simple en \(z_0\).
  Entonces,
  como \(h(z_0) = 0\):
  \begin{equation*}
    \lim_{z \rightarrow z_0} (z - z_0) \frac{g(z)}{h(z)}
      = \frac{\lim_{z \rightarrow z_0} g(z)}
	     {\lim_{z \rightarrow z_0} \frac{h(z) - h(z_0)}
					    {z - z_0}}
      = \frac{g(z_0)}{h'(z_0)}
  \end{equation*}

  Si \(z_0\) es un polo de orden \(m\) de \(f\),
  tenemos:
  \begin{equation*}
    f(z)
      = \frac{a_{-m}}{(z - z_0)^m}
	  + \frac{a_{-m + 1}}{(z - z_0)^{m - 1}}
	  + \dotsb
	  + \frac{a_{-1}}{z - z_0}
	  + g(z)
  \end{equation*}
  Entonces es holomorfa:
  \begin{equation*}
    (z - z_0)^m f(z)
      = a_{-m}
	 + a_{- m + 1} (z - z_0)
	 + \dotsb
	 + a_{-1} (z - z_0)^{m - 1}
	 + (z - z_0)^m g(z)
  \end{equation*}
  Derivando \(m - 1\) veces:
  \begin{equation*}
    \frac{\mathrm{d}^{m - 1}}{\mathrm{d} z^{m - 1}} \,
      ((z - z_0)^m f(z))
      = a_{-1} (m - 1)!
	  + \frac{\mathrm{d}^{m - 1}}{\mathrm{d} z^{m - 1}} \,
	      ((z - z_0)^m f(z))
  \end{equation*}
  Como \(g\) es holomorfa,
  el segundo término tiende a 0 cuando \(z \rightarrow z_0\):
  \begin{equation*}
    a_{-1}
      = \frac{1}{(m - 1)!} \,
	  \lim_{z \rightarrow z_0}
	    \frac{\mathrm{d}^{m - 1}}{\mathrm{d} z^{m - 1}} \,
	      ((z - z_0)^m f(z))
  \end{equation*}

  Un resultado que requeriremos más adelante es el siguiente.
  Para \(0 < \alpha < 1\) tenemos la integral real:
  \begin{equation}
    \label{eq:integral-Gamma(z)Gamma(1-z)}
    \int_{-\infty}^\infty
      \frac{\mathrm{e}^{\alpha x}}{1 + \mathrm{e}^x}
	 \, \mathrm{d} x
      = \frac{\pi}{\sin \pi \alpha}
  \end{equation}
  Usamos como contorno
  el rectángulo con vértices en
  \(-R, R, R + 2 \pi \mathrm{i}, -R + 2 \pi \mathrm{i}\),
  luego haremos tender \(R \rightarrow \infty\).
  El numerador del integrando es una función entera,
  el denominador tiene un único cero simple
  en \(z = \pi \mathrm{i}\)
  dentro de la curva:
  \begin{equation*}
    \res \left(
	   \frac{\mathrm{e}^{\alpha z}}{1 + \mathrm{e}^z},
	   \pi \mathrm{i}
	 \right)
      = \frac{\mathrm{e}^{\alpha \pi \mathrm{i}}}
	     {\mathrm{e}^{\pi \mathrm{i}}}
      = - \mathrm{e}^{\alpha \pi \mathrm{i}}
  \end{equation*}
  Veamos las integrales sobre los lados del rectángulo.
  Sea \(I_R\) la integral que nos interesa,
  a lo largo del eje real desde \(-R\) a \(R\);
  y similarmente \(I\) la integral de interés.
  La integral a lo largo del lado superior del rectángulo
  (recordar que estamos integrando de derecha a izquierda)
  es:
  \begin{equation*}
    - \mathrm{e}^{2 \pi \mathrm{i} \, \alpha} I_R
  \end{equation*}
  Finalmente,
  para el lado derecho
    \(V_R = \{ R + \mathrm{i} t \colon 0 \le t \le 2\pi \}\)
  resulta:
  \begin{equation*}
    \left\lvert
      \int_{V_R}
	\frac{\mathrm{e}^{\alpha z}}{1 + \mathrm{e}^z}
	\, \mathrm{d} z
    \right\rvert
      \le \int_0^{2 \pi}
	    \left\lvert
	      \frac{\mathrm{e}^{\alpha (R + \mathrm{i} t)}}
		   {1 + \mathrm{e}^{R + \mathrm{i} t}}
	      \, \mathrm{d} t
	    \right\rvert
      \le \int_0^{2 \pi}
	    \mathrm{e}^{R (\alpha - 1) t}
	      \cdot \lvert \mathrm{e}^{(\alpha - 1) \mathrm{i} t}
	    \rvert
	    \, \mathrm{d} t
      \le C \mathrm{e}^{R (\alpha - 1)}
  \end{equation*}
  Como \(\alpha < 1\),
  esto tiende a \(0\) al tender \(R\) a infinito.
  El lado izquierdo es casi lo mismo.
  Por el teorema de residuos:
  \begin{align*}
    I - \mathrm{e}^{2 \alpha \pi \mathrm{i}} I
      &= - 2 \pi \mathrm{i} \, \mathrm{e}^{\alpha \pi \mathrm{i}} \\
    I
      &= - 2 \pi \mathrm{i} \,
	   \frac{\mathrm{e}^{\alpha \pi \mathrm{i}}}
		{1 - \mathrm{e}^{2 \alpha \pi \mathrm{i}}}
       = \frac{2 \pi \mathrm{i}}
	      {\mathrm{e}^{\alpha \pi \mathrm{i}}
		 - \mathrm{e}^{- \alpha \pi \mathrm{i}}}
       = \frac{\pi}{\sin \alpha \pi}
  \end{align*}

\subsection{Principio del argumento}
\label{sec:argument-principle}

  Aplicado adecuadamente,
  el teorema de residuos de Cauchy
  (teorema~\ref{theo:Cauchy-residues})
  permite calcular el número de ceros y de polos
  de funciones meromorfas.
  \begin{lemma}
    \label{lem:zero-residue}
    Sea \(f\) holomorfa en un entorno de \(z_0\),
    donde tiene un cero de multiplicidad \(m\).
    Entonces la función \(f'(z) / f(z)\)
    es holomorfa en un entorno perforado de \(z_0\),
    con un polo simple en \(z_0\) con residuo \(m\).
  \end{lemma}
  \begin{lemma}
    \label{lem:pole-residue}
    Sea \(f\) holomorfa en un entorno de \(z_0\),
    donde tiene un polo de multiplicidad \(m\).
    Entonces la función \(f'(z) / f(z)\)
    es holomorfa en un entorno perforado de \(z_0\),
    con un polo simple en \(z_0\) con residuo \(-m\).
  \end{lemma}
  \begin{proof}[Demostración del lema~\ref{lem:zero-residue}]
    Podemos escribir \(f(z) = (z - z_0)^m g(z)\),
    con \(g\) holomorfa en un entorno de \(z_0\) y \(g(z_0) \ne 0\).
    Entonces:
    \begin{equation*}
      \frac{f'(z)}{f(z)}
	= \frac{m (z - z_0)^{m - 1} g(z) + (z - z_0)^m g'(z)}
	       {(z - z_0)^m g(z)}
	= \frac{m}{z - z_0} + \frac{g'(z)}{g(z)}
    \end{equation*}
    El segundo término es holomorfo en el entorno de \(z_0\)
    y tenemos lo prometido.
  \end{proof}
  \begin{proof}[Demostración del lema~\ref{lem:pole-residue}]
    Podemos escribir \(f(z) = (z - z_0)^{-m} g(z)\),
    con \(g\) holomorfa en un entorno de \(z_0\) y \(g(z_0) \ne 0\).
    Entonces:
    \begin{equation*}
      \frac{f'(z)}{f(z)}
	= \frac{-m (z - z_0)^{-m - 1} g(z) + (z - z_0)^{-m} g'(z)}
	       {(z - z_0)^{-m} g(z)}
	= \frac{-m}{z - z_0} + \frac{g'(z)}{g(z)}
    \end{equation*}
    El segundo término es holomorfo en el entorno de \(z_0\)
    y tenemos lo prometido.
  \end{proof}
  Uniendo los lemas~\ref{lem:zero-residue} y~\ref{lem:pole-residue}
  resulta:
  \begin{theorem}[Principio del argumento]
    \index{principio del argumento|textbfhy}
    \label{theo:argument-principle}
    Sea \(f\) meromorfa en la región \(D\),
    y \(\gamma \subset D\) una curva cerrada simple
    trazada en dirección positiva,
    y tal que no hayan ceros ni polos de \(f\) sobre \(\gamma\).
    Sea \(N\) el número de ceros de \(f\) al interior de \(\gamma\),
    contados con sus multiplicidades;
    y análogamente \(P\)
    el número de polos de \(f\) al interior de \(\gamma\),
    contados con sus multiplicidades.
    Entonces:
    \begin{equation}
      \label{eq:argument-principle}
      \frac{1}{2 \pi \mathrm{i}} \,
	\int_\gamma \frac{f'(z)}{f(z)} \, \mathrm{d} z
	= N - P
    \end{equation}
  \end{theorem}
  El nombre del teorema~\eqref{theo:argument-principle}
  viene de lo siguiente:
  \begin{equation*}
    \int_\gamma \frac{f'(z)}{f(z)} \, \mathrm{d} z
  \end{equation*}
  es la variación del logaritmo de \(f(z)\) al trazar \(\gamma\),
  como \(\gamma\) es cerrada
  esto es únicamente la variación del argumento de \(f\)
  en unidades de \(2 \pi \mathrm{i}\).

  Para aplicación concreta del principio del argumento
  al contar ceros
  el resultado siguiente permite obviar el principio mismo,
  o al menos aplicarlo a una función más simple.
  \begin{theorem}[Rouché]
    \index{Rouche, teorema de@Rouché, teorema de|textbfhy}
    \label{theo:Rouche}
    Sean \(f\) y \(g\) holomorfas en \(D\),
    y sea \(\gamma \subset D\) una curva conexa simple.
    Suponga además que \(\lvert f(z) \rvert > \lvert g(z) \rvert\)
    sobre \(\gamma\).
    Entonces el número de ceros de \(f\) y \(f + g\)
    al interior de \(\gamma\) es el mismo.
  \end{theorem}
  La demostración que daremos se debe a Chen~%
      \cite{chen08:_intro_complex_anal}.
  \begin{proof}
    Para \(\tau \in [0, 1]\) defina:
    \begin{equation*}
      N(\tau)
	= \int_\gamma \frac{f'(z) + \tau g'(z)}{f(z) + \tau g(z)}
	    \, \mathrm{d} z
    \end{equation*}
    Como en \(\gamma\)
    es \(\lvert f(z) \rvert > \lvert g(z) \rvert\),
    esto asegura:
    \begin{equation*}
      \lvert f(z) + \tau g(z) \rvert
	\ge \lvert f(z) \rvert - \tau \lvert g(z) \rvert
	\ge \lvert f(z) \rvert - \lvert g(z) \rvert
	> 0
    \end{equation*}
    de forma que \(f + \tau g\) no tiene ceros sobre \(\gamma\),
    con lo que \(N(\tau)\) es continua.
    Siendo entero el valor,
    la única posibilidad es que \(N(\tau)\) sea constante.
    Pero \(N(0)\) es el número de ceros de \(f\)
    al interior de \(\gamma\),
    y \(N(1)\) el número de ceros de \(f + g\).
  \end{proof}

\section{Aplicaciones discretas}
\label{sec:complex-discrete-applications}

  Para ir a nuestro tema,
  veremos algunas aplicaciones discretas del análisis complejo.

\subsection{Sumas infinitas}
\label{sec:complex-infinite-sums}

  Para mostrar la técnica general,
  veamos un par de ejemplos.

  Consideremos la función:
  \begin{equation*}
    f(z)
      = \frac{\pi \cot \pi z}{z^2}
  \end{equation*}
  Esta función tiene singularidades en \(z = k\)
  para todo \(k \in \mathbb{Z}\)
  (\(\sin \pi z\) tiene ceros simples allí).
  En \(z = 0\) es un polo de orden 3,
  los demás son polos simples.
  En el origen:
  \begin{equation*}
    \res(f, 0)
      = \frac{1}{2!} \,
	  \lim_{z \rightarrow 0}
	    \frac{\mathrm{d}^2}{\mathrm{d} z^2} z^3 f(z)
      = - \frac{\pi^2}{3}
  \end{equation*}
  Para los demás polos:
  \begin{equation*}
    \res(f, k)
      = \frac{\pi \cos k \pi}{k^2 \pi \cos k \pi + 2 k \sin k \pi}
      = \frac{1}{k^2}
  \end{equation*}
  Nuestra estrategia será hallar una curva
  que encierre todas las singularidades de interés de \(f\),
  y sobre la cual sea sencillo calcular la integral
  (o demostrar que dicha integral tiende a cero).
  Elegir adecuadamente la curva es un arte,
  no hay recetas simples.
  Con eso tenemos la suma.

  Una curva simple de manejar es \(\gamma_n\),
  el cuadrado
  con esquinas en \(\pm (n + 1/2) \pm \mathrm{i}(n + 1/2)\).
  Es importante mantenerse alejado de los polos,
  ya que en ellos la función tiende a infinito
  y acotar la integral sobre un camino que pase cerca de un polo
  será complicado.

  Otras alternativas populares
  son circunferencias centradas en el origen
  cuyo radio tiende a infinito,
  y también semicircunferencias
  (en el semiplano imaginario positivo o negativo)
  completadas con el eje \(x\).

  Consideremos la función \(\cot z\) para \(z = x + \mathrm{i} y\).
  Podemos expresar:
  \begin{equation}
    \label{eq:cot(z)}
    \cot z
      = \frac{\cos x \cosh y - \mathrm{i} \, \sin x \sinh y}
	     {\sin x \cosh y + \mathrm{i} \, \cos x \sinh y}
  \end{equation}
  En los lados verticales del cuadrado es \(\cos x = 0\),
  con lo que \(\sin x = 1\) y~\eqref{eq:cot(z)} queda:
  \begin{align}
    \cot z
      &= \frac{- \mathrm{i} \, \sinh y}{\cosh y}
       = - \mathrm{i} \, \tanh y \notag \\
    \lvert \cot z \rvert
      &= \lvert \tanh y \rvert
       \le 1 \label{eq:complex-cot-vertical-bound}
  \end{align}
  Obtenemos la cota:
  \begin{equation*}
    \left\lvert
	\int_\text{vertical} f(z) \, \mathrm{d} z
    \right\rvert
      \le \max( \lvert f(z) \rvert) \cdot (2 n + 1)
      \le \frac{1}{(n + 1/2)^2} \cdot (2 n + 1)
  \end{equation*}

  En los lados horizontales escribimos:
  \begin{equation*}
    \lvert \cot z \rvert^2
      =	 \frac{\cos^2 x \cosh^2 y + \sin^2 x \sinh^2 y}
	      {\sin^2 x \cosh^2 y + \cos^2 x \sinh^2 y}
  \end{equation*}
  Por~\eqref{eq:cot(z)} podemos expresar:
  \begin{align*}
    \lvert \cot z \rvert^2
      &= \frac{\cos^2 x \cosh^2 y + \sin^2 x \sinh^2 y}
	      {\sin^2 x \cosh^2 y + \cos^2 x \sinh^2 y} \\
      &= \frac{\cos^2 x \cosh^2 y + (1 - \cos^2 x) (\cosh^2 y - 1)}
	      {(1 - \cos^2 x) \cosh^2 y
		+ \cos^2 x (\cosh^2 y - 1)} \\
      &= 1 + \frac{2 \cos^2 x - 1}{\cosh^2 y - \cos^2 x}
  \end{align*}
  Como cuando \(y \rightarrow \pm \infty\)
  también \(\cosh y \rightarrow \infty\),
  para \(y\) suficientemente grande es:
  \begin{equation}
    \label{eq:complex-cot-horizontal-bound}
    \lvert \cot z \rvert
      \le 2
  \end{equation}
  Con este entendido resulta la cota:
  \begin{equation*}
    \left\lvert \int_\text{horizontal} f(z) \, \mathrm{d} z \right\rvert
      \le \max(\lvert f(z) \rvert) \cdot (2 n + 1)
      \le \frac{2}{(n + 1/2)^2} \cdot (2 n + 1)
  \end{equation*}
  Uniendo las anteriores
  tenemos la cota para la integral sobre \(\gamma_n\):
  \begin{equation*}
    \left\lvert \int_{\gamma_n} f(z) \, \mathrm{d} z \right\rvert
      \le 2 \cdot \frac{1}{(n  + 1/2)^2} \cdot (2 n + 1)
	   + 2 \cdot \frac{2}{(n + 1/2)^2} \cdot (2 n + 1)
      = \frac{24}{2 n + 1}
  \end{equation*}
  Esto tiende a cero cuando \(n \rightarrow \infty\).
  Por lo tanto:
  \begin{align*}
    \lim_{n \rightarrow \infty} \int_{\gamma_n} f(z) \, \mathrm{d} z
      &= 0 \\
    \sum_{k \in \mathbb{Z}} \res(f, k \pi)
      &= 0
       = - \frac{\pi^2}{3} + 2 \sum_{k \ge 1} \frac{1}{k^2}
  \end{align*}
  De acá:%
    \index{Basilea, problema de}
  \begin{equation}
    \label{eq:sum-reciprocal-squares}
    \sum_{k \ge 1} \frac{1}{k^2}
      = \frac{\pi^2}{6}
  \end{equation}
  Otra solución al problema de Basilea.
  El mismo método entrega valores
  para \(\zeta(2 n)\) para todo \(n\).

  Consideremos ahora la función:
  \begin{equation*}
    f(z)
      = \frac{\pi \csc \pi z}{z^2}
  \end{equation*}
  Nuevamente tenemos singularidades aisladas
  en \(z \in \mathbb{Z}\).
  En el origen es un polo de orden tres:
  \begin{equation*}
    \res(f, 0)
      = \lim_{z \rightarrow 0} \frac{1}{2!} \,
	  \frac{\mathrm{d}^2}{\mathrm{d} z^2} f(z)
      = \frac{\pi^2}{6}
  \end{equation*}
  Para los demás polos,
  que son todos simples:
  \begin{equation*}
    \res(f, k)
      = \frac{\pi}{\pi k^2 \cos k \pi + 2 k \sin k \pi}
      = \frac{(-1)^k}{k^2}
  \end{equation*}
  Veamos \(\csc z\) para \(z = x + \mathrm{i} y\)
  sobre el mismo cuadrado \(\gamma_n\).
  Primero:
  \begin{equation*}
    \csc z
      = \frac{1}{\sin x \cosh y + \mathrm{i} \, \cos x \sinh y}
  \end{equation*}
  En los lados verticales,
  donde \(\cos x = 0\) y \(\sin x = 1\):
  \begin{equation}
    \label{eq:complex-csc-vertical-bound}
    \lvert \csc z \rvert
      = \frac{1}{\cosh y}
      \le 1
  \end{equation}
  Para los horizontales interesa una cota superior para \(\cot z\),
  que es lo mismo que una cota inferior para \(\sin z\):
  \begin{align*}
    \lvert \sin z \rvert^2
      &= \lvert \sin^2 x \, \cosh^2 y
	    + \cos^2 x \, \sinh^2 y \rvert \\
      &= \lvert (1 - \cos^2 x) (1 + \sinh^2 y)
		   + \cos^2 x \, \sinh^2 y \rvert \\
      &= \lvert 1 - \cos^2 x + \sinh^2 y \rvert
  \end{align*}
  Para \(y \rightarrow \pm \infty\)
  tenemos \(\sinh^2 y \rightarrow \infty\),
  por lo que para \(y\) suficientemente grande
    \(\lvert \csc z \rvert \le 1\).
  Similar a antes:
  \begin{equation*}
    \left\lvert \int_{\gamma_n} f(z) \, \mathrm{d} z \right\rvert
      \le 2 \cdot \frac{1}{(n  + 1/2)^2} \cdot (2 n + 1)
	   + 2 \cdot \frac{1}{(n + 1/2)^2} \cdot (2 n + 1)
      = \frac{16}{2 n + 1}
  \end{equation*}
  Resulta:
  \begin{align*}
    \lim_{n \rightarrow \infty} \int_{\gamma_n} f(z) \, \mathrm{d} z
      &= 0 \\
    \sum_{k \in \mathbb{Z}} \res(f, k \pi)
      &= 0
       = - \frac{\pi^2}{6} + 2 \sum_{k \ge 1} \frac{(-1)^k}{k^2}
  \end{align*}
  De acá:
  \begin{equation}
    \label{eq:sum-alternating-reciprocal-squares}
    \sum_{k \ge 1} \frac{(-1)^k}{k^2}
      = \frac{\pi^2}{12}
  \end{equation}

  Si se revisa el desarrollo,
  es claro que para cualquier función meromorfa \(g(z)\)
  par en \(\mathbb{R}\) y tal que:
  \begin{equation*}
    \lim_{\lvert z \rvert \rightarrow \infty} z g(z) = 0
  \end{equation*}
  podemos aplicar las mismas técnicas para evaluar las sumas:
  \begin{equation*}
    \sum_{k \ge 1} g(k)
    \qquad
    \sum_{k \ge 1} (-1)^k g(k)
  \end{equation*}

  Podemos usar el mismo método para evaluar series.
  Por ejemplo,
  si nos interesa evaluar
  la siguiente serie para \(w \in \mathbb{C}\):
  \begin{equation*}
    \sum_{k \ge 1} \frac{w}{k^2 - w^2}
  \end{equation*}
  Si consideramos \(w\) como una constante,
  podemos aplicar la idea precedente con:
  \begin{equation*}
    f(z)
      = \frac{\pi \cot \pi z}{z^2 - w^2}
  \end{equation*}
  Esta función tiene polos simples en \(z \in \mathbb{Z}\)
  y en \(z = \pm w\).
  Igual que antes,
  en las últimas singularidades:
  \begin{equation*}
    \res(f, \pm w)
      = \lim_{z \rightarrow \pm w}
	  \frac{\pi \cot \pi z}{z \pm w}
      = \pm \frac{\pi \cot (\pm \pi w)}{2 w}
      = \frac{\pi \cot \pi w}{2 w}
  \end{equation*}
  En \(z = k \in \mathbb{Z}\):
  \begin{equation*}
    \res(f, k)
      = \lim_{z \rightarrow k}
	  \frac{\pi \cos \pi z}
	{(z^2 - w^2) \, \pi \cos \pi z + 2 z \sin \pi z}
      = \frac{1}{k^2 - w^2}
  \end{equation*}
  Igual que arriba,
  la integral sobre el cuadrado se anula:
  \begin{align*}
    \sum_{k \in \mathbb{Z}} \frac{1}{k^2 - w^2}
      + 2 \cdot \frac{\pi \cot (\pi w)}{2 w}
      &= 0 \\
    2 \sum_{k \ge 1} \frac{1}{k^2 - w^2}
      - \frac{1}{w^2}
      + \frac{\pi \cot (\pi w)}{w}
      &= 0
  \end{align*}
  \begin{align*}
    \sum_{k \ge 1} \frac{1}{k^2 - w^2}
      &= \frac{1}{2 w^2} - \frac{\pi \cot (\pi w)}{2 w} \\
    \sum_{k \ge 1} \frac{w}{k^2 - w^2}
      &= \frac{1}{2 w} - \frac{\pi}{2} \cot \pi w
  \end{align*}

\subsection{Números de Fibonacci}
\label{sec:complex-Fibonacci-numbers}

  Los números de Fibonacci quedan definidos por la recurrencia
  (ver sección~\ref{sec:Fibonacci}):%
     \index{Fibonacci, numeros de@Fibonacci, números de}
  \begin{equation}
    \label{eq:Fibonacci-recurrence-2}
    F_{n + 2}
      = F_{n + 1} + F_n
    \qquad
    F_0 = 0, F_1 = 1
  \end{equation}
  Sabemos que la función generatriz ordinaria
  es~\eqref{eq:Fibonacci-explicit}:
  \begin{equation}
    \label{eq:Fibonacci-explicit}
    F(z)
      = \sum_{k \ge 0} F_k z^k
      = \frac{z}{1 - z - z^2}
  \end{equation}
  Los ceros del denominador de~\eqref{eq:Fibonacci-explicit}
  son \((-1 \pm \sqrt{5}) / 2\),
  siendo \((-1 + \sqrt{5}) / 2\) el más cercano al origen.
  Esto determina el radio de convergencia
  de la serie.
  Recordamos las definiciones:
  \begin{equation*}
    \tau
      = \frac{1 + \sqrt{5}}{2}
    \qquad
    \phi
      = \frac{1 - \sqrt{5}}{2}
  \end{equation*}
  con lo que:
  \begin{equation*}
    1 - z - z^2
      = - (z + \tau) (z + \phi)
  \end{equation*}
  y también:
  \begin{equation}
    \label{eq:phi*barphi}
    \tau \phi
      = -1
  \end{equation}
  Nuestro operador de extracción de coeficientes es:
  \begin{equation*}
    \left[ z^n \right] F(z)
      = \res \left( \frac{F(z)}{z^{n + 1}}, 0 \right)
      = F_n
  \end{equation*}
  Las otras singularidades son polos simples:
  \begin{align*}
    \res \left( \frac{F(z)}{z^{n + 1}}, -\tau \right)
      &= - \lim_{z \rightarrow -\tau}
	     \frac{1}{z^n (z + \phi)}
      = - \frac{1}{(-\tau)^n (-\tau + \phi)}
      = \frac{\phi^n}{\tau - \phi} \\
    \res \left( \frac{F(z)}{z^{n + 1}}, -\phi \right)
      &= - \lim_{z \rightarrow -\phi}
	     \frac{1}{z^n (z + \tau)}
      = - \frac{1}{(-\phi)^n (-\phi + \tau)}
      = - \frac{\tau^n}{\tau - \phi}
  \end{align*}
  Acá usamos la ecuación~\eqref{eq:phi*barphi} para los recíprocos
  de \(\tau\) y \(\phi\).
  Integremos sobre \(C_R\),
  la circunferencia de radio \(R\) centrada en el origen,
  donde \(R > \tau\).
  Por la desigualdad triangular,%
    \index{desigualdad triangular}
  teorema~\ref{theo:desigualdad-triangular}:
  \begin{equation*}
    \lvert z \rvert^2 - \lvert z \rvert - 1
      \le \lvert z^2 + z - 1 \rvert
  \end{equation*}
  Para \(\lvert z \rvert\) suficientemente grande:
  \begin{equation*}
    \lvert z \rvert^2
       \left(
	 1 - \frac{\lvert z \rvert + 1}{\lvert z \rvert^2}
       \right)
       \ge \frac{1}{2} \lvert z \rvert^2
  \end{equation*}
  Usando esto en la integral:
  \begin{equation*}
    \left\lvert
      \int_{C_R} \frac{z}{z^{n + 1}(1 - z - z^2)} \, \mathrm{d} z
    \right\rvert
      \le \frac{2}{R^{n + 1} R^2}
	    \cdot 2 \pi R
      = \frac{4 \pi}{R^{n + 2}}
  \end{equation*}
  Esto tiende a cero al tender \(R\) a infinito:
  \begin{equation*}
    F_n
      = \frac{1}{\tau - \phi} \,
	  \left(
	    \tau^n - \phi^n
	  \right)
      = \frac{1}{\sqrt{5}} \,
	  \left(
	    \left( \frac{1 + \sqrt{5}}{2} \right)^n
	      - \left( \frac{1 - \sqrt{5}}{2} \right)^n
	  \right)
  \end{equation*}
  De nuevo la fórmula de Binet~\eqref{eq:Binet-Fibonacci}.%
    \index{Binet, formula de@Binet, fórmula de}

% gamma.tex
%
% Copyright (c) 2013-2014 Horst H. von Brand
% Derechos reservados. Vea COPYRIGHT para detalles

\section[La función \texorpdfstring{$\Gamma$}{gamma}]
	{\protect\boldmath
	    La función $\Gamma$
	 \protect\unboldmath}
\label{sec:gamma-function}
\index{\(\Gamma\)|textbfhy}

  Tendremos ocasión de usar la función \(\Gamma\)
  (gamma mayúscula),
  interesa deducir sus propiedades elementales.
  Para \(z\) complejo,
  se define:
  \begin{equation}
    \label{eq:Gamma-definition}
    \Gamma(z)
      = \int_0^\infty t^{z - 1} \mathrm{e}^{-t} \, \mathrm{d} t
  \end{equation}
  Para \(t > 0\) fijo el integrando de~\eqref{eq:Gamma-definition}
  es holomorfo en \(z\),
  con lo que esto define una función holomorfa en el semiplano
  \(\Re z > 1\).%
    \index{C (numeros complejos)@\(\mathbb{C}\) (números complejos)!funcion holomorfa@función holomorfa}
% Fixme: Completar
  Integrando por partes:
  \begin{equation*}
    \Gamma(z)
      = \int_0^\infty t^{z - 1} \mathrm{e}^{-t} \, \mathrm{d} t
      = \frac{1}{z}
	  \, \int_0^\infty \mathrm{e}^{-t} \, \mathrm{d} t^z
      = \left. \frac{1}{z} \, t^z \mathrm{e}^{-t} \right|_0^\infty
	  + \frac{1}{z}
	      \, \int_0^\infty t^z \mathrm{e}^{-t} \, \mathrm{d} t
      = \frac{1}{z} \, \Gamma(z + 1)
  \end{equation*}
  Tenemos así la \emph{fórmula de reducción}:
  \begin{equation}
    \label{eq:Gamma-reduction}
    \index{\(\Gamma\)!formula de reduccion@fórmula de reducción}
    \Gamma(z + 1)
      = z \, \Gamma(z)
  \end{equation}
  También tenemos \(\Gamma(1) = 1\),
  lo que para \(n \in \mathbb{N}\) por la fórmula de reducción da:
  \begin{equation}
    \label{eq:Gamma-factorial}
    \index{\(\Gamma\)!factorial}
    \Gamma(n)
      = (n - 1)!
  \end{equation}
  Incidentalmente,
  como para \(z \in - \mathbb{N}_0\)
  resulta \(1 / \Gamma(z) = 0\)
  es consistente nuestra convención~\eqref{eq:1/k!-convention}
  que \(1 / n! = 0\) si \(n\) es un entero negativo.

  Si \(\Re z > 1\),
  tenemos el valor de~\eqref{eq:Gamma-definition}.
  Fijemos entonces \(z\) con \(\Re z \le 1\),
  y sea \(n\) tal que \(\Re (z + n) > 0\).
  En un entorno de \(z + n\) la función \(\Gamma\) es holomorfa,
  y por~\eqref{eq:Gamma-reduction} tenemos:
  \begin{equation*}
    \Gamma(z + n)
      = z^{\underline{n + 1}} \, \Gamma(z)
  \end{equation*}
  Vale decir,
  si \(\Re z < 0\),
  tiene sentido definir con \(n = \lceil - \Re z \rceil\):%
    \index{potencia!factorial}
  \begin{equation}
    \label{eq:Gamma-reduction-n}
    \Gamma(z)
      = \frac{\Gamma(z + n)}{z^{\underline{n + 1}}}
  \end{equation}
  Es claro
  que esto nos mete en problemas solo si \(z \in -\mathbb{N}_0\).
  En el entorno de \(-n\),
  para \(\lvert w \rvert\) chico,
  la función queda representada por:
  \begin{equation*}
    \Gamma(- n + w)
      = \frac{\Gamma(1 + w)}{(-n + w)^{\underline{n + 1}}}
  \end{equation*}
  Como \(z^{\underline{n + 1}}\) tiene un cero simple en \(-n\),
  vemos que \(\Gamma(z)\)
  tiene polos simples en los enteros negativos.
  Es fácil calcular sus residuos:%
    \index{\(\Gamma\)!residuo}
  \begin{equation}
    \label{eq:Gamma-residues}
    \begin{split}
      \res(\Gamma, 0)
	&= \lim_{z \rightarrow 0} z \frac{\Gamma(z + 1)}{z}
	 = 1 \\
      \res(\Gamma, -n)
	&= \lim_{z \rightarrow -n} (z + n)
	     \frac{\Gamma(z + n + 1)}{z^{\underline{n + 1}}}
	 = \frac{1}{(-n)^{\underline{n}}}
	 = \frac{(-1)^n}{n!}
    \end{split}
  \end{equation}

  La función \(\Gamma\) satisface muchas identidades notables.
  Por ejemplo,
  tenemos:
  \begin{theorem}[Fórmula de reflexión de Euler]
    \index{Euler, formula de reflexion para \(\Gamma\)@Euler, fórmula de reflexión para \(\Gamma\)|see{\(\Gamma\)!fórmula de reflexión}}
    \index{\(\Gamma\)!formula de reflexion@fórmula de reflexión}
    Se cumple:
    \begin{equation}
      \label{eq:Gamma-reflection}
      \Gamma(z) \, \Gamma(1 - z)
	= \frac{\pi}{\sin \pi z}
      \end{equation}
  \end{theorem}
  Seguimos la demostración de Stein y~Shakarchi~%
    \cite{stein10:_compl_analy}.
  \begin{proof}
    Primeramente, \(\pi / \sin \pi z\)
    tiene polos en \(\mathbb{Z}\).
    El residuo en \(n \in \mathbb{Z}\) es:
    \begin{equation*}
      \res \left( \frac{\pi}{\sin \pi z}, n \right)
	= \frac{\pi}{\pi \cos \pi n}
	= (-1)^n
    \end{equation*}
    La función \(\Gamma(1 - z)\) tiene polos simples
    en \(z \in - \mathbb{N}\),
    allí \(\Gamma(z)\) es holomorfa.
    Para \(n \in \mathbb{N}\):
    \begin{equation*}
      \res(\Gamma(z) \, \Gamma(1 - z), - n)
	= \res(\Gamma(z), -n) \, \Gamma(n + 1)
	= \frac{(-1)^n}{n!} \, n!
	= (-1)^{-n}
    \end{equation*}
    De la misma forma,
    para \(n \in \mathbb{N}_0\):
    \begin{equation*}
      \res(\Gamma(z) \, \Gamma(1 - z), n)
	= (-1)^n
    \end{equation*}
    O sea,
    los polos y residuos respectivos de ambas funciones coinciden.

    Enseguida,
    \(\Gamma(z) \, \Gamma(1 - z)\)
    y \(\pi / \sin \pi z\) son ambas periódicas,
    con período 1:
    \begin{equation*}
      \Gamma(z + 1) \, \Gamma(1 - (z + 1))
	= z \Gamma(z) \cdot \frac{\Gamma(1 - z)}{z}
	= \Gamma(z) \, \Gamma(1 - z)
    \end{equation*}

    Finalmente,
    demostramos que ambas funciones coinciden en \(0 < s < 1\).
    Podemos escribir:
    \begin{equation*}
      \Gamma(1 - s)
	= \int_0^\infty u^{-s} \mathrm{e}^{-u} \, \mathrm{d} u
	= t \int_0^\infty \mathrm{e}^{-v t} (v t)^{-s}
	      \, \mathrm{d} v
    \end{equation*}
    Acá usamos el cambio de variables \(u = v t\).
    Luego:
    \begin{align*}
      \Gamma(s) \, \Gamma(1 - s)
	&= \int_0^\infty
	       \mathrm{e}^{-t} t^{s - 1} \, \Gamma(1 - s)
	     \, \mathrm{d} t \\
	&= \int_0^\infty
	     \mathrm{e}^{-t} t^{s - 1}
	       \left(
		 t \int_0^\infty \mathrm{e}^{- v t} (v t)^{-s}
		     \, \mathrm{d} v
	       \right) \, \mathrm{d} t \\
	&= \int_0^\infty
	     \int_0^\infty
	       \mathrm{e}^{-t (1 + v)} v^{-s} \,
		 \mathrm{d} v \, \mathrm{d} t \\
	&= \int_0^\infty \frac{v^{-s}}{1 + v} \, \mathrm{d} v
    \end{align*}
    Con el cambio de variables \(v = \mathrm{e}^t\)
    queda la integral que evaluamos
    en~\eqref{eq:integral-Gamma(z)Gamma(1-z)}:
    \begin{equation*}
      \int_{-\infty}^\infty
	\frac{\mathrm{e}^{- s t}}{1 + \mathrm{e}^t} \, \mathrm{d} t
	= \frac{\pi}{\sin s \pi}
    \end{equation*}

    Uniendo todas las piezas,
    ambas funciones deben ser iguales.
  \end{proof}
  De acá resulta directamente
  el valor para argumento no entero más usado:
  \begin{align}
    ( \Gamma( 1/2 ))^2
      &= \Gamma(1/2) \, \Gamma(1 - 1/2)
       = \frac{\pi}{\sin \pi / 2}
       = \pi \notag \\
    \Gamma(1/2)
      &= \sqrt{\pi} \label{eq:Gamma(1/2)}
  \end{align}

  Íntimamente relacionada es la función \(\mathrm{B}\)
  (beta mayúscula),%
    \index{\(\mathrm{B}\)|textbfhy}
  definida para \(\Re x, \Re y > 0\):
  \begin{equation}
    \label{eq:definition-Beta}
    \mathrm{B}(x, y)
      = \int_0^1 t^{x - 1} (1 - t)^{y - 1} \, \mathrm{d} t
  \end{equation}
  Es claro que es simétrica:
  \begin{equation}
    \label{eq:Beta-symmetry}
    \mathrm{B}(x, y)
      = \mathrm{B}(y, x)
  \end{equation}
  También:
  \begin{equation*}
    \Gamma(x) \, \Gamma(y)
      = \int_0^\infty \mathrm{e}^{-u} u^{x - 1} \, \mathrm{d} u
	  \int_0^\infty \mathrm{e}^{-v} v^{y - 1} \, \mathrm{d} v
      = \int_0^\infty \int_0^\infty
	  \mathrm{e}^{-u - v} u^{x - 1} v^{y - 1}
	     \, \mathrm{d} u \, \mathrm{d} v
  \end{equation*}
  El cambio de variables \(u = s t\) y \(v = s (1 - t)\) da:
  \begin{equation*}
    \Gamma(x) \, \Gamma(y)
      = \int_0^\infty \mathrm{e}^{-s} s^{x + y - 1} \, \mathrm{d} s
	  \int_0^1 t^{x - 1} (1 - t)^{y - 1} \, \mathrm{d} t
      = \Gamma(x + y) \mathrm{B}(x, y)
  \end{equation*}
  de donde resulta la identidad básica,
  que sirve para definir \(\mathrm{B}(x, y)\):
  \begin{equation}
    \label{eq:Gamma-Beta}
    \mathrm{B}(x, y)
      = \frac{\Gamma(x) \, \Gamma(y)}{\Gamma(x + y)}
  \end{equation}

  Por la fórmula de reducción~\eqref{eq:Gamma-reduction-n}%
    \index{\(\Gamma\)!formula de reduccion@fórmula de reducción}
  tenemos de~\eqref{eq:coeficiente-binomial}:%
    \index{coeficiente binomial}
  \begin{equation*}
    \binom{\alpha}{n}
      = \frac{\alpha^{\underline{n}}}{n!}
      = \frac{\Gamma(\alpha + 1)}{\Gamma(\alpha - n + 1) \, n!}
  \end{equation*}
  Con~\eqref{eq:Gamma-factorial}
  esto se parece a~\eqref{eq:coeficiente-binomial-factorial},
  que hace sentido adoptar como definición:
  \begin{equation}
    \label{eq:complex-binomial-coefficient}
    \binom{m + n}{m}
      = \frac{(m + n)!}{m! \, n!}
      = \frac{\Gamma(m + n + 1)}{\Gamma(m + 1) \, \Gamma(n + 1)}
  \end{equation}

%%% Local Variables:
%%% mode: latex
%%% TeX-master: "clases"
%%% End:


%%% Local Variables:
%%% mode: latex
%%% TeX-master: "clases"
%%% End:


% estimaciones-asintoticas.tex
%
% Copyright (c) 2013-2014 Horst H. von Brand
% Derechos reservados. Vea COPYRIGHT para detalles

\chapter{Estimaciones asintóticas}
\label{cha:estim-asint}
\index{asintotica@asintótica}

  Es común requerir una estimación ajustada de alguna cantidad,
  como una suma
  o el valor aproximado del coeficiente de una serie horrible.
  Hay una variedad de técnicas disponibles para esta tarea,
  desde simples a complejas.
  Interesan estimaciones asintóticas
  que entreguen resultados numéricos precisos y simples
  para el número de estructuras combinatorias,
  generalmente en el ámbito de rendimiento de algoritmos
  con el objetivo de comparar alternativas
  o estimar los recursos requeridos para resolver un problema particular.
  Describiremos sólo algunos de los métodos más importantes.
  Resúmenes detallados de las principales técnicas
  ofrecen Odlyzko~\cite{odlyzko95:_asympt_enum_method}
  y Flajolet y Sedgewick~\cite{flajolet09:_analy_combin}.

\section{Estimar sumas}
\label{sec:estimar-sumas}
\index{asintotica@asintótica!suma}

  Primero una técnica simple,
  que ya habíamos visto
  (capítulo~\ref{cha:euler-maclaurin}).%
    \index{Euler-Maclaurin, formula de@Euler-Maclaurin, fórmula de}
  \begin{theorem}
    \label{theo:sum-integral}
    Sea \(f(x)\) una función continua y monótona.
    Entonces:
    \begin{equation*}
      \sum_{0 \le k \le n} f(n)
	\approx \int_0^n f(x) d x
	    + \frac{f(0) + f(n)}{2}
    \end{equation*}
  \end{theorem}
  \begin{proof}
    Si \(f(x)\) es monótona creciente:
    \begin{alignat*}{2}
      f(\lfloor x \rfloor)
	&\le f(x)
	&&\le f(\lceil x \rceil) \\
     \int_0^n f(\lfloor x \rfloor) d x
	&\le \int_0^n f(x) d x
	&&\le \int_0^n f(\lceil x \rceil) d x \\
     \sum_{0 \le k \le n - 1} f(k)
	&\le \int_0^n f(x) d x
	&&\le \sum_{0 \le k \le n - 1} f(k + 1) d x \\
     \sum_{0 \le k \le n} f(k) - f(n)
	&\le \int_0^n f(x) d x
	&&\le \sum_{0 \le k \le n} f(k) - f(0)
    \end{alignat*}
    y directamente:
    \begin{equation*}
      \int_0^n f(x) d x + f(0)
	\le \sum_{0 \le k \le n} f(n)
	\le \int_0^n f(x) d x + f(n)
    \end{equation*}
    Si \(f(x)\) es monótona decreciente,
    se intercambian \(f(0)\) y \(f(n)\) en este último resultado.
  \end{proof}

%% Partly filched from aspnes14:_notes_discr_mathem
  Otras estimaciones simples resultan de acotar usando sumas conocidas.
  Si tenemos una suma finita de términos positivos,
  y la suma infinita correspondiente converge,
  tenemos una cota obvia.
  Suelen ser útiles como cotas
  la suma de secuencias geométricas
  (teorema~\ref{theo:suma-geometrica}):
  \begin{align}
    \sum_{k \ge 0} a^k
      &= \frac{1}{1 - a} \label{eq:suma-geometrica} \\
    \sum_{0 \le k \le n} a^k
      &= \frac{1 - a^{n + 1}}{1 - a} \label{eq:suma-geometrica-finita}
  \end{align}
  También es común la suma de secuencias aritméticas:%
     \index{suma!aritmetica@aritmética}
  \begin{equation}
    \label{eq:suma-aritmetica}
    \sum_{0 \le k \le n} k
      = \frac{n (n + 1)}{2}
  \end{equation}
  Ocasionalmente aparecen números harmónicos
  (sección~\ref{sec:numeros-harmonicos}):%
     \index{numeros harmonicos@números harmónicos}
  \begin{equation}
    \label{eq:3}
    \sum_{1 \le k \le n} \frac{1}{k}
      = H_n
      = \ln n + \gamma + O\left( \frac{1}{n} \right)
  \end{equation}

  Una técnica simple es acotar parte de la suma.
  Por ejemplo,
  sabemos cómo derivar una fórmula para la suma de los primeros cubos,
  pero podemos obtener cotas sencillas mediante las siguientes observaciones:
  \begin{align*}
    \sum_{1 \le k \le n} k^3
      &\le \sum_{1 \le k \le n} n^3 \\
      &=   n^4 \\
    \sum_{1 \le k \le n} k^3
      &\ge \sum_{n / 2 \le k \le n} k^3 \\
      &\ge \sum_{n / 2 \le k \le n} \left( \frac{n}{2} \right)^3 \\
      &=   \frac{n^4}{8}
  \end{align*}
  O sea:
  \begin{equation*}
    \sum_{1 \le k \le n} k^3
      = \Theta(n^4)
  \end{equation*}

\section{Estimar coeficientes}
\label{sec:estimar-coeficientes}
\index{asintotica@asintótica!coeficiente}

  Hay diferentes técnicas aplicables en este caso,
  de diferentes grados de complejidad.
  Discutiremos algunos de aplicabilidad general,
  el lector interesado
  podrá hallar referencias a técnicas adicionales.

\subsection{Cota trivial}
\label{sec:trivial-sum-bound}

  Hay una cota bastante trivial,
  que resulta sorprendentemente ajustada en muchos casos.
  Típicamente el error cometido es de un factor de \(n^{1/2}\).
  Si consideramos la secuencia \(\langle a_n \rangle_{n \ge 0}\),
  donde en aplicaciones combinatorias los \(a_n\) no son negativos:
  \begin{equation}
    \label{eq:example-gf}
    A(z)
      = \sum_{n \ge 0} a_n z^n
  \end{equation}
  Es claro que para \(u\) positivo:
  \begin{equation*}
    a_n u^n
      \le A(u)
  \end{equation*}
  de donde tenemos la cota:
  \begin{equation}
    \label{eq:trivial-bound}
    a_n
      \le \min_{u \ge 0} \left( \frac{A(u)}{u^n} \right)
  \end{equation}

\subsection{Singularidades dominantes}
\label{sec:dominating-singularities}
\index{asintotica@asintótica!coeficiente!singularidad}

  Sabemos
  (ver capítulo~\ref{cha:analisis-complejo},%
    \index{analisis complejo@análisis complejo}
   particularmente el teorema~\ref{theo:complex-f1+f2-singularity})
  que una función meromorfa cerca de una singularidad
  queda representada por la suma de una función entera
  y una función holomorfa
  en un disco perforado centrado en la singularidad.
  Esta segunda función
  (básicamente la parte principal
   de la serie de Laurent de la función)%
    \index{Laurent, serie de}
  domina la expansión en serie,
  o,
  lo que es lo mismo,
  pone la mayor parte de los coeficientes de la serie de potencias.
  Un ejemplo de esto
  ya lo vimos en la sección~\ref{sec:d&c:division-fija},
  donde vimos que el cero del denominador más cercano al origen
  da los términos dominantes.
  Considerando únicamente un polo simple \(z_0\)
  como singularidad más cercana al origen,
  estamos aproximando:
  \begin{equation*}
    A(z)
      = \sum_{n \ge 0} a_n z^n
  \end{equation*}
  mediante:
  \begin{equation*}
    A(z)
      \approx \res(A, z_0) \, \frac{1}{z - z_0}
      =	      - \res(A, z_0) \, \frac{1}{z_0 (1 - z / z_0)}
  \end{equation*}
  lo que se traduce en:
  \begin{equation}
    \label{eq:ae:dominating-single-pole-approximation}
    a_n
      \approx - \res(A, z_0) \, \frac{1}{z_0^{n + 1}}
  \end{equation}
  Incluir singularidades adicionales
  significa añadir términos similares para ellas.
  En detalle,
  tenemos:
  \begin{theorem}
    \label{theo:poles-approximation}
    Sea \(f\) meromorfa en la región \(D\) que contiene el origen.
    Sea \(R > 0\) el módulo de los polos de mínimo módulo,
    y sean \(z_0, \dotsc, z_s\) estos polos.
    Sea \(R' > R\) el módulo de los polos de siguiente módulo mayor,
    y sea \(\epsilon > 0\) dado.
    Entonces:
    \begin{equation}
      \label{eq:poles-approximation}
      \left[ z^n  \right] f(z)
	= \left[ z^n \right] \,
	    \left( \sum_{0 \le k \le s} \pp(f; z_k) \right)
	    + O \, \left( \left( \frac{1}{R'}
	    + \epsilon \right)^n \right)
    \end{equation}
  \end{theorem}
  \begin{proof}
    Basta demostrar que
    al restar las partes principales de \(f\) indicadas
    de \(f\) el resultado
    es una función que es holomorfa en el disco
    \(D_{R'}(0)\),
    la cota para el error
    es esencialmente el corolario~\ref{cor:ratio-test}.
    Por el teorema~\ref{theo:complex-f1+f2-singularity},
    la función \(f(z) - \pp(f; z_0)\) es holomorfa en \(z_0\),
    y debemos demostrar que su parte principal en \(z_1\)
    es la misma que la de \(f\):
    \begin{align*}
      \pp(f - \pp(f; z_0); z_1)
	&= \pp(f; z_1) - \pp(\pp(f; z_0); z_1) \\
	&= \pp(f; z_1)
    \end{align*}
    ya que \(\pp(f; z_0)\) es holomorfa en \(z_1\).
    Podemos ir aplicando esto polo a polo
    hasta demostrar
    que restando las partes principales indicadas de \(f\)
    obtenemos una función holomorfa
    en el disco de radio \(R'\) indicado.
  \end{proof}

  Para un ejemplo,
  vimos
  (sección~\ref{sec:desarreglos})
  que la función generatriz exponencial
  de los números de desarreglos está dada por:%
    \index{desarreglo!asintotica@asintótica}
  \begin{equation}
    \label{eq:ae:egf-derangements}
    \widehat{D}(z)
      = \frac{\mathrm{e}^{-z}}{1 - z}
  \end{equation}
  Esta función generatriz tiene un polo simple en \(z = 1\),
  con residuo \(- \mathrm{e}^{-1}\).
  Obtenemos directamente
  la aproximación~\eqref{eq:n-derangements-approx}.
  Sabemos también que la función:
  \begin{equation}
    \label{eq:ae:derangements-error}
    d(z)
      = \frac{\mathrm{e}^{-z}}{1 - z}
	  - \frac{\mathrm{e}^{-1}}{1 - z}
      = \frac{\mathrm{e}^{-z} - \mathrm{e}^{-1}}{1 - z}
  \end{equation}
  es entera
  (tiene una singularidad removible en \(z = 1\)),
  sus coeficientes son el error que comete la aproximación.
  Como el radio de convergencia
  para una función entera es \(R = \infty\),
  nuestra cota~\eqref{eq:poles-approximation} es:
  \begin{equation}
    \label{eq:ae:derangements-approximation}
    D_n
      = n! \mathrm{e}^{-1} + O(n! \epsilon^n)
  \end{equation}

  Otro caso es el número de resultados de competencias con empate
  (números de Bell ordenados).%
    \index{Bell, numeros de (ordenados)@Bell, números de (ordenados)!asintotica@asintótica}
  En la sección~\ref{sec:campeonatos-empate} llegamos a:
  \begin{equation}
    \label{eq:ae:ranking-gf}
    R(z)
      = \frac{1}{2 - \mathrm{e}^z}
  \end{equation}
  Esto tiene singularidades cuando \(\mathrm{e}^z = 2\),
  o sea en los puntos para \(k \in \mathbb{Z}\):
  \begin{equation}
    \label{eq:ae:ranking-gf-sigularities}
    \log 2
      = \ln 2 + 2 k \pi \mathrm{i}
  \end{equation}
  Domina la singularidad en \(\ln 2\),
  podemos refinar el resultado incluyendo singularidades adicionales
  en orden de cercanía de \(0\)
  (\(\ln 2\),
   \(\ln 2 \pm 2 \pi \mathrm{i}\),
   \(\ln 2 \pm 4 \pi \mathrm{i}\),
   \ldots).
  Si escribimos \(z = \log 2 + u\)
  y expandimos la exponencial en serie:
  \begin{equation*}
    R(z)
      = \frac{1}{2 - 2 \mathrm{e}^u}
      = \frac{1}{2} \, \frac{1}{1 - \mathrm{e}^u}
  \end{equation*}
  Las singularidades son todas polos simples,
  sus residuos son:%
    \index{C (numeros complejos)@\(\mathbb{C}\) (números complejos)!residuo}
  \begin{equation}
    \label{eq:R-residues}
    \res \left( \frac{1}{2 - \mathrm{e}^z},
		\ln 2 + 2 k \pi \mathrm{i} \right)
      = \lim_{z \rightarrow \ln 2 + 2 k \pi \mathrm{i}}
	  \frac{1}{- \mathrm{e}^z}
      = - \frac{1}{2}
  \end{equation}
  El polo más cercano al origen es \(\ln 2\),
  los demás polos están en \(\ln 2 \pm 2 \pi \mathrm{i}\),
  de módulo \(\sqrt{\ln 2 + 4 \pi^2} = 6,32\).
  En consecuencia,
  de~\eqref{eq:ae:dominating-single-pole-approximation}:
  \begin{equation}
    \label{eq:ae:R-approximation}
    R_n
      = \frac{n!}{2 (\ln 2)^{n + 1}} + O(n! \cdot 0,16^n)
  \end{equation}
  \begin{table}[ht]
    \centering
    \begin{tabular}{*{2}{>{\(}r<{\)}}D{.}{,}{4}}
      \multicolumn{1}{c}{\boldmath \(n\) \unboldmath} &
	\multicolumn{1}{c}{\boldmath \(R_n\) \unboldmath} &
	\multicolumn{1}{c}{\textbf{\eqref{eq:ae:R-approximation}}} \\
      \hline
      0 &   1 &	  0.7213 \\
      1 &   1 &	  1.0407 \\
      2 &   3 &	  3.0028 \\
      3 &  13 &	 12.9963 \\
      4 &  75 &	 74.9987 \\
      5 & 541 & 541.0015 \\
      \hline
    \end{tabular}
    \caption{Números de Bell ordenados}
    \label{tab:R-approximation}
  \end{table}
  El cuadro~\ref{tab:R-approximation}
  muestra los primeros valores exactos y aproximados,
  la concordancia es extremadamente buena.

  Para un ejemplo un poquito más complejo,
  tenemos los números de Bernoulli,%
    \index{Bernoulli, numeros de@Bernoulli, números de}
  con función generatriz exponencial
  (ver la sección~\ref{sec:propiedades-Bernoulli}):
  \begin{equation}
    \label{eq:ae:B(0,z)}
    B(z)
      = \frac{z}{\mathrm{e}^z - 1}
  \end{equation}
  Tanto \(z\) como \(\mathrm{e}^z - 1\) son enteras,
  \(B(z)\) solo puede tener polos.
  En los polos \(\mathrm{e}^z = 1\),
  o sea son polos \(z = 2 k \pi \mathrm{i}\)
  para todo \(k \in \mathbb{Z}\).
  En este caso tenemos dos polos a la misma distancia del origen,
  debemos considerar el aporte de ambos.
  Es fácil corroborar que los polos son simples,
  interesan:
  \begin{equation}
    \label{eq:residues-B(z)}
    \res(B(z), 2 k \pi \mathrm{i})
      = \lim_{z \rightarrow 2 k \pi \mathrm{i}}
	  \frac{z}{\mathrm{e}^z}
      = \mathrm{2 k \pi \mathrm{i}}
  \end{equation}
  Usando los polos \(\pm 2 \pi \mathrm{i}\)
  tenemos la aproximación:%
    \index{Bernoulli, numeros de@Bernoulli, números de!asintotica@asintótica}
  \begin{equation}
    \label{eq:ae:Bn-1}
    \frac{B_n}{n!}
      = - \frac{2 \pi \mathrm{i}}{(2 \pi \mathrm{i})^{n + 1}}
	    + \frac{2 \pi \mathrm{i}}{(- 2 \pi \mathrm{i})^{n + 1}}
	    + O \left( \left( \frac{1}{4 \pi} \right)^n \right)
  \end{equation}
  Vemos que para \(n\) impar los aportes se cancelan,
  esta técnica no entrega demasiada información en ese caso.
  Para \(n = 2 k\)
  volvemos a obtener~\eqref{eq:Bernoulli-approximation}:
  \begin{equation}
    \label{eq:ae:Bernoulli-approximation}
    B_{2 k}
      \sim \frac{- 2 (2 k)!}{(2 \pi \mathrm{i})^{2 k}}
      = (-1)^{k + 1} \, \frac{2 (2 k)!}{(4 \pi^2)^k}
  \end{equation}
  \begin{table}[ht]
    \centering
    \begin{tabular}{>{\(}r<{\)}>{\(}c<{\)}
		    D{.}{,}{4}D{.}{,}{4}}
      \multicolumn{1}{c}{\boldmath\(n\)\unboldmath} &
	\multicolumn{2}{c}{\boldmath\(B_n\)\unboldmath} &
	\multicolumn{1}{c}{\textbf{\eqref{eq:ae:Bernoulli-approximation}}} \\
      \hline
       0 & 1		      &	 1.0000 & -2.0000 \\
       2 & \phantom{-} 1 / 6  &	 1.1666 &  0.1013 \\
       4 & - 1 / 30	      & -0.0333 & -0.0308 \\
       6 & \phantom{-} 1 / 42 &	 0.0238 &  0.0234 \\
       8 & - 1 / 30	      & -0.0333 & -0.0332 \\
      10 & \phantom{-} 5 / 56 &	 0.0758 &  0.0757 \\
      12 & - 691 / 2730	      & -0.2531 & -0.2531 \\
      14 & \phantom{-} 7 / 6  &	 1.1666 &  1.1666 \\
      16 & - 3617 / 510	      & -7.0922 & -7.0920 \\
      \hline
    \end{tabular}
    \caption{Números de Bernoulli pares}
    \label{tab:Bernoulli-approximation}
  \end{table}
  Los valores exactos y aproximados de \(B_{2 k}\)
  se contrastan en el cuadro~\ref{tab:Bernoulli-approximation}.
  La aproximación no es particularmente buena en los índices bajos,
  se siente la influencia de los polos más lejanos.
  Igualmente la aproximación es muy buena.

  Si sumamos los aportes de todos los polos,
  resulta de nuevo la fórmula para \(\zeta(2 k)\):
  \begin{equation}
    \label{eq:ae:zeta(2k)}
    B_{2 k}
      = (-1)^{k + 1} \, \frac{2 (2 k)!}{(4 \pi^2)^k} \, \zeta(2 k)
  \end{equation}

  Una aplicación instructiva de las técnicas presentadas
  dan Odlyzko y Wilf~\cite{odlyzko88:_coins_fountain}
  al tratar el caso de fuentes no necesariamente de bloques
  (ver la sección~\ref{sec:FG-combinatoria}).

% asintotica-palabras-sin-patron.tex
%
% Copyright (c) 2013-2014 Horst H. von Brand
% Derechos reservados. Vea COPYRIGHT para detalles

\subsection{Número de palabras sin $k$ símbolos repetidos}
\label{sec:asymptotics-words-no-pattern}
\index{palabra!asintotica@asintótica}

  Vimos en la sección~\ref{sec:strings-excluding-pattern}
  que el número de palabras de largo~\(n\)
  sobre un alfabeto de~\(s\) símbolos
  que no contienen el patrón~\(p\)
  de largo~\(k\)
  tiene función generatriz ordinaria~\eqref{eq:Bp-gf}:
  \begin{equation*}
    \frac{c_p(z)}{(1 - s z) c_p(z) + z^k}
  \end{equation*}
  Acá \(c_p(z)\)
  es el polinomio de autocorrelación del patrón~\(p\).
  Obtener una estimación asintótica de este número
  sirve de ejemplo detallado
  de la aplicación de las herramientas discutidas.
  El caso general es tratado por Guibas y Odlyzko~%
    \cite{guibas81:_strin_overl_patterns}.

  Para simplificar,
  trataremos únicamente el caso en que el patrón
  es un símbolo repetido~\(k\) veces,
  en cuyo caso \(c_p(z) = 1 + z + \dotsb + z^{k - 1}\).
  En este caso la función generatriz se reduce a:
  \begin{equation}
    \label{eq:Bak-gf}
    \frac{1 - z^k}{1 - s z + (s - 1) z^{k + 1}}
  \end{equation}
  Nos interesa el cero más cercano al origen
  del denominador de~\eqref{eq:Bak-gf} para \(k\) dado.
  Sólo tiene sentido el caso \(k > 1\).
  Llamemos:
  \begin{equation}
    \label{eq:h-definition}
    h(z)
      = 1 - s z + (s - 1) z^{k + 1}
  \end{equation}
  Claramente \(h(1) = 0\)
  con \(h'(1) = k (s - 1) - 1 > 0\),
  y \(h(1/s) = (s - 1) s^{-k - 1} > 0\) es pequeño.
  Podemos acotar:
  \begin{equation*}
    h((s - 1)^{-1})
      = 1 - s (s - 1)^{-1} + (s - 1)^{-k}
      \le 1 - s + 1
      = 2 - s
  \end{equation*}
  Hay un cero entre \(s^{-1}\) y \((s - 1)^{-1}\).
  Debemos asegurarnos que sea único.
  Para ello aplicamos el teorema de Rouché
  (\ref{theo:Rouche}).

  Consideremos el caso \(s > 2\).
  Tomemos las funciones:
  \begin{align*}
    f(z)
      &= -s z \\
    g(z)
      &= 1 + (s - 1) z^{k + 1}
  \end{align*}
  Sobre la circunferencia \(\lvert z \rvert = r\)
  tenemos las cotas:
  \begin{align*}
    \lvert f(z) \rvert
      &= s r \\
    \lvert g(z) \rvert
      &= 1 + (s - 1) r^{k + 1}
  \end{align*}
  Nos interesa demostrar que:
  \begin{align*}
    s r
      &> 1 + (s - 1) r^{k + 1} \\
    0 &> 1 - s r + (s - 1) r^{k + 1}
       = h(r)
  \end{align*}
  Como interesan \(k \ge 2\) y \(s \ge 2\),
  tenemos:
  \begin{align*}
    h(0)
      &= 1 \\
    h(1)
      &= 0 \\
    h'(1)
      &= - s + (k + 1) (s - 1)
       = k (s - 1) - 1
       > 0
  \end{align*}
  En consecuencia,
  hay \(0 < r^* < 1\) tal que \(h(r) < 0\),
  o,
  lo que es lo mismo,
  \(\lvert f(z) \rvert > \lvert g(z) \rvert\)
  sobre la circunferencia \(\lvert z \rvert = r^*\)
  Por el teorema de Rouché,
  \(f(z)\) y \(f(z) + g(z) = h(z)\) tienen el mismo número de ceros
  al interior de la circunferencia,
  uno solo.
  Como \(h(z)\) es un polinomio de coeficientes reales,
  sus ceros son reales o vienen en pares complejos.
  El cero que nos interesa es simple y real.
  Si lo llamamos \(\rho\),
  por el teorema de Bender
  (\ref{theo:Bender}):
  \begin{equation}
    \label{eq:asymptotic-strings-no-ak}
    P_n
      \sim \frac{1 - \rho^k}{h'(\rho)} \cdot \rho^n
      = \frac{1 - \rho^k}{(s - 1) (k + 1) \rho^k - s} \cdot \rho^n
  \end{equation}

  Tenemos cotas para \(\rho\),
  para obtener una mejor aproximación del cero
  aplicamos una iteración del método de Newton%
    \index{Newton, metodo de@Newton, método de}
  (para detalles,
   ver textos de análisis numérico,
   como Acton~%
     \cite[capítulo~2]{acton90:_numerical_methods_work}
   o Ralston y Rabinowitz~%
     \cite[capítulo~8]{ralston12:_first_cours_numer_analy})
  partiendo con la aproximación \(1 / s\).
  Resulta que la expresión final
  es mucho más sencilla si partimos con:
  \begin{equation*}
    h_r(u)
      = u^{k + 1} h(1 / u)
      = u^{k + 1} - s u^k + s - 1
  \end{equation*}
  Los ceros de \(h_r(u)\) son recíprocos de los ceros de \(h(z)\).
  Tenemos:
  \begin{equation*}
    (\rho^*)^{-1}
      \approx s - \frac{h_r(s)}{h_r'(s)}
      = s - \frac{s - 1}{s^k}
  \end{equation*}
  Para el peor caso posible,
  \(s = k = 2\),
  esto da la aproximación \(\rho^* \approx 0,571\),
  cuando el valor correcto es \(\rho = 1 / \tau = 0,618\).

%%% Local Variables:
%%% mode: latex
%%% TeX-master: "clases"
%%% End:


\subsection{Singularidades algebraicas}
\label{sec:algebraic-singularities}

  Supongamos ahora
  que la singularidad \(z_0\) de \(f\) más cercana al origen
  es algebraica
  (un punto de ramificación),
  vale decir hay \(\alpha \in \mathbb{C}\)
  con \(\alpha \notin \mathbb{N}\)
  tal que la función \(g\) definida por lo siguiente
  es holomorfa en un entorno de \(z_0\):
  \begin{equation*}
    f(z)
      = (z_0 - z)^\alpha g(z)
  \end{equation*}
  El desarrollo de la sección anterior
  hace sospechar que en tal caso la serie:
  \begin{equation*}
    f(z)
      = (z_0 - z)^\alpha \sum_{k \ge 0} g_k (z_0 - z)^k
      = \sum_{k \ge 0} g_k (z_0 - z)^{k + \alpha}
  \end{equation*}
  es clave.

  Seguimos básicamente el desarrollo de Knuth y Wilf~%
    \cite{knuth89:_short_proof_Darboux}.
  Sin pérdida de generalidad,
  podemos suponer que \(z_0 = 1\)
  (basta considerar \(f(z z_0)\) en caso contrario),
  y que hay una única singularidad de interés.
  Primero un par de resultados auxiliares,
  de interés independiente.
  \begin{theorem}[Bender]
    \label{theo:Bender}
    \index{Bender, teorema de|textbfhy}
    Sean \(A(z) = \sum a_k z^k\) y \(B(z) = \sum b_k z^k\)
    series de potencias
    con radios de convergencia \(\alpha > \beta \ge 0\),
    respectivamente.
    Suponga que:
    \begin{equation*}
      \lim_{n \rightarrow \infty} \frac{b_{n - 1}}{b_n} = b
    \end{equation*}
    Si \(A(b) \ne 0\),
    y \(A(z) B(z) = \sum c_n z^n\),
    entonces \(c_n \sim A(b) b_n\).
  \end{theorem}
  La demostración sigue la de Bender~%
    \cite{bender74:_asymp_method_enumer}.
  \begin{proof}
    Basta demostrar que \(c_n / b_n \sim A(b)\).
    Sabemos que:
    \begin{equation*}
      c_n
	= \sum_{0 \le k \le n} a_k b_{n - k}
    \end{equation*}
    Con esto:
    \begin{align*}
      \left\lvert A(b) - \frac{c_n}{b_n} \right\rvert
	&=   \left\lvert
	       A(b)
		 - \sum_{0 \le k \le n} a_k \frac{b_{n - k}}{b_n}
	     \right\rvert \\
	&=   \left\lvert
	       \sum_{k > n} a_k b^k
		 - \sum_{0 \le k \le n}
		     a_k \left( b^k - \frac{b_{n - k}}{b_n} \right)
	      \right\rvert \\
	&\le \left\lvert
	       \sum_{k > n} a_k b^k
	      \right\rvert
		+ \left\lvert
		    \sum_{0 \le k \le n}
		      a_k \left( b^k - \frac{b_{n - k}}{b_n} \right)
		  \right\rvert
    \end{align*}
    El primer término es la cola de una serie convergente,
    tiende a cero al crecer \(n\).
    Para el segundo término,
    dividiendo la suma en \(n/2\):
    \begin{align*}
      \left\lvert
	\sum_{0 \le k \le n}
	  a_k \left( b^k - \frac{b_{n - k}}{b_n} \right)
      \right\rvert
	&\le \left\lvert
	       \sum_{0 \le k < n /2}
		 a_k \left( b^k - \frac{b_{n - k}}{b_n} \right)
	     \right\rvert
	       + \sum_{n/2 \le k \le n}
		   \lvert a_k b^k \rvert
		     \cdot \left\lvert
			     \frac{b_{n - k}}{b_n b^k}
			   \right\rvert \\
	&\le \left\lvert
	       \sum_{0 \le k < n /2}
		 a_k \left( b^k - \frac{b_{n - k}}{b_n} \right)
	     \right\rvert
	       + \max_{n/2 \le k \le n}
		   \left\lvert
		     \frac{b_{n - k}}{b_n b^k}
		   \right\rvert
		   \cdot \sum_{n/2 \le k \le n}
			   \lvert a_k b^k \rvert
    \end{align*}
    El primer término tiende a cero,
    ya que \(b_{n - k} / b_n \sim b^k\);
    la suma del segundo término
    está acotada por la cola de una serie convergente.
    Para el factor:
    \begin{equation*}
      \max_{n/2 \le k \le n}
	\left\lvert
	  \frac{b_{n - k}}{b_n b^k}
	\right\rvert
	  = \max_{n/2 \le k \le n}
	      \left\lvert
		\frac{b_{n - k} b^{n - k}}{b_n b^n}
	      \right\rvert
	  = \max_{0 \le k < n/2}
	      \left\lvert
		\frac{b_k b^k}{b_n b^n}
	      \right\rvert
    \end{equation*}
    Como la serie \(\sum b_k b^k\) converge,
    esto está acotado.
  \end{proof}
  Una aplicación simple del teorema de Bender
  es obtener una expansión asintótica para los números de Motzkin~%
    \cite{motzkin48:_relat_between_hyper_cross_ratios}%
    \index{Motzkin, numeros de@Motzkin, números de!asintotica@asintótica}
  (ver la discusión detallada de Donaghey y Shapiro~%
    \cite{donaghey77:_motzkin_numbers}).
  Dedujimos~\eqref{eq:ae-Motzkin-gf} en la sección~\ref{sec:numeros-motzkin}:
  \begin{equation}
    \label{eq:ae-Motzkin-gf}
    M(z)
      = \frac{1 - z - \sqrt{1 - 2 z - 3 z^2}}{2 z^2}
  \end{equation}
  Es claro que para \(n \ge 2\):
  \begin{align*}
    M_n
      &= \left[ z^n \right] \, \frac{1 - z - \sqrt{1 - 2 z - 3 z^2}}{2 z^2} \\
      &= - \frac{1}{2} \,
	     \left[ z^{n + 2} \right]
	       \, (1 + z)^{1/2} (1 - 3 z)^{1/2}
  \end{align*}
  Vemos que el radio de convergencia de esto último es \(1 / 3\),
  igual que el segundo factor;
  el primer factor tiene radio de convergencia \(1\).
  Por el teorema de Bender:
  \begin{align}
    M_n
      &\sim - \frac{1}{2} \,
		(1 + 1 / 3)^{1/2}
		   \left[ z^{n + 2} \right] (1 - 3 z)^{1/2}
	 \notag \\
      &= - \frac{\sqrt{3}}{3} \, \binom{1/2}{n + 2} \, (-3)^{n + 2}
	 \notag \\
      &= \frac{3 \sqrt{3}}{8 (n + 2)} \, \binom{2 n + 2}{n + 1}
	   \left( \frac{3}{4} \right)^n
	 \label{eq:ae:M-approximation}
  \end{align}
  \begin{table}[ht]
    \centering
    \begin{tabular}{*{2}{>{\(}r<{\)}}D{.}{,}{3}}
      \multicolumn{1}{c}{\boldmath \(n\) \unboldmath} &
	\multicolumn{1}{c}{\boldmath \(R_n\) \unboldmath} &
	\multicolumn{1}{c}{\textbf{\eqref{eq:ae:M-approximation}}} \\
      \hline
	0 &    1 & \text{\textemdash} \\
	1 &    1 & \text{\textemdash} \\
	2 &    2 &    1.827 \\
	3 &    4 &    3.836 \\
	4 &    9 &    8.631 \\
	5 &   21 &   20.346 \\
	6 &   51 &   49.593 \\
	7 &  127 &  123.981 \\
	8 &  323 &  316.153 \\
	9 &  835 &  819.123 \\
       10 & 2188 & 2150.198 \\
      \hline
    \end{tabular}
    \caption{Números de Motzkin}
    \label{tab:Motzkin-numbers}
  \end{table}
  El cuadro~\ref{tab:Motzkin-numbers} contrasta los valores exactos
  con la aproximación~\eqref{eq:ae:M-approximation}.
  No es particularmente buena,
  pero invertimos muy poco esfuerzo en ella.

  Un ejercicio simple
  de la función \(\Gamma\)	y la fórmula de Stirling
  da:
  \begin{lemma}
    \label{lem:asymptotics-binomial}
    \index{coeficiente binomial!asintotica@asintótica}
    Para \(\beta\) fijo cuando \(n \rightarrow \infty\):
    \begin{equation*}
      \left[ z^n \right] (1 - z)^\beta
	\begin{cases}
	  \sim n^{- \beta - 1} / \Gamma(- \beta)
	    & \text{si \(\beta \notin \mathbb{N}\)} \\
	  \rightarrow 0
	    & \text{si \(\beta \in \mathbb{N}\)} \\
	\end{cases}
    \end{equation*}
  \end{lemma}
  \begin{proof}
    Para el caso \(\beta \in \mathbb{N}\),
    por el teorema del binomio%
      \index{binomio, teorema del}
    sabemos que si \(n > \beta\) el coeficiente es cero.

    En caso que \(\beta \notin \mathbb{N}\):
    \begin{align*}
      \left[ z^n \right] (1 - z)^\beta
	= \binom{\beta}{n} (-1)^n
	= \frac{\beta^{\underline{n}}}{n!} \, (-1)^n
	= \frac{(-\beta)^{\overline{n}}}{n!}
	= \frac{\Gamma(n - \beta)}{\Gamma(-\beta) n!}
    \end{align*}
    El resultado
    sigue de la fórmula de Stirling~\eqref{eq:Stirling}:%
      \index{Stirling, formula de@Stirling, fórmula de}
    \begin{equation*}
      \Gamma(n + 1)
	= n!
	\sim \sqrt{2 \pi n} \, \left( \frac{n}{e} \right)^n
    \end{equation*}
    Tenemos:
    \begin{align*}
      \frac{\Gamma(n - \beta)}{\Gamma(-\beta) n!}
	&\sim \frac{1}{\Gamma(-\beta)}
		\cdot \frac{(n - \beta - 1)^{n - \beta - 3/2}}
			   {\mathrm{e}^{n - \beta - 2}}
		\cdot \frac{\mathrm{e}^n}{n^{n + 1/2}} \\
	&=    \frac{1}{\Gamma(-\beta)}
		\cdot \frac{n^{n - \beta - 1/2}
			      \cdot (1 - (\beta + 2) / n)^n
			      \cdot (1 - (\beta + 2) / n)^{-\beta - 3/2}
			      \cdot \mathrm{e}^{- \beta - 2}}
			   {n^{n + 1/2}} \\
	&\sim \frac{n^{- \beta - 1}}{\Gamma(-\beta)} \\
    \end{align*}
    Acá usamos el límite clásico:
    \begin{equation*}
      \lim_{n \rightarrow \infty}
	\left( 1 + \frac{\alpha}{n} \right)^n
	= \mathrm{e}^\alpha
     \qedhere
    \end{equation*}
  \end{proof}
  Con estos:
  \begin{lemma}
    \label{lem:bound-algebraic}
    Sea \(u(z) = (1 - z)^\gamma v(z)\),
    donde \(v(z)\) es holomorfa
    en algún disco \(\lvert z \rvert < 1 + \eta\)
    (acá \(\eta > 0\)).
    Entonces:
    \begin{equation*}
      \left[ z^n \right] u(z)
	= O(n^{-\gamma - 1})
    \end{equation*}
  \end{lemma}
  \begin{proof}
    Aplicando el teorema de Bender,%
      \index{Bender, teorema de}
    teorema~\ref{theo:Bender},
    queda:
    \begin{equation*}
      \left[ z^n \right] (1 - z)^\gamma v(z)
	\sim v(1) \left[ z^n \right] (1 - z)^\gamma
    \end{equation*}
    El lema~\ref{lem:asymptotics-binomial}
    entrega el resultado prometido.
  \end{proof}
  Estamos en condiciones de demostrar:
  \begin{theorem}[Lema de Darboux]
    \index{Darboux, lema de}
    \label{theo:Darboux-lemma}
    Sea \(f\) holomorfa en un disco \(\lvert z \rvert < 1 + \eta\),
    donde \(\eta > 0\).
    Suponga que en un entorno de \(z = 1\) tiene una expansión
    \begin{equation*}
      f(z)
	= \sum_{k \ge 0} f_k (1 - z)^k
    \end{equation*}
    Entonces para todo \(\beta \in \mathbb{C}\)
    y todo \(m \in \mathbb{N}_0\):
    \begin{align*}
      \left[ z^n \right] (1 - z)^\beta f(z)
	&= \left[ z^n \right] \sum_{k \ge 0} f_k (1 - z)^{\beta + k}
	     + O(n^{-m - \beta - 2}) \\
	&= \sum_{0 \le k \le m} f_k \binom{n -\beta - k - 1}{n}
	     + O(n^{-m - \beta - 2})
    \end{align*}
  \end{theorem}
  \begin{proof}
    Tenemos:
    \begin{align*}
      (1 - z)^\beta f(z)
	- \sum_{0 \le k \le m} f_k (1 - z)^{\beta + k}
	&= \sum_{k > m} f_k (1 - z)^{\beta + k} \\
	&= (1 - z)^{\beta + m + 1} \widetilde{f}(z)
    \end{align*}
    Las regiones de holomorfismo
    de \(f\) y \(\widetilde{f}\) son las mismas.
    El resultado sigue del lema~\ref{lem:bound-algebraic}.
  \end{proof}
  Obtengamos una expansión más precisa de los números de Motzkin.%
    \index{Motzkin, numeros de@Motzkin, números de!asintotica@asintótica}
  Tenemos para \(n \ge 2\),
  vía el cambio de variables \(u = 3 z\):
  \begin{align*}
    M_n
      &= - \frac{1}{2} \,
	     \left[ z^{n + 2} \right]
	       \, (1 + z)^{1/2} (1 - 3 z)^{1/2} \\
      &= - \frac{3^{n + 2}}{2} \,
	     \left[ u^{n + 2} \right] \,
	       \left( \frac{4}{3} \right)^{1/2} \,
		 \left( 1 - \frac{1 - u}{4} \right)^{1/2}
	       \cdot (1 - u)^{1/2} \\
      &= - 3 \sqrt{3} \cdot 3^n \cdot
	     \sum_{k \ge 0} \binom{1/2}{k} (-4)^{-k}
	       \left[ u^{n + 2} \right] \, (1 - u)^{1/2}
  \end{align*}
  Si expandimos para \(m = 0\)
  y no aproximamos los coeficientes binomiales
  resulta nuevamente~\eqref{eq:ae:M-approximation}.
  Extendamos a \(m = 1\),
  aproximando los coeficientes binomiales
  mediante el lema~\ref{lem:asymptotics-binomial}.%
    \index{coeficiente binomial!asintotica@asintótica}
  Necesitamos:
  \begin{align*}
    \binom{1/2}{n + 2}
      &\sim \frac{(n + 2)^{-3/2}}{\Gamma(-1/2)}
       =    - \frac{(n + 2)^{-3/2}}{2 \sqrt{\pi}} \\
    \binom{3/2}{n + 2}
      &\sim \frac{(n + 2)^{-5/2}}{\Gamma(-3/2)}
       =    \frac{3 (n + 2)^{-5/2}}{4 \sqrt{\pi}}
  \end{align*}
  porque:
  \begin{align*}
    \left( - \frac{1}{2} \right)
	\cdot \Gamma \left( - \frac{1}{2} \right)
      &= \Gamma \left( \frac{1}{2} \right)
      = \sqrt{\pi} \\
    \left( - \frac{3}{2} \right)
       \cdot \Gamma \left( - \frac{3}{2} \right)
      &= \Gamma \left( - \frac{1}{2} \right)
  \end{align*}
  y tenemos la estimación:
  \begin{equation}
    \label{eq:ae:M-approximation-2}
    M_n
      = \frac{\sqrt{3} \cdot 3^n}{16 \sqrt{\pi}} \cdot
	  \left(
	    8 (n + 2)^{-3/2} + 3 (n + 2)^{-5/2}
	  \right)
  \end{equation}
  \begin{table}[ht]
    \centering
    \begin{tabular}{rrD{.}{,}{3}}
      \multicolumn{1}{c}{\boldmath \(n\) \unboldmath} &
	\multicolumn{1}{c}{\boldmath \(R_n\) \unboldmath} &
	\multicolumn{1}{c}{\textbf{\eqref{eq:ae:M-approximation-2}}} \\
      \hline
	0 &    1 & \textemdash \\
	1 &    1 & \textemdash \\
	2 &    2 &    1.804 \\
	3 &    4 &    3.805 \\
	4 &    9 &    8.583 \\
	5 &   21 &   20.263 \\
	6 &   51 &   49.438 \\
	7 &  127 &  123.678 \\
	8 &  323 &  315.526 \\
	9 &  835 &  817.783 \\
       10 & 2188 & 2147.245 \\
      \hline
    \end{tabular}
    \caption{Números de Motzkin nuevamente}
    \label{tab:Motzkin-numbers-2}
  \end{table}
  La aproximación~\eqref{eq:ae:M-approximation-2}
  no es tan buena como~\eqref{eq:ae:M-approximation},
  pero hay que considerar
  que usamos una aproximación bastante cruda
  para los coeficientes binomiales
  a cambio de reemplazarlos por potencias.
  Igual la fórmula diseñada para índices muy grandes
  da excelentes resultados ya para \(n = 10\).

\subsection{Singularidades algebraico-logarítmicas}
\label{sec:singularidades-logaritmicas}
\index{Odlyzko-Flajolet, metodo de@Odlyzko-Flajolet, método de}

  Una técnica que resuelve muchos de los casos de interés en combinatoria
  es la que presentan Flajolet y Odlyzko\cite{flajolet90:_singular_anal}
  (ver también el resumen más accesible de Flajolet y Sedgewick~%
    \cite[sección~VI.2]{flajolet09:_analy_combin}).
  Las demostraciones son bastante engorrosas,
  nos remitiremos a citar los resultados.
  \begin{theorem}
    \label{theo:(1-z)^alpha}
    Sea \(\alpha \in \mathbb{C} \smallsetminus \mathbb{Z}_{\le 0}\).
    Entonces:
    \begin{equation}
      \label{eq:(1-z)^alpha-asy}
      [z^n] (1 - z)^{- \alpha}
	 \sim \frac{n^{\alpha - 1}}{\Gamma(\alpha)}
    \end{equation}
  \end{theorem}
  Si \(\alpha\) es un entero negativo,
  \((1 - z)^{- \alpha}\) es un polinomio,
  y los coeficientes del caso eventualmente se anulan.
  La exposición de Flajolet y Sedgewick~%
    \cite[teorema~VI.1]{flajolet09:_analy_combin}
  da la expansión asintótica.
  \begin{theorem}
    \label{theo:(1-z)^alpha*log(1-z)^beta}
    Sea \(\alpha \in \mathbb{C} \smallsetminus \mathbb{Z}_{\le 0}\).
    Entonces:
    \begin{equation}
      \label{eq:(1-z)^alpha*log(1-z)^beta-asy}
      [z^n] (1 - z)^{- \alpha}
	       \left(\frac{1}{z} \ln \frac{1}{1 - z}\right)^\beta
	 \sim \frac{n^{\alpha - 1}}{\Gamma(\alpha)}
		\ln^\beta n
    \end{equation}
  \end{theorem}
  Se introduce un factor \(1/z\) en frente del logaritmo
  para obtener una serie en \(z\),
  no altera la expansión cerca de \(z = 1\).
  Nuevamente Flajolet y Sedgewick~%
    \cite[teorema~VI.2]{flajolet09:_analy_combin}
  dan la expansión asintótica completa
  para el caso en que \(\beta\) no es un natural,
  citan a Jungen~\cite{jungen31:_sur_taylor} para completar ese caso.

\subsection{El método de Hayman}
\label{sec:Hayman-method}
\index{Hayman, metodo de@Hayman, método de}

  Las técnicas precedentes dan excelentes resultados
  cuando las funciones generatrices de interés
  tienen singularidades cerca del origen.
  Son totalmente inútiles si la función generatriz es entera.
  Nos interesaría,
  por ejemplo,
  obtener una expansión asintótica de \(n!\)
  vía considerar la serie para la exponencial.
  Para ello podemos razonar como sigue:
  De la fórmula generalizada de Cauchy,%
    \index{Cauchy, formula de (generalizada)@Cauchy, fórmula de (generalizada)}
  teorema~\ref{theo:Cauchy-formula-f-(n)},
  tenemos que para toda curva simple cerrada que incluya el origen:
  \begin{equation*}
    \frac{1}{n!}
      = \frac{1}{2 \pi \mathrm{i}} \,
	  \int_\gamma \frac{\mathrm{e}^z}{z^{n + 1}} \, \mathrm{d} z
  \end{equation*}
  Si usamos la circunferencia de radio \(R\) centrada en el origen,
  tomando valores absolutos:
  \begin{align*}
    \frac{1}{n!}
      &\le \frac{1}{2 \pi} \,
	     \max_{\lvert z \rvert = R}
	       \left\{
		 \frac{\lvert \mathrm{e}^z \rvert}
		      {\lvert z \rvert^{n + 1}}
	       \right\}
	     \cdot 2 \pi R \\
      &=   \frac{\mathrm{e}^R}{R^n}
  \end{align*}
  Pero el valor de \(R\) es arbitrario,
  podemos elegir aquel que minimice esta expresión,
  que resulta ser \(R = n\),
  dando la cota:
  \begin{equation*}
    \frac{1}{n!}
      \le \left( \frac{\mathrm{e}}{n} \right)^n
  \end{equation*}
  Nada mal,
  si se compara con la fórmula de Stirling~\eqref{eq:Stirling}.%
    \index{Stirling, formula de@Stirling, fórmula de}

  Si queremos mayor precisión,
  debemos tratar la integral en forma más cuidadosa.
  Hayman~\cite{hayman56:_generalization_Stirling}
  desarrolló maquinaria poderosa para esta situación.
  Seguimos la exposición de Wilf~\cite{wilf06:_gfology}.
  En una circunferencia alrededor del origen
  el módulo de una función con coeficientes reales no negativos
  tiene un máximo marcado en el eje real positivo,
  y es precisamente en ese caso que el método es más efectivo.
  Y este es el caso de las funciones generatrices
  de la combinatoria.
  El método en realidad es aplicable siempre,
  pero las técnicas basadas en singularidades
  son más sencillas de aplicar.

  Sea \(f(z)\) holomorfa en el disco \(\lvert z \rvert < R\),
  donde \(0 < R \le \infty\),
  y suponga que \(f(z)\) es \emph{admisible} para el método.
  Las condiciones de admisibilidad son bastante complicadas,
  veremos algunas condiciones suficientes más adelante.
  En la práctica,
  que \(f(z)\) sea admisible
  significa simplemente que la técnica funciona.

  Defina:
  \begin{equation}
    \label{eq:Hayman-M-definition}
    M(r)
      = \max_{\lvert z \rvert = r}
	  \left\{ \lvert f(z) \rvert \right\}
  \end{equation}
  Una consecuencia de las condiciones de admisibilidad es que
  para \(r\) suficientemente grande:
  \begin{equation}
    \label{eq:Hayman-M-value}
    M(r)
      = f(r)
  \end{equation}
  Esto porque,
  como notamos arriba,
  el método apunta a funciones que toman el valor máximo
  en la dirección real positiva.
  Defina funciones auxiliares:
  \begin{align}
    a(r)
      &= r \frac{f'(r)}{f(r)}
	 \label{eq:Hayman-a-definition} \\
    b(r)
      &= r a'(r)
       = r \, \frac{f'(r)}{f(r)}
	   + r^2 \, \frac{f''(r)}{f(r)}
	   - r^2 \, \left( \frac{f'(r)}{f(r)} \right)^2
	 \label{eq:Hayman-b-definition}
  \end{align}
  El resultado central es:
  \begin{theorem}[Hayman]
    \index{Hayman, teorema de}
    \label{theo:Hayman}
    Sea \(f(z) = \sum f_n z^n\) una función admisible.
    Sea \(r_n\) el cero positivo de \(a(r_n) = n\)
    para cada \(n \in \mathbb{N}\),
    donde \(a(r)\) es la función~\eqref{eq:Hayman-a-definition}.
    Entonces para \(n \rightarrow \infty\):
    \begin{equation}
      \label{eq:Hayman-asymptotic}
      f_n
	\sim \frac{f(r_n)}{r_n^n \sqrt{2 \pi b(r_n)}}
    \end{equation}
  \end{theorem}
  La receta misma es de aplicación directa,
  lo complicado es determinar si \(f(z)\) es admisible.

  Continuemos con nuestro ejemplo \(\mathrm{e}^z\),
  aceptando por ahora que es admisible.
  Resultan:
  \begin{equation*}
    a(r)
      = r
    \qquad
    b(r)
      = r \mathrm{e}^r
  \end{equation*}
  con lo que \(r_n = n\).
  La estimación de Hayman~\eqref{eq:Hayman-asymptotic} es:
  \begin{equation*}
    \frac{1}{n!}
      \sim \frac{\mathrm{e}^n}{n^n \sqrt{2 \pi n}}
  \end{equation*}
  La fórmula de Stirling.%
    \index{Stirling, formula de@Stirling, fórmula de}

  Veamos las condiciones de admisibilidad.
  En lo que sigue,
  las funciones \(a\) y \(b\)
  son las definidas
  por las ecuaciones~\eqref{eq:Hayman-a-definition}
  y~\eqref{eq:Hayman-b-definition},
  respectivamente.
  Sea \(f(z) = \sum_{n \ge 0} f_n z^n\)
  holomorfa en \(\lvert z \rvert < R\),
  donde \(0 < R \le \infty\).
  Suponga que:
  \begin{enumerate}[label=(\alph*)]
  \item
    Existe \(R_0 < R\) tal que para \(R_0 < r < R\) es \(f(r) > 0\)
  \item
    Hay una función \(\delta(r)\),
    definida para \(R_0 < r < R\),
    tal que \(0 < \delta(r) < \pi\) en ese rango,
    y tal que cuando \(r \rightarrow R\),
    uniformemente para \(\lvert \theta \rvert \le \delta(r)\),
    tenemos:
    \begin{equation*}
      f(r \mathrm{e}^{\mathrm{i} \theta})
	\sim f(r) \mathrm{e}^{\mathrm{i} \theta a(r)
				- \frac{1}{2} \, \theta^2 b(r)}
    \end{equation*}
  \item
    Uniformemente
    para \(\delta(r) \le \lvert \theta \rvert \le \pi\)
    tenemos cuando \(r \rightarrow R\):
    \begin{equation*}
      f(r \mathrm{e}^{\mathrm{i} \theta})
	= \frac{o(f(r))}{\sqrt{b(r)}}
    \end{equation*}
  \end{enumerate}
  Si esto se cumple,
  \(f\) es admisible y el teorema~\ref{theo:Hayman}
  da una estimación asintótica de los coeficientes.

  Como puede verse,
  las condiciones son complejas de verificar.
  Hay teoremas que dan condiciones suficientes,
  mucho más sencillas de manejar.
  Para nuestra fortuna,
  corresponden a operaciones comunes con funciones generatrices.%
    \index{Hayman, metodo de@Hayman, método de!funciones admisibles}
  \begin{enumerate}[label=(\Alph*), ref=(\Alph*)]
  \item
    \label{enum:Hayman-admisible-exp(f)}
    Si \(f(z)\) es admisible,
    lo es \(\mathrm{e}^{f(z)}\).
  \item
    \label{enum:Hayman-admisible-fg}
    Si \(f(z)\) y \(g(z)\)
    son admisibles para \(\lvert z \rvert < R\),
    lo es \(f(z) g(z)\).
  \item
    \label{enum:Hayman-admisible-fp}
    Sea \(f(z)\) admisible en \(\lvert z \rvert < R\).
    Sea \(p(z)\) un polinomio con coeficientes reales
    tal que \(p(R) > 0\) si \(R \ne \infty\),
    o tal que el coeficiente de máximo grado
    es positivo si \(R = \infty\).
    Entonces \(f(z) p(z)\) es admisible en \(\lvert z \rvert < R\).
  \item
    \label{enum:Hayman-admisible-p(f)}
    Sea \(p(z)\) un polinomio de coeficiente reales,
    y sea \(f(z)\) admisible en \(\lvert z \rvert < R\).
    Si \(f(z) + p(z)\) es admisible,
    y el coeficiente de máximo grado de \(p\) es positivo,
    entonces \(p(f(z))\) es admisible.
  \item
    \label{enum:Hayman-admisible-exp(p)}
    Sea \(p(z)\) un polinomio no constante con coeficientes reales,
    y sea \(f(z) = \mathrm{e}^{p(z)}\).
    Si \(\left[ z^n \right] \mathrm{e}^{p(z)} > 0\)
    para todo \(n\) suficientemente grande,
    entonces \(f(z)\) es admisible en \(\mathbb{C}\).
  \end{enumerate}
  Como un ejemplo,
  tomemos las involuciones,
  con función generatriz exponencial~\eqref{eq:involution-egf}:
  \begin{equation*}
    \mathrm{e}^{z + z^2 / 2}
  \end{equation*}
  Es claro
  que se cumplen las condiciones~\ref{enum:Hayman-admisible-exp(p)}.
  Obtenemos:
  \begin{equation*}
    a(r)
      = r + r^2
    \qquad
    b(r)
      = r + 2 r^2
  \end{equation*}
  Tenemos una excelente aproximación para \(r_n\):
  \begin{align*}
    r_n
      &= \sqrt{n + 1/4} - 1/2 \\
      &= \sqrt{n}
	   \left( 1 + \frac{1}{4 n} \right)^n - \frac{1}{2} \\
      &= \sqrt{n}
	   \left(
	     1 + \frac{1}{8 n} - \frac{1}{128 n^2} + \dotsb
	   \right)
	   - \frac{1}{2} \\
      &= \sqrt{n}
	   - \frac{1}{2}
	   + \frac{1}{8 n^{1/2}}
	   - \frac{1}{128 n^{3/2}}
	   + \dotsb
  \end{align*}
  Una revisión de la fórmula~\eqref{eq:Hayman-asymptotic} nos dice
  que requerimos estimaciones
  estilo \(\sim\) cuando \(n \rightarrow \infty\)
  de las cantidades \(f(r_n)\), \(b(r_n)\) y \(r_n^n\).
  Por turno:
  \begin{equation*}
    f(r_n)
      = \mathrm{e}^{r_n + \frac{1}{2} r_n}
      = \mathrm{e}^{\frac{1}{2} (r_n + n)}
      = \mathrm{e}^{n/2} \mathrm{e}^{r_n / 2}
  \end{equation*}
  Pero:
  \begin{align*}
    \mathrm{e}^{r_n / 2}
      = \exp \left(
		\frac{\sqrt{n}}{2} - \frac{1}{4} + O(n^{-1/2}
	      \right)
      \sim \mathrm{e}^{\frac{1}{2} \sqrt{n} - \frac{1}{4}}
  \end{align*}
  En la lista sigue \(b(r_n)\):
  \begin{equation*}
    b(r_n)
      = r_n + 2 r_n^2
      \sim 2 (r_n^2 + r_n)
      = 2 n
  \end{equation*}
  El último es el más complicado:
  \begin{align*}
    r_n^n
      &= \left(
	   \sqrt{n} - \frac{1}{2} + \frac{1}{8 \sqrt{n}} - \dotsb
	 \right)^n \\
      &= n^{n/2}
	   \left(
	     1 - \frac{1}{2 n^{1/2}} + \frac{1}{8 n} - \dotsb
	   \right)^n
  \end{align*}
  Esta situación debe tratarse con cuidado,
  a pesar de tender a 1 el paréntesis
  la potencia no necesariamente tiende a 1.
  La técnica general en estos casos es usar logaritmos
  para calcular la potencia:
  \begin{equation*}
    \left(
      1 - \frac{1}{2 n^{1/2}} + \frac{1}{8 n} - \dotsb
    \right)^n
      = \exp \left( n \ln \left(
			    1 - \frac{1}{2 n^{1/2}}
			      + \frac{1}{8 n}
			      - \dotsb
			  \right)
	     \right)
  \end{equation*}
  Luego expandimos el logaritmo en serie,%
    \index{serie de potencias!logaritmo}
  hasta llegar a términos de orden \(o(n^{-1})\).
  En nuestro caso:
  \begin{align*}
    \exp \left( n \ln \left(
			1 - \frac{1}{2 n^{1/2}}
			  + \frac{1}{8 n}
			  - \dotsb
		      \right)
	 \right)
      &= \exp \left(
		n \left(
		    \left(
		      - \frac{1}{2 n^{1/2}} + \frac{1}{8 n}
		    \right)
		      - \frac{1}{2}
			  \left(
			    - \frac{1}{2 n^{1/2}} + \frac{1}{8 n}
			  \right)^2
		      + O(n^{-3/2})
		     \right)
		  \right) \\
      &\sim \exp(- \sqrt{n} / 2)
  \end{align*}
  con lo que:
  \begin{equation*}
    r_n^n
     \sim n^{n/2} \exp(- \sqrt{n} / 2)
  \end{equation*}
  Finalmente,
  uniendo las distintas piezas:
  \begin{equation*}
    \frac{i_n}{n!}
      \sim \frac{\mathrm{e}^{\frac{n}{2} + \sqrt{n} - \frac{1}{4}}}
		{2 n^{n/2} \sqrt{\pi n}}
  \end{equation*}
  Mediante la fórmula de Stirling:%
    \index{Stirling, formula de@Stirling, fórmula de}
  \begin{equation}
    \label{eq:ae:involutions}
    i_n
      \sim \frac{1}{\sqrt{2}} n^{n/2}
	      \exp \left(
		     - \frac{n}{2} + \sqrt{n} - \frac{1}{4}
		   \right)
  \end{equation}

%%% Local Variables:
%%% mode: latex
%%% TeX-master: "clases"
%%% End:


\backmatter

\bibliography{references,rfc,iso,url}

\printindex
\end{document}

%%% Local Variables:
%%% mode: latex
%%% TeX-master: t
%%% End:

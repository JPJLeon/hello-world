% aplicaciones.tex
%
% Copyright (c) 2009-2014 Horst H. von Brand
% Derechos reservados. Vea COPYRIGHT para detalles

\chapter{Aplicaciones}
\label{cha:aplicaciones}

  Veremos varias aplicaciones concretas adicionales
  de la maquinaria de funciones generatrices,
  hallando funciones generatrices
  y también derivando (y demostrando) identidades.
  De particular interés para nosotros es la solución de recurrencias,
  que comúnmente aparecen en problemas combinatorios,
  en particular aplicaciones al análisis de algoritmos.
  En el camino estudiaremos algunas de las secuencias más comunes
  en la combinatoria.

\section{Números harmónicos}
\label{sec:numeros-harmonicos}
\index{numeros harmonicos@números harmónicos}

  Los números harmónicos se definen como:
  \begin{equation}
    \label{eq:harmonic-number}
    H_n = \sum_{1 \le k \le n} \frac{1}{k}
  \end{equation}
  Además definimos
  \(H_0 = 0\)
  (consistente con que sumas vacías son cero).
  Esta secuencia es importante en teoría de números,
  se requiere para calcular una variedad de funciones especiales,
  además que aparece con frecuencia al analizar algoritmos.

  Los primeros valores son:
  \begin{equation*}
    \left\langle
      0, 1, \frac{3}{2}, \frac{11}{6}, \frac{25}{12},
      \frac{137}{60},  \frac{49}{20}, \frac{363}{140},
      \frac{761}{280}, \frac{7\,129}{2\,520},
      \dotsc
    \right\rangle
  \end{equation*}
  Buscamos una expresión para:
  \begin{equation}
    \label{eq:harmonic-number-ogf}
    H(z)
      = \sum_{n \ge 0} H_n z^n
  \end{equation}
  Esto se reduce a:
  \begin{align*}
    H(z)
      &= \sum_{n \ge 1} \biggl( \,
			  \sum_{1 \le k \le n} \frac{1}{k}
			\biggr) z^n \\
      &= z \sum_{n \ge 1} \biggl( \,
			     \sum_{0 \le k \le n - 1}
			       \frac{1}{k + 1}
			   \biggr) z^{n - 1} \\
      &= z \sum_{n \ge 0} \biggl( \,
			     \sum_{0 \le k \le n} \frac{1}{k + 1}
			   \biggr) z^n
  \end{align*}
  Como tenemos una suma entre manos,
  usamos la regla de sumas parciales:%
    \index{generatriz!numeros harmonicos@números harmónicos}
  \begin{align}
    H(z)
      &= z \cdot \frac{1}{1 - z}
	   \cdot \sum_{k \ge 0} \frac{z^k}{k + 1}
	   \notag \\
      &= \frac{1}{1 - z} \, \ln \frac{1}{1 - z}
	   \label{eq:H(z)}
  \end{align}
  Acá usamos la suma~\eqref{eq:ln(1-z)} para el logaritmo
  derivada en la sección~\ref{sec:otras-series}.

  Aprovechando que la serie~\eqref{eq:H(z)}
  converge para \(\lvert z \rvert < 1\)
  (la expresión entra en problemas en \(z = 1\),
   el radio de convergencia es \(\lvert z \rvert = 1\)),
  podemos evaluar expresiones como:
  \begin{equation*}
    \sum_{n \ge 0} H_n \cdot 2^{-n}
      = H(1 / 2)
      = 2 \, \ln 2
  \end{equation*}

% logaritmos.tex
%
% Copyright (c) 2015 Horst H. von Brand
% Derechos reservados. Vea COPYRIGHT para detalles

\section{Funciones generatrices con logaritmos}
\label{sec:gf-logs}

  Requeriremos coeficientes de varias series involucrando logaritmos,
  de la forma:
  \begin{equation}
    \label{eq:ln:alpha-beta}
    \frac{1}{(1 - z)^\alpha} \ln^\beta \frac{1}{1 - z}
  \end{equation}
  para \(\alpha\) y \(\beta\) enteros.
  El caso más simple es \(\alpha = 0\) y \(\beta = 1\):
  \begin{equation}
    \label{eq:ln:0-1}
    \ln \frac{1}{1 - z}
      = \sum_{n \ge 1} \frac{z^n}{n}
  \end{equation}
  De~\eqref{eq:ln:0-1} resulta directamente
  la función generatriz de los números harmónicos:%
    \index{numeros harmonicos@números harmónicos!generatriz}
  \begin{equation}
    \label{eq:ln:1-1}
    \frac{1}{1 - z} \ln \frac{1}{1 - z}
      = \sum_{n \ge 1} H_n z^n
  \end{equation}
  Aplicando la fórmula de Leibnitz:%
    \index{Leibnitz!formula de@fórmula de}
  \begin{equation}
    \label{eq:Leibnitz-derivative}
    \frac{\mathrm{d}^m}{\mathrm{d} z^m} (f(z) \cdot g(z))
      = \sum_{0 \le r \le m} \binom{m}{r} f^{(r)}(z) \cdot g^{(m - r)}(z)
  \end{equation}
  a~\eqref{eq:ln:1-1} resulta:
  \begin{align}
    \frac{\mathrm{d}^{k - 1}}{\mathrm{d} z^{k - 1}}
      \left( \frac{1}{1 - z} \ln \frac{1}{1 - z} \right)
      &= \sum_{0 \le r \le k - 1}
	   \binom{k - 1}{r}
	      \frac{\mathrm{d}^r}{\mathrm{d} z^r} \ln \frac{1}{1 - z}
	      \cdot
	      \frac{\mathrm{d}^{k - 1 - r}}{\mathrm{d} z^{k - 1 - r}}
		 \frac{1}{1 - z}
		       \notag \\
      &= \frac{(k - 1)!}{(1 - z)^k} \ln \frac{1}{1 - z}
	   + \sum_{1 \le r \le k - 1}
	       \binom{k - 1}{r}
		 \frac{(r - 1)!}{(1 - z)^r}
		   \cdot \frac{(k - 1 - r)!}{(1 - z)^{k - r}}
		       \notag \\
      &= \frac{(k - 1)!}{(1 - z)^k}
	   \left(
	     \ln \frac{1}{1 - z}
	       + \sum_{1 \le r \le k - 1} \frac{1}{r}
	   \right)
		       \notag \\
      &= \frac{(k - 1)!}{(1 - z)^k}
	   \left(
	     \ln \frac{1}{1 - z}
	       + H_{k - 1}
	   \right)
  \end{align}
  Despejando el término que nos interesa:
  \begin{equation}
    \label{eq:ln:k-1}
    \frac{1}{(1 - z)^k} \ln \frac{1}{1 - z}
      = \frac{1}{(k - 1)!}
	  \frac{\mathrm{d}^{k - 1}}{\mathrm{d} z^{k - 1}}
	    \left( \frac{1}{1 - z} \ln \frac{1}{1 - z} \right)
	  - \frac{1}{(1 - z)^k} H_{k - 1}
\end{equation}
  Con~\eqref{eq:ln:1-1},
  las propiedades conocidas de las funciones generatrices%
    \index{funcion generatriz@función generatriz!propiedades}
  y el teorema del binomio:%
    \index{teorema del binomio}
  \begin{align}
    [z^n] \frac{1}{(1 - z)^k} \ln \frac{1}{1 - z}
      &= \frac{(n + k  - 1)^{\underline{k - 1}}}{(k - 1)!} H_{n + k - 1}
	   - \binom{n + k - 1}{k - 1} H_{k - 1} \notag \\
      &= \binom{n + k - 1}{k - 1} (H_{n + k - 1} - H_{k - 1})
		      \label{eq:ln:k-1:coef}
  \end{align}
  Vemos que para \(n = 0\) el coeficiente se anula
  y para \(n = 1\) es \(1\),
  tal como debiera ser.
%% Checked against the Taylor series for n = 1, 2, 4 and k = 1, 2, 10
%%   HvB 2015-01-16

  Otra colección de interés resulta partiendo con:
  \begin{align}
    [z^n] \ln^2 \frac{1}{1 - z}
      &= \sum_{1 \le r \le n - 1} \frac{1}{r (n - r)} \notag \\
  \intertext{Aplicamos fracciones parciales al sumando:}
      &= \frac{1}{n}
	   \sum_{1 \le r \le n - 1}
	     \left( \frac{1}{r} + \frac{1}{n - r} \right) \notag \\
      &= \frac{2}{n} H_{n - 1}
	     \label{eq:ln:0-2:coef}
  \end{align}
%% Checked against the Taylor series for n = 2, 3, 4, 7, 10
%%   HvB 2015-01-16
  De esto:
  \begin{align*}
    [z^n]\frac{1}{1 - z} \ln^2 \frac{1}{1 - z}
      &= \sum_{1 \le r \le n} \frac{2}{r} H_{r - 1} \\
      &= 2 \sum_{1 \le r \le n}
	     \frac{1}{r} \sum_{1 \le s \le r - 1} \frac{1}{s} \\
      &= 2 \sum_{1 \le s < r \le n} \frac{1}{r s} \\
  \intertext{Por simetría en \(r\) y \(s\) podemos escribir:}
      &= \sum_{1 \le s < r \le n} \frac{1}{r s}
	   + \sum_{1 \le r < s \le n} \frac{1}{r s} \\
      &= \sum_{\substack{1 \le r \le n \\
			 1 \le s \le n \\
			 r \ne s}} \frac{1}{r s} \\
      &= H^2_n - H^{(2)}_n
  \end{align*}
  Acá usamos la definición de números harmónicos generalizados:%
    \index{numeros harmonicos@números harmónicos!generalizados|textbfhy}
  \begin{equation}
    \label{eq:H(m)n}
    H^{(m)}_n
      = \sum_{1 \le k \le n} \frac{1}{k^m}
  \end{equation}
  Con esto el coeficiente que nos interesa es:
  \begin{equation}
    \label{eq:ln:1-2:coef}
    [z^n] \frac{1}{1 - z} \ln^2 \frac{1}{1 - z}
      = H^2_n - H^{(2)}_n
  \end{equation}
%% Checked against the Taylor series for n = 1, 2, 3, 4, 10
%%   HvB 2015-01-16

  Interesan un par de funciones generatrices adicionales.
  Primero:
  \begin{align*}
    \frac{\mathrm{d}}{\mathrm{d} z}
      \left( \frac{1}{1 - z} \ln^2 \frac{1}{1 - z} \right)
      &= \frac{1}{(1 - z)^2} \ln^2 \frac{1}{1 - z}
	   + 2 \frac{1}{(1 - z)^2} \ln \frac{1}{1 - z} \\
    \frac{\mathrm{d}^2}{\mathrm{d} z^2}
      \left( \frac{1}{1 - z} \ln^2 \frac{1}{1 - z} \right)
      &= 2 \frac{1}{(1 - z)^3} \ln^2 \frac{1}{1 - z}
	   + 6 \frac{1}{(1 - z)^3} \ln \frac{1}{1 - z}
	   + \frac{2}{(1 - z)^3}
  \end{align*}
  de donde despejamos:
  \begin{align*}
    \frac{1}{(1 - z)^2} \ln^2 \frac{1}{1 - z}
      &= \frac{\mathrm{d}}{\mathrm{d} z}
	   \left( \frac{1}{1 - z} \ln^2 \frac{1}{1 - z} \right)
	   - 2 \frac{1}{(1 - z)^2} \ln \frac{1}{1 - z} \\
    \frac{1}{(1 - z)^3} \ln^2 \frac{1}{1 - z}
      &=  \frac{1}{2} \frac{\mathrm{d}^2}{\mathrm{d} z^2}
			\left( \frac{1}{1 - z} \ln^2 \frac{1}{1 - z} \right)
	   - 3 \frac{1}{(1 - z)^3} \ln \frac{1}{1 - z}
	   - \frac{1}{(1 - z)^3}
  \end{align*}
  Derivando término a término
  la serie con coeficientes~\eqref{eq:ln:1-2:coef}
  tenemos los coeficientes de las derivadas,
  usamos los coeficientes~\eqref{eq:ln:k-1:coef} deducidos antes:
  \begin{align}
    [z^n] \frac{1}{(1 - z)^2} \ln^2 \frac{1}{1 - z}
      &= (n + 1) \left( H^2_{n + 1} - H^{(2)}_{n + 1} \right)
	  - 2 \binom{n + 1}{1} \left( H_{n + 1} - H_1 \right) \notag \\
      &= \binom{n + 1}{1}
	   \left(
	     H^2_{n + 1} - H^{(2)}_{n + 1} - 2 H_{n + 1} + 2
	   \right) \label{eq:ln:2-2:coef} \\
%% Checked against the Taylor series for n = 2, 3, 4, 5, 10
%%   HvB 2015-01-20
    [z^n] \frac{1}{(1 - z)^3} \ln^2 \frac{1}{1 - z}
      &= \frac{1}{2} (n + 2) (n + 1)
	   \left( H^2_{n + 2} - H^{(2)}_{n + 2} \right) \notag \\
	   &\qquad
	   - 3 \binom{n + 2}{2}
	       \left(
		 H_{n + 2}
		   - H_2
	       \right) \notag
	   - \binom{n + 2}{2} \notag \\
      &= \binom{n + 2}{2}
	   \left(
	     H^2_{n + 2} - H^{(2)}_{n + 2}
	       - 3 H_{n + 2} + \frac{7}{2}
	   \right)
%% Checked against the Taylor series for n = 2, 3, 4, 5, 10
%%   HvB 2015-01-20
  \end{align}

%%% Local Variables:
%%% mode: latex
%%% TeX-master: t
%%% End:


\section{Potencias factoriales}
\label{sec:potencias-factoriales}

  Definamos:
  \begin{equation*}
    G(z, u)
      = \sum_{n \ge 0} u^{\underline{n}} \, \frac{z^n}{n!}
  \end{equation*}
  Como:
  \begin{equation*}
    \frac{u^{\underline{n}}}{n!}
      = \binom{u}{n}
  \end{equation*}
  tenemos:
  \begin{equation*}
    G(z, u)
      = \sum_{n \ge 0} \binom{u}{n} z^n
      = (1 + z)^u
  \end{equation*}
  Esto implica:
  \begin{equation*}
    G(z, u) \cdot G(z, v)
      = G(z, u + v) \\
  \end{equation*}
  Podemos evaluar entonces de dos formas:
  \begin{align}
    G(z, u) \cdot G(z, v)
      &= \sum_{n \ge 0}
	   \biggl( \,
	     \sum_{0 \le k \le n}
	       \binom{n}{k}
		 \, u^{\underline{k}} v^{\underline{n - k}}
	   \biggr) \, \frac{z^n}{n!}
	       \label{eq:G(u)G(v)}\\
    G(z, u + v)
      &= \sum_{n \ge 0} (u + v)^{\underline{n}} \, \frac{z^n}{n!}
	       \label{eq:G(u+v)}
  \end{align}
  Comparando los coeficientes de \(z^n\)
  en~\eqref{eq:G(u)G(v)} y~\eqref{eq:G(u+v)} resulta:
  \begin{equation}
    \index{binomio, teorema del!potencias factoriales}
    \label{eq:binomial-falling}
    (u + v)^{\underline{n}}
      = \sum_{0 \le k \le n}
	  \binom{n}{k} \, u^{\underline{k}} v^{\underline{n - k}}
  \end{equation}
  Curioso equivalente de la fórmula
  para la potencia de un binomio.
  Aprovechando la relación
  entre potencias factoriales en subida y en bajada
  se puede derivar una relación similar
  para las potencias factoriales en subida.

\section{Números de Fibonacci}
\label{sec:Fibonacci}
\index{Fibonacci, numeros de@Fibonacci, números de}

  Consideremos la secuencia:
  \begin{equation}
    \label{eq:Fibonacci-sequence}
    \langle 0, 1, 1, 2, 5, 8, 13, 21, 34, \dotsc \rangle
  \end{equation}
  que se obtiene de la recurrencia válida para \(n \ge 0\):
  \begin{equation}
    \label{eq:recurrence-Fibonacci}
    F_{n + 2}
      = F_{n + 1} + F_n
      \qquad \text{\(F_0 = 0\), \(F_1 = 1\)}
  \end{equation}
  Esta la encontramos al analizar el algoritmo de Euclides%
    \index{Euclides, algoritmo de!analisis@análisis}
  para el máximo común divisor
  en la sección~\ref{sec:gcd}.
  Aparece en una gran variedad de situaciones
  relacionadas con nuestra área,
  y muchos fenómenos naturales,
  como el crecimiento de los árboles
  y las espirales que se observan en los girasoles,
  siguen aproximadamente esta secuencia.
  Véanse por ejemplo el libro de Dunlap~%
    \cite{dunlap98:_golden_ratio_fibonacci}
  para una variedad de situaciones donde aparecen,
  y la muy detallada discusión de Vajda~%
    \cite{vajda89:_fibonacci_lucas_number_golden_section}.

\subsection{Solución mediante funciones generatrices ordinarias}
\label{sec:Fibonacci-ordinarias}

  Definimos:
  \begin{equation*}
    F(z)
      = \sum_{n \ge 0} F_n z^n
  \end{equation*}
  Aplicando las propiedades de funciones generatrices ordinarias
  a~\eqref{eq:recurrence-Fibonacci}:
  \begin{equation*}
    \frac{F(z) - F_0 - F_1 \cdot z}{z^2}
      = \frac{F(z) - F_0}{z} + F(z)
  \end{equation*}
  Substituyendo los valores de \(F_0\) y \(F_1\)
  y despejando resulta:
  \begin{equation}
    \index{Fibonacci, numeros de@Fibonacci, números de!generatriz|textbfhy}
    \label{eq:gf-Fibonacci}
    F(z)
      = \frac{z}{1 - z - z^2}
  \end{equation}
  Necesitamos reducir~\eqref{eq:gf-Fibonacci}
  a fracciones con denominadores lineales,
  usando fracciones parciales.
  Buscamos factorizar de la siguiente manera:
  \begin{equation*}
    1 - z - z^2 = (1 - r_{+} z) (1 - r_{-} z)
  \end{equation*}
  Para obtener esta factorización
  realizamos el cambio de variable \(y = 1 / z\)
  y tenemos:
  \begin{align*}
    y^2 - y - 1
      &= (y - r_{+}) (y - r_{-}) \\
    r_{\pm}
      &= \frac{1 \pm \sqrt{5}}{2}
  \end{align*}
  y denotamos \(r_{+} = \tau\) y \(r_{-} = \phi\).
  El número \(\tau\) es la \emph{sección áurea}%
    \index{\(\tau\) (seccion aurea)@\(\tau\) (sección áurea)|see{sección áurea}}%
    \index{seccion aurea@sección áurea|textbfhy}
  (por la palabra griega para \emph{corte}),
  que ya habíamos encontrado antes
  al analizar el algoritmo de Euclides%
    \index{Euclides, algoritmo de!analisis@análisis}
  para el máximo común divisor.
  Una notación común para la sección áurea
  (particularmente en matemáticas recreativas)
  es \(\phi\) o \(\varphi\),
  en honor al escultor ateniense Fidias,%
    \index{Fidias}
  quien se dice usó esta razón extensamente en su trabajo.
  Otros usan \(\phi = - r_{-}\) y \(\Phi = r_{+}\),
  de forma de tener números positivos siempre.

  Podemos expresar:
  \begin{align*}
    y^2 - y - 1
      &= (y - \tau) (y - \phi) \\
      &= y^2 - (\tau + \phi) y + \tau \phi
  \end{align*}
  Comparando coeficientes
  resulta \(\phi = 1 - \tau = -1 / \tau\).
  Las fracciones parciales resultan ser:
  \begin{equation}
    \label{eq:gf-Fibonacci-fracciones-parciales}
    F(z) = \frac{1}{\tau - \phi} \cdot
	     \left(
		\frac{1}{1 - \tau z} - \frac{1}{1 - \phi z}
	     \right)
  \end{equation}
  En~\eqref{eq:gf-Fibonacci-fracciones-parciales}
  se reconocen dos series geométricas.
  Esto da la sorprendente relación,
  conocida como fórmula de Binet,%
    \index{Binet, formula de@Binet, fórmula de|textbfhy}%
    \index{Fibonacci, numeros de@Fibonacci, números de!formula de Binet@fórmula de Binet}
  que expresa los números de Fibonacci
  (enteros)
  en términos de números irracionales:%
    \index{numero@número!irracional}
  \begin{align}
    F_n
      &= \frac{\tau^n - \phi^n}{\tau - \phi} \\
      &= \frac{(1 + \sqrt{5})^n - (1 - \sqrt{5})^n}{2^n \sqrt{5}}
	   \label{eq:Binet-Fibonacci}
  \end{align}
  Ahora bien:
  \begin{equation*}
    \frac{1}{2 \tau - 1}
      = \frac{1}{\sqrt{5}}
      = 0,4472\dotso
    \hspace{3em}
    \tau
      = 1,618\dotso
    \hspace{3em}
    \phi
      = -0,6180\dotso
  \end{equation*}
  Resulta para todo \(n \ge 0\):
  \begin{equation*}
    \left| \frac{\phi^n}{2 \tau - 1} \right| < 0,5
  \end{equation*}
  Por lo tanto,
  \(F_n = \tau^n / \sqrt{5}\),
  redondeado al entero más cercano.
  De todas formas:%
    \index{Fibonacci, numeros de@Fibonacci, números de!asintotica@asintótica}
  \begin{equation}
    \label{eq:Fibonacci-asymptotic}
    F_n
      \sim \frac{\tau^n}{\sqrt{5}}
  \end{equation}

  Podemos obtener otras relaciones
  de la función generatriz~\eqref{eq:gf-Fibonacci}:
  \begin{equation*}
    F(z)
      = \frac{z}{1 - z - z^2}
      = \frac{z}{1 - z(1 + z)}
      = \sum_{r \ge 0} z^{r + 1} (1 + z)^r
      = \sum_{r, s \ge 0} \binom{r}{s} z^{r + s + 1}
  \end{equation*}
  De aquí,
  como con \(n = r + s\) tenemos \(r = n - s\):
  \begin{equation}
    \index{Fibonacci, numeros de@Fibonacci, números de!relacion con coeficientes binomiales@relación con coeficientes binomiales}
    \label{eq:Fibonacci-binomial}
    F_{n + 1}
      = \sum_{0 \le s \le n} \binom{n - s}{s}
  \end{equation}

\subsection{Solución mediante funciones generatrices exponenciales}
\label{sec:Fibonacci-exponenciales}

  Definimos la función generatriz exponencial:
  \begin{equation*}
    \widehat{F}(z)
      = \sum_{n \ge 0} \frac{F_n z^n}{n!}
  \end{equation*}
  Aplicando las propiedades de funciones generatrices exponenciales%
    \index{generatriz!exponencial}
  a~\eqref{eq:recurrence-Fibonacci}:
  \begin{equation*}
    \widehat{F}''(z)
      = \widehat{F}'(z) + \widehat{F}(z)
      \quad \text{\(\widehat{F}(0) = 0\), \(\widehat{F}'(0) = 1\)}
  \end{equation*}
  Resolvemos esta ecuación diferencial
  ordinaria lineal de segundo orden,
  homogénea y de coeficientes constantes
  por el método de la ecuación característica:
  \begin{equation*}
    r^2
      = r + 1
  \end{equation*}
  Los ceros son
  \(\tau\) y \(\phi\),
  con lo que tenemos:
  \begin{equation*}
    \widehat{F}(z)
      = \alpha \mathrm{e}^{\tau z} + \beta \mathrm{e}^{\phi z}
  \end{equation*}
  de donde:
  \begin{align*}
    \widehat{F}(0)  &= \alpha + \beta = 0 \\
    \widehat{F}'(0) &= \alpha \tau + \beta \phi = 1
  \end{align*}
  La solución de estas ecuaciones es:
  \begin{equation*}
    \alpha
      = \frac{1}{\sqrt{5}} \qquad
    \beta
      = -\frac{1}{\sqrt{5}}
  \end{equation*}
  y finalmente resulta
  la misma fórmula~\eqref{eq:Binet-Fibonacci} anterior:
  \begin{align*}
    F_n
      = \frac{1}{\sqrt{5}}
	  n! \left[ z^n \right] \,
	    \left(
	      \mathrm{e}^{\tau z} - \mathrm{e}^{\phi z}
	    \right)
      = \frac{\tau^n - \phi^n}{\sqrt{5}}
  \end{align*}
  Si comparamos las derivaciones,
  obtener la ecuación y sus condiciones de borde es más simple
  al usar funciones generatrices exponenciales,
  luego debemos resolver una ecuación diferencial,
  pero obtener el resultado
  de la solución de la ecuación diferencial es inmediato.
  En la derivación usando funciones generatrices ordinarias
  obtener la ecuación era algo más trabajo,
  y tuvimos que usar fracciones parciales
  para poder obtener la secuencia;
  pero tratar la ecuación misma era más simple.
  De todas formas,
  siempre tendremos las dos opciones.
  Cuál resulta más conveniente dependerá de la situación específica.

\subsection{Números de Fibonacci y fuentes}
\label{sec:Fibonacci-fuentes}

  Si comparamos la secuencia~\eqref{eq:fountains-sequence}
  de números de fuentes de base \(n\)%
    \index{fuente}
  con los números de Fibonacci,
  parecieran ser los términos alternos:
  \begin{equation*}
    \langle F_{2 n + 1} \rangle_{n \ge 0}
      = \langle 1, 2, 5, 13, 34, \dotsc \rangle
  \end{equation*}
  Una manera de verificar esto es extraer los términos impares
  de la función generatriz~\eqref{eq:gf-Fibonacci},%
    \index{serie de potencias!decimar}
  o sea encontrar una función generatriz
  para \(\langle F_{2 n + 1} \rangle_{n \ge 0}\)
  y comparar con~\eqref{eq:gf-fountain-pf}.

  Aplicamos la técnica descrita en la sección~\ref{sec:decimar}
  a la función generatriz~\eqref{eq:gf-Fibonacci}.
  Para los números de Fibonacci impares resulta:
  \begin{equation*}
    \frac{F(\sqrt{z}) - F(-\sqrt{z})}{2 \sqrt{z}}
      = \frac{1 - z}{1 - 3 z + z^2}
  \end{equation*}
  En nuestro caso tenemos la secuencia desplazada en uno,
  de~\eqref{eq:gf-fountain}:
  \begin{equation*}
    \frac{f(z) - f_0}{z}
      = \frac{1 - z}{1 - 3 z + z^2}
  \end{equation*}
  Coinciden,
  o sea:
  \begin{equation}
    \label{eq:fountain-Fibonacci}
    f_n
      = \begin{cases}
	  1	      & \text{si \(n = 0\)} \\
	  F_{2 n - 1} & \text{si \(n \ge 1\)}
	\end{cases}
  \end{equation}

\subsection{Búsqueda de Fibonacci}
\label{sec:busqueda-Fibonacci}
\index{Fibonacci, busqueda de@Fibonacci, búsqueda de|textbfhy}

  La \emph{búsqueda de Fibonacci}
  (ver por ejemplo a Kiefer~%
    \cite{kiefer53:_seq_minimax_search_maximum})
  es un método para encontrar el mínimo
  de una función en un rango dado.
  Resulta incluso que esta técnica es óptima
  en cuanto a número de veces que se evalúa la función
  siempre que nos restrinjamos a solo comparar valores.
  \begin{definition}
    Una función \(f : \mathbb{R} \rightarrow \mathbb{R}\)
    se dice \emph{unimodal} sobre el rango \([a, b]\)%
      \index{funcion@función!unimodal}
    si hay un único \(\xi\) con \(a \le \xi \le b\) tal que
    \(f\) es decreciente en \([a, \xi]\)
    y creciente en \([\xi, b]\).
  \end{definition}
  Esto describe el caso en que la función tenga un único mínimo
  en el rango,
  de forma muy similar se define el caso que tiene un único máximo,
  y en ambas situaciones se llama unimodal la función.

  Nos interesa acotar el mínimo en el rango \([a, b]\)
  de una función unimodal \(f(z)\)
  recurriendo únicamente a evaluar la función.
  Por ejemplo,
  la función está dada por una computación compleja
  y no hay forma de calcular su derivada.
  Supongamos que tenemos los valores de la función
  en los puntos \(a\) y \(b\).
  Elegimos dos puntos adicionales \(c < d\)
  dentro del rango \([a, b]\),
  y evaluamos la función en ellos,
  resultando la situación
  de la figura~\ref{fig:Fibonacci-search-step}.
  \begin{figure}[htbp]
    \centering
    \pgfimage{images/Fibonacci-search-step}
    \caption{Búsqueda de Fibonacci}
    \label{fig:Fibonacci-search-step}
  \end{figure}
  Al ser unimodal \(f\)
  sabemos que \(f(c)\) y \(f(d)\)
  son ambos menores que \(\max \{ f(a), f(b) \}\).
  Si \(f(c) < f(d)\),
  el mínimo está en el subintervalo \([a, d]\),
  descartamos el tramo \((d, b]\)
  y trabajamos con el nuevo rango \([a, d]\).
  De la misma manera,
  si \(f(c) > f(d)\),
  el mínimo está en el tramo \([c, b]\),
  descartamos el rango \([a, c)\).

  Consideremos el ejemplo
  de la figura~\ref{fig:Fibonacci-search-step}.
  Descartamos el rango \([a, c)\)
  y elegimos un nuevo punto \(e\) entre \(d\) y \(b\),
  y seguimos con nuevos puntos \(a'\), \(b'\), \(c'\) y \(d'\).
  Para solo calcular una vez la función en la iteración
  se reutilizan los valores calculados
  en los puntos \(a\), \(c\) y \(d\)
  (de descartar \((d, b]\))
  o en los puntos \(c\), \(d\) y \(b\)
  (de descartar \([a, c)\)).
  Queremos además
  que el método reduzca el tramo en la misma proporción
  en ambos casos.
  O sea,
  debe ser \(d - a = b - c\).
  En el siguiente paso queremos que se vuelva a repetir esto,
  debe ser también
  \(b - d = e - c\).

  Definamos \(r\) mediante \(d - a = r (b - a)\),
  con lo que \(c - a = (1 - r) (b - a)\).
  Restando:
  \begin{align*}
    d - c
      &= (d - a) - (c - a) \\
      &= r (b - a) - (1 - r) (b - a) \\
      &= (2 r  - 1) (b - a)
  \end{align*}
  Para el paso siguiente,
  \(a' = c\),
  \(c' = d\),
  \(d' = e\)
  y \(b' = b\).
  Elegimos \(r'\) mediante \(d' - a' = r' (b' - a')\),
  de donde resulta \(c' - a' = (1 - r') (b' - a')\),
  que es decir \(d - c = (1 - r') (b' - a')\).
  El intervalo \([a, b]\) se redujo a \([a', b']\),
  con \(b' - a' = r (b - a)\).
  Igualando los valores de \(d - c\),
  y substituyendo el valor de \(b' - a'\) resulta:
  \begin{align*}
    (2 r - 1) (b - a)
      &= (1 - r') (b ' - a') \\
    (2 r - 1) (b - a)
      &= (1 - r') r (b - a) \\
    2 r - 1
      &= (1 - r') r
  \end{align*}
  Despejando \(r\):
  \begin{equation}
    \label{eq:Fibonacci-search-recurrence-r}
    r = \frac{1}{r' + 1}
  \end{equation}
  Partiendo del final,
  esto permite calcular las razones previas.
  En el caso extremo reducimos el intervalo en una razón de \(1\)
  (vale decir, se mantiene el tamaño).
  \begin{figure}[htbp]
    \centering
     \pgfimage{images/Fibonacci-search-final}
    \caption{Búsqueda de Fibonacci: Juego final}
    \label{fig:Fibonacci-search-final}
  \end{figure}
  El paso final
  lo ilustra la figura~\ref{fig:Fibonacci-search-final}.
  El algoritmo retorna el rango marcado \(a\) a \(b\)
  como resultado final,
  la mejor aproximación a \(\xi\) es \(c = d = (a + b) / 2\).

  Esto sugiere la recurrencia:
  \begin{equation}
    \label{eq:Fibonacci-search-recurrence}
    r_{k + 1}
      = \frac{1}{1 + r_k}
      \quad (k \ge 1)
      \qquad r_0 = 1
  \end{equation}
  Intentando algunos valores obtenemos:
  \begin{equation*}
    \left\langle
      1, \frac{1}{2}, \frac{2}{3}, \frac{3}{5}, \frac{5}{8}, \dotsc
    \right\rangle
  \end{equation*}
  Da la impresión que:
  \begin{equation}
    \label{eq:Fibonacci-search-solution}
    r_k
      = \frac{F_{k + 1}}{F_{k + 2}}
  \end{equation}
  Esto es fácil de demostrar por inducción,
  los detalles los proveerá el amable lector.
  Con esto estamos en condiciones de plantear
  la búsqueda de Fibonacci,
  algoritmo~\ref{alg:Fibonacci-search}.
  Suponemos dados el intervalo \([a, b]\)
  y la tolerancia \(\epsilon\)
  (el largo del último intervalo).
  Para reducir el largo del intervalo en un factor \(F_n\)
  se calcula la función \(n + 4\) veces.
  En vista de~\eqref{eq:Fibonacci-asymptotic}
  para reducir el rango de largo \(L_0\) a \(L_f\)
  el número de llamadas de la función es:
  \begin{align*}
    n &\sim \frac{\ln (L_0 / L_f)}{\ln \tau}
	      + 4 + \frac{\ln 5}{2 \ln \tau} \\
      &\sim 4,78497 \, \ln \frac{L_0}{L_f} + 5,67723
  \end{align*}
  \begin{algorithm}[htbp]
    \DontPrintSemicolon
    \SetKwFunction{Fibonacci}{FibonacciSearch}

    \KwFunction \Fibonacci{\(f,\; a, \; b, \; \epsilon\)} \;
    \BlankLine
    \(L \leftarrow (b - a) / \epsilon\) \;
    \((F_a, F_b) \leftarrow (1, 1)\) \;
    \While{\(F_b < L\)}{
      \((F_a, F_b) \leftarrow (F_b, F_a + F_b)\) \;
    }
    \(c \leftarrow b - (b - a) \cdot F_a / F_b\) \;
    \(d \leftarrow a + (b - a) \cdot F_a / F_b\) \;
    \((f_a, f_b, f_c, f_d) \leftarrow (f(a), f(b), f(c), f(d))\) \;
    \While{\(F_a \ne 1\)}{
      \((F_a, F_b) \leftarrow (F_b - F_a, F_a)\) \;
      \eIf{\(f_d < f_c\)}{
	\(e \leftarrow c + (b - c) \cdot F_a / F_b\) \;
	\(f_e \leftarrow f(e)\) \;
	\((a, c, d, b) \leftarrow (c, d, e, b)\) \;
	\((fa, fc, fd, fb) \leftarrow (fc, fd, fe, fb)\) \;
      }{
	\(e \leftarrow d - (d - a) \cdot F_a / F_b\) \;
	\(f_e \leftarrow f(e)\) \;
	\((a, c, d, b) \leftarrow (a, e, c, d)\) \;
	\((fa, fc, fd, fb) \leftarrow (fa, fe, fc, fd)\) \;
      }
    }
    \eIf{\(f_a < f_b\)}{
      \Return \([a, d]\) \;
    }{
      \Return \([d, b]\) \;
    }
    \caption{Búsqueda de Fibonacci}
    \label{alg:Fibonacci-search}
  \end{algorithm}

  Un método relacionado es la búsqueda de sección áurea.%
    \index{seccion aurea@sección áurea!busqueda de@búsqueda de|see{Fibonacci, búsqueda de}}
  La idea es similar,
  solo que en vez de ir modificando \(r\) se usa el valor límite:
  \begin{equation*}
    \lim_{n \rightarrow \infty} \frac{F_{n + 1}}{F_n}
      = \tau
  \end{equation*}
  El programa es un poco más sencillo,
  pero algo menos eficiente
  (requiere más evaluaciones de la función).

\section{Coeficientes binomiales}
\label{sec:coeficientes-binomiales}
\index{coeficiente binomial}

  Hagamos como que nada sabemos\ldots
  ¿Cuántos subconjuntos de \(k\)
  elementos podemos obtener de un conjunto de \(n\) elementos?
  Obviamente,
  exactamente qué conjunto de \(n\) elementos tomemos da lo mismo,
  podemos usar el conjunto \(\{1, 2, \dotsc, n\}\)
  sin pérdida de generalidad.
  Podemos deducir algunas propiedades de estas:
  \begin{itemize}
  \item
    Tomar un número negativo de elementos
    del conjunto no tiene sentido,
    o sea,
    \(\binom{n}{k} = 0\) si \(k < 0\).
  \item
    Tomar más de \(n\) elementos es imposible,
    así que \(\binom{n}{k} = 0\) si \(k > n\).
  \item
    Hay una única forma de elegir cero elementos,
    y \(\binom{n}{0} = 1\).
  \item
    De la misma forma,
    hay una única manera de elegirlos todos,
    y \(\binom{n}{n} = 1\).
  \item
    Elegir \(k\) elementos a poner en el subconjunto
    es lo mismo que elegir los \(n - k\) que se dejan fuera,
    o sea \(\binom{n}{k} = \binom{n}{n - k}\).
  \end{itemize}
  Ahora buscamos encontrar
  una recurrencia para los \(\binom{n}{k}\).
  Podemos descomponer los \(\binom{n}{k}\) subconjuntos
  en dos grupos:
  \begin{description}
  \item[\boldmath Aquellos conjuntos
	que no contienen a \(n\):\unboldmath]
    Corresponden simplemente a tomar \(k\) elementos
    de los restantes \(n - 1\),
    de estos hay \(\binom{n - 1}{k}\).
  \item[\boldmath Aquellos conjuntos
	que contienen a \(n\):\unboldmath]
    Tomamos \(n\),
    y \(k - 1\) elementos más de entre los restantes \(n - 1\),
    de estos hay \(\binom{n - 1}{k - 1}\).
  \end{description}
  Como estas dos posibilidades son excluyentes,
  y corresponden a todas las formas
  de armar subconjuntos de \(k\) elementos:
  \begin{equation*}
    \binom{n}{k} = \binom{n - 1}{k} + \binom{n - 1}{k - 1}
  \end{equation*}
  Esto en principio es válido para \(1 \le k \le n - 1\).
  Pero si substituimos \(k = n\) bajo los entendidos de arriba
  resulta \(\binom{n}{n} = 1\),
  y con \(k > n\) se reduce a \(\binom{n}{k} = 0\),
  y la recurrencia en realidad es válida para \(k \ge 1\).
  Partiremos de \(k + 1\) y \(n + 1\)
  para poder sumar desde \(k = 0\) y \(n = 0\):
  \begin{equation}
    \label{eq:binomial-recurrencia}
    \binom{n + 1}{k + 1}
      = \binom{n}{k + 1} + \binom{n}{k}
  \end{equation}
  Como condiciones de contorno bastan:
  \begin{equation*}
    \binom{n}{0}
      = 1 \qquad
    \binom{0}{k}
      = [k = 0]
  \end{equation*}
  Ahora tenemos tres opciones de función generatriz ordinaria:
  \begin{equation*}
    A_n(x)
      = \sum_{k \ge 0} \binom{n}{k} x^k
      \qquad
    B_k(y)
      = \sum_{n \ge 0} \binom{n}{k} y^n
      \qquad
    C(x, y)
      = \sum_{\substack{k \ge 0 \\
			n \ge 0}}
	  \binom{n}{k} x^k y^n
  \end{equation*}
  Las primeras dos opciones
  llevarán a recurrencias en la función generatriz,
  cosa que la tercera resuelve automáticamente.
  Nada indica particulares complicaciones
  (más allá del uso de dos variables y no una
   como ha sido común hasta acá),
  por lo que optaremos por esta.
  Aplicando las propiedades de funciones generatrices ordinarias%
    \index{generatriz!ordinaria}
  a la recurrencia queda:
  \begin{equation}
    \label{eq:binomial-bivariada}
    \frac{C(x, y) - C(x, 0) - C(0, y) + C(0, 0)}{x y}
      = \frac{C(x, y) - C(0, y)}{x} + C(x, y)
  \end{equation}
  El numerador del lado izquierdo
  de~\eqref{eq:binomial-bivariada} corresponde a:
  \begin{equation*}
    \sum_{\substack{k \ge 0 \\
		    n \ge 0}}
      \binom{n + 1}{k + 1} x^{k + 1} y^{n + 1}
	= \sum_{\substack{k \ge 0 \\
			  n \ge 0}}
		\binom{n}{k} x^k y^n
	    - \sum_{k \ge 0} \binom{0}{k} x^k y^0
	    - \sum_{n \ge 0} \binom{n}{0} x^0 y^n
	    + \binom{0}{0}
  \end{equation*}
  que resulta de eliminar la primera fila y columna de la suma
  (de eso se hacen cargo los dos siguientes términos);
  al restarlas estamos restando dos veces \(C(0, 0)\),
  y debemos reponerlo,
  lo que da lugar al último término.
  Esto es el principio de inclusión y exclusión,%
    \index{inclusion y exclusion, principio de@inclusión y exclusión, principio de}
  capítulo~\ref{cha:pie},
  haciendo su trabajo.
  De nuestras condiciones de contorno:
  \begin{equation*}
    C(0, 0)
      = \binom{0}{0} = 1 \qquad
    C(x, 0)
      = \sum_{k \ge 0} \binom{0}{k} \, x^k = 1 \qquad
    C(0, y)
      = \sum_{n \ge 0} \binom{n}{0} \, y^n = \frac{1}{1 - y}
  \end{equation*}
  Despejando obtenemos:
  \begin{equation*}
    C(x, y)
      = \frac{1}{1 - (1 + x) y}
  \end{equation*}
  Expandiendo la serie geométrica:%
    \index{serie geometrica@serie geométrica}
  \begin{equation*}
    \binom{n}{k}
      = \left[ x^k y^n \right] C(x, y)
      = \left[ x^k y^n \right]
	  \sum_{r \ge 0} (1 + x)^r y^r
      = \left[ x^k \right] (1 + x)^n
  \end{equation*}
  Tenemos nuevamente la relación
  entre los coeficientes binomiales
  y el número de combinaciones de \(k\) elementos
  tomados entre \(n\).

  La recurrencia~\eqref{eq:binomial-recurrencia}
  mostrada de la siguiente manera:
  \begin{equation*}
    \xymatrix{
      \displaystyle \binom{n}{k} \ar[dr] &
	 & \displaystyle \binom{n}{k + 1} \ar[dl] \\
	 & \displaystyle\binom{n + 1}{k + 1} &
    }
  \end{equation*}
  y recordando \(\binom{n}{0} = \binom{n}{n} = 1\)
  da el famoso triángulo de Pascal,%
    \index{Pascal, triangulo de@Pascal, triángulo de|textbfhy}
  ver cuadro~\ref{tab:triangulo-Pascal}
  (comparar también con la sección~\ref{sec:conteos-recurrentes},
   en particular el teorema~\ref{theo:identidad-Pascal}).
  \begin{table}[htbp]
    \centering
    \begin{tabular}{>{\(}r<{\)}*{12}{>{\(}c<{\)}@{\hspace{1ex}}}>{\(}c<{\)}}
      n=0:& \phantom{00}
		& \phantom{00}
		    & \phantom{00}
			& \phantom{00}
			    & \phantom{00}
				 & \phantom{00}
				      &	 1 \\
	 \noalign{\smallskip\smallskip}
      n=1:&	&   &	&   &	 &  1 &	   &  1 \\
	 \noalign{\smallskip\smallskip}
      n=2:&	&   &	&   & 1	 &    &	 2 & \phantom{00}
					       &  1 \\
	 \noalign{\smallskip\smallskip}
      n=3:&	&   &	& 1 &	 &  3 &	   &  3 & \phantom{00}
						    &  1 \\
	 \noalign{\smallskip\smallskip}
      n=4:&	&   & 1 &   & 4	 &    &	 6 &	&  4 & \phantom{00}
							 &  1 \\
	 \noalign{\smallskip\smallskip}
      n=5:&	& 1 &	& 5 &	 & 10 &	   & 10 &    &	5 & \phantom{00}
							      &	 1
	      & \phantom{00} \\
	 \noalign{\smallskip\smallskip}
      n=6:& 1 &	  & 6	 &  & 15 &    & 20 &	& 15 &	  & 6 & \phantom{00}
								   & 1 \\
	 \noalign{\smallskip\smallskip}
    \end{tabular}
    \caption{Triángulo de Pascal}
    \label{tab:triangulo-Pascal}
  \end{table}

% recurrencia-2.tex
%
% Copyright (c) 2013-2014 Horst H. von Brand
% Derechos reservados. Vea COPYRIGHT para detalles

\section{Otras recurrencias de dos índices}
\label{sec:recurrencia-2-indices}

  Consideremos el problema de calcular
  cuántos subconjuntos de \(k\) elementos de \(\{1, 2, \dotsc, n\}\)
  hay tal que no contienen números consecutivos.
  Llamemos \(s(n, k)\) a este valor.
  Para construir una recurrencia para ellos,
  aplicamos el método general de ver qué pasa al incluir o excluir \(n\).
  \begin{description}
  \item[\boldmath No incluye \(n\):\unboldmath]
    Esto es simplemente elegir \(k\) de entre los primeros \(n - 1\),
    o sea \(s(n - 1, k)\).
  \item[\boldmath Incluye \(n\):\unboldmath]
    Quedan por agregar \(k - 1\) elementos,
    que no pueden incluir a \(n - 1\),
    o sea corresponde a \(s(n - 2, k - 1)\).
  \end{description}
  Esto nos da la recurrencia:
  \begin{equation*}
    s(n, k)
      = s(n - 1, k) + s(n - 2, k - 1)
  \end{equation*}
  Ajustando índices:
  \begin{equation}
    \label{eq:recurrence-s-1}
    s(n + 2, k + 1)
      = s(n + 1, k + 1) + s(n, k)
  \end{equation}
  Es claro que para \(n \ge 1\):
  \begin{equation}
    s(n, 1)
      = n
	   \label{eq:recurrence-s(n,1)-boundary}
  \end{equation}
  Requeriremos
  los valores \(s(0, 0)\), \(s(n, 0)\), \(s(0, k)\), \(s(1, k)\).
  De la recurrencia,
  con \(n \ge 0\):
  \begin{align*}
    s(n + 2, 1)
      &= s(n + 1, 1) + s(n, 0) \\
    n + 2
      &= n + 1 + s(n, 0)
  \end{align*}
  Por tanto,
  definimos \(s(n, 0) = 1\).
  Similarmente:
  \begin{equation*}
    s(2, k + 1)
      = s(1, k + 1) + s(0, k)
  \end{equation*}
  Para \(k \ge 1\) resulta \(s(0, k) = 0\),
  con lo que \(s(0, k) = [k = 0]\).
  Además,
  uniendo los casos \(s(1, 0) = s(1, 1) = 1\)
  con \(s(1, k) = 0\) para \(k > 1\):
  \begin{equation*}
    s(1, k)
      = [ 0 \le k \le 1 ]
  \end{equation*}

  Definamos la función generatriz:
  \begin{equation}
    \label{eq:recurrence-s-GF}
    S(x, y)
      = \sum_{n, k \ge 0} s(n, k) x^n y^k
  \end{equation}
  Para aplicar nuestra técnicas de solución de recurrencias,
  necesitaremos las sumas:
  \begin{align*}
    x^2 y \sum_{n, k \ge 0} s(n + 2, k + 1) x^n y^k
      &= S(x, y)
	   - \sum_{k \ge 0} s(0, k) y^k
	   - \sum_{k \ge 0} s(1, k) x y^k
	   - \sum_{n \ge 0} s(n, 0) x^n \\
      &\hspace{6em}
	   + s(0, 0)
	   + s(1, 0) x \\
      &= S(x, y)
	   - 1
	   - x (1 + y)
	   - \frac{1}{1 - x}
	   + 1
	   + x \\
      &= S(x, y) - x y - \frac{1}{1 - x} \\
    x y \sum_{n, k \ge 0} s(n + 1, k + 1) x^n y^k
      &= S(x, y)
	   - \sum_{k \ge 0} s(0, k) y^k
	   - \sum_{n \ge 0} s(n, 0) x^n
	   + s(0, 0) \\
      &= S(x, y)
	   - 1
	   - \frac{1}{1 - x}
	   + 1 \\
      &= S(x, y) - \frac{1}{1 - x}
  \end{align*}
  Los términos que se suman se han restado dos veces,
  y deben reponerse.
  Esto con la recurrencia da:
  \begin{equation*}
    \frac{S(x, y)
	    - x (1 + y)
	    - \frac{1}{1 - x}}
	 {x^2 y}
      = \frac{S(x, y) - \frac{1}{1 - x}}{x y}
	  + S(x, y)
  \end{equation*}
  Despejando:
  \begin{equation}
    \label{eq:GF-s}
    S(x, y)
      = \frac{1 + x y}{1 - x - x^2 y}
  \end{equation}
  Podemos escribir~\eqref{eq:GF-s} como:
  \begin{align*}
    S(x, y)
      &= \frac{1 + x y}{1 - x (1 + x y)} \\
      &= \sum_{r \ge 0} x^r (1 + x y)^{r + 1} \\
      &= \sum_{r \ge 0} x^r
	   \sum_{s \ge 0} \binom{r + 1}{s} \, x^s y^s \\
      &= \sum_{r, s \ge 0} \binom{r + 1}{s} \, x^{r + s} y^s
  \end{align*}
  De aquí:
  \begin{equation}
    \label{eq:s(n,k)}
    s(n, k)
      = \left[ x^n y^k \right] \, S(x, y)
      = \binom{n - k + 1}{k}
  \end{equation}

  Para \(y = 1\) la función generatriz~\eqref{eq:GF-s} da:
  \begin{equation*}
    S(x, 1)
      = \frac{1 + x}{1 - x - x^2}
      = \frac{F(x) - x}{x^2}
  \end{equation*}
  Acá \(F(x)\) es la función generatriz
  de los números de Fibonacci~\eqref{eq:gf-Fibonacci}.
  Esto concuerda con sumar la recurrencia~\eqref{eq:recurrence-s-1}
  sobre todo \(k\),
  como \(s(0, 0) = 1 = F_2\) y \(s(1, 0) + s(1, 1) = 2 = F_3\).

  Otro caso de interés son los números de Delannoy~%
    \cite{delannoy95:_quest_probab}
  (ver también la discusión de Banderier y Schwer~%
    \cite{banderier05:_why_delannoy_numbers}).
  Se define \(D(m, n)\)
  como el número de caminos entre \((0, 0)\) y \((m, n)\)
  en una cuadrícula,
  si se permiten únicamente pasos hacia el norte, el nordeste o este.
  De la definición es clara la recurrencia:
  \begin{equation}
    \label{eq:Delannoy-recurrence}
    D(m, n)
      = D(m - 1, n) + D(m, n - 1) + D(m - 1, n - 1)
    \qquad D(0, 0) = 1
  \end{equation}
  Para aplicar nuestra técnica requeriremos:
  \begin{equation}
    \label{eq:Delannoy-boundary}
    D(m, 0) = D(0, n) = 1
  \end{equation}
  Definiendo la función generatriz ordinaria:
  \begin{equation}
    \label{eq:Delannoy-GF-definition}
    d(x, y)
      = \sum_{m, n \ge 0} D(m, n) x^m y^n
  \end{equation}
  obtenemos la ecuación funcional:
  \begin{equation}
    \label{eq:Delannoy-FE}
    \frac{d(x, y) - d(0, y) - d(y, 0) + d(0, 0)}{x y}
      = \frac{d(x, y) - d(0, y)}{x} + \frac{d(x, y) - d(x, 0)}{y} + d(x, y)
  \end{equation}
  Las condiciones de contorno dan:
  \begin{equation}
    \label{eq:Delannoy-boundary-GF}
    d(0, y)
      = \frac{1}{1 - y}
    \hspace{3em}
    d(x, 0)
      = \frac{1}{1 - x}
  \end{equation}
  Substituyendo~\eqref{eq:Delannoy-boundary-GF} en~\eqref{eq:Delannoy-FE}
  y despejando \(d(x, y)\) resulta:
  \begin{equation}
    \label{eq:Delannoy-GF}
    d(x, y)
      = \frac{1}{1 - x - y - x y}
  \end{equation}
  El lector interesado verificará que expandir como serie geométrica
  y extraer los coeficientes respectivos resulta en:
  \begin{equation}
    \label{eq:Delannoy}
      D(m, n)
	= \sum_t \binom{m + n - t}{n} \, \binom{n}{t}
  \end{equation}
  La asimetría de~\eqref{eq:Delannoy}
  ofende las sensibilidades del autor.
  Puede escribirse en forma simétrica en términos de coeficientes trinomiales,
  eso sí.

  La siguiente idea da una expansión más simétrica:
  \begin{align*}
    d(x, y)
      &= \frac{1}{(1 - x) (1 - y) - 2 x y} \\
      &= \frac{(1 - x) (1 - y)}{1 - \frac{2 x y}{(1 - x) (1 - y)}} \\
      &= (1 - x) (1 - y)
	   \sum_{r \ge 0} \left( \frac{2 x y}{(1 - x) (1 - y)} \right)^r \\
      &= \sum_{r \ge 0}
	  \frac{2^r x^r y^r}{(1 - x)^{r - 1} \, (1 - y)^{r - 1}} \\
      &= \sum_{r \ge 0} 2^r \,
	   \sum_{s \ge 0} \binom{r + s}{s} \, x^{r + s}
	   \sum_{t \ge 0} \binom{r + t}{t} \, y^{r + t}
  \end{align*}
  Al extraer el coeficiente de \(x^m y^n\)
  solo sobreviven los términos con \(r + s = m\) y \(r + t = n\),
  aprovechando la simetría de los coeficientes binomiales:
  \begin{equation}
    \label{eq:Delannoy-alt}
    D(m, n)
      = \sum_{r \ge 0} 2^r \, \binom{m}{r} \, \binom{n}{r}
  \end{equation}
  En todo caso,
  ya habíamos demostrado esta identidad como~\eqref{eq:so:binomial-identity}
  usando aceite de serpiente.

%%% Local Variables:
%%% mode: latex
%%% TeX-master: "clases"
%%% End:


\section{Dividir y conquistar}
\label{sec:dividir-y-conquistar}

% Fixme: Agregar los algoritmos, eliminar comentarios abajo
% Fixme: Más ejemplos de algoritmos! Derivar alguno?

  Una de las estrategias más fructíferas para diseñar algoritmos
  es la que se llama \emph{dividir y conquistar}%
    \index{dividir y conquistar}
  (ver por ejemplo Cormen, Leiserson, Rivest y Stein~%
    \cite{cormen09:_introd_algor}).
  La idea es resolver un problema ``grande''
  por la vía de expresarlo
  en términos de varios problemas menores del mismo tipo,
  resolver estos (recursivamente)
  y luego combinar los resultados.
  Ejemplos típicos son el ordenamiento por intercalación%
    \index{ordenamiento!intercalacion@intercalación}
%  (ver algoritmo~\ref{alg:mergesort})
  y búsqueda binaria.%
    \index{busqueda binaria@búsqueda binaria}
%   (ver algoritmo~\ref{alg:busqueda-binaria}).

  Un ejemplo menos conocido
  es el algoritmo de Karatsuba
  para multiplicación de números enteros~%
    \cite{karatsuba62:_multiplication}.%
    \index{Karatsuba, algoritmo de}
%   (ver algoritmo~\ref{alg:Karatsuba}).
  Se desean multiplicar números de \(2 n\) dígitos,
  llamémosles \(A\) y \(B\),
  los dividimos en mitades más y menos significativas.
  Si la base es \(10\),
  escribimos:
  \begin{equation*}
    A = a \cdot 10^n + b
    \hspace{3em}
    B = c \cdot 10^n + d
  \end{equation*}
  donde \(0 \le a, b, c, d < 10^n\),
  y tenemos:
  \begin{equation*}
    A \cdot B
      = a c \cdot 10^{2 n}
	  + (a d + b c) \cdot 10^n
	  + b d
  \end{equation*}
  Esta fórmula permite calcular un producto de dos números
  de \(2 n\) dígitos
  mediante cuatro multiplicaciones de números de \(n\) dígitos
  (y algunas operaciones adicionales,
   como sumas de números de a lo más \(2 n\) dígitos).
  Si definimos:
  \begin{align*}
    u = a + b
    \hspace{3em}
    v = c + d
    \hspace{3em}
    u v
      = a c + a d + b c + b d
  \end{align*}
  podemos expresar:
  \begin{equation*}
    A \cdot B
      = a c \cdot 10^{2 n} + (u v - a c - b d) \cdot 10^n + b d
  \end{equation*}
  Esta fórmula significa usar tres
  (no cuatro)
  multiplicaciones,
  a costa de más operaciones de suma.
  Si comenzamos con números con \(2^n\) dígitos,
  podemos aplicar esta estrategia recursivamente,
  y los ahorros se suman.

  Un ejemplo lo da el producto \(23\,316\,384 \cdot 20\,936\,118\).
  Tenemos:
  \begin{align*}
    n &= 8 \\
    A &= 23\,316\,384  \\
    B &= 20\,936\,118 \\
    a &= 2\,331 \quad b = 6\,384
	 \qquad c = 2\,093 \quad d = 6\,118 \\
    u &= 8\,715 \quad v = 8\,211
  \end{align*}
  Debemos ahora calcular:
  \begin{align*}
    a c
      &= 2\,331 \cdot 2\,093 \\
      &= (23 \cdot 20) \cdot 10^4
	   + ((23 + 31) \cdot (20 + 93)
	   - 23 \cdot 20
	   - 31 \cdot 93) \cdot 10^2
	   + 31 \cdot 93 \\
      &= 460 \cdot 10^4 + 2\,759 \cdot 10^2 + 2\,883 \\
      &= 4\,878\,783
  \end{align*}
  En esto hemos calculado,
  por ejemplo:
  \begin{equation*}
    23 \cdot 20
      = 2 \cdot 2 \cdot 10^2
	  + ((2 + 3) \cdot (2 + 0)
	  - 2 \cdot 2
	  - 3 \cdot 0) \cdot 10
	  + 3 \cdot 0
      = 4 \cdot 10^2 + 6 \cdot 10 + 0
      = 460
  \end{equation*}
  Los otros valores intermedios a calcular son:
  \begin{equation*}
    b d
      = 39\,057\,312
    \hspace{3em}
    u v
      = 71\,558\,865
    \hspace{3em}
    u v - a c - b d
      = 27\,622\,770
  \end{equation*}
  Combinando los anteriores queda finalmente:
  \begin{align*}
     23\,316\,384 \cdot 20\,936\,118
       &= 4\,878\,783 \cdot 10^8
	    + 27\,622\,770 \cdot 10^4
	    + 39\,057\,312 \\
       &= 488\,154\,566\,757\,312
  \end{align*}
%   \begin{algorithm}[htbp]
%     \caption{Ordenamiento por intercalación}
%     \label{alg:mergesort}
%   \end{algorithm}
%   \begin{algorithm}[htbp]
%     \caption{Búsqueda binaria}
%     \label{alg:busqueda-binaria}
%   \end{algorithm}
%   \begin{algorithm}[htbp]
%     \caption{Multiplicación eficiente}
%     \label{alg:Karatsuba}
%   \end{algorithm}

  Otro ejemplo de esta estrategia es el algoritmo de Strassen~%
    \cite{strassen69:_matrix_multiplication}%
    \index{Strassen, multiplicacion de@Strassen, multiplicación de}
%   (ver algoritmo~\ref{alg:Strassen})%
  para multiplicar matrices.
  Consideremos primeramente
  el producto de dos matrices de \(2 \times 2\):
  \begin{equation*}
    \begin{pmatrix}
      c_{1 1} & c_{1 2} \\
      c_{2 1} & c_{2 2}
    \end{pmatrix}
      = \begin{pmatrix}
	  a_{1 1} & a_{1 2} \\
	  a_{2 1} & a_{2 2}
	\end{pmatrix}
	  \cdot
	    \begin{pmatrix}
	      b_{1 1} & b_{1 2} \\
	      b_{2 1} & b_{2 2}
	    \end{pmatrix}
  \end{equation*}
  Sabemos que:
  \begin{equation*}
    \begin{array}{l@{\qquad}l}
      c_{1 1}
	= a_{1 1} b_{1 1} + a_{1 2} b_{2 1} &
      c_{1 2}
	= a_{1 1} b_{1 2} + a_{1 2} b_{2 2} \\
      c_{2 1}
	= a_{2 1} b_{1 1} + a_{2 2} b_{2 1} &
      c_{2 2}
	= a_{2 1} b_{1 2} + a_{2 2} b_{2 2}
    \end{array}
  \end{equation*}
  Esto corresponde a \(8\) multiplicaciones.
  Definamos los siguientes productos:
  \begin{equation*}
    \begin{array}{l@{\qquad}l}
      m_1
	= (a_{1 1} + a_{2 2}) \, (b_{1 1} + b_{2 2}) &
      m_2
       = (a_{2 1} + a_{2 2}) \, b_{1 1} \\
      m_3
       = a_{1 1} \, (b_{1 2} - b_{2 2}) &
      m_4
       = a_{2 2} \, (b_{2 1} - b_{1 1}) \\
      m_5
       = (a_{1 1} + a_{1 2}) \, b_{2 2} &
      m_6
       = (a_{2 1} - a_{1 1}) \, (b_{1 1} + b_{1 2}) \\
      m_7
       = (a_{1 2} - a_{2 2}) \, (b_{2 1} + b_{2 2})
    \end{array}
  \end{equation*}
  Entonces podemos expresar:
  \begin{align*}
    \begin{array}{l@{\qquad}l}
      c_{1 1}
	= m_1 + m_4 - m_5 + m_7 &
      c_{1 2}
	= m_3 + m_5 \\
      c_{2 1}
	= m_2 + m_4 &
      c_{2 2}
	= m_1 - m_2 + m_3 + m_6
    \end{array}
  \end{align*}
  Con estas fórmulas se usan \(7\) multiplicaciones
  para evaluar el producto de dos matrices.
  Cabe hacer notar que estas fórmulas
  no hacen uso de conmutatividad,
  por lo que son aplicables también
  para multiplicar matrices de \(2 \times 2\)
  cuyos elementos son a su vez matrices.
  Podemos usar esta fórmula recursivamente
  para multiplicar matrices de \(2^n \times 2^n\).
%   \begin{algorithm}[htbp]
%     \caption{Multiplicación eficiente de matrices}
%     \label{alg:Strassen}
%   \end{algorithm}

  Tal vez el algoritmo más importante
  basado en dividir y conquistar
  es el que se conoce
  como \emph{transformada rápida de Fourier},%
    \index{Fourier, transformada rapida de@Fourier, transformada rápida de}%
    \index{FFT|see{Fourier, transformada rápida de}}
  generalmente abreviado \emph{FFT}
  (de \emph{\foreignlanguage{english}{Fast Fourier Transform}} en inglés);
  se acredita a Cooley y~Tukey~%
    \cite{cooley65:_FFT}
  (aunque para variar un poco,
   más de dos siglos antes Gauß%
     \index{Gauss, Carl Friedrich@Gauß, Carl Friedrich}
   ya lo empleaba,
   como relatan Heideman, Johnson y Burrus~%
     \cite{heideman84:_gauss_history_FFT}).
  Elegido
  como uno de los \(10\) algoritmos más importantes del siglo~XX~%
    \cite{dongarra00:_top10_algorithms},
  es la base de mucho de lo que es procesamiento de señales hoy día,
  es el corazón del algoritmo de Schönhage y Strassen,%
    \index{Schonhage y Strassen, algoritmo de@Schönhage y Strassen, algoritmo de}
  el mejor algoritmo conocido para multiplicar que resulta práctico
  para números muy grandes~%
    \cite{schoehage71:_schnel_multip_zahlen},
  y es central en el algoritmo de Fürer~%
    \cite{fuerer07:_faster_integ_multip},
  el mejor que se conoce
  (aunque este último solo sería ventajoso
   para números fuera del rango útil)

  Otro ejemplo clásico es el algoritmo Quicksort%
    \index{ordenamiento!quicksort}%
    \index{quicksort|see{ordenamiento}}%
  (ver la sección~\ref{sec:quicksort}),
  claro que en este la división no es equitativa
  (como en los otros que se mencionan).
  También fue considerado
  uno de los \(10\) algoritmos más importantes del siglo~XX~%
    \cite{dongarra00:_top10_algorithms}.

\subsection{Análisis de división fija}
\label{sec:d&c:division-fija}

  Consideraremos primero el caso en que el problema original
  se traduce en varios problemas
  de una fracción fija del tamaño del original.
  Si el tiempo de ejecución de un algoritmo de este tipo
  para una entrada de tamaño \(n\)
  lo denotamos por \(t(n)\),
  el problema se reduce a \(a\) problemas de tamaño \(n / b\),
  y el costo de reducir el problema y luego combinar las soluciones
  es \(f(n)\),
  al sumar el tiempo para resolver los subproblemas
  y las otras operaciones
  obtendremos recurrencias de la forma:
    \index{dividir y conquistar!recurrencia}
  \begin{equation*}
    t(n) = a \, t(n / b) + f(n) \qquad t(1) = t_1
  \end{equation*}
  El restringir el análisis a potencias de \(b\)
  es válido ya que interesa el comportamiento asintótico
  de la solución a la recurrencia.
  Intuitivamente es claro que los algoritmos considerados
  se ejecutan en un tiempo intermedio para tamaños intermedios,
  y en cualquier caso podemos ``rellenar''
  los datos hasta completar la potencia respectiva.
  Hacer esto no cambia nuestras conclusiones más abajo.

  En el caso de ordenamiento por intercalación,%
    \index{ordenamiento!intercalacion@intercalación}
  dividimos en dos partes iguales
  que se procesan recursivamente.
  El proceso de dividir
  puede implementarse vía tomar elementos alternativos
  y ubicarlos en grupos separados,
  el combinar las partes ordenadas
  toma tiempo proporcional a su tamaño.
  Por lo tanto,
  el crear los subproblemas y combinar sus soluciones
  toma un tiempo proporcional al número de elementos a ordenar.
  Así tenemos que
  \(a = b = 2\), \(f(n) = c n\) para alguna constante \(c\).
  Para búsqueda binaria,%
    \index{busqueda binaria@búsqueda binaria}
  se divide en dos partes iguales
  de las cuales se procesa recursivamente solo una,
  y el proceso de división es simplemente ubicar el elemento medio
  y comparar con él,
  y no hay combinación de subproblemas;
  todo esto toma un tiempo constante.
  En este caso es
  \(a = 1\), \(b = 2\), \(f(n) = c\) para alguna constante.
  En el algoritmo de Karatsuba%
    \index{Karatsuba, algoritmo de}
  se transforma la multiplicación de dos números de largo \(2 n\)
  en \(3\) multiplicaciones de números de \(n\) dígitos,
  las tareas adicionales son dividir los números en mitades
  y efectuar varias sumas y restas de números de \(2 n\) dígitos,
  y finalmente juntar las piezas.
  El costo de estas operaciones es simplemente proporcional a \(n\).
  Resulta \(a = 3\), \(b = 2\) y \(f(n) = c n\).
  En la multiplicación de matrices de Strassen%
    \index{Strassen, algoritmo de}
  el multiplicar matrices de \(2 n \times 2 n\)
  se traduce en \(7\) multiplicaciones de matrices de \(n \times n\)
  y algunas sumas de matrices.
  Tenemos entonces \(a = 7\),
  \(b = 2\),
  y las operaciones adicionales son básicamente sumas de matrices,
  lo que da \(f(n) = c n^2\).
  Para cubrir el patio de \(2^n \times 2^n\)
  de la Universidad de Miskatonic con losas en L%
    \index{pavimentacion@pavimentación}
  que vimos al discutir inducción fuerte
  (sección~\ref{sec:induccion-fuerte}),
  la demostración que dimos reduce el problema de \(2^n \times 2^n\)
  a 4~problemas de \(2^{n - 1} \times 2^{n - 1}\)
  haciendo una cantidad fija de trabajo,
  lo que hace \(a = 4\), \(b = 2\), \(d = 0\).
  Si aprovechamos simetrías
  con el cuadradito a cubrir siempre en una esquina,
  es un solo trabajo menor,
  con lo que \(a = 1\), \(b = 2\), \(d = 0\).
  En caso que la posición de August es arbitraria,
  hay 2~tipos de subproblemas
  (uno con el espacio libre en la esquina,
   el otro con el espacio para August en una posición arbitraria),
  y \(a = 2\), \(b = 2\), \(d = 0\).

  Estos ejemplos son bastante representativos.
  El análisis es simple si \(f(n) = c n^d\).
  Para búsqueda binaria tenemos \(d = 0\),
  para ordenamiento por intercalación%
    \index{ordenamiento!intercalacion@intercalación}
  y en Karatsuba \(d = 1\),
  Strassen da \(d = 2\).
  El cuadro~\ref{tab:dividir-conquistar} resume los parámetros
  para los algoritmos dados.

  Consideramos entonces la recurrencia,
  válida para \(n\) una potencia de \(b\):
  \begin{equation*}
    t(b n) = a t(n) + c n^d \qquad t(1) = t_1
  \end{equation*}

  Efectuamos el cambio de variables:
  \begin{equation*}
    \begin{array}{l@{\quad}l}
      n	   = b^k  & k	   = \log_b n \\
      t(n) = T(k) & t(b n) = T(k + 1)
    \end{array}
  \end{equation*}
  En estos términos,
  dadas las condiciones del problema
  para constantes \(c > 0\)
  (el costo de dividir y combinar no es nulo)
  y \(t_1 > 0\)
  (el resolver un problema de tamaño mínimo tiene algún costo)
  tenemos para \(k \ge 0\):
  \begin{equation*}
    T(k + 1) = a T(k) + c (b^d)^k \qquad T(0) = t_1
  \end{equation*}
  Para resolver la recurrencia
  definimos la función generatriz:
  \begin{equation*}
    g(z) = \sum_{k \ge 0} T(k) z^k
  \end{equation*}
  y aplicamos nuestra técnica a la recurrencia lineal resultante:
  \begin{align*}
    \frac{g(z) - t_1}{z}
      &= g(z) + c \, \frac{1}{1 - b^d z} \\
    g(z)
      &= \frac{t_1 - (b^d t_1 - c) z}{(1 - b^d z) (1 - a z)}
  \end{align*}
  Si \(a \ne b^d\):
  \begin{equation}
    \label{eq:DaC-ne}
    g(z)
      = \frac{c}{b^d - a} \cdot \frac{1}{1 - b^d z}
	  + \frac{(b^d - a) t_1 - c}{b^d - a}
	      \cdot \frac{1}{1 - a z}
  \end{equation}
  Cuando \(a = b^d\):
  \begin{equation}
    \label{eq:DaC-eq}
    g(z)
      = \frac{c}{a} \cdot \frac{1}{(1 - a z)^2}
	  + \frac{a t_1 - c}{a} \cdot \frac{1}{1 - a z}
  \end{equation}
  El comportamiento asintótico
  queda determinado por \(a\) y \(b^d\).
  Si \(a > b^d\),
  domina el segundo término de~\eqref{eq:DaC-ne}:
  \begin{equation}
    \label{eq:DaC-gt-asymp}
    T(k)
      \sim \left( t_1 + \frac{c}{a - b^d} \right) \cdot a^k
  \end{equation}
  Si \(a < b^d\),
  es el primer término de~\eqref{eq:DaC-ne} el dominante:
  \begin{equation}
    \label{eq:DaC-lt-asymp}
    T(k)
      \sim \frac{c}{b^d - a} \cdot b^{k d}
  \end{equation}
  En caso que \(a = b^d\) debemos recurrir a~\eqref{eq:DaC-eq},
  y es dominante el primer término:
  \begin{equation}
    \label{eq:DaC-eq-asymp}
    T(k)
      \sim \frac{c}{a} \cdot k a^k
  \end{equation}
  Las constantes indicadas son siempre diferentes de cero.

  En términos de las variables originales,
  es \(k = \log_b n\)
  y \(a^k = a^{\log_b n}
	  = b^{\log_b a \cdot \log_b n}
	  = n^{\log_b a}\):%
    \index{analisis de algoritmos@análisis de algoritmos!dividir y conquistar!asintotica@asintótica}
  \begin{equation*}
    t(n)
      \sim
      \begin{cases}
	\left( t_1 + \frac{c}{a - b^d} \right) \cdot n^{\log_b a}
	   & \text{si \(a > b^d\)} \\
	\frac{c}{a} \, n^{\log_b a} \log n)
	   & \text{si \(a = b^d\)} \\
	\frac{c}{b^d - a} \cdot n^d
	   & \text{si \(a < b^d\)}
      \end{cases}
  \end{equation*}
  Para los algoritmos que describimos
  tenemos las complejidades
  resumidas en el cuadro~\ref{tab:dividir-conquistar}.%
    \index{analisis de algoritmos@análisis de algoritmos!dividir y conquistar}
  \begin{table}[htbp]
    \centering
    \begin{tabular}{|l|*{3}{>{\(}r<{\)}|}>{\(}l<{\)}|}
      \hline
      \multicolumn{1}{|c|}
	 {\rule[-0.7ex]{0pt}{3ex}\textbf{Nombre}} &
	\multicolumn{1}{c|}{\(\boldsymbol{a}\)} &
	\multicolumn{1}{c|}{\(\boldsymbol{b}\)} &
	\multicolumn{1}{c|}{\(\boldsymbol{d}\)} &
	\multicolumn{1}{c|}{\textbf{Complejidad}} \\
      \hline\rule[-0.7ex]{0pt}{3.5ex}%
      Búsqueda binaria		     & 1 & 2 & 0
	& O(\log n)	      \\
      \hline\rule[-0.7ex]{0pt}{3.5ex}%
      Ordenamiento por intercalación & 2 & 2 & 1
	& O(n \log n)	      \\
      \hline\rule[-0.7ex]{0pt}{3.5ex}%
      Karatsuba			     & 3 & 2 & 1
	& O \left( n^{\log_2 3} \right) \\
      \hline\rule[-0.7ex]{0pt}{3.5ex}%
      \multirow{3}*{Pavimentación}   & 4 & 2 & 0
	& O \left( n^2 \right) \\
				     & 2 & 2 & 0
	& O( n) \\
				     & 1 & 2 & 0
	& O( n) \\
      \hline\rule[-0.9ex]{0pt}{3.5ex}%
      Strassen			     & 7 & 2 & 2
	& O \left( n^{\log_2 7} \right) \\
      \hline
    \end{tabular}
    \caption{Complejidad de algunos algoritmos}
    \label{tab:dividir-conquistar}
  \end{table}

  Este tipo de recurrencias puede resolverse exactamente.
  Por ejemplo,
  Sedgewick y Flajolet~\cite{sedgewick13:_introd_anal_algor}
  muestran que la solución a la recurrencia
  para el número de comparaciones
  en mergesort de \(n\) elementos diferentes:
    \index{ordenamiento!intercalacion@intercalación!analisis@análisis}
  \begin{equation}
    \label{eq:mergesort-compares-exact}
    C_n
      = C_{\lfloor n / 2 \rfloor} + C_{\lceil n / 2 \rceil} + n
    \qquad C_1 = 0
  \end{equation}
  es:
  \begin{equation}
    \label{eq:mergesort-compares-exact-solution}
    C_n
      = n \log_2 n + n \theta(1 - \{ \log_2 n \})
  \end{equation}
  donde:
  \begin{equation}
    \label{eq:mergesort-theta}
    \theta(x)
      = 1 + x - 2^x
  \end{equation}
  Resulta \(\theta(0) = \theta(1) = 0\)
  y \(0 < \theta(x) < 0,086\)
  para \(0 < x < 1\).
  Esta clase de comportamiento ``periódico''
  complica el análisis preciso de muchos algoritmos.

  Un desarrollo didáctico de resultados de este tipo
  se encuentra en el texto de Stein, Drysdale y Bogarth~%
    \cite[apéndice A]{stein10:_discr_mathem_comput_scien}.
  Una visión alternativa,
  incluyendo técnicas
  para acotar el caso de funciones forzantes diferentes,
  dan Bentley, Haken y Saxe~%
    \cite{bentley80:_gener_method_solvin_divid_conquer_recur}.
  Una extensión a estos resultados
  es el teorema de Akra-Bazzi~\cite{akra98:_sol_lin_recurr_eqs}.
  Leighton~\cite{leighton96:_notes_better_master_theo}
  da la variante que reseñamos,
  extensiones interesantes da Roura~%
     \cite{roura01:_improv_master_theor_divid_conquer_recur},
  soluciones más precisas para versiones discretas
  (con techos/pisos)
  ofrecen Drmota y Szpankowski~%
     \cite{Drmota:2011:MTD:2133036.2133064}.
  \begin{theorem}[Akra-Bazzi]
    \index{Akra-Bazzi, teorema de|textbfhy}
    \label{theo:Akra-Bazzi}
    Sea una recurrencia de la forma:
    \begin{equation*}
      T(z)
	= g(z) + \sum_{1 \le k \le n} a_k T(b_k z + h_k(z))
	   \quad \text{para \(z \ge z_0\)}
    \end{equation*}
    donde \(z_0\), \(a_k\) y \(b_k\) son constantes,
    sujeta a las siguientes condiciones:
    \begin{itemize}
    \item
      Hay suficientes casos base.
    \item
      Para todo \(k\) se cumplen \(a_k > 0\) y \(0 < b_k < 1\).
    \item
      Hay una constante \(c\)
      tal que \(\lvert g(z) \rvert = O(z^c)\).
    \item
      Para todo \(k\)
      se cumple \(\lvert h_k(z) \rvert = O(z / (\log z)^2)\).
    \end{itemize}
    Entonces,
    si \(p\) es tal que:
    \begin{equation*}
      \sum_{1 \le k \le n} a_k b_k^p
	= 1
    \end{equation*}
    la solución a la recurrencia cumple:
    \begin{equation*}
      T(z)
	= \Theta
	    \left(
	      z^p \left(
		     1 + \int_1^z \frac{g(u)}{u^{p + 1}}
			   \, \mathrm{d} u
		  \right)
	    \right)
    \end{equation*}
  \end{theorem}
  Frente a nuestro tratamiento tiene la ventaja
  de manejar divisiones desiguales
  (\(b_k\) diferentes),
  y explícitamente
  considera pequeñas perturbaciones en los términos,
  como lo son aplicar pisos o techos,
  a través de los \(h_k(z)\).
  Diferencias con pisos y techos están acotados por una constante,
  mientras la cota del teorema permite que crezcan.
  Por ejemplo,
  la recurrencia correcta para el número de comparaciones
  en ordenamiento por intercalación es:
  \begin{equation*}
    T(n)
      = T(\lfloor n / 2 \rfloor) + T(\lceil n / 2 \rceil) + n - 1
  \end{equation*}
  El teorema de Akra-Bazzi es aplicable.
  La recurrencia es:
  \begin{equation*}
    T(n)
      = T(n / 2 + h_{+}(n)) + T(n / 2 + h_{-}(n)) + n - 1
  \end{equation*}
  Acá \(\lvert h_{\pm}(n) \rvert \le 1/2\),
  además \(a_{\pm} = 1\) y \(b_{\pm} = 1/2\).
  Estos cumplen las condiciones del teorema,
  de:
  \begin{equation*}
    \sum_{1 \le k \le 2} a_k b_k^p = 1
  \end{equation*}
  resulta \(p = 1\),
  y tenemos la cota:
  \begin{equation*}
    T(z)
      = \Theta
	  \left(
	    z \left(
	  1 + \int_1^z \frac{u - 1}{u^2} \, \mathrm{d} u
	      \right)
	  \right)
      = \Theta
	  \left(
	    z \ln z + 1
	  \right)
      = \Theta(z \log z)
  \end{equation*}

  Otro ejemplo son los árboles de búsqueda aleatorizados
  (\emph{\foreignlanguage{english}{Randomized Search Trees}},
   ver por ejemplo Aragon y Seidel~%
     \cite{aragon89:_random_search_tree},
   Martínez y Roura~%
     \cite{martinez98:_random_binar_searc_trees}
   y Seidel y Aragon~%
     \cite{seidel96:_random_search_trees})
  en uno de ellos de tamaño~\(n\)
  una búsqueda toma tiempo aproximado:
  \begin{equation*}
    T(n)
      = \frac{1}{4} \, T(n / 4) + \frac{3}{4} \, T(3 n / 4) + 1
  \end{equation*}
  Nuevamente es aplicable el teorema~\ref{theo:Akra-Bazzi},
  de:
  \begin{equation*}
    \frac{1}{4} \, \left(\frac{1}{4}\right)^p
	+ \frac{3}{4} \left(\frac{3}{4}\right)^p
      = 1
  \end{equation*}
  obtenemos \(p = 0\),
  y por tanto la cota
  \begin{equation*}
    T(z)
      = \Theta \left(
	  z^0 \left( 1 + \int_1^z \frac{\mathrm{d} u}{u} \right)
	\right)
      = \Theta ( \log z )
  \end{equation*}

% quicksort.tex
%
% Copyright (c) 2010-2014 Horst H. von Brand
% Derechos reservados. Vea COPYRIGHT para detalles

\subsection{Quicksort}
\label{sec:quicksort}
\index{ordenamiento!quicksort|textbfhy}
\index{analisis de algoritmos@análisis de algoritmos!ordenamiento!quicksort}

  Quicksort,
  debido a Hoare~\cite{hoare62:_quicksort},
  es otro algoritmo basado en dividir y conquistar,
  pero en este caso la división no es fija.
  Dado un rango de elementos de un arreglo a ser ordenado,
  se elige un elemento \emph{pivote} de entre ellos
  y se reorganizan los elementos en el rango
  de forma que todos los elementos menores que el pivote
  queden antes de este,
  y todos los elementos mayores queden después.
  \begin{figure}[htbp]
    \centering
    \pgfimage{images/qsort-idea}
    \caption{Idea de Quicksort}
    \label{fig:qsort:idea}
  \end{figure}
  Con esto el pivote ocupa su posición final en el arreglo,
  y bastará ordenar recursivamente cada uno
  de los dos nuevos rangos generados
  para completar el trabajo.
  \begin{figure}[htbp]
    \centering
    \pgfimage{images/qsort-particionar}
    \caption{Particionamiento en Quicksort}
    \label{fig:qsort:particionamiento}
  \end{figure}
  La figura~\ref{fig:qsort:particionamiento} indica una manera popular
  de efectuar esta \emph{partición}:
  Se elige un pivote de forma aleatoria
  y el pivote elegido se intercambia con el primer elemento del rango
  (para sacarlo de en medio),
  luego se busca un elemento mayor que el pivote desde la izquierda
  y uno menor desde la derecha.
  Estos están fuera de orden,
  se intercambian y se continúa de la misma forma hasta agotar el rango.
  Después se repone el pivote en su lugar,
  intercambiándolo con el último elemento menor que él.
  El rango finalmente queda como indica la figura~\ref{fig:qsort:idea}.
  El listado~\ref{lst:quicksort} muestra una versión simple del programa,
  que elige siempre el primer elemento del rango como pivote.
  \lstinputlisting[language=C,
		   float,
		   caption={Versión simple de Quicksort},
		   label=lst:quicksort]
		   {code/quicksort.c}
% LaTeX bug: Leaving out the "float, ..." line out gives a crash
  Evaluaremos el tiempo promedio de ejecución del algoritmo.%
    \index{quicksort!analisis@análisis}
  Supondremos~\(n\) elementos todos diferentes,
  que las \(n!\)~permutaciones de los \(n\)~elementos
  son igualmente probables,
  y que el pivote se elige al azar en cada etapa.
  En este caso está claro que el método de particionamiento planteado
  no altera el orden de los elementos en las particiones
  respecto del orden que tenían originalmente.
  Luego,
  los elementos en cada partición también son una permutación al azar.

  Para efectos del análisis del algoritmo
  tomaremos como medida de costo el número promedio de comparaciones
  que efectúa Quicksort al ordenar un arreglo de \(n\)~elementos.
  El trabajo adicional que se hace en cada partición
  será aproximadamente proporcional a esto,
  por lo que esta es una buena vara de medida.
  Al particionar,
  cada uno de los \(n - 1\) elementos fuera del pivote
  se comparan con este exactamente una vez en el método planteado,
  y además es obvio que este es el mínimo número de comparaciones necesario
  para hacer este trabajo.
  Si llamamos \(k\) a la posición final del pivote,
  el costo de las llamadas recursivas que completan el ordenamiento
  será \(C(k - 1) + C(n - k)\).
  Si elegimos el pivote al azar
  la probabilidad de que \(k\) tenga un valor cualquiera entre~\(1\)
  y~\(n\) es la misma.
  Cuando el rango es vacío no se efectúan comparaciones.
  Estas consideraciones llevan a la recurrencia:
  \begin{equation*}
    C(n)
       =  n - 1 +
	   \frac{1}{n} \, \sum_{1 \le k \le n - 1}
	      \left(C(k - 1) + C(n - k)\right) \quad C(0)  = 0
  \end{equation*}
  Por simetría podemos simplificar la suma,
  dado que estamos sumando los mismos términos
  en orden creciente y decreciente.
  Cambiando el rango de la suma y multiplicando por \(n\) queda:
  \begin{equation*}
    n C(n)
      = n (n - 1) + 2 \sum_{0 \le k \le n - 1} C(k)
  \end{equation*}
  Ajustando los índices:
  \begin{equation*}
    (n + 1) C(n + 1)
      = n (n + 1) + 2 \sum_{0 \le k \le n} C(k) \quad C(0) = 0
  \end{equation*}

  Definimos la función generatriz ordinaria:
  \begin{equation*}
    c(z)
      = \sum_{n \ge 0} C(n) z^n
  \end{equation*}
  Aplicando las propiedades de funciones generatrices ordinarias
  a la recurrencia
  queda la ecuación diferencial:
  \begin{align*}
    \left( z \mathrm{D} + 1 \right) \, \frac{c(z)}{z}
      &= \left( (z \mathrm{D})^2 + z D \right) \, \frac{1}{1 - z}
	   + \frac{2 c(z)}{1 - z}
	   \qquad c(0) = 0 \\
    c'(z)
      &= \frac{2 c(z)}{1 - z} + \frac{2 z}{(1 - z)^3}
  \end{align*}
  La solución a esta ecuación es:
  \begin{equation*}
    c(z)
      = - 2 \, \frac{\ln (1 - z)}{(1 - z)^2}
	   - \frac{2 z}{(1 - z)^2}
  \end{equation*}
  El primer término corresponde
  a la suma parcial de la secuencia de números harmónicos
  (derivamos su función generatriz
   en la sección~\ref{sec:numeros-harmonicos}),
  el segundo término da un coeficiente binomial:
  \begin{align*}
    C(n)
      &= 2 \sum_{0 \le k \le n} H_k - 2 \, \binom{n}{1} \\
      &= 2 \sum_{0 \le k \le n} H_k - 2 n
  \end{align*}
  Interesa obtener una fórmula más simple
  para la suma de los números harmónicos.
  Por la fórmula para la función generatriz de las sumas parciales:
  \begin{equation*}
    H(z)
      = \sum_{n \ge 0} H_n z^n
      = \frac{1}{1 - z} \, \ln \frac{1}{1 - z}
  \end{equation*}
  con lo que la función generatriz de las sumas de números harmónicos es:
  \begin{equation}
    \label{eq:Hn-sum-gf}
    \frac{H(z)}{1 - z}
      = \frac{1}{(1 - z)^2} \, \ln \frac{1}{1 - z}
  \end{equation}
  Esto se parece a la derivada de \(H(z)\):
  \begin{equation*}
    H'(z)
      = \frac{1}{(1 - z)^2} \, \ln \frac{1}{1 - z} + \frac{1}{(1 - z)^2}
  \end{equation*}
  Pero sabemos que \(H'(z)\) es la función generatriz
  de la secuencia \(\left\langle (n + 1) H_{n + 1} \right\rangle_{n \ge 0}\),
  mientras \((1 - z)^{-2}\) corresponde simplemente
  a \(\left\langle n + 1 \right\rangle_{n \ge 0}\).
  De esta forma tenemos:
  \begin{align}
    \sum_{0 \le k \le n} H_k
      &= (n + 1) H_{n + 1} - (n + 1)
	   \notag \\
      &= (n + 1) H_n - n
	   \label{eq:Hn-sum}
  \end{align}
  Esto da finalmente:
  \begin{equation*}
    C(n)
      = 2 (n + 1) H_n - 4 n
  \end{equation*}
  Vimos en el capítulo~\ref{cha:euler-maclaurin}%
    \index{numeros harmonicos@números harmónicos!aproximacion@aproximación}
  que \(H_n = \ln n + O(1)\),
  con lo que \(C(n) = 2 n \ln n + O(n)\).

  Pero podemos hacer más.
  En el peor caso,%
    \index{quicksort!analisis@análisis!peor caso}
  al particionar en cada paso elegimos uno de los elementos extremos,
  con lo que las particiones son de largo 0 y \(n - 1\),
  lo que da lugar a la recurrencia:
  \begin{equation*}
    C_{\text{peor}}(n)
      = n - 1 + C_{\text{peor}}(n - 1) \quad C_{\text{peor}}(0) = 0
  \end{equation*}
  Las técnicas estándar dan como solución:
  \begin{align*}
    C_{\text{peor}}(n)
      &= \frac{n (n - 1)}{2} \\
      &= \frac{1}{2} n^2 + O(n)
  \end{align*}
  El mejor caso es cuando en cada paso la división es equitativa,
  lo que lleva casi a la situación de dividir y conquistar
  analizada antes
  (sección~\ref{sec:d&c:division-fija}),
  con \(a = 2\), \(b = 2\) y \(d = 1\),
  cuya solución sabemos es \(C_{\text{mejor}}(n) = O(n \log n)\).
  Un análisis más detallado,
  restringido al caso en que \(n = 2^k - 1\)
  de manera que los dos rangos siempre resulten del mismo largo,
  es como sigue.%
    \index{quicksort!analisis@análisis!mejor caso}
  La recurrencia original se reduce a:
  \begin{equation*}
    C_{\text{mejor}}(n)
      = n - 1 + 2 C_{\text{mejor}}((n - 1) / 2)
      \quad C_{\text{mejor}}(0) = 0
  \end{equation*}
  Con el cambio de variables:
  \begin{equation*}
    n = 2^k - 1 \quad F(k) = C_{\text{mejor}}(2^k - 1)
  \end{equation*}
  esto se transforma en:
  \begin{equation*}
    F(k)
      = 2^k - 2 + 2 F(k - 1) \quad F(0) = 0
  \end{equation*}
  cuya solución es:
  \begin{align*}
    F(k)
      &= k 2^k + 2^{k + 1} + 2 \\
    C_{\text{mejor}}(n)
      &= (n + 1) \log_2 (n + 1) + 2 (n + 1) + 2 \\
      &= \frac{1}{\ln 2} \, n \ln n + O(n)
  \end{align*}
  La constante en este caso es aproximadamente \(1,443\),
  el mejor caso no es demasiado mejor que el promedio;
  pero el peor caso es mucho peor.

  Una variante común
  es usar un método de ordenamiento simple para rangos chicos.
  Una opción es cortar la recursión
  no cuando el rango se reduce a un único elemento
  sino cuando cae bajo un cierto margen;
  y luego se ordena todo mediante inserción,
  que funciona muy bien si los datos vienen ``casi ordenados'',
  como resulta de lo anterior.
  Para analizar esto se requieren medidas más ajustadas
  del costo de los métodos,
  y se cambian las condiciones de forma que para valores de \(n\)
  menor que el límite se usa el costo del método alternativo.
  Esto puede hacerse,
  pero es bastante engorroso y no lo veremos acá.

  Para evitar el peor caso
  (que se da cuando el pivote es uno de los elementos extremos)
  una opción es tomar una muestra de elementos y usar la mediana
  (el elemento del medio de la muestra)
  como pivote.
  La forma más simple de hacer esto es tomar tres elementos.
  Como además es frecuente que se invoque el procedimiento con un arreglo
  ``casi ordenado''
  (o incluso ya ordenado),
  conviene tomar como muestra el primero,
  el último
  y un elemento del centro,
  de forma de elegir un buen pivote incluso en ese caso patológico.
  A esta idea se le conoce como \emph{mediana de tres}.
  Esta estrategia disminuye un tanto la constante
  por efecto de una división más equitativa.
  Tiene la ventaja adicional
  que tener elementos menor que el pivote al comienzo del rango
  y mayor al final
  no es necesario comparar índices para determinar
  si se llegó al borde del rango.
  El análisis detallado se encuentra por ejemplo en Sedgewick y Flajolet~%
    \cite{sedgewick13:_introd_anal_algor}.

  Por el otro lado,
  McIllroy~\cite{mcillroy99:_killer_adver_quicksort}
  muestra cómo lograr que siempre tome el máximo tiempo posible.
  Quicksort
  (haciendo honor a su nombre)
  es muy rápido
  ya que las operaciones en sus ciclos internos
  implican únicamente una comparación
  y un incremento o decremento de un índice.
  Es ampliamente usado,
  y como su peor caso es muy malo,
  vale la pena hacer un estudio detallado de la ingeniería del algoritmo,
  como hacen Bentley y McIllroy~%
    \cite{bentley93:_engin_sort_funct}.
  Debe tenerse cuidado con Quicksort por su peor caso,
  si un atacante puede determinar los datos
  puede hacer que el algoritmo consuma muchísimos recursos.
  Para evitar el peor caso se ha propuesto cambiar a Heapsort,%
    \index{ordenamiento!heapsort}
    debido a Williams~%
    \cite{williams64:_alg_heapsort}
  (garantizadamente \(O(n \log n)\),
   pero mucho más lento que Quicksort)
  si se detecta un caso malo,
  como propone Musser~\cite{musser97:_introsort}.

%%% Local Variables:
%%% mode: latex
%%% TeX-master: "clases"
%%% End:


%%% Local Variables:
%%% mode: latex
%%% TeX-master: "clases"
%%% End:

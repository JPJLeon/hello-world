% sigil.tex
%
% Copyright (c) 2013-2014 Horst H. von Brand
% Derechos reservados. Vea COPYRIGHT para detalles

\documentclass[spanish, english, greek]{article}
% Fonts
\usepackage{fourier}
\usepackage[utf8]{inputenc}
% Languages
\usepackage{babel}
\usepackage{babelbib}
% General LaTeXe
\usepackage{pgf}
\usepackage{fixltx2e}
% microtype should come after fonts and languages, and most packages
\usepackage[babel=true, stretch=10]{microtype}
% Hyperref (PDF output settings), should be last
\usepackage[ocgcolorlinks]{hyperref}

%%%
%%% PDF information settings
%%%

\hypersetup{pdftitle = {Sigil},
	    pdfauthor = {Horst H. von Brand},
	    unicode, pdfdisplaydoctitle,
	    urlcolor=blue
	   }

\title{Sobre el sello oficial de\\
       Fundamentos de Informática}
\author{\href{mailro:vonbrand@inf.utfsm.cl}{Horst H. von Brand}}

\begin{document}
\selectlanguage{spanish}
\bibliographystyle{plain}

\maketitle
\thispagestyle{empty}

  \begin{figure}[ht]
    \centering
    \pgfimage[width=0.4\textwidth]{images/necronomicon}
    \caption{El sello de Yog-Sothoth}
    \label{fig:sigil}
  \end{figure}
  Primero,
  unas pinceladas de (pre)historia,
  en su mayor parte de \cite{SDSAB:_necronomicon_real},
  \cite{Wikipedia14:_necronomicon}
  y~\cite{gilmore03:_necronomicon_facts}.
  El escritor estadounidense Howard Phillip Lovecraft
  (1890--1937)
  es reconocido como maestro de lo macabro.
  En \selectlanguage{english}{``The Hound''} (1922)
  menciona por primera vez
  un ``libro terrible y prohibido'',
  el Necronomicon,
  texto general de demonología,
  lo oculto y magia negra.
  Es tal el poder del libro que solo estudiar su contenido
  suele tener consecuencias espantosas.
  A su supuesto autor,
  el árabe loco Abdul Alhazred,
  ya lo había mencionado en \selectlanguage{english}{``The Nameless City''}
  (1921).
  Con el cuento \selectlanguage{english}{``The Call of Cthulhu''}
  (1926)
  comenzó una serie de cuentos
  que incorporan mitología basada en seres sobrenaturales
  (los ``antiguos'',
   como Cthulhu o Yog-Sothoth)
  e incluyen textos míticos.
  Lovecraft se distingue por haber urdido una elaborada y original mitología,
  que fue a su vez adoptada por un grupo de autores
  conocidos o corresponsales de Lovecraft.
  Este círculo tuvo la costumbre de inventar textos míticos,
  y
  (para darles verosimilitud)
  citarlos junto con textos reales,
  e incluso inventar historias de variadas ediciones de tales textos.
  Por ejemplo,
  el mismo Lovecraft escribió una historia del texto %
    \cite{lovecraft38:_history_necronomicon},
  donde reseña que su autor fue el poeta yemení demente Abdul Alhazred,
  quien lo transcribió en Damasco alrededor del año~700,
  bajo el nombre de \emph{Al Azif}
  (en árabe,
   \emph{azif} es el sonido nocturno de insectos que se supone es el aullido
   de los demonios).
  Según su biógrafo Ebn Khallikan,
  en recompensa por sus esfuerzos fue devorado públicamente
  por un monstruo invisible el año~738.
  El texto fue traducido al griego por Theodorus Philetas el año 950,
  adquiriendo el nombre.

  Según Lovecraft,
  el nombre del libro viene del griego
  \textgreek{νεκρός}
  (cadáver),
  \textgreek{νόμος}
  (ley)
  e \textgreek{εἰκών}
  (imagen),
  con lo que significaría algo como ``La imagen de la ley de los muertos'',
  pero esto es inconsistente con la gramática griega clásica
  (el sufijo \textgreek{-ἰκών} es adjetivo neutro,
   no tiene relación con \textgreek{εἰκών}).
  Asimismo,
  el nombre del supuesto autor es árabe incorrecto,
  el \emph{-ul} de Abdul es redundante con \emph{al} de Alhazred,
  correcto sería Abd el Hazred
  (claro que no suena tan bien).
  Por lo demás,
  Abd es \emph{servidor} o \emph{adorador},
  mientras Alhazred hace referencia a su lugar de origen,
  con lo que el nombre del autor no tiene sentido.
  Lovecraft dijo que el nombre lo habría sacado de las 1000 y una noches,
  o que fue sugerido por un abogado amigo de la familia.

  En 1977,
  un escritor anónimo aficionado a lo oculto
  que se identifica únicamente como Simon publicó una supuesta traducción
  del texto~\cite{simon77:_necronomicon},
  en cuya tapa figura prominentemente el sello de la figura~\ref{fig:sigil}.
  De muchos supuestas ediciones del texto esta es lejos la más popular.
  Se caracteriza además porque no reconoce que es falsa.
  Un análisis somero del texto de marras muestra
  que la mitología en que se basa es sumeria con añadidos,
  muy posterior a la historia que supuestamente narra el Necronomicon.
  Otra edición notable es la de Hay~\cite{hay78:_necronomicon},
  escrita a propósito como una elaborada inocentada.

  Volvamos a nuestro tema ahora.
  Poco después de comenzar a escribir el presente apunte
  para los ramos de Fundamentos de Informática
  ya los estudiantes consideraban que era la máxima expresión de lo maligno,
  en particular que su estudio provocaba serios males,
  y lo bautizaron ``el Necronomicón''.
  En busca de una imagen que representara cabalmente el mal que encierra
  es que se eligió este sello.

  En sus cuentos Lovecraft menciona las ciudades ficticias
  de \selectlanguage{english}{Arkham},
  \selectlanguage{english}{Dunwich} e \selectlanguage{english}{Innsmouth}.
  Supuestamente la única copia existente del texto se encuentra
  en la biblioteca de la también ficticia
  Universidad de \selectlanguage{english}{Miskatonic}
  en \selectlanguage{english}{Arkham}.
  Asimismo,
  hay personajes que se repiten en sus escritos.
  Donde es posible en el texto se usan lugares y nombres
  de la tradición de Lovecraft.

\bibliography{necronomicon}
\end{document}

%%% Local Variables:
%%% mode: latex
%%% TeX-master: t
%%% End:

% espacios-vectoriales.tex
%
% Copyright (c) 2012-2014 Horst H. von Brand
% Derechos reservados. Vea COPYRIGHT para detalles

\section{Espacios vectoriales}
\label{sec:espacios-vectoriales}
\index{espacio vectorial}

  Una estructura algebraica común es el espacio vectorial.
  Es aplicable a una gran variedad de situaciones,
  algunas bastante inesperadas.
  \begin{definition}
    Sea \(F\) un campo
    (sus elementos los llamaremos \emph{escalares})%
      \index{espacio vectorial!escalar|textbfhy}
    y \(V\) un conjunto
    (los \emph{vectores},%
      \index{espacio vectorial!vector|textbfhy}
     que por convención anotaremos en negrita).
    Hay operaciones \emph{suma de vectores}%
      \index{espacio vectorial!operaciones}
    (anotada \(+\))
    y \emph{producto escalar}
    entre un escalar y un vector
    (anotada \(\cdot\)).
    Se dice que \(V\) es un \emph{espacio vectorial sobre \(F\)}
    si cumple con los siguientes axiomas,%
      \index{espacio vectorial!axiomas}%
      \index{axioma!espacio vectorial}
    donde \(\alpha, \beta, \dotsc \in F\),
    y \(\boldsymbol{v}_1, \boldsymbol{v}_2, \dotsc \in V\).
    \begin{enumerate}[label=\textbf{V\arabic{*}:}, ref=V\arabic{*}]
    \item\label{ax:V:associative}
      \((\boldsymbol{v}_1 + \boldsymbol{v}_2) + \boldsymbol{v}_3
	   = \boldsymbol{v}_1
	       + (\boldsymbol{v}_2
	       + \boldsymbol{v}_3)\)
    \item\label{ax:V:neutral}
      Hay un elemento \(\boldsymbol{0} \in V\)
      tal que para todo \(\boldsymbol{v} \in V\)
      se cumple
      \(\boldsymbol{v} + \boldsymbol{0} = \boldsymbol{v}\)
    \item\label{ax:V:inverse}
      Para cada \(\boldsymbol{v} \in V\)
      hay \(- \boldsymbol{v} \in V\)
      tal que \(\boldsymbol{v} + (- \boldsymbol{v})
		  = \boldsymbol{0}\)
    \item\label{ax:V:commutative}
       \(\boldsymbol{v}_1 + \boldsymbol{v}_2
	   = \boldsymbol{v}_2 + \boldsymbol{v}_1\)
    \item\label{ax:V:scalar(vectorsum)}
      \(\alpha \cdot ( \boldsymbol{v}_1 + \boldsymbol{v}_2 )
	  = \alpha \cdot \boldsymbol{v}_1
	      + \alpha \cdot \boldsymbol{v}_2\)
    \item\label{ax:V:(scalarsum)vector}
      \((\alpha + \beta) \cdot \boldsymbol{v}
	  = \alpha \cdot \boldsymbol{v}
	      + \beta \cdot \boldsymbol{v}\)
    \item\label{ax:V:scalar-scalar-vector}
      \(\alpha \cdot (\beta \cdot \boldsymbol{v})
	  = (\alpha \beta) \cdot \boldsymbol{v}\)
     \item\label{ax:V:1-vector}
       Si \(1\) es el neutro multiplicativo de \(F\),
       \(1 \cdot \boldsymbol{v} = \boldsymbol{v}\)
    \end{enumerate}
  \end{definition}
  \noindent
  En resumen,
  \((V, +)\) es un grupo abeliano%
    \index{grupo!abeliano}
  (axiomas~\ref{ax:V:associative} a~\ref{ax:V:commutative}),
  junto con el campo \(F\) y multiplicación escalar que cumple
  los axiomas adicionales~\ref{ax:V:scalar(vectorsum)}
  a~\ref{ax:V:1-vector}.
  Normalmente indicaremos la multiplicación escalar
  por simple yuxtaposición.
  Dejamos de ejercicio
  demostrar que \(0 \cdot \boldsymbol{v} = \boldsymbol{0}\)
  y que \((- \alpha) \cdot \boldsymbol{v}
	    = - ( \alpha \cdot \boldsymbol{v})\).
  \begin{definition}
    Sea \(V\) un espacio vectorial sobre el campo \(F\).
    Si para el conjunto de vectores \(B\)
    es:
    \begin{equation*}
      \sum_{\boldsymbol{b} \in B}
	\alpha_{\boldsymbol{b}} \boldsymbol{b}
	= \boldsymbol{0}
    \end{equation*}
    solo si \(\alpha_{\boldsymbol{b}} = 0\)
    para todo \(\boldsymbol{b} \in B\)
    se dice que esos vectores
    son \emph{linealmente independientes}.%
      \index{espacio vectorial!independencia lineal}
  \end{definition}
  Si un conjunto de vectores no es linealmente independiente
  se dice que son \emph{linealmente dependientes}.
  Nótese que \(\boldsymbol{0}\)
  nunca pertenece
  a un conjunto de vectores linealmente independientes,
  ya que al multiplicarlo
  por cualquier escalar obtenemos \(\boldsymbol{0}\).
  \begin{definition}
    Sea \(V\) un espacio vectorial sobre \(F\),
    y \(B \subseteq V\) un conjunto de vectores.
    El \emph{espacio vectorial generado por \(B\)}
    es el conjunto:
    \begin{equation*}
      \langle B \rangle
	= \left\{
	    \sum_{\boldsymbol{b} \in B}
	      \alpha_{\boldsymbol{b}} \boldsymbol{b}
	       \colon \alpha_{\boldsymbol{b}} \in F
	  \right\}
    \end{equation*}
    Si \(V = \langle B \rangle\),
    se dice que \(B\) \emph{abarca} \(V\).
  \end{definition}
  En particular:
  \begin{definition}
    \index{espacio vectorial!base|textbfhy}
    Una \emph{base} del espacio vectorial \(V\)
    es un conjunto linealmente independiente de vectores \(B\)
    que abarca \(V\).
  \end{definition}
  La representación de \(\boldsymbol{v} \in V\)
  en términos de la base \(B\)
  es única,
  ya que si hubieran dos representaciones diferentes
  darían una dependencia lineal en \(B\).
  Para el vector:
  \begin{equation*}
    \boldsymbol{v}
      = \sum_{\boldsymbol{b} \in B}
	  \alpha_{\boldsymbol{b}} \boldsymbol{b}
  \end{equation*}
  a los coeficientes \(\alpha_{\boldsymbol{b}}\)
  se les llama \emph{componentes} de \(\boldsymbol{v}\)%
    \index{espacio vectorial!componentes (de un vector)|textbfhy}
  (en la base \(B\)).
  \begin{definition}
    \index{espacio vectorial!dimension@dimensión|textbfhy}
    Al número de vectores en una base de \(V\)
    se le llama la \emph{dimensión} de \(V\),
    anotada \(\dim V\).
    Un espacio vectorial abarcado por un conjunto finito de vectores
    se dice de \emph{dimensión finita},
    en caso contrario es de \emph{dimensión infinita}.
    Al espacio vectorial \(\{\boldsymbol{0}\}\)
    se le asigna dimensión cero.
    Se anota \([V : F]\)
    para la dimensión de \(V\) sobre el campo \(F\).
  \end{definition}
  En el caso de espacios vectoriales de dimensión finita
  es simple demostrar que todas las bases
  tienen la misma cardinalidad,
  con lo que nuestra definición de dimensión tiene sentido.
  \begin{theorem}
    \label{theo:espacio-vectorial-li}
    Si \(V\) es un espacio vectorial
    con base
      \(B = \{\boldsymbol{b}_1, \boldsymbol{b}_2,
	       \dotsc, \boldsymbol{b}_n\}\),
    y \(A = \{\boldsymbol{a}_1, \boldsymbol{a}_2,
	       \dotsc, \boldsymbol{a}_r\}\)
    es un conjunto linealmente independiente de vectores en \(V\),
    entonces \(r \le n\).
  \end{theorem}
  \begin{proof}
    Como \(B\) abarca \(V\),
    \(B \cup \{\boldsymbol{a}_1\}\) también abarca \(V\).
    Como \(\boldsymbol{a}_1 \ne \boldsymbol{0}\)
    (\(A\) es linealmente independiente),
    podemos expresar \(\boldsymbol{a}_1\)
    como combinación lineal de los \(B\),
    y en ella algún \(\boldsymbol{b}_t\)
    tendrá coeficiente diferente de 0.
    Ese \(\boldsymbol{b}_t\) puede expresarse en términos
    de \(B_1 = \{\boldsymbol{a}_1, \boldsymbol{b}_1,
		   \boldsymbol{b}_2,
		   \dotsc, \boldsymbol{b}_{t - 1},
		   \boldsymbol{b}_{t + 1},
		 \dotsc, \boldsymbol{b}_n\}\).
    Como todo \(\boldsymbol{v} \in V\)
    puede escribirse como combinación lineal
    de los \(B\),
    también puede escribirse como combinación lineal de los \(B_1\)
    (substituyendo la combinación
     de \(B_1\) que da \(\boldsymbol{b}_t\)
     en la combinación lineal para \(\boldsymbol{v}\)
     se obtiene una nueva combinación lineal).
    Este proceso puede repetirse
    intercambiando un \(A\) por uno de los \(B\),
    manteniendo siempre \(B_k\) como base,
    finalmente llegando
    a \(B_r = \{\boldsymbol{a}_1, \dotsc, \boldsymbol{a}_r,
		\boldsymbol{b}_{m_1}, \boldsymbol{b}_{m_2},
		\dotsc, \boldsymbol{b}_{m_s}\}\)
    (posiblemente no queden \(\boldsymbol{b}_{m_k}\) en \(B_r\)).
    No pueden quedar \(A\) si se acaban los \(B\),
    ya que si fuera así un \(\boldsymbol{a}_i\) sobrante
    no podría representarse como combinación lineal
    de los \(B\),
    y \(B\) no sería una base.
    Tenemos \(A \subseteq B_r\),
    y claramente
      \(\lvert A \rvert \le \lvert B_r \rvert = \lvert B \rvert\).
  \end{proof}
  Esto justifica la definición de la dimensión
  en el caso de espacios vectoriales de dimensión finita:
  \begin{corollary}[Teorema de dimensión de espacios vectoriales]
    \label{cor:espacio-vectorial-dimension}
    Si \(A\) y \(B\) son bases
    de un espacio vectorial de dimensión finita,
    entonces \(\lvert A \rvert = \lvert B \rvert\).
  \end{corollary}
  \begin{proof}
    La base \(A\) es linealmente independiente,
    con lo que por el teorema~\ref{theo:espacio-vectorial-li}
    es \(\lvert A \rvert \le \lvert B \rvert\).
    Por el mismo argumento,
    intercambiando los roles de \(A\) y \(B\),
    \(\lvert B \rvert \le \lvert A \rvert\),
    con lo que \(\lvert A \rvert = \lvert B \rvert\).
  \end{proof}
  Esto nos lleva a:
  \begin{theorem}
    \label{theo:espacio-vectorial-isomorfos}
    Todos los espacios vectoriales de la misma dimensión finita
    sobre \(F\) son isomorfos.
  \end{theorem}
  \begin{proof}
    Sean \(U\) y \(V\) espacios vectoriales
    de la misma dimensión finita,
    con bases \(\{\boldsymbol{a}_k\}_{1 \le k \le n}\)
    y \(\{\boldsymbol{b}_k\}_{1 \le k \le n}\),
    respectivamente.
    Podemos representar todos los vectores \(\boldsymbol{u} \in U\)
    y \(\boldsymbol{v} \in V\)
    mediante:
    \begin{equation*}
      \boldsymbol{u}
	= \sum_{1 \le k \le n} a_k \boldsymbol{a}_k \hspace{2em}
      \boldsymbol{v}
	= \sum_{1 \le k \le n} b_k \boldsymbol{b}_k
    \end{equation*}
    Definimos la biyección \(f \colon U \rightarrow V\)
    mediante:
    \begin{equation*}
      f \colon \sum_{1 \le k \le n} a_k \boldsymbol{a}_k
	\mapsto \sum_{1 \le k \le n} a_k \boldsymbol{b}_k
    \end{equation*}
    Demostrar que la suma vectorial
    y el producto escalar se preservan
    es rutinario.
  \end{proof}
  \noindent
  En vista de la demostración
  del teorema~\ref{theo:espacio-vectorial-isomorfos},
  en un espacio vectorial de dimensión finita
  basta elegir una base,
  cada vector puede representarse
  mediante la secuencia de los coeficientes en \(F\).
  La suma vectorial es sumar componente a componente,
  el producto escalar es multiplicar cada componente por el escalar.
  Es por esta representación que a secuencias de largo fijo
  les llaman vectores.

  Lo anterior solo cubre una peueña parte
  de la extensa teoría relacionada con operaciones lineales.
  Para profundizar en ella recomendamos el texto de Treil~%
    \cite{treil14:_linear_algeb_done_wrong}.

%%% Local Variables:
%%% mode: latex
%%% TeX-master: "clases"
%%% End:

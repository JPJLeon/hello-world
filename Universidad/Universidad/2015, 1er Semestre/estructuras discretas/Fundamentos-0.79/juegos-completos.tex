% juegos-completos.tex
%
% Copyright (c) 2011-2014 Horst H. von Brand
% Derechos reservados. Vea COPYRIGHT para detalles

\section{Juegos completos de poker}
\label{sec:poker-juegos}
\index{juegos de poker|see{combinatoria!juegos de poker}}
\index{combinatoria!juegos de poker|textbfhy}

  Los señores George G.~Akeley,
  Robert Blake,
  Randolph Carter
  y Edward P.~Davis
  juegan poker.
  Interesa saber
  de cuántas maneras se pueden repartir las \(52\)~cartas
  en \(4\)~manos de \(5\)~cartas,
  quedando \(32\)~cartas en el mazo.

  Podemos atacar el problema considerando que Akeley
  elige \(5\) cartas de las \(52\),
  que Blake elige \(5\) de las restantes,
  y así sucesivamente.
  El resultado es:
  \begin{align*}
    \binom{52}{5}
	\cdot \binom{52 - 5}{5}
	\cdot \binom{52 - 2 \cdot 5}{5}
	\cdot \binom{52 - 3 \cdot 5}{5}
      &= \frac{52!}{5! \, 5! \, 5! \, 5! \, 32!} \\
      &= \binom{52}{5 \; 5 \; 5 \; 5 \; 32}
  \end{align*}

  Si consideramos las cartas en un orden cualquiera,
  podemos representar la distribución
  asociando cada posición con quien la tiene.
  De esta forma,
  tenemos una biyección
  entre secuencias de \(52\)~dueños de las cartas respectivas
  y las distribuciones de las cartas.
  Para simplificar notación,
  denotamos a los caballeros
  por las primeras letras de sus apellidos,
  y el mazo por \(\mathtt{M}\).
  Buscamos entonces el número de secuencias de \(52\)~símbolos
  elegidos
  entre
    \(\{\mathtt{A}, \mathtt{B}, \mathtt{C}, \mathtt{D},
	\mathtt{M}\}\)
  formadas con \(5\)~\(\mathtt{A}\),
  \(5\)~\(\mathtt{B}\),
  \(5\)~\(\mathtt{C}\),
  \(5\)~\(\mathtt{D}\) y \(32\)~\(\mathtt{M}\).
  Esto nos lleva directamente al resultado anterior
  al aplicar el tao,
  sección~\ref{sec:tao-bookkeeper}.%
    \index{combinatoria!secuencias con repeticiones}

%%% Local Variables:
%%% mode: latex
%%% TeX-master: "clases"
%%% End:

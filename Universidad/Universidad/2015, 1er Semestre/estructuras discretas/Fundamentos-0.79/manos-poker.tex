% manos-poker.tex
%
% Copyright (c) 2009-2014 Horst H. von Brand
% Derechos reservados. Vea COPYRIGHT para detalles

\section{Manos de poker}
\label{sec:manos-poker}
\index{manos de poker|see{combinatoria!manos de poker}}
\index{combinatoria!manos de poker|textbfhy}

% Fixme: Revisar las reglas, contabilizar _todas_ las manos
%	 (hay que restar alternativas repetidas, etc)
%	 ¿Algún juego más complejo (2 mazos, ...)?

  Nuestro siguiente tema de interés es contar subconjuntos
  que cumplen ciertas restricciones.
  Como conjuntos,
  siguiendo a Lehman, Leighton y Meyer~%
    \cite{lehman15:_mathem_comput_scien},
  usaremos manos de \foreignlanguage{english}{poker}.

  En \foreignlanguage{english}{poker}
  a cada jugador se le da una \emph{mano} de cinco cartas,
  elegidas del mazo inglés,
  formado por cuatro \emph{pintas}:%
    \index{carta!pinta}
  Pica (\(\spadesuit\)),
    \index{carta!pinta!pica (\(\spadesuit\))|textbfhy}
  corazón (\(\heartsuit\)),
    \index{carta!pinta!corazon (\(\heartsuit\))@corazón (\(\heartsuit\))|textbfhy}
  trébol (\(\clubsuit\))
    \index{carta!pinta!trebol (\(\clubsuit\))@trébol (\(\clubsuit\))|textbfhy}
  y diamante (\(\diamondsuit\));
    \index{carta!pinta!diamante (\(\diamondsuit\))|textbfhy}
  en cada pinta hay trece \emph{valores}:%
    \index{carta!valor|textbfhy}
  As, \(2\) a \(10\),
  \foreignlanguage{english}{Jack},
  \foreignlanguage{english}{Queen}
  y \foreignlanguage{english}{King}.
  El número total de manos posibles es:
  \begin{equation*}
    \binom{52}{5} = 2\,598\,960
  \end{equation*}

  Como estrategia general,
  buscaremos secuencias que describan las manos que queremos contar
  (porque contar secuencias es fácil),
  y nos aseguraremos que hay una biyección
  (o que haya alguna otra relación clara,
   como un mapa \(2\) a \(1\))
  entre descripciones
  y manos.

\subsection{Royal Flush}
\label{sec:royal-flush}

  Es la mano más alta en \foreignlanguage{english}{poker}.
  Consta de \foreignlanguage{english}{As},
  \foreignlanguage{english}{King},
  \foreignlanguage{english}{Queen},
  \foreignlanguage{english}{Jack}, \(10\) de la misma pinta,
  por ejemplo:
  \begin{equation*}
    \begin{array}{@{\{}*{4}{c@{\;\;}}c@{\}}}
      A \spadesuit & K \spadesuit & Q \spadesuit &
	J \spadesuit & 10 \spadesuit
    \end{array}
  \end{equation*}
  Está claro que hay una mano de éstas para cada pinta,
  con lo que hay exactamente \(4\).

\subsection{Straight Flush}
\label{sec:straight-flush}

  Consta de \(5\) cartas de la misma pinta en secuencia,
  donde \foreignlanguage{english}{As} cuenta como \(1\)
  (no después de \foreignlanguage{english}{King},
   como en el \foreignlanguage{english}{Royal Flush}).
  Ejemplos son:
  \begin{equation*}
    \begin{array}{@{\{}*{4}{c@{\;\;}}c@{\}}}
      8 \spadesuit & 9 \spadesuit & 10 \spadesuit &
	J \spadesuit & Q \spadesuit \\
      A \heartsuit & 2 \heartsuit & 3 \heartsuit  &
	4 \heartsuit & 5\heartsuit \\
      3 \clubsuit  & 4 \clubsuit  & 5 \clubsuit	  &
	6\clubsuit  & 7\clubsuit
    \end{array}
  \end{equation*}
  Estas manos podemos describirlas mediante
  una secuencia que indica:
  \begin{itemize}
  \item
    El valor de la primera carta en la secuencia.
    Este puede elegirse de \(9\) maneras
    (entre \(1\) y \(9\)).
  \item
    La pinta,
    que puede elegirse de \(4\) maneras.
  \end{itemize}
  En nuestros ejemplos:
  \begin{equation*}
    \begin{array}{l@{{} \longleftrightarrow \{}*{4}{c@{\;\;}}c@{\}}}
      (8, \spadesuit) &
	8 \spadesuit  & 9 \spadesuit & 10 \spadesuit &
	J \spadesuit  & Q \spadesuit \\
      (1, \heartsuit) &
	A \heartsuit  & 2 \heartsuit & 3 \heartsuit  &
	4 \heartsuit  & 5 \heartsuit \\
      (3, \clubsuit)  &
	3 \clubsuit   & 4 \clubsuit  & 5 \clubsuit   &
	6 \clubsuit   & 7 \clubsuit
    \end{array}
  \end{equation*}
  Por la regla del producto,%
    \index{combinatoria!regla del producto}
  el número total de estas manos es:
  \begin{equation*}
    9 \cdot \binom{4}{1} = 36
  \end{equation*}
  Como esto no describe un \foreignlanguage{english}{Royal Flush},
  no hace falta ningún ajuste adicional.

\subsection{Four of a Kind}
\label{sec:four-of-a-kind}

  Esta es una mano con cuatro cartas del mismo valor.
  Por ejemplo:
  \begin{align*}
    \begin{array}{@{\{}*{4}{c@{\;\;}}c@{\}}}
      8 \spadesuit  & 8 \heartsuit & 8 \clubsuit &
	8 \diamondsuit	 & 5 \diamondsuit \\
      2 \spadesuit  & 2 \heartsuit & 2 \clubsuit  &
	2 \diamondsuit	 & 3 \clubsuit
    \end{array}
  \end{align*}
  Para calcular cuántas de estas hay,
  armamos un mapa de secuencias a manos de este tipo
  y contamos las secuencias.
  En este caso,
  una mano queda descrita por:
  \begin{itemize}
  \item El valor que se repite.
  \item El valor de la quinta carta.
  \item La pinta de la quinta carta.
  \end{itemize}
  Hay una biyección entre secuencias de estos tres elementos
  y manos.
  En nuestros ejemplos,
  las correspondencias son:
  \begin{align*}
    \begin{array}{l@{{} \longleftrightarrow \{}*{4}{c@{\;\;}}c@{\}}}
      (8, 5, \diamondsuit) &
	8 \spadesuit & 8 \heartsuit & 8 \clubsuit & 8 \diamondsuit
	   & 5 \diamondsuit \\
      (2, 3, \clubsuit)	   &
	2 \spadesuit & 2 \heartsuit & 2 \clubsuit & 2 \diamondsuit
	   & 3 \clubsuit
    \end{array}
  \end{align*}
  Para el valor tenemos \(13\) posibilidades,
  para el valor de la quinta carta quedan \(12\) posibilidades,
  y hay \(4\) opciones para la pinta de la quinta carta.
  En total,
  usando la regla del producto,%
    \index{combinatoria!regla del producto}
  son \(13 \cdot 12 \cdot 4 = 624\) posibilidades.
  Hay \(1\) en \(2\,598\,960 / 624 = 4\,165\) manos,
  no sorprende que se considere muy buena.

\subsection{Full House}
\label{sec:full-house}

  Es una mano con tres cartas de un valor
  y dos de otro.
  Ejemplos:
  \begin{equation*}
    \begin{array}{@{\{}*{4}{c@{\;\;}}c@{\}}}
       2 \heartsuit & 2 \clubsuit & 2 \diamondsuit & Q \spadesuit
	  & Q \diamondsuit \\
       5 \spadesuit & 5 \clubsuit & 5 \diamondsuit & K \spadesuit
	  & K \heartsuit
    \end{array}
  \end{equation*}
  Nuevamente un mapa con secuencias:
  \begin{itemize}
  \item
    El valor del trío,
    que puede especificarse de \(13\) maneras.
  \item
    Las pintas del trío,
    que son elegir \(3\) de entre \(4\).
  \item
    El valor del par,
    que se puede tomar de \(12\) maneras.
  \item
    Las pintas del par,
    que se eligen \(2\) entre \(4\).
  \end{itemize}
  Las manos ejemplo corresponden con:
  \begin{equation*}
    \begin{array}{l@{{} \longleftrightarrow \{}*{4}{c@{\;\;}}c@{\}}}
      (2, \{\heartsuit, \clubsuit, \diamondsuit\},
       Q, \{\spadesuit, \diamondsuit\}) &
	 2 \heartsuit & 2 \clubsuit & 2 \diamondsuit & Q \spadesuit
	    & Q \diamondsuit \\
      (5, \{\spadesuit, \clubsuit, \diamondsuit\},
       K, \{\spadesuit, \heartsuit\})	&
	 5 \spadesuit & 5 \clubsuit & 5 \diamondsuit & K \spadesuit
	    & K \heartsuit
    \end{array}
  \end{equation*}
  Por la regla del producto%
    \index{combinatoria!regla del producto}
  el número de \foreignlanguage{english}{Full Houses} es entonces:
  \begin{equation*}
    13 \cdot \binom{4}{3} \cdot 12 \cdot \binom{4}{2} = 3\,744
  \end{equation*}

\subsection{Flush}
\label{sec:flush}

  Mano con \(5\) cartas de la misma pinta,
  como por ejemplo:
  \begin{equation*}
    \begin{array}{@{\{}*{4}{c@{\;\;}}c@{\}}}
      A \heartsuit & 3 \heartsuit & 4 \heartsuit &
	8 \heartsuit & K \heartsuit
    \end{array}
  \end{equation*}
  Esto se describe mediante la secuencia que da:
  \begin{itemize}
  \item
    Un conjunto de \(5\) valores,
    se eligen \(5\) de entre \(13\).
  \item
    Una pinta,
    se elige una entre \(4\).
  \end{itemize}
  En nuestro ejemplo:
  \begin{equation*}
    \begin{array}{l@{{} \longleftrightarrow \{}*{4}{c@{\;\;}}c@{\}}}
      (\{A, 3, 4, 8, K\}, \heartsuit) &
	 A \heartsuit & 3 \heartsuit & 4 \heartsuit &
	   8 \heartsuit & K \heartsuit
    \end{array}
  \end{equation*}
  De estas manos hay entonces:
  \begin{equation*}
    \binom{13}{5} \cdot \binom{4}{1} = 5\,148
  \end{equation*}
  Esto también describe al \foreignlanguage{english}{Royal Flush}
  y al \foreignlanguage{english}{Straight Flush},
  debemos restar éstos
  (regla de la suma):%
    \index{combinatoria!regla de la suma}
  \begin{equation*}
    5\,148 - 4 - 36 = 5\,108
  \end{equation*}

\subsection{Manos con dos pares}
\label{sec:dos-pares}

  Interesa calcular cuántas manos con dos pares hay,
  vale decir,
  dos cartas de un valor,
  dos cartas de otro valor,
  y una carta de un tercer valor.
  Ejemplos son:
  \begin{align*}
    \begin{array}{@{\{}*{4}{c@{\;\;}}c@{\}}}
      3 \heartsuit & 3 \diamondsuit & Q \spadesuit & Q \clubsuit
	 & 5 \diamondsuit \\
      9 \heartsuit & 9 \clubsuit    & K \spadesuit & K \diamondsuit
	 & 2 \spadesuit
    \end{array}
  \end{align*}
  Cada mano queda descrita por:
  \begin{itemize}
  \item
    El valor del primer par,
    puede elegirse de \(13\) maneras.
  \item
    Las pintas del primer par,
    se toman \(2\) de entre \(4\).
  \item
    El valor del segundo par,
    que se puede elegir de \(12\) maneras.
  \item
    Las pintas del segundo par,
    se eligen \(2\) entre \(4\).
  \item
    El valor de la carta extra,
    es uno de \(11\).
  \item
    La pinta de la carta extra,
    que es una de \(4\).
  \end{itemize}
  Se pensaría entonces que el número buscado es:
  \begin{equation*}
    13 \cdot \binom{4}{2} \cdot 12 \cdot \binom{4}{2}
     \cdot 11 \cdot \binom{4}{1}
  \end{equation*}
  ¡Esto es incorrecto!
  El mapa entre secuencias y manos no es una biyección,
  es \(2\) a \(1\)
  (podemos elegir cualquiera de los dos pares como primero).
  El valor correcto es:
  \begin{equation*}
    \frac{13 \cdot \binom{4}{2} \cdot 12 \cdot \binom{4}{2}
	    \cdot 11 \cdot \binom{4}{1}}{2}
      = 123\,552
  \end{equation*}
  No es una mano particularmente buena.

  Pero además es perturbadora:
  Es fácil omitir el detalle de que no es una biyección.
  Hay dos salidas:
  \begin{enumerate}
  \item
    Cada vez que se usa
    un mapa \(f \colon \mathcal{A} \rightarrow \mathcal{B}\),
    verifique que el mismo número de elementos de \(\mathcal{A}\)
    llevan a cada elemento de \(\mathcal{B}\);
    si este número es \(k\),
    aplique la regla de división con \(k\).%
      \index{combinatoria!regla de division@regla de división}
  \item
    Intente otra forma de resolver el problema.
    Muchas veces hay varias formas de enfrentarlo --
    y debieran dar el mismo resultado.
    Claro que suele ocurrir que métodos distintos
    dan resultados que se \emph{ven} diferentes,
    aunque resultan ser iguales.
  \end{enumerate}

  Arriba usamos un método,
  veamos un segundo:
  Hay una biyección entre estas manos y secuencias que especifican:
  \begin{itemize}
  \item
    Los valores de los dos pares,
    se pueden elegir \(2\) entre \(13\).
  \item
    Las pintas del par de menor valor,
    se eligen \(2\) entre \(4\).
  \item
    Las pintas del par de mayor valor,
    se eligen \(2\) entre \(4\).
  \item
    El valor de la carta extra,
    es \(1\) entre \(11\).
  \item
    La pinta de la carta extra,
    es \(1\) entre \(4\).
  \end{itemize}
  Para nuestro ejemplo:
  \begin{equation*}
    \begin{array}{l@{{} \longleftrightarrow \{}*{4}{c@{\;\;}}c@{\}}}
      (\{3, Q\}, \{\diamondsuit, \heartsuit\},
	 \{\spadesuit, \clubsuit\},
       5, \diamondsuit\}) &
	 3 \diamondsuit & 3 \heartsuit & Q \spadesuit &
	 Q \clubsuit  & 5 \diamondsuit \\
      (\{9, K\}, \{\clubsuit, \heartsuit\},
	 \{\spadesuit, \heartsuit\},
       2, \spadesuit\})	  &
	 9 \clubsuit	& 9 \heartsuit & K \spadesuit &
	 K \diamondsuit & 2 \spadesuit
    \end{array}
  \end{equation*}
  Esto lleva a:
  \begin{equation*}
    \binom{13}{2} \cdot \binom{4}{2} \cdot \binom{4}{2} \cdot 11
      \cdot \binom{4}{1}
  \end{equation*}
  Es el mismo resultado anterior,
  claro que escrito de forma ligeramente diferente.

\subsection{Manos con todas las pintas}
\label{sec:todas-las-pintas}

  Buscamos el número de manos con cartas de todas las pintas.
  Por ejemplo:
  \begin{equation*}
    \begin{array}{@{\{}*{4}{c@{\;\;}}c@{\}}}
      7 \heartsuit & 8 \diamondsuit & K \clubsuit &
	A \spadesuit & 3 \heartsuit
    \end{array}
  \end{equation*}
  Esto podemos describirlo mediante:
  \begin{itemize}
  \item
    Los valores de las cartas de cada pinta,
    o sea \(13 \cdot 13 \cdot 13 \cdot 13\) posibilidades.
  \item
    El valor de la carta extra,
    con \(12\) selecciones posibles.
  \item
    La pinta de la carta extra,
    \(4\) opciones.
  \end{itemize}
  La mano del ejemplo se describe mediante:
  \begin{equation*}
    \begin{array}{l@{{} \longleftrightarrow \{}*{4}{c@{\;\;}}c@{\}}}
      (A, 7, 8, K, 3, \heartsuit) &
	 7 \heartsuit & 8 \diamondsuit & K \clubsuit &
	 A \spadesuit & 3 \heartsuit
    \end{array}
  \end{equation*}
  El problema es nuevamente que esto no es una biyección,
  en el ejemplo podemos considerar \(3\heartsuit\) o \(7\heartsuit\)
  como la carta extra,
  y el mapa es \(2\) a \(1\).
  El número buscado es:
  \begin{equation*}
    \frac{13^4 \cdot 4 \cdot 12}{2}
      = 685\,464
  \end{equation*}

  Una forma alternativa
  es dar los valores del par de la misma pinta,
  y la pinta del par;
  y luego los valores de las tres cartas de las pintas restantes.
  Nuestro ejemplo se describe mediante:
  \begin{equation*}
    \begin{array}{l@{{} \longleftrightarrow \{}*{4}{c@{\;\;}}c@{\}}}
      (\{3, 7\}, \heartsuit, A, K, 8) &
	 7 \heartsuit & 8 \diamondsuit & K \clubsuit &
	 A \spadesuit & 3 \heartsuit
    \end{array}
  \end{equation*}
  Acá hemos supuesto
  el orden \(\spadesuit, \heartsuit, \clubsuit, \diamondsuit\)
  de las pintas.
  Esto da nuevamente:
  \begin{equation*}
    \binom{13}{2} \cdot \binom{4}{1} \cdot 13 \cdot 13 \cdot 13
      = 685\,464
  \end{equation*}

\subsection{Manos con valores diferentes}
\label{sec:poker-different-values}

  Nos interesa ahora contar el número de manos
  en las cuales todos los valores son diferentes.
  Una forma alternativa de describir estas manos
  es diciendo que no tienen pares.
  Veremos varias maneras para obtener esto.

  Una primera forma de enfrentar esto
  es considerar
  que la primera carta se puede elegir de entre \(52\),
  la segunda entre las \(48\) que no tienen el valor de la primera,
  y así sucesivamente.
  Pero el hablar de la primera, segunda y sucesivas cartas
  presupone orden,
  alerta del riesgo de contar demás.%
    \index{combinatoria!sobrecontar}
  Como son todas diferentes,
  basta dividir por el número de ordenamientos de \(5\) cartas,
  vale decir \(5! = 120\).
  O sea:
  \begin{equation*}
    \frac{52 \cdot 48 \cdot 44 \cdot 40 \cdot 36}{5!}
      = 1\,317\,888
  \end{equation*}

  Una solución alternativa
  consiste en seleccionar los valores de las \(5\) cartas,
  entre los \(13\) valores posibles,
  y luego a cada carta asignarle una pinta entre \(4\).
  Esto da:
  \begin{equation*}
    \binom{13}{5} \cdot 4^5
      = 1\,317\,888
  \end{equation*}

%%% Local Variables:
%%% mode: latex
%%% TeX-master: "clases"
%%% End:

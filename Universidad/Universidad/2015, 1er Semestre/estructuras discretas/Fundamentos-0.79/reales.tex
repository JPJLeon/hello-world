% reales.tex
%
% Copyright (c) 2009-2014 Horst H. von Brand
% Derechos reservados. Vea COPYRIGHT para detalles

\chapter{Números reales}
\label{cha:numeros-reales}
\index{numero@número!real|textbfhy}
\index{R (numeros reales)@\(\mathbb{R}\) (números reales)}

  Los números reales son fundamentales en mucha de la matemática
  que usamos diariamente.
  No nos detendremos en un estudio detallado de ellos,
  los exploraremos como ejemplo de campo,
  estructura algebraica que trataremos en algún detalle más adelante
  (mostrando el funcionamiento del método axiomático
   y algunas técnicas de demostración).

\section{Axiomas de los reales}
\label{sec:axiomas-reales}
\index{R (numeros reales)@\(\mathbb{R}\) (números reales)!axioma}

  Daremos una breve introducción a los números reales y sus propiedades,
  siguiendo en lo general a Chen~%
    \cite[capítulo 1]{chen08:_first_year_calculus}.
  Anotamos \(\mathbb{R}\) para el conjunto de números reales,%
    \index{R (números reales)@\(\mathbb{R}\) (números reales)}
  con operaciones \(+\) y \(\cdot\).
  Los números reales con sus operaciones cumplen los siguientes axiomas,%
    \index{axioma!numeros reales@números reales|textbfhy}%
    \index{numero@número!real!axiomas|see{axioma!números reales}}
  que simplemente daremos por hechos.
  Estos axiomas describen lo que se conoce como un \emph{campo}%
    \index{campo (algebra)@campo (álgebra)}
  (en inglés \emph{\foreignlanguage{english}{field}}).%
    \index{field@\emph{\foreignlanguage{english}{field}}|see{campo (álgebra)}}
  En esta lista \(a\), \(b\), \(c\) son reales cualquiera.
  \begin{enumerate}[label=\textbf{R\arabic{*}:}, ref=R\arabic{*}]
  \item\label{Re:suma-asociativa}%
    \index{operacion@operación!asociativa}
    La suma es asociativa:
    \((a + b) + c = a + (b + c)\).
  \item\label{Re:cero}%
    \index{operacion@operación!elemento neutro}
    Hay un \emph{elemento neutro para la suma} \(0 \in \mathbb{R}\)
    tal que \(a + 0 = a\)
  \item\label{Re:inverso-aditivo}%
    \index{operacion@operación!inverso}
    Hay un elemento \(-a \in \mathbb{R}\)
    tal que \(a + (-a) = 0\).
    A \(-a\) se le llama \emph{inverso aditivo} de \(a\).
  \item\label{Re:suma-conmutativa}%
    \index{operacion@operación!conmutativa}
    La suma es conmutativa:
    \(a + b = b + a\).
  \item\label{Re:multiplicacion-asociativa}%
    \index{operacion@operación!asociativa}
    La multiplicación es asociativa:
    \((a \cdot b) \cdot c = a \cdot (b \cdot c)\)
  \item\label{Re:distributiva}%
    \index{operacion@operación!distributiva}
    La multiplicación distribuye sobre la suma:
    \((a + b) \cdot c = (a \cdot c) + (b \cdot c)\).
  \item\label{Re:uno}
    Hay un elemento \emph{neutro para la multiplicación} \(1 \in \mathbb{R}\)
    tal que \(a \cdot 1 = a\).
  \item\label{Re:multiplicacion-conmutativa}
    La multiplicación es conmutativa:
    \(a \cdot b = b \cdot a\).
  \item\label{Re:inverso-multiplicativo}
    Para \(a \ne 0\) hay un elemento \(a^{-1} \in \mathbb{R}\)
    tal que \(a \cdot a^{-1} = 1\).
    A \(a^{-1}\) se le llama \emph{inverso multiplicativo} de \(a\).
  \end{enumerate}

  Entre los reales tenemos además un orden,%
    \index{relacion@relación!orden}
  una relación \(<\) que cumple los siguientes axiomas adicionales,
  donde \(a\), \(b\) y \(c\) nuevamente denotan números reales cualquiera.
  Anotamos \(a > b\) si \(b < a\),
  como es convencional.
  \begin{enumerate}[label=\textbf{O\arabic{*}:}, ref=O\arabic{*}]
  \item\label{Re:tricotomia}%
    \index{relacion@relación!orden!tricotomia@tricotomía}
    Se cumple exactamente uno de \(a < b\), \(a = b\) o \(a > b\).
    Esta propiedad se conoce como \emph{tricotomía}.
  \item\label{Re:transitidad<}%
    \index{relacion@relación!transitiva}
    Si \(a < b\) y \(b < c\) entonces \(a < c\).
  \item\label{Re:>+}
    Si \(a < b\),
    entonces \(a + c < b + c\).
  \item\label{Re:>*}
    Si \(a < b\) y \(c > 0\)
    entonces \(a \cdot c < b \cdot c\).
  \end{enumerate}
  En términos de las propiedades de relaciones,
  diríamos que \(<\) es transitiva e irreflexiva,%
    \index{relacion@relación!irreflexiva}
  y que la relación \(\le\) en \(\mathbb{R}\),
  definida mediante \mbox{\(a \le b \equiv (a < b) \vee (a = b)\)},
  es un orden total.%
    \index{relacion@relación!orden total}

  Un subconjunto de los reales
  es el conjunto de los números naturales,%
    \index{N (números naturales)@\(\mathbb{N}\) (números naturales)}
  \(\mathbb{N} = \{1, 2, 3, \dotsc\}\).
  Los siguientes axiomas dan sus principales características.%
    \index{axioma!numeros naturales@números naturales|textbfhy}
  Primeramente,
  \(\mathbb{N} \subseteq \mathbb{R}\) con las mismas operaciones
  y relación de orden.
  Enseguida:
  \begin{enumerate}[label=\textbf{N\arabic{*}:}, ref=N\arabic{*}]
  \item\label{N:uno}
    \(1 \in \mathbb{N}\)
  \item\label{N:sucesor}
    Si \(n \in \mathbb{N}\),
    entonces \(n + 1 \in \mathbb{N}\).
    A \(n + 1\) se le llama el \emph{sucesor} de \(n\).
  \item\label{N:solo-sucesores}
    Todo \(n \in \mathbb{N}\) tal que \(n \ne 1\)
    es el sucesor de un único número natural.
  \item\label{N:buen-orden}
    Todo subconjunto no vacío de \(\mathbb{N}\) contiene un elemento mínimo.
    Esto se conoce como \emph{principio de buen orden}.%
      \index{buen orden, principio de}
  \end{enumerate}
  Puede demostrarse que el principio de buen orden
  es equivalente al \emph{principio de inducción},
    \index{induccion@inducción!principio de}
  acá demostraremos este último partiendo de buen orden:
  \begin{theorem}[Principio de inducción]
    \label{theo:principio-induccion}
    Sea \(p(\cdot)\) un predicado que cumple:
    \begin{enumerate}[label=(\roman{*})]
    \item
      \(p(1)\) es verdadero
    \item
      \(p(n) \implies p(n + 1)\)
    \end{enumerate}
    Entonces \(p(n)\) es verdadero para todo \(n \in \mathbb{N}\).
  \end{theorem}
  \begin{proof}
    Por contradicción.
    Supongamos un conjunto no vacío
    \(\mathcal{C} \subseteq \mathbb{N}\) para el que \(p(\cdot)\) no vale.
    Por el principio del buen orden,
    \(\mathcal{C}\) contiene su elemento mínimo,
    llamémosle \(m\).
    Sabemos que \(p(1)\) es cierto,
    así que \(m > 1\) y es el sucesor de un natural \(n\).
    Como \(m\) es el mínimo para el que no vale \(p(\cdot)\),
    \(p(n)\) es cierto,
    pero entonces es cierto \(p(n + 1) = p(m)\),
    contradiciendo la elección de \(m\).
  \end{proof}

  El conjunto \(\mathbb{Z}\) de los enteros
  es la extensión de \(\mathbb{N}\)
  para incluir a \(0\)
  y los números de la forma \(-n\) para \(n \in \mathbb{N}\).
  El conjunto \(\mathbb{Q}\) de los racionales
  es el conjunto de números de la forma \(a \cdot b^{-1}\),
  con \(a \in \mathbb{Z}\) y \(b \in \mathbb{N}\).

  Los axiomas de campo y de orden valen para \(\mathbb{Q}\).%
    \index{Q (números racionales)@\(\mathbb{Q}\) (números racionales)}
  Pero vimos que \(\mathbb{Q}\) es incompleto
  (por el teorema~\ref{theo:sqrt2-irracional},
   \(\sqrt{2} \notin \mathbb{Q}\)).
  Veremos una propiedad que distingue a \(\mathbb{R}\) de \(\mathbb{Q}\).
  Se le conoce como \emph{axioma de completitud},%
    \index{completitud, axioma de}
  que nosotros definiremos en términos de cotas.
  \begin{definition}
    \index{numero@número!irracional|textbfhy}
    A un número \(x \in \mathbb{R} \smallsetminus \mathbb{Q}\)
    se le llama \emph{irracional}.%
      \index{numero@número!irracional|textbfhy}
  \end{definition}

  \begin{definition}
    Un conjunto no vacío \(\mathcal{S}\) de números reales
    se dice \emph{acotado por arriba} si hay \(C \in \mathbb{R}\)
    tal que \(x \le C\) para todo \(x \in \mathcal{S}\).
    A \(C\) se le llama \emph{cota superior} de \(\mathcal{S}\).%
      \index{cota!superior}
    Si hay \(c \in \mathbb{R}\)
    tal que \(x \ge c\) para todo \(x \in \mathcal{S}\),
    se dice \emph{acotado por abajo}
    y a \(c\) se le llama \emph{cota inferior} de \(\mathcal{S}\).%
      \index{cota!inferior}
    Si \(\mathcal{S}\) es acotado por abajo y por arriba,
    de dice \emph{acotado}.%
      \index{acotado}%
      \index{cota}
  \end{definition}
  Por ejemplo,
  \(\mathbb{N}\) es acotado por abajo pero no por arriba
  (cosa que demostraremos en el teorema~\ref{theo:arquimedeana}),
  el conjunto \(\mathbb{Q}\) no tiene cota inferior ni superior,
  mientras \(\{x \in \mathbb{R} \colon 1 < x \le 3\}\) es acotado.

  \begin{axiom}[Supremo]
    \label{Re:supremo}
    \index{supremo, axioma de|textbfhy}
    Sea \(\mathcal{S} \subseteq \mathbb{R}\) no vacío,
    acotado por arriba.
    Entonces existe \(M \in \mathbb{R}\) tal que
    \begin{enumerate}[label=(\alph{*})]
    \item
      \(M\) es cota superior de \(\mathcal{S}\).
    \item
      Para todo \(\epsilon > 0\) hay \(s \in \mathcal{S}\)
      tal que \(s > M - \epsilon\).
    \end{enumerate}
  \end{axiom}
  Esto dice que no pueden haber cotas superiores menores que \(M\),
  y asegura que \(M\) es un número real.
  \begin{definition}
    A \(M\) se le llama el \emph{supremo}
    (o mínima cota superior)
    de \(\mathcal{S}\),
    se anota \(M = \sup \mathcal{S}\).
  \end{definition}
  El axioma del supremo puede expresarse en la forma obviamente equivalente:
  \begin{axiom}[Ínfimo]
    \label{Re:infimo}
    \index{infimo, axioma de@ínfimo, axioma de|textbfhy}
    Sea \(\mathcal{S} \subseteq \mathbb{R}\) no vacío,
    acotado por abajo.
    Entonces existe \(m \in \mathbb{R}\) tal que
    \begin{enumerate}[label=(\alph{*})]
    \item
      \(m\) es cota inferior de \(\mathcal{S}\).
    \item
      Para todo \(\epsilon > 0\) hay \(s \in \mathcal{S}\)
      tal que \(s > m + \epsilon\).
    \end{enumerate}
  \end{axiom}
  \begin{definition}
    A este \(m\) se le llama \emph{ínfimo}
    (máxima cota inferior)
    de \(\mathcal{S}\),
    se anota \mbox{\(m = \inf \mathcal{S}\)}.
  \end{definition}

  Como un ejemplo,
  demostraremos que \(\sqrt{2}\) es real,%
    \index{numero@número!irracional!\(\sqrt{2}\)}
  y por tanto irracional.
  \begin{theorem}
    \label{theo:sqrt(2)-real}
    Hay un real positivo \(r\) que cumple \(r^2 = 2\).
  \end{theorem}
  \begin{proof}
    Sea \(\mathcal{S} = \{x \in \mathbb{R} \colon x^2 < 2\}\).
    Entonces \(\mathcal{S}\) no es vacío,
    ya que \(1^2 < 2\);
    y tiene a \(2\) como cota superior,
    ya que si \(x > 2\) entonces \(x^2 > 4 > 2\).
    Por el axioma del supremo,
    hay \(r \in \mathbb{R}\) tal que \(r = \sup \mathcal{S}\).
    Claramente \(r > 0\),
    ya que \(1 \in \mathcal{S}\).
    Demostraremos por contradicción que \(r^2 = 2\).
    Supongamos \(r^2 \ne 2\).
    Por el axioma~\ref{Re:tricotomia},
    debe ser entonces \(r^2 < 2\) o \(r^2 > 2\).
    Para demostrar que estas no se cumplen,
    basta exhibir un \(\epsilon > 0\) para cada caso
    para el cual falla.
    Por turno:
    \begin{description}
    \item[\boldmath\(r^2 < 2\):\unboldmath]
      Sea \(\epsilon > 0\)
      y consideremos:
      \begin{equation}
	\label{eq:r+eps}
	(r + \epsilon)^2
	  = r^2 + 2 r \epsilon + \epsilon^2
	  < r^2 + (2 r + 1) \epsilon
      \end{equation}
      Si ahora \(\epsilon < (2 - r^2) / (2 r + 1)\),
      la última expresión en~\eqref{eq:r+eps} es menor a \(2\),
      lo que contradice la elección de \(r\) como supremo.
    \item[\boldmath\(r^2 > 2\):\unboldmath]
      Para \(\epsilon > 0\)
      calculamos:
      \begin{equation}
	\label{eq:r-eps}
	(r - \epsilon)^2
	  = r^2 - 2 r \epsilon + \epsilon^2
	  > r^2 - 2 r \epsilon
      \end{equation}
      Si elegimos \(\epsilon < (r^2 - 2) / (2 r)\),
      la última expresión de~\eqref{eq:r-eps} es mayor a \(2\),
      nuevamente contradiciendo la elección de \(r\) como supremo.
    \end{description}
    En consecuencia,
    debe ser \(r^2 = 2\).
  \end{proof}

  Algunas consecuencias de la completitud de los reales son las siguientes.
  \begin{theorem}[Propiedad arquimedeana]
    \index{propiedad arquimedeana}
    \label{theo:arquimedeana}
    Para todo \(x \in \mathbb{R}\)
    hay \(n \in \mathbb{N}\)
    tal que \(n > x\).
  \end{theorem}
  \begin{proof}
    La demostración es por contradicción.
    Supongamos que \(x \in \mathbb{R}\),
    y que para todo \(n \in \mathbb{N}\) es \(n \le x\).
    Entonces \(x\) es una cota superior para \(\mathbb{N}\),
    y el conjunto \(\mathbb{N}\) tiene un supremo por completitud,
    sea \(M = \sup \mathbb{N}\).
    Así \(M \ge n\) para \(n \in \mathbb{N}\),
    en particular es \(M \ge n\) para \(n = 2, 3, 4, \dotsc\).
    Pero cada número natural
    (salvo \(1\))
    es el sucesor de un número natural,
    con lo que \(M \ge k + 1\) para \(k = 1, 2, 3, \dotsc\),
    o \(M - 1 \ge k\) para todo \(k \in \mathbb{N}\).
    Pero habíamos elegido \(M\) como el supremo de \(\mathbb{N}\),
    no puede tener cotas superiores menores.
  \end{proof}
  Esta demostración completa nuestra aseveración anterior que \(\mathbb{N}\)
  no tiene cota superior.

  \begin{theorem}
    \label{theo:Q-dense}
    Los racionales e irracionales son densos en \(\mathbb{R}\),
      \index{conjunto!denso|textbfhy}
    o sea entre cada par de reales distintos hay un racional y un irracional.
  \end{theorem}
  \begin{proof}
    Supongamos \(x, y \in \mathbb{R}\),
    con \(x < y\).
    Primero hay \(r \in \mathbb{Q}\) tal que \(x < r < y\).
    Supongamos \(x > 0\) por ahora.
    Por la propiedad arquimedeana,%
      \index{propiedad arquimedeana}
    teorema~\ref{theo:arquimedeana},
    existe \(b \in \mathbb{N}\) tal que \(b > (y - x)^{-1}\),
    de manera que \(b (y - x) > 1\).
    Por la propiedad arquimedeana existe \(n \in \mathbb{N}\)
    tal que \(n > b x\),
    con lo que el conjunto
      \(\mathcal{S} = \{n \in \mathbb{N} \colon n > b x\}\)
    no es vacío,
    y por el axioma~\ref{N:buen-orden} contiene su mínimo,%
      \index{buen orden, principio de}
    llamémosle \(a\).
    Entonces \(a - 1 \le b x\):
    Si fuera \(a = 1\),
    \(a - 1 = 0 < b x\);
    en caso contrario \(a - 1 > b x\)
    contradice la elección de \(a\) como mínimo.
    Se sigue:
    \begin{equation*}
      b x
	< a
	= (a - 1) + 1
	< b x + b (y - x)
	= b y
    \end{equation*}
    de forma que:
    \begin{equation*}
      x < a \cdot b^{-1} < y
    \end{equation*}
    Veamos ahora el caso \(x \le 0\).
    Por la propiedad arquimedeana,
    existe \(k \in \mathbb{N}\) tal que \(k > -x\),
    de manera que \(k + x > 0\).
    Por lo anterior,
    hay \(s \in \mathbb{Q}\) tal que \(x + k < s < y + k\),
    y \(x < s - k < y\)
    donde \(s - k \in \mathbb{Q}\).

    Para hallar un irracional entre \(x\) e \(y\),
    vemos por lo anterior que hay \(r_1 \in \mathbb{Q}\)
    tal que \(x < r_1 < y\);
    de la misma forma hay \(r_2 \in \mathbb{Q}\)
    tal que \(r_1 < r_2 < y\).
    Como \(1 < \sqrt{2} < 2\),
    es:
    \begin{equation*}
      x < r_1 < r_1 + (r_2 - r_1) / \sqrt{2} < r_2 < y
    \end{equation*}
    y \(r_1 + (r_2 - r_1) / \sqrt{2}\) claramente es irracional.
  \end{proof}

%%% Local Variables:
%%% mode: latex
%%% TeX-master: "clases"
%%% End:

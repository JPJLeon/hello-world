% irreducible-polynomials.tex
%
% Copyright (c) 2013-2014 Horst H. von Brand
% Derechos reservados. Vea COPYRIGHT para detalles

\subsection{Polinomios irreductibles en \(\mathbb{F}_q\)}
\label{sec:count-irreductible-polynomials}

  Recordamos del capítulo~\ref{cha:campos-finitos}
  que los polinomios
  \(\mathbb{F}_q[x]\) para \(q\) la potencia de un primo
  son un dominio euclidiano,
  por lo que
  por la teoría de la sección~\ref{sec:dominios-euclidianos},
  en particular el teorema~\ref{theo:PID=>UFD},
  nos asegura que hay factorización única
  (salvo unidades)
  en \(\mathbb{F}_q[x]\).
  Para obviar las unidades,
  consideremos polinomios mónicos.

  Si para el polinomio \(\alpha(x) \in \mathbb{F}_q[x]\)
  consideramos su grado como tamaño,
  vemos que multiplicar polinomios es simplemente sumar sus tamaños.
  Podemos entonces considerar
  la clase \(\mathcal{P}\)
  de polinomios mónicos en \(\mathbb{F}_q[x]\)
  con \(\lvert \alpha(x) \rvert = \deg(\alpha)\),
  y combinar polinomios corresponde a multiplicarlos.
  Es claro que hay \(q^n\) polinomios mónicos de grado \(n\),
  o sea la función generatriz ordinaria
  que cuenta polinomios mónicos es:
  \begin{equation*}
    P(z)
      = \sum_{n \ge 0} q^n z^n
      = \frac{1}{1 - q z}
  \end{equation*}

  Consideremos la clase \(\mathcal{I}\)
  de polinomios mónicos irreductibles,
  contados por la función generatriz ordinaria:
  \begin{equation*}
    I(z)
      = \sum_{n \ge 0} N_n z^n
  \end{equation*}
  Factorización única significa que todo polinomio
  corresponde a un multiconjunto de polinomios irreductibles:
  \begin{equation}
    \label{eq:polynomials=MSet(irreducibles)}
    \mathcal{P}
      = \MSet(\mathcal{I})
  \end{equation}
  Resulta interesante
  contar con una forma de resolver ecuaciones implícitas
  como~\eqref{eq:polynomials=MSet(irreducibles)}.
  \begin{theorem}
    \label{theo:A=MSet(B)}
    Sean \(\mathcal{A}\) y \(\mathcal{B}\)
    clases de objetos no rotulados
    relacionadas mediante:
    \begin{equation*}
      \mathcal{A}
	= \MSet(\mathcal{B})
    \end{equation*}
    Entonces las funciones generatrices ordinarias
    respectivas cumplen:
    \begin{equation}
      \label{eq:A=MSet(B)-->B}
      B(z)
	= \sum_{k \ge 1} \frac{\mu(k)}{k} \, \ln A(z^k)
    \end{equation}
  \end{theorem}
  \begin{proof}
    El método simbólico da:
    \begin{equation*}
      A(z)
	= \exp \left( \sum_{k \ge 1} \frac{B(z^k)}{k} \right)
    \end{equation*}
    Tomando logaritmos:
    \begin{align*}
      \ln A(z)
	&= \sum_{r \ge 1} \frac{B(z^r)}{r} \\
	&= \sum_{r \ge 1} \frac{1}{r} \, \sum_{s \ge 1} b_s z^{r s}
    \end{align*}
    Extraemos coeficientes:
    \begin{align*}
      n \left[ z^n \right] \ln A(z) \\
	&= \sum_{r \ge 1} \frac{n}{r} \, \sum_{s \ge 1} b_s z^{r s} \\
	&= \sum_{r s = n} s b_s
    \end{align*}
    Este es exactamente el caso
    del lema~\ref{lem:GF-Moebius-inversion},
    lo que entrega lo enunciado.
  \end{proof}

  Con el teorema~\ref{theo:A=MSet(B)}
  queda de~\eqref{eq:polynomials=MSet(irreducibles)}:
  \begin{align*}
    I(z)
      = \sum_{k \ge 1} \frac{\mu(k)}{k} \, \frac{1}{1 - q z^k}
  \end{align*}
  Tenemos nuevamente el resultado
  del teorema~\ref{theo:number-irreducible-polynomials}:
  \begin{equation*}
    N_n
      = \frac{1}{n} \, \sum_{d \mid n} \mu(n / d)  \, q^d
  \end{equation*}

%%% Local Variables:
%%% mode: latex
%%% TeX-master: "clases"
%%% End:

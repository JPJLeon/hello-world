% combinatoria-elemental.tex
%
% Copyright (c) 2010-2014 Horst H. von Brand
% Derechos reservados. Vea COPYRIGHT para detalles

\chapter{Combinatoria elemental}
\label{cha:combinatoria-elemental}
\index{combinatoria}

  Consideremos el problema de contar sistemáticamente
  los elementos de colecciones de objetos.
  Nuestro interés en el tema es que por ejemplo
  el comportamiento de un algoritmo de ordenamiento
  dependerá del número de disposiciones distintas
  en que pueden venir los datos,
  y ciertas características de dicho orden.
  Contar éstos,
  particularmente para conjuntos de datos de tamaño interesante,
  generalmente no es posible manualmente.
  Acá nos concentraremos en algunas técnicas simples,
  de aplicabilidad sorprendentemente amplia.
  Más adelante veremos herramientas adicionales.

\section{Técnicas básicas}
\label{sec:tecnicas-basicas-conteo}
\index{combinatoria!tecnicas basicas@técnicas básicas}

  Las herramientas básicas son:
  \begin{description}
  \item[Biyecciones (funciones 1--1):]
    \index{combinatoria!biyecciones}
    Si hay una función \(1\) a \(1\)
    como \(f \colon \mathcal{X} \rightarrow \mathcal{Y}\!\),
    entonces \(\lvert \mathcal{X} \rvert
		 = \lvert \mathcal{Y} \rvert\)
    (esto incluso lo usamos para definir cardinalidades
     en el capítulo~\ref{cha:numerabilidad}).
    Más en general para funciones \(k\) a \(1\):
    Si hay tal función
      \(g \colon \mathcal{X} \rightarrow \mathcal{Y}\!\),
    entonces
      \(\lvert \mathcal{X} \rvert
	  = k \cdot \lvert \mathcal{Y} \rvert\).
  \item[Regla de la suma:]
    \index{combinatoria!regla de la suma}
    Si \(\mathcal{A} \cap \mathcal{B} = \varnothing\),
    entonces
      \(\lvert \mathcal{A} \cup \mathcal{B} \rvert
	   = \lvert \mathcal{A} \rvert
	       + \lvert \mathcal{B} \rvert\).
    Esto se generaliza en forma obvia a un número mayor
    de conjuntos disjuntos a pares:
    \begin{equation*}
      \lvert \mathcal{A}_1 \cup \mathcal{A}_2
	\cup \dotso \cup \mathcal{A}_r \rvert
	  = \lvert \mathcal{A}_1 \rvert
	      + \lvert \mathcal{A}_2 \rvert
	      + \dotsb + \lvert \mathcal{A}_r \rvert
    \end{equation*}
    Si la intersección no es vacía,
    al sumar los tamaños estamos contando la intersección dos veces,
    o sea para dos conjuntos debemos hacer:
    \begin{equation*}
      \lvert \mathcal{A} \cup \mathcal{B} \rvert
	 = \lvert \mathcal{A} \rvert + \lvert \mathcal{B} \rvert
	   - \lvert \mathcal{A} \cap \mathcal{B} \rvert
    \end{equation*}
    Más adelante
    (capítulo~\ref{cha:pie})%
      \index{inclusion y exclusion, principio de@inclusión y exclusión, principio de}
    veremos cómo se puede manejar esto
    si hay más de dos conjuntos involucrados.
  \item[Contar por filas y por columnas:]
    \index{combinatoria!contar por filas y columnas}
    Si \(\mathcal{S} \subseteq \mathcal{X} \times \mathcal{Y}\),
    y para \(x \in \mathcal{X}\) e \(y \in \mathcal{Y}\) definimos:
    \begin{align*}
      r_x(\mathcal{S})
	&= \lvert \{(x, y) \in \mathcal{S}
		      \colon y \in \mathcal{Y}\} \rvert \\
      c_y(\mathcal{S})
	&= \lvert \{(x, y) \in \mathcal{S}
		      \colon x \in \mathcal{X}\} \rvert
    \end{align*}
    Entonces:
    \begin{equation*}
      \lvert \mathcal{S} \rvert
	 = \sum_{x \in \mathcal{X}} r_x(\mathcal{S})
	 = \sum_{y \in \mathcal{Y}} c_y(\mathcal{S})
    \end{equation*}

    En forma más general,
    si hay dos (o más) maneras de contar algo,
    debieran coincidir los resultados.
  \item[Regla del producto:]
    \index{combinatoria!regla del producto}
    Si al contar por filas y columnas
    tomamos \(\mathcal{S} = \mathcal{X} \times \mathcal{Y}\),
    resulta:
    \begin{equation*}
      \lvert \mathcal{S} \rvert
	= \lvert \mathcal{X} \rvert
	    \cdot \lvert \mathcal{Y} \rvert
    \end{equation*}
    dado que en ese caso
    \(r_x(\mathcal{S}) = \lvert \mathcal{Y} \rvert \)
    y \(c_y(\mathcal{S}) = \lvert \mathcal{X} \rvert\).
  \end{description}

  Algunos ejemplos simples:
  \begin{itemize}
  \item
    ¿Cuántas patentes antiguas
    (\(2\) letras, pero no \(\mathtt{Q}\) ni \(\mathtt{W}\);
    y \(4\) dígitos)
    hay?

    Podemos considerarlo como una tupla.
    Como hay \(24\) letras permitidas y \(10\) dígitos,
    por la regla del producto esto corresponde a:
    \begin{equation*}
      24 \cdot 24 \cdot 10 \cdot 10 \cdot 10 \cdot 10
	= 5\,760\,000
    \end{equation*}
    posibilidades.
  \item
    Se estaban acabando los números con el esquema anterior,
    se agregó la letra \(\mathtt{W}\).
    ¿Cuántos números se agregan?

    Nuevamente una tupla,
    pero ahora de \(25\) letras y \(10\) dígitos.
    La regla del producto da para el nuevo total:
    \begin{equation*}
      25 \cdot 25 \cdot 10 \cdot 10 \cdot 10 \cdot 10
	= 6\,250\,000
    \end{equation*}
    Por la regla de la suma,
    las patentes actuales son las antiguas y las agregadas,
    con lo que las agregadas son:
    \begin{equation*}
      6\,250\,000 - 5\,760\,000
	= 490\,000
    \end{equation*}
  \item
    ¿Cuántas patentes nuevas
    (\(4\) consonantes y \(2\) dígitos)
    hay?

    Otra vez una tupla.
    Son \(21 \cdot 21 \cdot 21 \cdot 21 \cdot 10 \cdot 10
	    = 19\,448\,100\).
  \item
    ¿Cuántas patentes hay en total?

    Son el conjunto de patentes antiguas y las nuevas,
    lo que da por la regla de la suma:
    \begin{equation*}
      6\,250\,000 + 19\,448\,100
	= 25\,698\,100
    \end{equation*}
  \item
    En la Universidad de Miskatonic,
    el decano Halsey insiste en que
    todos los estudiantes
    deben tomar exactamente cuatro cursos por semestre.
    Pide a los profesores
    que le hagan llegar las listas de los alumnos en sus cursos,
    pero estos solo le informan los números de estudiantes,
    ver el cuadro~\ref{tab:numero-cursos}.
    \begin{table}[htbp]
      \centering
      \begin{tabular}[htbp]{|l|l|>{\(}r<{\)}|}
	\hline
	\multicolumn{1}{|c|}{\rule[-0.7ex]{0pt}{3ex}\textbf{Profesor}} &
	   \multicolumn{1}{c|}{\textbf{Materia}} &
	\multicolumn{1}{c|}{\textbf{Nº}} \\
	\hline
	\rule[-0.7ex]{0pt}{3ex}%
	Ashley	 & Física      &  45 \\
	Dexter	 & Zoología    &  29 \\
	Dyer	 & Geología    &  33 \\
	Ellery	 & Química     &   2 \\
	Lake	 & Biología    &  12 \\
	Morgan	 & Arqueología &   5 \\
	Pabodie	 & Ingeniería  & 103 \\
	Upham	 & Matemáticas &  95 \\
	Wilmarth & Inglés      &   7 \\
	\hline
      \end{tabular}
      \caption{Número de alumnos por curso}
      \label{tab:numero-cursos}
    \end{table}
    ¿Qué puede decir el decano Halsey con estos datos?

    Si consideramos los pares (alumno, curso),
    la suma de cada fila es el número de cursos que el alumno toma.
    Por tanto,
    si cada alumno toma cuatro cursos,
    la suma total debe ser divisible por cuatro.
    La suma de cada columna es el número de alumnos en el curso.
    Pero en este caso la suma total
    de los alumnos por curso es \(331\),
    así que la condición del decano no se está cumpliendo.
  \end{itemize}

  Como un ejemplo más complejo,
  usando estas ideas podemos demostrar nuevamente
  para la función \(\phi\) de Euler:
  \begin{theorem}[Identidad de Gauß]
    \index{Gauss, identidad de@Gauß, identidad de}
    \label{theo:Gauss-identity-2}
    Para todo entero \(n\),
    tenemos:
    \begin{equation*}
      n = \sum_{d \mid n} \phi(d)
    \end{equation*}
    donde la suma se extiende
    sobre los enteros \(d\) que dividen a \(n\).
  \end{theorem}
  La idea de la siguiente demostración
  viene del conjunto de fracciones:
  \begin{equation*}
    \left\{ \frac{1}{n}, \frac{2}{n},
	    \dotsc, \frac{n - 1}{n} \right\}
      = \left\{
	  \frac{a_1}{b_1}, \frac{a_2}{b_2}, \dotsc,
	    \frac{a_{n - 1}}{b_{n - 1}}
	\right\}
  \end{equation*}
  donde \(a_r / b_r\) está en mínimos términos,
  o sea con \(d_r = \gcd(r, n)\):
  \begin{equation*}
    \frac{a_r}{b_r}
      = \frac{r / d_r}{n / d_r}
  \end{equation*}
  Cada \(b_r\) aparece exactamente \(\phi(b_r)\) veces.
  \begin{proof}
    Sean \(\mathcal{S}\) los pares \((d, f)\)
    tales que \(d \mid n\), \(1 \le f \le d\) y \(\gcd(f, d) = 1\).
    Sumando por filas tenemos:
    \begin{equation*}
      \lvert \mathcal{S} \rvert
	= \sum_{d \mid n} \phi(d)
    \end{equation*}
    Para demostrar que \(n = \lvert \mathcal{S} \rvert\),
    construimos una biyección \(\beta\)
    entre \(\mathcal{S}\) y los enteros entre \(1\) y \(n\).

    Sea \(\beta(d, f) =	 f n / d\).
    Esto siempre es un entero positivo,
    ya que \(d \mid n\);
    y como \(1 \le f \le d\),
    es a lo más \(n\).
    Para demostrar que es una inyección,
    consideremos:
    \begin{align*}
      \beta(d, f)
	&= \beta(d', f') \\
      f n / d
	&= f' n / d' \\
      f d'
	&= f' d
    \end{align*}
    Esto último es \(d \mid f d'\),
    y como \(f\) y \(d\) son relativamente primos,
    por el lema~\ref{lem:gcd}
    significa que \(d \mid d'\).
    De la misma forma \(d' \mid d\),
    y resulta \(d = d'\).
    Con esto también es \(f = f'\).

    Para demostrar que es sobre,
    supongamos dado \(1 \le k \le n\),
    y sean:
    \begin{align*}
      g_k
	&= \gcd(k, n) \\
      d_k
	&= n / g_k    \\
      f_k
	&= k / g_k
    \end{align*}
    Tanto \(d_k\) como \(f_k\) son enteros,
    y además \(\gcd(d_k, f_k) = 1\).
    Resulta:
    \begin{align*}
      \beta(d_k, f_k)
	&= \frac{f_k n}{d_k} \\
	&= \frac{k n / g_k}{n / g_k} \\
	&= k
	   \qedhere
    \end{align*}
  \end{proof}

\section{Situaciones recurrentes}
\label{sec:conteos-recurrentes}

  Según Albert~%
    \cite{albert09:_basic_count_princ}
  algunas circunstancias comunes
  se organizan bajo las siguientes ideas:
  \begin{description}
  \item[Objetos distinguibles o no:]
    Al jugar cartas interesa fundamentalmente su pinta y valor,
    mientras al discutir un canasto de frutas
    no interesa la identidad de cada una de las manzanas.
  \item[Repeticiones o no:]
    En un juego de cartas
    se considera de bastante mal gusto
    que una misma carta aparezca varias veces;
    si nos preguntamos
    de cuántas formas pueden entregarse \$\,\(100\)
    usando monedas de \$\,\(1\), \$\,\(5\) y \$\,\(10\),
    claramente se permite que una moneda se repita.
  \item[Orden interesa:]
    Al jugar cartas,
    una mano queda determinada por el conjunto de cartas
    (el orden no importa),
    al discutir números escritos en decimal
    el orden de los dígitos es fundamental.
  \end{description}
  Esto da lugar a varias situaciones diferentes,
  ordenadas aproximadamente en orden de complejidad creciente
  del análisis:
  \begin{description}
  \item[Secuencias:]
    Se dan siempre que el orden interesa.
    Pueden darse tanto situaciones donde se permiten repeticiones
    como cuando no se permiten.
  \item[Conjuntos:]
    No hay repetición
    y no interesa el orden,
    solo si el elemento pertenece a la colección o no.
  \item[Multiconjuntos:]
    Se permiten repeticiones
    y no interesa el orden.
    Un elemento dado puede pertenecer varias veces a la colección.
  \end{description}
  Veamos las distintas situaciones por turno,
  buscando expresiones simples
  para el número total de posibilidades
  suponiendo que estamos tomando \(k\) elementos
  de entre \(n\) opciones.
  \begin{description}
  \item[Secuencias sin repeticiones:]
    Esta situación se conoce como \emph{permutaciones},
    suele anotarse \(P(n, k)\) para el número de permutaciones
    de \(k\) objetos tomados entre un total de \(n\).
    El primer elemento puede elegirse de \(n\)~formas,
    el segundo de \(n - 1\) maneras,
    y así hasta llegar al último,
    que se puede elegir de \(n - k + 1\) maneras.
    Aplicando la regla del producto,
    tenemos:
    \begin{align}
      P(n, k)
	&= n \cdot (n - 1) \dotsm (n - k + 1) \notag \\
	&= n^{\underline{k}} \label{eq:Perm=ff}
    \end{align}
    En el caso particular en que \(k = n\) resulta:
    \begin{align}
      P(n, n)
	&= n^{\underline{n}} \notag \\
	&= n! \label{eq:Perms=n!}
    \end{align}
  \item[Secuencias con repeticiones:]
    Generalmente se llaman
    usando el término inglés
      \emph{\foreignlanguage{english}{strings}}%
      \index{string@\emph{\foreignlanguage{english}{string}}|see{palabra}}
    (también \emph{palabras},%
      \index{palabra}
     o las podemos considerar como tuplas
     cuyos elementos se toman todos del mismo conjunto).
    No hay notación en uso común para este caso.
    Aplicando la regla de multiplicación,
    viendo que cada uno de los \(k\) elementos
    puede elegirse de \(n\) maneras independientemente,
    en este tenemos:
    \begin{equation*}
      n^k
    \end{equation*}
    Un caso de interés
    es contar todas las secuencias hasta cierto largo \(k\).
    Vimos que hay \(n^r\) secuencias de largo \(r\),
    con lo que por el teorema~\ref{theo:suma-geometrica}
    el número buscado es:
    \begin{equation*}
      \sum_{0 \le r \le k} n^r
	= \frac{n^{k + 1} - 1}{n - 1}
    \end{equation*}
  \item[Conjuntos:]
    Para elegir \(k\) elementos de entre \(n\)
    sin interesar el orden
    (lo que se llaman \emph{combinaciones},%
      \index{combinatoria!combinacion@combinación|textbfhy}
     y suele anotarse \(C(n, k)\))
    podemos elegirlos en orden
    (hay \(P(n, k)\) maneras de hacer esto),
    y luego considerar que hay \(P(k, k) = k!\) maneras de ordenar
    los \(k\) elementos elegidos
    (un mapa \(k!\) a \(1\)
     entre las secuencias ordenadas
     y el conjunto de \(k\) elementos elegidos).
    Vale decir,
    el número buscado es:
    \begin{align}
      C(n, k)
	&= \frac{P(n, k)}{P(k, k)} \label{eq:Comb=Perm/Perm} \\
	&= \frac{n^{\underline{k}}}{k!} \label{eq:Comb=ff/f} \\
	&= \frac{n!}{k! (n - k)!} \label{eq:Comb=f/f*f} \\
	&= \binom{n}{k} \label{eq:Comb=binomial}
    \end{align}
    Debido a esto suele leerse \(\binom{n}{k}\)
    como ``\(n\) elija \(k\)''%
      \index{coeficiente binomial}
    (en inglés
      \emph{\(n\) \foreignlanguage{english}{choose} \(k\)}).
    Nótese que:
    \begin{equation*}
      \binom{n}{k} = \binom{n}{n - k}
    \end{equation*}
    lo que puede interpretarse
    diciendo que al elegir
    los \(k\) elementos incluidos en el subconjunto,
    lo que en realidad estamos haciendo
    es elegir los \(n - k\) elementos
    que estamos dejando fuera.
    A esta clase de razonamiento
    se le llama \emph{demostración combinatoria}.%
      \index{demostracion@demostración!combinatoria|textbfhy}
  \item[Multiconjuntos:]
    \index{combinatoria!multiconjunto|see{multiconjunto!número}}
    \index{multiconjunto!numero@número|textbfhy}
    No hay una notación especial aceptada comúnmente para este caso.
    Algunos autores usan:
    \begin{equation*}
      \multiset{n}{k}
    \end{equation*}
    para el caso en que tenemos \(n\) tipos de elementos
    de los cuales tomamos en total \(k\).
    Una manera de representar esta situación
    es mediante variables \(x_r\),
    donde \(x_r\) representa el número de elementos
    de tipo \(r\) elegidos.
    Entonces el número de multiconjuntos de tamaño \(k\)
    tomando de entre \(n\) alternativas
    es el número de soluciones en números naturales
    a la ecuación:
    \begin{equation*}
      x_1 + x_2 + \dotsb + x_n = k
    \end{equation*}
    \begin{figure}[ht]
      \centering
      \pgfimage{images/stars-bars}
      \caption{Una distribución de \(6\) elementos en \(4\) grupos}
      \label{fig:stars-bars}
    \end{figure}
    La solución \(x_1 = 2\), \(x_2 = 0\), \(x_3 = 3\), \(x_4 = 1\)
    al caso \(n = 4\) y \(k = 6\)
    queda ilustrada en la figura~\ref{fig:stars-bars}.
    Esta distribución puede describirse
    con un total de \(n = 6\) asteriscos para la suma,
    separados por \(k - 1 = 3\) barras para marcar las separaciones
    (los extremos son fijos,
     y los omitimos):
    \begin{equation*}
      ** | | *** | *
    \end{equation*}
    Visto de esta forma,
    lo que estamos haciendo es distribuir \(n - 1\) barras
    en \(n + k - 1\) posiciones,
    un total de \(C(n + k - 1, n - 1)\).
    A este tipo de razonamiento se le conoce
    como \emph{\foreignlanguage{english}{stars and bars}}%
      \index{stars and bars@\emph{\foreignlanguage{english}{stars and bars}}}
    en inglés.
    Así,
    el número de soluciones se expresa:
    \begin{equation*}
      \multiset{n}{k}
	= \binom{n + k - 1}{n - 1}
    \end{equation*}
    Nótese que puede escribirse,
    de forma afín a los coeficientes binomiales:
    \begin{equation*}
      \multiset{n}{k}
	= \binom{n + k - 1}{k}
	= \frac{n^{\overline{k}}}{k!}
    \end{equation*}
    Como \(n^{\overline{k}} = (-1)^k \, (-n)^{\underline{k}}\)
    se cumple la curiosa identidad:
    \begin{equation}
      \label{eq:multiset=negative-binomial}
      \multiset{n}{k}
	= (-1)^k \binom{-n}{k}
    \end{equation}
  \end{description}

  Determinemos el número de subconjuntos
  de \(k\) elementos de \([n]\)
  que no contienen elementos consecutivos.%
    \index{combinatoria!subconjuntos sin elementos consecutivos}
  Es claro que si \(k = 0\) hay un único subconjunto
  (el vacío),
  si \(k = 1\) hay \(n\).
  Otros casos simples son:
  \begin{align*}
    n = 3, k = 2 &\colon 1 \quad \{1, 3\} \\
    n = 4, k = 2 &\colon 3 \quad \{1, 3\}, \{1, 4\}, \{2, 4\} \\
    n = 5, k = 2 &\colon 6 \quad \{1, 3\}, \{1, 4\}, \{1, 5\},
				 \{2, 4\}, \{2, 5\},
				 \{3, 5\} \\
    n = 5, k = 3 &\colon 1 \quad \{1, 3, 5\}
  \end{align*}

  Podemos nombrar un subconjunto de \([n]\)
  como \(\{ a_1, a_2, \dotsc, a_k \}\),
  con \(1 \le a_1 < a_2 < \dotsb < a_k \le n\).
  La restricción que no contenga elementos adyacentes
  se traduce en \(a_{r + 1} \ge a_r + 2\)
  para \(1 \le r < k\).
  Definamos nuevas variables:
  \begin{align*}
    d_1
      &= a_1 - 1 \\
    d_{r + 1}
      &= a_{r + 1} - a_r - 2 \quad \text{para \(1 \le r < k\)} \\
    d_{k + 1}
      &= n - a_k
  \end{align*}
  Es claro que la restricción es que \(d_r \ge 0\),
  y suman \(n - (k - 1) \cdot 2 - 1 = n - 2 k + 1\).
  Por lo anterior,
  el número de soluciones a esto es:
  \begin{equation*}
    \multiset{k + 1}{n - 2 k + 1}
      = \binom{n - k + 1}{k}
  \end{equation*}
  Esto coincide con los valores obtenidos antes.

  Una aplicación simple de los resultados anteriores
  es la siguiente:
  \begin{theorem}
    \index{combinatoria!numero de funciones@número de funciones|see{función!número}}
    \index{funcion@función!numero@número|textbfhy}
    \label{theo:numero-funciones}
    Sean \(\mathcal{X}\) e \(\mathcal{Y}\) conjuntos finitos.
    Entonces el número total
    de funciones
      \(f \colon \mathcal{X} \rightarrow \mathcal{Y}\) es:
    \begin{equation*}
      \lvert \mathcal{Y} \rvert^{\lvert \mathcal{X} \rvert}
    \end{equation*}
  \end{theorem}
  \begin{proof}
    Supongamos que \(\lvert \mathcal{X} \rvert = m\).
    Entonces podemos considerar esta situación
    como contar las tuplas
    \((f(1), f(2), \dotsc, f(m))\),
    en las cuales cada elemento
    toma un valor de entre \(\lvert \mathcal{Y} \rvert = n\),
    con lo que por la regla del producto hay \(n^m\) funciones.
  \end{proof}
  Es por este resultado que una notación común
  para el conjunto de funciones de \(\mathcal{X}\) a \(\mathcal{Y}\)
  es \(\mathcal{Y}^{\mathcal{X}}\).

  Una manera de describir un subconjunto \(\mathcal{S}\)
  de un conjunto \(\mathcal{U}\)
  es mediante su \emph{función característica}%
    \index{funcion caracteristica@función característica|see{conjunto!función característica}}%
    \index{conjunto!funcion caracteristica@función característica|textbfhy}
  \(\chi_{\mathcal{S}} \colon \mathcal{U} \rightarrow \{0, 1\}\),
  donde \(\chi_{\mathcal{S}}(u) = 0\)
  significa que \(u\) no pertenece al subconjunto,
  y \(\chi_{\mathcal{S}}(u) = 1\) que pertenece.
  Esta forma de ver las cosas lleva a:
  \begin{corollary}
    \index{combinatoria!numero de subconjuntos@número de subconjuntos|see{conjunto!subconjunto!número}}
    \index{conjunto!subconjunto!numero@número|textbfhy}
    \label{cor:numero-subconjuntos}
    Sea \(\mathcal{A}\) un conjunto finito.
    Entonces hay \(2^{\lvert \mathcal{A} \rvert}\)
    subconjuntos de \(\mathcal{A}\).
  \end{corollary}
  \begin{proof}
    Aplicar el teorema~\ref{theo:numero-funciones}
    al conjunto de funciones características.
  \end{proof}
  Es por esta razón que el conjunto
  de los subconjuntos de \(\mathcal{A}\)
  suele anotarse \(2^{\mathcal{A}}\).
  \begin{corollary}
    \index{combinatoria!numero de relaciones@número de relaciones|see{relación!número}}
    \index{relacion@relación!numero@número|textbfhy}
    \label{cor:numero-relaciones}
    Sean \(\mathcal{A}\) y \(\mathcal{B}\) conjuntos finitos.
    Entonces hay \(2^{\lvert \mathcal{A} \rvert
		      \cdot \lvert \mathcal{B} \rvert}\)
    relaciones de \(\mathcal{A}\) a \(\mathcal{B}\).
  \end{corollary}
  \begin{proof}
    \index{relacion@relación}
    Una relación entre \(\mathcal{A}\) y \(\mathcal{B}\)
    no es más que un subconjunto
    de \(\mathcal{A} \times \mathcal{B}\),
    aplicando la regla del producto
    y luego~(\ref{cor:numero-subconjuntos}) obtenemos lo prometido.
  \end{proof}

  Otro caso importante es contabilizar el número de inyecciones.
  \begin{theorem}
    \index{combinatoria!numero de inyecciones@número de inyecciones|see{función!inyectiva!nùmero}}
    \index{funcion@función!inyectiva!numero@número|textbfhy}
    \label{theo:numero-inyecciones}
    Sean \(\mathcal{X}\) e \(\mathcal{Y}\) conjuntos finitos,
    de cardinalidades \(\lvert \mathcal{X} \rvert = m\)
    e \(\lvert \mathcal{Y} \rvert = n\).
    Entonces el número total de funciones inyectivas
    \(i \colon \mathcal{X} \rightarrow \mathcal{Y}\)
    es \(n^{\underline{m}} = n! / (n - m)!\)
  \end{theorem}
  \begin{proof}
    Si es una inyección,
    no se repiten valores de la función
    (y por tanto \(m \le n\)).
    Si consideramos que \(\mathcal{X}\) son índices
    (definen las posiciones),
    estamos frente a permutaciones de \(n\) elementos
    de los que se eligen \(m\),
    vale decir es:
    \begin{equation*}
      P(n, m)
	= n^{\underline{m}}
	= \frac{n!}{(n - m)!}
      \qedhere
    \end{equation*}
  \end{proof}
  Directamente resulta:
  \begin{corollary}
    \index{combinatoria!numero de biyecciones@número de biyecciones|see{biyección!número}}
    \index{biyeccion@biyección!numero@número|textbfhy}
    \label{cor:numero-biyecciones}
    Sean \(\mathcal{X}\) e \(\mathcal{Y}\) conjuntos finitos
    tales que
      \(\lvert \mathcal{X} \rvert = \lvert \mathcal{Y} \rvert = n\).
    Entonces el número de biyecciones
      \(b \colon \mathcal{X} \rightarrow \mathcal{Y}\)
    es \(n!\).
  \end{corollary}
  \begin{proof}
    Para el caso \(n = m\)
    el teorema~\ref{theo:numero-inyecciones}
    da \(n^{\underline{n}} = n!\).
  \end{proof}

  Otra forma de interpretar
  el corolario~\ref{cor:numero-biyecciones}
  es que hay \(n!\) maneras de ordenar \(n\) elementos diferentes.

  Los números de combinaciones cumplen una colección inmensa
  de equivalencias curiosas.
  \begin{theorem}[Identidad de Pascal]
    \index{Pascal, identidad de|textbfhy}
    \label{theo:identidad-Pascal}
    Para \(n, k \in \mathbb{N}\) se cumplen:
    \begin{align*}
      \binom{n}{0}
	&= \binom{n}{n} = 1 \\
      \binom{n + 1}{k + 1}
	&= \binom{n}{k + 1} + \binom{n}{k}
     \end{align*}
  \end{theorem}
  \begin{proof}
    Primero:
    \begin{align*}
      \binom{n}{n}
	&= \binom{n}{n - n}
	 = \binom{n}{0} \\
      \binom{n}{0}
	&= \frac{n!}{n! \; 0!}
	 = 1
    \end{align*}
    Por el otro lado,
    podemos considerar que \(\binom{n + 1}{k + 1}\)
    corresponde a elegir \(k + 1\) elementos de entre \(n + 1\),
    cosa que se puede hacer fijando uno de los elementos,
    y luego considerar aquellos conjuntos de \(k + 1\) elementos
    que lo incluyen
    (corresponde a elegir los demás \(k\)
     de entre los \(n\) restantes,
     hay \(\binom{n}{k}\) casos de éstos),
    y los que no
    (corresponde a elegir \(k + 1\) elementos de entre los \(n\)
     que son elegibles,
     hay \(\binom{n}{k + 1}\) de estos casos).
    Como el conjunto de los subconjuntos
    que incluyen al elemento seleccionado
    y los que no son disjuntos,
    podemos aplicar la regla de la suma
    para obtener la recurrencia indicada.
  \end{proof}
  Una demostración alternativa es:
  \begin{proof}
    Primeramente,
    siempre es:
    \begin{align*}
      \binom{n}{0}
	&= \frac{n^{\underline{0}}}{0!}
	= 1 \\
      \binom{n}{n}
	&= \frac{n^{\underline{n}}}{n!}
	= \frac{n!}{n!}
	= 1
    \end{align*}
    Luego:
    \begin{align*}
      \binom{n}{k + 1} + \binom{n}{k}
	&= \frac{n^{\underline{k + 1}}}{(k + 1)!}
	     + \frac{n^{\underline{k}}}{k!} \\
	&= \frac{n^{\underline{k}} (n - k)
		   + (k + 1) n^{\underline{k}}}
		{(k + 1)!} \\
	&= \frac{n^{\underline{k}} (n + 1)}{(k + 1)!} \\
	&= \frac{(n + 1)^{\underline{k + 1}}}{(k + 1)!} \\
	&= \binom{n + 1}{k + 1}
      \qedhere
    \end{align*}
  \end{proof}
  \noindent
  Nótese que salvo en \(\binom{n}{n} = 1\)
  no presupone \(n \in \mathbb{N}_0\).

  Un resultado extremadamente importante
  es el que sigue:
  \begin{theorem}[Binomio]
    \index{binomio, teorema del|textbfhy}
    \label{theo:binomio}
    Para \(n \in \mathbb{N}\) tenemos:
    \begin{equation*}
      (a + b)^n
	= \sum_{0 \le k \le n} \binom{n}{k} a^k b^{n - k}
    \end{equation*}
  \end{theorem}
  \begin{proof}
    Por inducción sobre \(n\).%
      \index{demostracion@demostración!induccion@inducción}
    \begin{description}
    \item[Base:]
      Cuando \(n = 0\),
      tenemos:
      \begin{equation*}
	\sum_{0 \le k \le 0}
	  \binom{0}{k} a^k b^{0 - k}
	  = \binom{0}{0} a^0 b^0
	  = 1
      \end{equation*}
    \item[Inducción:]
      Nótese que en las sumatorias siguientes
      el rango de las sumas
      es exactamente los índices
      para los cuales
      no se anulan los coeficientes binomiales respectivos,
      con lo que podemos obviar los límites de las sumas.

      Tenemos:
      \begin{align*}
	(a + b)^{n + 1}
	  &= (a + b)^n \cdot (a + b) \\
	  &= \left(\sum_k \binom{n}{k} a^k b^{n - k}\right)
	       \cdot (a + b) \\
	  &= \sum_k \binom{n}{k} a^{k + 1} b^{n - k}
	       + \sum_k \binom{n}{k} a^k b^{n + 1 - k} \\
	  &= \sum_k \binom{n}{k - 1} a^k b^{n + 1 - k}
	       + \sum_k \binom{n}{k} a^k b^{n + 1 - k} \\
	  &= \sum_k
	       \left(
		 \binom{n}{k - 1}
		   + \binom{n}{k}
	       \right) \, a^k b^{n + 1} \\
	  &= \sum_k \binom{n + 1}{k} \, a^k b^{n + 1 - k} \\
	  &= \sum_{0 \le k \le n + 1}
	       \binom{n + 1}{k} \, a^k b^{n + 1 - k}
      \end{align*}
    \end{description}
    Por inducción,
    vale para \(n \in \mathbb{N}_0\).
  \end{proof}
  Por el teorema~\ref{theo:binomio}
  es que los números \(\binom{n}{k}\)
  se llaman \emph{coeficientes binomiales}.%
    \index{coeficiente binomial|textbfhy}

  Otro resultado importante es el siguiente.
  \begin{theorem}[Multinomio]
    \index{multinomio, teorema del|textbfhy}
    \label{theo:multinomio}
    Para \(n \in \mathbb{N}\)
    tenemos:
    \begin{equation*}
      (a_1 +a_2 + \dotsb a_r)^n
	 = \sum_{k_1 + k_2 + \dotsb + k_r = n}
	      \binom{n}{k_1, k_2, \dotsc, k_r}
		  a_1^{k_1} a_2^{k_2} \dotsb a_r^{k_r}
    \end{equation*}
    donde:
    \begin{equation*}
      \binom{n}{k_1, k_2, \dotsc, k_r}
	= \frac{n!}{k_1! k_2! \dotsm k_r!}
    \end{equation*}
    Esta expresión está definida
    solo si \(n = k_1 + k_2 + \dotsb + k_r\).
  \end{theorem}
  \begin{proof}
    Por inducción fuerte sobre \(r\).%
      \index{demostracion@demostración!induccion@inducción}
    \begin{description}
    \item[Base:]
      Cuando \(r = 2\),
      se reduce al teorema del binomio:
      \begin{equation*}
	\sum_{k_1 + k_2 = n}
	     \binom{n}{k_1, k_2} a_1^{k_1} a_2^{k_2}
	  = \sum_{0 \le k \le n}
	      \binom{n}{k, n - k} a_1^k a_2^{n - k}
	  = \sum_{0 \le k \le n} \binom{n}{k} a_1^k a_2^{n - k}
      \end{equation*}
    \item[Inducción:]
      Tenemos:
      \begin{align*}
	((a_1 &+ \dotsb + a_r) + a_{r + 1})^n \\
	  &= \sum_{0 \le k_{r + 1} \le n}
	       \binom{n}{k_{r + 1}} \,
		 \left(
		   \sum_{k_1 + k_2 + \dotsb + k_r = n - k_{r + 1}}
		   \binom{n - k_{r + 1}}{k_1, k_2, \dotsc, k_r}
		     a_1^{k_1} a_2^{k_2} \dotsm a_r^{k_r}
		 \right)
	       \cdot a_{r + 1}^{n - k_{r + 1}} \\
	  &= \sum_{k_1 + \dotsb + k_{r + 1} = n}
	       \binom{n}{k_{r + 1}}
		 \binom{n - k_{r + 1}}{k_1, k_2, \dotsc, k_r} \,
		   a_1^{k_1} a_2^{k_2} \dotsm a_{k_{r + 1}} \\
	  &= \sum_{k_1 + \dotsb + k_{r + 1} = n}
	       \binom{n}{k_1, k_2, \dotsc, k_{r + 1}} \,
		 a_1^{k_1} a_2^{k_2} \dotsm a_{k_{r + 1}}
      \end{align*}
      Acá usamos:
      \begin{align*}
	\binom{n}{k_{r + 1}}
	  \binom{n - k_{r + 1}}{k_1, k_2, \dotsc, k_r}
	  &= \frac{n!}{k_{r + 1}! (n - k_{r + 1})!}
	       \cdot \frac{(n - k_{r + 1})!}
			  {k_1! \; k_2! \dotsm k_r!} \\
	  &= \frac{n!}{k_1! \; k_2! \dotsm k_{r + 1}!} \\
	  &= \binom{n}{k_1, k_2, \dotsc, k_{r + 1}}
      \end{align*}
    \end{description}
    Por inducción es válido para \(r \ge 2\),
    y claramente es válido para \(r = 0\) y \(r = 1\),
    con lo que vale para \(r \in \mathbb{N}_0\).
  \end{proof}
  Por razones obvias,
  a los \(\binom{n}{k_1, k_2, \dotsc, k_r}\)
  se les llama \emph{coeficientes multinomiales},%
    \index{coeficiente multinomial|textbfhy}
  y tenemos también:
  \begin{equation*}
    \binom{n}{k, n - k} = \binom{n}{k}
  \end{equation*}

% manos-poker.tex
%
% Copyright (c) 2009-2014 Horst H. von Brand
% Derechos reservados. Vea COPYRIGHT para detalles

\section{Manos de poker}
\label{sec:manos-poker}
\index{manos de poker|see{combinatoria!manos de poker}}
\index{combinatoria!manos de poker|textbfhy}

% Fixme: Revisar las reglas, contabilizar _todas_ las manos
%	 (hay que restar alternativas repetidas, etc)
%	 ¿Algún juego más complejo (2 mazos, ...)?

  Nuestro siguiente tema de interés es contar subconjuntos
  que cumplen ciertas restricciones.
  Como conjuntos,
  siguiendo a Lehman, Leighton y Meyer~%
    \cite{lehman15:_mathem_comput_scien},
  usaremos manos de \foreignlanguage{english}{poker}.

  En \foreignlanguage{english}{poker}
  a cada jugador se le da una \emph{mano} de cinco cartas,
  elegidas del mazo inglés,
  formado por cuatro \emph{pintas}:%
    \index{carta!pinta}
  Pica (\(\spadesuit\)),
    \index{carta!pinta!pica (\(\spadesuit\))|textbfhy}
  corazón (\(\heartsuit\)),
    \index{carta!pinta!corazon (\(\heartsuit\))@corazón (\(\heartsuit\))|textbfhy}
  trébol (\(\clubsuit\))
    \index{carta!pinta!trebol (\(\clubsuit\))@trébol (\(\clubsuit\))|textbfhy}
  y diamante (\(\diamondsuit\));
    \index{carta!pinta!diamante (\(\diamondsuit\))|textbfhy}
  en cada pinta hay trece \emph{valores}:%
    \index{carta!valor|textbfhy}
  As, \(2\) a \(10\),
  \foreignlanguage{english}{Jack},
  \foreignlanguage{english}{Queen}
  y \foreignlanguage{english}{King}.
  El número total de manos posibles es:
  \begin{equation*}
    \binom{52}{5} = 2\,598\,960
  \end{equation*}

  Como estrategia general,
  buscaremos secuencias que describan las manos que queremos contar
  (porque contar secuencias es fácil),
  y nos aseguraremos que hay una biyección
  (o que haya alguna otra relación clara,
   como un mapa \(2\) a \(1\))
  entre descripciones
  y manos.

\subsection{Royal Flush}
\label{sec:royal-flush}

  Es la mano más alta en \foreignlanguage{english}{poker}.
  Consta de \foreignlanguage{english}{As},
  \foreignlanguage{english}{King},
  \foreignlanguage{english}{Queen},
  \foreignlanguage{english}{Jack}, \(10\) de la misma pinta,
  por ejemplo:
  \begin{equation*}
    \begin{array}{@{\{}*{4}{c@{\;\;}}c@{\}}}
      A \spadesuit & K \spadesuit & Q \spadesuit &
	J \spadesuit & 10 \spadesuit
    \end{array}
  \end{equation*}
  Está claro que hay una mano de éstas para cada pinta,
  con lo que hay exactamente \(4\).

\subsection{Straight Flush}
\label{sec:straight-flush}

  Consta de \(5\) cartas de la misma pinta en secuencia,
  donde \foreignlanguage{english}{As} cuenta como \(1\)
  (no después de \foreignlanguage{english}{King},
   como en el \foreignlanguage{english}{Royal Flush}).
  Ejemplos son:
  \begin{equation*}
    \begin{array}{@{\{}*{4}{c@{\;\;}}c@{\}}}
      8 \spadesuit & 9 \spadesuit & 10 \spadesuit &
	J \spadesuit & Q \spadesuit \\
      A \heartsuit & 2 \heartsuit & 3 \heartsuit  &
	4 \heartsuit & 5\heartsuit \\
      3 \clubsuit  & 4 \clubsuit  & 5 \clubsuit	  &
	6\clubsuit  & 7\clubsuit
    \end{array}
  \end{equation*}
  Estas manos podemos describirlas mediante
  una secuencia que indica:
  \begin{itemize}
  \item
    El valor de la primera carta en la secuencia.
    Este puede elegirse de \(9\) maneras
    (entre \(1\) y \(9\)).
  \item
    La pinta,
    que puede elegirse de \(4\) maneras.
  \end{itemize}
  En nuestros ejemplos:
  \begin{equation*}
    \begin{array}{l@{{} \longleftrightarrow \{}*{4}{c@{\;\;}}c@{\}}}
      (8, \spadesuit) &
	8 \spadesuit  & 9 \spadesuit & 10 \spadesuit &
	J \spadesuit  & Q \spadesuit \\
      (1, \heartsuit) &
	A \heartsuit  & 2 \heartsuit & 3 \heartsuit  &
	4 \heartsuit  & 5 \heartsuit \\
      (3, \clubsuit)  &
	3 \clubsuit   & 4 \clubsuit  & 5 \clubsuit   &
	6 \clubsuit   & 7 \clubsuit
    \end{array}
  \end{equation*}
  Por la regla del producto,%
    \index{combinatoria!regla del producto}
  el número total de estas manos es:
  \begin{equation*}
    9 \cdot \binom{4}{1} = 36
  \end{equation*}
  Como esto no describe un \foreignlanguage{english}{Royal Flush},
  no hace falta ningún ajuste adicional.

\subsection{Four of a Kind}
\label{sec:four-of-a-kind}

  Esta es una mano con cuatro cartas del mismo valor.
  Por ejemplo:
  \begin{align*}
    \begin{array}{@{\{}*{4}{c@{\;\;}}c@{\}}}
      8 \spadesuit  & 8 \heartsuit & 8 \clubsuit &
	8 \diamondsuit	 & 5 \diamondsuit \\
      2 \spadesuit  & 2 \heartsuit & 2 \clubsuit  &
	2 \diamondsuit	 & 3 \clubsuit
    \end{array}
  \end{align*}
  Para calcular cuántas de estas hay,
  armamos un mapa de secuencias a manos de este tipo
  y contamos las secuencias.
  En este caso,
  una mano queda descrita por:
  \begin{itemize}
  \item El valor que se repite.
  \item El valor de la quinta carta.
  \item La pinta de la quinta carta.
  \end{itemize}
  Hay una biyección entre secuencias de estos tres elementos
  y manos.
  En nuestros ejemplos,
  las correspondencias son:
  \begin{align*}
    \begin{array}{l@{{} \longleftrightarrow \{}*{4}{c@{\;\;}}c@{\}}}
      (8, 5, \diamondsuit) &
	8 \spadesuit & 8 \heartsuit & 8 \clubsuit & 8 \diamondsuit
	   & 5 \diamondsuit \\
      (2, 3, \clubsuit)	   &
	2 \spadesuit & 2 \heartsuit & 2 \clubsuit & 2 \diamondsuit
	   & 3 \clubsuit
    \end{array}
  \end{align*}
  Para el valor tenemos \(13\) posibilidades,
  para el valor de la quinta carta quedan \(12\) posibilidades,
  y hay \(4\) opciones para la pinta de la quinta carta.
  En total,
  usando la regla del producto,%
    \index{combinatoria!regla del producto}
  son \(13 \cdot 12 \cdot 4 = 624\) posibilidades.
  Hay \(1\) en \(2\,598\,960 / 624 = 4\,165\) manos,
  no sorprende que se considere muy buena.

\subsection{Full House}
\label{sec:full-house}

  Es una mano con tres cartas de un valor
  y dos de otro.
  Ejemplos:
  \begin{equation*}
    \begin{array}{@{\{}*{4}{c@{\;\;}}c@{\}}}
       2 \heartsuit & 2 \clubsuit & 2 \diamondsuit & Q \spadesuit
	  & Q \diamondsuit \\
       5 \spadesuit & 5 \clubsuit & 5 \diamondsuit & K \spadesuit
	  & K \heartsuit
    \end{array}
  \end{equation*}
  Nuevamente un mapa con secuencias:
  \begin{itemize}
  \item
    El valor del trío,
    que puede especificarse de \(13\) maneras.
  \item
    Las pintas del trío,
    que son elegir \(3\) de entre \(4\).
  \item
    El valor del par,
    que se puede tomar de \(12\) maneras.
  \item
    Las pintas del par,
    que se eligen \(2\) entre \(4\).
  \end{itemize}
  Las manos ejemplo corresponden con:
  \begin{equation*}
    \begin{array}{l@{{} \longleftrightarrow \{}*{4}{c@{\;\;}}c@{\}}}
      (2, \{\heartsuit, \clubsuit, \diamondsuit\},
       Q, \{\spadesuit, \diamondsuit\}) &
	 2 \heartsuit & 2 \clubsuit & 2 \diamondsuit & Q \spadesuit
	    & Q \diamondsuit \\
      (5, \{\spadesuit, \clubsuit, \diamondsuit\},
       K, \{\spadesuit, \heartsuit\})	&
	 5 \spadesuit & 5 \clubsuit & 5 \diamondsuit & K \spadesuit
	    & K \heartsuit
    \end{array}
  \end{equation*}
  Por la regla del producto%
    \index{combinatoria!regla del producto}
  el número de \foreignlanguage{english}{Full Houses} es entonces:
  \begin{equation*}
    13 \cdot \binom{4}{3} \cdot 12 \cdot \binom{4}{2} = 3\,744
  \end{equation*}

\subsection{Flush}
\label{sec:flush}

  Mano con \(5\) cartas de la misma pinta,
  como por ejemplo:
  \begin{equation*}
    \begin{array}{@{\{}*{4}{c@{\;\;}}c@{\}}}
      A \heartsuit & 3 \heartsuit & 4 \heartsuit &
	8 \heartsuit & K \heartsuit
    \end{array}
  \end{equation*}
  Esto se describe mediante la secuencia que da:
  \begin{itemize}
  \item
    Un conjunto de \(5\) valores,
    se eligen \(5\) de entre \(13\).
  \item
    Una pinta,
    se elige una entre \(4\).
  \end{itemize}
  En nuestro ejemplo:
  \begin{equation*}
    \begin{array}{l@{{} \longleftrightarrow \{}*{4}{c@{\;\;}}c@{\}}}
      (\{A, 3, 4, 8, K\}, \heartsuit) &
	 A \heartsuit & 3 \heartsuit & 4 \heartsuit &
	   8 \heartsuit & K \heartsuit
    \end{array}
  \end{equation*}
  De estas manos hay entonces:
  \begin{equation*}
    \binom{13}{5} \cdot \binom{4}{1} = 5\,148
  \end{equation*}
  Esto también describe al \foreignlanguage{english}{Royal Flush}
  y al \foreignlanguage{english}{Straight Flush},
  debemos restar éstos
  (regla de la suma):%
    \index{combinatoria!regla de la suma}
  \begin{equation*}
    5\,148 - 4 - 36 = 5\,108
  \end{equation*}

\subsection{Manos con dos pares}
\label{sec:dos-pares}

  Interesa calcular cuántas manos con dos pares hay,
  vale decir,
  dos cartas de un valor,
  dos cartas de otro valor,
  y una carta de un tercer valor.
  Ejemplos son:
  \begin{align*}
    \begin{array}{@{\{}*{4}{c@{\;\;}}c@{\}}}
      3 \heartsuit & 3 \diamondsuit & Q \spadesuit & Q \clubsuit
	 & 5 \diamondsuit \\
      9 \heartsuit & 9 \clubsuit    & K \spadesuit & K \diamondsuit
	 & 2 \spadesuit
    \end{array}
  \end{align*}
  Cada mano queda descrita por:
  \begin{itemize}
  \item
    El valor del primer par,
    puede elegirse de \(13\) maneras.
  \item
    Las pintas del primer par,
    se toman \(2\) de entre \(4\).
  \item
    El valor del segundo par,
    que se puede elegir de \(12\) maneras.
  \item
    Las pintas del segundo par,
    se eligen \(2\) entre \(4\).
  \item
    El valor de la carta extra,
    es uno de \(11\).
  \item
    La pinta de la carta extra,
    que es una de \(4\).
  \end{itemize}
  Se pensaría entonces que el número buscado es:
  \begin{equation*}
    13 \cdot \binom{4}{2} \cdot 12 \cdot \binom{4}{2}
     \cdot 11 \cdot \binom{4}{1}
  \end{equation*}
  ¡Esto es incorrecto!
  El mapa entre secuencias y manos no es una biyección,
  es \(2\) a \(1\)
  (podemos elegir cualquiera de los dos pares como primero).
  El valor correcto es:
  \begin{equation*}
    \frac{13 \cdot \binom{4}{2} \cdot 12 \cdot \binom{4}{2}
	    \cdot 11 \cdot \binom{4}{1}}{2}
      = 123\,552
  \end{equation*}
  No es una mano particularmente buena.

  Pero además es perturbadora:
  Es fácil omitir el detalle de que no es una biyección.
  Hay dos salidas:
  \begin{enumerate}
  \item
    Cada vez que se usa
    un mapa \(f \colon \mathcal{A} \rightarrow \mathcal{B}\),
    verifique que el mismo número de elementos de \(\mathcal{A}\)
    llevan a cada elemento de \(\mathcal{B}\);
    si este número es \(k\),
    aplique la regla de división con \(k\).%
      \index{combinatoria!regla de division@regla de división}
  \item
    Intente otra forma de resolver el problema.
    Muchas veces hay varias formas de enfrentarlo --
    y debieran dar el mismo resultado.
    Claro que suele ocurrir que métodos distintos
    dan resultados que se \emph{ven} diferentes,
    aunque resultan ser iguales.
  \end{enumerate}

  Arriba usamos un método,
  veamos un segundo:
  Hay una biyección entre estas manos y secuencias que especifican:
  \begin{itemize}
  \item
    Los valores de los dos pares,
    se pueden elegir \(2\) entre \(13\).
  \item
    Las pintas del par de menor valor,
    se eligen \(2\) entre \(4\).
  \item
    Las pintas del par de mayor valor,
    se eligen \(2\) entre \(4\).
  \item
    El valor de la carta extra,
    es \(1\) entre \(11\).
  \item
    La pinta de la carta extra,
    es \(1\) entre \(4\).
  \end{itemize}
  Para nuestro ejemplo:
  \begin{equation*}
    \begin{array}{l@{{} \longleftrightarrow \{}*{4}{c@{\;\;}}c@{\}}}
      (\{3, Q\}, \{\diamondsuit, \heartsuit\},
	 \{\spadesuit, \clubsuit\},
       5, \diamondsuit\}) &
	 3 \diamondsuit & 3 \heartsuit & Q \spadesuit &
	 Q \clubsuit  & 5 \diamondsuit \\
      (\{9, K\}, \{\clubsuit, \heartsuit\},
	 \{\spadesuit, \heartsuit\},
       2, \spadesuit\})	  &
	 9 \clubsuit	& 9 \heartsuit & K \spadesuit &
	 K \diamondsuit & 2 \spadesuit
    \end{array}
  \end{equation*}
  Esto lleva a:
  \begin{equation*}
    \binom{13}{2} \cdot \binom{4}{2} \cdot \binom{4}{2} \cdot 11
      \cdot \binom{4}{1}
  \end{equation*}
  Es el mismo resultado anterior,
  claro que escrito de forma ligeramente diferente.

\subsection{Manos con todas las pintas}
\label{sec:todas-las-pintas}

  Buscamos el número de manos con cartas de todas las pintas.
  Por ejemplo:
  \begin{equation*}
    \begin{array}{@{\{}*{4}{c@{\;\;}}c@{\}}}
      7 \heartsuit & 8 \diamondsuit & K \clubsuit &
	A \spadesuit & 3 \heartsuit
    \end{array}
  \end{equation*}
  Esto podemos describirlo mediante:
  \begin{itemize}
  \item
    Los valores de las cartas de cada pinta,
    o sea \(13 \cdot 13 \cdot 13 \cdot 13\) posibilidades.
  \item
    El valor de la carta extra,
    con \(12\) selecciones posibles.
  \item
    La pinta de la carta extra,
    \(4\) opciones.
  \end{itemize}
  La mano del ejemplo se describe mediante:
  \begin{equation*}
    \begin{array}{l@{{} \longleftrightarrow \{}*{4}{c@{\;\;}}c@{\}}}
      (A, 7, 8, K, 3, \heartsuit) &
	 7 \heartsuit & 8 \diamondsuit & K \clubsuit &
	 A \spadesuit & 3 \heartsuit
    \end{array}
  \end{equation*}
  El problema es nuevamente que esto no es una biyección,
  en el ejemplo podemos considerar \(3\heartsuit\) o \(7\heartsuit\)
  como la carta extra,
  y el mapa es \(2\) a \(1\).
  El número buscado es:
  \begin{equation*}
    \frac{13^4 \cdot 4 \cdot 12}{2}
      = 685\,464
  \end{equation*}

  Una forma alternativa
  es dar los valores del par de la misma pinta,
  y la pinta del par;
  y luego los valores de las tres cartas de las pintas restantes.
  Nuestro ejemplo se describe mediante:
  \begin{equation*}
    \begin{array}{l@{{} \longleftrightarrow \{}*{4}{c@{\;\;}}c@{\}}}
      (\{3, 7\}, \heartsuit, A, K, 8) &
	 7 \heartsuit & 8 \diamondsuit & K \clubsuit &
	 A \spadesuit & 3 \heartsuit
    \end{array}
  \end{equation*}
  Acá hemos supuesto
  el orden \(\spadesuit, \heartsuit, \clubsuit, \diamondsuit\)
  de las pintas.
  Esto da nuevamente:
  \begin{equation*}
    \binom{13}{2} \cdot \binom{4}{1} \cdot 13 \cdot 13 \cdot 13
      = 685\,464
  \end{equation*}

\subsection{Manos con valores diferentes}
\label{sec:poker-different-values}

  Nos interesa ahora contar el número de manos
  en las cuales todos los valores son diferentes.
  Una forma alternativa de describir estas manos
  es diciendo que no tienen pares.
  Veremos varias maneras para obtener esto.

  Una primera forma de enfrentar esto
  es considerar
  que la primera carta se puede elegir de entre \(52\),
  la segunda entre las \(48\) que no tienen el valor de la primera,
  y así sucesivamente.
  Pero el hablar de la primera, segunda y sucesivas cartas
  presupone orden,
  alerta del riesgo de contar demás.%
    \index{combinatoria!sobrecontar}
  Como son todas diferentes,
  basta dividir por el número de ordenamientos de \(5\) cartas,
  vale decir \(5! = 120\).
  O sea:
  \begin{equation*}
    \frac{52 \cdot 48 \cdot 44 \cdot 40 \cdot 36}{5!}
      = 1\,317\,888
  \end{equation*}

  Una solución alternativa
  consiste en seleccionar los valores de las \(5\) cartas,
  entre los \(13\) valores posibles,
  y luego a cada carta asignarle una pinta entre \(4\).
  Esto da:
  \begin{equation*}
    \binom{13}{5} \cdot 4^5
      = 1\,317\,888
  \end{equation*}

%%% Local Variables:
%%% mode: latex
%%% TeX-master: "clases"
%%% End:


% tao-bookkeeper.tex
%
% Copyright (c) 2009, 2011-2014 Horst H. von Brand
% Derechos reservados. Vea COPYRIGHT para detalles

\section{El tao de \texttt{BOOKKEEPER}}
\label{sec:tao-bookkeeper}
\index{Tao de \texttt{BOOKKEEPER}|see{combinatoria!secuencias con repeticiones}}
\index{combinatoria!secuencias con repeticiones|textbfhy}

  Veremos maneras de contar secuencias que incluyen elementos repetidos.
  Para llegar a la iluminación
  siguiendo los pasos de Lehman, Leighton y Meyer~%
    \cite{lehman15:_mathem_comput_scien},
  meditemos sobre la palabra \(\mathtt{BOOKKEEPER}\).
  \begin{enumerate}
  \item
    ¿De cuántas maneras se pueden ordenar las letras de \(\mathtt{POKE}\)?
  \item
    ¿De cuántas maneras se pueden ordenar las letras de
    \(\mathtt{B} \mathtt{O}_1 \mathtt{O}_2 \mathtt{K}\)?
    (Note que los subíndices
     hacen que las dos \(\mathtt{O}\) sean distintas).
  \item
    Pequeño saltamontes,
    mapea los ordenamientos de
    \(\mathtt{B} \mathtt{O}_1 \mathtt{O}_2 \mathtt{K}\)
    (las \(\mathtt{O}\) son diferentes)
    a \(\mathtt{BOOK}\)
    (las dos \(\mathtt{O}\) son idénticas).
    ¿Qué clase de mapa es este?
  \item
    ¡Muy bien,
    joven maestro!
    Dime ahora,
    ¿de cuántas maneras pueden ordenarse las letras de
    \(\mathtt{K} \mathtt{E}_1 \mathtt{E}_2 \mathtt{P}
      \mathtt{E}_3 \mathtt{R}\)?
  \item
    Mapea cada ordenamiento de
    \(\mathtt{K} \mathtt{E}_1 \mathtt{E}_2 \mathtt{P}
      \mathtt{E}_3 \mathtt{R}\)
    a un ordenamiento de \(\mathtt{KEEPER}\)
    tal que,
    borrando los subíndices,
    lista todos los que leen \(\mathtt{REPEEK}\).
    ¿Que clase de mapa es este?
  \item
    En vista de lo anterior,
    ¿cuántos ordenamientos de
    \(\mathtt{\foreignlanguage{english}{KEEPER}}\) hay?
  \item
    \emph{¡Ahora ya estás en posición de enfrentarte
      al terrible \(\mathtt{BOOKKEEPER}\)!}
    ¿Cuántos ordenamientos de
    \(\mathtt{B} \mathtt{O}_1 \mathtt{O}_2 \mathtt{K}_1
      \mathtt{K}_2 \mathtt{E}_1 \mathtt{E}_2 \mathtt{P}
      \mathtt{E}_3 \mathtt{R}\)
    hay?
  \item
    ¿Cuántos ordenamientos de
    \(\mathtt{BOO} \mathtt{K}_1
      \mathtt{K}_2 \mathtt{E}_1 \mathtt{E}_2 \mathtt{P}
      \mathtt{E}_3 \mathtt{R}\)
    hay?
  \item
    ¿Cuántos ordenamientos de
    \(\mathtt{BOOKKEEPER}\) hay?
  \item
    ¿Cuántos ordenamientos de
    \(\mathtt{VOODOODOLL}\) hay?
  \item
    Esta es muy importante,
    pequeño saltamontes.
    ¿Cuántas secuencias de \(n\) bits
    tienen \(k\) ceros y \(n - k\) unos?
  \end{enumerate}
  Prender subíndices,
  apagar subíndices.
  Ese es el tao de \(\mathtt{BOOKKEEPER}\).%
    \index{coeficiente multinomial}

%%% Local Variables:
%%% mode: latex
%%% TeX-master: "clases"
%%% End:


% juegos-completos.tex
%
% Copyright (c) 2011-2014 Horst H. von Brand
% Derechos reservados. Vea COPYRIGHT para detalles

\section{Juegos completos de poker}
\label{sec:poker-juegos}
\index{juegos de poker|see{combinatoria!juegos de poker}}
\index{combinatoria!juegos de poker|textbfhy}

  Los señores George G.~Akeley,
  Robert Blake,
  Randolph Carter
  y Edward P.~Davis
  juegan poker.
  Interesa saber
  de cuántas maneras se pueden repartir las \(52\)~cartas
  en \(4\)~manos de \(5\)~cartas,
  quedando \(32\)~cartas en el mazo.

  Podemos atacar el problema considerando que Akeley
  elige \(5\) cartas de las \(52\),
  que Blake elige \(5\) de las restantes,
  y así sucesivamente.
  El resultado es:
  \begin{align*}
    \binom{52}{5}
	\cdot \binom{52 - 5}{5}
	\cdot \binom{52 - 2 \cdot 5}{5}
	\cdot \binom{52 - 3 \cdot 5}{5}
      &= \frac{52!}{5! \, 5! \, 5! \, 5! \, 32!} \\
      &= \binom{52}{5 \; 5 \; 5 \; 5 \; 32}
  \end{align*}

  Si consideramos las cartas en un orden cualquiera,
  podemos representar la distribución
  asociando cada posición con quien la tiene.
  De esta forma,
  tenemos una biyección
  entre secuencias de \(52\)~dueños de las cartas respectivas
  y las distribuciones de las cartas.
  Para simplificar notación,
  denotamos a los caballeros
  por las primeras letras de sus apellidos,
  y el mazo por \(\mathtt{M}\).
  Buscamos entonces el número de secuencias de \(52\)~símbolos
  elegidos
  entre
    \(\{\mathtt{A}, \mathtt{B}, \mathtt{C}, \mathtt{D},
	\mathtt{M}\}\)
  formadas con \(5\)~\(\mathtt{A}\),
  \(5\)~\(\mathtt{B}\),
  \(5\)~\(\mathtt{C}\),
  \(5\)~\(\mathtt{D}\) y \(32\)~\(\mathtt{M}\).
  Esto nos lleva directamente al resultado anterior
  al aplicar el tao,
  sección~\ref{sec:tao-bookkeeper}.%
    \index{combinatoria!secuencias con repeticiones}

%%% Local Variables:
%%% mode: latex
%%% TeX-master: "clases"
%%% End:


% secuencias-restringidas.tex
%
% Copyright (c) 2009, 2012-2014 Horst H. von Brand
% Derechos reservados. Vea COPYRIGHT para detalles

\section{Secuencias con restricciones}
\label{sec:secuencias-restringidas}
\index{combinatoria!secuencias restringidas}

  También interesa poder contar reordenamientos
  en los cuales hay ciertas restricciones,
  como elementos en posiciones fijas
  o elementos en posiciones fijas relativas entre sí.

  Seguimos con nuestro ejemplo de \(\mathtt{BOOKKEEPER}\).
  \begin{enumerate}
  \item
    ¿De cuántas formas se puede escribir esta palabra
    si las dos \(\mathtt{O}\) siempre están juntas?
  \item
    ¿Cuántas formas hay de ordenar las letras
    si siempre están \(\mathtt{BPR}\) juntas en ese orden?
  \item
    ¿Si \(\mathtt{BPR}\) están juntas,
    pero en cualquier orden?
  \item
    ¿En cuántos aparecen \(\mathtt{BPR}\) en ese orden,
    no necesariamente juntas?
    % Podemos elegir las posiciones de BPR en \binom{10}{3} formas
    % (el orden está predeterminado), las demás por multinomio
  \item
    ¿Cuántas maneras hay de ordenar las letras
    si las \(\mathtt{O}\) están separadas por una letra?
  \item
    ¿Cuántas maneras hay de ordenarlas
    si las \(\mathtt{E}\) están separadas siempre por una letra?
  \item
    ¿De cuántas maneras se pueden ordenar las letras
    si las \(5\) vocales están al principio y las \(5\) consonantes al final?
  \item
    ¿Y si las vocales están en las posiciones impares?
  \item
    ¿Que pasa si las vocales
    ocupan las posiciones \(2\), \(3\), \(6\), \(7\), \(9\)?
  \item
    ¿Cuántos ordenamientos hay
    en los cuales las vocales están todas juntas?
  \item
    ¿Cuántos ordenamientos con \(\mathtt{B}\) en una posición impar hay?
  \item
    ¿Y si solo pedimos una \(\mathtt{O}\) en una posición par?
  \item
    ¿Cuántos ordenamientos hay con las tres \(\mathtt{E}\)
    en posiciones impares?
  \item
    ¿Cuántos órdenes tienen la \(\mathtt{B}\) separadas de la \(\mathtt{R}\)
    por dos letras?
  \item
    ¿Cuántos tienen la \(\mathtt{B}\) separadas de la \(\mathtt{R}\)
    por \(k\) letras?
  \end{enumerate}

  Otra situación,
  que puede enfrentarse mediante nuestra estrategia general
  de construir el objeto de interés
  en fases independientes,%
    \index{combinatoria!regla del producto}
  se presenta si queremos determinar
  el número de maneras de ordenar las letras de \(\mathtt{MISSISSIPPI}\)
  de forma que las vocales siempre estén separadas por consonantes.
  Vemos que hay \(4\) \(\mathtt{I}\),
  lo que deja \(5\) espacios en los cuales distribuir las consonantes.
  Si llamamos \(x_0\) al número de consonantes
  antes de la primera \(\mathtt{I}\),
  \(x_1\) a \(x_3\) al número de consonantes entre \(\mathtt{I}\)
  y finalmente \(x_4\) al número de consonantes
  después de la última \(\mathtt{I}\),
  quedan la ecuación:
  \begin{equation*}
    x_0 + x_1 + x_2 + x_3 + x_4
      = 7
  \end{equation*}
  Restricciones son que \(x_0 \ge 0\),
  \(x_k \ge 1\) para \(1 \le k \le 3\)
  y \(x_4 \ge 0\).
  Si definimos nuevas variables \(y_0 = x_0\),
  \(y_k = x_k - 1\) para \(1 \le k \le 3\)
  e \(y_4 = x_4\),
  queda:
  \begin{equation*}
    y_0 + y_1 + y_2 + y_3 + y_4
      = 4
  \end{equation*}
  lo que nos lleva a contar multiconjuntos:%
    \index{multiconjunto!numero@número}
  El número de soluciones es el número de multiconjuntos
  de \(5\) elementos de los que tomamos \(4\).
  Luego ordenamos el multiconjunto de consonantes
  \(\{\mathtt{M}, \mathtt{S}^4, \mathtt{P}^2\}\),
  distribuyendo las consonantes según los tramos definidos anteriormente.
  Como estas dos decisiones
  (número de consonantes en cada tramo
   y orden de las consonantes)
  son independientes,
  aplicamos la regla del producto:
  \begin{equation*}
    \multiset{5}{4} \cdot \binom{7}{1, 4, 2}
      = 7\,350
  \end{equation*}

% Fixme: Anunciar PIE, etc

%%% Local Variables:
%%% mode: latex
%%% TeX-master: "clases"
%%% End:


%%% Local Variables:
%%% mode: latex
%%% TeX-master: "clases"
%%% End:

% estimaciones-asintoticas.tex
%
% Copyright (c) 2013-2014 Horst H. von Brand
% Derechos reservados. Vea COPYRIGHT para detalles

\chapter{Estimaciones asintóticas}
\label{cha:estim-asint}
\index{asintotica@asintótica}

  Es común requerir una estimación ajustada de alguna cantidad,
  como una suma
  o el valor aproximado del coeficiente de una serie horrible.
  Hay una variedad de técnicas disponibles para esta tarea,
  desde simples a complejas.
  Interesan estimaciones asintóticas
  que entreguen resultados numéricos precisos y simples
  para el número de estructuras combinatorias,
  generalmente en el ámbito de rendimiento de algoritmos
  con el objetivo de comparar alternativas
  o estimar los recursos requeridos para resolver un problema particular.
  Describiremos sólo algunos de los métodos más importantes.
  Resúmenes detallados de las principales técnicas
  ofrecen Odlyzko~\cite{odlyzko95:_asympt_enum_method}
  y Flajolet y Sedgewick~\cite{flajolet09:_analy_combin}.

\section{Estimar sumas}
\label{sec:estimar-sumas}
\index{asintotica@asintótica!suma}

  Primero una técnica simple,
  que ya habíamos visto
  (capítulo~\ref{cha:euler-maclaurin}).%
    \index{Euler-Maclaurin, formula de@Euler-Maclaurin, fórmula de}
  \begin{theorem}
    \label{theo:sum-integral}
    Sea \(f(x)\) una función continua y monótona.
    Entonces:
    \begin{equation*}
      \sum_{0 \le k \le n} f(n)
	\approx \int_0^n f(x) d x
	    + \frac{f(0) + f(n)}{2}
    \end{equation*}
  \end{theorem}
  \begin{proof}
    Si \(f(x)\) es monótona creciente:
    \begin{alignat*}{2}
      f(\lfloor x \rfloor)
	&\le f(x)
	&&\le f(\lceil x \rceil) \\
     \int_0^n f(\lfloor x \rfloor) d x
	&\le \int_0^n f(x) d x
	&&\le \int_0^n f(\lceil x \rceil) d x \\
     \sum_{0 \le k \le n - 1} f(k)
	&\le \int_0^n f(x) d x
	&&\le \sum_{0 \le k \le n - 1} f(k + 1) d x \\
     \sum_{0 \le k \le n} f(k) - f(n)
	&\le \int_0^n f(x) d x
	&&\le \sum_{0 \le k \le n} f(k) - f(0)
    \end{alignat*}
    y directamente:
    \begin{equation*}
      \int_0^n f(x) d x + f(0)
	\le \sum_{0 \le k \le n} f(n)
	\le \int_0^n f(x) d x + f(n)
    \end{equation*}
    Si \(f(x)\) es monótona decreciente,
    se intercambian \(f(0)\) y \(f(n)\) en este último resultado.
  \end{proof}

%% Partly filched from aspnes14:_notes_discr_mathem
  Otras estimaciones simples resultan de acotar usando sumas conocidas.
  Si tenemos una suma finita de términos positivos,
  y la suma infinita correspondiente converge,
  tenemos una cota obvia.
  Suelen ser útiles como cotas
  la suma de secuencias geométricas
  (teorema~\ref{theo:suma-geometrica}):
  \begin{align}
    \sum_{k \ge 0} a^k
      &= \frac{1}{1 - a} \label{eq:suma-geometrica} \\
    \sum_{0 \le k \le n} a^k
      &= \frac{1 - a^{n + 1}}{1 - a} \label{eq:suma-geometrica-finita}
  \end{align}
  También es común la suma de secuencias aritméticas:%
     \index{suma!aritmetica@aritmética}
  \begin{equation}
    \label{eq:suma-aritmetica}
    \sum_{0 \le k \le n} k
      = \frac{n (n + 1)}{2}
  \end{equation}
  Ocasionalmente aparecen números harmónicos
  (sección~\ref{sec:numeros-harmonicos}):%
     \index{numeros harmonicos@números harmónicos}
  \begin{equation}
    \label{eq:3}
    \sum_{1 \le k \le n} \frac{1}{k}
      = H_n
      = \ln n + \gamma + O\left( \frac{1}{n} \right)
  \end{equation}

  Una técnica simple es acotar parte de la suma.
  Por ejemplo,
  sabemos cómo derivar una fórmula para la suma de los primeros cubos,
  pero podemos obtener cotas sencillas mediante las siguientes observaciones:
  \begin{align*}
    \sum_{1 \le k \le n} k^3
      &\le \sum_{1 \le k \le n} n^3 \\
      &=   n^4 \\
    \sum_{1 \le k \le n} k^3
      &\ge \sum_{n / 2 \le k \le n} k^3 \\
      &\ge \sum_{n / 2 \le k \le n} \left( \frac{n}{2} \right)^3 \\
      &=   \frac{n^4}{8}
  \end{align*}
  O sea:
  \begin{equation*}
    \sum_{1 \le k \le n} k^3
      = \Theta(n^4)
  \end{equation*}

\section{Estimar coeficientes}
\label{sec:estimar-coeficientes}
\index{asintotica@asintótica!coeficiente}

  Hay diferentes técnicas aplicables en este caso,
  de diferentes grados de complejidad.
  Discutiremos algunos de aplicabilidad general,
  el lector interesado
  podrá hallar referencias a técnicas adicionales.

\subsection{Cota trivial}
\label{sec:trivial-sum-bound}

  Hay una cota bastante trivial,
  que resulta sorprendentemente ajustada en muchos casos.
  Típicamente el error cometido es de un factor de \(n^{1/2}\).
  Si consideramos la secuencia \(\langle a_n \rangle_{n \ge 0}\),
  donde en aplicaciones combinatorias los \(a_n\) no son negativos:
  \begin{equation}
    \label{eq:example-gf}
    A(z)
      = \sum_{n \ge 0} a_n z^n
  \end{equation}
  Es claro que para \(u\) positivo:
  \begin{equation*}
    a_n u^n
      \le A(u)
  \end{equation*}
  de donde tenemos la cota:
  \begin{equation}
    \label{eq:trivial-bound}
    a_n
      \le \min_{u \ge 0} \left( \frac{A(u)}{u^n} \right)
  \end{equation}

\subsection{Singularidades dominantes}
\label{sec:dominating-singularities}
\index{asintotica@asintótica!coeficiente!singularidad}

  Sabemos
  (ver capítulo~\ref{cha:analisis-complejo},%
    \index{analisis complejo@análisis complejo}
   particularmente el teorema~\ref{theo:complex-f1+f2-singularity})
  que una función meromorfa cerca de una singularidad
  queda representada por la suma de una función entera
  y una función holomorfa
  en un disco perforado centrado en la singularidad.
  Esta segunda función
  (básicamente la parte principal
   de la serie de Laurent de la función)%
    \index{Laurent, serie de}
  domina la expansión en serie,
  o,
  lo que es lo mismo,
  pone la mayor parte de los coeficientes de la serie de potencias.
  Un ejemplo de esto
  ya lo vimos en la sección~\ref{sec:d&c:division-fija},
  donde vimos que el cero del denominador más cercano al origen
  da los términos dominantes.
  Considerando únicamente un polo simple \(z_0\)
  como singularidad más cercana al origen,
  estamos aproximando:
  \begin{equation*}
    A(z)
      = \sum_{n \ge 0} a_n z^n
  \end{equation*}
  mediante:
  \begin{equation*}
    A(z)
      \approx \res(A, z_0) \, \frac{1}{z - z_0}
      =	      - \res(A, z_0) \, \frac{1}{z_0 (1 - z / z_0)}
  \end{equation*}
  lo que se traduce en:
  \begin{equation}
    \label{eq:ae:dominating-single-pole-approximation}
    a_n
      \approx - \res(A, z_0) \, \frac{1}{z_0^{n + 1}}
  \end{equation}
  Incluir singularidades adicionales
  significa añadir términos similares para ellas.
  En detalle,
  tenemos:
  \begin{theorem}
    \label{theo:poles-approximation}
    Sea \(f\) meromorfa en la región \(D\) que contiene el origen.
    Sea \(R > 0\) el módulo de los polos de mínimo módulo,
    y sean \(z_0, \dotsc, z_s\) estos polos.
    Sea \(R' > R\) el módulo de los polos de siguiente módulo mayor,
    y sea \(\epsilon > 0\) dado.
    Entonces:
    \begin{equation}
      \label{eq:poles-approximation}
      \left[ z^n  \right] f(z)
	= \left[ z^n \right] \,
	    \left( \sum_{0 \le k \le s} \pp(f; z_k) \right)
	    + O \, \left( \left( \frac{1}{R'}
	    + \epsilon \right)^n \right)
    \end{equation}
  \end{theorem}
  \begin{proof}
    Basta demostrar que
    al restar las partes principales de \(f\) indicadas
    de \(f\) el resultado
    es una función que es holomorfa en el disco
    \(D_{R'}(0)\),
    la cota para el error
    es esencialmente el corolario~\ref{cor:ratio-test}.
    Por el teorema~\ref{theo:complex-f1+f2-singularity},
    la función \(f(z) - \pp(f; z_0)\) es holomorfa en \(z_0\),
    y debemos demostrar que su parte principal en \(z_1\)
    es la misma que la de \(f\):
    \begin{align*}
      \pp(f - \pp(f; z_0); z_1)
	&= \pp(f; z_1) - \pp(\pp(f; z_0); z_1) \\
	&= \pp(f; z_1)
    \end{align*}
    ya que \(\pp(f; z_0)\) es holomorfa en \(z_1\).
    Podemos ir aplicando esto polo a polo
    hasta demostrar
    que restando las partes principales indicadas de \(f\)
    obtenemos una función holomorfa
    en el disco de radio \(R'\) indicado.
  \end{proof}

  Para un ejemplo,
  vimos
  (sección~\ref{sec:desarreglos})
  que la función generatriz exponencial
  de los números de desarreglos está dada por:%
    \index{desarreglo!asintotica@asintótica}
  \begin{equation}
    \label{eq:ae:egf-derangements}
    \widehat{D}(z)
      = \frac{\mathrm{e}^{-z}}{1 - z}
  \end{equation}
  Esta función generatriz tiene un polo simple en \(z = 1\),
  con residuo \(- \mathrm{e}^{-1}\).
  Obtenemos directamente
  la aproximación~\eqref{eq:n-derangements-approx}.
  Sabemos también que la función:
  \begin{equation}
    \label{eq:ae:derangements-error}
    d(z)
      = \frac{\mathrm{e}^{-z}}{1 - z}
	  - \frac{\mathrm{e}^{-1}}{1 - z}
      = \frac{\mathrm{e}^{-z} - \mathrm{e}^{-1}}{1 - z}
  \end{equation}
  es entera
  (tiene una singularidad removible en \(z = 1\)),
  sus coeficientes son el error que comete la aproximación.
  Como el radio de convergencia
  para una función entera es \(R = \infty\),
  nuestra cota~\eqref{eq:poles-approximation} es:
  \begin{equation}
    \label{eq:ae:derangements-approximation}
    D_n
      = n! \mathrm{e}^{-1} + O(n! \epsilon^n)
  \end{equation}

  Otro caso es el número de resultados de competencias con empate
  (números de Bell ordenados).%
    \index{Bell, numeros de (ordenados)@Bell, números de (ordenados)!asintotica@asintótica}
  En la sección~\ref{sec:campeonatos-empate} llegamos a:
  \begin{equation}
    \label{eq:ae:ranking-gf}
    R(z)
      = \frac{1}{2 - \mathrm{e}^z}
  \end{equation}
  Esto tiene singularidades cuando \(\mathrm{e}^z = 2\),
  o sea en los puntos para \(k \in \mathbb{Z}\):
  \begin{equation}
    \label{eq:ae:ranking-gf-sigularities}
    \log 2
      = \ln 2 + 2 k \pi \mathrm{i}
  \end{equation}
  Domina la singularidad en \(\ln 2\),
  podemos refinar el resultado incluyendo singularidades adicionales
  en orden de cercanía de \(0\)
  (\(\ln 2\),
   \(\ln 2 \pm 2 \pi \mathrm{i}\),
   \(\ln 2 \pm 4 \pi \mathrm{i}\),
   \ldots).
  Si escribimos \(z = \log 2 + u\)
  y expandimos la exponencial en serie:
  \begin{equation*}
    R(z)
      = \frac{1}{2 - 2 \mathrm{e}^u}
      = \frac{1}{2} \, \frac{1}{1 - \mathrm{e}^u}
  \end{equation*}
  Las singularidades son todas polos simples,
  sus residuos son:%
    \index{C (numeros complejos)@\(\mathbb{C}\) (números complejos)!residuo}
  \begin{equation}
    \label{eq:R-residues}
    \res \left( \frac{1}{2 - \mathrm{e}^z},
		\ln 2 + 2 k \pi \mathrm{i} \right)
      = \lim_{z \rightarrow \ln 2 + 2 k \pi \mathrm{i}}
	  \frac{1}{- \mathrm{e}^z}
      = - \frac{1}{2}
  \end{equation}
  El polo más cercano al origen es \(\ln 2\),
  los demás polos están en \(\ln 2 \pm 2 \pi \mathrm{i}\),
  de módulo \(\sqrt{\ln 2 + 4 \pi^2} = 6,32\).
  En consecuencia,
  de~\eqref{eq:ae:dominating-single-pole-approximation}:
  \begin{equation}
    \label{eq:ae:R-approximation}
    R_n
      = \frac{n!}{2 (\ln 2)^{n + 1}} + O(n! \cdot 0,16^n)
  \end{equation}
  \begin{table}[ht]
    \centering
    \begin{tabular}{*{2}{>{\(}r<{\)}}D{.}{,}{4}}
      \multicolumn{1}{c}{\boldmath \(n\) \unboldmath} &
	\multicolumn{1}{c}{\boldmath \(R_n\) \unboldmath} &
	\multicolumn{1}{c}{\textbf{\eqref{eq:ae:R-approximation}}} \\
      \hline
      0 &   1 &	  0.7213 \\
      1 &   1 &	  1.0407 \\
      2 &   3 &	  3.0028 \\
      3 &  13 &	 12.9963 \\
      4 &  75 &	 74.9987 \\
      5 & 541 & 541.0015 \\
      \hline
    \end{tabular}
    \caption{Números de Bell ordenados}
    \label{tab:R-approximation}
  \end{table}
  El cuadro~\ref{tab:R-approximation}
  muestra los primeros valores exactos y aproximados,
  la concordancia es extremadamente buena.

  Para un ejemplo un poquito más complejo,
  tenemos los números de Bernoulli,%
    \index{Bernoulli, numeros de@Bernoulli, números de}
  con función generatriz exponencial
  (ver la sección~\ref{sec:propiedades-Bernoulli}):
  \begin{equation}
    \label{eq:ae:B(0,z)}
    B(z)
      = \frac{z}{\mathrm{e}^z - 1}
  \end{equation}
  Tanto \(z\) como \(\mathrm{e}^z - 1\) son enteras,
  \(B(z)\) solo puede tener polos.
  En los polos \(\mathrm{e}^z = 1\),
  o sea son polos \(z = 2 k \pi \mathrm{i}\)
  para todo \(k \in \mathbb{Z}\).
  En este caso tenemos dos polos a la misma distancia del origen,
  debemos considerar el aporte de ambos.
  Es fácil corroborar que los polos son simples,
  interesan:
  \begin{equation}
    \label{eq:residues-B(z)}
    \res(B(z), 2 k \pi \mathrm{i})
      = \lim_{z \rightarrow 2 k \pi \mathrm{i}}
	  \frac{z}{\mathrm{e}^z}
      = \mathrm{2 k \pi \mathrm{i}}
  \end{equation}
  Usando los polos \(\pm 2 \pi \mathrm{i}\)
  tenemos la aproximación:%
    \index{Bernoulli, numeros de@Bernoulli, números de!asintotica@asintótica}
  \begin{equation}
    \label{eq:ae:Bn-1}
    \frac{B_n}{n!}
      = - \frac{2 \pi \mathrm{i}}{(2 \pi \mathrm{i})^{n + 1}}
	    + \frac{2 \pi \mathrm{i}}{(- 2 \pi \mathrm{i})^{n + 1}}
	    + O \left( \left( \frac{1}{4 \pi} \right)^n \right)
  \end{equation}
  Vemos que para \(n\) impar los aportes se cancelan,
  esta técnica no entrega demasiada información en ese caso.
  Para \(n = 2 k\)
  volvemos a obtener~\eqref{eq:Bernoulli-approximation}:
  \begin{equation}
    \label{eq:ae:Bernoulli-approximation}
    B_{2 k}
      \sim \frac{- 2 (2 k)!}{(2 \pi \mathrm{i})^{2 k}}
      = (-1)^{k + 1} \, \frac{2 (2 k)!}{(4 \pi^2)^k}
  \end{equation}
  \begin{table}[ht]
    \centering
    \begin{tabular}{>{\(}r<{\)}>{\(}c<{\)}
		    D{.}{,}{4}D{.}{,}{4}}
      \multicolumn{1}{c}{\boldmath\(n\)\unboldmath} &
	\multicolumn{2}{c}{\boldmath\(B_n\)\unboldmath} &
	\multicolumn{1}{c}{\textbf{\eqref{eq:ae:Bernoulli-approximation}}} \\
      \hline
       0 & 1		      &	 1.0000 & -2.0000 \\
       2 & \phantom{-} 1 / 6  &	 1.1666 &  0.1013 \\
       4 & - 1 / 30	      & -0.0333 & -0.0308 \\
       6 & \phantom{-} 1 / 42 &	 0.0238 &  0.0234 \\
       8 & - 1 / 30	      & -0.0333 & -0.0332 \\
      10 & \phantom{-} 5 / 56 &	 0.0758 &  0.0757 \\
      12 & - 691 / 2730	      & -0.2531 & -0.2531 \\
      14 & \phantom{-} 7 / 6  &	 1.1666 &  1.1666 \\
      16 & - 3617 / 510	      & -7.0922 & -7.0920 \\
      \hline
    \end{tabular}
    \caption{Números de Bernoulli pares}
    \label{tab:Bernoulli-approximation}
  \end{table}
  Los valores exactos y aproximados de \(B_{2 k}\)
  se contrastan en el cuadro~\ref{tab:Bernoulli-approximation}.
  La aproximación no es particularmente buena en los índices bajos,
  se siente la influencia de los polos más lejanos.
  Igualmente la aproximación es muy buena.

  Si sumamos los aportes de todos los polos,
  resulta de nuevo la fórmula para \(\zeta(2 k)\):
  \begin{equation}
    \label{eq:ae:zeta(2k)}
    B_{2 k}
      = (-1)^{k + 1} \, \frac{2 (2 k)!}{(4 \pi^2)^k} \, \zeta(2 k)
  \end{equation}

  Una aplicación instructiva de las técnicas presentadas
  dan Odlyzko y Wilf~\cite{odlyzko88:_coins_fountain}
  al tratar el caso de fuentes no necesariamente de bloques
  (ver la sección~\ref{sec:FG-combinatoria}).

% asintotica-palabras-sin-patron.tex
%
% Copyright (c) 2013-2014 Horst H. von Brand
% Derechos reservados. Vea COPYRIGHT para detalles

\subsection{Número de palabras sin $k$ símbolos repetidos}
\label{sec:asymptotics-words-no-pattern}
\index{palabra!asintotica@asintótica}

  Vimos en la sección~\ref{sec:strings-excluding-pattern}
  que el número de palabras de largo~\(n\)
  sobre un alfabeto de~\(s\) símbolos
  que no contienen el patrón~\(p\)
  de largo~\(k\)
  tiene función generatriz ordinaria~\eqref{eq:Bp-gf}:
  \begin{equation*}
    \frac{c_p(z)}{(1 - s z) c_p(z) + z^k}
  \end{equation*}
  Acá \(c_p(z)\)
  es el polinomio de autocorrelación del patrón~\(p\).
  Obtener una estimación asintótica de este número
  sirve de ejemplo detallado
  de la aplicación de las herramientas discutidas.
  El caso general es tratado por Guibas y Odlyzko~%
    \cite{guibas81:_strin_overl_patterns}.

  Para simplificar,
  trataremos únicamente el caso en que el patrón
  es un símbolo repetido~\(k\) veces,
  en cuyo caso \(c_p(z) = 1 + z + \dotsb + z^{k - 1}\).
  En este caso la función generatriz se reduce a:
  \begin{equation}
    \label{eq:Bak-gf}
    \frac{1 - z^k}{1 - s z + (s - 1) z^{k + 1}}
  \end{equation}
  Nos interesa el cero más cercano al origen
  del denominador de~\eqref{eq:Bak-gf} para \(k\) dado.
  Sólo tiene sentido el caso \(k > 1\).
  Llamemos:
  \begin{equation}
    \label{eq:h-definition}
    h(z)
      = 1 - s z + (s - 1) z^{k + 1}
  \end{equation}
  Claramente \(h(1) = 0\)
  con \(h'(1) = k (s - 1) - 1 > 0\),
  y \(h(1/s) = (s - 1) s^{-k - 1} > 0\) es pequeño.
  Podemos acotar:
  \begin{equation*}
    h((s - 1)^{-1})
      = 1 - s (s - 1)^{-1} + (s - 1)^{-k}
      \le 1 - s + 1
      = 2 - s
  \end{equation*}
  Hay un cero entre \(s^{-1}\) y \((s - 1)^{-1}\).
  Debemos asegurarnos que sea único.
  Para ello aplicamos el teorema de Rouché
  (\ref{theo:Rouche}).

  Consideremos el caso \(s > 2\).
  Tomemos las funciones:
  \begin{align*}
    f(z)
      &= -s z \\
    g(z)
      &= 1 + (s - 1) z^{k + 1}
  \end{align*}
  Sobre la circunferencia \(\lvert z \rvert = r\)
  tenemos las cotas:
  \begin{align*}
    \lvert f(z) \rvert
      &= s r \\
    \lvert g(z) \rvert
      &= 1 + (s - 1) r^{k + 1}
  \end{align*}
  Nos interesa demostrar que:
  \begin{align*}
    s r
      &> 1 + (s - 1) r^{k + 1} \\
    0 &> 1 - s r + (s - 1) r^{k + 1}
       = h(r)
  \end{align*}
  Como interesan \(k \ge 2\) y \(s \ge 2\),
  tenemos:
  \begin{align*}
    h(0)
      &= 1 \\
    h(1)
      &= 0 \\
    h'(1)
      &= - s + (k + 1) (s - 1)
       = k (s - 1) - 1
       > 0
  \end{align*}
  En consecuencia,
  hay \(0 < r^* < 1\) tal que \(h(r) < 0\),
  o,
  lo que es lo mismo,
  \(\lvert f(z) \rvert > \lvert g(z) \rvert\)
  sobre la circunferencia \(\lvert z \rvert = r^*\)
  Por el teorema de Rouché,
  \(f(z)\) y \(f(z) + g(z) = h(z)\) tienen el mismo número de ceros
  al interior de la circunferencia,
  uno solo.
  Como \(h(z)\) es un polinomio de coeficientes reales,
  sus ceros son reales o vienen en pares complejos.
  El cero que nos interesa es simple y real.
  Si lo llamamos \(\rho\),
  por el teorema de Bender
  (\ref{theo:Bender}):
  \begin{equation}
    \label{eq:asymptotic-strings-no-ak}
    P_n
      \sim \frac{1 - \rho^k}{h'(\rho)} \cdot \rho^n
      = \frac{1 - \rho^k}{(s - 1) (k + 1) \rho^k - s} \cdot \rho^n
  \end{equation}

  Tenemos cotas para \(\rho\),
  para obtener una mejor aproximación del cero
  aplicamos una iteración del método de Newton%
    \index{Newton, metodo de@Newton, método de}
  (para detalles,
   ver textos de análisis numérico,
   como Acton~%
     \cite[capítulo~2]{acton90:_numerical_methods_work}
   o Ralston y Rabinowitz~%
     \cite[capítulo~8]{ralston12:_first_cours_numer_analy})
  partiendo con la aproximación \(1 / s\).
  Resulta que la expresión final
  es mucho más sencilla si partimos con:
  \begin{equation*}
    h_r(u)
      = u^{k + 1} h(1 / u)
      = u^{k + 1} - s u^k + s - 1
  \end{equation*}
  Los ceros de \(h_r(u)\) son recíprocos de los ceros de \(h(z)\).
  Tenemos:
  \begin{equation*}
    (\rho^*)^{-1}
      \approx s - \frac{h_r(s)}{h_r'(s)}
      = s - \frac{s - 1}{s^k}
  \end{equation*}
  Para el peor caso posible,
  \(s = k = 2\),
  esto da la aproximación \(\rho^* \approx 0,571\),
  cuando el valor correcto es \(\rho = 1 / \tau = 0,618\).

%%% Local Variables:
%%% mode: latex
%%% TeX-master: "clases"
%%% End:


\subsection{Singularidades algebraicas}
\label{sec:algebraic-singularities}

  Supongamos ahora
  que la singularidad \(z_0\) de \(f\) más cercana al origen
  es algebraica
  (un punto de ramificación),
  vale decir hay \(\alpha \in \mathbb{C}\)
  con \(\alpha \notin \mathbb{N}\)
  tal que la función \(g\) definida por lo siguiente
  es holomorfa en un entorno de \(z_0\):
  \begin{equation*}
    f(z)
      = (z_0 - z)^\alpha g(z)
  \end{equation*}
  El desarrollo de la sección anterior
  hace sospechar que en tal caso la serie:
  \begin{equation*}
    f(z)
      = (z_0 - z)^\alpha \sum_{k \ge 0} g_k (z_0 - z)^k
      = \sum_{k \ge 0} g_k (z_0 - z)^{k + \alpha}
  \end{equation*}
  es clave.

  Seguimos básicamente el desarrollo de Knuth y Wilf~%
    \cite{knuth89:_short_proof_Darboux}.
  Sin pérdida de generalidad,
  podemos suponer que \(z_0 = 1\)
  (basta considerar \(f(z z_0)\) en caso contrario),
  y que hay una única singularidad de interés.
  Primero un par de resultados auxiliares,
  de interés independiente.
  \begin{theorem}[Bender]
    \label{theo:Bender}
    \index{Bender, teorema de|textbfhy}
    Sean \(A(z) = \sum a_k z^k\) y \(B(z) = \sum b_k z^k\)
    series de potencias
    con radios de convergencia \(\alpha > \beta \ge 0\),
    respectivamente.
    Suponga que:
    \begin{equation*}
      \lim_{n \rightarrow \infty} \frac{b_{n - 1}}{b_n} = b
    \end{equation*}
    Si \(A(b) \ne 0\),
    y \(A(z) B(z) = \sum c_n z^n\),
    entonces \(c_n \sim A(b) b_n\).
  \end{theorem}
  La demostración sigue la de Bender~%
    \cite{bender74:_asymp_method_enumer}.
  \begin{proof}
    Basta demostrar que \(c_n / b_n \sim A(b)\).
    Sabemos que:
    \begin{equation*}
      c_n
	= \sum_{0 \le k \le n} a_k b_{n - k}
    \end{equation*}
    Con esto:
    \begin{align*}
      \left\lvert A(b) - \frac{c_n}{b_n} \right\rvert
	&=   \left\lvert
	       A(b)
		 - \sum_{0 \le k \le n} a_k \frac{b_{n - k}}{b_n}
	     \right\rvert \\
	&=   \left\lvert
	       \sum_{k > n} a_k b^k
		 - \sum_{0 \le k \le n}
		     a_k \left( b^k - \frac{b_{n - k}}{b_n} \right)
	      \right\rvert \\
	&\le \left\lvert
	       \sum_{k > n} a_k b^k
	      \right\rvert
		+ \left\lvert
		    \sum_{0 \le k \le n}
		      a_k \left( b^k - \frac{b_{n - k}}{b_n} \right)
		  \right\rvert
    \end{align*}
    El primer término es la cola de una serie convergente,
    tiende a cero al crecer \(n\).
    Para el segundo término,
    dividiendo la suma en \(n/2\):
    \begin{align*}
      \left\lvert
	\sum_{0 \le k \le n}
	  a_k \left( b^k - \frac{b_{n - k}}{b_n} \right)
      \right\rvert
	&\le \left\lvert
	       \sum_{0 \le k < n /2}
		 a_k \left( b^k - \frac{b_{n - k}}{b_n} \right)
	     \right\rvert
	       + \sum_{n/2 \le k \le n}
		   \lvert a_k b^k \rvert
		     \cdot \left\lvert
			     \frac{b_{n - k}}{b_n b^k}
			   \right\rvert \\
	&\le \left\lvert
	       \sum_{0 \le k < n /2}
		 a_k \left( b^k - \frac{b_{n - k}}{b_n} \right)
	     \right\rvert
	       + \max_{n/2 \le k \le n}
		   \left\lvert
		     \frac{b_{n - k}}{b_n b^k}
		   \right\rvert
		   \cdot \sum_{n/2 \le k \le n}
			   \lvert a_k b^k \rvert
    \end{align*}
    El primer término tiende a cero,
    ya que \(b_{n - k} / b_n \sim b^k\);
    la suma del segundo término
    está acotada por la cola de una serie convergente.
    Para el factor:
    \begin{equation*}
      \max_{n/2 \le k \le n}
	\left\lvert
	  \frac{b_{n - k}}{b_n b^k}
	\right\rvert
	  = \max_{n/2 \le k \le n}
	      \left\lvert
		\frac{b_{n - k} b^{n - k}}{b_n b^n}
	      \right\rvert
	  = \max_{0 \le k < n/2}
	      \left\lvert
		\frac{b_k b^k}{b_n b^n}
	      \right\rvert
    \end{equation*}
    Como la serie \(\sum b_k b^k\) converge,
    esto está acotado.
  \end{proof}
  Una aplicación simple del teorema de Bender
  es obtener una expansión asintótica para los números de Motzkin~%
    \cite{motzkin48:_relat_between_hyper_cross_ratios}%
    \index{Motzkin, numeros de@Motzkin, números de!asintotica@asintótica}
  (ver la discusión detallada de Donaghey y Shapiro~%
    \cite{donaghey77:_motzkin_numbers}).
  Dedujimos~\eqref{eq:ae-Motzkin-gf} en la sección~\ref{sec:numeros-motzkin}:
  \begin{equation}
    \label{eq:ae-Motzkin-gf}
    M(z)
      = \frac{1 - z - \sqrt{1 - 2 z - 3 z^2}}{2 z^2}
  \end{equation}
  Es claro que para \(n \ge 2\):
  \begin{align*}
    M_n
      &= \left[ z^n \right] \, \frac{1 - z - \sqrt{1 - 2 z - 3 z^2}}{2 z^2} \\
      &= - \frac{1}{2} \,
	     \left[ z^{n + 2} \right]
	       \, (1 + z)^{1/2} (1 - 3 z)^{1/2}
  \end{align*}
  Vemos que el radio de convergencia de esto último es \(1 / 3\),
  igual que el segundo factor;
  el primer factor tiene radio de convergencia \(1\).
  Por el teorema de Bender:
  \begin{align}
    M_n
      &\sim - \frac{1}{2} \,
		(1 + 1 / 3)^{1/2}
		   \left[ z^{n + 2} \right] (1 - 3 z)^{1/2}
	 \notag \\
      &= - \frac{\sqrt{3}}{3} \, \binom{1/2}{n + 2} \, (-3)^{n + 2}
	 \notag \\
      &= \frac{3 \sqrt{3}}{8 (n + 2)} \, \binom{2 n + 2}{n + 1}
	   \left( \frac{3}{4} \right)^n
	 \label{eq:ae:M-approximation}
  \end{align}
  \begin{table}[ht]
    \centering
    \begin{tabular}{*{2}{>{\(}r<{\)}}D{.}{,}{3}}
      \multicolumn{1}{c}{\boldmath \(n\) \unboldmath} &
	\multicolumn{1}{c}{\boldmath \(R_n\) \unboldmath} &
	\multicolumn{1}{c}{\textbf{\eqref{eq:ae:M-approximation}}} \\
      \hline
	0 &    1 & \text{\textemdash} \\
	1 &    1 & \text{\textemdash} \\
	2 &    2 &    1.827 \\
	3 &    4 &    3.836 \\
	4 &    9 &    8.631 \\
	5 &   21 &   20.346 \\
	6 &   51 &   49.593 \\
	7 &  127 &  123.981 \\
	8 &  323 &  316.153 \\
	9 &  835 &  819.123 \\
       10 & 2188 & 2150.198 \\
      \hline
    \end{tabular}
    \caption{Números de Motzkin}
    \label{tab:Motzkin-numbers}
  \end{table}
  El cuadro~\ref{tab:Motzkin-numbers} contrasta los valores exactos
  con la aproximación~\eqref{eq:ae:M-approximation}.
  No es particularmente buena,
  pero invertimos muy poco esfuerzo en ella.

  Un ejercicio simple
  de la función \(\Gamma\)	y la fórmula de Stirling
  da:
  \begin{lemma}
    \label{lem:asymptotics-binomial}
    \index{coeficiente binomial!asintotica@asintótica}
    Para \(\beta\) fijo cuando \(n \rightarrow \infty\):
    \begin{equation*}
      \left[ z^n \right] (1 - z)^\beta
	\begin{cases}
	  \sim n^{- \beta - 1} / \Gamma(- \beta)
	    & \text{si \(\beta \notin \mathbb{N}\)} \\
	  \rightarrow 0
	    & \text{si \(\beta \in \mathbb{N}\)} \\
	\end{cases}
    \end{equation*}
  \end{lemma}
  \begin{proof}
    Para el caso \(\beta \in \mathbb{N}\),
    por el teorema del binomio%
      \index{binomio, teorema del}
    sabemos que si \(n > \beta\) el coeficiente es cero.

    En caso que \(\beta \notin \mathbb{N}\):
    \begin{align*}
      \left[ z^n \right] (1 - z)^\beta
	= \binom{\beta}{n} (-1)^n
	= \frac{\beta^{\underline{n}}}{n!} \, (-1)^n
	= \frac{(-\beta)^{\overline{n}}}{n!}
	= \frac{\Gamma(n - \beta)}{\Gamma(-\beta) n!}
    \end{align*}
    El resultado
    sigue de la fórmula de Stirling~\eqref{eq:Stirling}:%
      \index{Stirling, formula de@Stirling, fórmula de}
    \begin{equation*}
      \Gamma(n + 1)
	= n!
	\sim \sqrt{2 \pi n} \, \left( \frac{n}{e} \right)^n
    \end{equation*}
    Tenemos:
    \begin{align*}
      \frac{\Gamma(n - \beta)}{\Gamma(-\beta) n!}
	&\sim \frac{1}{\Gamma(-\beta)}
		\cdot \frac{(n - \beta - 1)^{n - \beta - 3/2}}
			   {\mathrm{e}^{n - \beta - 2}}
		\cdot \frac{\mathrm{e}^n}{n^{n + 1/2}} \\
	&=    \frac{1}{\Gamma(-\beta)}
		\cdot \frac{n^{n - \beta - 1/2}
			      \cdot (1 - (\beta + 2) / n)^n
			      \cdot (1 - (\beta + 2) / n)^{-\beta - 3/2}
			      \cdot \mathrm{e}^{- \beta - 2}}
			   {n^{n + 1/2}} \\
	&\sim \frac{n^{- \beta - 1}}{\Gamma(-\beta)} \\
    \end{align*}
    Acá usamos el límite clásico:
    \begin{equation*}
      \lim_{n \rightarrow \infty}
	\left( 1 + \frac{\alpha}{n} \right)^n
	= \mathrm{e}^\alpha
     \qedhere
    \end{equation*}
  \end{proof}
  Con estos:
  \begin{lemma}
    \label{lem:bound-algebraic}
    Sea \(u(z) = (1 - z)^\gamma v(z)\),
    donde \(v(z)\) es holomorfa
    en algún disco \(\lvert z \rvert < 1 + \eta\)
    (acá \(\eta > 0\)).
    Entonces:
    \begin{equation*}
      \left[ z^n \right] u(z)
	= O(n^{-\gamma - 1})
    \end{equation*}
  \end{lemma}
  \begin{proof}
    Aplicando el teorema de Bender,%
      \index{Bender, teorema de}
    teorema~\ref{theo:Bender},
    queda:
    \begin{equation*}
      \left[ z^n \right] (1 - z)^\gamma v(z)
	\sim v(1) \left[ z^n \right] (1 - z)^\gamma
    \end{equation*}
    El lema~\ref{lem:asymptotics-binomial}
    entrega el resultado prometido.
  \end{proof}
  Estamos en condiciones de demostrar:
  \begin{theorem}[Lema de Darboux]
    \index{Darboux, lema de}
    \label{theo:Darboux-lemma}
    Sea \(f\) holomorfa en un disco \(\lvert z \rvert < 1 + \eta\),
    donde \(\eta > 0\).
    Suponga que en un entorno de \(z = 1\) tiene una expansión
    \begin{equation*}
      f(z)
	= \sum_{k \ge 0} f_k (1 - z)^k
    \end{equation*}
    Entonces para todo \(\beta \in \mathbb{C}\)
    y todo \(m \in \mathbb{N}_0\):
    \begin{align*}
      \left[ z^n \right] (1 - z)^\beta f(z)
	&= \left[ z^n \right] \sum_{k \ge 0} f_k (1 - z)^{\beta + k}
	     + O(n^{-m - \beta - 2}) \\
	&= \sum_{0 \le k \le m} f_k \binom{n -\beta - k - 1}{n}
	     + O(n^{-m - \beta - 2})
    \end{align*}
  \end{theorem}
  \begin{proof}
    Tenemos:
    \begin{align*}
      (1 - z)^\beta f(z)
	- \sum_{0 \le k \le m} f_k (1 - z)^{\beta + k}
	&= \sum_{k > m} f_k (1 - z)^{\beta + k} \\
	&= (1 - z)^{\beta + m + 1} \widetilde{f}(z)
    \end{align*}
    Las regiones de holomorfismo
    de \(f\) y \(\widetilde{f}\) son las mismas.
    El resultado sigue del lema~\ref{lem:bound-algebraic}.
  \end{proof}
  Obtengamos una expansión más precisa de los números de Motzkin.%
    \index{Motzkin, numeros de@Motzkin, números de!asintotica@asintótica}
  Tenemos para \(n \ge 2\),
  vía el cambio de variables \(u = 3 z\):
  \begin{align*}
    M_n
      &= - \frac{1}{2} \,
	     \left[ z^{n + 2} \right]
	       \, (1 + z)^{1/2} (1 - 3 z)^{1/2} \\
      &= - \frac{3^{n + 2}}{2} \,
	     \left[ u^{n + 2} \right] \,
	       \left( \frac{4}{3} \right)^{1/2} \,
		 \left( 1 - \frac{1 - u}{4} \right)^{1/2}
	       \cdot (1 - u)^{1/2} \\
      &= - 3 \sqrt{3} \cdot 3^n \cdot
	     \sum_{k \ge 0} \binom{1/2}{k} (-4)^{-k}
	       \left[ u^{n + 2} \right] \, (1 - u)^{1/2}
  \end{align*}
  Si expandimos para \(m = 0\)
  y no aproximamos los coeficientes binomiales
  resulta nuevamente~\eqref{eq:ae:M-approximation}.
  Extendamos a \(m = 1\),
  aproximando los coeficientes binomiales
  mediante el lema~\ref{lem:asymptotics-binomial}.%
    \index{coeficiente binomial!asintotica@asintótica}
  Necesitamos:
  \begin{align*}
    \binom{1/2}{n + 2}
      &\sim \frac{(n + 2)^{-3/2}}{\Gamma(-1/2)}
       =    - \frac{(n + 2)^{-3/2}}{2 \sqrt{\pi}} \\
    \binom{3/2}{n + 2}
      &\sim \frac{(n + 2)^{-5/2}}{\Gamma(-3/2)}
       =    \frac{3 (n + 2)^{-5/2}}{4 \sqrt{\pi}}
  \end{align*}
  porque:
  \begin{align*}
    \left( - \frac{1}{2} \right)
	\cdot \Gamma \left( - \frac{1}{2} \right)
      &= \Gamma \left( \frac{1}{2} \right)
      = \sqrt{\pi} \\
    \left( - \frac{3}{2} \right)
       \cdot \Gamma \left( - \frac{3}{2} \right)
      &= \Gamma \left( - \frac{1}{2} \right)
  \end{align*}
  y tenemos la estimación:
  \begin{equation}
    \label{eq:ae:M-approximation-2}
    M_n
      = \frac{\sqrt{3} \cdot 3^n}{16 \sqrt{\pi}} \cdot
	  \left(
	    8 (n + 2)^{-3/2} + 3 (n + 2)^{-5/2}
	  \right)
  \end{equation}
  \begin{table}[ht]
    \centering
    \begin{tabular}{rrD{.}{,}{3}}
      \multicolumn{1}{c}{\boldmath \(n\) \unboldmath} &
	\multicolumn{1}{c}{\boldmath \(R_n\) \unboldmath} &
	\multicolumn{1}{c}{\textbf{\eqref{eq:ae:M-approximation-2}}} \\
      \hline
	0 &    1 & \textemdash \\
	1 &    1 & \textemdash \\
	2 &    2 &    1.804 \\
	3 &    4 &    3.805 \\
	4 &    9 &    8.583 \\
	5 &   21 &   20.263 \\
	6 &   51 &   49.438 \\
	7 &  127 &  123.678 \\
	8 &  323 &  315.526 \\
	9 &  835 &  817.783 \\
       10 & 2188 & 2147.245 \\
      \hline
    \end{tabular}
    \caption{Números de Motzkin nuevamente}
    \label{tab:Motzkin-numbers-2}
  \end{table}
  La aproximación~\eqref{eq:ae:M-approximation-2}
  no es tan buena como~\eqref{eq:ae:M-approximation},
  pero hay que considerar
  que usamos una aproximación bastante cruda
  para los coeficientes binomiales
  a cambio de reemplazarlos por potencias.
  Igual la fórmula diseñada para índices muy grandes
  da excelentes resultados ya para \(n = 10\).

\subsection{Singularidades algebraico-logarítmicas}
\label{sec:singularidades-logaritmicas}
\index{Odlyzko-Flajolet, metodo de@Odlyzko-Flajolet, método de}

  Una técnica que resuelve muchos de los casos de interés en combinatoria
  es la que presentan Flajolet y Odlyzko\cite{flajolet90:_singular_anal}
  (ver también el resumen más accesible de Flajolet y Sedgewick~%
    \cite[sección~VI.2]{flajolet09:_analy_combin}).
  Las demostraciones son bastante engorrosas,
  nos remitiremos a citar los resultados.
  \begin{theorem}
    \label{theo:(1-z)^alpha}
    Sea \(\alpha \in \mathbb{C} \smallsetminus \mathbb{Z}_{\le 0}\).
    Entonces:
    \begin{equation}
      \label{eq:(1-z)^alpha-asy}
      [z^n] (1 - z)^{- \alpha}
	 \sim \frac{n^{\alpha - 1}}{\Gamma(\alpha)}
    \end{equation}
  \end{theorem}
  Si \(\alpha\) es un entero negativo,
  \((1 - z)^{- \alpha}\) es un polinomio,
  y los coeficientes del caso eventualmente se anulan.
  La exposición de Flajolet y Sedgewick~%
    \cite[teorema~VI.1]{flajolet09:_analy_combin}
  da la expansión asintótica.
  \begin{theorem}
    \label{theo:(1-z)^alpha*log(1-z)^beta}
    Sea \(\alpha \in \mathbb{C} \smallsetminus \mathbb{Z}_{\le 0}\).
    Entonces:
    \begin{equation}
      \label{eq:(1-z)^alpha*log(1-z)^beta-asy}
      [z^n] (1 - z)^{- \alpha}
	       \left(\frac{1}{z} \ln \frac{1}{1 - z}\right)^\beta
	 \sim \frac{n^{\alpha - 1}}{\Gamma(\alpha)}
		\ln^\beta n
    \end{equation}
  \end{theorem}
  Se introduce un factor \(1/z\) en frente del logaritmo
  para obtener una serie en \(z\),
  no altera la expansión cerca de \(z = 1\).
  Nuevamente Flajolet y Sedgewick~%
    \cite[teorema~VI.2]{flajolet09:_analy_combin}
  dan la expansión asintótica completa
  para el caso en que \(\beta\) no es un natural,
  citan a Jungen~\cite{jungen31:_sur_taylor} para completar ese caso.

\subsection{El método de Hayman}
\label{sec:Hayman-method}
\index{Hayman, metodo de@Hayman, método de}

  Las técnicas precedentes dan excelentes resultados
  cuando las funciones generatrices de interés
  tienen singularidades cerca del origen.
  Son totalmente inútiles si la función generatriz es entera.
  Nos interesaría,
  por ejemplo,
  obtener una expansión asintótica de \(n!\)
  vía considerar la serie para la exponencial.
  Para ello podemos razonar como sigue:
  De la fórmula generalizada de Cauchy,%
    \index{Cauchy, formula de (generalizada)@Cauchy, fórmula de (generalizada)}
  teorema~\ref{theo:Cauchy-formula-f-(n)},
  tenemos que para toda curva simple cerrada que incluya el origen:
  \begin{equation*}
    \frac{1}{n!}
      = \frac{1}{2 \pi \mathrm{i}} \,
	  \int_\gamma \frac{\mathrm{e}^z}{z^{n + 1}} \, \mathrm{d} z
  \end{equation*}
  Si usamos la circunferencia de radio \(R\) centrada en el origen,
  tomando valores absolutos:
  \begin{align*}
    \frac{1}{n!}
      &\le \frac{1}{2 \pi} \,
	     \max_{\lvert z \rvert = R}
	       \left\{
		 \frac{\lvert \mathrm{e}^z \rvert}
		      {\lvert z \rvert^{n + 1}}
	       \right\}
	     \cdot 2 \pi R \\
      &=   \frac{\mathrm{e}^R}{R^n}
  \end{align*}
  Pero el valor de \(R\) es arbitrario,
  podemos elegir aquel que minimice esta expresión,
  que resulta ser \(R = n\),
  dando la cota:
  \begin{equation*}
    \frac{1}{n!}
      \le \left( \frac{\mathrm{e}}{n} \right)^n
  \end{equation*}
  Nada mal,
  si se compara con la fórmula de Stirling~\eqref{eq:Stirling}.%
    \index{Stirling, formula de@Stirling, fórmula de}

  Si queremos mayor precisión,
  debemos tratar la integral en forma más cuidadosa.
  Hayman~\cite{hayman56:_generalization_Stirling}
  desarrolló maquinaria poderosa para esta situación.
  Seguimos la exposición de Wilf~\cite{wilf06:_gfology}.
  En una circunferencia alrededor del origen
  el módulo de una función con coeficientes reales no negativos
  tiene un máximo marcado en el eje real positivo,
  y es precisamente en ese caso que el método es más efectivo.
  Y este es el caso de las funciones generatrices
  de la combinatoria.
  El método en realidad es aplicable siempre,
  pero las técnicas basadas en singularidades
  son más sencillas de aplicar.

  Sea \(f(z)\) holomorfa en el disco \(\lvert z \rvert < R\),
  donde \(0 < R \le \infty\),
  y suponga que \(f(z)\) es \emph{admisible} para el método.
  Las condiciones de admisibilidad son bastante complicadas,
  veremos algunas condiciones suficientes más adelante.
  En la práctica,
  que \(f(z)\) sea admisible
  significa simplemente que la técnica funciona.

  Defina:
  \begin{equation}
    \label{eq:Hayman-M-definition}
    M(r)
      = \max_{\lvert z \rvert = r}
	  \left\{ \lvert f(z) \rvert \right\}
  \end{equation}
  Una consecuencia de las condiciones de admisibilidad es que
  para \(r\) suficientemente grande:
  \begin{equation}
    \label{eq:Hayman-M-value}
    M(r)
      = f(r)
  \end{equation}
  Esto porque,
  como notamos arriba,
  el método apunta a funciones que toman el valor máximo
  en la dirección real positiva.
  Defina funciones auxiliares:
  \begin{align}
    a(r)
      &= r \frac{f'(r)}{f(r)}
	 \label{eq:Hayman-a-definition} \\
    b(r)
      &= r a'(r)
       = r \, \frac{f'(r)}{f(r)}
	   + r^2 \, \frac{f''(r)}{f(r)}
	   - r^2 \, \left( \frac{f'(r)}{f(r)} \right)^2
	 \label{eq:Hayman-b-definition}
  \end{align}
  El resultado central es:
  \begin{theorem}[Hayman]
    \index{Hayman, teorema de}
    \label{theo:Hayman}
    Sea \(f(z) = \sum f_n z^n\) una función admisible.
    Sea \(r_n\) el cero positivo de \(a(r_n) = n\)
    para cada \(n \in \mathbb{N}\),
    donde \(a(r)\) es la función~\eqref{eq:Hayman-a-definition}.
    Entonces para \(n \rightarrow \infty\):
    \begin{equation}
      \label{eq:Hayman-asymptotic}
      f_n
	\sim \frac{f(r_n)}{r_n^n \sqrt{2 \pi b(r_n)}}
    \end{equation}
  \end{theorem}
  La receta misma es de aplicación directa,
  lo complicado es determinar si \(f(z)\) es admisible.

  Continuemos con nuestro ejemplo \(\mathrm{e}^z\),
  aceptando por ahora que es admisible.
  Resultan:
  \begin{equation*}
    a(r)
      = r
    \qquad
    b(r)
      = r \mathrm{e}^r
  \end{equation*}
  con lo que \(r_n = n\).
  La estimación de Hayman~\eqref{eq:Hayman-asymptotic} es:
  \begin{equation*}
    \frac{1}{n!}
      \sim \frac{\mathrm{e}^n}{n^n \sqrt{2 \pi n}}
  \end{equation*}
  La fórmula de Stirling.%
    \index{Stirling, formula de@Stirling, fórmula de}

  Veamos las condiciones de admisibilidad.
  En lo que sigue,
  las funciones \(a\) y \(b\)
  son las definidas
  por las ecuaciones~\eqref{eq:Hayman-a-definition}
  y~\eqref{eq:Hayman-b-definition},
  respectivamente.
  Sea \(f(z) = \sum_{n \ge 0} f_n z^n\)
  holomorfa en \(\lvert z \rvert < R\),
  donde \(0 < R \le \infty\).
  Suponga que:
  \begin{enumerate}[label=(\alph*)]
  \item
    Existe \(R_0 < R\) tal que para \(R_0 < r < R\) es \(f(r) > 0\)
  \item
    Hay una función \(\delta(r)\),
    definida para \(R_0 < r < R\),
    tal que \(0 < \delta(r) < \pi\) en ese rango,
    y tal que cuando \(r \rightarrow R\),
    uniformemente para \(\lvert \theta \rvert \le \delta(r)\),
    tenemos:
    \begin{equation*}
      f(r \mathrm{e}^{\mathrm{i} \theta})
	\sim f(r) \mathrm{e}^{\mathrm{i} \theta a(r)
				- \frac{1}{2} \, \theta^2 b(r)}
    \end{equation*}
  \item
    Uniformemente
    para \(\delta(r) \le \lvert \theta \rvert \le \pi\)
    tenemos cuando \(r \rightarrow R\):
    \begin{equation*}
      f(r \mathrm{e}^{\mathrm{i} \theta})
	= \frac{o(f(r))}{\sqrt{b(r)}}
    \end{equation*}
  \end{enumerate}
  Si esto se cumple,
  \(f\) es admisible y el teorema~\ref{theo:Hayman}
  da una estimación asintótica de los coeficientes.

  Como puede verse,
  las condiciones son complejas de verificar.
  Hay teoremas que dan condiciones suficientes,
  mucho más sencillas de manejar.
  Para nuestra fortuna,
  corresponden a operaciones comunes con funciones generatrices.%
    \index{Hayman, metodo de@Hayman, método de!funciones admisibles}
  \begin{enumerate}[label=(\Alph*), ref=(\Alph*)]
  \item
    \label{enum:Hayman-admisible-exp(f)}
    Si \(f(z)\) es admisible,
    lo es \(\mathrm{e}^{f(z)}\).
  \item
    \label{enum:Hayman-admisible-fg}
    Si \(f(z)\) y \(g(z)\)
    son admisibles para \(\lvert z \rvert < R\),
    lo es \(f(z) g(z)\).
  \item
    \label{enum:Hayman-admisible-fp}
    Sea \(f(z)\) admisible en \(\lvert z \rvert < R\).
    Sea \(p(z)\) un polinomio con coeficientes reales
    tal que \(p(R) > 0\) si \(R \ne \infty\),
    o tal que el coeficiente de máximo grado
    es positivo si \(R = \infty\).
    Entonces \(f(z) p(z)\) es admisible en \(\lvert z \rvert < R\).
  \item
    \label{enum:Hayman-admisible-p(f)}
    Sea \(p(z)\) un polinomio de coeficiente reales,
    y sea \(f(z)\) admisible en \(\lvert z \rvert < R\).
    Si \(f(z) + p(z)\) es admisible,
    y el coeficiente de máximo grado de \(p\) es positivo,
    entonces \(p(f(z))\) es admisible.
  \item
    \label{enum:Hayman-admisible-exp(p)}
    Sea \(p(z)\) un polinomio no constante con coeficientes reales,
    y sea \(f(z) = \mathrm{e}^{p(z)}\).
    Si \(\left[ z^n \right] \mathrm{e}^{p(z)} > 0\)
    para todo \(n\) suficientemente grande,
    entonces \(f(z)\) es admisible en \(\mathbb{C}\).
  \end{enumerate}
  Como un ejemplo,
  tomemos las involuciones,
  con función generatriz exponencial~\eqref{eq:involution-egf}:
  \begin{equation*}
    \mathrm{e}^{z + z^2 / 2}
  \end{equation*}
  Es claro
  que se cumplen las condiciones~\ref{enum:Hayman-admisible-exp(p)}.
  Obtenemos:
  \begin{equation*}
    a(r)
      = r + r^2
    \qquad
    b(r)
      = r + 2 r^2
  \end{equation*}
  Tenemos una excelente aproximación para \(r_n\):
  \begin{align*}
    r_n
      &= \sqrt{n + 1/4} - 1/2 \\
      &= \sqrt{n}
	   \left( 1 + \frac{1}{4 n} \right)^n - \frac{1}{2} \\
      &= \sqrt{n}
	   \left(
	     1 + \frac{1}{8 n} - \frac{1}{128 n^2} + \dotsb
	   \right)
	   - \frac{1}{2} \\
      &= \sqrt{n}
	   - \frac{1}{2}
	   + \frac{1}{8 n^{1/2}}
	   - \frac{1}{128 n^{3/2}}
	   + \dotsb
  \end{align*}
  Una revisión de la fórmula~\eqref{eq:Hayman-asymptotic} nos dice
  que requerimos estimaciones
  estilo \(\sim\) cuando \(n \rightarrow \infty\)
  de las cantidades \(f(r_n)\), \(b(r_n)\) y \(r_n^n\).
  Por turno:
  \begin{equation*}
    f(r_n)
      = \mathrm{e}^{r_n + \frac{1}{2} r_n}
      = \mathrm{e}^{\frac{1}{2} (r_n + n)}
      = \mathrm{e}^{n/2} \mathrm{e}^{r_n / 2}
  \end{equation*}
  Pero:
  \begin{align*}
    \mathrm{e}^{r_n / 2}
      = \exp \left(
		\frac{\sqrt{n}}{2} - \frac{1}{4} + O(n^{-1/2}
	      \right)
      \sim \mathrm{e}^{\frac{1}{2} \sqrt{n} - \frac{1}{4}}
  \end{align*}
  En la lista sigue \(b(r_n)\):
  \begin{equation*}
    b(r_n)
      = r_n + 2 r_n^2
      \sim 2 (r_n^2 + r_n)
      = 2 n
  \end{equation*}
  El último es el más complicado:
  \begin{align*}
    r_n^n
      &= \left(
	   \sqrt{n} - \frac{1}{2} + \frac{1}{8 \sqrt{n}} - \dotsb
	 \right)^n \\
      &= n^{n/2}
	   \left(
	     1 - \frac{1}{2 n^{1/2}} + \frac{1}{8 n} - \dotsb
	   \right)^n
  \end{align*}
  Esta situación debe tratarse con cuidado,
  a pesar de tender a 1 el paréntesis
  la potencia no necesariamente tiende a 1.
  La técnica general en estos casos es usar logaritmos
  para calcular la potencia:
  \begin{equation*}
    \left(
      1 - \frac{1}{2 n^{1/2}} + \frac{1}{8 n} - \dotsb
    \right)^n
      = \exp \left( n \ln \left(
			    1 - \frac{1}{2 n^{1/2}}
			      + \frac{1}{8 n}
			      - \dotsb
			  \right)
	     \right)
  \end{equation*}
  Luego expandimos el logaritmo en serie,%
    \index{serie de potencias!logaritmo}
  hasta llegar a términos de orden \(o(n^{-1})\).
  En nuestro caso:
  \begin{align*}
    \exp \left( n \ln \left(
			1 - \frac{1}{2 n^{1/2}}
			  + \frac{1}{8 n}
			  - \dotsb
		      \right)
	 \right)
      &= \exp \left(
		n \left(
		    \left(
		      - \frac{1}{2 n^{1/2}} + \frac{1}{8 n}
		    \right)
		      - \frac{1}{2}
			  \left(
			    - \frac{1}{2 n^{1/2}} + \frac{1}{8 n}
			  \right)^2
		      + O(n^{-3/2})
		     \right)
		  \right) \\
      &\sim \exp(- \sqrt{n} / 2)
  \end{align*}
  con lo que:
  \begin{equation*}
    r_n^n
     \sim n^{n/2} \exp(- \sqrt{n} / 2)
  \end{equation*}
  Finalmente,
  uniendo las distintas piezas:
  \begin{equation*}
    \frac{i_n}{n!}
      \sim \frac{\mathrm{e}^{\frac{n}{2} + \sqrt{n} - \frac{1}{4}}}
		{2 n^{n/2} \sqrt{\pi n}}
  \end{equation*}
  Mediante la fórmula de Stirling:%
    \index{Stirling, formula de@Stirling, fórmula de}
  \begin{equation}
    \label{eq:ae:involutions}
    i_n
      \sim \frac{1}{\sqrt{2}} n^{n/2}
	      \exp \left(
		     - \frac{n}{2} + \sqrt{n} - \frac{1}{4}
		   \right)
  \end{equation}

%%% Local Variables:
%%% mode: latex
%%% TeX-master: "clases"
%%% End:

% relaciones-funciones.tex
%
% Copyright (c) 2009-2014 Horst H. von Brand
% Derechos reservados. Vea COPYRIGHT para detalles

\chapter{Relaciones y funciones}
\label{cha:relaciones-funciones}

  Conceptos básicos de todas las matemáticas
  son los de \emph{relación} y \emph{función}.
  A pesar de su simplicidad,
  ofrecen aspectos de bastante interés.
  Nuevamente,
  la mayor parte del material presentado
  debe considerarse como sistematización de conocimientos previos.
  Una buena sistematización del área es el texto de Düntsch y Gediga~%
    \cite{duentsch00:_sets_relat_funct}.

\section{Relaciones}
\label{sec:relaciones}
\index{relacion@relación|textbfhy}

  \begin{definition}
    Sean \(\mathcal{A}\) y \(\mathcal{B}\) conjuntos.
    Una \emph{relación} \(R\) entre \(\mathcal{A}\) y \(\mathcal{B}\)
    es un subconjunto de \(\mathcal{A} \times \mathcal{B}\).

    Si \((a, b) \in R\),
    se anota \(a \mathrel{R} b\).
    Similarmente,
    para \((a, b) \notin R\)
    se anota \(a \mathrel{\centernot R} b\).
  \end{definition}

  Esto en rigor solo describe \emph{relaciones binarias},
  es perfectamente posible considerar relaciones
  de un solo elemento, de dos, tres o más elementos.
  Las relaciones binarias son lejos las más importantes,
  así que nos restringiremos a ellas acá.

  Según esta definición son relaciones ``menor a''
  entre números naturales,
  ``pololea con'' entre personas,
  ``precio de'' entre libros y sus precios en las librerías de la ciudad.
  Nótese que perfectamente pueden haber varios elementos relacionados,
  como por ejemplo \(1 < 2\), \(1 < 17\), \(1 < 31\).
  De la misma forma,
  el mismo libro puede tener precios diferentes
  en distintas librerías,
  pueden haber varias ediciones,
  o una librería tiene copias usadas más o menos deterioradas.

  Relacionado a lo anterior están los siguientes conceptos:
  \begin{definition}
    \index{relacion@relación!transpuesta|textbfhy}
    Sea \(R\) una relación entre \(\mathcal{A}\) y \(\mathcal{B}\).
    A la relación:
    \begin{equation*}
      R^{-1}
	= \{(y, x) \colon x \mathrel{R} y\}
    \end{equation*}
    se le llama la \emph{transpuesta} de \(R\).
  \end{definition}
  \begin{definition}
    \index{relacion@relación!composicion@composición|textbfhy}
    Sea \(R_1\) una relación entre \(\mathcal{A}\) y \(\mathcal{B}\),
    y \(R_2\) una relación entre \(\mathcal{B}\) y \(\mathcal{C}\).
    A la relación \(R\) definida mediante:
    \begin{equation*}
      R = \{(x, z) \colon \exists y \in \mathcal{B} \colon
			    x \mathrel{R_1} y \wedge y \mathrel{R_2} z\}
    \end{equation*}
    se le llama la \emph{composición} de \(R_1\) y \(R_2\),
    y se anota \(R = R_2 \circ R_1\)
  \end{definition}
  Si consideramos que \(x \mathrel{R_1} y\)
  como que la relación \(R_1\) lleva de \(x\) a \(y\),
  y de la misma forma \(y \mathrel{R_2} z\)
  que \(R_2\) lleva de \(y\) a \(z\),
  entonces \(R_2 \circ R_1\)
  es una relación que lleva de \(x\) a \(z\).

  El caso más común de relación es el en que ambos conjuntos son el mismo.
  Si \(R \subseteq \mathcal{U} \times \mathcal{U}\)
  se habla de una relación sobre \(\mathcal{U}\).
  Algunas propiedades de relaciones entre elementos del mismo conjunto
  tienen nombres especiales:
  \begin{definition}
    Sea \(R\) una relación sobre \(\mathcal{U}\).
    Entonces:
    \begin{itemize}
    \item
      \index{relacion@relación!reflexiva|textbfhy}
      Si para todo \(a \in \mathcal{U}\) se cumple \(a \mathrel{R} a\),
      la relación se llama \emph{reflexiva}.
    \item
      \index{relacion@relación!irreflexiva|textbfhy}
      Si para ningún \(a \in \mathcal{U}\) se cumple \(a \mathrel{R} a\),
      la relación se llama \emph{irreflexiva}.
    \item
      \index{relacion@relación!transitiva|textbfhy}
      Si para todo \(a, b, c \in \mathcal{U}\)
      se cumple que si \(a \mathrel{R} b\)
      y \(b \mathrel{R} c\) entonces \(a \mathrel{R} c\)
      la relación se dice \emph{transitiva}.
    \item
      \index{relacion@relación!simetrica@simétrica|textbfhy}
      Si para todo \(a, b \in \mathcal{U}\)
      siempre que \(a \mathrel{R} b\) se tiene que \(b \mathrel{R} a\),
      a la relación se le llama \emph{simétrica}.
    \item
      \index{relacion@relación!antisimetrica@antisimétrica|textbfhy}
      Si para todo \(a, b \in \mathcal{U}\),
      siempre que \(a \mathrel{R} b\)
      y \(b \mathrel{R} a\) entonces \(a = b\)
      se dice que la relación es \emph{antisimétrica}.
    \item
      \index{relacion@relación!total|textbfhy}
      Si para todo \(a, b \in \mathcal{U}\),
      se cumple \(a \mathrel{R} b\) o \(b \mathrel{R} a\),
      se le llama relación \emph{total}.
    \end{itemize}
  \end{definition}
  Ejemplos de relaciones reflexivas son \(=\) y \(\le\) sobre~\(\mathbb{Z}\).
  La relación \(\ne\) no es reflexiva ni transitiva ni antisimétrica,
  pero es simétrica,
  la relación \(<\) en \(\mathbb{R}\) es irreflexiva,
  es transitiva,
  no es simétrica y es antisimétrica
  (en forma vacía,
   ya que no hay \(a, b \in \mathbb{R}\) con \(a < b\) y \(b < a\)).
  La relación ``pololea con'' es simétrica,
  y definitivamente no es antisimétrica.
  La relación \(\ge\) en \(\mathbb{Z}\) es antisimétrica,
  y no es simétrica.
  Tanto \(\le\) como \(\ge\) sobre \(\mathbb{R}\) son totales
  (para cada par \(a\), \(b\)
   se cumple una de \(a \le b\) o \(b \le a\),
   incluso cuando \(a = b\)).
  La relación ``conoce a'' entre personas no es total
  (hay gente que no se conoce entre sí).

% Fixme: Mostrar ejemplos de c/propiedad por separado (o ejercicios...)
% Fixme: Diagramas de Hasse para relaciones, funciones, ...

  Nótese que una relación total
  sobre un conjunto no vacío necesariamente es reflexiva,
  ya que la definición exige que para cualquier par \(a\), \(b\)
  (incluyendo el caso \(a = b\))
  debe darse una de \(a \mathrel{R} b\) o \(b \mathrel{R} a\).

  Algunas combinaciones de las propiedades se repiten frecuentemente
  y llevan a propiedades interesantes de la relación,
  con lo que merecen nombres especiales.
  \begin{definition}
    \index{relacion@relación!equivalencia|textbfhy}
    Sea \(R\) una relación sobre \(\mathcal{U}\).
    Si \(R\) es reflexiva, simétrica y transitiva
    se le llama \emph{relación de equivalencia}.
  \end{definition}
  El caso clásico de relación de equivalencia es la igualdad.
  Otros ejemplos son la congruencia y semejanza geométricas.
  Veremos
  (y usaremos)
  muchas más en lo que sigue.

  La característica más importante de las relaciones de equivalencia
  está dada por el siguiente teorema.
  \begin{theorem}
    \label{theo:clases-equivalencia}
    \index{relacion@relación!equivalencia!clases|see{clases de equivalencia}}
    \index{clases de equivalencia|textbfhy}
    Sea \(R \subseteq \mathcal{U}^2\) una relación de equivalencia.
    Entonces los conjuntos definidos por:
    \begin{equation*}
      [a]_R
	= \{x \in \mathcal{U} \colon a \mathrel{R} x\}
    \end{equation*}
    son disjuntos y su unión es todo \(\mathcal{U}\).
  \end{theorem}
  \begin{proof}
    Hay dos cosas que demostrar acá:
    \begin{enumerate}
    \item
      \(\displaystyle \bigcup_{a \in \mathcal{U}} [a]_R = \mathcal{U}\)
    \item
      \(\displaystyle [a]_R \cap [b]_R = \varnothing\) o \([a]_R = [b]_R\)
    \end{enumerate}

    Para el primer punto,
    por reflexividad
    \(x \in [x]_R\),
    con lo que todo elemento \(x \in \mathcal{U}\)
    aparece al menos en la clase \([x]_R\),
    y la unión de todas las clases es \(\mathcal{U}\).

    Para el segundo punto
    consideremos dos elementos distintos \(a, b \in \mathcal{U}\),
    y veamos las clases \([a]_R\) y \([b]_R\).
    Si estos conjuntos son disjuntos,
    no hay nada que demostrar.
    Si no son disjuntos,
    habrá \(x \in \mathcal{U}\)
    en la intersección,
    o sea \(x \in [a]_R\) y \(x \in [b]_R\).
    Tenemos:
    \begin{alignat}{2}
      x &\mathrel{R} a
	&\qquad
	& \text{por suposición}		\label{eq:EC-1} \\
      x &\mathrel{R} b
	&&\text{por suposición}		\label{eq:EC-2} \\
      a &\mathrel{R} x
	&& \text{por simetría de \(R\) y~\eqref{eq:EC-2}} \label{eq:EC-3} \\
      a &\mathrel{R} b
	&& \text{por transitividad de \(R\) con~\eqref{eq:EC-2}
		 y~\eqref{eq:EC-3}} \label{eq:EC-4} \\
    \intertext{
      Ahora bien,
      si elegimos \(y \in [a]_R\):
    }
      y &\in [a]_R
	&& \text{por suposición} \label{eq:EC-5} \\
      y &\mathrel{R} a
	&& \text{por definición de \([a]_R\)} \label{eq:EC-6} \\
      y &\mathrel{R} b
	&& \text{por transitividad,
		 de~\eqref{eq:EC-6} con~\eqref{eq:EC-4}} \label{eq:EC-7} \\
      y &\in [b]_R
	&& \text{por definición de \([b]_R\)} \label{eq:EC-8}
    \end{alignat}
    Vale decir \([a]_R \subseteq [b]_R\).
    Por simetría,
    también \([b]_R \subseteq [a]_R\),
    y así \([a]_R = [b]_R\).
  \end{proof}
  \begin{definition}
    A los conjuntos \([a]_R\) les llamamos
    \emph{clases de equivalencia de \(R\)},
    al conjunto \([a]_R\) le llamamos
    la \emph{clase de equivalencia de \(a\) (en \(R\))}.
  \end{definition}
  Omitiremos la relación en la notación de clases de equivalencia
  cuando se subentienda cuál es la relación considerada.

  A la situación del teorema~\ref{theo:clases-equivalencia}
  se le dice que las clases \emph{particionan} el conjunto \(\mathcal{U}\).%
    \index{conjunto!particiones|textbfhy}
  Las clases corresponden precisamente
  a los conjuntos de elementos que la relación considera ``equivalentes''.
  \begin{example}
    Una relación \(R\) tal que si \(a \mathrel{R} b\) y \(a \mathrel{R} c\)
    entonces \(b \mathrel{R} c\) se llama \emph{euclidiana}.
    Demuestre que toda relación euclidiana reflexiva
    es una relación de equivalencia.

    Para demostrar que \(R\) es relación de equivalencia,
    debemos demostrar que es reflexiva,
    simétrica y transitiva.
    La relación dada es reflexiva por hipótesis,
    falta demostrar las otras dos propiedades.
    \begin{description}
    \item[Simetría:]
      Supongamos que \(a \mathrel{R} b\).
      Por reflexividad de \(R\),
      sabemos que \(a \mathrel{R} a\);
      y de \(a \mathrel{R} b\) y \(a \mathrel{R} a\)
      deducimos \(b \mathrel{R} a\).
    \item[Transitividad:]
      Supongamos \(a \mathrel{R} b\) y \(b \mathrel{R} c\).
      Por simetría,
      es \(b \mathrel{R} a\);
      y de \(b \mathrel{R} a\) y \(b \mathrel{R} c\)
      concluimos \(a \mathrel{R} c\).
    \end{description}
  \end{example}
  \begin{example}
    Consideremos las siguientes relaciones.
    Nótese que para demostrar que una de las propiedades \emph{no} vale,
    basta encontrar un único caso en que falla;
    para demostrar que \emph{si} vale hay que cubrir todos los casos.
    \begin{description}
    \item[\boldmath \(a \mathrel{R_1} b\) si \(a b = 100\),
	  sobre \(\mathbb{N}\):\unboldmath]
      Analizamos las distintas propiedades en turno:
      \begin{itemize}
      \item
	No es reflexiva,
	ya que por ejemplo \(5\cdot 5 \ne 100\).
      \item
	Es simétrica
	(si \(a b = 100\), entonces \(b a = 100\)).
      \item
	No es transitiva,
	ya que \(a b = 100\) y \(b c = 100\) no significa \(a c = 100\).
	Por ejemplo, tenemos \(5 \cdot 20 = 100\) y \(20 \cdot 5 = 100\),
	pero \(5 \cdot 5 \ne 100\).
      \item
	No es antisimétrica,
	ya que por ejemplo \(5 \mathrel{R_1} 20\)
	y \(20 \mathrel{R_1} 5\),
	pero \(5 \ne 20\).
      \item
	No es total,
	ya que por ejemplo el natural \(3\)
	no tiene ningún natural relacionado.
      \end{itemize}
    \item[\boldmath \(a \mathrel{R_2} b\) si \(a + b\) es par,
	  sobre \(\mathbb{N}\):\unboldmath]
      Esta es una relación de equivalencia.
      Las clases son los números pares y los impares.
      No es total.
    \item[\boldmath \(x \mathrel{R_3} y\) siempre que \(x - y\) es racional,
	  sobre \(\mathbb{R}\):\unboldmath]
      Es relación de equivalencia.
      En detalle:
      \begin{description}
      \item[Reflexiva:]
	\(x \mathrel{R_3} x\) corresponde a \(x - x\) racional,
	y \(0\) es racional.
      \item[Transitiva:]
	Si \(x \mathrel{R_3} y\) y también \(y \mathrel{R_3} z\),
	quiere decir que \(x -y\) e \(y - z\) son racionales,
	con lo que \(x - z = (x - y) + (y - z)\)
	también es racional.
      \item[Simétrica:]
	Si \(x \mathrel{R_3} y\),
	entonces \(x - y\) es racional,
	y lo es \(-(x - y) = y - x\),
	o sea,
	\(y \mathrel{R_3} x\).
      \end{description}
      Sabemos que hay clases de equivalencia de \(R_3\),
      aunque no son fáciles de describir.
    \item[\boldmath \((x_1, y_1) \mathrel{R_4} (x_2, y_2)\)
	  cuando \(x_1^2 + y_1^2 = x_2^2 + y_2^2\),
	  sobre \(\mathbb{R}^2\):\unboldmath]
      Equivalencia en \(\mathbb{R}^2\).
      Las clases de equivalencia son circunferencias de radio \(r\)
      alrededor del origen,
      definidas por \(x^2 + y^2 = r^2\).
    \end{description}
    Considere una relación \(R\) y su transpuesta \(R^{-1}\)
    sobre algún universo \(\mathcal{U}\).
    Veamos qué podemos decir acerca de \(R^{-1}\)
    si sabemos que:
    \begin{description}
    \item[\boldmath \(R\) es simétrica:\unboldmath]
      \index{relacion@relación!simetrica@simétrica!transpuesta}
      Que \(R\) sea simétrica significa
      que siempre que \(a \mathrel{R} b\) también \(b \mathrel{R} a\).
      Expresando esto en términos de \(R^{-1}\),
      es que \(b \mathrel{R^{-1}} a\) siempre que \(a \mathrel{R^{-1}} b\),
      y la transpuesta también lo es.
    \item[\boldmath \(R\) es antisimétrica:\unboldmath]
      \index{relacion@relación!antisimetrica@antisimétrica!transpuesta}
      Similar al caso anterior,
      se ve que en tal caso \(R^{-1}\) también es antisimétrica.
    \item[\boldmath \(R\) es transitiva:\unboldmath]
      \index{relacion@relación!transitiva!transpuesta}
      Si \(R\) es transitiva,
      \(R^{-1}\) también lo es
      (``caminando en la dirección contraria'').
    \item[\boldmath \(R\) es reflexiva:\unboldmath]
      \index{relacion@relación!reflexiva!transpuesta}
      Si \(a \mathrel{R} a\),
      entonces \(a \mathrel{R^{-1}} a\),
      y \(R^{-1}\) también es reflexiva.
    \item[\boldmath \(R\) es total:\unboldmath]
      \index{relacion@relación!total!transpuesta}
      Si \(a \mathrel{R} b\) o \(b \mathrel{R} a\)
      entonces también \(a \mathrel{R^{-1}} b\) o \(b \mathrel{R^{-1}} a\),
      y \(R^{-1}\) también es total.
    \end{description}
    Se recomienda al lector analizar en detalle estas observaciones,
    algunas no son tan simples como parecen.
  \end{example}
  \begin{definition}
    \index{relacion@relación!orden|textbfhy}
    Sea \(R\) una relación sobre un conjunto \(\mathcal{U}\).
    Si \(R\) es reflexiva, transitiva y antisimétrica
    se le llama \emph{relación de orden}.
    A una relación de orden que es total
    se le llama \emph{relación de orden total},
    en caso contrario es \emph{parcial}.
  \end{definition}

  Ejemplos clásicos de relaciones de orden son \(\le\) y \(\ge\).
  En \(\mathbb{Z}\) ambas son totales.
  Otra relación de orden es \(\subseteq\) entre conjuntos.
  No es total,
  ya que dos conjuntos
  no necesariamente se relacionan uno como subconjunto del otro.

  Otro ejemplo es la relación ``divide a'',
  definida sobre \(\mathbb{N}\) mediante:
  \begin{equation*}
    a \mid b
       \text{\ si y solo si existe \(c \in \mathbb{N}\) tal que\ }
	  b = a \cdot c
  \end{equation*}
  Veamos en detalle esto último:
  \begin{description}
  \item[Reflexividad:]
    \(a \mid a\) ya que \(a = a \cdot 1\)
    y \(1 \in \mathbb{N}\).
  \item[Transitividad:]
    \(a \mid b\) y \(b \mid c\)
    significa que existen \(m, n \in \mathbb{N}\)
    tales que:
    \begin{equation*}
      b = a \cdot m \qquad
      c = b \cdot n
    \end{equation*}
    con lo que:
    \begin{equation*}
      c
	= b \cdot n
	= (a \cdot m) \cdot n
	= a \cdot (m \cdot n)
    \end{equation*}
    que es decir \(a \mid c\).
  \item[Antisimetría:]
    Supongamos \(a \mid b\) y \(b \mid a\).
    Entonces existen \(m, n \in \mathbb{N}\) tales que:
    \begin{equation*}
      b = a \cdot m \qquad
      a = b \cdot n
    \end{equation*}
    Esto lleva a:
    \begin{align*}
      a
	&= (a \cdot m) \cdot n \\
      a \cdot 1
	&= a \cdot (m \cdot n) \\
      1
	&= m \cdot n
    \end{align*}
    Si ahora demostramos \(m = 1\),
    tenemos \(b = a \cdot m = a \cdot 1 = a\).
    Esto lo haremos por contradicción.
    Sabemos que \(1 \le m\);
    supongamos entonces que \(1 < m\),
    vale decir que para algún \(c \in \mathbb{N}\):
    \begin{align*}
      1 + c
	&= m \\
      1 \cdot n + c \cdot n
	&= m \cdot n \\
      n + c \cdot n
	&= 1
    \end{align*}
    Esto significa que \(n < 1\),
    y tal \(n\) no existe.
  \end{description}
  Este no es un orden total,
  ya que por ejemplo no se da ni \(6 \mid 15\) ni \(15 \mid 6\).

  Otros ejemplos dan las notaciones asintóticas de Bachmann-Landau%
    \index{Bachmann-Landau, notaciones de}
  definidas en la sección~\ref{sec:notacion-asintotica}.
  Consideremos la relación entre funciones \(f\) y \(g\)
  dada cuando \(f(n) = \Theta(g(n))\).
  Si \(f(n) = \Theta(g(n))\),
  hay \(n_0\) y constantes positivas \(c_1\) y \(c_2\)
  tales que para todo \(n \ge n_0\)
  se cumple \(c_1 g(n) \le f(n) \le c_2 g(n)\).
  Esta relación es reflexiva
  (podemos tomar \(n_0 = 1\), \(c_1 = 1/2\) y \(c_2 = 2\)),
  con lo que \(f(n) = \Theta(f(n))\).
  Es simétrica,
  ya que para \(n \ge n_0\) tenemos:
  \begin{equation*}
    \frac{1}{c_2} \, f(n) \le g(n) \le \frac{1}{c_1} \, f(n)
  \end{equation*}
  vale decir,
  \(g(n) = \Theta(f(n))\).
  También es transitiva,
  ya que si \(f(n) = \Theta(g(n))\)
  y además \(g(n) = \Theta(h(n))\),
  existen constantes \(n_0'\) y \(c_1'\) y \(c_2'\)
  con \(c_1' h(n) \le g(n) \le c_2' h(n)\) cuando \(n \ge n_0'\).
  Al tomar \(n \ge \max(n_0, n_0')\),
  combinando resulta
  \(c_1 c_1' h(n) \le f(n) \le c_2 c_2' h(n)\).
  Esto corresponde a la definición de \(f(n) = \Theta(g(n))\).

  Si consideramos \(f(n) = \Theta(g(n))\)
  como una especie de ``igualdad'' entre funciones,
  un desarrollo similar hará considerar \(f(n) = O(g(n))\)
  como un ``menor o igual que'',
  y similarmente \(f(n) = \Omega(g(n))\) como ``mayor o igual a''.
  Nótese eso sí que estas relaciones \emph{no} son totales.
  Tómense por ejemplo las funciones:
  \begin{align*}
    f(n)
      &= \begin{cases}
	   1 & \text{si \(n\) es par} \\
	   n & \text{caso contrario}
	 \end{cases}\\
    g(n)
      &= \begin{cases}
	   n & \text{si \(n\) es par} \\
	   1 & \text{caso contrario}
	 \end{cases}
  \end{align*}
  No se cumple \(f(n) = \Omega(g(n))\)
  ni \(f(n) = O(g(n))\),
  y tampoco \(g(n) = \Omega(f(n))\)
  ni \(g(n) = O(f(n))\).
  Estas funciones resultan ser no comparables.

\section{Funciones}
\label{sec:funciones}
\index{funcion@función|textbfhy}

  Históricamente,
  en los inicios del análisis se hablaba de ``curvas''
  (ver por ejemplo incluso el título del texto de l'Hôpital~%
    \cite{lHopital96:_analy_infin_petit_lignes_courb}),
  el que Euler hablara de ``funciones''
  (que inicialmente definiera esencialmente como expresiones algebraicas,
   para más adelante acercarse al concepto actual)
  fue un importante avance.
  El concepto se refinó,
  llegando a hacerse central en matemática
  en su forma actual,
  en que la función \(f\) asigna exactamente un valor \(f(x)\)
  a cada \(x\).

  Una \emph{función} es simplemente un tipo especial de relación.
  Para ser más precisos:
  \begin{definition}
    Sean \(\mathcal{D}\) y \(\mathcal{R}\) conjuntos.
    Decimos que una relación \(f\)
    es una \emph{función de \(\mathcal{D}\) a \(\mathcal{R}\)}
    si a cada \(x \in \mathcal{D}\)
    le relaciona exactamente un elemento \(z\) de \(\mathcal{R}\).
    Se anota \(f \colon \mathcal{D} \rightarrow \mathcal{R}\),
    llamamos \emph{dominio} a \(\mathcal{D}\)%
      \index{funcion@función!dominio|textbfhy}
    y \emph{codominio}%
      \index{funcion@función!codominio|textbfhy}
    o \emph{recorrido}%
      \index{funcion@función!recorrido|textbfhy}
    a \(\mathcal{R}\).
    Si \((x, z) \in f\) llamamos a \(z\) el \emph{valor de \(f\) en \(x\)},
    o \emph{imagen} de \(x\),%
      \index{funcion@función!imagen|textbfhy}
    y anotamos \(f(x)\).
    Al conjunto de todos los valores de la función
    se le llama su \emph{rango}.%
      \index{funcion@función!rango|textbfhy}
    Se le llama \emph{preimagen de \(z\)}%
      \index{funcion@función!preimagen|textbfhy}
    a cualquier \(x\) tal que \(f(x) = z\).
  \end{definition}

  Los puntos centrales de la definición son:
  \begin{enumerate}
  \item
    \(f(x)\) está definido para todos los \(x \in \mathcal{D}\).
  \item
    A cada \(x \in \mathcal{D}\)
    la función le asigna exactamente un valor en \(\mathcal{R}\).
  \end{enumerate}
  En vez de escribir \(f(x) = x^2 + 2\)
  anotaremos también \(f \colon x \mapsto x^2 + 2\).

  El caso más común de funciones en matemática elemental
  tiene dominio y rango conjuntos de números,
  por ejemplo \(\mathbb{N}\).
  Si rango y dominio son conjuntos de números,
  el método más simple de especificar la función
  es mediante una fórmula,
  como:
  \begin{equation*}
    f(n) = n^2 + n + 41
  \end{equation*}
  No siempre es posible llegar a una fórmula cerrada.
  En el caso particular en que el dominio es \(\mathbb{N}\)
  una alternativa es usar una definición recursiva.
  Un ejemplo importante es:
  \begin{equation*}
    f(1)
      = 1, \quad
	f(2)
	  = 1, \quad
	f(n + 2)
	  = f(n + 1) + f(n)\;\,(n \ge 1)
  \end{equation*}
  Esta función es la secuencia de los números de Fibonacci,
    \index{Fibonacci, numeros de@Fibonacci, números de}
  que comienza:
  \begin{equation*}
    \left\langle 1, 1, 2, 3, 5, 8, 13, 21, 34, \dotsc \right\rangle\
  \end{equation*}

  Nuevamente,
  hay clasificaciones:
  \begin{definition}
    Sea \(f \colon \mathcal{D} \rightarrow \mathcal{R}\) una función.
    Entonces:
    \begin{itemize}
    \item
      \index{funcion@función!inyectiva|textbfhy}%
      \index{funcion@función!uno a uno}
      Si para todo \(x, y \in \mathcal{D}\)
      con \(x \ne y\), \(f(x) \ne f(y)\),
      se dice \emph{inyectiva}
      (o \emph{uno a uno}).
    \item
      \index{funcion@función!sobreyectiva|textbfhy}
      Si para todo \(y \in \mathcal{R}\)
      hay \(x \in \mathcal{D}\) tal que \(f(x) = y\)
      se le llama \emph{sobreyectiva}
      (o simplemente \emph{sobre}).
    \item
      \index{funcion@función!biyectiva|textbfhy}%
      \index{biyeccion@biyección|see{función!biyectiva}}
      Si la función es inyectiva y sobreyectiva se dice \emph{biyectiva},
      también se le llama \emph{biyección}.
    \end{itemize}
  \end{definition}

  Para demostrar
  que una función \(f \colon \mathcal{X} \rightarrow \mathcal{Y}\)
  es inyectiva,
  debemos demostrar que si \(a \ne b\) entonces \(f(a) \ne f(b)\).
  La manera más sencilla de hacer esto suele ser demostrar el contrapositivo:
  Si \(f(a) = f(b)\),
  entonces \(a = b\).
  Para demostrar
  que una función \(f \colon \mathcal{X} \rightarrow \mathcal{Y}\)
  es sobreyectiva,
  hay que demostrar que para cada \(y \in \mathcal{Y}\)
  hay al menos un \(x \in \mathcal{X}\)
  tal que \(f(x) = y\).
  Para demostrar que una función es biyectiva
  hay que demostrar las dos anteriores.

  Como un ejemplo,
  anotamos para \(a, b \in \mathbb{R}\) con \(a < b\)
  el intervalo abierto \((a, b) = \{x \colon a < x < b\}\),
  y \(\mathbb{R}^+\) para los reales positivos.
  Definimos la función \(f \colon (a, b) \rightarrow \mathbb{R}^+\) mediante:
  \begin{equation*}
    f(t)
      = \frac{t - a}{b - t}
  \end{equation*}
  Primeramente,
  \(f\) es una función,
  ya que a cada \(t \in (a, b)\)
  le asigna un único valor en \(\mathbb{R}^+\).
  También es inyectiva,
  ya que:
  \begin{align*}
    f(t)
      &= z \\
      &= \frac{t - a}{b - t} \\
    t
      &= \frac{a + z b}{z + 1}
  \end{align*}
  Así,
  a un valor dado de \(z > 0\) le corresponde a un único valor de \(t\).
  Además es sobreyectiva,
  ya que si \(z > 0\)
  la última expresión siempre está definida.
  Como hay una biyección entre \(\mathbb{R}\) y un rango,
  podemos concluir que hay tantos números reales en un rango cualquiera
  como el total de los reales.
  Volveremos sobre este punto en el capítulo~\ref{cha:numerabilidad}.

  Podemos construir nuevas funciones partiendo de funciones dadas,
  dado que son simplemente relaciones:
  \begin{definition}
    \index{funcion@función!composicion@composición|textbfhy}
    Sean \(f \colon \mathcal{A} \rightarrow \mathcal{B}\)
    y \(g \colon \mathcal{B} \rightarrow \mathcal{C}\) funciones.
    La \emph{composición} de \(f\) y \(g\) está definida
    ya que son relaciones.
    La \emph{función inversa de \(f\)},
    \(f^{-1} \colon \mathcal{B} \rightarrow \mathcal{A}\)%
      \index{funcion@función!inversa|textbfhy}
    se define como la transpuesta de la relación,
    \(f^{-1}(z) = x\) siempre que \(f(x) = z\).
  \end{definition}
  La función inversa solo puede existir
  si el rango de \(f\) es su recorrido \(\mathcal{B}\)
  (\(f\) es sobreyectiva),
  y además un elemento de \(\mathcal{B}\) tiene una única preimagen
  (\(f\) es inyectiva).
  Combinando ambas,
  \(f\) es biyectiva.
  Además es fácil ver que en tal caso \((f^{-1})^{-1} = f\).

  Si \(f\), \(g\), \(h\) son funciones,
  es simple demostrar que
  \((f \circ g) \circ h = f \circ (g \circ h)\)
  (acá los paréntesis indican en qué orden se componen las funciones).
  \begin{theorem}
    \label{theo:gof}
    Sean \(f \colon \mathcal{A} \rightarrow \mathcal{B}\)
    y \(g \colon \mathcal{B} \rightarrow \mathcal{C}\) funciones.
    Entonces:
    \begin{enumerate}
    \item
      \label{theo:gof:inyectivas}
      Si \(f\) y \(g\) son inyectivas,
      lo es \(g \circ f\).
    \item
      \label{theo:gof:sobreyectivas}
      Si \(f\) y \(g\) son sobreyectivas,
      lo es \(g \circ f\).
    \item
      \label{theo:gof:biyectivas}
      Si \(f\) y \(g\) son biyectivas,
      lo es \(g \circ f\).
      La función inversa de la composición
      es \((g \circ f)^{-1} = f^{-1} \circ g^{-1}\).
    \item
      \label{theo:gof:gof-inyectiva}
      Si \(g \circ f\) es inyectiva,
      \(f\) es inyectiva
      (\(g\) puede no serlo).
    \item
      \label{theo:gof:gof-sobreyectiva}
      Si \(g \circ f\) es sobreyectiva,
      \(g\) es sobreyectiva
      (\(f\) puede no serlo).
    \end{enumerate}
  \end{theorem}
  \begin{proof}
    Demostramos cada punto por turno.
    \begin{enumerate}
    \item
      Supongamos \((g \circ f)(x) = (g \circ f)(y)\).
      Esto es \(g(f(x)) = g(f(y))\),
      y como supusimos \(g\) inyectiva,
      quiere decir que \(f(x) = f(y)\),
      y esto a su vez que \(x = y\) ya que \(f\) es inyectiva.
    \item
      Si \(f\) y \(g\) son sobreyectivas,
      quiere decir que para cada \(c \in \mathcal{C}\)
      hay algún \(b \in \mathcal{B}\) tal que \(g(b) = c\),
      y también que para cada \(b \in \mathcal{B}\) hay \(a \in \mathcal{A}\)
      tal que \(f(a) = b\).
      Combinando estas,
      para cada \(c \in \mathcal{C}\) hay algún \(a \in \mathcal{A}\)
      tal que \(g(f(a)) = c\),
      que es decir \((g \circ f)(a) = c\),
      y \(g \circ f\) es sobreyectiva también.
    \item
      Esto se obtiene combinando las partes~(\ref{theo:gof:inyectivas})
      y~(\ref{theo:gof:sobreyectivas}).
      La función \(g \circ f\) lleva
      (vía \(f\))
      de \(\mathcal{A}\) a \(\mathcal{B}\),
      y luego
      (vía \(g\))
      de \(\mathcal{B}\) a \(\mathcal{C}\).

      Para la inversa de \(g \circ f\)
      usamos asociatividad y la definición de inversa:
      \begin{equation*}
	(f^{-1} \circ g^{-1}) \circ (g \circ f)
	  = f^{-1} \circ (g^{-1} \circ g) \circ f
	  = f^{-1} \circ f
	  = \iota
      \end{equation*}
      Terminamos con la función identidad,
      y componer por la derecha se trata de forma análoga
      resultando la identidad también.
      Concluimos que \(f^{-1} \circ g^{-1}\) es la inversa prometida.
    \item
      Sean \(x, y \in \mathcal{A}\) tales que \(f(x) = f(y)\).
      Entonces	\(g(f(x)) = g(f(y))\),
      y como suponemos \(g \circ f\) inyectiva,
      \(x = y\),
      con lo que \(f\) es inyectiva.
      Mostraremos más adelante que \(g\) no tiene porqué ser inyectiva.
    \item
      Como \(g \circ f\) es sobreyectiva,
      para cada \(c \in \mathcal{C}\) hay \(a \in \mathcal{A}\)
      para el cual \((g \circ f)(a) = g(f(a)) = c\),
      y podemos elegir \(b = f(a)\)
      para demostrar que para todo \(c \in \mathcal{C}\)
      hay \(b \in \mathcal{B}\)
      tal que \(g(b) = c\).
      Mostraremos más adelante que \(f\) no tiene porqué ser sobreyectiva.
      \qedhere
    \end{enumerate}
  \end{proof}
  Nótese que en la demostración de la parte~(\ref{theo:gof:gof-inyectiva})
  nada podemos concluir sobre \(g\),
  puede ser que en lo anterior no ``usamos'' valores de \(g\)
  que hacen fallar la inyección.
  Ver figura~\ref{fig:gof-is:inyectiva}.
  De forma similar,
  en la parte~(\ref{theo:gof:gof-sobreyectiva})
  nada podemos decir sobre \(f\),
  ver figura~\ref{fig:gof-is:sobreyectiva}.
  \begin{figure}[htbp]
    \centering
    \subfloat[Inyectiva]
	     {\pgfimage[height=2in]{images/gof-inyectiva}
		\label{fig:gof-is:inyectiva}}
    \hspace{1in}
    \subfloat[Sobreyectiva]
	     {\pgfimage[height=2in]{images/gof-sobreyectiva}
		\label{fig:gof-is:sobreyectiva}}
    \caption{Funciones compuestas inyectivas y sobreyectivas}
    \label{fig:gof-is}
  \end{figure}

  En el caso especial en que dominio y codominio son iguales,
  podemos hacer algunas cosas adicionales:
  \begin{itemize}
  \item
    La \emph{función identidad},%
      \index{funcion@función!identidad|textbfhy}
    \(\iota \colon \mathcal{A} \rightarrow \mathcal{A}\),
    se define mediante \(\iota(x) = x\) para todo \(x \in \mathcal{A}\).
    Claramente cumple con \(\iota \circ f = f \circ \iota = f\)
    para toda función \(f\).
    Además,
    \(\iota^{-1} = \iota\).
  \item
    Es obvio de la definición que \(f \circ f^{-1} = f^{-1} \circ f = \iota\)
    si \(f\) es biyectiva.
  \end{itemize}

  En el caso de una función \(f \colon \mathcal{X} \rightarrow \mathcal{Z}\),
  con \(\mathcal{A} \subseteq \mathcal{X}\)
  se usa la notación
    \(f(\mathcal{A}) = \{f(x) \colon x \in \mathcal{A}\}\)
  para la imagen de un subconjunto del dominio.%
    \index{funcion@función!imagen de un conjunto|textbfhy}
  Igualmente,
  para \(\mathcal{B} \subseteq \mathcal{Z}\)
  se anota \(f^{-1} (\mathcal{B}) = \{x \colon f(x) \in \mathcal{B}\}\)
  para la preimagen de un conjunto.%
    \index{funcion@función!preimagen de un conjunto|textbfhy}
  En el último caso la función inversa no tiene porqué existir
  para que la notación tenga sentido.

\section{Operaciones}
\label{sec:operaciones}
\index{operacion@operación|textbfhy}

  Un caso particular importante de funciones son las \emph{operaciones}
  sobre un conjunto \(\mathcal{A}\),
  que formalmente no son más
  que funciones
    \(\operatorname{op} \colon \mathcal{A}^n \rightarrow \mathcal{A}\).
  Nótese que esta definición trae implícito
  que la operación entrega un valor en \(\mathcal{A}\)
  para todos sus posibles argumentos,
  cosa que suele indicarse diciendo que la operación es \emph{cerrada}.%
    \index{operacion@operación!cerrada|textbfhy}
  Se suelen distinguir operaciones \emph{unarias}%
    \index{operacion@operación!unaria|textbfhy}
  si tienen un único argumento,
  y \emph{binarias} si tienen dos.%
    \index{operacion@operación!binaria|textbfhy}
  Pueden considerarse operaciones de más de dos argumentos,
  pero son raras en la práctica.
  Nótese también que por ejemplo la división entre números reales
  no es una operación según nuestra definición
  (\(a / b\) no está definido si \(b = 0\)).
  Tampoco lo son las relaciones,
  ya que por ejemplo \(a < b\) toma dos reales y entrega verdadero o falso.

  Las operaciones de uso común suelen anotarse en forma especial,
  por ejemplo para la operación binaria
  de suma de \(a\) y \(b\) anotamos \(a + b\).
  Esto se conoce como \emph{notación infijo}.%
    \index{operacion@operación!notacion infijo@notación infijo|textbfhy}
  Para operaciones unarias es posible la notación \emph{prefijo}%
    \index{operacion@operación!notacion prefijo@notación prefijo|textbfhy}
  (como en \(- a\) o \(\tan \alpha\))
  o \emph{postfijo}%
    \index{operacion@operación!notacion postfijo@notación postfijo|textbfhy}
  (es el caso del factorial,
   como en \(n!\)).

  En rigor,
  una expresión como \(a + b + c\) no tiene sentido,
  debiera indicarse el orden en que se efectúan las operaciones
  mediante paréntesis.
  Para ahorrar notación,
  se suelen adoptar convenciones:
  Si \(a \circ b \circ c\) ha de interpretarse como
  \((a \circ b) \circ c\)
  se dice que la operación \emph{asocia hacia la izquierda}%
    \index{operacion@operación!asociativa izquierda|textbfhy}
  (o que es \emph{asociativa izquierda}),
  si en cambio \(a \circ b \circ c\)
  significa \(a \circ (b \circ c)\)
  se dice que \emph{asocia hacia la derecha}%
    \index{operacion@operación!asociativa derecha|textbfhy}
  o es \emph{asociativa derecha}.
  En caso que \(a \circ b \circ c\) no se le dé sentido,
  se le llama \emph{no asociativa}.%
    \index{operacion@operación!no asociativa|textbfhy}
  Las operaciones comunes se consideran todas asociativas izquierdas;
  salvo las potencias,
  que son asociativas derechas
  (o sea,
   \(2^{3^4} = 2^{(3^4)}\)).

  Si tenemos dos operaciones \(\odot\) y \(\oplus\),
  y se interpretan
  \(a \odot b \oplus c\) como \((a \odot b) \oplus c\)
  y \(a \oplus b \odot c\) como \(a \oplus (b \odot c)\)
  (siempre se efectúa \(\odot\) antes de \(\oplus\))
  se dice que \(\odot\) tiene \emph{mayor precedencia} que \(\oplus\).
    \index{operacion@operación!precedencia|textbfhy}
  En caso que ambas operaciones sean asociativas izquierdas,
  y \(a \odot b \oplus c\) se interpreta como \((a \odot b) \oplus c\)
  y \(a \oplus b \odot c\) como \((a \oplus b) \odot c\)
  (siempre se efectúan las operaciones de izquierda a derecha),
  se dice que tienen la \emph{misma precedencia}.

  Hay ciertas propiedades de las funciones mismas que son de interés también.
  \begin{itemize}
  \item
    \index{operacion@operación!conmutativa|textbfhy}
    Si \(a \circ b = b \circ a\) la operación se dice \emph{conmutativa}.
  \item
    \index{operacion@operación!asociativa|textbfhy}
    Si \((a \circ b) \circ c = a \circ (b \circ c)\),
    la operación se llama \emph{asociativa}.
    Esto no debe confundirse
    con las convenciones de notación mencionadas antes.
  \item
    \index{operacion@operación!elemento neutro|textbfhy}
    De existir un elemento \(e\) tal que
    \(a \circ e = e \circ a = a\) para todos los \(a\),
    a \(e\) se le llama \emph{elemento neutro} de la operación.
  \item
    \index{operacion@operación!distributiva|textbfhy}
    Si siempre se cumple
    \((a \oplus b) \odot c = (a \odot c) \oplus (b \odot c)\),
    se dice que \(\odot\) \emph{distribuye sobre} \(\oplus\)
    \emph{por la derecha},
    en forma similar
    si \(a \odot (b \oplus c) = (a \odot b) \oplus (a \odot c)\)
    \emph{por la izquierda}.
    Si \(\odot\) distribuye sobre \(\oplus\) tanto por la derecha
    como por la izquierda,
    se dice simplemente que distribuye sobre ella.
  \end{itemize}

%%% Local Variables:
%%% mode: latex
%%% TeX-master: "clases"
%%% End:

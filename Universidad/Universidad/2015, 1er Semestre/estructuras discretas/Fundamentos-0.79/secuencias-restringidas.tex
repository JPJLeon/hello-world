% secuencias-restringidas.tex
%
% Copyright (c) 2009, 2012-2014 Horst H. von Brand
% Derechos reservados. Vea COPYRIGHT para detalles

\section{Secuencias con restricciones}
\label{sec:secuencias-restringidas}
\index{combinatoria!secuencias restringidas}

  También interesa poder contar reordenamientos
  en los cuales hay ciertas restricciones,
  como elementos en posiciones fijas
  o elementos en posiciones fijas relativas entre sí.

  Seguimos con nuestro ejemplo de \(\mathtt{BOOKKEEPER}\).
  \begin{enumerate}
  \item
    ¿De cuántas formas se puede escribir esta palabra
    si las dos \(\mathtt{O}\) siempre están juntas?
  \item
    ¿Cuántas formas hay de ordenar las letras
    si siempre están \(\mathtt{BPR}\) juntas en ese orden?
  \item
    ¿Si \(\mathtt{BPR}\) están juntas,
    pero en cualquier orden?
  \item
    ¿En cuántos aparecen \(\mathtt{BPR}\) en ese orden,
    no necesariamente juntas?
    % Podemos elegir las posiciones de BPR en \binom{10}{3} formas
    % (el orden está predeterminado), las demás por multinomio
  \item
    ¿Cuántas maneras hay de ordenar las letras
    si las \(\mathtt{O}\) están separadas por una letra?
  \item
    ¿Cuántas maneras hay de ordenarlas
    si las \(\mathtt{E}\) están separadas siempre por una letra?
  \item
    ¿De cuántas maneras se pueden ordenar las letras
    si las \(5\) vocales están al principio y las \(5\) consonantes al final?
  \item
    ¿Y si las vocales están en las posiciones impares?
  \item
    ¿Que pasa si las vocales
    ocupan las posiciones \(2\), \(3\), \(6\), \(7\), \(9\)?
  \item
    ¿Cuántos ordenamientos hay
    en los cuales las vocales están todas juntas?
  \item
    ¿Cuántos ordenamientos con \(\mathtt{B}\) en una posición impar hay?
  \item
    ¿Y si solo pedimos una \(\mathtt{O}\) en una posición par?
  \item
    ¿Cuántos ordenamientos hay con las tres \(\mathtt{E}\)
    en posiciones impares?
  \item
    ¿Cuántos órdenes tienen la \(\mathtt{B}\) separadas de la \(\mathtt{R}\)
    por dos letras?
  \item
    ¿Cuántos tienen la \(\mathtt{B}\) separadas de la \(\mathtt{R}\)
    por \(k\) letras?
  \end{enumerate}

  Otra situación,
  que puede enfrentarse mediante nuestra estrategia general
  de construir el objeto de interés
  en fases independientes,%
    \index{combinatoria!regla del producto}
  se presenta si queremos determinar
  el número de maneras de ordenar las letras de \(\mathtt{MISSISSIPPI}\)
  de forma que las vocales siempre estén separadas por consonantes.
  Vemos que hay \(4\) \(\mathtt{I}\),
  lo que deja \(5\) espacios en los cuales distribuir las consonantes.
  Si llamamos \(x_0\) al número de consonantes
  antes de la primera \(\mathtt{I}\),
  \(x_1\) a \(x_3\) al número de consonantes entre \(\mathtt{I}\)
  y finalmente \(x_4\) al número de consonantes
  después de la última \(\mathtt{I}\),
  quedan la ecuación:
  \begin{equation*}
    x_0 + x_1 + x_2 + x_3 + x_4
      = 7
  \end{equation*}
  Restricciones son que \(x_0 \ge 0\),
  \(x_k \ge 1\) para \(1 \le k \le 3\)
  y \(x_4 \ge 0\).
  Si definimos nuevas variables \(y_0 = x_0\),
  \(y_k = x_k - 1\) para \(1 \le k \le 3\)
  e \(y_4 = x_4\),
  queda:
  \begin{equation*}
    y_0 + y_1 + y_2 + y_3 + y_4
      = 4
  \end{equation*}
  lo que nos lleva a contar multiconjuntos:%
    \index{multiconjunto!numero@número}
  El número de soluciones es el número de multiconjuntos
  de \(5\) elementos de los que tomamos \(4\).
  Luego ordenamos el multiconjunto de consonantes
  \(\{\mathtt{M}, \mathtt{S}^4, \mathtt{P}^2\}\),
  distribuyendo las consonantes según los tramos definidos anteriormente.
  Como estas dos decisiones
  (número de consonantes en cada tramo
   y orden de las consonantes)
  son independientes,
  aplicamos la regla del producto:
  \begin{equation*}
    \multiset{5}{4} \cdot \binom{7}{1, 4, 2}
      = 7\,350
  \end{equation*}

% Fixme: Anunciar PIE, etc

%%% Local Variables:
%%% mode: latex
%%% TeX-master: "clases"
%%% End:

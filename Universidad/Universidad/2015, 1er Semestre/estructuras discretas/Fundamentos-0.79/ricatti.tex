% ricatti.tex
%
% Copyright (c) 2013-2014 Horst H. von Brand
% Derechos reservados. Vea COPYRIGHT para detalles

\section{Recurrencia de Ricatti}
\label{sec:Ricatti}
\index{recurrencia!Ricatti|see{Ricatti, recurrencia de}}
\index{Ricatti, recurrencia de}

  Una recurrencia de la forma:
  \begin{equation}
    \label{eq:Ricatti}
    w_{n + 1}
      = \frac{a w_n + b}{c w_n + d}
  \end{equation}
  donde \(c \ne 0\)
  y \(a d - b c \ne 0\) se llama \emph{recurrencia de Ricatti}
  (si \(c = 0\) es simplemente una recurrencia lineal de primer orden;
   si \(a d = b c\) se reduce a \(w_{n + 1} = \text{constante}\)).
  Incidentalmente,
  la recurrencia~\eqref{eq:Fibonacci-search-recurrence-r}
  para~\(r\)
  que hallamos al analizar la búsqueda de Fibonacci
  en la sección~\ref{sec:busqueda-Fibonacci}%
    \index{Fibonacci, busqueda de@Fibonacci, búsqueda de!analisis@análisis}
  es una recurrencia de Ricatti.

  Para resolver este tipo de recurrencias hay varias opciones,
  que exploraremos en lo que sigue.

\subsection{Vía recurrencia de segundo orden}
\label{sec:Ricatti-2nd}

  Seguimos el esquema de Brand~\cite{brand55:_seq_def_difference_eq}.
  Si en~\eqref{eq:Ricatti} substituimos \(y_n \mapsto c w_n + d\),
  queda una recurrencia de la forma:
  \begin{equation}
    \label{eq:Ricatti-2nd-aux-1}
    y_n
      = \alpha - \frac{\beta}{y_{n - 1}}
  \end{equation}
  donde:
  \begin{align*}
    \alpha
      &= a + d \\
    \beta
      &= a d - b c
  \end{align*}
  Claramente eso solo vale si \(a d - b c \ne 0\).
  Substituyendo ahora:
  \begin{equation}
    \label{eq:Ricatti-2nd-y}
    y_n
      = \frac{x_{n + 1}}{x_n}
  \end{equation}
  resulta:
  \begin{equation}
    \label{eq:Ricatti-2nd-aux}
    x_{n + 2} - \alpha x_{n + 1} + \beta x_n
      = 0
  \end{equation}
  Esta es una recurrencia lineal de segundo orden,
  de coeficientes constantes y homogénea.
  Necesitamos dos valores iniciales para resolverla,
  podemos elegir bastante arbitrariamente \(x_0 = 1\),
  dando \(x_1 = y_0\),
  que a su vez podemos obtener de la condición inicial original.

  Para un ejemplo,
  tomemos \(w_0 = 3\) y:
  \begin{equation}
    \label{eq:Ricatti-ex}
    w_{n + 1}
      = \frac{5 w_n + 2}{3 w_n + 4}
  \end{equation}
  Siguiendo los pasos indicados:
  \begin{align*}
    3 w_{n + 1} + 4
      &= 3 \cdot \frac{5 w_n + 2}{3 w_n + 4} + 4 \\
      &= 9 - \frac{14}{3 w_n + 4}
  \end{align*}
  Substituyendo:
  \begin{equation*}
    3 w_n + 4
      = \frac{x_{n + 1}}{x_n}
  \end{equation*}
  y reordenando resulta:
  \begin{equation}
    \label{eq:Ricatti-ex-2nd}
    x_{n + 2} - 9 x_{n + 1} + 14 x_n
      = 0
  \end{equation}
  Con las condiciones iniciales \(x_0 = 1\), \(x_1 = 3 w_0 + 4 = 13\)
  la solución de~\eqref{eq:Ricatti-ex-2nd} es:
  \begin{equation*}
    x_n
      = \frac{11 \cdot 7^n - 6 \cdot 2^n}{5}
  \end{equation*}
  y finalmente:
  \begin{equation}
    \label{eq:Ricatti-ex-2nd-sol}
    w_n
      = \frac{11 \cdot 7^n + 2^{n + 2}}{11 \cdot 7^n - 3 \cdot 2^{n + 1}}
  \end{equation}

\subsection{Reducción a una recurrencia de primer orden}
\label{sec:Ricatti-1}

  Siguiendo a Mitchell~%
    \cite{mitchell00:_riccati_solution}
  definamos la secuencia auxiliar:
  \begin{equation}
    \label{eq:Ricatti-1st-x}
    x_n
      = \frac{1}{1 + \eta w_n}
  \end{equation}
  Expresamos la recurrencia~\eqref{eq:Ricatti}
  en términos de \(x_n\),
  y despejamos \(x_{n + 1}\):
  \begin{equation*}
    x_{n + 1}
      = \frac{(d \eta - c) x_n + c}
	     {(b \eta^2 - (a - d) \eta - c) x_n + a \eta + c}
  \end{equation*}
  Buscamos que esta recurrencia sea lineal,
  o sea:
  \begin{equation*}
    b \eta^2 - (a - d) \eta - c
      = 0
  \end{equation*}
  Arbitrariamente elegimos el signo positivo:
  \begin{equation}
    \label{eq:Ricatti-1st-aux-eta}
    \eta
      = \frac{a - d + \sqrt{(a - d)^2 + 4 b c}}{2 b}
  \end{equation}
  Esto lleva a la ecuación auxiliar:
  \begin{equation}
    \label{eq:Ricatti-1st-aux}
    x_{n + 1}
      = \frac{(d \eta - c) x_n + c}{a \eta + c}
  \end{equation}
  Esta es simple de resolver.

  Aplicado al mismo ejemplo anterior,
  tenemos:
  \begin{align*}
    \eta
      &= \frac{5 - 4 + \sqrt{(5 - 4)^2 + 4 \cdot 2 \cdot 3}}{2 \cdot 2}
       = \frac{3}{2}
  \end{align*}
  La recurrencia auxiliar es:
  \begin{align}
    x_{n + 1}
      &= \frac{(4 \cdot \frac{3}{2} - 3) x_n + 3}
	      {5 \cdot \frac{3}{2} + 3} \notag \\
      &= \frac{2 x_n + 2}{7}
	       \label{eq:Ricatti-ex-1st-aux}
  \end{align}
  De la condición inicial tenemos:
  \begin{equation}
    \label{eq:Ricatti-ex-1st-initial}
    x_0
      = \frac{1}{1 + \eta w_0}
      = \frac{2}{11}
  \end{equation}
  La tradicional danza para resolver recurrencias entrega:
  \begin{equation}
    \label{eq:Ricatti-ex-1st-aux-sol}
    x_n
      = \frac{2}{5} - \frac{84}{385} \cdot \left( \frac{2}{7} \right)^n
  \end{equation}
  De acá con~\eqref{eq:Ricatti-1st-x} resulta:
  \begin{equation}
    \label{eq:Ricatti-ex-1st-sol}
    w_n
      = \frac{11 \cdot 7^n + 2^{n + 2}}{11 \cdot 7^n - 3 \cdot 2^{n + 1}}
  \end{equation}
  Tal como antes.

\subsection{Transformación de Möbius}
\label{sec:Ricatti-Moebius}

  A una transformación de la forma:
  \begin{equation}
    \label{eq:Moebius-tranform}
    w = \frac{a_{1 1} z + a_{1 2}}{a_{2 1} z + a_{2 2}}
  \end{equation}
  donde \(a_{11} a_{22} - a_{12} a_{21} \ne 0\)
  (de lo contrario,
   la expresión se reduce a una constante)
  se le llama \emph{transformación de Möbius}.%
    \index{Mobius, transformacion de@Möbius, transformación de}%
    \index{recurrencia!Ricatti!transformacion de Möbius@transformación de Möbius}
  Una de sus características interesantes es que forman un grupo
  con la composición de funciones,%
    \index{grupo}
  como es fácil demostrar.
  A nosotros puntualmente nos interesa lo siguiente:
  Sean \(A(z)\), \(B(z)\) transformaciones de Möbius.
  \begin{align}
    A(z)
     &= \frac{a_{1 1} z + a_{1 2}}{a_{2 1} z + a_{2 2}}
	  \label{eq:Moebius-A} \\
    B(z)
     &= \frac{b_{1 1} z + b_{1 2}}{b_{2 1} z + b_{2 2}}
	  \label{eq:Moebius-B}
  \end{align}
  La composición es:
  \begin{equation}
    \label{eq:Moebius-AB}
    A(B(z))
      = \frac{(a_{11} b_{11} + a_{12} b_{21}) z
		 + (a_{11} b_{12} + a_{12} b_{22})}
	     {(a_{21} b_{11} + a_{22} b_{21}) z
		 + (a_{21} b_{12} + a_{22} b_{22})}
  \end{equation}
  Si representamos las transformaciones
  por las respectivas matrices de coeficientes,
  vemos que la composición corresponde al producto de las matrices.
  Lo que hace la recurrencia~\eqref{eq:Ricatti}
  es aplicar la transformación de Möbius repetidas veces:
  \begin{equation}
    \label{eq:Ricatti-Moebius}
    w_n
      = A^n(w_0)
  \end{equation}
  a lo que naturalmente corresponde
  el calcular la potencia de la matriz respectiva.
  Para cálculo numérico puede usarse
  entonces una técnica eficiente para calcular potencias,
  como las dadas en la sección~\ref{sec:potencias}.%
    \index{potencia!calculo eficiente@cálculo eficiente}

  En nuestro caso de matrices de \(2 \times 2\)
  los valores propios son simples de obtener:%
    \index{matriz!valor propio}
  \begin{equation}
    \label{eq:diagonalizable-lambda-eqn}
    (a_{11} - \lambda) (a_{22} - \lambda) - a_{12} a_{21}
      = 0
  \end{equation}
  La fórmula cuadrática da:
  \begin{equation}
    \label{eq:diagonalizable-lambda}
    \lambda
      = \frac{a_{11} + a_{22}
		\pm \sqrt{(a_{11} + a_{22})^2
			     - 4 (a_{11} a_{22} - a_{12} a{21})}}
	     {2}
  \end{equation}
  Las columnas de la matriz \(\boldsymbol{P}\) a su vez
  son los vectores propios correspondientes:
  \begin{equation}
    \label{eq:diagonalizable-p-eqn}
    \begin{pmatrix}
      a_{11} - \lambda_i & a_{12} \\
      a_{21}		 & a_{22} - \lambda_i
    \end{pmatrix}
    \cdot
    \begin{pmatrix}
      p_{1 i} \\
      p_{2 i}
    \end{pmatrix}
      =
      \begin{pmatrix}
	0 \\
	0
      \end{pmatrix}
  \end{equation}
  Por construcción,
  la matriz del sistema~\eqref{eq:diagonalizable-p-eqn}
  es singular,
  y solo da una relación entre \(p_{1 i}\) y \(p_{2 i}\).
  Podemos imponer cualquier condición que resulte cómoda.
  Conociendo \(\boldsymbol{D}\) y \(\boldsymbol{P}\)
  podemos calcular las potencias
  y así obtener la solución buscada.

  Volviendo a nuestro manoseado ejemplo,
  tenemos:
  \begin{equation}
    \label{eq:Ricatti-ex-Moebius-A}
    \boldsymbol{A} =
    \begin{pmatrix}
      5 & 2 \\
      3 & 4
    \end{pmatrix}
  \end{equation}
  Los valores propios son \(\lambda_1 = 7\) y \(\lambda_2 = 2\),
  eligiendo valores \(p_{1 i} = 1\):
  \begin{equation}
    \label{eq:eq:Ricatti-ex-Moebius-P}
    \boldsymbol{D} =
    \begin{pmatrix}
	7  & 0 \\
	0  & 2
    \end{pmatrix} \qquad
    \boldsymbol{P} =
    \begin{pmatrix}
	1  & 1 \\
	1  & - \sfrac{3}{2}
    \end{pmatrix} \qquad
    \boldsymbol{P}^{-1} =
    \begin{pmatrix}
      \sfrac{3}{5} & \sfrac{2}{5} \\
      \sfrac{2}{5} & -\sfrac{2}{5}
    \end{pmatrix}
  \end{equation}
  Podemos entonces calcular:
  \begin{align}
    \boldsymbol{A}^n
     &= \boldsymbol{P}^{-1}
	  \cdot \boldsymbol{D}^n
	  \cdot \boldsymbol{P} \notag \\
     &=
     \begin{pmatrix}
       \displaystyle \frac{3 \cdot 7^n + 2^{n + 1}}{5} &
	 \displaystyle \frac{3 \cdot 7^n - 3 \cdot 2^n}{5} \\
       \\
       \displaystyle \frac{2 \cdot 7^n - 2^{n + 1}}{5} &
	 \displaystyle \frac{2 \cdot 7^n + 3 \cdot 2^n}{5}
     \end{pmatrix}
	\label{eq:Ricatti-ex-Moebius-An}
  \end{align}
  Con esto es:
  \begin{equation}
    \label{eq:Ricatti-ex-Moebius-sol-gral}
    w_n
      = \frac{(3 \cdot 7^n + 2^{n + 1}) w_0
		+ (2 \cdot 7^n - 2^{n + 1})}
	     {(2 \cdot 7^n - 2^{n + 1}) w_0
		+ (2 \cdot 7^n + 3 \cdot 2^n)}
  \end{equation}
  Nótese que esta solución muestra explícitamente la dependencia
  de la condición inicial.
  Con nuestra condición inicial \(w_0 = 3\) resulta nuevamente:
  \begin{equation}
    \label{eq:Ricatti-ex-Moebius-sol}
    w_n
      = \frac{11 \cdot 7^n + 2^{n + 2}}{11 \cdot 7^n - 3 \cdot 2^{n + 1}}
  \end{equation}

%%% Local Variables:
%%% mode: latex
%%% TeX-master: "clases"
%%% End:

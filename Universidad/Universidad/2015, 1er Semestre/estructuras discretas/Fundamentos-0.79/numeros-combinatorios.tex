% numeros-combinatorios.tex
%
% Copyright (c) 2011-2014 Horst H. von Brand
% Derechos reservados. Vea COPYRIGHT para detalles

\chapter{Números combinatorios}
\label{cha:numeros-combinatorios}
\index{combinatoria}

  Las funciones generatrices tienen muchas aplicaciones en combinatoria,%
    \index{generatriz}
  por las razones que quedarán claras
  a través de variados ejemplos.
  En muchos casos es directo obtener recurrencias
  para los números que interesan,
  y funciones generatrices
  dan entonces forma de llegar a expresiones explícitas
  o relaciones adicionales.
  Aplicamos el método simbólico
  (desarrollado en el capítulo~\ref{cha:metodo-simbolico})
  donde es posible,
  dado que simplifica inmensamente los desarrollos.

\section{Subconjuntos y multiconjuntos}
\label{sec:sub+multi-conjuntos}

  Para obtener el número de subconjuntos%
    \index{conjunto!subconjunto}
  de \(k\) elementos tomados entre \(n\),
  podemos razonar en forma afín a la demostración
  del teorema~\ref{theo:ms-OGF} para conjuntos:
  Cada elemento aporta \(0\) o \(1\) al tamaño de un subconjunto,
  por lo que la función generatriz
  para los subconjuntos de un conjunto de \(n\) elementos no es más que:
  \begin{equation*}
    (1 + z)^n
  \end{equation*}
  y el número de interés es entonces directamente:
  \begin{equation*}
    \binom{n}{k}
      = \left[ z^k \right] (1 + z)^n
  \end{equation*}
  como ya sabíamos.

  Otra situación interesante son los multiconjuntos.%
    \index{multiconjunto!subconjunto}
  Siguiendo nuevamente el razonamiento de la demostración
  del teorema~\ref{theo:ms-OGF},
  el aporte de cada elemento es:
  \begin{equation*}
    1 + z + z^2 + \dotsb
      = \frac{1}{1 - z}
  \end{equation*}
  con lo que el número de multiconjuntos
  de \(k\) elementos tomados entre \(n\)
  es simplemente:
  \begin{equation*}
    \multiset{n}{k}
      = \left[ z^k \right] \, \left( \frac{1}{1 - z} \right)^n
      = (-1)^k \, \binom{-n}{k}
      = \binom{n + k - 1}{n - 1}
  \end{equation*}

\section[¿Cuántas secuencias
	   de \texorpdfstring{$2 n$ }{}paréntesis balanceados hay?]
	{\protect\boldmath
	   ¿Cuántas secuencias
	   de \texorpdfstring{$2 n$ }{}paréntesis balanceados
	   hay?%
       \protect\unboldmath}
\label{sec:Catalan}
\index{Catalan, numeros de@Catalan, números de}
\index{parentesis balanceados@paréntesis balanceados|see{Catalan, números de}}

  Llamemos \(C_n\)
  al número de secuencias
  de \(2 n\) paréntesis balanceados.
  Los primeros valores son:
  \begin{align*}
    1 & \quad \epsilon \\
    1 & \quad () \\
    2 & \quad () \quad (()) \\
    5 & \quad ()()() \quad ()(()) \quad (())() \quad (()()) \quad ((()))
  \end{align*}

  Toda secuencia
  de paréntesis balanceados se puede dividir en dos partes:
  El \(2 k\)\nobreakdash-ésimo paréntesis
  cierra el primer paréntesis (\(1 \le k \le n\)),
  y el resto.
  Marcando este comienzo en rojo
  está la secuencia
  \textcolor{red}{(()())}(()(()))()(()()()).
  Llamamos \emph{perfecta} a una secuencia
  con \(n = k\).
  Hay \(C_{n - 1}\) secuencias perfectas
  de largo \(2 n\),
  como muestra la siguiente biyección:
  \begin{itemize}
  \item
    Tomar una secuencia balanceada cualquiera
    de largo \(2 n - 2\),
    y encerrarla entre paréntesis
    da una secuencia perfecta de largo \(2 n\).
  \item
    Tomar una secuencia perfecta,
    y eliminar los paréntesis de más afuera
    da una secuencia balanceada.
  \end{itemize}

  Si llamamos \(\mathcal{C}\)
  la clase de secuencias de paréntesis balanceados,
  tenemos la relación simbólica:%
    \index{metodo simbolico@método simbólico}
  \begin{equation*}
    \mathcal{C}
      = \mathcal{E} + ( \mathcal{C} ) \mathcal{C}
  \end{equation*}
  Al usar \(z\) para marcar un par de paréntesis
  (sus posiciones exactas realmente no interesan),
  esto da:
  \begin{equation}
    \label{eq:Catalan}
    C(z)
      = 1 + z C^2(z)
  \end{equation}
  Ya nos habíamos tropezado con la ecuación~\eqref{eq:Catalan}
  al derivar la función generatriz para el número de árboles binarios
  en el método simbólico en el capítulo~\ref{cha:metodo-simbolico},
  y vimos que los \(C_n\)
  son los números de Catalan~\eqref{eq:Catalan-numbers}:
    \index{Catalan, numeros de@Catalan, números de}%
    \index{Catalan, numeros de@Catalan, números de!formula@fórmula}
  \begin{equation*}
    C_n
      = \frac{1}{n + 1} \, \binom{2 n}{n}
  \end{equation*}
  Expandiendo la serie:
  \begin{equation*}
    C(z)
      = 1 + z + 2 z^2 + 5 z^3 + 14 z^4 + 42 z^5 + 132 z^6
	  + 429 z^7 + 1\,430 z^8 + 4\,862  z^9 + 16\,796 z^{10} + \dotsb
  \end{equation*}

  Consideremos ahora la manera de dividir un polígono convexo
  (vale decir,
   toda diagonal cae completamente en su interior)
  en triángulos mediante diagonales.%
    \index{poligono@polígono!triangulacion@triangulación}
  La figura~\ref{fig:triangulations-pentagon}
  muestra que para el pentágono hay cinco triangulaciones.
  \begin{figure}[ht]
    \centering
    \subfloat{\pgfimage{images/triangulation-1}}
    \hspace{2em}
    \subfloat{\pgfimage{images/triangulation-2}}
    \hspace{2em}
    \subfloat{\pgfimage{images/triangulation-3}}
    \\
    \subfloat{\pgfimage{images/triangulation-4}}
    \hspace{2em}
    \subfloat{\pgfimage{images/triangulation-5}}
    \caption{División del pentágono en triángulos}
    \label{fig:triangulations-pentagon}
  \end{figure}
  Podemos considerar un polígono convexo triangulado
  como un polígono triangulado,
  un triángulo y otro polígono triangulado.
  El caso extremo es el ``polígono'' con dos vértices
  (una línea)
  que tiene una única triangulación
  (en cero triángulos).
  Si representamos la clase de triangulaciones por \(\mathcal{T}\)
  y los triángulos por \(\mathcal{Z}\),
  tenemos la expresión simbólica:%
    \index{metodo simbolico@método simbólico}
  \begin{equation}
    \label{eq:triangulation-se}
    \mathcal{T}
      = \mathcal{E} + \mathcal{T} \times \mathcal{Z} \times \mathcal{T}
  \end{equation}
  con la correspondiente ecuación funcional:
  \begin{equation}
    \label{eq:triangulation-fe}
    T(z)
      = 1 + z T^2(z)
  \end{equation}
  La solución es nuevamente los números de Catalan;%
    \index{Catalan, numeros de@Catalan, números de}
  solo que expresa el número de triangulaciones
  en términos del número de triángulos,
  no de lados del polígono.
  Vemos que un polígono de \(n\) lados se divide en \(n - 2\) triángulos,
  con lo que finalmente el número de triangulaciones
  de un polígono de \(n\)~lados es \(C_{n - 2}\).

\section{Números de Motzkin}
\label{sec:numeros-motzkin}
\index{Motzkin, numeros de@Motzkin, números de|textbfhy}

  Donaghey y Shapiro~\cite{donaghey77:_motzkin_numbers}
  indican una estrecha relación entre los números de Motzkin
  y los de Catalan,
    \index{Catalan, numeros de@Catalan, números de}%
  por lo que debieran aparecer con frecuencia similar.
  Una de las tantas estructuras que cuentan
  es el número de maneras en que pueden dibujarse cuerdas
  entre puntos sobre una circunferencia
  de manera que no se intersecten al interior ni sobre la circunferencia.
  La figura~\ref{fig:Motzkin} muestra que \(m_5 = 21\).
  \begin{figure}[ht]
    \centering
    \subfloat{\pgfimage{images/Motzkin-01}}
    \hspace{1em}
    \subfloat{\pgfimage{images/Motzkin-02}}
    \hspace{1em}
    \subfloat{\pgfimage{images/Motzkin-03}}
    \hspace{1em}
    \subfloat{\pgfimage{images/Motzkin-04}}
    \hspace{1em}
    \subfloat{\pgfimage{images/Motzkin-05}}
    \hspace{1em}
    \subfloat{\pgfimage{images/Motzkin-06}}
    \hspace{1em}
    \subfloat{\pgfimage{images/Motzkin-07}}
    \\
    \subfloat{\pgfimage{images/Motzkin-08}}
    \hspace{1em}
    \subfloat{\pgfimage{images/Motzkin-09}}
    \hspace{1em}
    \subfloat{\pgfimage{images/Motzkin-10}}
    \hspace{1em}
    \subfloat{\pgfimage{images/Motzkin-11}}
    \hspace{1em}
    \subfloat{\pgfimage{images/Motzkin-12}}
    \hspace{1em}
    \subfloat{\pgfimage{images/Motzkin-13}}
    \hspace{1em}
    \subfloat{\pgfimage{images/Motzkin-14}}
    \\
    \subfloat{\pgfimage{images/Motzkin-15}}
    \hspace{1em}
    \subfloat{\pgfimage{images/Motzkin-16}}
    \hspace{1em}
    \subfloat{\pgfimage{images/Motzkin-17}}
    \hspace{1em}
    \subfloat{\pgfimage{images/Motzkin-18}}
    \hspace{1em}
    \subfloat{\pgfimage{images/Motzkin-19}}
    \hspace{1em}
    \subfloat{\pgfimage{images/Motzkin-20}}
    \hspace{1em}
    \subfloat{\pgfimage{images/Motzkin-21}}
    \caption{Cuerdas entre cinco puntos sobre la circunferencia}
    \label{fig:Motzkin}
  \end{figure}
  Analizando la situación de la figura~\ref{fig:Motzkin}
  podemos construir una recurrencia para \(m_n\).
  Si elegimos uno de los \(n\) puntos
  este puede participar en alguna cuerda o no.
  Si no participa,
  es como si no existiera,
  esa situación aporta \(m_{n - 1}\) casos.
  Si ese punto participa en una cuerda,
  sus dos extremos quedan excluidos,
  y quedan por tender cuerdas entre los demás \(n - 2\) puntos,
  de forma que no crucen la cuerda entre manos.
  Vale decir,
  la cuerda corta el círculo en dos,
  una parte de \(k\) nodos y otra de \(n - k - 2\) nodos,
  las formas de tender cuerdas entre ellas se pueden combinar a gusto:%
    \index{recurrencia}
  \begin{equation*}
    m_n
      = m_{n - 1} + \sum_{0 \le k \le n - 2} m_k m_{n - k - 2}
  \end{equation*}
  Como condiciones de contorno tenemos \(m_0 = m_1 = 1\)
  (con la recurrencia entregan los valores conocidos
   \(m_2 = 2\) y \(m_3 = 4\)).
  Resulta:
  \begin{equation}
    \label{eq:Motzkin-recurrence}
    m_{n + 2}
      = m_{n + 1} + \sum_{0 \le k \le n} m_k m_{n - k}
    \qquad m_0 = m_1 = 1
  \end{equation}
  Aplicando las reglas con la función generatriz ordinaria \(M(z)\)
  y simplificando queda:%
    \index{generatriz}
  \begin{equation}
    \label{eq:Motzkin-fuctional}
    M(z)
      = 1 + z M(z) + z^2 M^2(z)
  \end{equation}
  de donde:
  \begin{equation}
    \label{eq:Motzkin-gf}
    M(z)
      = \frac{1 - z - \sqrt{1 - 2 z - 3 z^2}}{2 z^2}
  \end{equation}
  (elegimos el signo negativo ya que \(M(0) = m_0 = 1\)).
  Expandiendo en serie:
  \begin{equation*}
    M(z)
      = 1 + z + 2 z^2 + 4 z^3 + 9 z^4 + 21 z^5 + 51 z^6 + 127 z^8
	  + 323 z^8 + 835 z^9 + 2\,188 z^{10} + \dotsb
  \end{equation*}

  Alternativamente,
  si \(\mathcal{M}\) es la clase de las maneras de tender cuerdas,%
    \index{metodo simbolico@método simbólico}
  podemos descomponerla
  en ningún punto
  (aporta \(\mathcal{E}\));
  agregar un punto que no participa en ninguna cuerda
  (aporta \( \mathcal{Z} \times \mathcal{M}\)
  y tender una cuerda,
  lo que da cuenta de dos puntos y divide en dos conjuntos de puntos
  entre los cuales tender cuerdas
  (aporta
     \(\mathcal{Z} \times \mathcal{Z} \times \mathcal{M} \times \mathcal{M}\)).
  En total:
  \begin{equation}
    \label{eq:Motzkin-se}
    \mathcal{M}
      = \mathcal{E}
	  + \mathcal{Z} \times \mathcal{M}
	  + \mathcal{Z} \times \mathcal{Z}
	      \times \mathcal{M} \times \mathcal{M}
  \end{equation}
  Esto lleva nuevamente a la ecuación funcional~(\ref{eq:Motzkin-fuctional}).

\section{Números de Schröder}
\label{sec:numeros-Schroeder}
\index{Schroder, numeros de@Schröder, números de|textbfhy}

  De interés ocasional son los números de Schröder,
  quien los planteó como el segundo de sus cuatro problemas~%
    \cite{schroeder70:_vier_probleme},
  ver también a Stanley~\cite{stanley97:_hippar_plutar_schroed_hough}.
  Tenemos \(n\) símbolos
  (por ejemplo, \(x\))
  e interesa saber de cuántas formas se pueden ``parentizar'',
  donde la regla es más fácil de explicar recursivamente:
  \(x\) mismo es una parentización;
  y si lo son \(\sigma_1\) a \(\sigma_k\) con \(k \ge 2\),
  también lo es \((\sigma_1 \sigma_2 \dotsm \sigma_k)\).
  Por ejemplo,
  \((((x x) x (x x x)) (x x))\)
  es una parentización de \(x x x x x x x x\).
  El método simbólico aplicado a esta descripción recursiva
  lleva a:%
    \index{metodo simbolico@método simbólico}
  \begin{equation}
    \label{eq:Schroeder-symbolic}
    \mathcal{S}
      = \mathcal{Z} + \Seq_{\ge 2}(\mathcal{S})
  \end{equation}
  y a la ecuación funcional:
  \begin{equation*}
    S(z)
      = z + \frac{S^2(z)}{1 - S(z)}
  \end{equation*}
  No es aplicable directamente la fórmula de inversión de Lagrange,%
    \index{Lagrange, inversion de@Lagrange, inversión de}
  pero podemos resolver la cuadrática:
  \begin{equation}
    \label{eq:Schroeder-functional}
    2 S^2(z) - (z + 1) S(z) + z
      = 0
  \end{equation}
  lo que al descartar la solución espuria da:
  \begin{equation}
    \label{eq:Schroeder-gf}
    S(z)
      = \frac{1}{4} \, \left( 1 + z - \sqrt{1 - 6 z + z^2} \right)
  \end{equation}
  Expandiendo en serie:
  \begin{equation*}
    S(z)
      = z + z^2 + 3 z^3 + 11 z^4 + 45 z^5 + 197 z^6 + 903 z^7
	  + 4\,279 z^8 + 20\,793 z^9 + 103\,049 z^{10} + \dotsb
  \end{equation*}

  La ecuación~\eqref{eq:Schroeder-gf} es incómoda.
  Derivando~\eqref{eq:Schroeder-functional}
  y despejando \(S'(z)\):
  \begin{equation}
    \label{eq:Schroeder-differential}
    S'(z)
      = \frac{S(z) - 1}{4 S(z) - z - 1}
  \end{equation}
  Observamos de~\eqref{eq:Schroeder-gf} que:
  \begin{equation*}
    4 S(z) - z - 1
      = \sqrt{1 - 6 z + z^2}
  \end{equation*}
  Amplificando la fracción en~\eqref{eq:Schroeder-differential} por esto,
  substituyendo el resultado de despejar \(S^2(z)\)
  de~\eqref{eq:Schroeder-functional}
  y simplificando:
  \begin{equation}
    \label{eq:Schroeder-differential-2}
    (z^2 - 6 z + 1) S'(z) - (z - 3) S(z)
      = -z + 1
  \end{equation}
  Substituyendo la serie de potencias \(S(z)\)
  en~\eqref{eq:Schroeder-differential-2}
  e igualando coeficientes de \(z^n\)
  resulta la recurrencia para \(n \ge 1\)
  (esto permite evitar las situaciones especiales
   que introduce el lado derecho de~\eqref{eq:Schroeder-differential-2}):
  \begin{equation}
    \label{eq:Schroeder-recurrence}
    (n + 2) s_{n + 2} - 3 (2 n + 1) s_{n + 1} + (n - 1) s_n
      = 0
  \end{equation}
  Expandiendo~\eqref{eq:Schroeder-gf} tenemos \(s_1 = s_2 = 1\)
  como puntos de partida.

  Acá obtuvimos una ecuación diferencial lineal%
    \index{ecuacion diferencial@ecuación diferencial}
  manipulando la ecuación funcional para la función generatriz,
  y de ella extrajimos una recurrencia,
  más cómoda para calcular los coeficientes que la función generatriz.
  Esto puede extenderse
  al caso general de ecuaciones funcionales algebraicas,
  como muestra Bostan, Chyzak, Salvy, Lecerf y Schost~%
    \cite{bostan07:_differ_equat_algeb_funct}.

  Otra opción es el camino siguiente:
  \begin{align*}
    \frac{S(z)}{z}
      &= 1 + \frac{S(z)}{z} \, \frac{S(z)}{1 - S(z)} \\
    \intertext{Despejando \(S / z\) resulta:}
    S(z)
      &= z \, \frac{1 - S(z)}{1 - 2 S(z)}
  \end{align*}
  donde sí es aplicable inversión de Lagrange.
  El resultado es complicado,
  y lo omitiremos.

\section{Números de Stirling de segunda especie}
\label{sec:Stirling-2}
\index{Stirling, numeros de@Stirling, números de!segunda especie|textbfhy}

% Fixme: Agregar más manipulación, identidades, manipulación (GKP, TAoCP)

  Interesa el número de maneras
  de dividir el conjunto \(\{1, 2, 3, 4\}\) en dos clases,
  como ilustra el cuadro~\ref{tab:S-4-2}.
  \begin{table}[htbp]
    \centering
    \begin{tabular}{*{2}{>{\(}l<{\)}}}
      \{1\}	  & \{2, 3, 4\} \\
      \{1, 2\}	  & \{3, 4\}	\\
      \{1, 3\}	  & \{2, 4\}	\\
      \{1, 4\}	  & \{2, 4\}	\\
      \{1, 2, 3\} & \{4\}	\\
      \{1, 2, 4\} & \{3\}	\\
      \{1, 3, 4\} & \{2\}
    \end{tabular}
    \caption{Las 7 particiones de 4 elementos en 2 clases}
    \label{tab:S-4-2}
  \end{table}
  Esto muestra que hay 7 particiones de 4 elementos en 2 clases.
  El número de maneras de dividir un conjunto en clases
  lo da el \emph{número de Stirling de segunda especie},
  se anota \(\classes{n}{k}\) para el número de formas
  de dividir un conjunto de \(n\) elementos en \(k\) particiones.
  Cabe hacer notar que esta notación,
  originada por Karamata~%
    \cite{karamata35:_sommab_exp},
  es relativamente común
  (uno de sus campeones es Knuth,%
     \index{Knuth, Donald E.}
   por ejemplo~\cite{graham94:_concr_mathem, knuth92:_two_notes_notat}),
  aunque hay una gran variedad de notaciones,
  algunas con signo.
  Claramente para los números de Bell%
    \index{Bell, Eric Temple}%
    \index{Bell, numeros de@Bell, números de}
  vistos en la sección~\ref{sec:rotulados}:
  \begin{equation}
    \label{eq:Bell-Stirling}
    B_n
      = \sum_{1 \le k \le n} \classes{n}{k}
  \end{equation}

  Una aplicación es
  contar el número de funciones sobre de \([n]\) a \([k]\):%
    \index{funcion@función!sobreyectiva!numero@número}
  Corresponde
  a particionar el dominio en las preimágenes de cada elemento del rango,
  y podemos asignar valores de la función
  a cada una de las \(k\)~particiones
  de \(k!\)~maneras,
  con lo que \(\classes{n}{k} k!\) es el valor buscado.

  Para obtener \(\classes{n}{k}\),
  consideremos dos grupos de particiones:%
    \index{Stirling, numeros de@Stirling, números de!segunda especie!recurrencia}
  \begin{description}
  \item[\boldmath Aquellas en que \(n\) está solo:\unboldmath]
    Corresponden a tomar \(k - 1\) particiones
    de los demás \(n - 1\) elementos,
    hay \(\classes{n - 1}{k - 1}\) de estas.
  \item[\boldmath Aquellas en que \(n\) está con otros elementos:\unboldmath]
    Se construyen en base a \(k\) clases
    de los restantes \(n - 1\) elementos
    vía agregar \(n\) a cada clase,
    hay \(k \cdot \classes{n - 1}{k}\) de estas.
  \end{description}
  Estas dos opciones son excluyentes y exhaustivas,
  y:
  \begin{equation}
    \label{eq:recurrence-Stirling-2}
    \classes{n}{k} = \classes{n - 1}{k - 1} + k \classes{n - 1}{k}
  \end{equation}
  Donde:
  \begin{equation*}
    \classes{n}{0}
      = [n = 0]
    \qquad
    \classes{n}{n}
      = 1
  \end{equation*}
  Si además decretamos:
  \begin{equation*}
    \classes{n}{k} =
      \begin{cases}
	0 & n < 0 \\
	0 & k < 0 \\
	0 & k > n
      \end{cases}
  \end{equation*}
  la recurrencia \emph{siempre} se cumple.
  Una tabla de los números de Stirling de segunda especie
  en forma de triángulo
  (como el triángulo de Pascal del cuadro~\ref{tab:triangulo-Pascal})
  comienza como ilustra el cuadro~\ref{tab:triangulo-Stirling-2}.
  \begin{table}[htbp]
    \centering
    \begin{tabular}{>{\(}r<{\)}*{12}{>{\(}c<{\)}@{\hspace{1ex}}}>{\(}c<{\)}}
      n=0:& \phantom{00}
		& \phantom{00}
		    & \phantom{00}
			& \phantom{00}
			    & \phantom{00}
				 & \phantom{00}
				      & 1 \\
	 \noalign{\smallskip\smallskip}
      n=1:&	&   &	&   &	 &  0 & \phantom{00}
					  &  1 \\
	 \noalign{\smallskip\smallskip}
      n=2:&	&   &	&   &  0 &    &	 1 & \phantom{00}
					       &  1 \\
	 \noalign{\smallskip\smallskip}
      n=3:&	&   &	& 0 &	 &  1 &	   &  3 & \phantom{00}
						    &  1 \\
	 \noalign{\smallskip\smallskip}
      n=4:&	&   & 0 &   &  1 &    &	 7 &	&  6 & \phantom{00}
							 &  1 \\
	 \noalign{\smallskip\smallskip}
      n=5:&	& 0 &	& 1 &	 & 15 &	   & 25 &    & 10 & \phantom{00}
							      &	 1
	      & \phantom{00} \\
	 \noalign{\smallskip\smallskip}
      n=6:& 0 &	  & 1 &	  & 31 &    & 90 &	& 65 &	  & 15 & \phantom{00}
								    &  1 \\
	 \noalign{\smallskip\smallskip}
    \end{tabular}
    \caption{Números de Stirling de segunda especie}
    \label{tab:triangulo-Stirling-2}
    \index{Stirling, numeros de@Stirling, números de!segunda especie!cuadro}
  \end{table}
  Podemos optar entre tres funciones generatrices:
  Multiplicar por \(u^k\) y sumar sobre \(k\),
  multiplicar por \(z^n\) y sumar sobre \(n\)
  o multiplicar por \(u^k z^n\) sumar sobre ambos índices.
  Al sumar sobre \(k\) el factor \(k\) resulta en una derivada,
  se optaría por sumar solo sobre \(n\),
  que da ecuaciones más simples de tratar.
  Resulta eso sí una recurrencia para la función generatriz del caso.
  El desarrollo es largo,
  y lo omitiremos,
  vea el texto de Wilf~\cite{wilf06:_gfology}.

  Es más simple aplicar el método simbólico.%
    \index{metodo simbolico@método simbólico}
  Para obtener el número de particiones de \(n\) elementos en \(k\) clases
  partimos de la expresión simbólica:
  \begin{equation*}
    \mathcal{S}
      = \MSet(\mathcal{U} \times \MSet_{\ge 1}(\mathcal{Z}))
  \end{equation*}
  en la que \(\mathcal{U}\) contiene un único elemento de tamaño uno
  (usado para contabilizar el número de clases
   asociándolo a la variable \(u\),
   mientras \(z\) cuenta el número de elementos total),
  lo que lleva a la función generatriz mixta%
    \index{generatriz!multivariada}
  (exponencial en \(z\),
   los elementos están rotulados;
   y ordinaria en \(u\),
   las clases no lo están):
  \begin{equation}
    \label{eq:Stirling-2-EGF}
    \index{Stirling, numeros de@Stirling, números de!segunda especie!funcion generatriz@función generatriz}
    S(z, u)
      = \mathrm{e}^{u (\mathrm{e}^z - 1)}
  \end{equation}
  De acá:
  \begin{align}
    \classes{n}{k}
      &= n! \left[ z^n u^k \right] S(u, z) \notag \\
      &= n! \left[ z^n \right] \, \frac{(\mathrm{e}^z - 1)^k}{k!} \notag \\
      &= \frac{n!}{k!} \left[ z^n \right] \,
	    \sum_{0 \le r \le k} (-1)^{k - r} \,
	       \binom{k}{r} \, \mathrm{e}^{r z} \notag \\
      &= \frac{n!}{k!}
	    \sum_{0 \le r \le k} (-1)^{k - r} \, \binom{k}{r} \, \frac{r^n}{n!}
		\notag \\
      &= \sum_{0 \le r \le k} \frac{(-1)^{k - r} r^n}{r! (k - r)!}
		\label{eq:Stirling-2-explicit}
  \end{align}

\section{Números de Stirling de primera especie}
\label{sec:Stirling-1}
\index{Stirling, numeros de@Stirling, números de!primera especie|textbfhy}

  Interesa el número de maneras
  de organizar \(n\) elementos en \(k\) ciclos.
  Esto queda representado por la expresión simbólica:%
    \index{metodo simbolico@método simbólico}
  \begin{equation*}
    \mathcal{C}
      = \MSet(\mathcal{U} \times \Cyc(\mathcal{Z}))
  \end{equation*}
  de donde el método simbólico entrega directamente
  la función generatriz mixta%
    \index{generatriz!multivariada}
  (exponencial en \(z\),
   los elementos están rotulados;
   y ordinaria en \(u\),
   los ciclos no lo están):
  \begin{equation}
    \label{eq:Stirling-1-EGF}
    \index{Stirling, numeros de@Stirling, números de!primera especie!funcion generatriz@función generatriz}
    C(z, u)
      = \exp \left( u \ln \frac{1}{1 - z} \right)
      = (1 - z)^{-u}
  \end{equation}
  Los coeficientes
  se conocen como \emph{números de Stirling de primera especie}
  y se anota \(\cycle{n}{k}\)
  (nuevamente notación impulsada por Knuth~%
    \index{Knuth, Donald E.}%
    \cite{knuth92:_two_notes_notat}).
  Se lee ``\(n\) ciclo \(k\)''
  (en inglés
   se expresa \emph{\(n\) \foreignlanguage{english}{cycle} \(k\)}).

  Para derivar una recurrencia para ellos,%
    \index{Stirling, numeros de@Stirling, números de!primera especie!recurrencia}
  consideremos cómo podemos construir
  una organización de \(n\) objetos con \(k\) ciclos
  a partir de \(n - 1\) objetos.
  Al agregar el nuevo objeto,
  podemos ponerlo en un ciclo por sí mismo,
  lo que puede hacer de una única manera
  partiendo
  de cada una de las \(\cycle{n - 1}{k - 1}\) organizaciones
  de \(n - 1\) elementos con \(k - 1\) ciclos.
  La otra opción
  es insertarlo en alguno de los \(k\) ciclos ya existentes.
  Si suponemos \(n - 1\) elementos y \(k\) ciclos:
  \begin{equation*}
    (a_1 \, a_2 \dotso a_{j_1})
    (a_{j_1 + 1} \, a_{j_1 + 2} \dotso a_{j_2})
    \dotso
    (a_{j_{k - 1} + 1} \, a_{j_{k - 1} + 2} \dotso a_{n - 1})
  \end{equation*}
  En ella podemos insertar el nuevo elemento antes de cada elemento,
  agregándolo al ciclo al que este pertenece;
  insertar el elemento al final del ciclo
  es lo mismo que ubicarlo al comienzo de éste,
  por lo que esto no aporta nuevas opciones.
  De esta forma para cada caso
  hay \(n - 1\) posibilidades de insertar el nuevo elemento:
  \begin{equation}
    \label{eq:recurrence-Stirling-1}
    \cycle{n}{k}
      = (n - 1) \cycle{n - 1}{k} + \cycle{n - 1}{k - 1}
  \end{equation}
  Para condiciones de contorno,
  tenemos:
  \begin{equation*}
    \cycle{n}{0}
      = [n = 0]
    \qquad
    \cycle{n}{n}
      = 1
  \end{equation*}
  Si además decretamos:
  \begin{equation*}
    \cycle{n}{k} =
      \begin{cases}
	0 & n < 0 \\
	0 & k < 0 \\
	0 & k > n
      \end{cases}
  \end{equation*}
  la recurrencia \emph{siempre} se cumple.

  En forma de triángulo à la Pascal
  tenemos el cuadro~\ref{tab:triangulo-Stirling-1}.
  \begin{table}[htbp]
    \centering
    \begin{tabular}{>{\(}r<{\)}*{12}{>{\(}c<{\)}@{\hspace{1ex}}}>{\(}c<{\)}}
      n=0:& \phantom{000}
		  & \phantom{000}
		       & \phantom{000}
			    & \phantom{000}
				 & \phantom{000}
				      & \phantom{000}
					   &  1 \\
	 \noalign{\smallskip\smallskip}
      n=1:&	  &    &    &	 &    &	 0 & \phantom{000}
						&  1 \\
	 \noalign{\smallskip\smallskip}
      n=2:&	  &    &    &	 &  0 &	   &  1 & \phantom{000}
						     &	1 \\
	 \noalign{\smallskip\smallskip}
      n=3:&	  &    &    &  0 &    &	 2 &	&  3 & \phantom{000}
							  &  1 \\
	 \noalign{\smallskip\smallskip}
      n=4:&	  &    &  0 &	 &  6 &	   & 11 &    &	6 & \phantom{000}
							       &  1 \\
	 \noalign{\smallskip\smallskip}
      n=5:&	  &  0 &    & 24 &    & 50 &	& 35 &	  & 10 & \phantom{000}
								    &  1
	       & \phantom{000} \\
	 \noalign{\smallskip\smallskip}
      n=6:& 0	  &    &120 &	 &274 &	   &225 &    & 85 &    & 15
	       & \phantom{000} &  1 \\
	 \noalign{\smallskip\smallskip}
    \end{tabular}
    \caption{Números de Stirling de primera especie}
    \label{tab:triangulo-Stirling-1}
    \index{Stirling, numeros de@Stirling, números de!primera especie!cuadro}
  \end{table}
  Hay fórmulas explícitas,
  pero son complicadas y las omitiremos.

\section{Números de Lah}
\label{sec:numeros-lah}
\index{Lah, numeros de@Lah, números de|textbfhy}
\index{Stirling, numeros de@Stirling, números de!tercera especie|see{ Lah, números de}}

  Los números de Lah~%
    \cite{lah54:_new_kind_number}
  cuentan el número de maneras
  de ordenar \(n\) elementos en \(k\) secuencias.
  Se les ha llamado ``números de Stirling de tercera especie''
  por analogía a los anteriores,
  y Petkovšek y Pisanski~%
    \cite{petko02:_combin_lah_stirling}
  les dan la notación \(\lah{n}{k}\) que usaremos.
  También es común \(L_{n, k}\).
  Queda representado por la expresión simbólica:%
    \index{metodo simbolico@método simbólico}
  \begin{equation*}
    \mathcal{L}
      = \MSet(\mathcal{U} \times \Seq_{\ge 1}(\mathcal{Z}))
  \end{equation*}
  y el método simbólico da la función generatriz mixta%
    \index{generatriz!multivariada}
  (elementos rotulados,
   secuencias sin rotular):
  \begin{equation}
    \label{eq:Lah-EGF}
    L(z, u)
      = \mathrm{e}^{u ((1 - z)^{-1} - 1)}
      = \mathrm{e}^{u z (1 - z)^{-1}}
  \end{equation}
  Podemos extraer una fórmula explícita de~\eqref{eq:Lah-EGF}:%
    \index{Lah, numeros de@Lah, números de!formula@fórmula}
  \begin{align}
    \lah{n}{k}
      &= n! \left[ u^k z^n \right]
	      \, \exp \left( u z (1 - z)^{-1} \right) \\
      &= n! \left[ z^n \right] \, \frac{z^k (1 - z)^{-k}}{k!} \\
      &= \frac{n!}{k!} \, \left[ z^{n - k} \right] (1 - z)^{-k} \\
      &= \frac{n!}{k!} \, (-1)^{n - k} \binom{-k}{n - k} \\
      &= \frac{n!}{k!} \, \binom{n - 1}{k - 1}
  \end{align}

  Para derivar una recurrencia para estos números,%
    \index{Lah, numeros de@Lah, números de!recurrencia}
  usamos la misma técnica anterior.
  Veamos cómo podemos construir
  \(k\) secuencias tomadas entre \(n\) elementos
  partiendo con las configuraciones de \(n - 1\) elementos.
  Hay dos posibilidades exhaustivas y excluyentes:
  \(n\) forma una secuencia por sí solo,
  cosa que puede hacerse de una única forma
  partiendo
  con cada una de las \(\lah{n - 1}{k - 1}\) configuraciones
  con \(n - 1\) elementos y \(k - 1\) secuencias;
  o podemos agregar \(n\) a alguna de \(k\) secuencias
  en configuraciones de \(n - 1\) elementos.
  Podemos agregar un nuevo elemento a una secuencia de largo \(e\)
  de \(e + 1\) formas
  (antes del primero,
   o después de cada uno de los \(e\) elementos).
  Como los largos de las secuencias suman \(n - 1\),
  en total
  creamos \((n + k - 1) \lah{n - 1}{k}\) nuevas configuraciones.
  Resulta:
  \begin{equation}
    \label{eq:recurrence-Lah}
    \lah{n}{k}
      = (n + k - 1) \lah{n - 1}{k} + \lah{n - 1}{k - 1}
  \end{equation}
  Para condiciones de contorno tenemos:
  \begin{equation*}
    \lah{n}{0}
      = [n = 0]
    \qquad
    \lah{n}{n}
      = 1
  \end{equation*}
  Si además decretamos:
  \begin{equation*}
    \lah{n}{k} =
      \begin{cases}
	0 & n < 0 \\
	0 & k < 0 \\
	0 & k > n
      \end{cases}
  \end{equation*}
  la recurrencia \emph{siempre} se cumple.

  Al estilo del triángulo de Pascal
  tenemos el cuadro~\ref{tab:triangulo-Lah}.
  \begin{table}[htbp]
    \centering
    \begin{tabular}{>{\(}r<{\)}*{12}{>{\(}c<{\)}@{\hspace{1ex}}}>{\(}c<{\)}}
      n=0:& \phantom{0000}
		  & \phantom{0000}
		       & \phantom{0000}
			    & \phantom{0000}
				 & \phantom{0000}
				      & \phantom{0000}
					   &  1 \\
	 \noalign{\smallskip\smallskip}
      n=1:&	  &    &     &	  &	 &   0 & \phantom{0000} &  1 \\
	 \noalign{\smallskip\smallskip}
      n=2:&	  &    &     &	  &    0 &     &  1 & \phantom{0000} &	 1 \\
	 \noalign{\smallskip\smallskip}
      n=3:&	  &    &     &	 0 &	 &   6 &	 &   6
	       & \phantom{0000} &  1 \\
	 \noalign{\smallskip\smallskip}
      n=4:&	  &    &   0 &	  &   24 &     & 36 &	  & 12
	       & \phantom{0000} &  1 \\
	 \noalign{\smallskip\smallskip}
      n=5:&	  &  0 &     & 120 &	 & 240 &	  & 120 &    &	20
	       & \phantom{0000} & 1 & \phantom{0000} \\
	 \noalign{\smallskip\smallskip}
      n=6:& 0	  &    & 720 &	   &1800 &	   &1200 &    & 300 &	 & 30
	       & \phantom{0000} &  1 \\
	 \noalign{\smallskip\smallskip}
    \end{tabular}
    \caption{Números de Lah}
    \label{tab:triangulo-Lah}
    \index{Lah, numeros de@Lah, números de!cuadro}
  \end{table}

\section{Potencias, números de Stirling y de Lah}
\label{sec:potencias-Stirling-Lah}
\index{Stirling, numeros de@Stirling, números de!y potencias}
\index{Lah, numeros de@Lah, números de!y potencias}

  Las potencias factoriales aparecen con bastante regularidad%
    \index{potencia!factorial}
  al calcular con diferencias finitas,%
    \index{diferencias finitas}
  como muestran entre otros Graham, Knuth y~Patashnik~%
    \cite{graham94:_concr_mathem}.
  Como vimos
  en la sección~\ref{sec:preliminares-potencias-factoriales}
  hay paralelos entre las diferencias finitas
  y la derivada,
  y entre la sumatoria y la integral.
  Este tipo de relaciones se explotan en el cálculo umbral~%
    \cite{roman78:_umbral_calculus}.%
    \index{calculo umbral@cálculo umbral}

  Interesa expresar la potencia \(z^n\)
  en términos de los \(z^{\underline{k}}\),
  o sea obtener los coeficientes \(S(n, k)\)
  en la expansión siguiente:
  \begin{equation*}
    z^n = \sum_{0 \le k \le n} S(n, k) z^{\underline{k}}
  \end{equation*}
  Cuando \(k < 0\) o \(k > n\) debe ser \(S(n, k) = 0\),
  con lo que los límites en realidad son superfluos.
  Además,
  para \(n = 0\)
  resulta \(S(0, 0) = 1\),
  es \(S(n, 0) = 0\) si \(n > 0\)
  y es claro que \(S(n, n) = 1\).

  Escribamos:
  \begin{align*}
    z^{n + 1}
      &= \sum_k S(n, k) z^{\underline{k}} \cdot z \\
      &= \sum_k
	   S(n, k) (z^{\underline{k + 1}} + k z^{\underline{k}}) \\
      &= \sum_k (S(n, k - 1) + k S(n, k)) z^{\underline{k}}
  \end{align*}
  Comparando coeficientes de esto con la expansión de \(z^{n + 1}\):
  \begin{equation*}
    S(n + 1, k) = k S(n, k) + S(n, k - 1)
  \end{equation*}
  Tenemos las condiciones de contorno:
  \begin{equation*}
    S(n, 0)
      = [n = 0]
    \qquad
    S(n, n)
      = 1
  \end{equation*}
  El lector astuto reconocerá esto
  como la recurrencia~\eqref{eq:recurrence-Stirling-2}
  que obtuvimos para los números de Stirling de segunda especie,
  tenemos:
  \begin{equation}
    \label{eq:Stirling-2-down}
    z^n = \sum_k \classes{n}{k} z^{\underline{k}}
  \end{equation}

  Veamos ahora los coeficientes \(C(n, k)\) en:
  \begin{equation*}
    z^{\overline{n}}
      = \sum_k C(n, k) z^k
  \end{equation*}
  Con esto:
  \begin{align*}
    z^{\overline{n + 1}}
      &= \sum_k C(n, k) z^k (z + n) \\
      &= \sum_k (C(n, k) z^{k + 1} + n C(n, k) z^k) \\
      &= \sum_k (C(n, k - 1) + n C(n, k)) z^k
  \end{align*}
  Comparando coeficientes
  con la expansión de \(z^{\overline{n + 1}}\):
  \begin{equation*}
    C(n + 1, k)
      = n C(n, k) + C(n, k - 1)
  \end{equation*}
  con condiciones de contorno:
  \begin{equation*}
    C(n, 0) = [n = 0] \qquad C(n, n) = 1
  \end{equation*}
  Esto coincide con los números de Stirling de primera especie,
  o sea es:%
    \index{potencia!factorial}
  \begin{equation}
    \label{eq:Stirling-1-up}
    z^{\overline{n}}
      = \sum_k \cycle{n}{k} z^k
  \end{equation}

  En vista que nos ha ido tan bien con esto,
  consideremos los coeficientes \(L(n, k)\) en:
  \begin{equation*}
    z^{\overline{n}}
      = \sum_k L(n, k) z^{\underline{k}}
  \end{equation*}
  Aplicando la misma estrategia:
  \begin{align*}
    z^{\overline{n + 1}}
      &= \sum_k L(n, k) z^{\underline{k}} (z + n) \\
      &= \sum_k L(n, k) (z^{\underline{k + 1}} + n + k) \\
      &= \sum_k (L(n, k - 1) + (n + k) L(n, k)) z^{\underline{k}}
  \end{align*}
  Comparar coeficientes da:
  \begin{equation*}
    L(n + 1, k)
      = (n + k) L(n, k) + L(n, k - 1)
  \end{equation*}
  con condiciones de borde:
  \begin{equation*}
    L(n, 0) = [n = 0] \qquad L(n, n) = 1
  \end{equation*}
  Esta es la recurrencia~\eqref{eq:recurrence-Lah}
  de los números de Lah:
  \begin{equation}
    \label{eq:Lah-up}
    z^{\overline{n}}
      = \sum_k \lah{n}{k} z^{\underline{k}}
  \end{equation}

  Podemos usar
  la identidad \((-z)^{\underline{m}} = (-1)^m z^{\overline{m}}\)
  y viceversa para obtener de~\eqref{eq:Stirling-2-down},
  \eqref{eq:Stirling-1-up}
  y~\eqref{eq:Lah-up}:
  \begin{align}
    z^n
      &= \sum_k (-1)^{n - k} \classes{n}{k} z^{\overline{k}}
	      \label{eq:Stirling-2-up} \\
    z^{\underline{n}}
      &= \sum_k (-1)^{n - k} \cycle{n}{k} z^k
	      \label{eq:Stirling-1-down} \\
    z^{\underline{n}}
      &= \sum_k (-1)^{n - k} \lah{n}{k} z^{\overline{k}}
	      \label{eq:Lah-down}
  \end{align}

  La impresionante colección de identidades~%
    \eqref{eq:Stirling-2-down},
    \eqref{eq:Stirling-1-up},
    \eqref{eq:Lah-up},
    \eqref{eq:Stirling-2-up},
    \eqref{eq:Stirling-1-down}
    y \eqref{eq:Lah-down}
  cumple la promesa dada por el título.

\section{Desarreglos}
\label{sec:desarreglos}
\index{desarreglo}

  Ya calculamos el número de desarreglos
  (permutaciones sin puntos fijos)
  como ejemplo del uso del principio de inclusión y exclusión
  en el capítulo~\ref{cha:pie}
  y como ejemplo del método simbólico
  en la sección~\ref{sec:rotulados}.
  Acá veremos técnicas alternativas.

  Llamamos \(D_n\) al número de desarreglos
  de \(n\) elementos.
  Valores de \(D_n\) para \(n\) pequeños son interesantes
  para contrastar nuestros resultados luego:
  \(D_0 = 1\)
  (hay una única manera de ordenar cero elementos,
   y en esa ningún elemento está en su posición),
  \(D_1 = 0\)
  (un elemento puede ordenarse de una manera solamente,
   y ese siempre está en su posición),
  \(D_2 = 1\)
  (solo \((2\,1)\)),
  \(D_3 = 2\)
  (son \((3\,1\,2)\) y \((2\,3\,1)\)).

  Interesa obtener una recurrencia para los \(D_n\).%
    \index{desarreglo!recurrencia}
  Comenzaremos considerando una permutación de 7 elementos
  como \((2 \; 6 \; \mathbf{3} \; \mathbf{4} \; 1 \; 5 \; \mathbf{7})\).
  Esta permutación tiene tres puntos fijos
  (marcados con negrilla):
  El \(3\) está en la posición \(3\),
  el \(4\) en la posición \(4\)
  y el \(7\) en la posición \(7\).
  Hay \(D_{7 - 3}\) formas
  de ordenar los demás \(4 = 7 - 3\) elementos
  sin introducir puntos fijos adicionales
  (debemos dejarlos desordenados
   con respecto a sus posiciones propias,
   que estas se llamen \(1, 2, 3, 4\) o \(1, 2, 5, 6\) da lo mismo);
  a su vez los \(3\) elementos fijos
  se pueden elegir de \(\binom{7}{3}\) formas.
  O sea,
  hay un total de \(\binom{7}{3} \cdot D_{7 - 3}\) permutaciones
  de \(7\) elementos con \(3\) puntos fijos.

  En general,
  si hay \(k\) puntos fijos en una permutación de \(n\) elementos,
  estos pueden elegirse de \(\binom{n}{k}\) maneras;
  por lo tanto
  hay exactamente \(\binom{n}{k} \cdot D_{n - k}\) permutaciones
  con \(k\) puntos fijos.
  Toda permutación tiene puntos fijos (\(0, 1, \dotsc, n\) de ellos)
  y hay un total de \(n!\)~permutaciones,
  por lo que concluimos:
  \begin{equation}
    \label{eq:relacion-desarreglos}
    n! = \sum_{0 \le k \le n} \binom{n}{k} \cdot D_{n - k}
  \end{equation}
  Esta relación es válida para \(n \ge 0\).

  En la recurrencia~\eqref{eq:relacion-desarreglos}
  se observa que el lado derecho
  es la convolución binomial
  de la secuencia \(\left\langle 1 \right\rangle_{n \ge 0}\)
  con la secuencia \(\left\langle D_n \right\rangle_{n \ge 0}\).
  En consecuencia definimos:
  \begin{equation*}
    \widehat{D}(z)
      = \sum_{n \ge 0} D_n \, \frac{z^n}{n!}
  \end{equation*}
  Aplicamos las propiedades de funciones generatrices exponenciales%
    \index{generatriz!exponencial}
  a~\eqref{eq:relacion-desarreglos}.
  Al lado izquierdo queda una suma geométrica
  al simplificarse los factoriales:
  \begin{align}
    \frac{1}{1 - z}
      &= \mathrm{e}^z \cdot \widehat{D}(z) \notag \\
    \widehat{D}(z)
      &= \frac{\mathrm{e}^{-z}}{1 - z}
	\label{eq:EGF-desarreglos}
	\index{desarreglo!funcion generatriz@función generatriz}
  \end{align}
  De~\eqref{eq:EGF-desarreglos}:
  \begin{align*}
    D_n
      &= n! \left[ z^n \right] \widehat{D}(z) \\
      &= n! \sum_{0 \le k \le n} \! \frac{(-1)^k}{k!}
  \end{align*}
  En términos de la exponencial truncada
  definida por~\eqref{eq:exp-truncada}
  resulta~\eqref{eq:n-derangements}:%
    \index{desarreglo!expresion@expresión}
  \begin{equation*}
    D_n
      = n! \cdot \exp \rvert_n (-1)
  \end{equation*}
  Tenemos:
  \begin{equation*}
    \widehat{D}(z)
      = 1 + \frac{1}{2!} z^2 + \frac{2}{3!} z^3
	  + \frac{9}{4!} z^4 + \frac{44}{5!} z^5
	  + \frac{265}{6!} z^6 + \frac{1\,854}{7!} z^7
	  + \frac{14\,833}{8!} z^8 + \frac{133\,496}{9!} z^9
	  + \dotsb
  \end{equation*}

  Otra forma de resolver esto es partir derivando directamente
  una recurrencia para los \(D_n\).%
    \index{desarreglo!recurrencia}
  Consideremos \(n\) personas
  que eligen entre \(n\) sombreros
  de manera que ninguna se lleva el suyo.
  Numeramos a las personas
  y los respectivos sombreros de~\(1\) a~\(n\).
  La persona~\(n\)
  puede elegir el sombrero equivocado de~\(n - 1\) maneras,
  supongamos que elige el sombrero~\(k\).
  Ahora hay dos posibilidades:
  Si la persona~\(k\) toma el sombrero~\(n\),
  podemos eliminar los sombreros
  (y las personas)~\(n\) y~\(k\) de consideración,
  y el problema se reduce a distribuir
  los~\(n - 2\) sombreros restantes
  entre las otras~\(n - 2\) personas.
  Si la persona~\(k\) no toma el sombrero~\(n\),
  podemos renombrar ese como~\(k\)
  (no lo toma~\(k\) porque ahora le corresponde)
  y quedan por distribuir los \(n - 1\) sombreros restantes
  sin que a nadie le toque el suyo.
  Al revés,
  de un par de desarreglos de \(n - 1\) y \(n - 2\) elementos
  podemos construir \(n - 1\) desarreglos de \(n\) elementos
  (podemos elegir \(k\) arriba de \(n - 1\) maneras en ambos casos).
  Obtenemos:
  \begin{equation}
    \label{eq:recurrence-desarreglos}
    D_n
      = (n - 1) (D_{n - 1} + D_{n - 2})
    \quad (n \ge 2)
    \qquad \text{\(D_0 = 1\), \(D_1 = 0\)}
    \index{desarreglo!recurrencia}
  \end{equation}

  Para resolver la recurrencia~\eqref{eq:recurrence-desarreglos},
  definimos una función generatriz exponencial.%
    \index{generatriz!exponencial}
  La razón de intentar una función generatriz exponencial
  en este caso
  es que el factor \(n - 1\)
  se compensa parcialmente con \((n - 1)!\) en el denominador.
  Para \(n \ge 1\) podemos escribir:
  \begin{equation}
    \label{eq:recurrence-desarreglos-1}
    D_{n + 1}
      = n D_n + n D_{n - 1}
  \end{equation}
  Las propiedades de las funciones generatrices exponenciales
  dan:
  \begin{align*}
    \widehat{D}'(z)
      &\egf \left\langle D_{n + 1} \right\rangle_{n \ge 0} \\
    z \widehat{D}'(z)
      &\egf \left\langle n D_n \right\rangle_{n \ge 0}
  \end{align*}
  Falta el segundo término del lado derecho
  de~\eqref{eq:recurrence-desarreglos-1}:
  \begin{equation*}
    \sum_{n \ge 1} n D_{n - 1} \, \frac{z^n}{n!}
      = z \sum_{n \ge 1} D_{n - 1} \, \frac{z^{n - 1}}{(n - 1)!}
      = z \widehat{D}(z)
  \end{equation*}
  Otra forma de ver esto
  es que la secuencia
    \(\left\langle D_{n - 1} \right\rangle_{n \ge 0}\)
  corresponde a la antiderivada de \(\widehat{D}(z)\)
  (correr una posición a la derecha),
  y al multiplicar por \(n\)
  (que corresponde al operador \(z \mathrm{D}\))
  la derivada y la antiderivada se cancelan.
  Combinando las anteriores:
  \begin{align*}
    \widehat{D}'(z)
      &= z \widehat{D}'(z) + z \widehat{D}(z)
	   \qquad \widehat{D}(0) = D_0 = 1 \\
    \frac{\widehat{D}'(z)}{\widehat{D}(z)}
      &= \frac{z}{1 - z} \\
  \end{align*}
  La solución de esta ecuación diferencial es:%
    \index{ecuacion diferencial@ecuación diferencial}
  \begin{equation*}
    \widehat{D}(z)
      = \frac{\mathrm{e}^{-z}}{1 - z}
  \end{equation*}
  Con esto nuevamente tenemos~\eqref{eq:n-derangements}:
  \begin{equation*}
    D_n
      = n! \cdot  \exp \rvert_n (-1)
  \end{equation*}

  Aún otra manera de enfrentar este problema
  es masajear la recurrencia~\eqref{eq:recurrence-desarreglos-1}
  para obtener una recurrencia más simple de resolver:%
    \index{desarreglo!recurrencia}
  \begin{align}
    D_n - n D_{n - 1}
      &= - D_{n - 1} + (n - 1) D_{n - 2} \notag \\
      &= - \left( D_{n - 1} - (n - 1) D_{n - 2} \right) \notag \\
    (-1)^n \left( D_n - n D_{n - 1} \right)
      &= (-1)^{n - 1} \left( D_{n - 1} - (n - 1) D_{n - 2} \right)
	\label{eq:identity-derangements}
  \end{align}
  Lo que dice~\eqref{eq:identity-derangements}
  es que la expresión indicada es independiente de \(n\).
  Sabemos que \(D_0 = 1\) y \(D_1 = 0\),
  evaluándola para \(n = 1\) obtenemos:
  \begin{equation}
    \label{eq:identity-derangements-value}
    (-1)^1 \left( D_1 - 1 \cdot D_0 \right)
      = 1
  \end{equation}
  Esto nos da la recurrencia:
  \begin{equation}
    \label{eq:recurrence-derangements-3}
    \index{recurrencia!desarreglo}
    D_n
      = n D_{n - 1} + (-1)^n
    \qquad D_0 = 1
  \end{equation}
  Esta es una recurrencia lineal de primer orden.
  Nuevamente el factor \(n\)
  sugiere una función generatriz exponencial.%
    \index{generatriz!exponencial}
  Multiplicando por \(z^n / n!\) y sumando:
  \begin{align*}
    \sum_{n \ge 1} D_n \, \frac{z^n}{n!}
      &= \sum_{n \ge 1} n D_{n - 1} \, \frac{z}{n!}
	   + \sum_{n \ge 1} (-1)^n \, \frac{z^n}{n!} \\
    \widehat{D}(z) - D_0
      &= z \sum_{n \ge 1} D_{n - 1} \, \frac{z^{n - 1}}{(n - 1)!}
	   + \mathrm{e}^{-z} - 1 \\
  \intertext{Como \(D_0 = 1\):}
    \widehat{D}(z)
      &= z \widehat{D}(z) + \mathrm{e}^{-z} \\
    \widehat{D}(z)
      &= \frac{\mathrm{e}^{-z}}{1 - z}
  \end{align*}
  Esta es nuevamente~\eqref{eq:EGF-desarreglos},
  y tenemos~\eqref{eq:n-derangements} una vez más.

  Queda de ejercicio intentar las anteriores
  con una función generatriz ordinaria.

\section{Resultados de competencias con empate}
\label{sec:campeonatos-empate}
\index{Bell, numeros de (ordenados)@Bell, números de (ordenados)|textbfhy}

  Interesa saber cuántos resultados finales
  pueden producirse en un campeonato entre \(n\) participantes
  si pueden producirse empates,
  vale decir pueden haber varios primeros puestos,
  y en general varios en cada puesto.
  No hay puestos vacantes,
  si hay participantes en el puesto \(j\),
  los hay en todos los puestos \(1 \le i \le j\).
  A estos números se les llama \emph{números de Bell ordenados},
  como los números de Bell
  vistos en la sección~\ref{sec:rotulados}%
    \index{Bell, numeros de@Bell, números de}
  cuentan el número total de particiones
  sin especificar el orden de las mismas,
  acá las estamos ordenando.
  Good~\cite{good75:_number_orders_cand_ties_permit}
  trata estos números en gran detalle.

  Llamemos \(R_n\) al número de posibilidades indicado.
  Claramente \(R_0 = 1\), \(R_1 = 1\), \(R_2 = 3\).

  Si hay \(k\) en primer lugar,
  habrán \(n - k\) que se distribuyen de la misma forma
  desde el segundo lugar,
  o sea,
  hay \(R_{n - k}\) distribuciones de los demás.
  Como a los \(k\) campeones los estamos eligiendo entre los \(n\),
  y pueden haber de \(1\) a \(n\) en primer lugar,
  resulta la recurrencia:%
    \index{Bell, numeros de (ordenados)@Bell, números de (ordenados)!recurrencia}
  \begin{equation}
    \label{eq:ranking-recurrence-1}
    R_n
      = \sum_{ 1 \le k \le n} \binom{n}{k} \, R_{n - k}
  \end{equation}
  Para \(n \ge 1\)
  podemos completar la suma en~\eqref{eq:ranking-recurrence-1}:
  \begin{equation*}
    2 R_n
      = \sum_{0 \le k \le n} \binom{n}{k} \, R_{n - k}
  \end{equation*}
  Para \(n = 0\) tenemos:
  \begin{equation*}
    \sum_{0 \le k \le 0} \binom{0}{k} \, R_{0 - k}
      = R_0
  \end{equation*}
  En vista de esto,
  como \(R_0 = 1\),
  la recurrencia completa,
  válida para \(n \ge 0\),
  es:
  \begin{equation}
    \label{eq:ranking-recurrence-2}
    2 R_n
      = [n = 0] + \sum_{0 \le k \le n} \binom{n}{k} \, R_{n - k}
  \end{equation}
  Como la sumatoria en~\eqref{eq:ranking-recurrence-2}
  es la convolución binomial
  de las secuencias \(\langle 1 \rangle_{n \ge 0}\)
  y \(\langle R_n \rangle_{n \ge 0}\),
  definimos la función generatriz exponencial:%
    \index{generatriz!exponencial}
  \begin{equation}
    \label{eq:ranking-gf-def}
    \widehat{R}(z)
       = \sum_{n \ge 0} R_n \, \frac{z^n}{n!}
  \end{equation}
  Con~\eqref{eq:ranking-recurrence-2} tenemos así:
  \begin{align}
    2 \widehat{R}(z)
      &= 1 + \widehat{R}(z) \mathrm{e}^z \notag \\
    \widehat{R}(z)
      &= \frac{1}{2 - \mathrm{e}^z}
	    \label{eq:ranking-gf}
  \end{align}
  Expandiendo en serie:
  \begin{equation*}
    \widehat{R}(z)
      = 1 + z + \frac{3}{2!} z^2 + \frac{13}{3!} z^3
	  + \frac{75}{4!} z^4 + \frac{541}{5!} z^5
	  + \frac{4\,683}{6!} z^6 + \frac{47\,293}{7!} z^7
	  + \frac{545\,835}{8!} z^8
	  + \dotsb
  \end{equation*}

  Alternativamente,
  por el método simbólico:%
    \index{metodo simbolico@método simbólico}
  \begin{equation*}
    \mathcal{R}
      = \Seq(\Set_{\ge 1}(\mathcal{Z}))
  \end{equation*}
  y directamente:
  \begin{equation*}
    \widehat{R}(z)
      = \frac{1}{2 - \mathrm{e}^z}
  \end{equation*}

  Podemos expandir~\eqref{eq:ranking-gf} como serie geométrica:
  \begin{align*}
    \widehat{R}(z)
      &= \frac{1}{2} \, \sum_{k \ge 0}
			  \frac{\mathrm{e}^{k z}}{2^k} \\
      &= \frac{1}{2} \, \sum_{k \ge 0}
			  \frac{1}{2^k} \,
			     \left(
			       \sum_{n \ge 0} \frac{(k z)^n}{n!}
			     \right) \\
      &= \sum_{n \ge 0}
	   \frac{z^n}{n!} \,
	   \sum_{k \ge 0}
	     \frac{k^n}{2^{k + 1}}
  \end{align*}
  De acá tenemos:
  \begin{equation}
    \label{eq:ranking-explicit}
    R_n
      = \sum_{k \ge 0} \frac{k^n}{2^{k + 1}}
  \end{equation}
  Difícilmente habríamos adivinado esta expresión
  vía calcular valores
  (con la idea de demostrarla luego por inducción).%
    \index{demostracion@demostración!induccion@inducción}

\section{Particiones de enteros}
\label{sec:funciones-generatrices:particiones}
\index{numero natural@número natural!particion@partición}

  En el siglo~XX
  tema de investigaciones importantes en teoría de números
  involucraban la teoría de las particiones de enteros,
  un área en la que Euler fue pionero.
  Acá veremos solo un par de resultados curiosos,
  que ya demostró Euler.

\subsection{Particiones en general}
\label{sec:particiones-generales}

  Sea \(p(n)\) el número de formas de escribir \(n\) como suma.
  Tenemos:
  \begin{align*}
    p(1)
      &= 1 \qquad 1 \\
    p(2)
      &= 2 \qquad 2 = 1 + 1 \\
    p(3)
      &= 3 \qquad 3 = 1 + 1 + 1 = 2 + 1 \\
    p(4)
      &= 4 \qquad 4 = 1 + 1 + 1 + 1 = 2 + 1 + 1 = 2 + 2 = 3 + 1 \\
    \vdots
  \end{align*}
  Tratamos esto mediante el método simbólico,
  una partición de \(n\)
  corresponde a un multiconjunto de \(\mathbb{N}\),
  o sea nos interesa \(\MSet(\mathbb{N})\).
  Cada número aparece una vez en \(\mathbb{N}\),
  así:
  \begin{equation}
    \label{eq:partition-gf}
    P(z)
      = \sum_{n \ge 1} p(n) z^n
      = \prod_{n \ge 1} \frac{1}{1 - z^n}
  \end{equation}

\subsection{Sumandos diferentes e impares}
\label{sec:particiones-diferentes}

  Poniendo como condición
  que los sumandos no se pueden repetir
  tenemos subconjuntos de \(\mathbb{N}\),
  interesa \(\Set(\mathbb{N})\).
  Anotamos \(p_d(n)\) para el número de estas particiones,
  por \emph{\foreignlanguage{english}{different}} en inglés,
  y distinguimos así la función generatriz también:
  \begin{align}
    P_d(z)
      &= \sum_{n \ge 0} p_d (n) z^n \notag \\
      &= \prod_{n \ge 1} (1 + z^n)
	    \label{eq:partition-different-gf}
  \end{align}
  Este producto podemos expresarlo de otra forma:
  \begin{align}
    P_d(z)
      &= \prod_{n \ge 1} \frac{1 - z^{2 n}}{1 - z^n} \notag \\
      &= \prod_{n \ge 0} \frac{1}{1 - z^{2 n + 1}}
	    \label{eq:partition-odd-gf}
  \end{align}
  Este producto corresponde a sumandos impares,
  si escribimos \(p_o(n)\)
  por el número de particiones en sumandos impares
  (de \emph{\foreignlanguage{english}{odd}} en inglés),
  tenemos:
  \begin{equation}
    \label{eq:partition-odd=different}
    p_d(n) = p_o(n)
  \end{equation}
  Curioso resultado~\eqref{eq:partition-odd=different},
  obtenido únicamente
  considerando las funciones generatrices del caso.
  Y ni siquiera evaluamos ninguna de~\eqref{eq:partition-gf},
  \eqref{eq:partition-different-gf} o~\eqref{eq:partition-odd-gf}.

\subsection{Un problema de Moser y Lambek}
\label{sec:Moser-Lambek}

  En 1959,
  Leo Moser y Joe Lambek plantearon~%
    \cite{novakovic07:_gener_func}:
  \begin{theorem}
    \label{theo:Moser-Lambek}
    Hay una única forma de particionar \(\mathbb{N}_0\)
    en conjuntos \(\mathcal{A}\) y \(\mathcal{B}\)
    tales que el número de maneras
    en que se puede representar \(n \in \mathbb{N}_0\)
    como sumas \(n = a_1 + a_2\)
    con \(a_1 \ne a_2\) y \(a_1, a_2 \in \mathcal{A}\)
    es igual al número de representaciones como
    \(n = b_1 + b_2\)
    con \(b_1 \ne b_2\) y \(b_1, b_2 \in \mathcal{B}\).
  \end{theorem}
  \begin{proof}
    Definamos funciones generatrices:%
      \index{generatriz}
    \begin{equation*}
      A(z)
	= \sum_{a \in \mathcal{A}} z^a \qquad
      B(z)
	= \sum_{b \in \mathcal{B}} z^b
    \end{equation*}
    Como \(\mathcal{A}\) y \(\mathcal{B}\)
    particionan \(\mathbb{N}_0\),
    los coeficientes son \(0\) o \(1\):
    \begin{equation*}
      A(z) + B(z)
	= \frac{1}{1 - z}
    \end{equation*}
    Para las maneras en que se puede escribir \(n\)
    como suma de dos \(\mathcal{A}\) distintos:
    \begin{equation*}
      \sum_{\substack{
	      a_1 \ne a_2 \\
	      a_1, a_2 \in A
	   }}
	     z^{a_1 + a_2}
	= \frac{1}{2} \, \left( A^2(z) - A(z^2) \right)
    \end{equation*}
    De la misma forma podemos expresar el caso de \(\mathcal{B}\),
    e interesa que sean las mismas:
    \begin{align*}
      A^2(z) - A(z^2)
	&= B^2(z) - B(z^2) \\
      A(z^2) - B(z^2)
	&= A^2(z) - B^2(z) \\
	&= (A(z) - B(z)) \cdot (A(z) + B(z)) \\
	&= \frac{A(z) - B(z)}{1 - z}
    \end{align*}
    O equivalentemente:
    \begin{equation*}
      A(z) - B(z)
	= \left( A(z^2) - B(z^2) \right) (1 - z)
    \end{equation*}
    Substituyendo
      \(z \rightsquigarrow z^2, z^4, \dotsc, z^{2^{n - 1}}\)
    obtenemos:
    \begin{equation}
      \label{eq:Moser-Lambek-recurrence}
      A(z) - B(z)
	= \left( A(z^{2^n}) - B(z^{2^n}) \right)
	    \prod_{0 \le k \le n - 1} \left( 1 - z^{2^k} \right)
    \end{equation}
    lo que indica:
    \begin{equation}
      \label{eq:Moser-Lambek}
      A(z) - B(z)
	= \prod_{k \ge 0} \left( 1 - z^{2^k} \right)
    \end{equation}
    Como series formales
    (ver la sección~\ref{sec:series-secuencias})%
      \index{serie formal!convergencia}
    la secuencia
    al lado derecho de~\eqref{eq:Moser-Lambek-recurrence}
    converge a la expresión~\eqref{eq:Moser-Lambek}.

    Los términos del lado derecho de~\eqref{eq:Moser-Lambek}
    tienen coeficientes \(\pm 1\),
    con lo que determinan en forma única
    los coeficientes de \(A(z)\) y \(B(z)\),
    que son cero o uno.
    El conjunto \(\mathcal{A}\)
    son los que tienen un número par de unos
    en su expansión binaria.
  \end{proof}

\section{Contando secuencias}
\label{sec:ejemplos-secuencias}
\index{secuencia}

  Muchas situaciones
  llevan a enfrentar problemas de contar secuencias
  con ciertas restricciones.
  Veremos algunos ejemplos representativos.

  ¿Cuántas palabras de largo \(n\)
  formadas únicamente por las \(5\) vocales
  pueden formarse,
  si deben contener un número par de vocales fuertes
  (`a', `e' y `o')?

  De las solicitadas podemos formar \(1\) de largo \(0\),
  \(2\) de largo \(1\),
  \(2 \cdot 2 + 3 \cdot 3 = 13\) de largo \(2\).
  Estos valores sirven para verificar luego.

  Definamos las clases \(\mathcal{U}\)
  para palabras con un número par
  de vocales fuertes,
  y \(\mathcal{V}\) si tienen un número impar,
  con las respectivas funciones generatrices \(U(z)\) y \(V(z)\)
  donde \(z\) cuenta el número de vocales.
  Podemos definir el sistema de ecuaciones simbólicas:
  \begin{align*}
    \mathcal{U}
      &= \mathcal{E}
	    + \{i, u\} \times \mathcal{U}
	    + \{a, e, o\} \times \mathcal{V} \\
    \mathcal{V}
      &= \{a, e, o\} \times \mathcal{U}
	    + \{i, u\} \times \mathcal{V}
  \end{align*}
  El método simbólico entrega:%
    \index{metodo simbolico@método simbólico}
  \begin{align*}
    U(z)
      &= 1 + 2 z U(z) + 3 z V(z) \\
    V(z)
      &= 3 z U(z) + 2 z V(z)
  \end{align*}
  Resolvemos el sistema de ecuaciones para \(U(z)\),
  que es lo único que realmente interesa,
  y descomponemos en fracciones parciales:
  \begin{equation*}
    U(z)
      = \frac{1}{2} \cdot \frac{1}{1 - 5 z}
	  + \frac{1}{2} \cdot \frac{1}{1 + z}
  \end{equation*}
  Podemos leer los \(u_n\) de esto último,
  que son simplemente dos series geométricas:
  \begin{equation*}
    u_n = \frac{1}{2} \,\left( 5^n + (-1)^n \right)
  \end{equation*}
  Esto coincide con los valores obtenidos antes.

  Volvamos nuevamente
  a la situación de las secuencias de unos y ceros
  sin ceros seguidos,
  pero ahora interesa contar no solo el largo
  sino simultáneamente el número de unos.
  Para la clase \(\mathcal{S}\) de tales secuencias obtuvimos:
  \begin{equation*}
    \mathcal{S}
      = \mathcal{E} + \{0\} + \mathcal{S} \times \{1, 10\}
  \end{equation*}
  Si usamos \(x\) para marcar los ceros,
  e \(y\) para el número total de símbolos,
  obtenemos para la función generatriz \(S(x, y)\):%
    \index{generatriz!multivariada}
  \begin{equation*}
    S(x, y)
      = 1 + y + (x y + x y^2) S(x, y)
  \end{equation*}
  Despejando:
  \begin{equation*}
    S(x, y)
      = \frac{1 + y}{1 - x (y + y^2)}
  \end{equation*}
  Podemos expandir como serie geométrica en \(x (y + y^2)\):
  \begin{equation*}
    S(x, y)
      = \sum_{r \ge 0} x^r y^r (1 + y)^{r + 1}
  \end{equation*}
  El número de estas secuencias de largo \(n\) con \(k\) unos es:
  \begin{align*}
    \left[ x^k y^n \right] \sum_{r \ge 0} x^r y^r (1 + y)^{r + 1}
      &= \left[ y^n \right] y^k (1 + y)^{k + 1} \\
      &= \left[ y^{n - k} \right] (1 + y)^{k + 1} \\
      &= \binom{k + 1}{n - k}
  \end{align*}
  Pero \(n - k\) es el número de ceros.
  Podemos interpretar esto
  como diciendo que debemos distribuir \(n - k\) ceros
  en las \(k + 1\) posiciones separadas por los \(k\) unos.

%%% Local Variables:
%%% mode: latex
%%% TeX-master: "clases"
%%% End:

% anillos-polinomios.tex
%
% Copyright (c) 2009-2014 Horst H. von Brand
% Derechos reservados. Vea COPYRIGHT para detalles

\chapter{Anillos de polinomios}
\label{cha:anillos-polinomios}
\index{anillo!polinomios|textbfhy}

  Los polinomios se cuentan entre las funciones más importantes,
  dada su simplicidad.
  Estudiarlos desde un punto de vista algebraico,
  tanto para determinar propiedades de sus ceros
  y su factorización
  como desde el punto de vista más abstracto como ejemplo de anillo,
  es fructífero.
  En particular,
  el estudio de anillos de polinomios lleva naturalmente
  a anillos de series formales,
  que nos ocuparán intensamente más adelante.

\section{Algunas herramientas}
\label{sec:algunas-herramientas-poly}

% Fixme: Agregar ejemplos/ejercicios para aterrizar
% (Iván Lazo <ilazo@alumnos.inf.utfsm.cl> 20130813)

  Un paquete de álgebra simbólica,%
    \index{algebra simbolica@álgebra simbólica}
  como \texttt{maxima}~\cite{maxima14b:_computer_algebra},%
    \index{maxima@\texttt{maxima}}
  ayuda bastante con la operatoria.
  El paquete \texttt{PARI/GP}~\cite{PARI:2.7.2}%
    \index{PARI/GP@\texttt{PARI/GP}}
  incluye extenso soporte para trabajar con polinomios.
  La biblioteca GiNaC~%
    \cite{bauer02:_ginac_fram_symbol_comput, GiNaC:1.6.2}%
    \index{GiNaC@\texttt{GiNaC}}
  permite manipular expresiones simbólicas y numéricas
  directamente en \cplusplus~%
    \cite{ISO11:_C++, stroustrup00:_C++_progr_languag}.%
    \index{C++ (lenguaje de programacion)@\cplusplus{} (lenguaje de programación)}
  \begin{definition}
    \index{polinomio|textbfhy}
    \label{def:ring-polynomials}
    Sea \((R, +, \cdot)\) un anillo,
    y \(x\) un \emph{símbolo}
    (también llamado \emph{indeterminada} o \emph{variable}).
    Definimos \(R[x]\),
    los \emph{polinomios sobre \(R\)},
    como el conjunto de las expresiones:
    \begin{equation*}
      f(x)
	= a_n \cdot x^n + a_{n - 1} \cdot x^{n - 1} + \dotsb + a_0
    \end{equation*}
    El \emph{grado} de \(f\) es la máxima potencia de \(x\)
      \index{polinomio!grado|textbfhy}
    que aparece multiplicada por un coeficiente no cero,
    se anota \(\deg(f)\).
    Al polinomio con todos los coeficientes cero
    (el \emph{polinomio cero})
      \index{polinomio!cero|textbfhy}
    se le asigna el grado \(-\infty\).
    Si solo el término constante (\(a_0\)) es diferente de cero,
    el grado del polinomio es cero,
    y se dice que es un \emph{polinomio constante}.%
      \index{polinomio!constante|textbfhy}
    A polinomios de grado \(1\) se les llama \emph{lineales},%
      \index{polinomio!lineal|textbfhy}
    a los de grado \(2\) \emph{cuadráticos}%
      \index{polinomio!cuadratico@cuadrático|textbfhy}
    y a los de grado \(3\) \emph{cúbicos}.%
      \index{polinomio!cubico@cúbico|textbfhy}
    Se habla del \emph{coeficiente principal}%
      \index{polinomio!coeficiente principal|textbfhy}
    para referirse al coeficiente
    de la máxima potencia de \(x\) en el polinomio.
    Si el coeficiente principal es \(1\),
    el polinomio se llama \emph{mónico}.%
      \index{polinomio!monico@mónico|textbfhy}
  \end{definition}
  Nótese que algunos autores
  simplemente no le asignan grado al polinomio cero.

  No asignamos significado a \(x\) ni a sus potencias.
  Podemos desarrollar toda la teoría
  hablando únicamente de tuplas de coeficientes.
  La notación es sugestiva,
  y más adelante sí consideraremos
  los polinomios como definiendo funciones.

  \begin{definition}
    \index{operacion@operación!polinomios}
    Para polinomios \(f, g \in R[x]\),
    definimos la suma y producto entre ellos
    (bajo el supuesto
     que es una secuencia infinita de coeficientes \(0\)
     a partir de un cierto punto para simplificar)
    como si tratáramos con expresiones en \(R\),
    solo que \(x^k\) conmuta con los elementos de \(R\)
    y se cumple \(x^i x^j = x^{i + j}\):
    \begin{align}
      f(x)
	 &= a_0 + a_1 \cdot x + a_2 \cdot x^2 + \dotsb
	   \label{eq:poly-f} \\
      g(x)
	 &= b_0 + b_1 \cdot x + b_2 \cdot x^2 + \dotsb
	   \label{eq:poly-g}
    \end{align}
    \begin{align}
      f(x) + g(x)
	 &= (a_0 + b_0)
	      + (a_1 + b_1) \cdot x
	      + \dotsb
	      + (a_k + b_k) \cdot x^k
	      + \dotsb \notag \\
	 &= \sum_{k  \ge 0} (a_ k + b_k) x^k \label{eq:poly-f+g}
    \end{align}
    \begin{align}
      f(x) \cdot g(x)
	 &= a_0 \cdot b_0
	      + (a_1 b_0 + a_0 b_1) \cdot x
	      + (a_2 b_0 + a_1 b_1 + a_0 b_2) \cdot x^2
	      + \dotsb \notag \\
	 &= \sum_{\substack{0 \le i \le m \\
			    0 \le j \le n
		 }} a_i b_j x^{i + j} \notag \\
	 &= \sum_{k \ge 0}
	      \left(
		\sum_{0 \le i \le k} a_{k - i} b_i
	      \right) \cdot x^k \label{eq:poly-f*g}
    \end{align}
  \end{definition}
  Resulta que \(R[x]\) con las operaciones
  definidas por~\eqref{eq:poly-f+g} y~\eqref{eq:poly-f*g}
  es un anillo.%
    \index{anillo}
  Es cómodo considerar \(\alpha \in R\)
  como el polinomio constante \(\alpha \in R[x]\).
  Las unidades de \(R[x]\) son los polinomios constantes
  \(\alpha \in R^\times\).
  Es fácil ver que:
  \begin{align}
    \deg(f + g)
      &\le \max\{\deg(f), \deg(g)\} \label{eq:poly-deg-f+g} \\
    \deg(f \cdot g)
      &\le \deg(f) + \deg(g)	    \label{eq:poly-deg-f*g}
  \end{align}
  Por esto resulta útil definir el grado del polinomio \(0\)
  como \(-\infty\),
  evita requerir casos especiales.

  En caso que no hayan divisores de cero
  diferentes de cero en \(R\),
  en~\eqref{eq:poly-deg-f*g} es igualdad.
  Si \(R\) es un dominio integral
    \index{dominio integral}
  (un anillo conmutativo sin divisores de cero distintos de cero),
  \(R[x]\) también es un dominio integral
  (el coeficiente del término de mayor grado
   del producto \(f(x) \cdot g(x)\)
   no es cero si ambos polinomios son diferentes de cero).

  Definimos la \emph{derivada formal}%
    \index{polinomio!derivada formal}
  de un polinomio mediante:
  \begin{align*}
    f(x)
      &= \sum_{0 \le k \le n} a_k x^k \\
    \mathrm{D} f(x)
      &= \sum_{0 \le k \le n - 1} (k + 1 ) a_{k + 1} x^k
  \end{align*}
  Anotaremos alternativamente:
  \begin{equation*}
    f'(x)
      = \mathrm{D} f(x)
  \end{equation*}
  Es fácil verificar que se cumplen las propiedades conocidas
  de las derivadas:
  \begin{theorem}
    \label{theo:polynomial-derivative}
    Sean \(f(x)\) y \(g(x)\) polinomios
    sobre el dominio integral \(R\),
    \(\alpha\) y \(\beta\) elementos de \(R\).
    Entonces:
    \begin{align*}
      \mathrm{D} (\alpha f(x) + \beta g(x))
	&= \alpha f'(x) + \beta g'(x) \\
      \mathrm{D} (f(x) \cdot g(x))
	&= f'(x) g(x) + f(x) g'(x) \\
      \mathrm{D} (f(x)^m)
	&= m f(x)^{m - 1} f'(x)
    \end{align*}
  \end{theorem}
  \begin{proof}
    Definamos:
    \begin{align*}
      f(x)
	&= \sum_{0 \le k \le m} f_k x^k \\
      g(x)
	&= \sum_{0 \le k \le n} g_k x^k
    \end{align*}
    Por comodidad,
    anotaremos sumas infinitas
    bajo el entendido que los términos son todos cero
    desde un índice en adelante.

    Para la primera parte,
    tenemos que:
    \begin{align*}
      \alpha f(x) + \beta g(x)
	&= \sum_{k \ge 0} (\alpha f_k + \beta g_k) x^k \\
      \mathrm{D} (\alpha f(x) + \beta g(x))
	&= \sum_{k \ge 0} (k + 1)
	     (\alpha f_{k + 1} + \beta g_{k + 1}) x^k \\
	&= \alpha \sum_{k \ge 0} (k + 1) f_{k + 1} x^k
	     + \beta \sum_{k \ge 0} (k + 1) g_{k + 1} x^k \\
	&= \alpha f'(x) + \beta g'(x)
    \end{align*}
    Acá usamos el que para \(k \in \mathbb{N}\) y \(a, b \in R\):
    \begin{align*}
      k (a b)
	&= a b + a b + \dotsb + a b \\
	&= a (b + b + \dotsb + b) \\
	&= a (k b)
    \end{align*}

    Para la segunda parte:
    \begin{align*}
      (\mathrm{D} f(x)) g(x) &+ f(x) (\mathrm{D} g'(x)) \\
	&= \sum_{k \ge 0}
	     \left(
	       \sum_{0 \le j \le k} (j + 1) f_{j + 1} g_{k - j}
		  + \sum_{0 \le j \le k}
		      (k + 1 - j) f_j g_{k + 1 - j}
	     \right) x^k \\
	&= \sum_{k \ge 0}
	     \left(
	       \sum_{0 \le j \le k + 1} j f_j g_{k + 1 - j}
		  + \sum_{0 \le j \le k + 1}
		      (k + 1 - j) f_j g_{k + 1 - j}
	     \right) x^k \\
	&= \sum_{k \ge 0}
	      (k + 1)
	      \left(
		\sum_{0 \le j \le k + 1} f_j g_{k + 1 - j}
	      \right) x^k \\
	&= \mathrm{D} (f(x) g(x))
    \end{align*}

    Para la tercera parte,
    usamos inducción sobre \(m\).%
      \index{demostracion@demostración!induccion@inducción}
    \begin{description}
    \item[Base:]
      Cuando \(m = 1\)
      lo aseverado ciertamente se cumple.
    \item[Inducción:]
      Suponiendo que vale para \(m\),
      demostramos que vale para \(m + 1\):
      \begin{align*}
	\mathrm{D} (f(x)^{m + 1})
	  &= \mathrm{D} (f(x)^m f(x)) \\
	  &= m f(x)^{m - 1} f'(x) f(x) + f(x)^m f'(x) \\
	  &= (m + 1) f(x)^m f'(x)
      \end{align*}
      Acá usamos la conmutatividad de \(R[x]\).
    \end{description}
    Por inducción,
    vale para todo \(m \in \mathbb{N}\).
  \end{proof}

  Los anillos de polinomios tienen varias propiedades interesantes,
  por ejemplo un algoritmo de división afín al de los enteros:%
    \index{algoritmo de division@algoritmo de división!polinomios}
  \begin{theorem}
    \label{theo:F[x]:division-algorithm}
    Sean \(a(x)\), \(b(x)\) polinomios sobre un campo \(F\),
    con \(b(x) \ne 0\).
    Entonces existen polinomios únicos \(q(x)\), \(r(x)\)
    tales que:
    \begin{equation*}
      a(x) = b(x) \cdot q(x) + r(x)
    \end{equation*}
    con \(\deg(r) < \deg(b)\)
  \end{theorem}
  \begin{proof}
    Consideremos el conjunto:
    \begin{equation*}
      \mathcal{R}
	= \{a(x) - c(x) \cdot b(x) \colon c(x) \in F[x]\}
    \end{equation*}
    Elijamos un elemento \(r\) de \(\mathcal{R}\) de grado mínimo.
    Entonces \(\deg(r) < \deg(b)\),
    ya que en caso contrario
    podríamos restar un múltiplo de \(b(x)\)
    que anule el término de grado mayor en \(r(x)\)
    y así obtener uno de grado menor.

    Demostramos que son únicos por contradicción.%
      \index{demostracion@demostración!contradiccion@contradicción}
    Supongamos que hay dos pares diferentes,
    o sea:
    \begin{equation*}
      a = b q' + r' \qquad
      a = b q'' + r''
    \end{equation*}
    Sin pérdida de generalidad
    podemos suponer que \(\deg(r') \le \deg(r'')\).
    Como \(F[x]\) es un anillo,
    con \(F\) un campo:
    \begin{align}
      r'' - r'
	&= b (q' - q'') \notag \\
      \deg(r'' - r')
	&= \deg(b (q' - q'')) \notag \\
	&= \deg(b) + \deg(q' - q'') \label{eq:deg(b(q2-q1))} \\
    \intertext{Pero:}
      \deg(r')
	&\le \deg(r'') < \deg(b) \notag \\
      \deg(r'' - r')
	&\le \deg(r'') < \deg(b) \label{eq:deg(r2-r1)}
    \end{align}
    En vista de~\eqref{eq:deg(r2-r1)}
    la única posibilidad en~\eqref{eq:deg(b(q2-q1))}
    es \(\deg(r'' - r') = \deg(q' - q'') = -\infty\),
    vale decir,
    \(q' = q''\) y \(r' = r''\).
    Esto contradice nuestra elección de dos pares diferentes.
  \end{proof}
  Vale la pena comparar esta demostración
  con la del algoritmo de división entre enteros,
  teorema~\ref{theo:division}.

% dominios-euclidianos.tex
%
% Copyright (c) 2012-2014 Horst H. von Brand
% Derechos reservados. Vea COPYRIGHT para detalles

\section{Dominios euclidianos}
\label{sec:dominios-euclidianos}

  A un dominio integral \(D\)
  equipado con una \emph{función euclidiana}
  (a veces llamada \emph{función grado} o simplemente \emph{grado})
  \(g \colon D \smallsetminus \{0\} \rightarrow \mathbb{N}\)
  tal que si \(a, b \in D\),
  con \(b \ne 0\),
  hay \(q, r \in D\)
  tales que \(a = q b + r\)
  con \(r = 0\) o \(g(r) < g(b)\)
  se le llama \emph{dominio euclidiano}.
  Estas estructuras tienen mucho en común con \(\mathbb{Z}\)
  (en particular,
   toman su nombre porque es aplicable el algoritmo de Euclides
   para calcular máximo común divisor,
   y tenemos el equivalente
   de la identidad de \foreignlanguage{french}{Bézout}).
  En el caso de los polinomios,
  el grado sirve como función euclidiana.

  Tenemos algunos resultados simples:
  \begin{theorem}
    \label{theo:ED:g-minimo=>unidad}
    Sea \(D\) un dominio euclidiano con función euclidiana \(g\).
    El valor \(g(a)\) es mínimo si \(a\) es una unidad.
  \end{theorem}
  \begin{proof}
    Tomemos \(a \ne 0\) en \(D\) tal que \(g(a)\) es mínimo.
    Por el algoritmo de división,
    teorema~\ref{theo:F[x]:division-algorithm},
    tenemos \(1 = q a + r\) con \(r = 0\) o \(g(r) < g(a)\).
    Pero \(g(a)\) es mínimo,
    por lo que debe ser \(r = 0\)
    y \(a\) es una unidad.
  \end{proof}
  También tenemos las propiedades
  (ver Rogers~%
    \cite{rogers71:_axiom_euclidean_domain}
   y Samuel~%
    \cite{samuel71:_about_euclidean_rings}):
  \begin{theorem}
    \label{theo:ED:propiedades-f}
    Sea \(D\) un dominio euclidiano
    con función euclidiana \(g\).
    La función definida por:
    \begin{equation*}
      f(a)
	= \min_{x \in D \smallsetminus \{0\}} g(a x)
    \end{equation*}
    es una función euclidiana,
    y cumple:
    \begin{enumerate}[label=(\alph{*})]
    \item
      \label{en:fe:a}
      \(f(a) \le f(a b)\) si \(a b \ne 0\)
    \item
      \label{en:fe:b}
      \(f(a) \le g(a)\) para todo \(a \in D \smallsetminus \{0\}\)
    \item
      \label{en:fe:c}
      \(f(a u) = f(a)\) si y solo si \(u \in D^\times\)
    \end{enumerate}
  \end{theorem}
  \begin{proof}
    Por la definición de \(f\)
    los puntos~\ref{en:fe:a} y~\ref{en:fe:b} son obvios.

    Para demostrar que \(f\) es euclidiana,
    consideremos elementos \(a, b\)
    cualquiera en \(D \smallsetminus \{0\}\).
    Debemos demostrar que si \(b = q a + r\)
    entonces \(r = 0\) o \(f(r) < f(a)\).

    El caso \(r = 0\) es trivial.
    Supongamos entonces \(r \ne 0\).
    Por definición
    es \(f(a) = g(a c)\)
    para algún \(c \in D \smallsetminus \{0\}\).
    De la definición de \(r\) tenemos que \(g(r) < g(a)\)
    por ser \(g\) euclidiana.
    De \(b c = q a c + r c\),
    por ser \(g\) euclidiana es \(g(r c) < g(a c) = f(a)\);
    y por la definición de \(f\) es también \(f(r) \le g(r c)\).
    Uniendo las anteriores queda \(f(r) < f(a)\),
    y \(f\) es euclidiana.

    Para~\ref{en:fe:c} demostramos implicancia en ambas direcciones.
    Primero,
    sea \(u \in D^\times\).
    Por el punto~\ref{en:fe:a}
    es \(f(a) \le f(a u) \le f((a u) u^{-1}) = f(a)\).
    Por otro lado,
    si \(f(a c) = f(a)\),
    escribimos \(a = q a c + r\) con \(f(r) < f(a c) = f(a)\);
    siendo \(r = a (1 - c q)\),
    por la parte~\ref{en:fe:a}
    si \(r \ne 0\) es \(f(r) \ge f(a)\),
    lo que es absurdo.
    Así \(r = 0\) y \(c\) es una unidad.
  \end{proof}
  Supondremos una función euclidiana
  tal que \(f(a) \le f(a b)\) para todo \(a, b \in D\)
  desde ahora,
  ya que simplifica mucha de la discusión que sigue.
  Nótese que en particular el grado de polinomios cumple esto.
  Como la función euclidiana no es única,
  no la incluimos en la definición del dominio.
  \begin{definition}
    Sea \(R\) un dominio integral.
    Si podemos escribir \(m = b c\),
    decimos que \(b\) \emph{divide a} \(m\),
    y anotamos \(b \mid m\).
  \end{definition}
  Esto también se expresa
  diciendo que \(b\) es un \emph{factor} de \(m\),
  o que \(m\) es un \emph{múltiplo} de \(b\).
  \begin{definition}
    Sea \(R\) un dominio integral.
    Dos elementos \(a, b \in R\) se dicen \emph{asociados}
    si \(a = u b\),
    donde \(u\) es una unidad.
    Se anota \(a \sim b\).
  \end{definition}
  Es fácil ver que \(\sim\) es una relación de equivalencia.
  \begin{definition}
    Sea \(R\) un dominio integral.
    Un elemento \(e \in R \smallsetminus R^\times\)
    se llama \emph{irreductible} si siempre que
    \(e = u \cdot v\),
    \(u\) o \(v\) es una unidad.
    En caso contrario,
    decimos que \(e\) es \emph{reductible}.
  \end{definition}
  \begin{definition}
    Sea \(R\) un dominio integral,
    \(p \in R \smallsetminus R^\times\).
    Si \(p \mid a b\) implica que \(p \mid a\) o \(p \mid b\),
    se dice que \(p\) es \emph{primo}.
  \end{definition}
  Vemos que si un elemento es primo,
  es irreductible:
  \begin{lemma}
    \label{lem:prime=>irreducible}
    Sea \(R\) un dominio integral.
    Si \(p \in R\) es primo,
    entonces es irreductible.
  \end{lemma}
  \begin{proof}
    Por contradicción.
    Supongamos \(p\) primo pero no irreductible.
    Así podemos escribir \(p = u v\),
    donde \(u, v \notin R^\times\).
    Pero entonces \(p \mid u v\),
    y por la definición de primo
    es \(p \mid u\) o \(p \mid v\).
    Sin pérdida de generalidad
    podemos suponer \(u = a p\),
    con lo que:
    \begin{align*}
      p
	&= a p v \\
      0
	&= p (1 - a v)
    \end{align*}
    Como no hay divisores de cero distintos de cero en \(R\),
    debe ser \(a v = 1\)
    y \(v\) es una unidad,
    lo que contradice su elección.
  \end{proof}
  El recíproco del lema~\ref{lem:prime=>irreducible}
  no siempre se cumple.
  Consideremos el dominio integral
  \(\mathbb{Z}[\sqrt{-5}]\)
  (ver la sección~\ref{sec:anillos-cuadraticos},
   solo que este es un subanillo de \(\mathbb{C}\)).
  Si \(3 = u \cdot v\),
  debe ser \(N(u) \cdot N(v) = N(3) = 9\),
  con lo que las normas posibles para \(u\) y \(v\)
  son los divisores de 9.
  Si \(N(u) = 1\),
  \(u\) es una unidad.
  Si \(N(u) = 3\),
  con \(u = u_1 + u_2 \sqrt{-5}\)
  es \(u_1^2 + 5 u_2^2 = 3\).
  Esto claramente es imposible con \(u_1\), \(u_2\) enteros.
  Así \(3\) es irreductible en \(\mathbb{Z}[\sqrt{-5}]\).
  Por el otro lado:
  \begin{align*}
    &(2 + \sqrt{-5}) \cdot (2 - \sqrt{-5})
      = 9 \\
  \intertext{con lo que}
    &3
      \mid (2 + \sqrt{-5}) \cdot (2 - \sqrt{-5})
  \end{align*}
  Claramente \(3 \centernot\mid 2 \pm \sqrt{-5}\),
  o sea 3 no es primo en \(\mathbb{Z}[\sqrt{-5}]\).

  \begin{definition}
    Sea \(R\) un dominio integral,
    y sea \(a \in R\) con \(a \ne 0\).
    Entonces se dice
    que \emph{\(a\) tiene factorización única en irreductibles}
    si hay una unidad \(u\)
    e irreductibles \(p_i\) tales que \(a = u p_1 p_2 \dotsm p_r\),
    y además,
    si \(a = v q_1 q_2 \dotsm q_s\) para una unidad \(v\)
    e irreductibles \(q_i\),
    entonces \(r = s\)
    y \(p_i = u_i q_i\)
    para unidades \(u_i\) salvo reordenamiento.
  \end{definition}
  \begin{definition}
    Se dice que \(R\) es un \emph{dominio de factorización única}
    (en inglés
     \emph{\foreignlanguage{english}{Unique Factorization Domain}}.
     abreviado \emph{UFD})
    si todo elemento de \(R\)
    tiene factorización única en irreductibles.
  \end{definition}

  En vista del algoritmo de división en el dominio euclidiano,
  tenemos:
  \begin{theorem}
    \label{theo:ED=>PID}
    Sea \(D\) un dominio euclidiano con función euclidiana \(f\)
    y \(a, b \in D\).
    Entonces el conjunto \(I = \{u a + v b \colon u, v \in D\}\)
    consta de todos los múltiplos de un elemento \(m\).
  \end{theorem}
  La demostración es muy similar
  a la discusión sobre máximo común divisor
  en el capítulo~\ref{cha:teoria-numeros}.
  \begin{proof}
    Si \(a = b = 0\),
    claramente \(I = \{0\}\),
    y lo aseverado se cumple.
    Supongamos entonces que al menos uno
    de \(a\), \(b\) es diferente de 0,
    en cuyo caso \(I\) contiene elementos diferentes de 0.
    Elijamos uno de ellos con \(f\) mínimo,
    llamémosle \(m\).
    Tomemos ahora \(n \in I\) cualquiera.
    Si \(n = 0\),
    se cumple \(m \mid n\),
    y estamos listos.
    Si \(n \ne 0\),
    podemos aplicar el algoritmo de división y escribir:
    \begin{equation*}
      n = q m + r
    \end{equation*}
    donde \(r = 0\) o \(f(r) < f(m)\).
    Dado que hay \(u, v, u', v'\) tales que \(n = u a + v b\)
    y \(m = u' a + v' b\) resulta:
    \begin{align*}
      r
	&= n - q m \\
	&= (u - q u') a + (v - q v') b
    \end{align*}
    con lo que \(r \in I\).
    Pero no puede ser \(f(r) < f(m)\),
    hemos elegido \(m\) precisamente por ser \(f(m)\) mínimo.
    En consecuencia,
    \(r = 0\) y \(m \mid n\).
  \end{proof}
  Conjuntos como \(I\)
  que aparece en la demostración del teorema~\ref{theo:ED=>PID}
  son muy importantes.
  Podemos definir \(m\)
  (o uno de sus asociados,
   que también son parte de \(I\);
   \(m\) no necesariamente es único)
  como un máximo común divisor de \(a\) y \(b\)
  (``máximo'' en el sentido de la función euclidiana \(f\)).
  \begin{definition}
    Sea \(R\) un anillo conmutativo.
    Un \emph{ideal} de \(R\)
    es un conjunto \(I \subseteq R\)
    tal que:
    \begin{enumerate}
    \item
      \((I, +)\) es un subgrupo de \((R, +)\)
    \item
      Para todo \(x \in I\)
      y para todo \(r \in R\)
      se cumple \(r \cdot x \in I\)
    \end{enumerate}
  \end{definition}
  Los ideales son casi subanillos de \(R\)
  (solo falta el elemento \(1\)).
  Hay quienes definen anillos sin \(1\),
  para ellos los ideales son subanillos.
  \begin{definition}
    Sea \(R\) un anillo conmutativo,
    y \(\{x_1, x_2, \dotsc, x_n\} \subseteq R\).
    Al ideal
      \(\{\sum_{1 \le k \le n} u_k x_k \colon u_k \in R\}\)
    se le llama
    el \emph{ideal generado por \(\{x_1, x_2, \dotsc, x_n\}\)},
    que se suele anotar \((x_1, x_2, \dotsc, x_n)\).
    Por la convención que sumas vacías son cero,
    \(\{0\}\) es generado por \(\varnothing\).
    A un ideal generado por un único elemento \(x_1\),
    anotado \((x_1)\),
    se le llama \emph{ideal principal}.
    Si en \(R\) todos los ideales son principales,
    se dice que \(R\) es un \emph{dominio de ideal principal}
    (en inglés
     \emph{\foreignlanguage{english}{Principal Ideal Domain}},
     abreviado \emph{PID}).
  \end{definition}
  En estos términos,
  el teorema~\ref{theo:ED=>PID} asevera que todo dominio euclidiano
  es un dominio de ideal principal.

  \begin{lemma}
    \label{lem:ED:irreductible-coprime}
    Sea \(p\) irreductible en un dominio euclidiano \(D\),
    y \(a\) otro elemento de \(D\).
    Si \(p\) no divide a \(a\),
    entonces \(1\) es un máximo común divisor entre \(a\) y \(p\).
  \end{lemma}
  \begin{proof}
    Sea \(m\) un máximo común divisor de \(a\) y \(p\).
    Por el teorema~\ref{theo:ED=>PID}
    existen \(x, y \in D\) tales que:
    \begin{equation*}
      m = x a + y p
    \end{equation*}
    Como \(m\) divide a \(p\),
    que es irreductible,
    \(m\) es una unidad o \(m \sim p\).
    Si \(m\) es una unidad,
    \(1\) es un máximo común divisor de \(a\) y \(p\)
    y estamos listos.
    En el otro caso,
    por ser \(m\) divisor de \(a\)
    es \(a = c m\) para algún \(c \in D\)
    y como a su vez \(m = u p\) para una unidad \(u\),
    entonces \(a = c u p\) y \(p \mid a\).
  \end{proof}
  Esto nos permite demostrar
  el recíproco del lema~\ref{lem:prime=>irreducible}
  en dominios euclidianos,
  como ya lo hicimos en el teorema~\ref{theo:Z:irreductible=>prime}
  para los enteros:
  \begin{theorem}
    \label{theo:ED:irreducible=>prime}
    En un dominio euclidiano,
    si \(p\) es irreductible
    entonces \(p\) es primo.
  \end{theorem}
  \begin{proof}
    Supongamos que el irreductible \(p\) divide a \(a b\).
    Debemos demostrar que \(p\) divide a \(a\) o a \(b\)
    (o a ambos).
    Si \(p \mid a\),
    estamos listos.
    En caso contrario,
    como \(p \mid a b\),
    hay un \(c \in D\) tal que \(a b = c p\).
    Por el lema~\ref{lem:ED:irreductible-coprime}
    tenemos que \(1\) es un máximo común divisor de \(a\) y \(p\),
    y por el teorema~\ref{theo:ED=>PID} podemos escribir:
    \begin{align*}
      1
	&= u p + v a \\
      b
	&= b u p + v a b \\
	&= (b u + v c) p
    \end{align*}
    con lo que \(p \mid b\).
  \end{proof}
  \begin{lemma}
    \label{lem:ED:primo-divide-producto}
    Si el primo \(p\) divide al producto \(x_1 x_2 \dotsm x_n\),
    entonces \(p \mid x_i\) para algún \(i\).
  \end{lemma}
  \begin{proof}
    Por inducción sobre \(n\).
    Si \(n = 1\),
    no hay nada que demostrar.
    \begin{description}
    \item[Base:]
      Para \(n = 2\),
      por la definición de primo tenemos que si \(p \mid x_1 x_2\),
      entonces \(p \mid x_1\) o \(p \mid x_2\).
    \item[Inducción:]
      Por la hipótesis de inducción,
      si \(p \mid x_1 x_2 \dotsm x_n\)
      entonces \(p \mid x_i\) para \(1 \le i \le n\).
      Si ahora \(p \mid x_1 x_2 \dotsm x_n x_{n + 1}\)
      por el caso \(n = 2\) significa que ya sea
      \(p \mid x_1 x_2 \dotsm x_n\)
      (lo que implica \(p \mid x_i\) para \(1 \le i \le n\))
      o \(p \mid x_{n + 1}\).
      En conjunto,
      \(p \mid x_i\) para \(1 \le i \le n + 1\).
    \end{description}
    Por inducción,
    vale para todo \(n \in \mathbb{N}\).
  \end{proof}
  Así tenemos:
  \begin{theorem}
    \label{theo:PID=>UFD}
    Si \(D\) es un dominio de ideal principal,
    entonces es un dominio de factorización única.
  \end{theorem}
  \begin{proof}
    Por contradicción.
    Consideremos un elemento \(a \in D \smallsetminus D^\times\)
    (distinto de \(0\))
    con \(f(a)\) mínimo
    y que no tiene factorización en irreductibles.
    Entonces \(a\) no es irreductible
    (sería el producto de un irreductible),
    por lo que podemos escribir \(a = b c\),
    con \(b, c \notin D^\times\).
    Por el teorema~\ref{theo:ED:propiedades-f}
    resulta \(f(b) < f(a)\) y \(f(c) < f(a)\).
    Pero entonces \(b\) y \(c\) son producto de irreductibles,
    con lo que lo es \(a\).
    Vale decir,
    tal \(a\) no existe.

    Para demostrar factorización única usamos reducción al absurdo.
    Sea \(a\) un elemento de mínimo \(f\)
    que tiene dos factorizaciones esencialmente diferentes:
    \begin{equation*}
      a
	= u p_1 p_2 \dotsm p_m
	= v q_1 q_2 \dotsm q_n
    \end{equation*}
    donde los \(p_i\) son primos
    (no necesariamente diferentes),
    y similarmente los \(q_i\),
    y \(u\) y \(v\) son unidades.
    Por el lema~\ref{lem:ED:primo-divide-producto},
    esto significa que \(p_1\) divide a \(q_i\) para algún \(i\),
    o sea \(q_i = u_i p_i\) para alguna unidad \(u_i\),
    con lo que:
    \begin{equation*}
      a / p_1
	= u p_2 \dotsm p_m
	= v u_i q_1 \dotsm q_{i -1} q_{i + 1} \dotsm q_n
    \end{equation*}
    tendría dos factorizaciones diferentes,
    pero \(f(a / p_1) < f(a)\),
    lo que contradice la elección de \(a\) como uno de mínimo \(f\)
    con dos factorizaciones.
  \end{proof}
  Esto viene a ser el equivalente
  del teorema fundamental de la aritmética
  (teorema~\ref{theo:fundamental-aritmetica}):
  En un dominio euclidiano
  todo elemento \(a\) es el producto de un número finito de primos.
  Además,
  si tenemos factorizaciones en primos \(p_i\) y \(q_i\):
  \begin{equation*}
    a
      = p_1 p_2 \dotsm p_m
      = q_1 q_2 \dotsm q_n
  \end{equation*}
  entonces cada \(p\) es el asociado de uno de los \(q\).
  En particular,
  \(m = n\).

%%% Local Variables:
%%% mode: latex
%%% TeX-master: "clases"
%%% End:


\section{Factorización de polinomios}
\label{sec:factorizacion-polinomios}
\index{polinomio!factorizacion@factorización}

  Vimos que el conjunto de polinomios \(F[x]\)
  sobre el campo \(F\) es un dominio euclidiano,%
    \index{dominio euclidiano}
  el grado del polinomio sirve de función euclidiana.
  Podemos elegir el polinomio mónico
  como el representante de la clase de asociados,
  con lo que tenemos:%
    \index{polinomio!teorema fundamental de la aritmetica@teorema fundamental de la aritmética}
  \begin{theorem}[Teorema fundamental de la aritmética]
    \label{theo:fundamental-arithmetic-polynomials}
    Todo polinomio en \(F[x]\)
    es una unidad
    o el producto de una unidad y polinomios mónicos irreductibles.
    Esta factorización es única
    (salvo el orden de los factores).
  \end{theorem}
  Podemos caracterizar polinomios con ceros repetidos:
  \begin{lemma}
    \index{polinomio!ceros repetidos}
    \label{lem:repeated-roots}
    Si \(\alpha\) es un cero repetido
    del polinomio \(f(x) \in F[x]\),
    es cero común de \(f(x)\) y \(f'(x)\).
  \end{lemma}
  \begin{proof}
    Consideremos \(f(x) = (x - \alpha)^m g(x)\),
    donde \((x - \alpha) \centernot\mid g(x)\) y \(m > 1\).
    Tenemos:
    \begin{equation*}
      f'(x)
	= m (x - \alpha)^{m - 1} g(x) + (x - \alpha)^m g'(x)
	= (x - \alpha)^{m - 1} (m g(x) + (x - \alpha) g(x))
    \end{equation*}
    Como \(m > 1\),
    esto siempre es divisible por \(x - \alpha\).
  \end{proof}
  Tenemos también:
  \begin{theorem}[Euclides]
    \label{theo:Euclides-polynomials}
    Hay infinitos polinomios irreductibles sobre el campo \(F\).
  \end{theorem}
  \begin{proof}
    Por contradicción.%
      \index{demostracion@demostración!contradiccion@contradicción}
    Supongamos que hay finitos irreductibles
    \(p_1(x)\) a \(p_n(x)\).
    Entonces \(p_1(x) \dotsm p_n(x) + 1\)
    no es divisible por ningún \(p_i(x)\).
  \end{proof}
  Sobre un campo infinito no da nada nuevo
  (los polinomios lineales son todos irreductibles),
  pero sobre un campo finito sí es interesante.

  Consideremos ahora los polinomios como funciones,
  substituyendo elementos del campo por \(x\).

  \begin{corollary}
    \label{cor:polinomio-raiz-factor}
    Sea \(F\) un campo,
    y \(f(x) = a_n x^n + a_{n - 1} x^{n - 1} + \dotsb + a_0\)
    un polinomio de grado \(n\) sobre \(F\),
    y \(\alpha \in F\).
    Entonces \(f(\alpha) = 0\)
    si y solo si \(f(x) = (x - \alpha) g(x)\)
    para algún polinomio \(g \in F[x]\).
  \end{corollary}
  \begin{proof}
    Demostramos implicancia en ambas direcciones.

    Primeramente,
    si \(f(x) = (x - \alpha) g(x)\),
    claramente \(f(\alpha) = 0 \cdot g(\alpha) = 0\).

    Por el algoritmo de división podemos escribir
    \(f(x) = q(x) \cdot (x - \alpha) + r(x)\),
    donde \(\deg(r) < 1\)
    con lo que \(r(x)\) es constante.
    Ahora bien,
    \(f(\alpha)
       = q(\alpha) \cdot (\alpha - \alpha) + r(\alpha) = 0\),
    con lo que al ser constante
      \(r(x)\) es \(r(x) = r(\alpha) = 0\).
    Así \(f(x) = q(x) \cdot (x - \alpha)\),
    y llamando \(g(x) = q(x)\) completa la demostración.
  \end{proof}
  Con esta herramienta básica podemos demostrar
  el resultado siguiente:
  \begin{corollary}
    \index{polinomio!numero de ceros@número de ceros}
    \label{cor:numero-raices-polinomio}
    Si \(f(x) \in F[x]\) es un polinomio de grado \(n \ge 0\),
    entonces \(f(x)\) tiene a lo más \(n\) ceros en \(F\).
  \end{corollary}
  \begin{proof}
    Por inducción.%
      \index{demostracion@demostración!induccion@inducción}
    \begin{description}
    \item[Bases:]
      Para \(n = 0\),
      no hay ceros.

      Para \(n = 1\),
      tenemos:
      \begin{align*}
	&a_1 \cdot x + a_0
	  = 0 \\
	&x
	  = - a_1^{-1} \cdot a_0
      \end{align*}
      que claramente es única.
    \item[Inducción:]
      Suponiendo ahora que todos los polinomios de grado \(n\)
      tienen a lo más \(n\) ceros,
      consideremos un polinomio \(f(x)\) de grado \(n + 1\).
      Si no hay \(\alpha\) tal que \(f(\alpha) = 0\),
      \(f(x)\) tiene 0 ceros y estamos listos.
      Si tiene un cero \(\alpha\),
      por el corolario~\ref{cor:polinomio-raiz-factor}
      podemos escribir \(f(x) = (x - \alpha) \cdot g(x)\)
      con \(\deg(g) = n\).
      Por inducción,
      \(g(x)\) tiene a lo más \(n\) ceros,
      y \(f(x)\) tiene a lo más \(n + 1\) ceros.
    \end{description}
    Por inducción vale lo enunciado.
  \end{proof}

  Suelen ser útiles los términos más fáciles de calcular
  del polinomio
  con ceros \(\alpha_1\), \(\alpha_2\), \ldots, \(\alpha_m\);
  o sea,
  los coeficientes en el producto:%
    \index{Vieta, formulas de@Vieta, fórmulas de}
  \begin{equation}
    \label{eq:poly-coeficientes-ceros}
    (x - \alpha_1) \dotsm (x - \alpha_m)
       = x^m
	 - (\alpha_1 + \alpha_2 + \dotsb + \alpha_m) x^{m - 1}
	 + \dotsb
	 + (-1)^m \alpha_1 \cdot \alpha_2 \dotsm \alpha_m
  \end{equation}
  El coeficiente de \(x^{m - 1}\)
  es el negativo de la suma de los ceros,
  y el término constante es su producto
  con signo que depende de si \(m\) es par o impar.
  Las expresiones~\eqref{eq:poly-coeficientes-ceros}
  son casos particulares de las fórmulas de Vieta:
  Para el polinomio
    \(a_n x^n + a_{n - 1} x^{n - 1} + \dotsb + a_0\),
  los ceros \(\alpha_1, \alpha_2, \dotsc, \alpha_n\) cumplen:
  \begin{equation}
    \label{eq:Vieta-formulas}
    \sum_{1 \le i_1 \le i_2 \le \dotsb \le i_k \le n}
       \alpha_{i_1} \alpha_{i_2} \dotsm \alpha_{i_k}
       = (-1)^k \, \frac{a_{n - k}}{a_n}
  \end{equation}
  La suma en~\eqref{eq:Vieta-formulas} es simplemente
  sobre todos los conjuntos de \(k\) de las ceros.
  Por ejemplo:
  \begin{equation*}
    a x^3 + b x^2 + c x + d
      = a (x - \alpha_1) (x - \alpha_2) (x - \alpha_3)
  \end{equation*}
  resulta en las siguientes tres expresiones:
  \begin{align*}
    \alpha_1 + \alpha_2 + \alpha_3
      &= - \frac{b}{a} \\
    \alpha_1 \alpha_2 + \alpha_1 \alpha_3 + \alpha_2 \alpha_3
      &= \frac{c}{a} \\
    \alpha_1 \alpha_2 \alpha_3
      &= - \frac{d}{a}
  \end{align*}

  \begin{theorem}
    \label{theo:F*-ciclico}
    Sea \(F\) un campo finito,%
      \index{campo (algebra)@campo (álgebra)}
    y \(F^\times\) su grupo de unidades.%
      \index{anillo!grupo de unidades}
    Entonces \(F^\times\) es cíclico.%
      \index{grupo!ciclico@cíclico}
  \end{theorem}
  \begin{proof}
    Sea \(e\) el \emph{exponente} de \(F^\times\),%
      \index{grupo!exponente|textbfhy}
    vale decir,
    el mínimo natural
    tal que \(a^e = 1\) para todo \(a \in F^\times\).
    Esto es el mínimo común múltiplo
    de los órdenes de los elementos de \(F^\times\).
    Por el teorema de Lagrange%
      \index{Lagrange, teorema de}
    todos ellos son factores de \(\lvert F^\times \rvert\),
    con lo que \(e\) es factor de \(\lvert F^\times \rvert\)
    y \(e \le \lvert F^\times \rvert\).

    Por otro lado,
    todos los elementos de \(F^\times\) cumplen:
    \begin{equation*}
      x^e - 1
	= 0
    \end{equation*}
    En el campo \(F\)
    este polinomio puede tener a lo más \(e\) ceros,
    o sea \(e \ge \lvert F^\times \rvert\),
    con lo que \(e = \lvert F^\times \rvert\).
    Siendo \(e\) el mínimo común múltiplo
    de los órdenes en \(F^\times\),
    debe haber un elemento de orden \(e\)
    y \(F^\times\) es cíclico.
  \end{proof}

  \begin{corollary}
    \label{cor:Un-ciclico}
    Si \(p\) es primo,
    \(\mathbb{Z}^\times_p\) es cíclico.
  \end{corollary}
  \begin{proof}
    \(\mathbb{Z}_p\) es un campo finito,
    y \(\mathbb{Z}^\times_p\) es su grupo de unidades.
  \end{proof}

  Esto implica el curioso resultado:
  \begin{theorem}[Wilson]
    \index{Wilson, teorema de}
    \label{theo:Wilson}
    \((n - 1)! \equiv -1 \pmod{n}\) si y solo si \(n\) es primo.
  \end{theorem}
  \begin{proof}
    Demostramos implicancia en ambas direcciones.
    Si \(n\) es primo,
    \(\mathbb{Z}_n\) es un campo,
    y por la demostración del teorema~\ref{theo:F*-ciclico}
    sabemos que:
    \begin{equation*}
      x^{n - 1} - 1
       = \prod_{1 \le k \le n - 1} (x - k)
    \end{equation*}
    La fórmula de Vieta~\eqref{eq:Vieta-formulas}%
      \index{Vieta, formulas de@Vieta, fórmulas de}
    da para el término constante:
    \begin{equation*}
      (-1)^{n - 1} (n - 1)! \equiv -1 \pmod{n}
    \end{equation*}
    Si \(n = 2\),
    es \((-1)^{n - 1} = -1\) y \(1 \equiv -1 \pmod{n}\);
    si \(n\) es un primo impar,
    \((-1)^{n - 1} = 1\).
    De cualquier forma,
    \((n - 1)! \equiv -1 \pmod{n}\).

    Para el recíproco,
    demostramos el contrapositivo:
    Si \(n\) no es primo,
    \((n - 1)! \centernot\equiv -1 \pmod{n}\).
    Hay varios casos:
    Si \(n = 4\)
    es \(3! = 6 \equiv 2 \centernot\equiv -1 \pmod{4}\),
    y se cumple lo enunciado.
    Sea ahora \(n > 4\) compuesto.
    En tal caso podemos escribir \(n = a \cdot b\),
    con \(2 \le a, b < n - 1\).
    De ser \(a \ne b\),
    ambos factores aparecen en \((n - 1)!\),
    y por tanto \(n \mid (n - 1)!\).
    Si es \(n = a^2\),
    será \(a > 2\),
    y entre los factores de \((n - 1)!\)
    estarán \(a\) y \(2 a\),
    con lo que nuevamente \(n \mid (n - 1)!\).
    En estas dos situaciones tenemos \((n - 1)! \equiv 0 \pmod{n}\),
    y obtenemos lo enunciado.
  \end{proof}
  Este resultado no es útil para computación,
  y de uso teórico bastante limitado.

  Una demostración alternativa es considerar un primo impar \(p\),
  y parear cada \(a \in \mathbb{Z}_p\) con su inverso.
  Los únicos elementos
  que no tienen pareja en esto son \(1\) y \(-1\)
  (son sus propios inversos,
   en el campo \(\mathbb{Z}_p\)
   la ecuación \(x^2 - 1 = 0\) puede tener
   a lo más dos raíces),
  por lo que el producto de todos ellos es \(-1\).

\section{Raíces primitivas}
\label{sec:raices-primitivas}
\index{raiz primitiva@raíz primitiva|textbfhy}

  Cuando \(R^\times\) es cíclico,%
    \index{grupo!ciclico@cíclico}
  a un generador de \(R^\times\)
  se le llama \emph{elemento primitivo} de \(R\).%
    \index{anillo!elemento primitivo}
  En el caso de \(\mathbb{Z}_n\)
  se le llama una \emph{raíz primitiva} módulo \(n\)
  (porque todo elemento de \(\mathbb{Z}^\times_n\)
   puede escribirse como potencia del generador).
  Lo anterior demuestra que todo primo \(p\)
  tiene raíces primitivas.
  Una pregunta obvia es qué valores de \(n\)
  tienen raíces primitivas
  (o sea,
   cuándo es cíclico \(\mathbb{Z}^\times_n\)).

  Consideremos \(n = 8\),
  con \(\phi(8) = 4\).
  Por cálculo directo tenemos los órdenes
  de los elementos de \(\mathbb{Z}^\times_8\)
  dados en el cuadro~\ref{tab:ordenes-U8}.
  \begin{table}[htbp]
    \centering
    \begin{tabular}[htbp]{|c|c|}
      \hline
      \multicolumn{1}{|c|}
	  {\rule[-0.7ex]{0pt}{3ex}\(\boldsymbol{k}\)} &
	\multicolumn{1}{c|}{\(\boldsymbol{\ord_8(k)}\)} \\
      \hline\rule[-0.7ex]{0pt}{3ex}%
      1 & 1 \\
      3 & 2 \\
      5 & 2 \\
      7 & 2 \\
      \hline
    \end{tabular}
    \caption{Órdenes de los elementos en $\mathbb{Z}^\times_8$}
    \label{tab:ordenes-U8}
  \end{table}
  Se aprecia que no hay elementos de orden \(4\),
  y \(8\) no tiene raíces primitivas.

  Por otro lado,
  para \(n = 2\) tenemos \(\phi(2) = 1\)
  y claramente 1 es raíz primitiva módulo 2.
  Para \(n = 4\) tenemos \(\phi(4) = 2\),
  y \(3\) es raíz primitiva módulo \(4\).

  Más generalmente,
  tenemos:
  \begin{theorem}
    \label{theo:raices-primitivas-2k}
    No hay raíces primitivas módulo \(2^m\) si \(m \ge 3\).
  \end{theorem}
  \begin{proof}
    Demostraremos esto por inducción.%
      \index{demostracion@demostración!induccion@inducción}
    \begin{description}
    \item[Base:]
      Cuando \(m = 3\),
      vimos antes que \(2^m = 8\) no tiene raíces primitivas.
    \item[Inducción:]
      Suponemos que la aseveración vale para \(m - 1\).
      Como \mbox{\(\phi(2^m) = 2^{m - 1}\)}
      y el único divisor de \(2^{m - 1}\) es \(2\),
      basta demostrar
      que el orden de todo elemento en \(\mathbb{Z}_{2^m}\)
      es divisor de \(2^{m - 2}\).
      Consideremos \(a\) cualquiera relativamente primo a \(2^m\).
      Por inducción
      sabemos que \(a\) no es raíz primitiva de \(2^{m - 1}\),
      que expresado en los términos anteriores es:
      \begin{align}
	a^{2^{m - 3}}
	  &\equiv 1 \pmod{2^{m - 1}} \notag \\
	a^{2^{m - 3}}
	  &= c \cdot 2^{m - 1} + 1   \notag \\
      \intertext{Elevando al cuadrado:}
	a^{2^{m - 2}}
	  &= \left(a^{2^{m - 3}}\right)^2 \notag \\
	  &= c^2 \cdot 2^m + 2 \cdot c \cdot 2^{m - 1} + 1 \notag \\
	  &= (c^2 + c) \cdot 2^m + 1 \notag \\
	  &\equiv 1 \pmod{2^m} \label{eq:a^(2^(m-2))=1}
      \end{align}
    \end{description}
    Por~\eqref{eq:a^(2^(m-2))=1}
    el orden de \(a\) módulo \(2^m\)
    es a lo más \(2^{m - 2} < 2^{m - 1} = \phi(2^m)\),
    y \(a\) no es raíz primitiva.
  \end{proof}
  Esto da un conjunto infinito de enteros sin raíces primitivas.
  Pero aún más:
  \begin{theorem}
    \label{theo:raices-primitivas-mn}
    No hay raíces primitivas módulo \(m n\)
    si \(m\) y \(n\) son enteros impares relativamente primos
    mayores que \(2\).
  \end{theorem}
  \begin{proof}
    Como \(m\) y \(n\) son impares y mayores que \(2\),
    sabemos que \(2\)
    es un factor común entre \(\phi(m)\) y \(\phi(n)\)
    (un primo impar \(p\)
     aporta \(p^{k - 1} (p - 1)\) a \(\phi(\cdot)\)).
    Para cualquier \(a\) tal que \(\gcd(a, m n) = 1\),
    por el teorema de Euler:
    \begin{align}
      a^{\phi(m) \phi(n) / 2}
	&\equiv \left(a^{\phi(m)}\right)^{\phi(n) / 2}
	   \equiv 1^{\phi(n) / 2}
	   \equiv 1 \pmod{m}  \label{eq:a^phiphi/2-m} \\
      a^{\phi(m) \phi(n) / 2}
	&\equiv \left(a^{\phi(n)}\right)^{\phi(m) / 2}
	   \equiv 1^{\phi(m) / 2}
	   \equiv 1 \pmod{n} \label{eq:a^phiphi/2-n}
    \end{align}
    Combinando~\eqref{eq:a^phiphi/2-m} con~\eqref{eq:a^phiphi/2-n}
    mediante el teorema~\ref{theo:congruencia-mn}
    resulta:
    \begin{equation}
      \label{eq:a^phiphi/2-mn}
      a^{\phi(m) \phi(n) / 2} \equiv 1 \pmod{m n}
    \end{equation}
    Por~\eqref{eq:a^phiphi/2-mn}
    el orden de \(a\) divide a \(\phi(m n) / 2\),
    y \(a\) no es raíz primitiva de \(m n\).
    Pero \(a\) es arbitrario,
    con lo que \(m n\) no tiene raíces primitivas.
  \end{proof}

  Lo anterior excluye \(2^k\) para \(k \ge 3\),
  y los números compuestos impares con factores primos distintos.
  Analicemos los restantes.
  \begin{theorem}
    \label{theo:raiz-primitiva-p2}
    Sea \(p\) un primo impar,
    entonces hay una raíz primitiva módulo \(p^2\)
  \end{theorem}
  \begin{proof}
    Sea \(r\) una raíz primitiva de \(p\),
    o sea \(\ord_p (r) = p - 1\).
    Sabemos que si \(n = \ord_{p^2}(r)\)
    entonces \(n \mid \phi(p^2)\),
    vale decir \(n \mid p (p - 1)\).
    Sabemos que si \(r^n \equiv 1 \pmod{p^2}\)
    entonces \(r^n \equiv 1 \pmod{p}\),
    de forma que \(\phi(p) \mid n\).
    Pero si \(p - 1 \mid n\) y \(n \mid p (p - 1)\),
    entonces \(n = p (p - 1)\) o \(n = p - 1\).
    En el primer caso,
    tenemos que \(r\) es raíz primitiva de \(p^2\),
    y estamos listos.

    En el segundo caso,
    consideremos el elemento \(r + p\),
    que sigue siendo raíz primitiva módulo \(p\),
    con lo que su orden es \(p - 1\) o \(p (p - 1)\) módulo \(p^2\).
    Calculamos:
    \begin{equation}
      \label{eq:(r+p)^(p-1)}
      (r + p)^{p - 1}
	= r^{p - 1}
	   + \binom{p - 1}{1} p r^{p - 2}
	   + \sum_{2 \le k \le p - 1}
	       \binom{p - 1}{k} p^k r^{p - 1 - k}
    \end{equation}
    Todos los elementos de la sumatoria en~\eqref{eq:(r+p)^(p-1)}
    son divisibles por \(p^2\).
    Como estamos suponiendo
    que el orden de \(r\) módulo \(p^2\) es \(p - 1\),
    y sabemos que:
    \begin{equation*}
      (p - 1) p r^{p - 2}
	\equiv - p r^{p - 2}
	\centernot\equiv 0 \pmod{p^2}
    \end{equation*}
    (ya que \(\gcd(r, p) = 1\)
     también es \(\gcd(r^{p - 2}, p) = 1\)),
    y queda:
    \begin{equation*}
      (r + p)^{p - 1}
	\equiv 1 - p r^{p - 2}
	\centernot\equiv 1 \pmod{p^2} \\
    \end{equation*}
    y \(r + p\) es raíz primitiva módulo \(p^2\).
  \end{proof}
  Esto parece ser solo un paso al ir de \(p\) a \(p^2\),
  pero resulta ser todo lo que se requiere.
  Antes de seguir,
  un lema técnico.
  \begin{lemma}
    \label{lem:raiz-primitiva-p2-orden-pk}
    Sea \(p\) un primo impar,
    \(r\) una raíz primitiva módulo \(p^2\).
    Entonces para \(m \ge 2\):
    \begin{equation}
      \label{eq:r^(p^(m-2)(p-1))<>1}
      r^{p^{m - 2} (p - 1)}
	\centernot\equiv 1 \pmod{p^m}
    \end{equation}
  \end{lemma}
  \begin{proof}
    La demostración es por inducción desde \(m = 2\).%
      \index{demostracion@demostración!induccion@inducción}
    De partida,
    si \(r\) es raíz primitiva módulo \(p^2\),
    lo es módulo \(p\) y \(\gcd(r, p) = 1\).
    \begin{description}
    \item[Base:]
      Para \(m = 2\)
      el orden de la raíz primitiva \(r\) de \(p^2\)
      es \(\phi(p^2) = p (p - 1)\),
      por lo que:
      \begin{equation}
	\label{eq:r^(p-1)<>1}
	r^{p - 1}
	  \centernot\equiv 1 \pmod{p^2}
      \end{equation}
    \item[Inducción:]
      Supongamos que el resultado vale para \(m - 1\).
      Del teorema de Euler%
	\index{Euler, teorema de}
      tenemos:
      \begin{align}
	r^{\phi(p^{m - 2})}
	  &\equiv 1 \pmod{p^{m - 2}} \notag \\
	r^{p^{m - 3} (p - 1)}
	  &=	  1 + c \cdot p^{m - 2} \label{eq:r^(p^(m-3)(p-1))}
      \end{align}
      Nótese que \(p \centernot\mid c\),
      ya que de lo contrario tendríamos:
      \begin{equation}
	\label{eq:r^(p^(m-3)(p-1))=1}
	r^{p^{m - 3} (p - 1)}
	  \equiv 1 \pmod{p^{m - 1}}
      \end{equation}
      y esto contradice
      nuestra hipótesis de inducción~\eqref{eq:r^(p^(m-2)(p-1))<>1}.
      Elevando~\eqref{eq:r^(p^(m-3)(p-1))=1} a la potencia \(p\),
      queda:
      \begin{align}
	r^{p^{m - 2} (p - 1)}
	  &= \left(1 + c p^{m - 2}\right)^p \notag \\
	  &= 1 + c p^{m - 1}
	      + \sum_{2 \le k \le p} \binom{p}{k} c^k p^{k (m - 2)}
	  \label{eq:r^(p^(m-2)(p-1))}
      \end{align}
      Interesa demostrar
      que cada término de la sumatoria~\eqref{eq:r^(p^(m-2)(p-1))}
      es divisible por \(p^m\).
      En ellos aparece el factor \(p^{k (m - 2) + 1}\)
      (incluyendo \(p\) del coeficiente binomial).%
	\index{coeficiente binomial}
      Interesa acotar el exponente \(k (m - 2) + 1\),
      con \(k \ge 2\) y \(m \ge 3\).
      Es \(m - 2 \ge 1\),
      el mínimo exponente se da para \(k = 2\)
      y tenemos \(k (m - 2) + 1 \ge 2 m - 3 \ge m\).
      En consecuencia:
      \begin{align*}
	r^{p^{m - 2} (p - 1)}
	  &\equiv     1 + c p^{m - 1} \pmod{p^m} \\
	  &\centernot\equiv 1 \pmod{p^m}
      \end{align*}
      como queríamos probar.
      \qedhere
    \end{description}
  \end{proof}
  \begin{theorem}
    \label{theo:raiz-primitiva-pk}
    Toda raíz primitiva módulo \(p^2\) para un primo impar \(p\)
    es raíz primitiva módulo \(p^m\) con \(m \ge 2\)
  \end{theorem}
  \begin{proof}
    Sea \(r\) raíz primitiva módulo \(p^2\),
    con lo que \(\ord_{p^2} (r) = p (p - 1)\)
    ya que \(\phi(p^2) = p (p - 1)\),
    y llamemos \(n = \ord_{p^m} (r)\).
    Como antes,
    por el teorema de Euler sabemos que:
    \begin{equation*}
      r^{\phi(p^m)}
	\equiv r^{p^{m - 1} (p - 1)} \equiv 1 \pmod{p^m}
    \end{equation*}
    con lo que:
    \begin{equation*}
      n \mid p^{m - 1} (p - 1)
    \end{equation*}
    Pero también:
    \begin{equation*}
      r^n
	\equiv 1 \pmod{p^m}
    \end{equation*}
    implica:
    \begin{equation*}
      r^n
	\equiv 1 \pmod{p^2}
    \end{equation*}
    con lo que \(p (p - 1) \mid n\),
    y \(n = p^k (p - 1)\) para algún \(1 \le k \le m - 1\).
    Por el lema~\ref{lem:raiz-primitiva-p2-orden-pk} sabemos que:
    \begin{equation*}
      r^{p^{m - 2} (p - 1)}
	\centernot\equiv 1 \pmod{p^m}
    \end{equation*}
    con lo que \(k = m - 1\),
    y \(r\) es raíz primitiva módulo \(p^m\).
  \end{proof}

  Resta el caso \(2 p^m\),
  donde por el corolario~\ref{cor:isomorfismo-anillo-Zm} resulta
  \(\mathbb{Z}_{2 p^m} \cong \mathbb{Z}_2 \oplus \mathbb{Z}_{p^m}\),
  y \(\mathbb{Z}^\times_{2 p^m}
	\cong \mathbb{Z}^\times_2 \oplus \mathbb{Z}^\times_{p^m}
	\cong \mathbb{Z}^\times_{p^m}\),
  y hay una raíz primitiva.
  En resumen,
  obtenemos el curioso resultado:
  \begin{theorem}
    \label{theo:raices-primitivas}
    Hay raíces primitivas módulo \(n\) si y solo si
    \(n = 2, 4, p^m, 2 p^m\),
    donde \(p\) es un primo impar y \(m \ge 1\).
  \end{theorem}

  Igual resulta interesante hallar el máximo orden módulo \(n\),
  al que llamaremos \(\lambda(n)\).%
    \index{\(\lambda(m)\) (maximo orden modulo \(m\))@\(\lambda(m)\) (máximo orden módulo \(m\))|textbfhy}
  En algunos casos
  lo conocemos por el teorema~\ref{theo:raices-primitivas},
  falta completarlo para los demás valores de \(n\).

  \begin{lemma}
    \label{lem:lambda-2^e}
    El máximo orden en \(\mathbb{Z}_{2^e}\) con \(e \ge 3\)
    es \(\lambda(2^e) = 2^{e - 2}\)
  \end{lemma}
  \begin{proof}
    Consideremos \(a\) impar.
    Entonces uno de \(a \pm 1\) es divisible por \(4\),
    y tenemos para algún \(f \ge 2\) y un \(c\):
    \begin{align*}
      a
	&\equiv 2^f \pm 1 \pmod{2^{f + 1}} \\
	&=	c \cdot 2^{f + 1} + 2^f \pm 1  \\
      a^2
	&=	c^2 \cdot 2^{2 (f + 1)} + 2^{2 f} + 1
		  + 2 \cdot c \cdot 2^{2 f + 1}
		  \pm 2 \cdot c \cdot 2^{f + 1}
		  \pm 2 \cdot 2^f		 \\
	&\equiv 2^{f + 1} + 1 \pmod{2^{f + 2}}
    \end{align*}
    Continuando de la misma forma,
    concluimos que para \(r \ge 1\):
    \begin{equation}
      \label{eq:a^(2^r)}
      a^{2^r}
	\equiv 2^{f + r} + 1 \pmod{2^{f + r + 1}}
    \end{equation}
    Cuando \(f + r + 1 = e\),
    la ecuación~\eqref{eq:a^(2^r)} lleva a:
    \begin{align*}
      a^{2^{e - f - 1}}
	&\equiv		  2^{e - 1} + 1 \pmod{2^e} \\
	&\centernot\equiv 1		\pmod{2^e} \\
      a^{2^{e - f}}
	&\equiv	    2^e + 1 \pmod{2^{e + 1}} \\
	&\equiv	    1	    \pmod{2^e}
    \end{align*}
    O sea,
    \(\ord_{2^e} (a) = 2^{e - f}\),
    donde \(f\) es el valor determinado anteriormente.
    El mínimo valor posible de \(f\) es \(2\),
    y siempre podemos elegir \(a = 4 - 1 = 3\) que da \(f = 2\),
    así \(\lambda(2^e) = 2^{e - 2}\).
  \end{proof}
  Incidentalmente,
  hemos hallado una manera simple de calcular \(\ord_{2^e}(a)\).
  \begin{theorem}
    \label{theo:max-orden-Z*}
    El máximo orden está dado por:
    \begin{align*}
      \lambda(2) = 1, \qquad
	&\lambda(4) = 2,
	  \qquad \lambda(2^e) = 2^{e - 2} \text{\ si\ } e \ge 3 \\
      \lambda(p^e)
	&= p^{e - 1} (p - 1) \text{\ si el primo\ } p > 2 \\
      \lambda(p_1^{e_1} p_2^{e_2} \dotsm p_r^{e_r}&)
	 = \lcm(\lambda(p_1^{e_1}), \lambda(p_2^{e_2}), \dotsc,
		\lambda(p_r^{e_r}))
	   \text{\ si \(p_1, \dotsc, p_r\) son primos}
    \end{align*}
  \end{theorem}
  \begin{proof}
    Los casos \(2\) y \(4\) son obvios.
    El caso \(2^e\) es el tema del lema~\ref{lem:lambda-2^e},
    el caso \(p^e\)
    es inmediato del teorema~\ref{theo:raices-primitivas}
    y \(\phi(p^e) = p^{e - 1} (p - 1)\).

    Ahora,
    para \(\gcd(m, n) = 1\)
    sabemos del teorema chino de los residuos%
      \index{residuo!teorema chino de los}
    (en realidad,
     del corolario~\ref{cor:isomorfismo-anillo-Zm})%
       \index{anillo!isomorfismo}
    que \(\mathbb{Z}^\times_{m n}
	    \cong \mathbb{Z}^\times_m \times \mathbb{Z}^\times_n\).
    El orden de un elemento \(a \in \mathbb{Z}^\times_{m n}\),
    que podemos descomponer en \(u v\)
    con \(u \in \mathbb{Z}^\times_m\)
    y \(v \in \mathbb{Z}^\times_n\),
    es \(\lcm(\ord_m(u), \ord_n(v))\).
    Si elegimos elementos de orden máximo para \(u\) y \(v\),
    \(a\) será de orden máximo,
    su orden es \(\lambda(m n) = \lcm(\lambda(m), \lambda(n))\),
    y el resultado sigue.
  \end{proof}

  Algunos algoritmos
  requieren usar raíces primitivas de un primo grande \(p\),%
    \index{raiz primitiva@raíz primitiva!numero@número}
  y los anteriores no dan muchas luces de cómo encontrar una.
  Por suerte son relativamente numerosas:
  Si tomamos una raíz primitiva \(r\),
  todos los elementos \(r^k\) con \(\gcd(k, p - 1) = 1\)
  también tendrán orden \(p - 1\) y son raíces primitivas.
  Vale decir,
  hay \(\phi(p - 1)\) raíces primitivas del primo \(p\).

%%% Local Variables:
%%% mode: latex
%%% TeX-master: "clases"
%%% End:

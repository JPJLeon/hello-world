% teoria-numeros.tex
%
% Copyright (c) 2009-2015 Horst H. von Brand
% Derechos reservados. Vea COPYRIGHT para detalles

\chapter{Teoría de números}
\label{cha:teoria-numeros}
\index{teoria de numeros@teoría de números}

  El estudio de los números enteros tiene una larga y distinguida historia,
  algo de la cual describe Ore~\cite{ore69:_invit_number_theo}.
  La importancia de la teoría de números en informática
  es porque mucho de lo que se hace en el computador
  es trabajar con números
  (o cosas que se representan como tales,
   o debemos contar objetos con ciertas características
   para determinar los recursos que se requieren).
  Mucha de la tecnología criptográfica moderna%
    \index{criptografia@criptología}
  se basa en resultados de la teoría de números
  y áreas afines.
  Nuestra presentación sigue la de Richman~%
    \cite{richman71:_number_theor}
  en que explícitamente discute las estructuras algebraicas involucradas.

\section{Algunas herramientas}
\label{sec:algunas-herramientas}

  Para calcular con números grandes
  resulta cómodo el programa \texttt{bc(1)},%
    \index{bc(1)@\texttt{bc(1)}}
  para computaciones más complejas
  son útiles \texttt{PARI/GP}~%
    \cite{PARI:2.7.2}%
    \index{PARI/GP@\texttt{PARI/GP}}
  o \texttt{maxima}~%
    \cite{maxima14b:_computer_algebra}.%
    \index{maxima@\texttt{maxima}}
  El sistema Sage~\cite{stein14:_Sage-6.3}%
    \index{Sage@\texttt{Sage}}
  agrupa varios sistemas para computación numérica y simbólica
  (incluyendo los mencionados)
  de código abierto bajo una interfaz común.%
    \index{codigo abierto@código abierto}
  Para uso en programas se recomiendan las bibliotecas GMP~%
    \cite{granlund14:_gnu_multip_precis_arith_librar},%
    \index{GMP@\texttt{GMP}}
 CLN~%
    \cite{haible14:_CLN_1.3.4}%
    \index{CLN@\texttt{CLN}}
  o NTL~%
    \cite{shoup14:_ntl}.%
    \index{NTL@\texttt{NTL}}
  Detalles de muchos algoritmos en \cplusplus{}%
    \index{C++ (lenguaje de programacion)@\cplusplus{} (lenguaje de programación)}
  da Arndt~%
    \cite{arndt11:_matters_computational}.

\section{Propiedades básicas}
\label{sec:propiedades-basicas}
\index{Z (numeros enteros)@\(\mathbb{Z}\) (números enteros)}

  Ya sabemos
  (ver el capítulo~\ref{cha:relaciones-funciones})
  que la relación ``divide a'' es una relación de orden en \(\mathbb{N}\)%
    \index{relacion@relación!orden}
  (en \(\mathbb{Z}\) no es antisimétrica,
   ya que por ejemplo \(-3 \mid 3\) y \(3 \mid -3\),
   pero \(3 \ne -3\)).
  Algunas propiedades adicionales son:
  \begin{enumerate}
  \item
    Si \(a \mid b\),
    entonces \(a \mid b \cdot c\)
  \item
    Si \(a \mid b\) y \(a \mid c\),
    entonces \(a \mid (s b + t c)\)
    para todo \(s, t \in \mathbb{Z}\)
  \item
    Siempre que \(c \ne 0\),
    \(a \mid b\) si y solo si \(a c \mid b c\)
  \end{enumerate}
  Si \(a \mid b\),
  decimos que \(b\) es múltiplo de \(a\),
  o que \(a\) es un factor de \(b\).

  Una propiedad básica de los enteros es la siguiente:
  \begin{theorem}[Algoritmo de división]
    \index{algoritmo de division@algoritmo de división}
    \label{theo:division}
    Sean \(n, d \in \mathbb{Z}\)
    con \(d > 0\).
    Entonces existen enteros \(q, r\) únicos tales que:
    \begin{equation*}
      n = q \cdot d + r \quad 0 \le r < d
    \end{equation*}
  \end{theorem}
  \begin{proof}
    El conjunto de ``restos'' es:
    \begin{equation}
      \mathcal{R} = \{n - s d \colon s \in \mathbb{Z} \wedge n - s d \ge 0\}
    \end{equation}
    Este conjunto no es vacío,
    siempre será \(n + (\lvert n \rvert + 1) \cdot d > 0\),
    y esto está en \(\mathcal{R}\).
    Siendo \(\mathcal{R}\) un conjunto de enteros acotado por debajo,
    contiene su mínimo,
    llamémosle \(r\),
    que podemos expresar \(r = n - q \cdot d\).
    Debemos demostrar que \(r < d\).
    Esto lo haremos por contradicción.%
      \index{demostracion@demostración!contradiccion@contradicción}
    Si suponemos \(r \ge d\),
    podemos escribir:
    \begin{equation*}
      r - d = n - (q + 1) \cdot d
    \end{equation*}
    donde \(r - d \ge 0\),
    pertenece a \(\mathcal{R}\)
    y es menor que \(r\),
    contradiciendo la elección de \(r\) como el menor elemento
    en \(\mathcal{R}\).

    Falta demostrar que \(q\) y \(r\) son únicos.
    Nuevamente procedemos por contradicción.%
      \index{demostracion@demostración!contradiccion@contradicción}
    Supongamos dos soluciones diferentes \(q_1, r_1\) y \(q_2, r_2\),
    donde podemos tomar \(r_2 \ge r_1\) sin pérdida de generalidad.
    Entonces:
    \begin{align}
      n
	&= q_1 \cdot d + r_1 \notag \\
      n
	&= q_2 \cdot d + r_2 \notag \\
      0
	&= (q_2 - q_1) \cdot d + (r_2 - r_1) \label{eq:(q2-q1)*d+(r2-r1)}
    \end{align}
    Sabemos que \(0 \le r_2 < d\) por lo anterior.
    Como asumimos \(r_1 \le r_2\),
    tenemos que \(0 \le r_2 - r_1\).
    Pero también,
    dado que \(0 \le r_1 < d\):
    \begin{align*}
      r_2
	&< d \\
      r_2 - r_1
	&< d - r_1
	  \le d
    \end{align*}
    En resumen,
    resulta:
    \begin{equation}
      \label{eq:r2-r1}
      0 \le r_2 - r_1 < d
    \end{equation}
    Pero de la ecuación~\eqref{eq:(q2-q1)*d+(r2-r1)} tenemos:
    \begin{equation}
      \label{eq:q1-q2}
      (q_1 - q_2) \cdot d
	= r_2 - r_1
    \end{equation}
    Como el lado izquierdo de~\eqref{eq:q1-q2} es divisible por \(d\)
    también lo es el derecho;
    pero por~\eqref{eq:r2-r1} esto es menor que \(d\),
    y la única posibilidad es que sea cero.
    Concluimos que \(r_1 = r_2\).
    Pero siendo cero el lado derecho de~\eqref{eq:q1-q2},
    y el izquierdo el producto de \(d\)
    y \(q_1 - q_2\),
    concluimos que \(q_1 - q_2 = 0\),
    o sea \(q_1 = q_2\).
    Pero habíamos supuesto que \((q_1, r_1) \ne (q_2, r_2)\),
    contradicción que demuestra que \(q\) y \(r\) son únicos.
  \end{proof}

  Es común que queramos hablar del resto de la división,%
    \index{modulo@módulo|textbfhy}
  para ello introducimos la notación:
  \begin{equation*}
    r = a \bmod b
  \end{equation*}
  El cociente respectivo puede expresarse simplemente como:%
    \index{cociente entero}
  \begin{equation*}
    q = \left\lfloor \frac{a}{b} \right\rfloor
  \end{equation*}
  No se requiere una notación especial para esto,
  se usa con menos frecuencia que el resto.

\section{Máximo común divisor}
\label{sec:GCD}
\index{maximo comun divisor@máximo común divisor|textbfhy}
\index{GCD@\emph{GCD}|see{máximo común divisor}}

  Sean \(a, b \in \mathbb{Z}\),
  y consideremos
  \(\mathcal{I} = \{u a + v b \colon u, v \in \mathbb{Z}\}\).%
    \index{ideal}
  Si \(a = b = 0\),
  entonces \(\mathcal{I} = \{0\}\).
  En caso contrario,
  este conjunto no es vacío,
  y contiene elementos positivos
  (siempre es \(a^2 + b^2 \in \mathcal{I}\)).
  Consideremos el mínimo elemento positivo de \(\mathcal{I}\),
  llamémosle \(m\).
  Entonces hay \(s, t \in \mathbb{Z}\)
  tales que:
  \begin{equation}
    \label{eq:base-Bezout}
    m = s a + t b
  \end{equation}
  Demostraremos que \(m\) divide a todos los elementos de \(\mathcal{I}\).

  Sea \(n \in \mathcal{I}\),
  que significa \(n = s' a + t' b\) para algún \(s'\) y \(t'\).
  Por el algoritmo de división:%
    \index{algoritmo de division@algoritmo de división}
  \begin{equation}
    \label{eq:n=qm+r}
    n = q m + r \quad 0 \le r < m
  \end{equation}
  Pero:
  \begin{align*}
    n
      &= s' a + t' b \\
      &= q (s a + t b) + r	 \\
    r
      &= (s' - q s) a + (t' - q t) b
  \end{align*}
  con lo que \(r \in \mathcal{I}\).
  Como \(m\) es el mínimo elemento positivo de \(\mathcal{I}\),
  por~\eqref{eq:n=qm+r}
  solo puede ser \(r = 0\),
  y \(m \mid n\);
  e \(\mathcal{I}\) es el conjunto de los múltiplos de \(m\).
  Como \(a\) y \(b\) pertenecen a \(\mathcal{I}\),
  \(m\) es un divisor común de ambos.
  Pero por otro lado,
  cualquier divisor de \(a\) y \(b\) debe dividir a \(m = s a + t b\),
  con lo que \(m\) es máximo.
  Este número lo anotaremos \(\gcd(a, b)\)
  (por \emph{\foreignlanguage{english}{greatest common divisor}} en inglés).
  Otra notación común es \((a, b)\).
  Para completar la definición de esta función,
  el máximo común divisor entre \(a \ne 0\) y \(0\) es simplemente \(a\),
  y podemos definir \(\gcd(0, 0) = 0\).
  A la importante relación \(\gcd(a, b) = s a + t b\)
  se le llama \emph{identidad de Bézout}.
    \index{Bezout, identidad de@Bézout, identidad de|textbfhy}%
    \index{Bezout, Etienne@Bézout, Étienne}
  En caso que \(\gcd(a, b) = 1\) se dice que \(a\) y \(b\)
  son \emph{relativamente primos}%
    \index{relativamente primos}%
  o \emph{coprimos}.%
    \index{coprimos|see{relativamente primos}}

  Nótese que si:
  \begin{equation}
    \label{eq:Bezout}
    \gcd(a, b)
      = s a + t b
  \end{equation}
  también son soluciones a \(\gcd(a, b) = s' a + t' b\)
  para todo \(k \in \mathbb{Z}\):
  \begin{equation}
    \label{eq:Bezout-coefficients}
    \begin{split}
      s' &= s + \frac{k b}{\gcd(a, b)} \\
      t' &= t - \frac{k a}{\gcd(a, b)}
    \end{split}
  \end{equation}
  Esto es fácil de ver substituyendo~\eqref{eq:Bezout-coefficients}.

  \begin{lemma}
    \index{maximo comun divisor@máximo común divisor!propiedades}
    \index{operacion@operación!maximo comun divisor@máximo común divisor}
    \label{lem:gcd}
    Tenemos las siguientes propiedades del máximo común divisor:
    \begin{enumerate}
    \item
      \label{lem:gcd:conmutativo}
      \(\gcd(a, b) = \gcd(b, a)\)
    \item
      \label{lem:gcd:signos}
      \(\gcd(a, b) = \gcd(\pm a, \pm b)\)
    \item
      \label{lem:gcd:gcd_maximal}
      Todo divisor común de \(a\) y \(b\) divide a \(\gcd(a, b)\).
    \item
      \label{lem:gcd:factor_comun}
      \(\gcd(k a, k b) = \lvert k  \rvert \cdot \gcd(a, b)\).
    \item
      \label{lem:gcd:dividir_gcd}
      Si \(m = \gcd(a, b)\),
      entonces \(\gcd(a / m, b / m) = 1\).
      Nótese que \(a / m\) y \(b / m\) son enteros acá,
      la división es exacta.
    \item
      \label{lem:gcd:producto}
      Si \(\gcd(a, b) = 1\) y \(\gcd(a, c) = 1\),
      entonces \(\gcd(a, b c) = 1\).
    \item
      \label{lem:gcd:divisor}
      Si \(a \mid b c\) y \(\gcd(a, b) = 1\),
      entonces \(a \mid c\).
    \item
      \label{lem:gcd:producto-coprimos-divide}
      Si \(\gcd(a, b) = 1\),
      y \(a \mid c\) y \(b \mid c\),
      entonces \(a b \mid c\).
    \end{enumerate}
  \end{lemma}
  \begin{proof}
    Demostramos cada parte por turno.
    \begin{enumerate}
    \item
      Los elementos de
      \(\mathcal{I} = \{u a + v b \colon u, v \in \mathbb{Z}\}\)
      e \(\mathcal{I}' = \{u b + v a \colon u, v \in \mathbb{Z}\}\)
      son los mismos,
      y lo son sus mínimos elementos positivos.
    \item
      Nuevamente,
      los elementos del conjunto \(\mathcal{I}\) respectivo
      son los mismos para ambos lados de la ecuación.
    \item
      Esto lo vimos antes.
    \item
      De la identidad de Bézout%
	\index{Bezout, identidad de@Bézout, identidad de}
      sabemos que hay \(s, t \in \mathbb{Z}\)
      tales que:
      \begin{align*}
	\gcd(k a, k b)
	  &= s (\lvert k  \rvert \cdot a)
	       + t (\lvert k  \rvert \cdot b) \\
	  &= \lvert k  \rvert \cdot (s a + t b)
      \end{align*}
      En particular,
      este es el mínimo de todos los valores positivos
      que se pueden obtener eligiendo \(s, t \in \mathbb{Z}\),
      por lo que \(s a + t b\)
      debe también ser el mínimo positivo de esta última expresión,
      \(s a + t b = \gcd(a, b)\).
    \item
      De la identidad de Bézout
      sabemos que hay \(s\) y \(t\) que dan \(m = \gcd(a, b)\) como:
      \begin{align*}
	m &= s a + t b \\
	1 &= s (a / m) + t (b / m)
      \end{align*}
      con lo que \(\gcd(a / m, b / m) = 1\).
      Esto también implica que \(\gcd(s, t) = 1\).
    \item
      Si \(\gcd(a, b) = \gcd(a, c) = 1\)
      existen \(s, t, u, v \in \mathbb{Z}\)
      tales que:
      \begin{align*}
	1 &= s a + t b \\
	1 &= u a + v c
      \end{align*}
      Entonces:
      \begin{align*}
	t b
	  &= 1 - s a \\
	v c
	  &= 1 - u a \\
	t v b c
	  &= 1 - (s + u) a + su a^2 \\
	1
	  &= (s + u - s u a) a + (t v) b c
      \end{align*}
      Esto es el mínimo positivo,
      y por tanto es \(\gcd(a, b c)\).
    \item
      \(a \mid b c\) significa que existe \(k\) tal que \(b c = k a\).
      Tenemos:
      \begin{align*}
	1
	  &= s a + t b \\
	c
	  &= s a c + t b c \\
	  &= (s c + t k) \cdot a
      \end{align*}
      y esto último dice que \(a \mid c\).
    \item
      Existen \(x, y \in \mathbb{Z}\) tales que \(c = a x = b y\).
      Por la identidad de Bézout
      existen \(u, v \in \mathbb{Z}\) con:
      \begin{align*}
	a u + b v
	  &= 1 \\
	a c u + b c v
	  &= c \\
	a b u y + a b v x
	  &= c \\
	a b (u y + v x)
	  &= c
      \end{align*}
      con lo que \(a b \mid c\).
      \qedhere
    \end{enumerate}
  \end{proof}

  Como ejemplo
  demostramos \(\gcd(x^2, y^2) = \left(\gcd(x, y)\right)^2\).
  Si \(\gcd(x, y) = 1\),
  aplicando la propiedad~(\ref{lem:gcd:producto})
  con \(a = x\), \(b = c = y\)
  tenemos \(\gcd(x, y^2) = 1\).
  Repitiendo esto con \(a = y^2\) y \(b = c = x\)
  resulta \(\gcd(x^2, y^2) = 1\).
  En realidad,
  podemos demostrar de la misma forma que \(\gcd(x^m, y^n) = 1\),
  para \(m \ge 1\) y \(n \ge 1\).
  Ahora,
  por~(\ref{lem:gcd:factor_comun}),
  si \(x = k u\) y \(y = k v\),
  donde \(k = \gcd(u, v)\) tendremos \(\gcd(u, v) = 1\),
  y \(\gcd(x^2, y^2) = \gcd(k^2 u^2, k^2 v^2)
		     = k^2 \cdot \gcd(u^2, v^2)
		     = \left(\gcd(x, y)\right)^2\).
  Queda como ejercicio demostrar de forma similar
  que \(\gcd(x^m, y^m) = \left(\gcd(x, y)\right)^m\).

  El máximo común divisor es muy importante,
  interesa obtener una forma de calcularlo eficientemente.
  De partida,
  sabemos que si \(m = \gcd(a, b)\),
  entonces \(m \mid u \cdot a + v \cdot b\)
  para todo par \(u, v\).
  En particular,
  \(m \mid a \bmod b\),
  ya que \(a \bmod b = a - q \cdot b\).
  Al revés,
  cualquier divisor común de \(a \bmod b\) y \(b\)
  divide a \(a = a \bmod b + q b\) y a \(b\),
  y por lo tanto a su máximo común divisor.
  Así \(\gcd(a, b) = \gcd(b, a \bmod b)\).
  Esto lleva directamente al algoritmo de Euclides~\ref{alg:Euclides}%
    \index{maximo comun divisor@máximo común divisor!algoritmo|see{Euclides, algoritmo de}}%
    \index{Euclides, algoritmo de}%
    \index{algoritmo!Euclides|see{Euclides, algoritmo de}}%
    \index{Euclides}
  (los 13 tomos de los \emph{Elementos} de este alejandrino del siglo III~AC%
    \index{Euclides, Elementos}
   incluyen teoría de números en los tomos 7 a 9).
  El algoritmo de Euclides de interés histórico también,
  es el algoritmo más antiguo que involucra ciclos,
  y fue el primer algoritmo cuyo rendimiento se analizó matemáticamente
  (por Gabriel Lamé en 1844).%
    \index{Lame, Gabriel Leon Jean Baptiste@Lamé, Gabriel Léon Jean Baptiste}
  El análisis lo discutiremos en la sección~\ref{sec:gcd}.
  \begin{algorithm}[htbp]
    \DontPrintSemicolon
    \SetKwFunction{Gcd}{gcd}

    \KwFunction \Gcd{\(a,\; b\)} \;
    \BlankLine
    \While{\(b > 0\)}{
      \((a, \; b) \leftarrow (b, \; a \bmod b)\) \;
    }
    \Return \(a\) \;
    \caption{Algoritmo de Euclides para calcular $\gcd(a, b)$}
    \label{alg:Euclides}
  \end{algorithm}

  Una función íntimamente relacionada con el máximo común divisor
  es el mínimo común múltiplo,
  que anotaremos \(\lcm(a, b)\)%
    \index{minimo comun multiplo@mínimo común múltiplo}%
    \index{LCM@\texttt{LCM}|see{mínimo común múltiplo}}
  (por \emph{\foreignlanguage{english}{least common multiple}} en inglés).
  Sea \(m\) un múltiplo común de \(a\) y \(b\),
  vale decir \(m = h a = k b\).
  Sean \(a = a_1 \gcd(a, b)\) y \(b = b_1 \gcd(a, b)\),
  dividiendo la relación para \(m\) por \(\gcd(a, b)\)
  resulta \(h a_1 = k b_1\).
  Como por el lema~\ref{lem:gcd} parte~\ref{lem:gcd:dividir_gcd}
  es \(\gcd(a_1, b_1) = 1\),
  por la parte~\ref{lem:gcd:divisor} debe ser  \(a_1 \mid k\),
  y obtenemos el mínimo cuando \(a_1 = k\),
  vale decir:
  \begin{equation}
    \label{eq:compute-lcm}
    \lcm(a, b) = \frac{\lvert a b  \rvert}{\gcd(a, b)}
  \end{equation}

  Una aplicación simple de lo anterior es el siguiente teorema:
  \begin{theorem}[Criterio de cero racional]
    \index{polinomio!criterio de cero racional}
    \label{theo:polinomial-rational-roots}
    Sea \(p(x) = a_n x^n + \dotsb + a_0\)
    un polinomio de coeficientes enteros.
    Todo cero racional \(r = u / v\) de \(p(x)\)
    expresada en mínimos términos cumple
    \(u \mid a_0\) y \(v \mid a_n\).
  \end{theorem}
  \begin{proof}
    Sin pérdida de generalidad
    podemos suponer que los \(a_i\) no tienen factores en común,
    que \(a_n \ne 0\) y que \(a_0 \ne 0\).
    Substituyendo \(u / v\) en \(p(x) = 0\),
    y multiplicando por \(v^n\),
    resulta:
    \begin{equation*}
      a_n u^n + a_{n - 1} u^{n - 1} v + \dotsb a_1 u v^{n - 1} + a_0 v^n
	= 0
    \end{equation*}
    Observamos que todos los términos,
    salvo posiblemente el primero,
    son divisibles por \(v\).
    En consecuencia,
    como la expresión completa es divisible por \(v\),
    tiene que serlo el primer término,
    o sea \(v \mid a_n\),
    ya que supusimos que \(u\) y \(v\) no tienen factores en común.
    Asimismo,
    todos los términos,
    salvo posiblemente el último,
    son divisibles por \(u\).
    Por el mismo razonamiento anterior,
    la única forma que esto se puede cumplir es que
    \(u \mid a_0\).
  \end{proof}
  El teorema~\ref{theo:polinomial-rational-roots}
  restringe los posibles ceros racionales del polinomio a un número finito.
  En el caso de polinomios mónicos
    \index{polinomio!monico@mónico}
  (cuyo coeficiente del término de mayor grado es uno)
  vemos que los ceros son enteros
  de ser racionales.
  Esto da una manera adicional de demostrar que \(\sqrt{2}\) es irracional:%
   \index{numero@número!irracional!\(\sqrt{2}\)}
  Es cero de \(x^2 - 2\),
  como \(\sqrt{2}\) no es entero,
  es irracional.
  Incluso más,
  toda raíz de un entero o es entera o es irracional,%
    \index{numero@número!irracional}
  por un razonamiento similar.

\subsection{Obtener los coeficientes de Bézout}
\label{sec:coeficientes-Bezout}

  Sabemos por la identidad de Bézout que existen enteros \(s, t\)
  tales que:%
    \index{Bezout, identidad de@Bézout, identidad de}
  \begin{equation*}
    \gcd(a, b) = s a + t b
  \end{equation*}
  y encontrar estos es de interés también.
  Una manera de proceder es ir registrando paso a paso los valores
  en el algoritmo de Euclides,%
    \index{Euclides, algoritmo de}
  y luego ir reemplazando en reversa.
  Por ejemplo,
  para calcular \(\gcd(40\,902, 24\,140)\) hacemos:

  \begin{center}
    \begin{tabular}[htbp]{>{\(}r<{\)}
			  @{${} = {}$}>{\(}r<{\)}
			     @{${} \cdot {}$}>{\(}r<{\)}
			     @{${} + {}$}>{\(}r<{\)}}
      40\,902 & 24\,140 & 1 & 16\,762 \\
      24\,140 & 16\,762 & 1 &  7\,378 \\
      16\,762 &	 7\,378 & 2 &  2\,006 \\
       7\,378 &	 2\,006 & 3 &  1\,360 \\
       2\,006 &	 1\,360 & 1 &	  646 \\
       1\,360 &	    646 & 2 &	   68 \\
	  646 &	     68 & 9 &	   34 \\
	   68 &	     34 & 2 &	    0
    \end{tabular}
  \end{center}
  Sabemos entonces que \(\gcd(40\,902, 24\,140) = 34\).
  De lo anterior también tenemos que:
  \begin{center}
    \begin{tabular}[htbp]{>{\(}r<{\)}
			  @{${} = {}$}>{\(}r<{\)}
			     @{${} - {}$}>{\(}r<{\)}
			     @{${} \cdot {}$}>{\(}r<{\)}}
	    34 &      646 &	 68 & 9 \\
	    68 &   1\,360 &	646 & 2 \\
	   646 &   2\,006 &  1\,360 & 1 \\
	1\,360 &   7\,378 &  2\,006 & 3 \\
	2\,006 &  16\,762 &  7\,378 & 2 \\
	7\,378 &  24\,140 & 16\,762 & 1 \\
       16\,762 &  40\,902 & 24\,140 & 1
    \end{tabular}
  \end{center}
  Substituyendo las expresiones para los restos
  finalmente obtenemos \(34 = 337 \cdot 40\,902 - 571 \cdot 24\,140\).

  Una manera de organizar mejor el trabajo es el algoritmo~\ref{alg:xgcd},
  debido a Blankinship~%
    \cite{blankinship63:_new_version_euclid_algor}.%
    \index{Euclides, algoritmo de!extendido}%
    \index{Blankinship, algoritmo de|see{Euclides, algoritmo de!extendido}}%
    \index{Blankinship, W. A.}
  Usa vectores auxiliares
  \((x_1, x_2, x_3)\), \((y_1, y_2, y_3)\) y \((t_1, t_2, t_3)\),
  que manipula
  de forma que siempre se cumple que:
  \begin{align*}
    x_1 a + x_2 b &= x_3 \\
    y_1 a + y_2 b &= y_3 \\
    t_1 a + t_2 b &= t_3
  \end{align*}
  \begin{algorithm}[htbp]
    \DontPrintSemicolon

    \KwFunction \FuncSty{xgcd(}\ArgSty{\(a, \; b\)}\FuncSty{)} \;
    \BlankLine
    \((x_1, x_2, x_3) \leftarrow (1, 0, a)\) \;
    \((y_1, y_2, y_3) \leftarrow (0, 1, b)\) \;
    \While{\(y_3 \ne 0\)}{
       \(q \leftarrow \lfloor x_3 / y_3 \rfloor\) \;
       \((t_1, t_2, t_3)
	    \leftarrow (x_1, x_2, x_3) - q \cdot (y_1, y_2, y_3)\) \;
       \((x_1, x_2, x_3) \leftarrow (y_1, y_2, y_3)\) \;
       \((y_1, y_2, y_3) \leftarrow (t_1, t_2, t_3)\) \;
    }
    \Return \(x_3 = x_1 \cdot a + x_2 \cdot b\) \;
    \caption{Algoritmo extendido de Euclides}
    \label{alg:xgcd}
  \end{algorithm}
  El algoritmo~\ref{alg:xgcd} es exactamente el mismo
  que el algoritmo~\ref{alg:Euclides}
  respecto de la manipulación de \(x_3\) e \(y_3\),
  con lo que calcula \(\gcd(a, b)\) correctamente;
  por la relación que se mantiene
  entre \(a\), \(b\), \(x_1\), \(x_2\) y \(x_3\)
  obtenemos los coeficientes de Bézout.

  Podemos eliminar buena parte de la computación
  del algoritmo~\ref{alg:xgcd}
  si omitimos \(x_2\), \(y_2\) y \(t_2\),
  y obtenemos \(x_2\) de la relación \(x_1 a + x_2 b = x_3\)
  al final.
  Manejar los casos de \(a\) o \(b\) negativos queda de ejercicio.

  La traza del algoritmo~\ref{alg:xgcd}
  para \(\gcd(40\,902, 24\,140)\)
  da el cuadro~\ref{tab:traza-xgcd}.
  \begin{table}[htbp]
    \centering
    \begin{tabular}{|>{\(}r<{\)}|*{6}{>{\(}r<{\)}|}}
      \hline
      \multicolumn{1}{|c|}{\rule[-0.7ex]{0pt}{3ex}\(\boldsymbol{x_1}\)} &
	\multicolumn{1}{|c|}{\(\boldsymbol{x_2}\)} &
	\multicolumn{1}{c|}{\(\boldsymbol{x_3}\)} &
	\multicolumn{1}{c|}{\(\boldsymbol{y_1}\)} &
	\multicolumn{1}{c|}{\(\boldsymbol{y_2}\)} &
	\multicolumn{1}{c|}{\(\boldsymbol{y_3}\)} &
	\multicolumn{1}{c|}{\(\boldsymbol{q}\)} \\
      \hline\rule[-0.7ex]{0pt}{3ex}%
	1  &	 0  & 40\,902  &     0	&     1	 & 24\,140  &	  1 \\
	0  &	 1  & 24\,140  &     1	&    -1	 & 16\,762  &	  1 \\
	1  &	-1  & 16\,762  &    -1	&     2	 &  7\,378  &	  2 \\
       -1  &	 2  &  7\,378  &     3	&    -5	 &  2\,006  &	  3 \\
	3  &	-5  &  2\,006  &   -10	&    17	 &  1\,360  &	  1 \\
      -10  &	17  &  1\,360  &    13	&   -22	 &     646  &	  2 \\
       13  &   -22  &	  646  &   -36	&    61	 &	68  &	  9 \\
      -36  &	61  &	   68  &   337	&  -571	 &	34  &	  2 \\
      337  &  -571  &	   34  &  -710	&  1203	 &	 0  &	    \\
      \hline
    \end{tabular}
    \caption{Traza del algoritmo extendido de Euclides}
    \label{tab:traza-xgcd}
  \end{table}
  nuevamente:
  \begin{equation*}
    \gcd(40\,902, 24\,140) = 34 = 337 \cdot 40\,902 - 571 \cdot 24\,140
  \end{equation*}

\subsection{Números primos}
\label{sec:primos}

  \begin{definition}
    \index{numero@número!primo|textbfhy}
    \label{def:Z:prime}
    Un entero positivo \(p \ge 2\) se llama \emph{primo}
    si siempre que \(p \mid a b\)
    es \(p \mid a\) o \(p \mid b\).
  \end{definition}
  Esto lo usamos al demostrar antes
  que \(\sqrt{2}\) es irracional
  (teorema~\ref{theo:sqrt2-irracional}):%
    \index{numero@número!irracional!\(\sqrt{2}\)}
  Si \(2 \mid a^2\),
  debe ser \(2 \mid a\) ya que \(2\) es primo.

  La definición más tradicional es que \(p\) es primo
  si sus únicos divisores positivos son \(1\) y \(p\):
  \begin{definition}
    \index{irreductible}
    \label{def:Z:irreducible}
    A un entero \(e\) se le llama \emph{irreductible}
    si siempre que \(e = a b\) es \(a = \pm 1\) o \(b = \pm 1\).
  \end{definition}
  Notar que si \(p\) es primo,
  es irreductible:
  Si escribimos \(p = a b\) por la definición de primo
  \(p\) divide \(a\) o \(b\).
  Sin pérdida de generalidad,
  supongamos que divide a \(b\),
  de forma que \(b = p c\).
  Entonces \(p = a c p\),
  y \(a c = 1\).
  O sea,
  \(a = \pm 1\).

  Y resulta:
  \begin{theorem}
    \label{theo:Z:irreductible=>prime}
    Si \(p\) es irreductible,
    \(p\) es primo.
  \end{theorem}
  \begin{proof}
    Por contradicción.%
      \index{demostracion@demostración!contradiccion@contradicción}
    Supongamos \(p\) primo y reductible,
    con lo que \(p = a b\)
    para \(a \ne \pm 1\) y \(b \ne \pm 1\).
    Así \(p \mid a b\),
    con lo que por definición de primo \(p \mid a\) o \(p \mid b\).
    Si \(p \mid a\),
    es \(a = m p\) para un entero \(m\).
    Pero entonces:
    \begin{equation*}
      p = a b = m p b
    \end{equation*}
    con lo que:
    \begin{equation*}
      p (1 - m b) = 1
    \end{equation*}
    O sea,
    \(p \mid 1\),
    lo que es absurdo.
  \end{proof}

  Todo número natural se puede escribir como un producto de primos.
  Para demostrarlo usamos un truco bastante común:
  Nos fijamos en el mínimo supuesto contraejemplo
  y usamos contradicción.
  \begin{theorem}
    \label{theo:natural=producto-primos}
    Todo número natural se puede expresar como producto de números primos.
  \end{theorem}
  \begin{proof}
    Por convención,
    \(1\) es el producto de cero primos.

    Para números mayores a \(1\),
    la demostración es por contradicción.
    Llamémosle \(m\) al mínimo número que no es un producto de primos.
    Entonces \(m\) no puede ser primo,
    ya que de serlo sería el producto de primos
    (uno solo);
    así podemos escribir \(m = a \cdot b\),
    donde \(1 < a, b < m\)
    y por tanto \(a\) y \(b\) son productos de primos.
    Pero entonces podemos escribir \(m\) como producto de primos,
    contrario a nuestra suposición de que tal cosa no era posible.
  \end{proof}

  Nuestro siguiente objetivo es demostrar
  que la factorización en primos es única.
  Un paso clave en esa dirección es:
  \begin{lemma}
    \label{lem:p-product}
    Sea \(p\) primo y \(x_1, x_2, \dotsc, x_n\) enteros
    tales que \(p \mid x_1 x_2 \dotsm x_n\).
    Entonces \(p \mid x_i\) para algún \(i\).
  \end{lemma}
  \begin{proof}
    Usamos inducción.%
      \index{induccion@inducción}
    \begin{description}
    \item[Base:]
      Cuando \(n = 1\),
      el supuesto se reduce a \(p \mid x_1\),
      y el resultado es inmediato.
      Por la definición de primo,
      para  \(n = 2\)
      si \(p \mid x_1 x_2\),
      divide a uno de los dos.
    \item[Inducción:]
      Supongamos que el resultado es cierto para \(n\),
      y queremos demostrar que es válido para \(n + 1\).
      Tenemos \(p \mid x_1 \dotsm x_n \cdot x_{n + 1}\).
      Por la definición de primo,
      es \(p \mid x_1 \dotsm x_n\)
      o \(p \mid x_{n + 1}\).
      En el primer caso (por inducción) \(p \mid x_i\)
      para algún \(1 \le i \le n\),
      en el segundo \(p \mid x_{n + 1}\).
      En resumen,
      \(p \mid x_i\) para algún \(1 \le i \le n + 1\).
      \qedhere
    \end{description}
  \end{proof}

  \begin{theorem}[Teorema fundamental de la aritmética]
    \index{fundamental de la aritmetica, teorema@fundamental de la aritmética, teorema|textbfhy}
    \label{theo:fundamental-aritmetica}
    Todo entero positivo
    tiene una factorización única en números primos,
    salvo el orden de los factores.
  \end{theorem}
  \begin{proof}
    El número \(1\) es un caso especial,
    se factoriza en \(0\) primos.

    Para los demás procedemos por contradicción,%
      \index{demostracion@demostración!contradiccion@contradicción}
    aplicando el mismo truco anterior.
    Si hay enteros para los que esto no es cierto,
    hay uno mínimo,
    llamémosle \(N\).
    Vale decir,
    podemos escribir \(N = p_1 p_2 \dotsm p_k\)
    y también \(N = q_1 q_2 \dotsm q_l\),
    donde los \(p_i\) son primos
    (no necesariamente distintos),
    y similarmente los \(q_j\).
    Ahora bien,
    como \(p_1 \mid N\),
    sabemos que \(p_1 \mid q_1 q_2 \dotsm q_l\),
    y por tanto \(p_1 \mid q_j\) para algún \(1 \le j \le l\).
    Como \(q_j\) es primo,
    es irreductible y esto significa que \(p_1 = q_j\).
    Pero entonces tenemos
    \begin{equation*}
      N'
	= p_2 p_3 \dotsm p_k
	= q_1 q_2 \dotsm q_{j - 1} q_{j + 1} \dotsm q_l
    \end{equation*}
    y  \(N' < N\) también tendría dos factorizaciones diferentes,
    contrario a la elección de \(N\) como el mínimo natural
    con esa característica.
  \end{proof}

  Otro hecho fundamental fue demostrado primeramente por Euclides.%
    \index{Euclides}
  De este importante resultado se da una variedad de bonitas demostraciones,
  basadas en conceptos totalmente diferentes,
  en Aigner y Ziegler~\cite{aigner14:_proof_the_book}.
  Nuestra variante de la demostración clásica de Euclides
  se debe a Ernst Kummer.%
    \index{Kummer, Ernst}
  \begin{theorem}
    \label{theo:infinitos-primos}
    Hay infinitos números primos.
  \end{theorem}
  \begin{proof}
    Procedemos por contradicción.%
      \index{demostracion@demostración!contradiccion@contradicción}
    Supongamos que hay un número finito de primos,
    \(p_1, p_2, \dotsc p_r\) en orden creciente,
    donde sabemos que \(r > 1\).
    Consideremos:
    \begin{equation*}
      N = p_1 p_2 \dotsm p_r
    \end{equation*}
    El número \(N - 1\) es compuesto,
    ya que es mayor que \(p_r\) y no es primo
    (no aparece en nuestra lista).
    Luego tiene un factor primo \(p\),
    y este factor lo tiene en común con \(N\).
    Entonces \(p\) divide tanto a \(N\) como a \(N - 1\),
    y divide a su diferencia,
    que es decir \(p \mid 1\),
    lo que es absurdo.
  \end{proof}

  Incluso se puede demostrar más.
  El siguiente resultado se debe a Leonhard Euler,%
    \index{Euler, Leonhard}
  la brillante demostración siguiente es de Paul Erdős~%
    \cite{erdos38:_ueber_reihe_1/p}.%
    \index{Erdos, Paul@Erdős, Paul}
  La importancia del mismo radica en que la divergencia de la serie
  da un indicio de la tasa de crecimiento de los números primos.
  Como dijo el mismo Euler,
  ``hay más primos que cuadrados''
  (por el teorema~\ref{theo:Basilea-converge} la serie de
   recíprocos de los cuadrados converge).%
     \index{Basilea, problema de}
  \begin{theorem}
    \label{theo:sum-1/p}
    La serie
    \begin{equation*}
      \sum_p \frac{1}{p}
    \end{equation*}
    (donde la suma es sobre los números primos)
    diverge.
  \end{theorem}
  \begin{proof}
    Por contradicción.%
      \index{demostracion@demostración!contradiccion@contradicción}
    Enumeremos los primos como \(p_1, p_2, \dotsc\) en orden creciente.
    Si la serie converge,
    hay un punto a partir del cual la suma es menor que \(1/2\),
    con lo que podemos escribir:
    \begin{equation*}
      \sum_{i \ge k + 1} \frac{1}{p_i} < \frac{1}{2}
    \end{equation*}
    Llamaremos \emph{primos chicos} a \(p_1, p_2, \dotsc p_k\)
    y \emph{primos grandes} a \(p_{k + 1}, p_{k + 2}, \dotsc\).
    Tomemos algún \(N > p_k\) a determinar más adelante,
    y llamemos \(N_1\) a la cantidad de números hasta \(N\)
    divisibles solo por primos chicos,
    y similarmente \(N_2\) los que tienen divisores grandes.
    Debe ser \(N = N_1 + N_2\),
    pero dada la suposición de arriba
    veremos que podemos elegir \(N\)
    tal que al acotar \(N_1\) y \(N_2\) de forma suficientemente precisa
    concluimos que \(N_1 + N_2 < N\).

    Como \(\lfloor N / p \rfloor\) de los números entre 1 y \(N\)
    son divisibles por el primo \(p\)
    y \(\lfloor x \rfloor \le x\),
    tenemos que:
    \begin{align}
      N_2
	&\le \sum_{i \ge k + 1} \left\lfloor \frac{N}{p_i} \right\rfloor
	   \notag \\
	&\le \sum_{i \ge k + 1} \frac{N}{p_i}
	   \notag \\
	&=    N \sum_{i \ge k + 1} \frac{1}{p_i}
	   \notag \\
	&<  \frac{N}{2}
	   \label{eq:suma-primos-reciprocos-N2}
    \end{align}
    Consideremos ahora los números menores que \(N\)
    que solo tienen factores primos chicos.
    Si tomamos uno cualquiera de ellos y le llamamos \(x\),
    podemos escribir \mbox{\(x = y \cdot z^2\)},
    donde \(y\) no es divisible por el cuadrado de ningún primo.
    Obviamente \(z^2 \le N\),
    con lo que hay a lo más \(\sqrt{N}\) valores posibles de \(z\).
    Como hay \(k\) primos chicos,
    hay a lo más \(2^k\) posibles valores distintos de \(y\).
    Esto da una sobre estimación bastante burda,
    pero suficiente para nuestras necesidades presentes:
    \begin{equation}
      \label{eq:suma-primos-reciprocos-N1}
      N_1 \le 2^k \cdot \sqrt{N}
    \end{equation}
    Ahora queremos elegir \(N\) de forma que \(N_1 < N / 2\),
    o sea por la cota~\eqref{eq:suma-primos-reciprocos-N1}:
    \begin{align}
      2^k \cdot \sqrt{N}
	&< \frac{N}{2} \notag \\
      2^{2 k + 2}
	&< N
      \label{eq:suma-primos-reciprocos-N}
    \end{align}
    Combinando~\eqref{eq:suma-primos-reciprocos-N}
    con nuestra estimación~\eqref{eq:suma-primos-reciprocos-N2},
    para el valor elegido de \(N\) tenemos:
    \begin{equation*}
      N
	= N_1 + N_2
	  < \frac{N}{2} + \frac{N}{2}
	    = N
    \end{equation*}
    Esto es ridículo.
  \end{proof}
  Nuevamente,
  si la serie diverge no puede tener finitos términos
  y hay infinitos números primos.

\section{Congruencias}
\label{sec:congruencias}

  El concepto de \emph{congruencia} está íntimamente relacionado
  con el resto de la división,
  e incluso usan notaciones similares.
  De todas formas es importante distinguirlos.
  \begin{definition}
    \index{congruencia|textbfhy}
    \label{def:Z:congruence}
    Sean \(a, b, m \in \mathbb{Z}\),
    con \(m \ne 0\).
    Definimos:
    \begin{equation*}
      a \equiv b \pmod{m}
    \end{equation*}
    si \(m \mid a - b\).
    Esto se expresa diciendo
    que \emph{\(a\) es congruente con \(b\) módulo \(m\)}.
  \end{definition}
  Esta es una relación de equivalencia:%
    \index{relacion@relación!equivalencia}
  \begin{description}
  \item[Reflexiva:]
    Si \(a = b\),
    la definición se reduce a \(m \mid 0\),
    lo que siempre es cierto.
  \item[Simétrica:]
    \(a \equiv b \pmod{m}\) significa que \(m \mid a - b\),
    pero entonces \(m \mid b - a\),
    que es decir \(b \equiv a \pmod{m}\).
  \item[Transitiva:]
    \(a \equiv b \pmod{m}\) y \(b \equiv c \pmod{m}\)
    significan \(m \mid a - b\) y \(m \mid b - c\),
    que es \(a - b = k_1 m\) y \(b - c = k_2 m\);
    pero entonces \(a - c = (k_1 + k_2) m\),
    o sea \(m \mid a - c\),
    que es decir \(a \equiv c \pmod{m}\).
  \end{description}
  Los siguientes teoremas dan algunas propiedades importantes.

  \begin{theorem}
    \index{congruencia!propiedades}
    \label{theo:+*mod}
    Sea \(m\) un entero positivo
    y sean \(x_1, x_2, y_1, y_2\) enteros tales que:
    \begin{align*}
      x_1
	&\equiv x_2 \pmod{m} \\
      y_1
	&\equiv y_2 \pmod{m}
    \end{align*}
    Entonces:
    \begin{align*}
      x_1 + y_1
	&\equiv x_2 + y_2 \pmod{m} \\
      x_1 \cdot y_1
	&\equiv x_2 \cdot y_2 \pmod{m} \\
    \end{align*}
  \end{theorem}
  \begin{proof}
    Nos dieron que
    \(x_1 - x_2 = m a\),
    \(y_1 - y_2 = m b\),
    para algunos \(x, y \in \mathbb{Z}\).
    Entonces para la suma
    \begin{align*}
      (x_1 + y_1) - (x_2 + y_2)
	&= (x_1 - x_2) + (y_1 - y_2) \\
	&= m a + m b \\
	&= m (a + b)
    \end{align*}
    lo que es decir
    \begin{equation*}
      x_1 + y_1
	\equiv x_2 + y_2 \pmod{m}
    \end{equation*}

    Similarmente,
    para el producto:
    \begin{align*}
      x_1 y_1 - x_2 y_2
	&= x_1 y_1 - x_2 y_1 + x_2 y_1 - x_2 y_2 \\
	&= (x_1 - x_2) y_1 + x_2 (y_1 -y_2) \\
	&= m a y_1 + x_2 m b \\
	&= m (a y_1 + b x_2)
    \end{align*}
    con lo que
    \begin{equation*}
      x_1 \cdot y_1 \equiv x_2 \cdot y_2 \pmod{m}
      \qedhere
    \end{equation*}
  \end{proof}

  \begin{theorem}
    \index{congruencia!propiedades}
    \label{theo:congruencia-mn}
    Sean \(m\), \(n\) enteros positivos.
    Entonces,
    si
    \begin{equation*}
      a \equiv b \pmod{m n}
    \end{equation*}
    tenemos
    \begin{align*}
      a &\equiv b \pmod{m} \\
      a &\equiv b \pmod{n}
    \end{align*}
    Además,
    si
    \begin{align*}
      a &\equiv b \pmod{m} \\
      a &\equiv b \pmod{n}
    \end{align*}
    entonces
    \begin{equation*}
      a \equiv b \pmod{\lcm(m, n)}
    \end{equation*}
    En particular,
    si \(\gcd(m, n) = 1\),
    entonces:
    \begin{equation*}
      a \equiv b \pmod{m n}
    \end{equation*}
  \end{theorem}
  \begin{proof}
    Para la primera aseveración,
    tenemos por definición que \(m n \mid a - b\),
    pero en tal caso \(m \mid a - b\)
    y también \(n \mid a - b\).

    Para la segunda,
    tenemos que:
    \begin{align*}
      m &\mid a - b \\
      n &\mid a - b
    \end{align*}
    Como tanto \(m\) y \(n\) dividen a \(a - b\),
    el mínimo común múltiplo lo divide:
    \begin{equation*}
      \lcm(m, n) \mid a - b
    \end{equation*}
    y obtenemos nuestro resultado.
  \end{proof}

% Fixme: ¿Más propiedades de congruencias? ~~-> TAoCP

\section[Aritmética \texorpdfstring{en $\mathbb{Z}_m$}{con congruencias}]
	{\protect\boldmath
	   Aritmética \texorpdfstring{en $\mathbb{Z}_m$}{con congruencias}%
       \protect\unboldmath}
\label{sec:aritmetica-Zm}

  Para cualquier entero \(x\) y un entero positivo \(m\)
  anotaremos \([x]_m\) para la clase de equivalencia de \(x\)
  en la relación de congruencia módulo \(m\).
  Vale decir,
  \([x]_m\) es el conjunto de todos los enteros \(x'\)
  tales que \(x - x'\) es múltiplo de \(m\):
  \begin{align*}
    [5]_3
      &= \{\dotsc, -4, -1, 2, 5, 8, \dotsc\} \\
    [-3]_7
      &= \{\dotsc, -10, -3, 4, 11, \dotsc\}
  \end{align*}
  Sabemos que las clases de equivalencia particionan \(\mathbb{Z}\).
  Por ejemplo:
  \begin{equation*}
    \mathbb{Z} = [0]_3 \cup [1]_3 \cup [2]_3
  \end{equation*}
  Para cualquier \(m\) dado,
  las clases son \([0]_m\), \([1]_m\), \ldots, \([m - 1]_m\),
  lo que sigue del algoritmo de división,
  ya que cualquier entero \(x\) puede expresarse como:
  \begin{equation*}
    x = q \cdot m + r
  \end{equation*}
  con \(0 \le r < m\),
  y \(x \in [r]_m\) en tal caso.
  Esto motiva:
  \begin{definition}
    \label{def:Zm}
    Sea \(m\) un entero positivo.
    El \emph{conjunto de enteros módulo \(m\)},
    anotado \(\mathbb{Z}_m\),
    es el conjunto de las clases de equivalencia \([x]_m\).
  \end{definition}
  Vale decir,
  \(\mathbb{Z}_m = \{[0]_m, [1]_m, \dotsc, [m - 1]_m\}\).

  Cabe enfatizar que los elementos de \(\mathbb{Z}_m\)
  son subconjuntos de \(\mathbb{Z}\),
  pero muchas veces resulta conveniente
  considerarlos como los enteros \(0, 1, \dotsc, m - 1\)
  (aunque podríamos elegir otro conjunto de representantes
   si resulta conveniente)
  con una estructura aritmética diferente.
  A los elementos de \(\mathbb{Z}_m\) se les suele llamar \emph{residuos}
  (módulo \(m\)).%
    \index{residuo}

  Definimos las operaciones:
  \begin{align*}
    [x]_m \oplus [y]_m
      &= [x + y]_m \\
    [x]_m \odot	 [y]_m
      &= [x \cdot y]_m
  \end{align*}
  Por nuestro teorema~\ref{theo:+*mod}
  estas operaciones están bien definidas.
  \begin{theorem}
    \label{theo:mod-rules}
    Sea \(m\) un entero positivo,
    y sean \(a, b, c \in \mathbb{Z}_m\) cualquiera.
    Para simplificar anotaremos \(0 = [0]_m\), \(1 = [1]_m\).
    Entonces:
    \begin{enumerate}[label=\textbf{G\arabic{*}:}, ref=G\arabic{*}]
    \item\label{G:asociativa}
      \(a \oplus (b \oplus c) = (a \oplus b) \oplus c\)
    \item\label{G:neutro}
      Hay\/ \(0 \in \mathbb{Z}_m\)
      tal que para todo \(a \in \mathbb{Z}_m\)
      se cumple \(a \oplus 0 = a\)
    \item\label{G:inverso}
      Para todo \(a \in \mathbb{Z}_m\) existe \(-a \in \mathbb{Z}_m\)
      tal que \(a \oplus (-a) = 0\)
    \item\label{G:conmutativa}
      \(a \oplus b = b \oplus a\)
    \end{enumerate}
    \begin{enumerate}[label=\textbf{R\arabic{*}:}, ref=R\arabic{*}]
    \item\label{R:asociativa}
      \((a \odot b) \odot c = a \odot (b \odot c)\)
    \item\label{R:distributiva}
      \(a \odot (b \oplus c) = (a \odot b) \oplus (a \odot c)\)
      y
      \((a \oplus b) \odot c = (a \odot c) \oplus (b \odot c)\)
    \item\label{R:neutro}
      Hay\/ \(1 \in \mathbb{Z}_m\)
      tal que para todo \(a \in \mathbb{Z}_m\)
      se cumple \(a \odot 1 = 1 \odot a = a\)
    \item\label{R:conmutativa}
      \(a \odot b = b \odot a\)
    \end{enumerate}
  \end{theorem}
  Hay una curiosa asimetría entre~\ref{G:neutro} y~\ref{R:neutro},
  generalmente se indica que \(a \oplus 0 = 0 \oplus a = a\).
  Demostraremos esto en el teorema~\ref{theo:group-two-sided-identity}.
  Asimismo,
  suele agregarse que \(a \oplus (-a) = (-a) \oplus a = 0\),
  lo que también sigue de las anteriores
  (teorema~\ref{theo:group-mutuality}).
  \begin{proof}
    Demostraremos algunas cosas,
    dejamos el resto a la imaginación del lector.

    Para \ref{G:conmutativa},
    sean \(a = [x]_m\) y \(b = [y]_m\).
    Entonces:
    \begin{equation*}
      \begin{array}{l@{{}={}}ll}
	a \oplus b
	  & [x]_m \oplus [y]_m & \\
	  & [x + y]_m	       & \text{(definición de \(\oplus\))} \\
	  & [y + x]_m	       & \text{(en \(\mathbb{Z}\))} \\
	  & [y]_m \oplus [x]_m & \text{(definición de \(\oplus\))} \\
	  & b \oplus a
      \end{array}
    \end{equation*}

    Para \ref{G:inverso},
    tomamos \(-[x]_m = [-x]_m\),
    que cumple lo solicitado.
  \end{proof}
  Si un conjunto \(G\) con una operación \(\oplus\) cumple~\ref{G:asociativa}
  a~\ref{G:inverso}
  para algún elemento \(0\)
  se le llama \emph{grupo},%
    \index{grupo|textbfhy}%
    \index{axioma!grupo}
  y se anota \(\langle G, \oplus \rangle\).
  Un grupo que cumple~\ref{G:conmutativa}
  es un \emph{grupo conmutativo} o \emph{abeliano}.%
    \index{grupo!abeliano|textbfhy}%
    \index{grupo!conmutativo|see{grupo!abeliano}}%
    \index{axioma!grupo!abeliano}
  Si es un grupo abeliano con \(\oplus\)
  y hay una segunda operación \(\odot\) y un elemento \(1\)
  que cumple con~\ref{R:asociativa} hasta~\ref{R:neutro}
  se le llama \emph{anillo},%
    \index{anillo|textbfhy}%
    \index{axioma!anillo}
  y se anota \(\langle G, \oplus, \odot \rangle\).
  Si además cumple~\ref{R:conmutativa}
  es un \emph{anillo conmutativo}.%
    \index{anillo!conmutativo|textbfhy}%
    \index{axioma!anillo!conmutativo}
  Nuestro teorema afirma entonces que \(\mathbb{Z}_m\)
  con las operaciones \(\oplus\) y \(\odot\) es un anillo conmutativo.
  Cuidado,
  hay quienes omiten el axioma~\ref{R:neutro} al definir anillos,
  y le llaman \emph{anillos con unidad} a lo que nosotros llamamos anillos.%
    \index{anillo!con unidad}
  En inglés algunos usan el nombre \emph{\foreignlanguage{english}{ring}}%
    \index{ring@\emph{\foreignlanguage{english}{ring}}|see{anillo}}
  para nuestros anillos,
  y usan \emph{\foreignlanguage{english}{rng}}
  (por \emph{\foreignlanguage{english}{ring}} sin identidad).
    \index{rng@\emph{\foreignlanguage{english}{rng}}|see{anillo}}
  Asimismo,
  exigiremos que \(0 \ne 1\),
  para evitar los casos especiales
  que produce un anillo con un único elemento.
  Para abreviar,
  generalmente solo se nombra el conjunto,
  omitiendo las operaciones como obvias.
  Así,
  en vez de hablar del grupo \(\langle \mathbb{Z}_m, \oplus \rangle\)
  o de los anillos \(\langle \mathbb{Z}_m, \oplus, \odot \rangle\)
  o \(\langle \mathbb{C}, +, \cdot \rangle\)
  se habla simplemente de \(\mathbb{Z}_m\) o \(\mathbb{C}\),
  subentendiendo las operaciones tradicionales.
  Esto no debiera llevar a confusión.
  Temas como este son el ámbito del álgebra abstracta,%
    \index{algebra abstracta@álgebra abstracta}
  para profundizar en ellos se recomienda el texto de Pinter~%
    \cite{pinter10:_book_abstr_algeb}
  o el de Judson~\cite{judson14:_abstr_algeb}.

  En el caso de grupos abelianos%
    \index{grupo!abeliano}
  por convención se le llama o anota como ``suma'' a la operación,
  y al neutro se le designa con \(0\).
  Al inverso \(-a\)  de la ``suma''
  se le llama \emph{inverso aditivo} en tal caso.%
    \index{operacion@operación!inverso}
  Escribiremos \(a \ominus b = a \oplus (-b)\) para grupos abelianos.
  Por convención se anota como ``multiplicación''
  la operación en grupos generales,
  al neutro se le llama \(1\)
  y el inverso de \(a\) en tal caso se anota \(a^{-1}\).

  Los elementos que llamamos \(0\) y \(1\) arriba
  (axiomas~\ref{G:neutro} y~\ref{R:neutro})%
    \index{operacion@operación!elemento neutro}
  no son necesariamente los enteros cero y uno,
  simplemente los usamos como nombres convenientes.
  Nótese que en el caso de grupos y anillos las operaciones son esenciales,
  debiera anotarse \((R, +)\) y \((R, +, \cdot)\)
  para explicitar las operaciones.
  Los elementos especiales resultan de las operaciones,
  no vale la pena nombrarlos.
  En anillos se anota \(-a\) para el inverso aditivo de \(a\)
  y \(a^{-1}\) para su inverso multiplicativo
  (si existe).

  Generalmente se indica únicamente el conjunto.

  Al número de elementos de un grupo o de un anillo \(R\)
  se le llama su \emph{orden},%
    \index{grupo!orden|textbfhy}%
    \index{anillo!orden|textbfhy}
  anotado \(\lvert R \rvert\).
  Si un subconjunto del grupo es a su vez un grupo con la operación del caso
  se le llama \emph{subgrupo}.%
    \index{subgrupo|textbfhy}
  De la misma forma podemos definir \emph{subanillos},%
    \index{subanillo|textbfhy}
  claro que insistiremos en que incluyan a \(1\)
  para evitar que el subanillo tenga estructura incompatible con el anillo.
  Por ejemplo,
  en \(\mathbb{Z}_6\),
  el subconjunto \(\{0, 3\}\) es cerrado respecto de suma y multiplicación,
  con neutro multiplicativo \(3\).
  Pero en \(\mathbb{Z}_6\) \(3\) ni siquiera tiene inverso multiplicativo.
  .
  Si \(A\) es subgrupo (o subanillo) de \(B\),
  se anota \(A \le B\).

  Un caso es el conjunto de un único elemento \(e\),
  y definimos \(e = e \circ e\).
  Por razones obvias se le llama el \emph{grupo trivial}.%
    \index{grupo!trivial}

  Otro grupo es el de las operaciones en el cubo de Rubik,%
    \index{Rubik, cubo de (grupo)}
  que intercambian las posiciones de las caras de colores de los cubitos.
  Joyner~\cite{joyner08:_adven_group_theory}
  usa este juguete como excusa para introducir la teoría de grupos,
  y muestra cómo aplicarla para resolver este puzzle
  y otros similares.

  Un ejemplo importante lo proveen
  las simetrías de objetos al moverse en el espacio.%
    \index{grupo!simetrias@simetrías}
  Por ejemplo,
  si consideramos un cuadrado
  \begin{figure}[htbp]
    \centering
    \pgfimage{images/cuadrado}
    \caption{Un cuadrado}
    \label{fig:simetria-cuadrado}
  \end{figure}
  (como el de la figura~\ref{fig:simetria-cuadrado}),
  tenemos las siguientes operaciones
  que lo hacen coincidir con su posición original:
  \begin{itemize}
  \item
    No hacer nada
    (identidad, \(\iota\)).
  \item
    Girar en sentido contra reloj en
    \(\pi / 2\), \(\pi\), \(3 \pi / 2\)
    (\(r_1\), \(r_2\) y	 \(r_3\)).
  \item
    Reflejar verticalmente
    (\(f_v\)),
    horizontalmente (\(f_h\)),
    y a través de las diagonales \(1\;3\) y \(2\;4\)
    (\(f_c\) y \(f_d\)).
  \end{itemize}
  Podemos componer estas operaciones vía efectuar una y luego la otra.%
    \index{grupo!simetrias@simetrías}
  En otras palabras,
  \(b \bullet a\) describe aplicar \(a\) y luego \(b\).
  Como una combinación de operaciones
  vuelve el cuadrado a su posición original
  (aunque intercambiando vértices),
  la composición es cerrada.
  Por ejemplo,
  si rotamos el cuadrado en \(3 \pi / 2\)
  y luego lo reflejamos en la horizontal
  el efecto es el mismo que si lo reflejamos en una diagonal:
  \begin{align*}
    f_h \bullet r_3 = f_d
  \end{align*}
  Esto lleva al cuadro~\ref{tab:D8},
  que describe al grupo conocido como \(\mathtt{D}_8\).
  \begin{table}[htbp]
    \centering
    \begin{tabular}{>{\(}c<{\)}|*{8}{>{\(}c<{\)}}}
      \bullet & \iota & r_1 & r_2 & r_3 & f_v & f_h & f_d & f_c \\
      \hline
	\rule[-0.7ex]{0pt}{3ex}%
      \iota & \iota & r_1   & r_2   & r_3   & f_v   & f_h   & f_d   & f_c \\
      r_1   & r_1   & r_2   & r_3   & \iota & f_v   & f_h   & f_d   & f_c \\
      r_2   & r_2   & r_3   & \iota & r_1   & f_c   & f_d   & f_v   & f_h \\
      r_3   & r_3   & \iota & r_1   & r_2   & f_d   & f_c   & f_h   & f_v \\
      f_v   & f_v   & f_d   & f_h   & f_c   & \iota & r_2   & r_1   & r_3 \\
      f_h   & f_h   & f_c   & f_v   & f_d   & r_2   & \iota & r_3   & r_1 \\
      f_d   & f_d   & f_h   & f_c   & f_v   & r_3   & r_1   & \iota & r_2 \\
      f_c   & f_c   & f_v   & f_d   & f_h   & r_1   & r_3   & r_2   & \iota
    \end{tabular}
    \caption{El grupo $\mathtt{D}_8$}
    \label{tab:D8}
  \end{table}
  Es \(\mathtt{D}\) por \emph{diedro},%
    \index{grupo!dihedral|textbfhy}
  el \(8\) por el número de operaciones del grupo.
  En general,
  el polígono regular de \(n\) lados%
    \index{poligono@polígono!regular}
  da lugar al grupo denominado \(\mathtt{D}_{2 n}\),
  hay \(n\) rotaciones y \(n\) reflexiones
  para un total de \(2 n\) operaciones.
  Estos grupos no son conmutativos.

  Un poco más de la teoría elemental de grupos
  provee Chen~%
    \cite[capítulo 4]{chen08:_misc_topics_first_year_math}.
  En particular,
  construye sistemáticamente todos los grupos de orden hasta \(7\).
  Resultados importantes son los siguientes:
  \begin{theorem}
    \label{theo:group-unique-identity}
    El elemento neutro de un grupo es único.
  \end{theorem}
  \begin{proof}
    Supongamos que en el grupo \((G, \odot)\)
    los elementos \(a\) y \(b\) son ambos neutros.
    Entonces:
    \begin{equation*}
      a = a \odot b = b
      \qedhere
    \end{equation*}
  \end{proof}
  \begin{theorem}[Ley de cancelación]
    \index{grupo!cancelacion@cancelación}
    \label{theo:group-cancellation}
    Si \(a, b, c \in G\) son tales que
    \(a \odot c = b \odot c\),
    entonces \(a = b\).
  \end{theorem}
  \begin{proof}
    Tenemos:
    \begin{equation*}
      a
	= a \odot 1
	= a \odot (c \odot c^{-1})
	= (a \odot c) \odot c^{-1}
	= (b \odot c) \odot c^{-1}
	= b \odot (c \odot c^{-1})
	= b
      \qedhere
    \end{equation*}
  \end{proof}
  Puede aplicarse exactamente la misma técnica
  para demostrar cancelación a la izquierda.
  \begin{theorem}[Mutualidad]
    \index{grupo!mutualidad}
    \label{theo:group-mutuality}
    Si en un grupo \(a \odot b = 1\),
    entonces \(b \odot a = 1\).
  \end{theorem}
  \begin{proof}
    Escribamos:
    \begin{equation*}
      1 \odot b
	= b \odot 1
	= b \odot (a \odot b)
	= (b \odot a) \odot b
    \end{equation*}
    Luego aplicamos la ley de cancelación.
  \end{proof}
  Esto nos dice que \((a^{-1})^{-1} = a\).
  \begin{theorem}
    \label{theo:group-two-sided-identity}
    En un grupo,
    \(1 \odot a = a\)
  \end{theorem}
  \begin{proof}
    Podemos escribir:
    \begin{equation*}
      1 \odot a
	= (a \odot a^{-1}) \odot a
	= a \odot (a^{-1} \odot a)
	= a \odot 1
	= a
      \qedhere
    \end{equation*}
  \end{proof}
  \begin{theorem}
    \label{theo:group-inverse-unique}
    El inverso de un elemento es único.
  \end{theorem}
  \begin{proof}
    Sea \(a \in G\) un elemento cualquiera,
    y supongamos que se cumplen \(a \odot b = 1\) y \(a \odot c = 1\).
    Por cancelación por la izquierda,
    \(b = c\).
    Por mutualidad,
    si \(a \odot b = 1\) entonces \(b \odot a = 1\),
    y \(b = a^{-1}\).
  \end{proof}

  En lo que sigue consideraremos un grupo \(G\) con operación \(\odot\)
  (o simplemente se omite).
  El elemento neutro de \(G\) lo denotaremos \(1\).
  Para simplificar notación,
  en un grupo con operación multiplicación
  usaremos la definición para potencias enteras,%
    \index{grupo!potencia}
  donde \(a\) es un elemento cualquiera del grupo:
  \begin{align}
    a^n
      &= \underbrace{a \odot a \odot \dotsm \odot a}_{\text{\(n\) veces}}
	   \notag \\
  \intertext{Formalmente:}
    a^0
      &= 1 \label{eq:definicion-a^0} \\
    a^{k + 1}
      &= a^k \odot a \qquad \text{si \(k \ge 0\)}
	   \label{eq:definicion-a^k}
  \end{align}
  Es fácil ver que si definimos:
  \begin{equation}
    \label{definicion-a^-k}
    a^{-k}
      = \left( a^{-1} \right)^k
  \end{equation}
  se cumplen las propiedades tradicionales de las potencias:
  \begin{align}
    a^{m + n}
      &= a^m \odot a^n \label{eq:propiedad-potencias-+} \\
    (a^m)^n
      &= a^{m n}       \label{eq:propiedad-potencias-*}
  \end{align}
  Si la operación es suma
  usamos la notación \(n \cdot a\)
  (o simplemente \(n a\))
  con \(n \in \mathbb{Z}\);
  si anotamos \(0\) para el neutro aditivo:
  \begin{align*}
    n \cdot a
      &= \underbrace{a \oplus a \oplus \dotsb \oplus a}_{\text{\(n\) veces}}
	   \notag \\
  \intertext{Formalmente:}
    0 \cdot a
      &= 0 \\
    (k + 1) \cdot a
      &= k \cdot a \oplus a \qquad \text{si \(k \ge 0\)} \\
    -k \cdot a
      &= k \cdot (-a)
  \end{align*}

  Sea \(a\) un elemento de un grupo \(G\),
  en el cual usamos notación de multiplicación.
  Los elementos \(a^k\) con \(k \in \mathbb{Z}\)
  forman un subgrupo abeliano de \(G\).%
    \index{subgrupo!generado}
  Si \(G\) es finito,
  el conjunto de los \(a^n\) para \(n \in \mathbb{N}\)
  tiene que contener repeticiones,
  ya que son todos elementos de \(G\).
  Si resulta que \(a^m = a^n\) con \(m > n\),
  tendremos \(a^{m - n} = 1\).
  El mínimo \(n > 0\) tal que \(a^n = 1\)
  (siempre existe si \(G\) es finito,
   aunque incluso hay grupos infinitos
   todos cuyos elementos son de orden finito)
  se llama el \emph{orden de \(a\)},%
    \index{grupo!orden de un elemento|textbfhy}
  que se anota \(\ord_G(a)\)
  (o simplemente \(\ord(a)\),
   si el grupo se subentiende).
  Más aún,
  si \(a^k = 1\),
  entonces \(\ord_G(a) \mid k\).
  Para demostrar esto,
  usamos el algoritmo de división.%
    \index{algoritmo de division@algoritmo de división}
  Con \(n = \ord_G(a)\)
  podemos escribir \(k = q n + r\),
  donde \(0 \le r < n\),
  y:
  \begin{align}
    a^k
      &= a^{q n + r} \notag \\
    1
      &= a^{q n} \odot a^r \notag \\
      &= (a^n)^q \odot a^r \notag \\
      &= a^r  \label{eq:a^k=r}
  \end{align}
  Como \(0 \le r < n\),
  por la definición de orden la única opción en~\eqref{eq:a^k=r} es \(r = 0\),
  y \(n \mid k\).

  \begin{definition}
    \index{grupo!ciclico@cíclico|textbfhy}
    \label{def:grupo-ciclico}
    Sea \(G\) un grupo.
    Si todos los elementos de \(G\) se pueden escribir como
    \(g^k\) para algún elemento \(g \in G\) y \(k \in \mathbb{Z}\)
    a \(G\) se le llama \emph{grupo cíclico},
    y a \(g\) se le llama
    \emph{generador} del grupo.%
      \index{grupo!generador}
  \end{definition}
  Si \(a\) tiene orden finito \(n\),
  los elementos \(1 = a^0, a, \dotsc, a^{n - 1}\)
  forman un grupo cíclico de orden \(n\),
  llamado \emph{el grupo generado por \(a\)},
  que se anota \(\langle a \rangle\).
  Tenemos \(a^{n - 1} = a^{-1}\),
  lo que completa la condición de subgrupo que dimos antes
  en el teorema~\ref{theo:sugroup}.

  Vamos ahora a anillos,
  que son las estructuras que realmente nos interesan acá.
  \begin{definition}
    \index{anillo!unidad|textbfhy}
    \label{def:ring-unit}
    En un anillo \(R\),
    si para \(r\) hay un \(x\) tal que \(r \odot x = x \odot r = 1\),
    se dice que \(r\) es \emph{invertible},
    que \(x\) es su \emph{inverso}
    y escribimos \(x = r^{-1}\).
    Nótese que también es \(x^{-1} = r\).
    A los elementos invertibles del anillo \(R\)
    se les llama \emph{unidades},
    y su conjunto se denota \(R^\times\).
  \end{definition}
  Notaciones alternativas para el grupo de unidades de \(R\) son
  \(U(R)\), \(R^*\) y \(E(R)\).

  \begin{theorem}
    \index{anillo!grupo de unidades}
    El conjunto de unidades de un anillo forma un grupo con la multiplicación.
  \end{theorem}
  \begin{proof}
    La multiplicación es asociativa en \(R\),
    así que lo es en el subconjunto \(R^\times\).
    Toda unidad tiene inverso por definición,
    y \(1\) siempre es una unidad.
    Si \(a\) y \(b\) son invertibles,
    entonces \(a \odot b\) tiene inverso \(b^{-1} \odot a^{-1}\),
    y la multiplicación es cerrada en \(R^\times\).
    Estas son las propiedades que definen a un grupo.
  \end{proof}

  Un caso especial muy importante es:
  \begin{definition}
    \index{campo (algebra)@campo (álgebra)|textbfhy}
    \index{axioma!campo}
    \label{def:field}
    \glossary{Campo}
	     {Anillo conmutativo en el cual
	      todos los elementos distintos de cero
	      son invertibles.}
    Un anillo conmutativo en el que todos los elementos distintos de \(0\)
    son invertibles se llama \emph{campo}.
  \end{definition}
  \noindent
  Ya habíamos encontrado este concepto al discutir \(\mathbb{R}\)
  en el capítulo~\ref{cha:numeros-reales}.

  Para simplificar la notación usaremos \(x\) con \(0 \le x \le m - 1\)
  para denotar al conjunto \([x]_m\),
  y usaremos \(+\) y \(\cdot\)
  (o simplemente omitiremos la multiplicación)
  en vez de \(\oplus\) y \(\odot\).
  Para eliminar paréntesis,
  usaremos la convención común de ``multiplicaciones antes de sumas''.
  El valor de \(m\) se debe indicar o quedar claro por el contexto.

  Nótese que en \(\mathbb{Z}_m\)
  no hay orden ni idea de ``positivo''.
  Como en \(\mathbb{Z}_m\) es
  \begin{equation*}
    0, 1, \dotsc, m - 1, 0, 1, \dotsc
  \end{equation*}
  (al llegar al final comienza nuevamente)
  a veces se le llama ``aritmética de reloj''.%
    \index{aritmetica de reloj@aritmética de reloj|see{congruencias}}

  Algunas de las propiedades conocidas de \(\mathbb{Z}\)
  no siempre valen en \(\mathbb{Z}_m\).
  Por ejemplo,
  en \(\mathbb{Z}\)
  si \(a \cdot c = b \cdot c\) con \(c \ne 0\)
  entonces \(a = b\).
  Sin embargo,
  en \(\mathbb{Z}_{15}\)
  tenemos
  \begin{equation*}
    4 \cdot 9 =	 14 \cdot 9 = 6
  \end{equation*}
  e incluso
  \begin{equation*}
    12 \cdot 10 = 0
  \end{equation*}
  También pueden haber elementos que tienen más de dos raíces cuadradas.
  Por ejemplo:
  \begin{equation*}
    4 = 2 \cdot 2 = 7 \cdot 7 = 8 \cdot 8 = 13 \cdot 13
  \end{equation*}
  Nótese que módulo \(15\) tenemos \(13 \equiv -2\) y \(8 \equiv -7\),
  con lo que las raíces cuadradas de \(4\) en \(\mathbb{Z}_{15}\)
  son \(\pm 2\) y \(\pm 7\),
  mientras en \(\mathbb{Z}\) solo están \(\pm 2\).
  Algunos elementos tienen inverso multiplicativo
  en \(\mathbb{Z}_{15}\),
  como:
  \begin{equation*}
    7 \cdot 13 = 8 \cdot 2 = 1
  \end{equation*}
  Otros no lo tienen,
  por ejemplo \(6\),
  como es fácil verificar revisando todos los productos \(6 \cdot k\)
  con \(0 \le k < 15\) módulo \(15\).
  Hay elementos con una única raíz cuadrada,
  como \(15 = 15 \cdot 15\) en \(\mathbb{Z}_{30}\),
  que incluso es su propia raíz cuadrada.
  \begin{definition}
    \index{anillo!divisor de cero|textbfhy}
    \label{def:ring-zero-divisor}
    Si en un anillo hay elementos \(a\), \(b\)
    tales que \(a \odot b = 0\)
    se les llama \emph{divisor izquierdo de cero} a \(a\)
    y \emph{divisor derecho de cero} a \(b\).
    A ambos se les llama \emph{divisores de cero}.
  \end{definition}
  \noindent
  Nótese que \(0\) siempre es un divisor de cero.

  Casos muy importantes de anillos son:
  \begin{definition}
    \index{dominio integral|textbfhy}
    \glossary{Dominio integral}
	     {Un anillo conmutativo sin divisores propios de cero.}
    \label{def:ID}
    Un anillo conmutativo en el cual sólo \(0\) es divisor de cero
    se llama \emph{dominio integral}.
  \end{definition}
  \noindent
  El ejemplo clásico de dominio integral es \(\mathbb{Z}\).
  En un dominio integral podemos cancelar en productos:
  \begin{theorem}
    \label{theo:ID-*-cancellation}
    En un dominio integral,
    si \(a x = b x\) con \(x \ne 0\)
    entonces \(a = b\).
  \end{theorem}
  \begin{proof}
    Tenemos:
    \begin{align*}
      a x
	&= b x \\
      a x - b x
	&= 0 \\
      (a - b) x
	&= 0
    \end{align*}
    Como \(x \ne 0\) es \(a - b = 0\),
    o sea \(a = b\).
  \end{proof}

  Veamos criterios para identificar subgrupos.
  \begin{theorem}
    \index{grupo!subgrupo}
    \label{theo:sugroup}
    Sea \(G\) un grupo,
    y \(H \subset G\).
    Si para todo \(a, b \in H\) es \(a \odot b^{-1} \in H\),
    \(H\) es un subgrupo de \(G\).
  \end{theorem}
  \begin{proof}
    Que \(H\) sea subgrupo de \(G\) significa simplemente
    que ese subconjunto es cerrado respecto de la operación e inversos
    (las demás propiedades se ``heredan'' del grupo),
    y que contiene al neutro \(e\).

    Si \(a \in H\),
    por hipótesis está \(a \odot a^{-1} = e\).
    Pero así también está \(e \odot a^{-1} = a^{-1}\).
    Con esto,
    si \(a, b \in H\),
    tenemos que \(a \odot (b^{-1})^{-1} = a \odot b \in H\),
    completando la demostración.
  \end{proof}
  Por ejemplo,
  podemos ver que en el grupo \(D_8\) mostrado en el cuadro~\ref{tab:D8}
  los elementos \(\{\iota, r_1, r_2, r_3\}\) forman un subgrupo.
  \begin{lemma}
    \label{lem:intersection-subgroups}
    Sean \(A\) y \(B\) subgrupos de \(G\).
    Entonces \(A \cap B\) es un subgrupo de \(G\).
  \end{lemma}
  \begin{proof}
    Supongamos \(x, y \in A \cap B\).
    Entonces \(y^{-1} \in A\) y también \(y^{-1} \in B\),
    por ser grupos.
    Pero entonces \(y^{-1} \in A \cap B\).
    De forma similar,
    \(x \cdot y^{-1} \in A \cap B\),
    y por el teorema~\ref{theo:sugroup} \(A \cap B\) es un subgrupo de \(G\).
  \end{proof}
  También para anillos:
  \begin{lemma}
    \label{lem:intersection-subrings}
    Sean \(A\) y \(B\) subanillos de \(R\).
    Entonces \(A \cap B\) es un subanillo de \(R\).
  \end{lemma}
  La demostración se omite,
  sigue la misma idea que la del lema~\ref{lem:intersection-subgroups}.

  Los axiomas de anillo tienen algunas consecuencias simples.
  \begin{theorem}
    \label{theo:ring-01}
    En un anillo,
    \(0\) y \(1\) son únicos,
    y para cada \(a\) hay un único \(-a\).
  \end{theorem}
  \begin{proof}
    Para la suma,
    el que \(0\) es único
    no es más que el teorema~\ref{theo:group-unique-identity};
    la misma técnica de la demostración puede aplicarse
    a \(1\) y la multiplicación.

    Que hay un único \(-a\) es el teorema~\ref{theo:group-inverse-unique}
    aplicado a la suma.
  \end{proof}

  Para evitar paréntesis,
  usaremos la convención tradicional
  que las operaciones son asociativas izquierdas,
  y que \(\odot\) tiene mayor precedencia que \(\oplus\).
  A veces anotaremos \(a \ominus b\) para \(a \oplus (-b)\).

  Otras consecuencias simples son:
  \begin{theorem}
    \label{theo:ring-0a}
    En un anillo,
    para todo \(a\) tenemos
    \(0 \odot a = a \odot 0 = 0\)
  \end{theorem}
  \begin{proof}
    Es aplicar los axiomas de anillo.
    Para \(0 \odot a\):
    \begin{align*}
      0 \odot a
	&= (0 \oplus 0) \odot a \\
	&= (0 \odot a) \oplus (0 \odot a)
    \end{align*}
    Sumando \(-(0 \odot a)\) a ambos lados obtenemos la conclusión buscada.
    La otra parte se demuestra de forma similar.
  \end{proof}
  De acá,
  si \(a\) es una unidad del anillo,
  entonces no es un divisor de cero,
  ya que si:
  \begin{align*}
    a \odot b
      &= 0 \\
    a^{-1} \odot (a \odot b)
      &= (a^{-1} \odot a) \odot b \\
      &= 1 \odot b \\
      &= b \\
  \intertext{Por el otro lado:}
    a^{-1} \odot 0
      &= 0
  \end{align*}
  O sea,
  \(b = 0\),
  y \(a\) no es divisor de cero.
  \begin{theorem}
    \label{theo:ring-(a*-b)}
    En un anillo,
    tenemos \((-a) \odot b = a \odot (-b) = -(a \odot b)\).
  \end{theorem}
  \begin{proof}
    Primero demostramos \((-1) \odot a = a \odot (-1) = -a\).
    \begin{align*}
      (-1) \odot a \oplus a
	&= (-1) \odot a \oplus 1 \odot a \\
	&= ((-1) \oplus 1) \odot a \\
	&= 0 \odot a \\
	&= 0
    \end{align*}
    Por conmutatividad de la suma en un anillo
    \(a \oplus ((-1) \odot a) = 0\),
    y en conjunto estas dos corresponden precisamente
    a la definición de \(-a\) en un grupo.
    La otra parte se demuestra de forma afín.

    Usando esto:
    \begin{align*}
      (-a) \odot b
	&= ((-1) \odot a) \odot b \\
	&= (-1) \odot (a \odot b) \\
	&= -(a \odot b)
    \end{align*}
    La otra parte se demuestra de la misma forma.
  \end{proof}
  Tenemos también:
  \begin{theorem}
    \label{theo:finite-ring-units}
    En un anillo finito \(R\),
    los elementos son unidades o divisores de cero.
  \end{theorem}
  \begin{proof}
    Tomemos \(a \ne 0\) del anillo,
    y consideremos el conjunto \(a R = \{a \odot x \colon x \in R\}\).
    Supongamos que hay elementos repetidos en \(a R\),
    digamos \(a \odot x_1 = a \odot x_2\) con \(x_1 \ne x_2\).
    Entonces:
    \begin{align*}
      a \odot x_1
	&= a \odot x_2 \\
      a \odot x_1 \ominus a \odot x_2
	&= 0 \\
      a \odot (x_1 \ominus x_2)
	&= 0
    \end{align*}
    Como \(x_1 \ominus x_2 \ne 0\),
    \(a\) es un divisor de cero.

    Si no hay elementos repetidos en \(a R\),
    como hay \(\lvert R \rvert\) elementos en ese conjunto
    tiene que estar 1,
    digamos \(a b = 1\).
    Entonces:
    \begin{align*}
      a \odot b
	&= 1 \\
      (a \odot b) \odot a
	&= a \\
      a \odot (b \odot a) \ominus a
	&= 0 \\
      a \odot ((b \odot a) \ominus 1)
	&= 0
    \end{align*}
    Sabemos que \(a \odot 0 = 0\),
    como en \(a R\) no hay elementos repetidos es:
    \begin{align*}
      b \odot a \ominus 1
	&= 0 \\
      b \odot a
	&= 1
    \end{align*}
    y \(b\) es el inverso de \(a\),
    \(a\) es un una unidad.
  \end{proof}
  Una consecuencia inmediata del teorema~\ref{theo:finite-ring-units}
  es:
  \begin{corollary}
    \label{cor:finite-ID=field}
    Todo dominio integral finito es un campo.
  \end{corollary}

  Ejemplos adicionales de anillo los ponen los enteros \(\mathbb{Z}\),%
    \index{Z (numeros enteros)@\(\mathbb{Z}\) (números enteros)}%
    \index{anillo}
  los complejos \(\mathbb{C}\)
    \index{C (numeros complejos)@\(\mathbb{C}\) (números complejos)}
  y matrices cuadradas sobre \(\mathbb{R}\).
    \index{anillo!matrices}

% curvas-elipticas.tex
%
% Copyright (c) 2012-2014 Horst H. von Brand
% Derechos reservados. Vea COPYRIGHT para detalles

\subsection{Curvas elípticas}
\label{sec:curvas-elipticas}
\index{curva eliptica@curva elíptica|textbfhy}

  Una \emph{curva elíptica}
  está definida por una ecuación de la forma:
  \begin{equation*}
    \label{eq:elliptic-curve}
    y^2
      = x^3 + a x + b
  \end{equation*}
  \begin{figure}
    \centering
    \subfloat[\(y^2 = x^3 - x\)]{
      \pgfimage{images/curva-eliptica-1}
    }%
    \hspace{1em}%
    \subfloat[\(y^2 = x^3 - x + 1\)]{
      \pgfimage{images/curva-eliptica-2}
    }
    \caption{Curvas elípticas}
    \label{fig:curvas-elipticas}
  \end{figure}
  que no tiene puntos aislados,
  no se intersecta a sí misma y no tiene cuernos.
  Algebraicamente,
  el discriminante \(\Delta = - 16 (4 a^3 + 27 b^2) \ne 0\).%
    \index{discriminante}
  El gráfico de la curva tiene dos componentes si \(\Delta > 0\)
  y uno solo si \(\Delta < 0\),
  ver la figura~\ref{fig:curvas-elipticas}.
  El nombre no tiene relación con la forma de la curva,
  sino con el hecho que se requieren funciones elípticas
  para representarlas paramétricamente.

  \begin{figure}
    \centering
    \subfloat[Suma de \(\mathtt{P}_1 = (x_1, y_1)\)
	      y \(\mathtt{P}_2 = (x_2, y_2)\)]{
      \pgfimage{images/suma-curva-eliptica}
      \label{subfig:suma-curva-eliptica}
    }%
    \hspace{1em}%
    \subfloat[Doble de \(\mathtt{P} = (x, y)\)]{
      \pgfimage{images/doble-curva-eliptica}
      \label{subfig:doble-curva-eliptica}
    }
    \caption{Sumas en curvas elípticas}
    \label{fig:sumas-curva-eliptica}
  \end{figure}
  Dados dos puntos \(\mathtt{P}_1 = (x_1, y_1)\)
  y \(\mathtt{P}_2 = (x_2, y_2)\) sobre una curva elíptica
  podemos definir la suma
  como el punto donde la recta entre los puntos corta la curva
  reflejado en el eje \(x\),
  véase~\ref{subfig:suma-curva-eliptica} para un ejemplo.
  Esto hace que si \(\mathtt{P}_1\),
  \(\mathtt{P}_2\) y \(\mathtt{P}_3\)
  son puntos sobre la curva,
  es \(\mathtt{P}_1 + \mathtt{P}_2 + \mathtt{P}_3 = 0\),
  donde 0 es el punto en el infinito.
  En caso que \(x_1 = x_2\)
  hay dos posibilidades:
  Si \(y_1 = - y_2\),
  (incluyendo el caso en que los puntos coinciden),
  la suma se define como \(0\)
  (el punto en el infinito).
  Tenemos así para \(\mathtt{P} = (x, y)\)
  que \(- \mathtt{P} = (x, -y)\).
  En caso contrario
  definimos \(\mathtt{P}_1 + \mathtt{P}_2 = \mathtt{P}_3\)
  con \(\mathtt{P}_3 = (x_3, y_3)\)
  mediante:
  \begin{align}
    s
      &= \frac{y_2 - y_1}{x_2 - x_1} \notag \\
    x_3
      &= s^2 - x_1 - x_2 \label{eq:suma-curva-eliptica} \\
    y_3
      &= y_1 + s (x_3 - x_1) \notag
  \end{align}
  Para sumar el punto \(\mathtt{P} = (x, y)\) consigo mismo
  corresponde usar la tangente a la curva,
  ver la figura~\ref{subfig:doble-curva-eliptica},
  lo que da \(\mathtt{P}_2 = (x_2, y_2)\):
  \begin{align}
    s
      &= \frac{3 x + a}{2 y} \notag \\
    x_2
      &= s^2 - 2 x \label{eq:doble-curva-eliptica} \\
    y_2
      &= y + s (x_2 - x) \notag
  \end{align}
  Es rutina verificar que esto define un grupo abeliano.%
    \index{curva eliptica@curva elíptica!grupo}%
    \index{grupo!abeliano}

  Lo interesante es que las relaciones
  que definen la suma en curvas elípticas
  valen en cualquier campo,
  por lo que podemos considerar
  el grupo definido por la curva elíptica sobre un campo cualquiera.
  Si la característica del campo \(F\) no es \(2\) ni \(3\)
  (vale decir,
   no es \(2 x = 0\) ni \(3 x = 0\) para todo \(x \in F\);
   la discusión formal deberá esperar
   al capítulo~\ref{cha:campos-finitos}),
  toda curva elíptica puede escribirse en la forma:
  \begin{equation*}
    y^2
      = x^3 - p x - q
  \end{equation*}
  tal que el lado derecho no tiene ceros repetidos.
  Interesan los puntos con coordenadas en \(F\).
  El teorema de Hasse~%
    \cite{hasse36:_EC-I, hasse36:_EC-II, hasse36:_EC-III}
  da las cotas para el número \(N\) de elementos en curvas elípticas
  sobre el campo finito de \(q\) elementos:
  \begin{equation*}
    \lvert N - (q + 1) \rvert
      \le 2 \sqrt{q}
  \end{equation*}

  Las curvas elípticas son importantes en teoría de números,
  y tienen bastantes aplicaciones prácticas,
  particularmente se están haciendo muy populares en criptografía.%
    \index{criptografia@criptografía}%
  El sistema \texttt{PARI/GP}~\cite{PARI:2.7.2}%
    \index{PARI/GP@\texttt{PARI/GP}}
  incluye soporte para operar en los grupos respectivos.
  El paquete GAP~%
    \cite{GAP:4.7.5}%
    \index{GAP@\texttt{GAP}}
  tiene extenso soporte para trabajar con grupos,
  incluyendo grupos de curvas elípticas.

%%% Local Variables:
%%% mode: latex
%%% TeX-master: "clases"
%%% End:


\subsection{Anillos cuadráticos}
\label{sec:anillos-cuadraticos}

  Un ejemplo menos conocido de anillo conmutativo lo pone
  \(\mathbb{Z}[\sqrt{2}]\),%
    \index{anillo!cuadratico@cuadrático}
  definido como el conjunto \(\{a + b \sqrt{2} \colon a, b \in \mathbb{Z}\}\),
  con las operaciones tradicionales de los reales.
  Primeramente,
  las operaciones en \(\mathbb{Z}[\sqrt{2}]\) están bien definidas:
  \begin{align*}
    (a_1 + b_1 \sqrt{2}) + (a_2 + b_2 \sqrt{2})
      &= (a_1 + a_2) + (b_1 + b_2) \sqrt{2} \\
    (a_1 + b_1 \sqrt{2}) \cdot (a_2 + b_2 \sqrt{2})
      &= (a_1 a_2 + 2 b_1 b_2) + (a_1 b_2 + a_2 b_1) \sqrt{2}
  \end{align*}
  Los coeficientes en estas expresiones son todos enteros,
  y al ser \(\sqrt{2}\) irracional no hay maneras alternativas%
    \index{numero@número!irracional!\(\sqrt{2}\)}
  de representar el mismo elemento.
  Como los elementos son simplemente números reales,
  las asociatividades y conmutatividades de las operaciones,
  y la distributividad,
  se ``heredan'' de  \(\mathbb{R}\).%
    \index{R (números reales)@\(\mathbb{R}\) (números reales)}
  Podemos representar:
  \begin{align*}
    0 &= 0 + 0 \cdot \sqrt{2} \\
    1 &= 1 + 0 \cdot \sqrt{2}
  \end{align*}
  Tenemos
  \begin{equation*}
    \left(a + b \sqrt{2}\right) + \left((-a) + (-b) \sqrt{2}\right)
      = 0
  \end{equation*}
  lo que provee de un inverso aditivo.
  Como este es un subanillo de los reales,
  no hay divisores de cero salvo \(0\),
  y es un dominio integral.%
    \index{dominio integral}

  Busquemos el inverso de \(a + b \sqrt{2}\) en \(\mathbb{Z}[\sqrt{2}]\):
  \begin{equation}
    \label{eq:inverso-Z[sqrt(2)]-1}
    \frac{1}{a + b \sqrt{2}}
      = \frac{a - b \sqrt{2}}{a^2 - 2 b^2}
  \end{equation}
  Para que~\eqref{eq:inverso-Z[sqrt(2)]-1} pertenezca a nuestro anillo,
  debe ser:
  \begin{equation}
    \label{eq:inverso-Z[sqrt(2)]-2}
    a^2 - 2 b^2
      = \pm 1
  \end{equation}
  Una solución es \(a = b = 1\),
  con lo que \(1 + \sqrt{2}\) es una unidad.

  A ecuaciones de la forma
  \begin{equation}
    \label{eq:Pell}
    x^2 - d y^2
      = 1
  \end{equation}
  con \(d > 1\) se les llama \emph{ecuaciones de Pell}%
    \footnote{Euler%
		\index{Euler, Leonhard}
	      erróneamente se la atribuyó
	      a John Pell~(1611--1685),%
		\index{Pell, John}
	      probablemente confundiéndolo con
	      William Brounker~(1620--1684)%
		\index{Brounker, William}
	      quien fue el primer europeo en estudiarla.
	      Brahmagupta%
		\index{Brahmagupta}
	      en la India en 628 ya describe cómo resolverla.
	      Los números de Pell%
		\index{Pell, numeros de@Pell, números de}
	      (soluciones para el caso \(d = 2\))
	      se conocen desde Pitágoras.%
		\index{Pitagoras@Pitágoras}},%
    \index{Pell, ecuacion de@Pell, ecuación de}
  discutimos el caso \(d = 2\) más arriba.
  La discusión siguiente sigue esencialmente a Djukić~%
      \cite{djukic07:_Pell_equation}.

  Vemos que si \(d\) es un cuadrado perfecto,
    \index{cuadrado perfecto}
  solo es posible la \emph{solución trivial} \(x = 1\), \(y = 0\).
    \index{Pell, ecuacion de@Pell, ecuación de!solucion trivial@solución trivial}
  Enseguida,
  podemos suponer que \(x\) e \(y\) son no negativos,
  por los cuadrados sus signos no importan.
  Podemos factorizar el lado derecho de la ecuación~\eqref{eq:Pell}
  \begin{equation*}
    x^2 - d y^2
      = (x + y \sqrt{d}) (x - y \sqrt{d})
  \end{equation*}
  lo que nos lleva de vuelta al anillo \(\mathbb{Z}[\sqrt{d}]\).%
    \index{anillo!cuadratico@cuadrático}

  \begin{definition}
    \index{anillo!cuadratico@cuadrático!conjugado}
    \index{anillo!cuadratico@cuadrático!norma}
    En el anillo \(\mathbb{Z}[\sqrt{d}]\)
    el \emph{conjugado} de \(z = a + b \sqrt{d}\)
    es \(\overline{z} = a - b \sqrt{d}\),
    y su \emph{norma} es \(N(z) = z \cdot \overline{z} = a^2 - d b^2\).
    Llamamos \emph{parte entera} a \(a\)
    y \emph{parte irracional} a \(b\).
  \end{definition}
  En estos términos,
  son unidades de \(\mathbb{Z}[\sqrt{d}]\)
  exactamente los elementos de norma~\(\pm 1\).
  El inverso de la unidad \(z\) es \(\pm \overline{z}\),
  dependiendo del signo de \(N(z)\).
  Resulta:
  \begin{theorem}
    En \(\mathbb{Z}[\sqrt{d}]\)
    la norma y el conjugado son multiplicativos,
    o sea \(N(z_1 z_2) = N(z_1) \cdot N(z_2)\)
    y \(\overline{z_1 z_2} = \overline{z_1} \cdot \overline{z_2}\)
  \end{theorem}
  \begin{proof}
    Primeramente,
    con \(z_1 = a_1 + b_1 \sqrt{d}\) y \(z_2 = a_2 + b_2 \sqrt{d}\),
    tenemos:
    \begin{align}
      \overline{z_1} \cdot \overline{z_2}
	&= (a_1 - b_1 \sqrt{d}) \cdot (a_2 - b_2 \sqrt{d}) \notag \\
	&= (a_1 a_2 + d b_1 b_2) - (a_1 b_2 + a_2 b_1) \sqrt{d} \notag \\
	&= \overline{z_1 z_2} \label{eq:conjugado-producto}
    \end{align}
    Con esto:
    \begin{align}
      N(z_1 z_2)
	&= (z_1 z_2) \cdot (\overline{z_1 z_2}) \notag \\
	&= (z_1 \overline{z_1}) \cdot (z_2 \overline{z_2}) \notag \\
	&= N(z_1) \cdot N(z_2) \label{eq:norma-producto}
	\qedhere
    \end{align}
  \end{proof}
  Pero también:
  \begin{theorem}
    Si \(z_0\) es el elemento mínimo de \(\mathbb{Z}[\sqrt{d}]\)
    con \(z_0 > 1\) y \(N(z_0) = 1\),
    todos los elementos \(z \in \mathbb{Z}[\sqrt{d}]\) con \(N(z) = 1\)
    están dados por \(z = \pm z_0^n\) con \(n \in \mathbb{Z}\).
  \end{theorem}
  \begin{proof}
    Suponga que \(N(z) = z \cdot \overline{z} = 1\) para \(z > 1\),
    con lo que \(z^{-1} = \overline{z}\).
    Hay un único \(k \in \mathbb{N}_0\)
    tal que \(z_0^k \le z < z_0^{k + 1}\).
    Así \(z_1 = z z_0^{-k} = z (\overline{z_0})^k\)
    cumple \(1 \le z_1 < z_0\),
    y tenemos \(N(z_1) = N(z) \cdot N(z_0)^{-k} = 1\).
    Por la minimalidad de \(z_0\),
    es \(z_1 = 1\)
    y \(z = z_0^k\).
  \end{proof}
  Al par \((x_0, y_0)\) o a \(z_0 = x_0 + y_0 \sqrt{d}\)
  se le llama \emph{solución fundamental} de la ecuación de Pell.%
    \index{Pell, ecuacion de@Pell, ecuación de!solucion fundamental@solución fundamental}

  Todos los anillos \(\mathbb{Z}[\sqrt{d}]\)
  tienen infinitas unidades.
  \begin{theorem}[Dirichlet]
    \label{theo:Dirichlet}
    Sea \(\alpha\) un número irracional%
      \index{numero@número!irracional}
    y \(n\) un entero positivo.
    Entonces hay \(p \in \mathbb{Z}\) y \(q \in [1, n]\)
    tales que:
    \begin{equation}
      \label{eq:desigualdad-Dirichlet}
      \left\lvert \alpha - \frac{p}{q} \right\rvert < \frac{1}{(n + 1) q}
    \end{equation}
  \end{theorem}
  \begin{proof}
    La desigualdad~\eqref{eq:desigualdad-Dirichlet}
    es equivalente a \(\lvert q \alpha - p \rvert < 1 / (n + 1)\).
    Entre los \(n + 2\) números
      \(0, \{\alpha\}, \{2 \alpha\}, \dotsc, \{n \alpha\}, 1\),
    por el principio del palomar%
      \index{principio del palomar}
    (teorema~\ref{theo:pigeonhole})
    en el segmento \([0, 1]\) hay dos que difieren en menos de \(1 / (n + 1)\)
    (si \(\alpha\) es racional podrían diferir en exactamente \(1 / (n + 1)\)).
    Si éstos son \(\{a \alpha\}\) y \(\{b \alpha\}\)
    basta hacer \(q = \lvert a - b \rvert\);
    si son \(\{a \alpha\}\) y 0 o 1,
    basta hacer \(q = a\).
    En cualquiera de los casos,
    \(p\) es el entero más cercano a \(a \alpha\).
  \end{proof}
  De acá,
  como \(n + 1 > q\) es \(1 / ((n + 1) q) < 1 / q^2\),
  sigue inmediatamente:
  \begin{corollary}
    \label{cor:Dirichlet}
    Si \(\alpha\) es un real arbitrario,
    hay infinitos pares de enteros positivos \((p, q)\) tales que:
    \begin{equation*}
       \left\lvert \alpha - \frac{p}{q} \right\rvert
	 < \frac{1}{q^2}
    \end{equation*}
  \end{corollary}
  Así resulta:
  \begin{theorem}
    \label{theo:Pell-soluble}
    La ecuación de Pell
    \begin{equation*}
      x^2 - d y^2
	= 1
    \end{equation*}
    donde \(d\) no es un cuadrado
    tiene una solución no trivial en los enteros.
  \end{theorem}
  \begin{proof}
    Aplicando el corolario~\ref{cor:Dirichlet} a \(\alpha = \sqrt{d}\),
    vemos que hay infinitos pares \((a, b)\) tales que:
    \begin{equation*}
      \left\lvert a - b \sqrt{d} \right\rvert
	< \frac{1}{b}
    \end{equation*}
    Notamos que por la desigualdad triangular,%
      \index{desigualdad triangular}
    teorema~\ref{theo:desigualdad-triangular}:
    \begin{equation*}
      \left\lvert a + b \sqrt{d} \right\rvert
	\le \left\lvert a - b \sqrt{d} \right\rvert
	      + \left\lvert 2 b \sqrt{d} \right\rvert
	\le \frac{1}{b} + 2 b \sqrt{d}
    \end{equation*}
    Con esto:
    \begin{equation*}
      \left\lvert a^2 - b^2 d \right\rvert
	= \left\lvert a + b \sqrt{d} \right\rvert
	    \cdot \left\lvert a - b \sqrt{d} \right\rvert
	\le \left( \frac{1}{b} + 2 b \sqrt{d} \right) \cdot \frac{1}{b}
	\le 2 \sqrt{d} + 1
    \end{equation*}
    Pero al haber infinitos pares
    con normas
      \(\left\lvert N(a + b \sqrt{d}) \right\rvert \le 2 \sqrt{d} + 1\),
    y siendo enteras las normas
    por el principio del palomar%
      \index{principio del palomar}
    hay infinitos pares
    con una misma norma \(N\).
    Acá \(N \ne 0\),
    ya que solo para \(z = 0\) es \(N(z) = 0\).
    Si ahora consideramos todos los pares de norma \(N\),
    nuevamente por el principio del palomar hay infinitos pares
    \(z_1 = (a_1, b_1)\) y \(z_2 = (a_2, b_2)\)
    tales que \(a_1 \equiv a_2\) y \(b_1 \equiv b_2 \pmod{N}\),
    por lo que debe haber \(z_1 \ne \pm z_2\) entre ellos.
    Consideremos:
    \begin{align*}
      z
	&= a + b \sqrt{d}
	 = \frac{z_1}{z_2}
	 = \frac{z_1 \overline{z_2}}{N(z_2)} \\
      N(z)
	&=\frac{N(z_1)}{N(z_2)}
	 = 1
    \end{align*}
    Como \(z_1 \ne \pm z_2\),
    sabemos que \(z \ne \pm 1\).
    Como \(N(z_2) = N\),
    resultan:
    \begin{align*}
      a
	&= \frac{a_1 a_2 - d b_1 b_2}{N} \\
      b
	&= \frac{a_1 b_2 - a_2 b_1}{N}
    \end{align*}
    Observamos que,
    ya que \(a_1 \equiv a_2 \pmod{N}\) y \(b_1 \equiv b_2 \pmod{N}\):
    \begin{align*}
      a_1 a_2 - d b_1 b_2
	&\equiv a_1 a_1 - d b_1 b_1
	 \equiv 0 \pmod{N} \\
      a_1 b_2 - a_2 b_1
	&\equiv a_1 b_1 - a_1 b_1 \phantom{d}
	 \equiv 0 \pmod{N}
    \end{align*}
    con lo que \(a, b \in \mathbb{Z}\),
    o sea \(z \in \mathbb{Z}[\sqrt{d}]\) con \(N(z) = 1\).
  \end{proof}
  También nos interesa saber
  si hay soluciones a \(x^2 - d y^2 = -1\),
  ya que también son unidades de \(\mathbb{Z}[\sqrt{d}]\).
  \begin{theorem}
    \label{theo:Pell-neg-soluble}
    La ecuación \(x^2 - d y^2 = -1\) tiene solución si y solo si
    existe \(z_1 \in \mathbb{Z}[\sqrt{d}]\)
    tal que \(z_1^2 = z_0\).
  \end{theorem}
  \begin{proof}
    Demostramos implicancias en ambas direcciones.
    Si hay tal \(z_1\)
    es menor que \(z_0\),
    y como \(N(z_0) = N(z_1^2) = N(z_1)^2 = 1\),
    debe ser \(N(z_1) = -1\).

    Si \(z\) es solución de \(N(z) = -1\),
    entonces \(N(z^2) = 1\).
    En particular,
    la mínima solución \(z_1 \in \mathbb{Z}[\sqrt{d}]\)
    de la ecuación \(N(z) = -1\) tal que \(z_1 > 1\)
    da lugar a la mínima solución \(z_0 = z_1^2\) de \(N(z) = 1\).
  \end{proof}
  Hace falta determinar raíces cuadradas en \(\mathbb{Z}[\sqrt{d}]\):
  \begin{equation*}
    (x_1 + y_1 \sqrt{d})^2
      = (x_1^2 + d y_1^2) + 2 x_1 y_1 \sqrt{d}
      = (2 d y_1 - 1) + 2 x_1 y_1 \sqrt{d}
      = x_0 + y_0 \sqrt{d}
  \end{equation*}
  Acá usamos \(x_1^2 + d y_1^2 = 1\).
  O sea:
  \begin{align*}
    y_1
      &= \frac{x_0 + 1}{2 d} \\
    x_1
      &= \frac{y_0}{2 y_1}
       = \frac{d y_0}{x_0 + 1}
  \end{align*}
  En \(\mathbb{Z}[\sqrt{2}]\) la solución fundamental es \((3, 2)\),
  como su raíz cuadrada resulta el par \((1, 1) \in \mathbb{Z}[\sqrt{2}]\),
  con \(N(1 + \sqrt{2}) = -1\),
  con lo que todas las unidades
  son \(\pm (1 + \sqrt{2})^n\) para \(n \in \mathbb{Z}\).

  Lenstra~%
    \cite{lenstra02:_solving_pell_equat}
  da algo de la historia de la ecuación de Pell%
    \index{Pell, ecuacion de@Pell, ecuación de}
  y describe algoritmos para obtener la solución fundamental.
  Mayores detalles de la fascinante teoría relacionada con estos anillos
  da por ejemplo Djukić~%
    \cite{djukic07:_arith_exten_Q}.

\subsection{Cuaterniones}
\label{sec:cuaterniones}
\index{cuaterniones|textbfhy}

  Otro ejemplo interesante de anillo lo ponen los \emph{cuaterniones}~%
    \cite{hamilton44:_quaternions},
  una extensión de los números complejos
  inventada para manejar posiciones en el espacio
  como se pueden manejar puntos en el plano con números complejos.
  Hoy han sido reemplazados casi universalmente por vectores,
    \index{vector}
  más flexibles y generales.
  Considerados una curiosidad histórica por mucho tiempo,
  últimamente han encontrado utilidad en diversas áreas,
  como representación eficiente de movimientos y rotaciones en el espacio%
    \index{cuaterniones!en computacion grafica@en computación gráfica}
  en computación gráfica,
  ver por ejemplo Dorst, Fontijne y Mann~%
    \cite{dorst07:_geometric_algebra_comp_sci}.

  Con \(a, b, c, d \in \mathbb{R}\)
  el cuaternión \(z \in \mathbb{H}\) puede describirse
  \begin{equation}
    \label{eq:definicion-cuaternion}
    z = a + b \mathrm{i} + c \mathrm{j} + d \mathrm{k}
  \end{equation}
  donde
  \begin{equation}
    \label{eq:cuaterniones-unidades}
    \mathrm{i}^2
      = \mathrm{j}^2
      = \mathrm{k}^2
      = \mathrm{i} \mathrm{j} \mathrm{k} = -1
  \end{equation}
  Definimos las operaciones con estos objetos
  como para polinomios en \(\mathrm{i}\), \(\mathrm{j}\) y \(\mathrm{k}\),
  aplicando~\eqref{eq:cuaterniones-unidades} luego.

  Resulta que el anillo de cuaterniones no es conmutativo,
    \index{anillo!cuaterniones|see{cuaterniones}}
  la tabla de multiplicación de los elementos unitarios
  es el cuadro~\ref{tab:cuaterniones},
  \begin{table}[htbp]
    \centering
    \begin{tabular}{>{\(}c<{\)}|*{4}{>{\(}r<{\)}}}
      \cdot & 1 & \mathrm{i} & \mathrm{j} & \mathrm{k} \\
      \hline
      1		 & 1	      & \mathrm{i}	& \mathrm{j}  &
	  \mathrm{k}  \\
      \mathrm{i} & \mathrm{i} & -1		& \mathrm{k}  &
	  -\mathrm{j} \\
      \mathrm{j} & \mathrm{j} & -\mathrm{k} & -1	  &
	  \mathrm{i}  \\
      \mathrm{k} & \mathrm{k} & \mathrm{j}	& -\mathrm{i} &
	  -1
    \end{tabular}
    \caption{Multiplicación de cuaterniones}
    \label{tab:cuaterniones}
  \end{table}
  pero todo elemento diferente de cero tiene inverso multiplicativo.
  En detalle,
  si definimos el \emph{conjugado} del cuaternión%
    \index{cuaterniones!conjugado}
    \(q = a + b \mathrm{i} + c \mathrm{j} + d \mathrm{k}\)
  como \(\overline{q} = a - b \mathrm{i} - c \mathrm{j} - d \mathrm{k}\),
  resulta que \(q \overline{q} = a^2 + b^2 + c^2 + d^2\);
  y la \emph{norma} de \(q\)%
    \index{cuaterniones!norma}
  se define como \(\lVert q \rVert = \sqrt{q \overline{q}}\).
  Nótese que \(\overline{p q} = \overline{q} \cdot \overline{p}\).
  Con esto,
  resulta que la norma es multiplicativa,
  ya que la multiplicación entre un cuaternión cualquiera y un real
  conmuta:
  \begin{align*}
    \lVert p q \rVert^2
      &= (p q) \cdot \overline{(p q)} \\
      &= p \cdot q \overline{q} \cdot \overline{p} \\
      &= \lVert p \rVert^2 \cdot \lVert q \rVert^2
  \end{align*}
  Así resulta el \emph{recíproco}%
    \index{cuaterniones!reciproco@recíproco}
  \begin{equation*}
    q^{-1}
      = \frac{\overline{q}}{\lVert q \rVert}
  \end{equation*}
  que claramente cumple \(q q^{-1} = q^{-1} q = 1\).
  La notación \(p / q\) no tiene sentido en cuaterniones,
  ya que \(p q^{-1} \ne q^{-1} p\) en general.

\section{Los teoremas de Lagrange, Euler y Fermat}
\label{sec:Lagrange-Euler-Fermat}

  Un resultado importante que relaciona grupos y subgrupos es el siguiente:%
    \index{grupo!subgrupo}
  \begin{theorem}[Lagrange]
    \index{Lagrange, teorema de}
    \index{Lagrange, Joseph-Louis}
    \label{theo:Lagrange}
    Sea \(G\) un grupo finito,
    y \(H\) un subgrupo de \(G\).
    Entonces \(\lvert H \rvert\) divide a \(\lvert G \rvert\).
  \end{theorem}
  \begin{proof}
    Sea \(a \in G\).
    Al conjunto \(a H = \{a \odot h \colon h \in H\}\)
    se le llama \emph{coset (izquierdo) de \(H\)}%
      \index{coset|textbfhy}
    (de forma afín el \emph{coset derecho}
     \(H a = \{h \odot a \colon h \in H\}\)).
    Demostraremos que todos los cosets tienen el mismo número de elementos,
    y que particionan \(G\),
    de lo que el resultado es inmediato.

    Primeramente,
    el coset \(a H = \{a \odot h \colon h \in H\}\)
    no tiene elementos repetidos,
    porque supongamos que hay \(h, g \in H\) tales que
    \(a \odot h = a \odot g\),
    por la ley de cancelación es \(h = g\).
    Resulta simplemente \(\lvert a H \rvert = \lvert H \rvert\).

    Definamos la relación \(R\) sobre \(G\) mediante
    \(x \mathrel{R} y\) si y solo si
    hay \(h \in H\) tal que \(x = y \odot h\).
    Entonces \(R\) es una relación de equivalencia:%
      \index{relacion@relación!equivalencia}
    \begin{description}
    \item[Reflexiva:]
      Necesariamente \(1 \in H\),
      con lo que \(x \mathrel{R} x\).
    \item[Simétrica:]
      Esto porque \(x \mathrel{R} y\) corresponde a \(x = y \odot h\)
      para \(h \in H\),
      pero entonces también \(y = x \odot h^{-1}\),
      y como \(h^{-1} \in H\) tenemos \(y \mathrel{R} x\).
    \item[Transitiva:]
      Si \(x \mathrel{R} y\) y \(y \mathrel{R} z\)
      entonces hay \(h_1, h_2 \in H\) tales que
      \(x = y \odot h_1\) y \(y = z \odot h_2\).
      Combinando éstos,
      \(x = z \odot (h_2 \odot h_1)\),
      y \(h_2 \odot h_1 \in H\),
      con lo que \(x \mathrel{R} z\).
    \end{description}
    Las clases de equivalencia de \(R\) son precisamente los cosets de \(H\):
    \(x \in [y]\) siempre que podemos escribir \(x = y \odot h\)
    con \(h \in H\),
     o sea, \(x \in y H\),
     con lo que \([y] = y H\).
    Pero \(\lvert a H \rvert = \lvert H \rvert\)
    como vimos antes,
    y tenemos nuestro resultado.
  \end{proof}

  En particular,
  consideremos el subgrupo generado por el elemento \(a \in G\),%
    \index{grupo!subgrupo generado}%
    \index{grupo!orden}
    \index{grupo!orden de un elemento}
  vale decir,
  si el orden de \(a\) es \(n\)
  los elementos \(a^0\), \(a^1\), \ldots, \(a^{n - 1}\).
  Este subgrupo tiene orden \(n\),
  con lo que \(n\) divide a \(\lvert G \rvert\).

  Nos abocaremos a un estudio más detallado de \(\mathbb{Z}^\times_m\),
  una vez adquiridas algunas herramientas algebraicas adicionales.
  \begin{theorem}
    \label{theo:a-invertible}
    El elemento \(a \in \mathbb{Z}_m\) es invertible
    si y solo si \(a\) y \(m\) son coprimos.
    En particular,
    si \(p\) es primo todos los elementos de \(\mathbb{Z}_p\)
    (salvo \(0\))
    son invertibles,
    y \(\mathbb{Z}_p\) es un campo.%
      \index{campo (algebra)@campo (álgebra)}
  \end{theorem}
  \begin{proof}
    Demostramos implicancia en ambos sentidos.
    Supongamos \(a\) invertible.
    Entonces existen enteros \(b\) y \(k\) tales que \(a b - 1 = k m\),
    que es decir \(a b - k m = 1\).
    Pero si existen tales \(b\) y \(k\) entonces \(\gcd(a, m) = 1\).

    Al revés,
    supongamos \(\gcd(a, m) = 1\).
    Entonces
    (por la identidad de Bézout)%
      \index{Bezout, identidad de@Bézout, identidad de}
    existen \(s\), \(t\) tales que:
    \begin{align*}
      s \cdot a + t \cdot m
	&=	1\\
      s \cdot a
	&\equiv 1 \pmod{m}
    \end{align*}
    y \(s\) es el inverso de \(a\).
  \end{proof}
  Resulta que el número de unidades de \(\mathbb{Z}_m\)
  es una cantidad muy importante.
  Por el teorema~\ref{theo:a-invertible},
  no es más que la cantidad de números en \(1, 2, \dotsc, m\)
  que son relativamente primos a \(m\),
  que se anota \(\phi(m)\)
  (función \(\phi\) de Euler).%
    \index{\(\phi\) de Euler}%
    \index{Euler, Leonhard}

  Un ejemplo lo pone \(\mathbb{Z}_{12}\),
  donde tenemos la tabla de multiplicación~\ref{tab:Z12}.
  \begin{table}[htbp]
    \centering
    \renewcommand{\tabcolsep}{3pt}
    \begin{tabular}{>{\(}r<{\)}|*{12}{>{\(}r<{\)}}}
      \multicolumn{1}{c|}{\(\cdot\)} &
	    \;0 & \phantom{0}1 & \phantom{0}2 & \phantom{0}3 & \phantom{0}4
		& \phantom{0}5 & \phantom{0}6 & \phantom{0}7 & \phantom{0}8
		& \phantom{0}9 & 10 & 11 \\
      \hline
	\rule[-0.7ex]{0pt}{3ex}%
       0 &  0 &	 0 &  0 &  0 &	0 &  0 &  0 &  0 &  0 &	 0 &  0 &  0 \\
       1 &  0 &	 1 &  2 &  3 &	4 &  5 &  6 &  7 &  8 &	 9 & 10 & 11 \\
       2 &  0 &	 2 &  4 &  6 &	8 & 10 &  0 &  2 &  4 &	 6 &  8 & 10 \\
       3 &  0 &	 3 &  6 &  9 &	0 &  3 &  6 &  9 &  0 &	 3 &  6 &  9 \\
       4 &  0 &	 4 &  8 &  0 &	4 &  8 &  0 &  4 &  8 &	 0 &  4 &  8 \\
       5 &  0 &	 5 & 10 &  3 &	8 &  1 &  6 & 11 &  4 &	 9 &  2 &  7 \\
       6 &  0 &	 6 &  0 &  6 &	0 &  6 &  0 &  6 &  0 &	 6 &  0 &  6 \\
       7 &  0 &	 7 &  2 &  9 &	4 & 11 &  6 &  1 &  8 &	 3 & 10 &  5 \\
       8 &  0 &	 8 &  4 &  0 &	8 &  4 &  0 &  8 &  4 &	 0 &  8 &  4 \\
       9 &  0 &	 9 &  6 &  3 &	0 &  9 &  6 &  3 &  0 &	 9 &  6 &  3 \\
      10 &  0 & 10 &  8 &  6 &	4 &  2 &  0 & 10 &  8 &	 6 &  4 &  2 \\
      11 &  0 & 11 & 10 &  9 &	8 &  7 &  6 &  5 &  4 &	 3 &  2 &  1
    \end{tabular}
    \caption{La tabla de multiplicación en $\mathbb{Z}_{12}$}
    \label{tab:Z12}
  \end{table}
  Pueden apreciarse los elementos invertibles
  \(\mathbb{Z}^\times_{12} = \{1, 5, 7, 11\}\),
  con lo que \(\phi(12) = 4\).
  Se ve también que los demás son todos divisores de cero,
  como asegura el teorema~\ref{theo:finite-ring-units}.

  Al considerar el subgrupo de \(\mathbb{Z}^\times_m\) generado por \(a\)
  tenemos del teorema de Lagrange%
    \index{Lagrange, teorema de}
  que el orden de \(a\) divide al orden de \(\mathbb{Z}^\times_m\),
  que es \(\phi(m)\),
  y así:
  \begin{theorem}[Euler]
    \index{Euler, teorema de}
    \index{Euler, Leonhard}
    \label{theo:Euler}
    Si \(a\) y \(m\) son relativamente primos,
    entonces
    \begin{equation*}
      a^{\phi(m)}
	\equiv 1 \pmod{m}
    \end{equation*}
  \end{theorem}
  En el caso de que \(m\) sea primo,
  como \(\phi(p) = p - 1\) para \(p\) primo,
  el teorema de Euler se reduce a:
  \begin{theorem}[Pequeño teorema de Fermat]
    \index{Fermat, pequeno teorema de@Fermat, pequeño teorema de}
    \label{theo:Fermat}
    Si	\(p\) es primo,
    y \(p \centernot\mid a\)
    entonces
    \begin{equation*}
      a^{p - 1}
	\equiv 1 \pmod{p}
    \end{equation*}
  \end{theorem}
  El hecho de que a este se le llame ``pequeño''
  no tiene ninguna relación con su importancia,
  veremos una gran variedad de aplicaciones en lo que sigue.
% Fixme: Un poquito de historia
  El ``gran'' (o ``último'') teorema de Fermat%
    \index{Fermat, ultimo teorema@Fermat, último teorema}
  es uno de los resultados más famosos de las matemáticas.
  Fermat anotó por 1637%
    \index{Fermat, Pierre de}
  en el margen de una copia de la Aritmética de Diofanto%
    \index{Diofanto}
  que había descubierto una demostración verdaderamente maravillosa
  de que \(a^n + b^n = c^n\)
  no tiene soluciones
  para números naturales \(a\), \(b\), \(c\) si \(n > 2\),
  pero que esta no cabía en el margen
  (tenía esta mala costumbre,
   la publicación del libro rayado después de su muerte
   dio trabajo a ejércitos de matemáticos durante bastante tiempo).
  Se le llamó el ``último teorema''
  no por ser la última de sus innumerables conjeturas,
  sino por ser la última importante que quedaba sin resolver
  una vez que Euler terminó de trabajar en ellas.
  Recién en 1995 Andrew Wiles con la ayuda de su estudiante Richard Taylor~%
    \cite{taylor95:_ring_theor_proper_hecke_alg,
	  wiles95:_modul_ellip_curves_Fermat}
  demostró el último teorema de Fermat,
  aunque mediante métodos muy recientes
  (y no es precisamente una demostración ``maravillosa'').
  Generalmente se piensa que Fermat se equivocó
  al creer que tenía una demostración.

% Fixme: Determinar órdenes, etc

%%% Local Variables:
%%% mode: latex
%%% TeX-master: "clases"
%%% End:

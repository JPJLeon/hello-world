% digrafos.tex
%
% Copyright (c) 2009-2014 Horst H. von Brand
% Derechos reservados. Vea COPYRIGHT para detalles

\chapter{Digrafos, redes, flujos}
\label{cha:digrafos}
\index{digrafo|textbfhy}
\index{grafo!dirigido|see{digrafo}}

  En muchas situaciones que podrían modelarse mediante grafos
  las conexiones no son bidireccionales.
  Por ejemplo,
  están las calles de una sola vía.
  En un proyecto hay actividades que deben efectuarse en orden,
  interesa representar estas dependencias
  y organizarlas de forma de completar el proyecto lo antes posible.
  Si queremos analizar el flujo a través de una red de tuberías
  o el flujo de bienes transportados
  hay una dirección definida.
  En esto suele interesar la capacidad de transporte de la red.
  Estas situaciones se modelan por grafos dirigidos rotulados.
  Generalmente se tratan solo en un capítulo de textos
  que se concentran en grafos,
  pero se puede argüir
  que la importancia práctica de los grafos dirigidos
  (digrafos, para abreviar)
  es similar a la de los grafos.
  Trataremos algunos algoritmos importantes del área
  con análisis somero de sus rendimientos.

\section{Definiciones básicas}
\label{sec:definiciones-basicas}

  Nos interesa representar estructuras similares a grafos,
  solo que los arcos tienen una dirección definida.
  No está la simetría entre ambos extremos del arco como en grafos.
  Por ejemplo,
  si queremos representar las llamadas de funciones en un programa,
  interesa cuál de las dos es quien llama y cuál es llamada.
  \begin{definition}
    Un \emph{digrafo}
    (o \emph{grafo dirigido})
    consta de un conjunto finito \(V\) de \emph{vértices},%
      \index{digrafo!vertice@vértice|textbfhy}
    y un subconjunto \(A\) de \(V \times V\),
    cuyos miembros se llaman \emph{arcos}.%
      \index{digrafo!arco|textbfhy}
    Anotaremos \(D = (V, A)\)
    para el digrafo definido de esta forma.
  \end{definition}
  Un texto reciente,
  que cubre desde temas elementales
  hasta áreas de investigación activa,
  es el de Bang-Jensen y~Gutin~%
    \cite{bang-jensen09:_digraphs}.

  Los digrafos se pueden representar gráficamente
  de forma similar a los grafos,
  solo que en este caso un arco es un par ordenado \((x, y)\),
  mientras en el grafo es un par no ordenado \(\{x, y\}\).
  Si \((x, y)\) es un arco,
  lo indicamos mediante una flecha de \(x\) a \(y\),
  si hay un arco \((x, x)\) lo indicamos mediante un ciclo,
  si hay arcos \((x, y)\) e \((y, x)\)
  los indicamos por dos flechas.
  Para simplificar notación,
  similar al caso de grafos
  usaremos \(u v\) para indicar el arco \((u, v)\).
  La figura~\ref{fig:digrafos} muestra ejemplos.
  \begin{figure}[htbp]
    \setbox1=\hbox{\pgfimage{images/digrafo-1}}
    \setbox2=\hbox{\pgfimage{images/digrafo-2}}
    \centering
    \subfloat{\raisebox{0.5\ht2-0.5\ht1}{\copy1}}%
    \hspace{3em}%
    \subfloat{\copy2}
    \caption{Ejemplos de digrafos}
    \label{fig:digrafos}
  \end{figure}
  Nuestras representaciones de grafos para uso computacional
  se aplican con cambios obvios a este caso.%
    \index{digrafo!representacion@representación}
  Igualmente,
  podemos aplicar los algoritmos de recorrido acá.
  Los algoritmos de Dijkstra,
  algoritmo~\ref{alg:Dijkstra},%
    \index{Dijkstra, algoritmo de}
  de Floyd-Warshall,
  algoritmo~\ref{alg:Floyd-Warshall}%
    \index{Floyd-Warshall, algoritmo de}
  (que en este caso es llamado simplemente algoritmo de Floyd)%
    \index{Floyd, algoritmo de}
  y de Bellman-Ford~\ref{alg:Bellman-Ford}%
    \index{Bellman-Ford, algoritmo de}
  son aplicables con modificaciones obvias.

  Un digrafo
  es simplemente otra manera de representar una relación \(R\)%
    \index{relacion@relación}
  entre elementos del \emph{mismo} conjunto
  (un grafo bipartito,
  por otro lado,
  representa una relación
  entre elementos de conjuntos \emph{disjuntos}).
  Las propiedades de las relaciones pueden fácilmente traducirse
  en propiedades del digrafo.
  Por ejemplo,
  si la relación es simétrica los arcos aparecen en pares
  (salvo los bucles),
  \(x y\) es un arco exactamente cuando lo es \(y x\).

  Las definiciones de camino,
  camino simple,
  circuito y ciclo en un grafo dirigido
  son análogas a las para grafos.
  Un \emph{camino dirigido} en \(D = (V, A)\)%
    \index{digrafo!camino dirigido|textbfhy}
  es una secuencia de vértices \(v_1, v_2, \dotsc, v_k\)
  tal que \(v_i v_{i + 1} \in A\) para \(1 \le i \le k - 1\),
  un \emph{camino dirigido simple} es un camino dirigido
  en que todos los vértices son diferentes,
  un \emph{circuito dirigido}%
    \index{digrafo!circuito dirigido|textbfhy}
  es un camino cuyo inicio y fin coinciden,
  mientras un \emph{ciclo dirigido}%
    \index{digrafo!ciclo dirigido|textbfhy}
  es un camino dirigido
  en que todos los vértices son distintos,
  solo que el inicial y el final coinciden.
  Un caso particularmente importante
  lo ponen los \emph{grafos dirigidos acíclicos}%
    \index{digrafo!aciclico@acíclico}%
    \index{grafo dirigido aciclico@grafo dirigido acíclico|see{digrafo!acíclico}}
  (abreviados comúnmente \emph{DAG},%
    \index{DAG|see{digrafo!acíclico}}
   por la frase en inglés
   \emph{\foreignlanguage{english}{Directed Acyclic Graph}}),
  de alguna forma análogos a los árboles.
    \index{grafo!arbol@árbol}

\section{Orden topológico}
\label{sec:topological-sort}
\index{digrafo!orden topologico@orden topológico}

  Dado un digrafo \(G = (V, E)\),
  un \emph{orden topológico}
  (en inglés,
   \emph{\foreignlanguage{english}{topological sort}})
  de los vértices de \(G\)
  es un ordenamiento de \(V\) tal que para cada arco \(u v \in E\)
  el vértice \(u\) aparece antes que \(v\).
  Esto claramente solo puede existir si el digrafo es acíclico.

  La aplicación típica es definir un orden
  para efectuar un conjunto de tareas
  que dependen entre sí.
  Por ejemplo,
  al vestirse uno debe ponerse los calcetines
  antes que los zapatos,
  pero no hay precedencia entre la camisa y los calcetines.
  La figura~\ref{fig:dressing} ilustra las restricciones.
  \begin{figure}[ht]
    \centering
    \pgfimage{images/vestirse}
    \caption{Restricciones al vestirse}
    \label{fig:dressing}
  \end{figure}
  Otras aplicaciones aparecen en compiladores,
  al reordenar instrucciones;%
    \index{generar codigo@generar código}
  al definir el orden
  en que se calculan las celdas en planillas de cálculo;
  y lo usa \texttt{make(1)}%
    \index{make@\texttt{make(1)}}
  para organizar el orden en que se generan los archivos.
  Unix%
    \index{Unix}
  ofrece el comando \texttt{tsort(1)},%
    \index{tsort@\texttt{tsort(1)}}
  que toma líneas indicando dependencias
  y entrega un orden consistente con ellas.
  Aplicando este último a la tarea de vestirse,
  sugiere el orden
  calcetines,
  camisa,
  pantalón,
  jersey,
  zapatos,
  cinturón
  y finalmente chaqueta.
  Este orden no es único,
  se ve de la figura que podríamos haber comenzado por la camisa,
  o haber terminado con los zapatos.

  Algoritmos para hallar un orden topológico
  se basan en que en un digrafo acíclico
  habrán vértices que no tienen arcos de entrada
  (respectivamente de salida).
  Kahn~\cite{kahn62:_topol_sorting_large_net}
  publicó el algoritmo clásico~\ref{alg:tsort-Kahn}.%
    \index{Kahn, algoritmo de}
  \begin{algorithm}[htbp]
    \DontPrintSemicolon

    \(L \leftarrow \text{lista vacía}\) \;
    \(S \leftarrow \text{conjunto de nodos sin arcos entrantes}\) \;
    \While{\(S \ne \varnothing\)}{
      Extraiga \(n\) cualquiera de \(S\) \;
      Agregue \(n\) al final de \(L\) \;
      \ForEach{nodo \(m\) con arco \(e = (n, m)\)}{
	Elimine \(e\) del grafo \;
	\If{\(m\) no tiene más arcos entrantes}{
	  Inserte \(m\) en \(S\) \;
	}
      }
    }
    \uIf{quedan arcos en el grafo}{
      \Return{error} \tcc*{El digrafo tiene ciclos}
    }
    \Else{
      \Return{\(L\)}
    }
    \caption{Ordenamiento topológico de Kahn}
    \label{alg:tsort-Kahn}
  \end{algorithm}
  Se basa en la observación que en un digrafo acíclico
  deben haber vértices sin arcos entrantes,
  y que cualquiera de ellos puede tomar el primer lugar en el orden.

  El algoritmo~\ref{alg:tsort-Tarjan} es debido a Tarjan~%
    \cite{tarjan76:_edge_disjoin_spann_trees_depth_first_searc}.%
    \index{Tarjan, algoritmo de}
  \begin{algorithm}[htbp]
    \DontPrintSemicolon
    \SetKwFunction{Visit}{visit}

    \(L \leftarrow \text{lista vacía}\) \;
    \(S \leftarrow \text{conjunto de todos los nodos sin arcos salientes}\) \;
    \BlankLine
    \KwProcedure \Visit{\(n\)} \;
    \Begin{
      \If{\(n\) no ha sido visitado}{
	Marque \(n\) como visitado \;
	\ForEach{nodo \(m\) con \((m, n) \in E\)}{
	  \Visit{\(m\)} \;
	}
	Agregar \(n\) al final de \(L\) \;
      }
    }
    \BlankLine
    \ForEach{nodo \(n\) en \(S\)}{
      \Visit{\(n\)} \;
    }
    \Return{L} \;
    \caption{Ordenamiento topológico de Tarjan}
    \label{alg:tsort-Tarjan}
  \end{algorithm}
  Es una aplicación de búsqueda en profundidad
  (ver la sección~\ref{sec:DFS}).%
    \index{grafo!busqueda en profundidad@búsqueda en profundidad}
  Notar que procesa los vértices desde el final hacia el comienzo
  (orden inverso al algoritmo de Kahn).
  Cuidado,
  como el algoritmo~\ref{alg:tsort-Tarjan}
  como está escrito falla si el digrafo tiene ciclos.
% Fixme: Revisar TAoCP1

% Fixme: Agregar más material (p.ej. de tarea sobre digrafos)

\section{Redes y rutas críticas}
\label{sec:redes-rutas-criticas}
\index{red}

  Es común que se asocien costos o distancias a los arcos.
  Con esta idea en mente,
  llamaremos \emph{red} a un digrafo \(D = (V, A)\)
  junto con una función \(w \colon A \rightarrow \mathbb{R}\),%
    \index{red|see{digrafo}}
  que representa costos de algún tipo
  o capacidades,
  según la aplicación.

  Una aplicación típica de redes
  son las \emph{redes de actividades}.%
    \index{red de actividades}
  Suponiendo un gran proyecto,
  este se subdivide en actividades menores.
  Las actividades a su vez están relacionadas,
  en el sentido que algunas no pueden iniciarse
  antes que terminen otras.
  Al planificar un proyecto de este tipo
  se suele usar una red de actividades,
  con arcos representando actividades
  y los vértices representando ``eventos'',
  donde un evento es el fin de una actividad.
  Es claro que tal digrafo es acíclico,
  ninguna actividad puede depender directa o indirectamente
  de sí misma.
  El peso de un arco es la duración de la actividad,
  y se busca organizar las actividades
  de forma de minimizar el tiempo total del proyecto.
  Técnicas basadas en esta idea son
  CPM (\foreignlanguage{english}{Critical Path Method})~%
    \cite{kelley59:_CPM}%
    \index{Critical Path Method@\emph{\foreignlanguage{english}{Critical Path Method}}}%
    \index{CPM|see{\emph{\foreignlanguage{english}{Critical Path Method}}}}
  y PERT (\foreignlanguage{english}
	  {Project Management and Evaluation Technique})~%
    \cite{malcolm59:_PERT}.%
    \index{Project Management and Evaluation Technique@\emph{\foreignlanguage{english}{Project Management and Evaluation Technique}}}%
    \index{PERT|see{\emph{\foreignlanguage{english}{Project Management and Evaluation Technique}}}}

  Consideremos un ejemplo concreto.
  El cuadro~\ref{tab:actividades}
  da las duraciones de las actividades
  (en meses),
  y las dependencias entre ellas
  (qué actividades deben estar completas
   antes de comenzar la actividad indicada).
  \begin{table}[htbp]
    \centering
    \begin{tabular}[c]{|c|l|l|>{\(}c<{\)}|}
      \hline
      \multicolumn{1}{|c|}{\rule[-0.7ex]{0pt}{3ex}\textbf{Act}} &
	 \multicolumn{1}{c|}{\textbf{Descripción}} &
	 \multicolumn{1}{c|}{\textbf{Requisitos}} &
	 \multicolumn{1}{c|}{\textbf{Dur}} \\
      \hline
	\rule[-0.7ex]{0pt}{3ex}%
      \(A\) & Diseño del producto    &			   & 5 \\
      \(B\) & Análisis de mercado    &			   & 1 \\
      \(C\) & Análisis de producción & \(A\)		   & 2 \\
      \(D\) & Prototipo del producto & \(A\)		   & 3 \\
      \(E\) & Diseño de folleto	     & \(A\)		   & 2 \\
      \(F\) & Análisis de costos     & \(C\)		   & 3 \\
      \(G\) & Pruebas del producto   & \(D\)		   & 4 \\
      \(H\) & Entrenamiento ventas   & \(B\), \(E\)	   & 2 \\
      \(I\) & Definición de precios  & \(H\)		   & 1 \\
      \(J\) & Reporte del proyecto   & \(F\), \(G\), \(I\) & 1 \\
      \hline
    \end{tabular}
    \caption{Actividades y dependencias}
    \label{tab:actividades}
  \end{table}

  El primer paso es construir la red de actividades,
  ver figura~\ref{fig:actividades}.
  \begin{figure}[htbp]
    \centering
    \pgfimage{images/actividades}
    \vspace*{2ex}

    \begin{tabular}[c]{l@{\quad}
		       *{9}{>{\(}c<{\)}@{\hspace{0.75em}}}>{\(}c<{\)}}
      Actividad:
	  & A	   & B	    & C	& D
	  & E	   & F	    & G	& H
	  & I	   & J \\
      Arco:
	  & (1, 2) & (1, 3) & (2, 4) & (2, 5)
	  & (2, 3) & (4, 7) & (5, 7) & (3, 6)
	  & (6, 7) & (7, 8) \\
      Duración:
	  &	 5 &	  1 &	   2 &	    3
	  &	 2 &	  3 &	   4 &	    2
	  &	 1 &	  1
    \end{tabular}
    \caption{Una red de actividades}
    \label{fig:actividades}
  \end{figure}
  Los arcos representan actividades,
  y los vértices sus inicios y finales.
  Es claro que resulta un digrafo sin ciclos.
  Para cada evento calcularemos \(E(v)\),
  el instante más temprano en que ese evento puede tener lugar.
  Esto corresponde al momento más temprano
  en que todas las actividades previas a \(v\) han terminado.
  Iniciamos el proceso con \(E(1) = 0\).
  Luego está claro que \(E(2) = 5\),
  ya que la única actividad involucrada es \(A = (1, 2)\).
  Ahora,
  como el evento 3 involucra a \(B = (1, 3)\),
  pero también \(A = (1, 2)\) y \(E = (2, 3)\),
  \(E(3) = \max \{E(1) + 1, E(2) + 2\}
	 = \max \{0 + 1, 5 + 2\}
	 = 7\).
  Continuando de esta forma,
  obtenemos los instantes de término
  dados en el cuadro~\ref{tab:actividades-E},
  \begin{table}[htbp]
    \centering
    \begin{tabular}[c]{l*{9}{>{\(}c<{\)}}}
      \(v\):	  & 1 &	 2 &  3 &  4 &	5 &  6 &  7 &  8 \\
      \(E(v)\):	  & 0 &	 5 &  7 &  7 &	8 &  9 & 12 & 13 \\
    \end{tabular}
    \caption{Términos más tempranos por actividad
	     para la red de la figura~\ref{fig:actividades}}
    \label{tab:actividades-E}
  \end{table}
  con lo que el plazo mínimo
  para completar el proyecto es de \(13\) meses.

  Esto es básicamente usar el algoritmo de Dijkstra%
    \index{Dijkstra, algoritmo de}
  (ver sección~\ref{sec:Dijkstra}),
  solo que estamos calculando
  el camino más \emph{largo} a través de la red.
  Éste está perfectamente definido en este caso,
  ya que no hay ciclos.
  Basta un barrido a lo ancho,
  al no haber ciclos
  se obtiene el valor final directamente.
  En cada vértice visitado calculamos \(E(v)\)
  partiendo del evento inicial \(s\)
  (el comienzo del proyecto)
  mediante la regla:
  \begin{equation*}
    E(s) = 0 \qquad E(v) = \max_x \{E(x) + w(x, v)\}
  \end{equation*}
  donde el máximo es sobre los vértices \(x\) predecesores de \(v\).

  Esto es parte de la técnica
  que se conoce como \emph{análisis de camino crítico}.%
    \index{analisis de camino critico@análisis de camino crítico}
  El resto de la técnica continúa como sigue:
  Calculamos \(L(v)\),
  el último instante en que puede ocurrir el evento \(v\)
  sin retrasar el proyecto completo
  de forma similar a como se calcularon los \(E(v)\),
  pero comenzando del evento final \(t\) y trabajando en reversa:
  \begin{equation*}
    L(t) = E(t) \qquad L(v) = \min_x \{L(x) - w(v, x)\}
  \end{equation*}
  donde el mínimo es sobre los vértices \(x\) sucesores de \(v\).
  Aplicando esto al ejemplo de la figura~\ref{fig:actividades}
  da el cuadro~\ref{tab:actividades-L}.
  \begin{table}[htbp]
    \centering
    \begin{tabular}[c]{l*{9}{>{\(}c<{\)}}}
      \(v\):	& 1 &  2 &  3 &	 4 &  5 &  6 &	7 &  8 \\
      \(L(v)\): & 0 &  5 &  9 &	 9 &  8 & 11 & 12 & 13
    \end{tabular}
    \caption{Inicio más tardío por actividad
	     para la red~\ref{fig:actividades}}
    \label{tab:actividades-L}
  \end{table}

  Podemos además calcular la \emph{holgura} de cada actividad%
    \index{analisis de camino critico@análisis de camino crítico!holgura}
  \(u v\),
  que representa el máximo retraso
  que puede sufrir el comienzo de la actividad
  sin retrasar el fin del proyecto:
  \begin{equation*}
    F(u, v) = L(v) - E(u) - w(u, v)
  \end{equation*}
  Las actividades sin holgura se dice que son \emph{críticas},
    \index{analisis de camino critico@análisis de camino crítico!actividad critica@actividad crítica}
  cualquier retraso en éstas
  se traduce en un retraso del proyecto completo.
  Toda red de actividades
  tiene al menos un camino dirigido entre principio y fin
  formado únicamente por actividades críticas,
  tal camino se llama \emph{ruta crítica}.
    \index{analisis de camino critico@análisis de camino crítico!ruta critica@ruta crítica}
  Combinando los valores en los cuadros~\ref{tab:actividades-E}
  y~\ref{tab:actividades-L}
  obtenemos las holguras
  dadas en el cuadro~\ref{tab:actividades-F}.
  \begin{table}[htbp]
    \centering
    \begin{tabular}[c]{l@{\hspace{0.5em}}*{10}{@{\hspace{0.75em}}>{\(}c<{\)}}}
      Actividad:
	 & A	  & B	   & C	   & D
	 & E	  & F	   & G	   & H
	 & I	  & J \\
      Arco:
	 & (1, 2) & (1, 3) & (2, 4) & (2, 5)
	 & (2, 3) & (4, 7) & (5, 7) & (3, 6)
	 & (6, 7) & (7, 8)  \\
      Holgura:
	 &	0 &	 8 &	  2 &	   0
	 &	2 &	 2 &	  0 &	   2
	 &	2 &	 0
    \end{tabular}
    \caption{Holguras para las actividades
	     de la figura~\ref{fig:actividades}}
    \label{tab:actividades-F}
  \end{table}
  Son actividades críticas las que no tienen holgura,
  en nuestro caso
  \(A\), \(D\), \(G\) y \(J\);
  y se ve en la figura~\ref{fig:actividades}
  que forman un camino a través del grafo.

  El procedimiento
  es fundamentalmente un par de recorridos a lo ancho%
    \index{grafo!recorrido a lo ancho}
  del digrafo.
  Como se discutió para este algoritmo en el caso de grafos
  en la sección~\ref{sec:BFS+DFS},
  la complejidad es \(O(\lvert V \rvert + \lvert E \rvert)\).

% Fixme: Discuss PERT a bit, references. Text book!

\section{Redes y flujos}
\label{sec:redes-flujos}
\index{red!flujo}

  En lo que sigue interpretaremos los arcos como ``tuberías''
  por las que puede fluir alguna mercadería.
  Un  ejemplo claro es el de circuitos eléctricos,%
    \index{circuito electrico@circuito eléctrico}
  con corrientes en los distintos conductores.
  Nótese que a diferencia de la aplicación anterior
  a redes de actividades
  acá un ciclo dirigido es perfectamente posible
  (aunque probablemente ineficiente).
  Los pesos numéricos representan la capacidad del arco.
  Además,
  habrá un vértice \(s\) con la propiedad que todos los arcos
  que contienen a \(s\) se alejan de él,
  y otro vértice \(t\)
  con la propiedad de que todos los arcos que lo contienen
  se dirigen a él.
  Al primero se le llama \emph{fuente}%
    \index{red!fuente}
    (en inglés \emph{\foreignlanguage{english}{source}}),%
    \index{red!source@\emph{\foreignlanguage{english}{source}}|see{red!fuente}}
  al segundo \emph{sumidero}%
    \index{red!sumidero}
    (en inglés \emph{\foreignlanguage{english}{sink}}.%
    \index{red!sink@\emph{\foreignlanguage{english}{sink}}|see{red!sumidero}}
  En resumen,
  trataremos con redes que incluyen:
  \begin{enumerate}[label=(\roman{*})]
  \item
    Un digrafo \(D = (V, A)\).
  \item
    Una función de capacidad \(c \colon A \rightarrow \mathbb{R}\).
    Comúnmente haremos referencia a capacidades
    de enlaces inexistentes,
    usamos la convención que si \(x y \notin A\)
    entonces \(c(x y) = 0\).
  \item
    Una fuente \(s\) y un sumidero \(t\).
  \end{enumerate}
  La figura~\ref{fig:flow-network} muestra una red,
  con los arcos rotulados con sus capacidades.
  También describe un flujo en esta red.
  \begin{figure}[htbp]
    \centering
    \pgfimage{images/flow-network}
    \\[2ex]
    \begin{tabular}[c]{r*{9}{c}}
      \((x, y)\):  & \((s, a)\) & \((s, b)\) & \((s, c)\) & \((a, d)\)
		   & \((b, d)\) & \((c, d)\) & \((a, t)\) & \((c, t)\)
		   & \((d, t)\) \\
      \(c(x, y)\): &	      5 &	   4 &		3 &	     7
		   &	      2 &	   7 &		3 &	     5
		   &	      4 \\
      \(f(x, y)\): &	      3 &	   2 &		3 &	     1
		   &	      2 &	   1 &		2 &	     2
		   &	      4
    \end{tabular}
    \caption{Una red, sus capacidades y un flujo en la red}
    \label{fig:flow-network}
  \end{figure}
% Fixme: Notación: Usar u, v en vez de x, y?

  Supongamos que algún material fluye por la red,
  y sea \(f(x, y)\) el flujo a lo largo del arco \(x y\),
  de \(x\) a \(y\).
  Si \(x y\) ni \(y x\) son arcos,
  por convención \(f(x, y) = 0\).
  Insistiremos para todos los vértices,
  salvo para la fuente y el sumidero,
  en que el flujo que entra al vértice debe ser el flujo que sale
  (no hay acumulación de material en los vértices).

  \begin{definition}
    \index{red!flujo|textbfhy}
    Un \emph{flujo} en una red es una función
    que asigna un número \(f(x, y)\) a cada arco \(x y\),
    sujeto a las condiciones:
    \begin{description}
    \item[Viabilidad:]
      El flujo en cada enlace es a lo más su capacidad,
      \(f(x, y) \le c(x, y)\) para todo arco \(x y \in A\).
    \item[Simetría torcida:]
      En inglés \emph{\foreignlanguage{english}{skew symmetry}},
      una convención de notación:
      \(f(y, x) = -f(x, y)\) para todo arco \(x y \in A\).
    \item[Conservación:]
      Para todo vértice \(y \in V\)
      que no sea la fuente ni el sumidero de la red
      requerimos que el flujo neto que entra en él sea cero:
      \begin{equation*}
	\sum_{x \in V} f(x, y) = 0
      \end{equation*}
    \end{description}
    El \emph{valor} del flujo es el flujo total que entra a la red:
    \begin{equation*}
      \val(f) = \sum_{x \in V} f(s, x)
    \end{equation*}
  \end{definition}

  Exploremos un poco las propiedades de la definición.
  La viabilidad indica únicamente que el flujo
  no puede exceder la capacidad del enlace.
  Simetría torcida
  es simplemente una conveniencia notacional
  (básicamente,
   si vamos ``contra la corriente''
   contabilizamos el flujo como negativo).
  La conservación indica que el flujo neto
  que entra a un vértice es nulo,
  y por simetría el que sale también:
  \begin{equation*}
    \sum_{y \in V} f(x, y)
      = \sum_{y \in V} -f(y, x)
      = -\sum_{y \in V} f(y, x)
      = 0
  \end{equation*}
  En el caso de circuitos eléctricos,
  la conservación es lo que se conoce como la ley de Kirchhoff.%
    \index{Kirchhoff, leyes de}
  Si no hay arco \(x y\),
  no puede haber flujo entre ellos,
  y \(f(x, y) = - f(y, x) = 0\).
  Como no se permite acumulación en los vértices intermedios,
  está claro que el flujo que sale de \(s\)
  debiera ser el flujo que entra en \(t\):
  \begin{equation*}
    \val(f) = \sum_{x \in V} f(x, t)
  \end{equation*}
  Esto lo demostraremos formalmente más adelante.

  Un problema obvio
  es obtener el valor máximo del flujo en una red.%
    \index{red!flujo maximo@flujo máximo}
  Este problema nos ocupará de ahora en adelante.

  Para conveniencia,
  definimos los flujos de entrada y salida de un vértice:
  \begin{align*}
    \inflow(v)
      &= \sum_{\mathclap{\substack{
			   u \in V	\\
			   f(u, v) > 0
	      }}} f(u, v) \\
    \outflow(u)
      &= \sum_{\mathclap{\substack{
			   v \in V	 \\
			   f(u, v) > 0
	      }}} f(u, v)
  \end{align*}
  Algunos autores anotan \(v^{-}\)
  para lo que llamamos \(\inflow(v)\),
  y similarmente \(v^{+}\) para \(\outflow(v)\).

  En estos términos,
  conservación se expresa simplemente
  como el flujo que entra a un vértice es igual al que sale
  si el vértice no es la fuente ni el sumidero:
  \begin{equation*}
    \inflow(v) = \outflow(v)
  \end{equation*}

% Fixme: (Paulina Silva <pasilva@alumnos.inf.utfsm.cl>) Agregar monito
%	 con vértice + flujos,
%	 vértice con entrantes + vertice con salientes

  Lo indicado en la figura~\ref{fig:flow-network}
  debe cumplir las condiciones para ser un flujo.
  La figura no da flujos ``contracorriente'',
  estamos suponiendo implícitamente
  que por ejemplo \(f(d, b) = - f(b, d) = -2\).
  Las condiciones de capacidad se cumplen,
  ya que por ejemplo \(f(c, d) = 1\) mientras \(c(c, d) = 7\).
  También debemos verificar conservación,
  por ejemplo que para el vértice \(d\)
  la suma de los flujos se anula:
  \begin{align*}
    f(a, d) + f(b, d) + f(c, d) + f(t, d)
      &= 1 + 2 + 1 - 4 \\
      &= 0
  \end{align*}
  El valor de este flujo
  es la suma de los flujos que salen de la fuente,
  vale decir:
  \begin{align*}
    \val(f)
      &= f(s, a) + f(s, b) + f(s, c) \\
      &= 8
  \end{align*}
  Resulta también,
  como esperábamos,
  que el flujo hacia el sumidero
  es igual al flujo que sale de la fuente:
  \begin{equation*}
    f(a, t) + f(d, t) + f(c, t)
      = 8
  \end{equation*}
  No contabilizamos flujos entre \(s\) y los demás vértices
  (respectivamente entre los otros vértices y \(t\))
  ya que no hay conexiones directas entre ellos.

\subsection{Trabajando con flujos}
\label{sec:trabajando-flujos}

  Usaremos una convención de \emph{suma implícita},%
    \index{red!suma implicita@suma implícita}
  en que si mencionamos un conjunto de vértices
  como argumento a \(f\),
  estamos considerando la suma de los flujos sobre ese conjunto,
  y similarmente para \(c\).
  Por ejemplo,
  al anotar \(f(X, Y)\),
  donde \(X\) e \(Y\) son conjuntos de vértices,
  entenderemos:
  \begin{equation*}
    f(X, Y) = \sum_{\substack{
		      x \in X \\
		      y \in Y
		   }} f(x, y)
  \end{equation*}
  En estos términos,
  la condición de conservación
  se reduce a \(f(V, x) = 0\) para todo \(x \notin \{s, t\}\)
  (recuérdese que por convención
   \(s\) es la fuente y \(t\) el sumidero de la red).
  Además,
  omitiremos las llaves al restar conjuntos de un solo elemento.
  Así,
  en \(f(s, V - s) = f(s, V)\)
  la notación
  \(V - s\) significa el conjunto \(V \smallsetminus \{s\}\).
  Esto simplifica mucho las ecuaciones que involucran flujos.
  El lema siguiente recoge varias de las identidades más comunes.
  La demostración queda como ejercicio.
  \begin{lemma}
    \label{lem:identidades}
    Sea \(D = (V, A)\) una red,
    y sea \(f\) un flujo en \(D\).
    Entonces:
    \begin{enumerate}
    \item
      Para todo \(X \subseteq V\),
      se cumple \(f(X, X) = 0\).
    \item
      Para todo \(X, Y \subseteq V\)
      se cumple \(f(X, Y) = - f(Y, X)\).
    \item
      Para todo \(X, Y, Z \subseteq V\) siempre que
      \(X \cap Y = \varnothing\).
      se cumplen:
      \begin{align*}
	f(X \cup Y, Z)
	  &= f(X, Z) + f(Y, Z) \\
	f(Z, X \cup Y)
	  &= f(Z, X) + f(Z, Y)
      \end{align*}
    \end{enumerate}
  \end{lemma}

  Como un ejemplo de uso de la notación
  y del lema~\ref{lem:identidades},
  demostraremos que \(\val(f) = f(V, t)\).
  Intuitivamente,
  lo que entra a la red por la fuente debe salir por el sumidero,
  ya que no se permiten acumulaciones entremedio.
  Formalmente:
  \begin{align*}
    f(V, t)
      & = f(V, V) - f(V, V - t) \\
      & = -f(V, V - t) \\
      & = f(V - t, V) \\
      & = f(s, V) + f(V - s - t, V) \\
      & = f(s, V) \\
      &= \val(f)
  \end{align*}
  En esto usamos el hecho:
  \begin{align*}
    f(V - s - t, V)
      &= \sum_{x \in V \smallsetminus \{s, t\}} f(x, V) \\
      &= -\sum_{x \in V \smallsetminus \{s, t\}} f(V, x) \\
      &= 0
  \end{align*}
  que sigue de conservación,
  ya que cada término de la última suma se anula.

  Las capacidades son los flujos máximos en la dirección indicada.
  Si hay tuberías de capacidades \(5\) de \(u\) a \(v\)
  y \(3\) de \(v\) a \(u\),
  el máximo flujo de \(u\) a \(v\) es \(5\)
  y el máximo de \(v\) a \(u\) es \(3\).
  Esto se traduce en que al aplicar la notación a capacidades
  los términos negativos se omiten.

\subsection{Método de Ford-Fulkerson}
\label{sec:ford-fulkerson}
\index{Ford-Fulkerson, metodo de@Ford-Fulkerson, método de}

  Presentaremos ahora una manera de obtener el flujo de máximo valor
  en una red.
  No lo llamaremos ``algoritmo'',
  ya que la estrategia general~%
    \cite{ford56:_max_flow_networks}
  puede implementarse de varias formas,
  con características diferentes.
  \begin{algorithm}[htbp]
    \DontPrintSemicolon

    Inicialice \(f\) en \(0\) \;
    \While{hay un camino aumentable \(p\)}{
      Aumente el flujo \(f\) a lo largo de \(p\) \;
    }
    \caption{El método de Ford-Fulkerson}
    \label{alg:ford-fulkerson}
  \end{algorithm}
  En el proceso introduciremos varias ideas importantes
  en muchos problemas relacionados con flujos.
  Supondremos que las capacidades son enteras,
  de otra forma puede ser que los métodos planteados
  no terminen nunca
  (aunque converjan hacia la solución).

  El método de Ford-Fulkerson es iterativo.
  Comenzamos con \(f(x, y) = 0\) para todo arco \(x y\),
  con un valor inicial cero.
  En cada iteración aumentamos el valor del flujo
  a través de identificar
  un \emph{camino aumentable}%
    \index{red!camino aumentable}
  (\emph{\foreignlanguage{english}{augmenting path}} en inglés,
   un camino entre \(s\) y \(t\) que no está en su máxima capacidad)
  y aumentamos el flujo a lo largo de este camino.
  Continuamos hasta que no se pueda encontrar
  otro camino aumentable.
  Por el teorema \emph{\foreignlanguage{english}{Max-Flow Min-Cut}}
  (teorema~\ref{theo:max-flow=min-cut},
   que demostraremos más adelante)
  al finalizar el valor del flujo es máximo.
  Si las capacidades
  están dadas por números enteros,
  los flujos también serán enteros.
  Los flujos no pueden crecer indefinidamente,
  los algoritmos terminan.

\subsection{Redes residuales}
\label{sec:red-residual}
\index{red!residual|textbfhy}

  Dada una red y un flujo \(f\),
  habrán arcos que admiten flujo adicional,
  y estos arcos con sus capacidades sin usar
  a su vez constituyen una red.
  Intuitivamente,
  esta nos dice cuáles son las posibles mejoras del flujo,
  así interesa analizar la relación entre esta red y la original.

  Más formalmente,
  supongamos una red \(D = (V, A)\),
  con capacidades \(c \colon A \rightarrow \mathbb{R^+}\),
  fuente \(s\) y sumidero \(t\).
  Sea \(f\) un flujo en \(D\),
  y consideremos un par de vértices \(x\) e \(y\).
  El flujo adicional que podemos enviar de \(x\) a \(y\)
  antes de sobrepasar la capacidad de ese enlace
  es la \emph{capacidad residual} del enlace \(x y\),
  que se anota \(c_f(x, y)\).
  Por ejemplo,
  si \(c(x, y) = 10\) y \(f(x, y) = 7\),
  podemos enviar un flujo adicional de \(3\) de \(x\) a \(y\)
  sin sobrepasar la capacidad de ese enlace,
  o podemos disminuir ese flujo en \(7\).
  O sea,
  para un enlace \(x y\)
  con flujo positivo \(f(x, y)\)
  tenemos una capacidad residual
  de \(c_f(x, y) = c(x, y) - f(x, y)\)
  de \(x\) a \(y\),
  y una capacidad residual
  de \(c_f(y, x) = f(x, y)\) de \(y\) a \(x\).
  De forma similar,
  para un enlace \(x y\) con flujo negativo \(f(x, y)\),
  podemos aumentar el flujo de \(x\) a \(y\)
  disminuyendo el flujo de \(y\) a \(x\),
  el máximo aumento posible es dejarlo en cero.
  La \emph{red residual} \(D_f = (V, A_f)\) inducida por \(f\)
  es simplemente la red formada por los enlaces
  y sus capacidades residuales
  (capacidad libre con la corriente,
   flujo a través del enlace en contracorriente),
  o sea \(A_f = \{u v \in V \times V \colon c_f (u, v) > 0\}\).
  En \(D_f\) todos los arcos pueden admitir flujos mayores a cero,
  los arcos de \(D_f\) son ya sea arcos de \(D\) o sus reversos.

  Conviene definir la suma de flujos \(f_1\) y \(f_2\):
  \begin{equation*}
    (f_1 + f_2)(u, v) = f_1 (u, v) + f_2 (u, v)
  \end{equation*}
  Nótese que esto no siempre es un flujo,
  ya que no necesariamente
  cumple las restricciones de nuestra definición.

  El siguiente lema relaciona flujos en \(D\) con flujos en \(D_f\).
  \begin{lemma}
    \label{lem:suma-flujos}
    Sea \(D = (V, A)\) una red con fuente \(s\) y sumidero \(t\),
    y \(f\) un flujo en \(D\).
    Sea \(D_f = (V, A_f)\) la red residual inducida por \(f\),
    y \(f'\) un flujo en \(D_f\).
    Entonces la suma \(f + f'\) es un flujo en \(D\),
    con valor \(\val(f + f') = \val(f) + \val(f')\).
  \end{lemma}
  \begin{proof}
    Primero,
    para verificar que es un flujo,
    debe cumplir las condiciones de la definición.
    Para simetría torcida tenemos:
    \begin{align*}
      (f + f')(x, y)
	&= f(x, y) + f' (x, y) \\
	&= -f(y, x) - f' (y, x) \\
	&= - (f + f')(y, x)
    \end{align*}
    Para las restricciones de capacidad,
    note que \(f' (x, y) \le c_f (x, y)\) para todo \(x, y \in V\)
    (incluso cuando \(f'(x, y) < 0\),
     o sea,
     el arco \(x y\) es contracorriente).
    Por tanto:
    \begin{align*}
      (f + f')(x, y)
	&= f(x, y) + f' (x, y) \\
	&\le f(x, y) + (c(x, y) - f(x, y)) \\
	&= c(x, y)
    \end{align*}
    Para conservación,
    al ser \(f\) y \(f'\) flujos,
    con \(x \ne s, t\) es \(f(x, V) = f'(x, V) = 0\),
    y \((f + f')(x, V) = 0\).

    Finalmente:
    \begin{align*}
      \val(f + f')
	&= (f + f')(s, V) \\
	&= f(s, V) + f' (s, V) \\
	&= \val(f) + \val(f')
    \end{align*}
    Como cumple con nuestra definición y nuestras convenciones,
    queda demostrado lo que perseguíamos.
  \end{proof}

\subsection{Caminos aumentables}
\label{sec:caminos-aumentables}
\index{red!camino aumentable|textbfhy}

  Dada una red \(D = (V, A)\) y un flujo \(f\),
  un \emph{camino aumentable}
  (en inglés \emph{\foreignlanguage{english}{augmenting path}})
  \(p\)
  es un camino dirigido entre \(s\) y \(t\)
  en la red residual \(D_f\).
  Por la definición de red residual,
  cada arco \(x y\) a lo largo de \(p\)
  admite flujo positivo de \(x\) a \(y\)
  sin violar la restricción de capacidad.
  Podemos aumentar el flujo a lo largo de \(p\) en:
  \begin{equation*}
    c_f(p) = \min_{(x, y) \in p} \{c_f (x, y)\}
  \end{equation*}
  sin sobrepasar la capacidad de ningún enlace.
  A \(c_f(p)\) se le llama la \emph{capacidad residual} de \(p\).
  Hemos demostrado:
  \begin{lemma}
    \label{lem:f_p}
    Sea \(D = (V, A)\) una red,
    sea \(f\) un flujo en \(D\),
    y sea \(p\) un camino aumentable en \(D_f\).
    Defina la función
    \(f_p \colon V \times V \rightarrow \mathbb{R}\) mediante:
    \begin{equation*}
      f_p (u, v) =
      \begin{cases}
	c_f (p)	 & \text{si \(x y\) está en \(p\)} \\
	-c_f (p) & \text{si \(y x\) está en \(p\)} \\
	0	 & \text{caso contrario}
      \end{cases}
    \end{equation*}
    Entonces \(f_p\) es un flujo en \(D_f\)
    con valor\/ \(\val(f_p) = c_f(p) > 0\).
  \end{lemma}

  De lo anterior tenemos:
  \begin{corollary}
    \label{cor:f.prime}
    Sea \(D = (V, A)\) una red,
    \(f\) un flujo en \(D\)
    y \(p\) un camino aumentable en \(D_f\).
    Sea \(f_p\) como definido en el lema~\ref{lem:f_p},
    y defina \(f' \colon V \times V \rightarrow \mathbb{R}\)
    como \(f' = f + f_p\).
    Entonces \(f'\) es un flujo en \(D\),
    y su valor es:
    \begin{equation*}
      \val(f')
	= \val(f) + \val(f_p)
	> \val(f)
    \end{equation*}
  \end{corollary}
  \begin{proof}
    Inmediata por los lemas~\ref{lem:suma-flujos} y~\ref{lem:f_p}.
  \end{proof}

  Para clarificar estas ideas,
  véanse las figuras~\ref{fig:red} y~\ref{fig:red-flujos}.
  La figura~\ref{fig:red} muestra una red,
  \begin{figure}[htbp]
    \centering
    \pgfimage{images/red}
    \caption{Una red}
    \label{fig:red}
  \end{figure}
  \begin{figure}[htbp]
    \centering
    \subfloat[Un flujo]{
      \pgfimage{images/red-flujo}
      \label{subfig:red-flujo}
    }

    \subfloat[La red residual y un camino aumentable]{
      \pgfimage{images/red-residual}
      \label{subfig:red-residual}
    }
    \caption{Flujo y red residual
	     en la red de la figura~\ref{fig:red}}
    \label{fig:red-flujos}
  \end{figure}
  la figura~\ref{subfig:red-flujo}
  muestra un flujo en la red de la figura~\ref{fig:red}.
  La figura~\ref{subfig:red-residual}
  muestra la red residual con ese flujo
  con un camino aumentable marcado.
  Hay otros caminos aumentables,
  como \((s, a, b, c, t)\).
  Nótese que en la red residual tenemos enlaces contracorriente
  cuyas capacidades son el flujo actual.
  Esto significa que podemos enviar hasta ese flujo contracorriente
  (disminuyendo el flujo actual)
  a través de ese enlace.
  \begin{figure}[htbp]
    \centering
    \pgfimage{images/red-flujo-2}
    \caption{El flujo aumentado según el camino aumentable
	     de la figura~\ref{subfig:red-residual}}
    \label{fig:red-flujo-aumentado}
  \end{figure}
  Este camino aumentable tiene capacidad residual~\(4\),
  y la figura~\ref{fig:red-flujo-aumentado}
  muestra el resultado de aumentar el flujo.

\subsection{Cortes}
\label{sec:cortes}
\index{red!corte}

  La idea del método de Ford-Fulkerson
  es hallar sucesivamente un camino aumentable
  y aumentar el flujo a lo largo de él.
  Cuando este proceso termina
  por no hallar un camino aumentable,
  tenemos un flujo de valor máximo.
  Al discutir estos métodos
  se restringen las capacidades a números naturales.
  De partida,
  capacidades negativas no tienen sentido,
  y resulta que si las capacidades son irracionales
  hay casos en que los algoritmos no terminan nunca
  (convergen hacia la solución,
   pero nunca la alcanzan).

  La correctitud de estos métodos
  la garantiza
  el teorema \foreignlanguage{english}{Max-Flow Min-Cut},%
    \index{max-flow min-cut, teorema}
  que es nuestro próximo objetivo.
  Primeramente consideraremos el concepto de un \emph{corte}
  (en inglés \emph{\foreignlanguage{english}{cut}})
  en una red.
  Un corte \((S, T)\) corresponde simplemente
  a una partición de los vértices de la red
  en conjuntos \(S\) y \(T = V \smallsetminus S\)
  tal que \(s \in S\) y \(t \in T\),
  véase por ejemplo la figura~\ref{fig:red-cut}.
  \begin{figure}[htbp]
    \centering
    \pgfimage{images/red-cut}
    \caption{Un corte en la red de la figura~\ref{fig:red}
	     con el flujo de la figura~\ref{subfig:red-flujo}}
  \label{fig:red-cut}
  \end{figure}
  Si \(f\) es un flujo,
  el \emph{flujo neto} a través del corte \((S, T)\)
  se define como \(f(S, T)\).
  En nuestro caso (figura~\ref{fig:red-cut}) el flujo neto
  es \(9 - 4 + 7 = 12\).
  La \emph{capacidad} del corte \((S, T)\) es \(c(S, T)\),
  que en la misma figura correspondería a \(12 + 14 = 26\).
  También se define un \emph{corte mínimo}
  (en inglés \emph{\foreignlanguage{english}{minimum cut}})
  como un corte de capacidad mínima.
  El flujo neto que cruza el corte incluye flujos de \(S\) a \(T\)
  (aportes positivos)
  y flujos de \(T\) a \(S\)
  (aportes negativos).
  Por otro lado,
  en las capacidades se incluyen solo las de arcos de \(S\) a \(T\).
  El lema siguiente
  relaciona los flujos
  con las capacidades a través de cortes de la red.
  \begin{lemma}
    \label{lem:flujos-cortes}
    Sea \(f\) un flujo en la red \(D = (V, A)\)
    con fuente \(s\) y sumidero \(t\),
    y sea \((S, T)\) un corte de la red.
    Entonces el flujo neto a través del corte es el valor del flujo.
  \end{lemma}
  \begin{proof}
    Por conservación de flujo \(f(S - s, V) = 0\),
    y aplicando el lema~\ref{lem:identidades},
    tenemos:
    \begin{align*}
      f(S, T)
	&= f(S, V) - f(S, S) \\
	&= f(S, V) \\
	&= f(s, V) + f(S - s, V) \\
	&= f(s, V) \\
	&= \val(f)
      \qedhere
    \end{align*}
  \end{proof}
  Un resultado inmediato del lema~\ref{lem:flujos-cortes}
  es el resultado que demostramos antes,
  que el flujo al sumidero es el valor del flujo en la red:
  Basta tomar el corte \((V - t, \{t\})\) para ello.

  Pero también podemos deducir:
  \begin{corollary}
    \label{cor:flow<cut}
    En una red \(D\) con un corte \((S, T)\),
    el valor de cualquier flujo
    está acotado por la capacidad del corte.
  \end{corollary}
  \begin{proof}
    Sea \((S, T)\) un corte cualquiera
    de \(D\) y sea \(f\) un flujo.
    Por el lema~\ref{lem:flujos-cortes}
    y las restricciones de capacidad:
    \begin{align*}
      \val(f)
	& =   f(S, T) \\
	& =   \sum_{x \in S} \sum_{y \in T} f(x, y) \\
	& \le \sum_{x \in S} \sum_{y \in T} c(x, y) \\
	& =   c(S, T)
      \qedhere
    \end{align*}
  \end{proof}
  Una consecuencia inmediata del corolario~\ref{cor:flow<cut}
  es que el valor de un flujo está acotado
  por la capacidad de un corte mínimo.
  El teorema siguiente
  nos dice que el flujo máximo en realidad es esta cota.

  \begin{theorem}[Max-Flow Min-Cut]
    \index{max-flow min-cut, teorema|textbfhy}
    \label{theo:max-flow=min-cut}
    Si \(f\) es un flujo en la red \(D = (V, A)\)
    con fuente \(s\) y sumidero \(t\),
    entonces las siguientes son equivalentes:
    \begin{enumerate}[label=(\arabic{*})]
    \item\label{item:mfmc:f-max}
      \(f\) es un flujo máximo en \(D\).
    \item\label{item:mfmc:res}
      La red residual \(D_f\) no contiene
      caminos aumentables.
    \item\label{item:mfmc:val=cap}
      \(\val(f) = c(S, T)\) para algún corte \((S, T)\) de \(D\).
    \end{enumerate}
  \end{theorem}
  \begin{proof}
    Demostramos la equivalencia
    a través de un ciclo de implicancias.%
      \index{demostracion@demostración!ciclo de implicancias}
    \begin{description}
    \item[\boldmath
	  \ref{item:mfmc:f-max}~\(\implies\)~\ref{item:mfmc:res}:
	  \unboldmath]
      Por contradicción.
      Supongamos en contrario que \(f\) es máximo en \(D\),
      pero que \(D_f\)
      tiene un camino aumentable \(p\).
      Por el corolario~\ref{cor:f.prime}
      sabemos que \(f + f_p\) es un flujo,
      cuyo valor es mayor que \(\val(f)\),
      lo que contradice la suposición de que \(f\) es máximo.
    \item[\boldmath
	  \ref{item:mfmc:res}~\(\implies\)~\ref{item:mfmc:val=cap}:
	  \unboldmath]
      Supongamos que \(D_f\)
      no tiene camino aumentable,
      es decir,
      no hay camino dirigido de \(s\) a \(t\) en \(D_f\).
      Definamos:
      \begin{align*}
	S & = \{x \in V \colon
		  \text{\ hay un camino de \(s\) a \(x\)
			en \(D_f\)\}} \\
	T & = V \smallsetminus S
      \end{align*}
      Entonces \((S, T)\) es un corte de \(D\),
      ya que obviamente \(s \in S\) y \(t \notin S\)
      ya que no hay camino de \(s\) a \(t\) en \(D_f\)
      por suposición.
      Para cada par de vértices \(x \in S\) e \(y \in T\) tenemos
      \(f(x, y) = c(x, y)\),
      dado que de lo contrario \(x y \in A_f\) y habría un camino
      \(s \rightsquigarrow x \rightarrow y\)
      y así \(y\) estaría en \(S\).
      Por el lema~\ref{lem:flujos-cortes} es:
      \begin{equation*}
	\val(f)
	  = f(S, T)
	  = c(S, T)
      \end{equation*}
    \item[\boldmath
	  \ref{item:mfmc:val=cap}~%
	     \(\implies\)~\ref{item:mfmc:f-max}:
	  \unboldmath]
      Por el corolario~\ref{cor:flow<cut},
      \(\val(f) \le c(S, T)\) para todo corte \((S, T)\).
      La condición \(\val(f) = c(S, T)\)
      entonces asegura que el flujo es máximo.
    \qedhere
    \end{description}
  \end{proof}

  Este teorema sirve para demostrar
  la validez del método de Ford-Fulkerson:%
    \index{Ford-Fulkerson, metodo de@Ford-Fulkerson, método de}
  Si en una iteración
  no hay un camino aumentable,
  quiere decir que el flujo actual es máximo.
  Una manera razonable
  de buscar un camino aumentable
  es usar búsqueda a lo ancho en la red residual.
  A esta forma de implementar el método de Ford-Fulkerson
  se le conoce como el algoritmo de Edmonds-Karp~%
    \cite{edmonds72:_theor_improv_algor_effic_networ_flow_probl}.%
    \index{Edmonds-Karp, algorithmo de}
  Una discusión detallada del problema,
  incluyendo historia de los algoritmos,
  presenta Wilf~\cite[capítulo~3]{wilf03:_algor_compl}.

% Fixme: Agregar un ejemplo completo (p.ej. de ayudantía)
% Sugerido por Franco Castro <fcastro@alumnos.inf.utfsm.cl>

%%% Local Variables:
%%% mode: latex
%%% TeX-master: "clases"
%%% End:

% correctitud-programas.tex
%
% Copyright (c) 2012-2014 Horst H. von Brand
% Derechos reservados. Vea COPYRIGHT para detalles

\chapter{Correctitud de programas}
\label{cha:programas}
\index{programa!correctitud}

  Cada día dependemos más de programas computacionales.
  Lamentablemente,
  muchos programas han demostrado ser incorrectos,
  errores de programación han dado lugar a costos inmensos
  (el reporte de Calude, Calude y Marcus~%
     \cite{calude07:_provin_progr}
   da algunos ejemplos notorios,
   pero todos hemos sufrido en menor o mayor medida
   a causa del comportamiento erróneo de programas).
  Dijkstra~\cite{dijkstra89:_cruelty_teach_comput_scien}%
    \index{Dijkstra, Edsger W.}
  observa que la computación
  significa un cambio radical en nuestro mundo,
  al introducir artefactos cuyo funcionamiento no es continuo,
  y más aún dar lugar a fenómenos que abarcan órdenes de magnitud
  absolutamente inimaginables en otras disciplinas.
  Veremos cómo aplicar técnicas de razonamiento matemático
  a la tarea de asegurar que un programa funcione como se espera.

\section{Lógica de Hoare}
\label{sec:logica-de-hoare}
\index{Hoare, logica de@Hoare, lógica de}

% Lógica de Hoare
%   \cite{floyd67:_assig_meanin_progr,
%	  hoare69:_axiom_basis_comput_progr,
%	  gries87:_scien_progr}
%
% Correctitud parcial/total
% Técnicas de razonamiento
% Ejemplos ~~--> Programming Pearls
% Herramientas como splint <http://www.splint.org>
% o sparse <http://www.kernel.org/pub/software/devel/sparse>
% (a la pasada cosas como doxygen, valgrind, ...?)

  Ya a fines de los años 60,
  Floyd~\cite{floyd67:_assig_meanin_progr}%
    \index{Floyd, Robert W.}
  y luego Hoare~\cite{hoare69:_axiom_basis_comput_progr}%
    \index{Hoare, Tony (Sir Charles Anthony Richard)}
  propusieron demostrar formalmente
  que los programas cumplen con sus especificaciones,
  idea que fue luego desarrollada por Dijkstra~%
    \cite{dijkstra76:_discip_progr}
  y de alguna manera completada por Gries~%
    \cite{gries87:_scien_progr}.%
    \index{Gries, David}
  Posiblemente la formalización más completa es la que da Apt~%
    \cite{apt81:_ten_years_hoare_logic,
	  apt84:_ten_years_hoare_logic}.
  Una descripción formal reciente
  (con orientación pedagógica)
  dan Gordon y~Collavizza~%
    \cite{gordon10:_forwar_hoare}.
  Jones~%
    \cite{jones03:_early_searc_tract_ways_reason_about_progr}
  resume la historia desde sus comienzos.
  Una crítica del área da Gutmann en su tesis~%
    \cite[capítulos~4 y~5]{gutmann04:_crypt_sec_arch}.

  Demostrar correctitud de programas aumenta la confianza
    \index{programa!verificacion formal@verificación formal}
  de que funcionen correctamente,
  pero las técnicas formales son complejas y caras
  de aplicar en general.
  Aún con ayuda del computador
  hoy solo es posible verificar completamente
  programas relativamente pequeños.
  Tal vez el ejemplo más notable de verificación
  es la del micronúcleo seL4~%
    \cite{Klein:2010:SFV:1743546.1743574, klein11:_l4},%
    \index{seL4}
  un sistema de unas \(8\,700\)~líneas de C
  de las cuales se verificaron formalmente unas \(7\,500\).
  Para contraste,
  el núcleo de un sistema operativo de uso común,
  como Linux,
    \index{Linux}
  tiene casi doce millones de líneas de código.

  Sigue vigente el famoso dicho:
  \hybridblockquote{english}
      [Donald E. Knuth~%
	 \cite{knuth77:_notes_emde_boas_const_prior_deques}]{%
    Beware of bugs in the above code;
    I have only proved it correct,
    not tried it%
  }.%
    \index{Knuth, Donald E.}
  Al efecto,
  Berry~\cite{berry92:_academ_legit_softw_eng_discipl}
  cita el ejemplo de un programita de \(25\)~líneas de Algol%
    \index{Algol (lenguaje de programacion)@Algol (lenguaje de programación)}
  especificado
  y demostrado correcto informalmente en 1969
  por Naur~\cite{naur69:_progr_action_clust},%
    \index{Naur, Peter}
  publicado sin probarlo.
  El año siguiente,
  revisando la publicación de Naur Leavenworth~%
    \cite{leavenworth70:_19420}
  encontró un error que habría sido evidente
  de ejecutar el programa.
  Aún después,
  London~\cite{london71:_softw_reliab_proving_prog_correct}
  halló tres errores adicionales que habrían sido fáciles de hallar
  probando el programa.
  Ofreció un programa corregido,
  acompañando esta vez una demostración formal de correctitud.
  Igualmente,
  Goodenough y Gerhart~%
    \cite{goodenough75:_towar_theor_test_data_selec}
  hallaron tres errores adicionales en el programa de London,
  que nuevamente habrían sido evidentes al probar el programa.
  A su vez,
  las especificaciones dieron lugar
  a una comedia de errores adicional.
    \index{programa!verificacion formal@verificación formal!comedia}
  El hecho que incluso ejemplos mínimos,
  de las manos de los máximos expertos del área,
  den lugar a tal espectáculo
  es prueba patente de lo complejo del tema.

  Siempre debe tenerse presente el dicho:
  \hybridblockquote{english}
      [C. A. R. Hoare~\cite{hoare81:_old_clothes}]{%
    There are two ways of constructing a software design:
    One way is to make it so simple
    that there are \emph{obviously} no deficiencies,
    and the other way is to make it so complicated
    that there are no \emph{obvious} deficiencies.
    The first method is far more difficult.%
  }.%
    \index{Hoare, Tony (Sir Charles Anthony Richard)}
  En la misma línea nos advierten:
  \hybridblockquote{english}
      [D. E. Knuth~%
	 \cite{knuth74:_struc_progr_goto, knuth89:_errors_TeX}]{%
    Premature optimization is the root of all evil.%
  }.%
    \index{Knuth, Donald E.}%
    \index{premature optimization@\emph{\foreignlanguage{english}{premature optimization}}}

\section{Lógica de Hoare}
\label{sec:logica-Hoare}
\index{Hoare, logica de@Hoare, lógica de|textbfhy}

  La idea tras el trabajo de Hoare y sus sucesores
  es que durante su ejecución un programa pasa por \emph{estados}
  bien definidos,
  y que estos estados cambian conforme se ejecutan instrucciones.
  Mediante \emph{predicados}
  sobre los valores de las variables
  se describen los aspectos de interés de cada estado,
  y se busca demostrar que dadas ciertas condiciones iniciales
  al finalizar el programa se cumple lo solicitado
  (\emph{correctitud parcial})%
    \index{programa!correctitud parcial|textbfhy}
  y,
  generalmente en forma separada de lo anterior,
  que el programa siempre termina con el resultado correcto
  (\emph{correctitud total}).%
    \index{programa!correctitud total|textbfhy}
  Esto involucra relacionar los predicados antes y después
  de las distintas instrucciones del lenguaje empleado.

  La forma general de expresar los resultados es:
  \begin{equation*}
    \{P\} \text{\ programa\ } \{Q\}
  \end{equation*}
  para expresar que si se cumple \(P\) antes del programa,
  cuando éste termine se cumple \(Q\).

\section{Búsqueda binaria}
\label{sec:busqueda-binaria}
\index{busqueda binaria@búsqueda binaria}

  Como un ejemplo,
  desarrollaremos paso a paso una función de búsqueda binaria,
  siguiendo la exposición de Bentley~%
    \cite[capítulo 4]{bentley00:_progr_pearl}.%
    \index{Bentley, Jon}
  Es también de interés histórico,
    \index{busqueda binaria@búsqueda binaria!historia}
  la primera exposición del programa fue publicada en 1946,
  la primera versión correcta recién en 1962.
  Bentley reporta haber usado el programa
  como ejercicio en cursos para programadores profesionales.
  Al cabo de una hora,
  revisaban sus programas durante media hora adicional,
  y consistentemente \(90\)\% de ellos hallaban errores.
  Bentley reconoce
  que no está seguro si el \(10\)\% restante era correcto.

  Nuestro problema es determinar si el arreglo ordenado
  \(x[1..N]\) contiene el valor \(t\).
  Más precisamente,
  sabemos que \(N \ge 0\)
  y que \(x[1] \le x[2] \le \dotso \le x[N]\).
  Los tipos de \(t\)
  y de los elementos de \(x\) son los mismos,
  el pseudocódigo debiera funcionar igualmente bien para enteros,
  reales o \foreignlanguage{english}{strings}.
  La respuesta es el entero \(p\);
  si \(p\) es cero
  \(t\) no está en \(x[1..N]\),
  en caso contrario \(1 \le p \le N\)
  y \(t = x[p]\).

  Búsqueda binaria resuelve este problema
  siguiendo la pista a un rango dentro del arreglo
  en el que \(t\) tiene que estar
  si es que está en el arreglo.
  Inicialmente el rango es el arreglo completo.
  El rango se acorta comparando \(t\)
  con el elemento medio del rango
  y descartando la mitad que no puede contenerlo.
  El proceso continúa hasta que se halle el elemento
  o el rango esté vacío.

\subsection{Escribiendo el programa}
\label{sec:busqueda-binaria-escribir}

  La idea clave es que siempre sabemos que si \(t\)
  está en alguna parte de \(x[1..N]\),
  está en un cierto rango.
  Usaremos \(\mathrm{DebeEstar}(\mathit{rango})\)
  para significar que si \(t\) está en el arreglo,
  debe estar en el \(\mathit{rango}\) indicado.
  Con esto,
  la descripción informal
  se puede transformar
  en el esbozo de programa~\ref{alg:bs-sketch-1}.
  \begin{algorithm}[H]
    \DontPrintSemicolon
    Inicialice \(\mathit{rango}\) para designar \(x[1..N]\) \;
    \Loop{
      \{ Invariante: \(\mathrm{DebeEstar}(\mathit{rango})\) \} \;
      \If{\(\mathit{rango}\) es vacío}{
	Retorne que \(t\) no está en el arreglo \;
      }
      Calcule \(m\), el punto medio de \(\mathit{rango}\) \;
      Use \(m\) como prueba para encoger el rango,
	si halla \(t\) en el proceso retorne su posición \;
    }
    \caption{Esbozo de búsqueda binaria}
    \label{alg:bs-sketch-1}
  \end{algorithm}
  Parte crucial de nuestro esbozo es el \emph{invariante de ciclo},
    \index{programa!invariante de ciclo|textbfhy}
  que encerramos entre llaves.
  Se cumple siempre al comienzo de cada ciclo
  (de allí su nombre),
  y formaliza nuestra idea intuitiva presentada antes.

  Refinamos el esbozo,
  asegurándonos que nuestras acciones respetan el invariante.
  Primeramente debemos decidir
  cómo representar el \(\mathit{rango}\).
  Usaremos dos índices \(i\) (inferior)
  y \(s\) (superior) para el rango \(i..s\).
  (Hay otras opciones,
   como inicio y largo.)
  El siguiente paso es la inicialización:
  ¿Qué valores deben tener \(i\) y \(s\) para que
  \(\mathrm{DebeEstar}(i, s)\) se cumpla inicialmente?
  La respuesta obvia es \(i = 1\) y \(s = N\):
  \(\mathrm{DebeEstar}(1, N)\) dice que si \(t\) está en \(x\),
  está en \(x[1..N]\),
  exactamente lo que sabemos al comenzar el programa.
  El siguiente paso es verificar si el rango es vacío,
  cosa que se da siempre que \(i > s\).
  Hallar el punto medio del rango es:
  \begin{equation*}
    m = \left\lfloor \frac{i + s}{2} \right\rfloor
  \end{equation*}
  Uniendo las piezas
  obtenemos el segundo esbozo~\ref{alg:bs-sketch-2}.
  Para evitar confusiones con la relación de igualdad u otras,
  en nuestros algoritmos
  usaremos \(\leftarrow\) para indicar asignación.
  \begin{algorithm}
    \DontPrintSemicolon

    \(i \leftarrow 1\); \(s \leftarrow N\) \;
    \Loop{
      \{ Invariante: \(\mathrm{DebeEstar}(i, s)\) \} \;
      \If{i > s}{
	\(p \leftarrow 0\); \KwBreak \;
      }
      \(m \leftarrow \lfloor (i + s) / 2 \rfloor\) \;
      Use \(m\) como prueba para encoger el rango \(i .. s\),
	si halla \(t\) en el proceso retorne su posición \;
    }
    \caption{Búsqueda binaria: Segundo esbozo}
    \label{alg:bs-sketch-2}
  \end{algorithm}
  Resta refinar la última acción del ciclo.
  Corresponderá a comparar \(t\) con \(x[m]\)
  y tomar la acción apropiada:

  \RestyleAlgo{plain}
  \begin{algorithm}[H]
    \DontPrintSemicolon

    \lCase{\(t < x[m]\):}{Acción a} \;
    \lCase{\(t = x[m]\):}{Acción b} \;
    \lCase{\(t > x[m]\):}{Acción c} \;
  \end{algorithm}
  \RestyleAlgo{algoruled}

  \noindent
  Sabemos que la acción correcta
  en el segundo caso es asignar \(m\) a \(p\)
  y quebrar el ciclo.
  En el primer caso el rango se reduce a \(i .. m - 1\),
  en el tercero a \(m + 1 .. s\).
  Esto se logra asignando a \(s\) o \(i\),
  respectivamente.
  Expresado en términos de las estructuras de control comunes
  resulta~\ref{alg:bs-sketch-3}.
  \begin{algorithm}
    \DontPrintSemicolon
    \(i \leftarrow 1\); \(s \leftarrow N\) \;
    \Loop{
      \{ Invariante: \(\mathrm{DebeEstar}(i, s)\) \} \;
      \If{i > s}{
	\(p \leftarrow 0\); \KwBreak \;
      }
      \(m \leftarrow \lfloor (i + s) / 2 \rfloor\) \;
      \uIf{\(t < x[m]\)}{
	\(s \leftarrow m - 1\) \;
       }
       \eIf{\(t = x[m]\)}{
	 \(p \leftarrow m\); \KwBreak \;
       }
       {
	 \(i \leftarrow m + 1\) \;
       }
    }
    \caption{Búsqueda binaria: Pseudocódigo final}
    \label{alg:bs-sketch-3}
  \end{algorithm}
  En rigor,
  solo hemos demostrado
  (muy informalmente,
   claro está)
  que si el ciclo termina entrega el resultado correcto.
  Debemos asegurar además que el ciclo siempre termina,
  vale decir,
  que el largo del rango disminuye en cada paso.
  Como el largo es un entero,
  no puede disminuir indefinidamente.

  Igualmente,
  en implementaciones en máquinas reales
  el algoritmo~\ref{alg:bs-sketch-3}
  tiene un error fatal:
  Si la suma de \(i + s\)
  es mayor que el máximo entero representable,
  el cálculo del nuevo valor de \(m\) falla.
  Esto puede sonar a sofisma,
  pero fue un error real en la biblioteca de Java.
  La solución correcta es escribir esa línea como
  \(m \leftarrow i + \lfloor (s - i) / 2 \rfloor\).
  De esa forma,
  las variables y valores intermedios están acotados por \(n\),
  y no hay rebalses.

  Sea \(l\) el largo del rango al comenzar un ciclo.
  Llamemos \(l'\) al largo después del ciclo,
  suponiendo \(t \ne x[m]\).
  Entonces:
  \begin{equation*}
    l'
      = \begin{cases}
	  (m - 1) - i + 1
	    = (\lfloor (i + s) / 2 \rfloor - 1) - i + 1
	    \le (s - i) / 2
	    = (l + 1) / 2 & \text{si \(t < x[m]\)} \\
	  s - (m + 1) - 1
	    = s - (\lfloor (i + s) / 2 \rfloor + 1) - 1
	    \le s - (i + s) / 2 - 1
	    = l / 2 - 3 / 2
	    < l & \text{si \(t > x[m]\)}
	\end{cases}
  \end{equation*}
  En el caso \(t < x[m]\),
  si \(l > 1\) claramente \(l' \le (l + 1) / 2 < l\),
  pero para \(l = 1\) resulta la cota inútil \(l' \le 1\).
  En esta última situación
  es \(i = s\),
  y el algoritmo asigna \(s = m - 1 = i - 1\),
  cumpliendo la condición de término.
  Cada iteración
  nos acerca a la condición que hace terminar el ciclo,
  el programa termina.

  Es sencillo ahora traducir el pseudocódigo~\ref{alg:bs-sketch-3}
  a un lenguaje de programación como C,%
    \index{C (lenguaje de programacion)@C (lenguaje de programación)}
  ver el listado~\ref{lst:busqueda-binaria}.
  Nuestra versión incluye la precaución contra rebalses aritméticos
  mencionada antes.
  \lstinputlisting[language=C,
		   caption={Búsqueda binaria en C},
		   label=lst:busqueda-binaria]
		   {code/binary-search.c}
  Nuestro desarrollo cuidadoso
  da gran confianza de que el programa resultante sea correcto.
  El registrar el invariante del ciclo ayudará a nuestros lectores
  convencerse de ello.

  Para no caer en un programa demostrado correcto
  que falla con el primer caso de prueba
    \index{programa!casos de prueba}
  (como se narra en la introducción del presente capítulo),
  usamos el computador para lo que es mejor:
  Tareas rutinarias.
  Un pequeño programa bombardea esta función con casos de prueba:
  Se crea un arreglo de \(22\) elementos
  que se llena con los valores \(0\) a \(21\),
  luego se buscan los valores de \(1\) a \(N\) en él
  dando el rango de \(1\) a \(N\) para la búsqueda,
  con \(N\) variando de \(1\) a \(20\).
  Después se buscan los valores \(0\) y \(N + 1\)
  (ambos presentes,
   pero fuera del rango indicado),
  y finalmente los valores de \(0,5\) a \(N + 0,5\)
  con paso \(1\).
  Con esto cubrimos valores populares para errores:
  \(1\), \(2\), \(3\),
  algunas potencias de \(2\),
  números que difieren de potencias de \(2\) en \(1\).
  Y se cumplió lo indicado
  respecto de que los errores se cometen
  principalmente en las partes simples
  del programa:
  Algunos de los casos de prueba iniciales fallaron,
  estaban escritos incorrectamente\ldots

\section{Exponenciación}
\label{sec:programa-exponenciar}
\index{programa!exponenciar}

  El calcular una potencia vía multiplicaciones sucesivas
  viene directamente de la definición de potencia.
  Sin embargo,
  hay métodos más eficientes.
  Partiendo de la identidad:
  \begin{equation*}
    n = d \cdot \left\lfloor \frac{n}{d} \right\rfloor + n \bmod d
  \end{equation*}
  donde \(n \bmod d\) es el resto de la división de \(n\) por \(d\),
  podemos expresar potencias:
  \begin{equation}
    \label{eq:exponenciacion-binaria}
    x^n
      = \left( x^{\lfloor n / 2 \rfloor} \right)^2
	  \cdot x^{n \bmod 2}
  \end{equation}
  Tenemos inmediatamente el algoritmo recursivo~%
    \ref{alg:exponenciacion-binaria-recursivo}.
  \begin{algorithm}
    \DontPrintSemicolon
    \SetKwFunction{Power}{power}

    \KwFunction \Power\FuncSty{(}\ArgSty{\(x, \; n\)}\FuncSty{)} \;
    \BlankLine
    \uIf{\(n = 0\)}{\Return{\(1\)}}
    \Else{
      \(s \leftarrow \Power(x, \; \lfloor n / 2 \rfloor)\) \;
      \(r \leftarrow s^2\) \;
      \If{\(n \bmod 2 = 1\)}{\(r \leftarrow r \cdot x\)}
      \Return{\(r\)} \;
    }
    \caption{Exponenciación binaria recursiva}
    \label{alg:exponenciacion-binaria-recursivo}
  \end{algorithm}
  La justificación
  es que suponiendo que la función \(\mathtt{power}(x, k)\)
  hace correctamente su trabajo para \(k < n\),
  por la identidad~\eqref{eq:exponenciacion-binaria}
  calcula correctamente la potencia \(n\).
  Para justificar que termina,
  vemos que cada vez se invoca la función recursivamente
  con un valor menor para el parámetro \(n\),
  por lo que tarde o temprano llega al valor \(0\),
  que no involucra recursión.
  Esto es en esencia una demostración por inducción fuerte.

  Una versión alternativa
  (no recursiva)
  del algoritmo~\ref{alg:exponenciacion-binaria-recursivo}
  es el algoritmo~\ref{alg:exponenciacion-binaria}.
  Usamos operaciones con bits
  al estilo C~\cite{kernighan88:_c_progr_lang}.
  \begin{algorithm}
    \DontPrintSemicolon
    \SetKwFunction{Power}{power}

    \KwFunction \Power\FuncSty{(}\ArgSty{\(x, \; n\)}\FuncSty{)} \;
    \BlankLine
    \If{\(n = 0\)}{
      \Return{\(1\)} \;
    }
    \BlankLine
    \For{\(i \leftarrow 0\);
	 \(n \mathbin{\&} (1 \ll i)\);
	 \(i \leftarrow i + 1\)}{
      \tcc{Nada}
    }
    \tcc{Ahora \((i \ge 0) \wedge (2^i \le n < 2^{i + 1})\)} \;
    \(r \leftarrow x\) \;
    \For{; \(i > 0\); \(i \leftarrow i - 1\)}{
      \tcc{Invariante: \(u = \lfloor n / 2^i \rfloor\),
		       \(v = n \bmod 2^i\),
	    \((r = x^u) \wedge (x^n = r^{2^i} \cdot x^v)\)}
      \(r \leftarrow r^2\) \;
      \If{\(n \mathbin{\&} (1 \ll (i - 1))\)}{
	\(r \leftarrow r \cdot x\) \;
      }
    }
    \tcc{Del invariante con \(i = 0\) resulta \(r = x^n\)}
    \Return{\(r\)} \;
    \caption{Exponenciación binaria no recursiva}
    \label{alg:exponenciacion-binaria}
  \end{algorithm}

  Un ejemplo adicional es el algoritmo~\ref{alg:sqrt-integer}%
    \index{programa!raiz cuadrada entera@raíz cuadrada entera}
  para obtener la raíz cuadrada entera,
  vale decir,
  calcular \(\lfloor \sqrt{n} \rfloor\).
  Todas las variables del algoritmo son enteras.
  Es simple verificar el invariante,
  y con la condición de término del ciclo
  tenemos que \(u^2 \le n < (u + 1)^2\),
  vale decir,
  \(u \le \sqrt{n} < u + 1\),
  o sea \(u = \lfloor \sqrt{n} \rfloor\).
  En cada ciclo aumenta \(u\) o disminuye \(v\),
  por lo que el algoritmo termina.

  \begin{algorithm}
    \DontPrintSemicolon
    \SetKwFunction{ISqrt}{isqrt}

    \KwFunction \ISqrt \FuncSty{(}\ArgSty{\(n\)}\FuncSty{)} \;
    \BlankLine
    \(u \leftarrow 0\) \;
    \(v \leftarrow n + 1\) \;
    \While{ \(u + 1 \ne v\)}{
      \tcc{Invariante: \(u^2 \le n \le v^2\) y \(u + 1 \le v\)}
      \(x \leftarrow \lfloor (u + v) / 2 \rfloor\) \;
      \uIf{ \(x^2 \le n\) }{
	\(u \leftarrow x\) \;
      }
      \Else{
	\(v \leftarrow x\) \;
      }
    }
    \Return{\(u\)} \;
    \caption{Cálculo de $\lfloor \sqrt{n} \rfloor$}
    \label{alg:sqrt-integer}
  \end{algorithm}

\section{Algunos principios}
\label{sec:algunos-principios}

  El ejemplo de búsqueda binaria
  ilustra las fortalezas de la verificación de programas:
  El problema a resolver es importante y requiere código cuidadoso,
  lo desarrollamos guiados por ideas de verificación,
  y el análisis de correctitud usa herramientas generales.
  En todo caso,
  en la práctica el nivel de detalle del desarrollo
  será substancialmente menor.
  Algunos principios generales resultan de la discusión:
  \begin{description}
  \item[Afirmaciones:]\index{programa!afirmaciones}
    Permiten expresar las relaciones entre datos de entrada,
    variables internas y resultados en forma precisa.
  \item[Estructuras de control secuenciales:]%
      \index{programa!estructuras de control}
    La estructura de control más simple es ejecutar una acción,
    seguida por otra.
    Entendemos un programa por afirmaciones entre las instrucciones
    y razonamos de una a la siguiente.
  \item[Estructuras de selección:]
    Durante la ejecución,
    se toma una de varias acciones.
    Demostramos la correctitud de tales estructuras
    razonando de la afirmación antes de la estructura
    junto a la condición especial que nos lleva al presente caso.
    Lo que podemos concluir luego de cada una de las opciones
    se cumple al finalizar la estructura completa.
  \item[Estructuras de iteración:]
    Argüir la correctitud de un ciclo tiene tres fases:
    Inicialización,
    preservación
    y término.
    Primero demostramos que el invariante del ciclo
    se establece antes del ciclo
    (inicialización),
    luego que si el invariante se cumple al comienzo del ciclo
    se cumple cuando el ciclo termina,
    y finalmente demostramos que el ciclo
    se ejecuta un número finito de veces.
    Después del ciclo sabemos que se cumple el invariante
    y la condición de término se cumplió.
  \item[Variables:]\index{programa!variables}
    Muchos programas modifican los valores de variables,
    pero interesa expresar
    que cierta relación se cumple respecto de los valores iniciales.
    Una posibilidad
    es usar la convención que si la variable es \(n\),
    su valor original se representa mediante \(n_0\).
    Otra opción es definir variables fantasmas
    que recogen valores de interés.
    Usamos variables fantasmas
    en el algoritmo~\ref{alg:exponenciacion-binaria}
    para expresar el invariante en forma más sencilla.
  \item[Subrutinas:]\index{programa!subrutinas}
    Para verificar una subrutina,
    explicitamos su propósito mediante dos afirmaciones:
    Su precondición describe el estado antes de ejecutarla,
    la postcondición describe lo que garantiza del estado
    una vez que finaliza.
    Externamente
    usamos estas afirmaciones para razonar sobre sus usos,
    internamente verificamos que dadas las precondiciones
    estamos garantizando las postcondiciones.
    Esto es válido también en el caso de rutinas recursivas:
    Suponemos que las llamadas recursivas
    ``hacen lo correcto'',
    y en base a ello demostramos que la rutina cumple su contrato.
  \item[Recursión:]\index{programa!recursión}
    Verificar la versión recursiva de la exponenciación binaria
    resulta substancialmente más sencillo
    que verificar la versión no recursiva.
    Esta observación es aplicable en general.
  \end{description}
  Hallar afirmaciones simples
  (particularmente invariantes de ciclos)%
    \index{programa!invariante de ciclo}
  no es nada fácil.
  Esta manera de enfrentar el problema de escribir un programa
  formaliza la manera en que entendemos un programa
  (muchas veces la explicación del funcionamiento
   es a través de una traza de la ejecución,
     \index{programa!traza}
   pero está claro
   que el detalle de los valores que toman las variables
   en esa instancia particular
   no es todo lo que se está transmitiendo).
  Por esta razón conviene familiarizarse con este enfoque.

  Las partes difíciles de un programa
  hacen que se recurra a métodos más formales,
  mientras las partes simples
  se desarrollan de la forma tradicional.
  La experiencia común es que las partes difíciles
  luego funcionan correctamente la primera vez,
  las fallas están en las partes simples.

  No se ha logrado el objetivo inicial de estos esfuerzos:
  Contar con algún artilugio que tome un programa
  e indique ``correcto'' o ``incorrecto''.
  Igualmente logramos algo muy valioso:
  Una comprensión mejor del proceso de programación.
  Lenguajes de programación actuales se definen al menos en parte
  cuidando que sea fácil razonar con sus operaciones.
  Técnicas similares a las usadas acá para demostrar correctitud
  emplean los compiladores para ``optimizar'' código
    \index{programa!optimizacion de codigo@optimización de código}
  (en realidad, solo intentan mejorar características de ejecución):
  Dependiendo de restricciones deducidas sobre los valores,
  el código puede especializarse o incluso omitirse
  sin cambiar los resultados.
  Refinamientos del análisis son técnicas de ejecución simbólica,%
    \index{programa!ejecucion simbolica@ejecución simbólica}
  que pueden hallar errores
  o al menos generar automáticamente casos de prueba%
    \index{programa!casos de prueba}
  para la mayoría del código,
  un ejemplo es KLEE~%
    \cite{cadar08:_klee}.%
    \index{KLEE}
  También hay herramientas que ayudan a construir y verificar
  demostraciones de correctitud,
  como Frama-C~%
    \cite{cuoq12:_frama_c_softw_analy_persp}.%
    \index{Frama-C}

%%% Local Variables:
%%% mode: latex
%%% TeX-master: "clases"
%%% End:

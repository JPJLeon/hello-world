% tao-bookkeeper.tex
%
% Copyright (c) 2009, 2011-2014 Horst H. von Brand
% Derechos reservados. Vea COPYRIGHT para detalles

\section{El tao de \texttt{BOOKKEEPER}}
\label{sec:tao-bookkeeper}
\index{Tao de \texttt{BOOKKEEPER}|see{combinatoria!secuencias con repeticiones}}
\index{combinatoria!secuencias con repeticiones|textbfhy}

  Veremos maneras de contar secuencias que incluyen elementos repetidos.
  Para llegar a la iluminación
  siguiendo los pasos de Lehman, Leighton y Meyer~%
    \cite{lehman15:_mathem_comput_scien},
  meditemos sobre la palabra \(\mathtt{BOOKKEEPER}\).
  \begin{enumerate}
  \item
    ¿De cuántas maneras se pueden ordenar las letras de \(\mathtt{POKE}\)?
  \item
    ¿De cuántas maneras se pueden ordenar las letras de
    \(\mathtt{B} \mathtt{O}_1 \mathtt{O}_2 \mathtt{K}\)?
    (Note que los subíndices
     hacen que las dos \(\mathtt{O}\) sean distintas).
  \item
    Pequeño saltamontes,
    mapea los ordenamientos de
    \(\mathtt{B} \mathtt{O}_1 \mathtt{O}_2 \mathtt{K}\)
    (las \(\mathtt{O}\) son diferentes)
    a \(\mathtt{BOOK}\)
    (las dos \(\mathtt{O}\) son idénticas).
    ¿Qué clase de mapa es este?
  \item
    ¡Muy bien,
    joven maestro!
    Dime ahora,
    ¿de cuántas maneras pueden ordenarse las letras de
    \(\mathtt{K} \mathtt{E}_1 \mathtt{E}_2 \mathtt{P}
      \mathtt{E}_3 \mathtt{R}\)?
  \item
    Mapea cada ordenamiento de
    \(\mathtt{K} \mathtt{E}_1 \mathtt{E}_2 \mathtt{P}
      \mathtt{E}_3 \mathtt{R}\)
    a un ordenamiento de \(\mathtt{KEEPER}\)
    tal que,
    borrando los subíndices,
    lista todos los que leen \(\mathtt{REPEEK}\).
    ¿Que clase de mapa es este?
  \item
    En vista de lo anterior,
    ¿cuántos ordenamientos de
    \(\mathtt{\foreignlanguage{english}{KEEPER}}\) hay?
  \item
    \emph{¡Ahora ya estás en posición de enfrentarte
      al terrible \(\mathtt{BOOKKEEPER}\)!}
    ¿Cuántos ordenamientos de
    \(\mathtt{B} \mathtt{O}_1 \mathtt{O}_2 \mathtt{K}_1
      \mathtt{K}_2 \mathtt{E}_1 \mathtt{E}_2 \mathtt{P}
      \mathtt{E}_3 \mathtt{R}\)
    hay?
  \item
    ¿Cuántos ordenamientos de
    \(\mathtt{BOO} \mathtt{K}_1
      \mathtt{K}_2 \mathtt{E}_1 \mathtt{E}_2 \mathtt{P}
      \mathtt{E}_3 \mathtt{R}\)
    hay?
  \item
    ¿Cuántos ordenamientos de
    \(\mathtt{BOOKKEEPER}\) hay?
  \item
    ¿Cuántos ordenamientos de
    \(\mathtt{VOODOODOLL}\) hay?
  \item
    Esta es muy importante,
    pequeño saltamontes.
    ¿Cuántas secuencias de \(n\) bits
    tienen \(k\) ceros y \(n - k\) unos?
  \end{enumerate}
  Prender subíndices,
  apagar subíndices.
  Ese es el tao de \(\mathtt{BOOKKEEPER}\).%
    \index{coeficiente multinomial}

%%% Local Variables:
%%% mode: latex
%%% TeX-master: "clases"
%%% End:

% numerabilidad.tex
%
% Copyright (c) 2011-2014 Horst H. von Brand
% Derechos reservados. Vea COPYRIGHT para detalles

\chapter{Numerabilidad}
\label{cha:numerabilidad}
\index{numerabilidad|textbfhy}

  La manera más básica de contar es construir una biyección%
    \index{biyeccion@biyección}
  entre dos conjuntos,
  que de esa forma tienen la misma cardinalidad
  (``número de elementos'').
  Por ejemplo,
  para determinar si en una sala hay tantos asistentes como sillas
  basta solicitar que todos se sienten.
  Si no sobran sillas vacías ni quedan personas de pie,
  hay tantas personas como sillas.

  Nuestro interés está en la existencia
  de diferentes infinitos,
  en particular la demostración de Cantor%
    \index{Cantor, Georg}
  de que hay más números reales que enteros.
  No haremos uso de esto en el texto presente,
  pero los conceptos y las técnicas de demostración usadas
  muestran ser centrales en el estudio de la computabilidad
  (los límites de lo que un algoritmo puede hacer).

\section{Cardinalidad}
\label{sec:cardinalidad}
\index{cardinalidad}

  Ya indicamos que la manera fundamental de asignar un ``tamaño'' a un conjunto
  es hallar una correspondencia con un conjunto prototipo.
  Si los conjuntos son finitos,
  esta operación es conocida;
  buscamos extender la definición de lo que significa a conjuntos infinitos
  de forma de poder razonar sobre ellos.
  \begin{definition}
    La \emph{cardinalidad} del conjunto
    \(\{1, 2, \dotsc, n\}\) con \(n \in \mathbb{N}\)
    es \(n\).
    La cardinalidad de \(\varnothing\) es \(0\).
    Un conjunto cuya cardinalidad es \(n \in \mathbb{N}_0\)
    se dice \emph{finito}.%
      \index{conjunto!finito}
  \end{definition}
  A partir de aquí podemos definir igualdad de cardinalidades
  mediante biyecciones,
  incluso para conjuntos infinitos.%
      \index{conjunto!infinito}
  Anotaremos \(\lvert \mathcal{A} \rvert\)
  para la cardinalidad del conjunto \(\mathcal{A}\).
  Formalmente:
  \begin{definition}
    \index{conjunto!cardinalidad|textbfhy}
    Dos conjuntos \(\mathcal{A}\) y \(\mathcal{B}\)
    tienen la misma cardinalidad,
    lo que se anota
      \(\lvert \mathcal{A} \rvert = \lvert \mathcal{B} \rvert\),
    si hay una biyección
      \(\phi: \mathcal{A} \rightarrow \mathcal{B}\).
    Decimos que
      \(\lvert \mathcal{A} \rvert \le \lvert \mathcal{B} \rvert\)
    si hay una inyección
      \(\gamma: \mathcal{A} \rightarrow \mathcal{B}\).
    Si \(\lvert \mathcal{A} \rvert \le \lvert \mathcal{B} \rvert\)
    pero no existe biyección
    entre \(\mathcal{A}\) y \(\mathcal{B}\),
    decimos
      \(\lvert \mathcal{A} \rvert < \lvert \mathcal{B} \rvert\).
  \end{definition}
  Las notaciones indicadas son sugerentes.
  Para justificarlas debemos demostrar que
  la igualdad de cardinalidades
  es una relación de equivalencia.%
    \index{relacion@relación!equivalencia}
  El que la función identidad es una biyección provee reflexividad,
  como la función inversa de una biyección
  es una biyección da simetría
  y ya que la composición de biyecciones
  es una biyección es transitiva.
  Además debemos verificar
  que \(\lvert \mathcal{A} \rvert \le \lvert \mathcal{B} \rvert\)
  es una relación de orden%
    \index{relacion@relación!orden}
  (es transitiva,
   reflexiva y simétrica
   bajo el entendido de la igualdad de cardinalidades).
  La reflexividad es obvia,
  la función identidad es una inyección.
  La transitividad es simple,
  ya que la composición de inyecciones es una inyección.
  Demostrar simetría es más complejo,
  es el tema de nuestro siguiente teorema.
  Como ya es costumbre,
  fue primeramente demostrado por Dedekind,%
    \index{Dedekind, Richard}
  quien no aparece entre los créditos.
  La demostración que mostramos se debe a Julius Kőnig.%
    \index{Konig, Julius@König, Julius}
  \begin{theorem}[Cantor-Bernstein-Schröder]
    \index{Cantor-Bernstein-Schroder, teorema de@Cantor-Bernstein-Schröder, teorema de}
    \label{theo:Cantor-Bernstein-Schroeder}
    Si hay inyecciones
      \(f \colon \mathcal{A} \rightarrow \mathcal{B}\),
    y \(g \colon \mathcal{B} \rightarrow \mathcal{A}\),
    entonces hay una biyección
    entre \(\mathcal{A}\) y \(\mathcal{B}\).
  \end{theorem}
  \begin{proof}
    Sin pérdida de generalidad podemos suponer que \(\mathcal{A}\)
    y \(\mathcal{B}\) son disjuntos.
    Partiendo de un elemento \(a \in \mathcal{A}\) cualquiera,
    podemos definir
    una secuencia en \(\mathcal{A}\) y \(\mathcal{B}\)
    en ambas direcciones
    aplicando repetidas veces \(f\) y \(g\),
    y \(f^{-1}\) y \(g^{-1}\) donde estén definidas:
    \begin{equation*}
      \cdots \rightarrow f^{-1}(g^{-1}(a))
	     \rightarrow g^{-1}(a)
	     \rightarrow a
	     \rightarrow f(a)
	     \rightarrow g(f(a))
	     \rightarrow \cdots
    \end{equation*}
    Por ser inyectivas
    (no hay preimágenes repetidas),
    esta es una cadena.
    Cada elemento de \(\mathcal{A} \cup \mathcal{B}\)
    pertenece a exactamente una cadena,
    esto define una partición.
    Hay varias posibilidades de cadenas diferentes,
    y juntando biyecciones construidas para cada cadena
    obtenemos una biyección entre \(\mathcal{A}\) y \(\mathcal{B}\).
    \begin{description}
    \item[La cadena es infinita en ambas direcciones:]
      En este caso,
      \(f\) es una biyección.
    \item[Es un ciclo:]
      Nuevamente,
      \(f\) es una biyección.
    \item[\boldmath Termina en \(\mathcal{A}\):\unboldmath]
      También en este caso \(f\) es una biyección
    \item[\boldmath Termina en \(\mathcal{B}\)
	  pero no en \(\mathcal{A}\):\unboldmath]
      En este caso \(g\) define una biyección
    \end{description}
    Tenemos la biyección prometida.
  \end{proof}

  Estudiaremos los conjuntos infinitos en algo más de detalle,
  partiendo por \(\mathbb{N}\).
  \begin{definition}
    \index{conjunto!numerable|textbfhy}
    Un conjunto \(\mathcal{X}\) se dice \emph{infinito numerable}
    si hay una biyección entre \(\mathcal{X}\) y \(\mathbb{N}\).
    Un conjunto se llama \emph{numerable} si es finito
    o es infinito numerable.
  \end{definition}
  Esta definición dice que si \(\mathcal{X}\) es infinito numerable,
  entonces podemos escribir \(\mathcal{X} = \{x_1, x_2, \dotsc\}\),
  bajo el entendido
  que existe la biyección
    \(\phi \colon \mathcal{X} \rightarrow \mathbb{N}\)
  con \(\phi(x_n) = n\) para cada \(n \in \mathbb{N}\).
  \begin{theorem}
    \label{theo:union-numerable}
    La unión numerable de conjuntos numerables es numerable.
  \end{theorem}
  \begin{proof}
    Sea \(\mathcal{I}\) un conjunto índice numerable,
    tal que para cada \(i \in \mathcal{I}\)
    el conjunto \(\mathcal{X}_i\) es numerable.
    Entonces \(\mathcal{I}\) es finito o es infinito numerable.
    Consideraremos solo el segundo caso,
    el primero requiere modificaciones menores.

    Como \(\mathcal{I}\) es infinito numerable,
    hay una biyección entre \(\mathcal{I}\) y \(\mathbb{N}\),
    con lo que podemos adoptar \(\mathbb{N}\) como conjunto índice
    sin pérdida de generalidad.
    Definamos:
    \begin{equation*}
      \mathcal{X}
	= \bigcup_{n \in \mathbb{N}} \mathcal{X}_n
    \end{equation*}
    Si \(\mathcal{X}\) es finito,
    estamos listos.

    Supongamos entonces \(\mathcal{X}\) infinito.
    Como para todo \(n \in \mathbb{N}\)
    el conjunto \(\mathcal{X}_n\) es numerable,
    podemos escribir
      \(\mathcal{X}_n = \{x_{n 1}, x_{n 2}, \dotsc\}\).
    Con la convención que si \(\mathcal{X}_n\) es finito
    la secuencia
       \(\left\langle
	   x_{n 1}, x_{n 2}, x_{n 3}, \dotsc
	 \right\rangle\)
    simplemente repite los elementos de \(\mathcal{X}_n\),
    podemos escribir la matriz doblemente infinita:
    \begin{equation*}
      \begin{array}{*{4}{c}}
	x_{1 1} & x_{1 2} & x_{1 3} & \dotso \\
	x_{2 1} & x_{2 2} & x_{2 3} & \dotso \\
	x_{3 1} & x_{3 2} & x_{3 3} & \dotso \\
	\vdots	& \vdots  & \vdots  & \ddots
      \end{array}
    \end{equation*}
    Esta matriz podemos recorrerla diagonalmente,
    siguiendo elementos \(x_{i j}\) en orden de \(i + j\) creciente:
    Primero \(x_{1 1}\),
    luego \(x_{1 2}\) y \(x_{2 1}\),
    después \(x_{1 3}\), \(x_{2 2}\) y \(x_{3 1}\),
    y así sucesivamente,
    omitiendo elementos ya listados.
    Así construimos una biyección
    entre \(\mathbb{N}\) y \(\mathcal{X}\).
  \end{proof}
  \begin{theorem}
    \label{theo:subconjunto-numerable}
    Todo subconjunto de un conjunto numerable es numerable.
  \end{theorem}
  \begin{proof}
    Sea \(\mathcal{X}\) un conjunto numerable.
    Si \(\mathcal{X}\) es finito,
    la conclusión es inmediata.
    Supongamos entonces \(\mathcal{X}\) infinito numerable,
    e \(\mathcal{Y} \subseteq \mathcal{X}\).
    Si \(\mathcal{Y}\) es finito,
    estamos listos.
    Supongamos entonces que \(\mathcal{Y}\) es infinito.
    Definimos la secuencia
      \(\left\langle n_1, n_2, \dotsc \right\rangle\)
    mediante:
    \begin{align*}
      n_1
	&= \min \{n \in \mathbb{N} \colon x_n \in \mathcal{Y}\} \\
      n_k
	&= \min \{n \in \mathbb{N} \colon
		     n > n_{k - 1} \wedge x_n \in \mathcal{Y}\}
    \end{align*}
    La secuencia \(\left\langle n_k \right\rangle_{k \ge 1}\)
    define una biyección entre \(\mathbb{N}\) e \(\mathcal{Y}\)
    (asocia un índice \(k\) con cada elemento de \(\mathcal{Y}\)).
  \end{proof}

  Llegamos así a los resultados más importantes que veremos acá.
  \begin{theorem}
    \label{theo:Z-numerable}
    El conjunto \(\mathbb{Z}\) es numerable.
  \end{theorem}
  \begin{proof}
    Tenemos la unión de tres conjuntos numerables:
    \begin{equation*}
      \mathbb{Z} = \mathbb{N} \cup \{0\} \cup \{-1, -2, -3, \dotsc\}
      \qedhere
    \end{equation*}
  \end{proof}
  \begin{theorem}
    \label{theo:Q-numerable}
    El conjunto \(\mathbb{Q}\) es numerable.
  \end{theorem}
  \begin{proof}
    Podemos representar \(r \in \mathbb{Q}\) como \(r = a / b\),
    con \(a \in \mathbb{Z}\) y \(b \in \mathbb{N}\).
    El conjunto de fracciones con denominador \(b\) es numerable,
    y la colección de tales conjuntos es numerable.
    Por el teorema~\ref{theo:union-numerable},
    su unión \(\mathbb{Q}\) es numerable.
  \end{proof}
  \begin{theorem}[Cantor]
    \index{Cantor, teorema de}
    \index{Cantor, Georg}
    \index{demostracion@demostración!argumento diagonal}
    \label{theo:R-no-numerable}
    El conjunto \(\mathbb{R}\) no es numerable.
  \end{theorem}
  \begin{proof}
    La demostración es por contradicción.
    En vista del teorema~\ref{theo:subconjunto-numerable},
    basta demostrar que \([0, 1)\) no es numerable.
    Supongamos entonces
    que hay una biyección entre \([0, 1)\) y \(\mathbb{N}\).
    Un número \(x \in [0, 1)\)
    puede expresarse en notación decimal como
    \(0,d_1 d_2 d_3 \dotso\),
    donde \(0 \le d_i \le 9\) son los dígitos correspondientes
    de su expansión decimal.
    Ponemos la condición adicional
    que la expansión no es solo nueves
    a partir de un punto dado,
    para evitar ambigüedades.
    Con la biyección supuesta tendremos una matriz
    en que \(d_{i j}\) corresponde al \(j\)\nobreakdash-ésimo dígito
    del \(i\)\nobreakdash-ésimo número:
    \begin{equation*}
      \begin{array}{r*{10}{c@{\,}}l}
	1: & d_{1, 1} & d_{1, 2} & d_{1, 3} & d_{1, 4} & d_{1, 5} & d_{1, 6} &
	     d_{1, 7} & d_{1, 8} & d_{1, 9} & d_{1, 10} & \cdots \\
	2: & d_{2, 1} & d_{2, 2} & d_{2, 3} & d_{2, 4} & d_{2, 5} & d_{2, 6} &
	     d_{2, 7} & d_{2, 8} & d_{2, 9} & d_{2, 10} & \cdots \\
	\vdots\;\;\,
	   & \multicolumn{10}{c}{\vdots} \\
	n: & d_{n, 1} & d_{n, 2} & d_{n, 3} & d_{n, 4} & d_{n, 5} & d_{n, 6} &
	     d_{n, 7} & d_{n, 8} & d_{n, 9} & d_{n, 10} & \cdots \\
	\vdots\;\;\,
	   & \multicolumn{10}{c}{\vdots}
      \end{array}
    \end{equation*}
    Consideremos el número \(y = 0,v_1 v_2 v_3 \dotso\),
    cuyos dígitos se definen por:
    \begin{equation*}
      v_i
	=
	\begin{cases}
	  2 & \text{\ si \(d_{i i} = 1\)} \\
	  1 & \text{\ si \(d_{i i} \ne 1\)}
	\end{cases}
    \end{equation*}
    Claramente \(y \in [0, 1)\),
    no tiene solo nueves a partir de una posición dada,
    y es diferente
    al menos en el dígito \(i\)\nobreakdash-ésimo
    del \(i\)\nobreakdash-ésimo número de la lista,
    con lo que \(y\) no está en esta lista
    que supuestamente los contiene a todos.
    Esta contradicción completa la demostración.
  \end{proof}
  Alguna variante de este argumento diagonal
  se usa en muchas demostraciones
  relacionadas a conjuntos infinitos.

  Nótese que el conjunto \(\mathbb{R} \smallsetminus \mathbb{Q}\)
  no es numerable,
  con lo que en cierto sentido
  hay más números irracionales que racionales.%
    \index{numero@número!irracional}
  Incluso más:
  Un número se llama \emph{algebraico}%
    \index{numero@número!algebraico}
  si es un cero de un polinomio con coeficientes enteros.%
     \index{polinomio!coeficientes enteros}
  Entonces:
  \begin{theorem}[Cantor]
    \label{theo:algebraicos-numerable}
    El conjunto de números reales algebraicos es numerable.
  \end{theorem}
  \begin{proof}
   Dado un polinomio
   \(a_n x^n + a_{n - 1} x^{n - 1} + \dotsb + a_0\),
   su \emph{altura} es
     \(n + \lvert a_n \rvert
	 + \lvert a_{n - 1} \rvert
	 + \dotsb
	 + \lvert a_0 \rvert
     \).
   Hay un número finito de polinomios de cada altura
   y un polinomio de grado \(n\)
   tiene a lo más \(n\) raíces reales
   (esto lo demostraremos
    en el capítulo~\ref{cha:anillos-polinomios}),
   con lo que hay un número finito de números algebraicos reales
   de cada altura.
   Los números algebraicos son entonces una unión numerable
   de conjuntos numerables,
   y por tanto numerables.
  \end{proof}
  Los números reales no algebraicos se llaman \emph{trascendentes}.%
    \index{numero@número!trascendente}
  Euler ya sospechó su existencia,%
    \index{Euler, Leonhard}
  la demostración anterior indica
  que hay muchos más números trascendentes que algebraicos,
  pero no exhibe ninguno.

  Un importante teorema es el siguiente:
  \begin{theorem}[Cantor]
    \label{theo:A<powerset(A)}
    \(\lvert \mathcal{A} \rvert < \lvert 2^{\mathcal{A}} \rvert\)
  \end{theorem}
  \begin{proof}
    Por contradicción.
    Sea \(f \colon \mathcal{A} \rightarrow 2^{\mathcal{A}}\)
    una función cualquiera,
    construimos un conjunto \(\mathcal{T} \in 2^{\mathcal{A}}\)
    que no es imagen de ningún \(\alpha \in \mathcal{A}\),
    con lo que \(f(\cdot)\) no puede ser una biyección
    por no ser sobre.
    Para todo \(\alpha \in \mathcal{A}\),
    debe ser \(\alpha \in \mathcal{T}\)
    o \(\alpha \notin \mathcal{T}\).
    Sea \(\mathcal{T}
	    = \{\alpha \in \mathcal{A} \colon
		  \alpha \notin f(\alpha)\}\).
    Si \(\alpha \in \mathcal{T}\),
    entonces \(\alpha \notin f(\alpha)\),
    de forma que \(f(\alpha) \ne \mathcal{T}\).
    Por el otro lado,
    si \(\alpha \notin \mathcal{T}\),
    entonces \(\alpha \in f(\alpha)\),
    y nuevamente \(f(\alpha) \ne \mathcal{T}\).
    La función \(f\) no es sobre
    ya que su rango no incluye a \(\mathcal{T}\),
    no puede ser biyección.
  \end{proof}
  Esto es nuevamente el argumento diagonal%
    \index{demostracion@demostración!argumento diagonal}
  que usamos para demostrar el teorema~\ref{theo:R-no-numerable}.
  Este teorema demuestra que hay infinitas cardinalidades
  mayores que la de \(\mathbb{N}\).

  Resulta que el conjunto
  de subconjuntos \emph{finitos} de un conjunto numerable
  es numerable.
  Si el conjunto universo es finito,
  el conjunto de sus subconjuntos es finito,
  y por tanto numerable.
  Si el conjunto universo es infinito,
  sin pérdida de generalidad podemos tomarlo como \(\mathbb{N}\).
  Los conjuntos \(S_n = 2^{[1, n]}\) son todos finitos,
  y la unión de todos ellos
  es la unión numerable de conjuntos numerables.

  Vemos también que rangos abiertos y cerrados de \(\mathbb{R}\)
  tienen la misma cardinalidad:
  Por ejemplo,
  como entre \((0, 1)\) y \([0, 1]\) tenemos las inyecciones
  \(x \mapsto x\) e \(y \mapsto (y + 1) / 3\),
  por el teorema~\ref{theo:Cantor-Bernstein-Schroeder}
  tienen la misma cardinalidad.
  Con la biyección \(x \mapsto (x + a) / (b - a)\)
  entre \([0, 1]\) y \([a, b]\)
  todos los rangos finitos tienen la misma cardinalidad.
  La biyección \(x \mapsto \tanh x\)
  entre \(\mathbb{R}\) y \((-1, 1)\)
  muestra que rangos infinitos comparten la misma cardinalidad
  de rangos finitos.

  Lo que sí resulta sorprendente es que \(\mathbb{R}\)
  y \(\mathbb{C}\)
  tienen la misma cardinalidad,
  e incluso en general
  \(\lvert \mathbb{R} \rvert
      = \lvert \mathbb{R}^n \rvert\) para \(n \ge 1\).
  Para ilustrar la demostración general,
  mostraremos una biyección entre \((0, 1]\)
  y el cuadrado \(0 < x, y \le 1\).
  Representamos \(z \in (0, 1]\)
  mediante su expansión decimal que no termina
  (en vez de \(0,5\) escribimos \(0,4\overline{9}\)).
  Dividimos \(z\) en el par \((x, y)\) en \((0, 1]\)
  por la vía de cortar la expansión de \(z\)
  en grupos de ceros y el dígito que sigue.
  Vale decir,
  si \(z = 0,1003049\dotso\) obtenemos
  \(x = 0,104\dotso\) e \(y = 0,0039\dotso\).
  Nótese que como la expansión decimal de \(z\)
  no termina en una secuencia infinita de ceros,
  siempre tendremos dónde cortar para el siguiente;
  y las expansiones resultantes para \(x\) e \(y\)
  nunca terminan en secuencias infinitas de ceros.
  Esto provee una biyección
  entre un rango finito y un cuadrado finito,
  podemos extender ambos como antes.
  La misma idea puede usarse para \(n\) mayor.

%%% Local Variables:
%%% mode: latex
%%% TeX-master: "clases"
%%% End:

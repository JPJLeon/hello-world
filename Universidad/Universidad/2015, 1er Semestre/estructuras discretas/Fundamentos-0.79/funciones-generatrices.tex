% funciones-generatrices.tex
%
% Copyright (c) 2009-2014 Horst H. von Brand
% Derechos reservados. Vea COPYRIGHT para detalles

\chapter{Funciones generatrices}
\label{cha:funciones-generatrices}
\index{funcion@función!generatriz|see{generatriz}}
\index{generatriz|textbfhy}

  Veremos cómo usar series de potencias%
    \index{serie de potencias}
  (una herramienta del análisis,
   vale decir matemáticas de lo continuo)
  para resolver una variedad de problemas discretos.
  La idea de funciones generatrices
  permite resolver muchos problemas
  de forma simple y transparente.
  Incluso cuando no da soluciones puede iluminar,
  indicando relaciones entre problemas
  que a primera vista no son obvias.
  El aplicar herramientas analíticas
  (especialmente la teoría de funciones de variables complejas)%
    \index{analisis complejo@análisis complejo}
  permite deducir resultados
  que de otra forma serían muy difíciles de obtener.

  Nos centramos en aplicaciones y en uso de las técnicas discutidas
  más que en exponer la teoría,
  nuestros ejemplos frecuentemente
  llevan a resultados de interés independiente.

\section{Detalles adicionales}
\label{sec:detalles-adicionales}

  Para detalles de la teoría y aplicaciones adicionales
  véanse por ejemplo a Flajolet y Segdewick~%
    \cite{flajolet09:_analy_combin}
  o Wilf~%
    \cite{wilf06:_gfology},
  mientras Kauers~%
    \cite{kauers11:_concr_tetrah}
  se centra en el uso de paquetes de álgebra simbólica
  alrededor de esto.%
    \index{algebra simbolica@álgebra simbólica}
  Referencia obligatoria para todo lo que es combinatoria
  son los textos de Stanley~%
    \cite{stanley12:_enumer_combin-1,
	  stanley99:_enumer_combin-2}.
  Un recurso indispensable
  es la enciclopedia de secuencias de enteros~%
    \cite{sloane:_OEIS},%
    \index{OEIS@\texttt{OEIS}}
  que registra muchos miles de secuencias,
  cómo se generan y da referencias al respecto.

  Sea una secuencia
  \(\left\langle a_n \right\rangle_{n \ge 0}
     = \left\langle
	 a_0, a_1, a_2, \dotsc, a_n, \dotsc
       \right\rangle\).
  La \emph{función generatriz} (ordinaria) de la secuencia es
  la serie de potencias:%
    \index{generatriz!ordinaria}
  \begin{equation}
    \label{eq:definition-gf}
    A(z) = \sum_{n \ge 0} a_n z^n
  \end{equation}
  El punto es que la serie~\eqref{eq:definition-gf}
  representa en forma compacta y manejable
  la secuencia infinita.
  Wilf~\cite{wilf06:_gfology} expresa esto
  diciendo que la función generatriz es una línea de ropa
  de la cual se cuelgan los coeficientes para exhibición.
  Acá entendemos el exponente de \(z\) como un contador,
  índice del coeficiente correspondiente.
  Como veremos,
  operaciones sobre la función generatriz
  corresponden a actuar sobre la secuencia,
  en muchos casos resulta bastante más sencillo manipular la serie
  que trabajar directamente con la secuencia.

  Para un primer ejemplo,
  volveremos al problema
  de la Competencia de Ensayos de la Universidad de Miskatonic
  (ver la sección~\ref{sec:conjetura->teorema}),
  que llevó a la relación:
  \begin{equation}
    \label{eq:recurrence-UMEC-pre}
    b_{2 r + 1} = b_{2 r - 1} + r + 1 \qquad b_1 = 1
  \end{equation}
  con lo que tenemos,
  como antes
  (arbitrariamente dando el valor cero
  a los que no quedan definidos por la recurrencia):
  \begin{equation*}
    \left\langle b_n \right\rangle_{n \ge 0}
      = \left\langle
	  0, 1, 0, 3, 0, 6, 0, 10, 0, 15, \dotsc
	\right\rangle
  \end{equation*}
  Contar con algunos valores sirve para verificar
  (y para ``sentir'' cómo se comportan).

  Para anotar en forma compacta recurrencias%
    \index{recurrencia}
  como~\eqref{eq:recurrence-UMEC-pre}
  usaremos la notación:
  \begin{equation}
    \label{eq:recurrence-UMEC}
    b_{2 r + 1}
      = b_{2 r - 1} + r + 1
      \quad (r \ge 1)
      \qquad b_1 = 1
  \end{equation}
  O sea,
  damos la recurrencia misma,
  los índices para los que la recurrencia vale,
  y valores iniciales.
  Normalmente será
  que la recurrencia vale a partir del término siguiente
  al indicado como valor inicial,
  y omitiremos el rango de validez.

  Definamos la serie
  (note que el subíndice en \(b_{2 r + 1}\) avanza de a dos):
  \begin{equation}
    \label{eq:gf-UMEC}
    B(z) = \sum_{r \ge 0} b_{2 r + 1} z^r
  \end{equation}
  Si multiplicamos la recurrencia~\eqref{eq:recurrence-UMEC}
  por \(z^r\)
  y sumamos sobre \(r \ge 1\)
  (para considerar solo índices positivos)
  queda:
  \begin{equation}
    \label{eq:gf-UMEC-0}
    \sum_{r \ge 1} b_{2 r + 1} z^r
       = \sum_{r \ge 1} b_{2 r - 1} z^r
	   + \sum_{r \ge 1} (r + 1) z^r
  \end{equation}
  Expresando lo anterior en términos de \(B(z)\),
  reconocemos:
  \begin{align}
    \sum_{r \ge 1} b_{2 r + 1} z^r
      &= \sum_{r \ge 0} b_{2 r + 1} z^r - b_1
       = B(z) - 1 \label{eq:suma-b(2r+1)} \\
    \sum_{r \ge 1} b_{2 r - 1} z^r
      &= \sum_{r \ge 0} b_{2 r + 1} z^{r + 1}
       = z \sum_{r \ge 0} b_{2 r + 1} z^r
       = z B(z) \label{eq:suma-b(2r-1)}
  \end{align}
  En la ecuación~\eqref{eq:gf-UMEC-0}
  aparecen términos \((r + 1) z^r\),
  que sugieren la derivada de \(z^{r + 1}\).
  Por el otro lado,
  de la sección~\ref{sec:induccion-comun}
  sabemos que para \(\lvert z \rvert < 1\) vale la serie geométrica:
  \begin{equation}
    \label{eq:serie-geometrica}
    \frac{1}{1 - z} = \sum_{r \ge 0} z^r
  \end{equation}
  Derivando la serie geométrica respecto de \(z\) término a término,
  lo que es válido dentro del radio de convergencia%
    \index{serie de potencias!radio de convergencia}
  (no nos detendremos en este punto,
   para la teoría que lo justifica véanse por ejemplo a Chen~%
    \cite{chen08:_fundam_analy},
   a Trench~\cite{trench13:_introd_real_analy}
   o refiérase al capítulo~\ref{cha:analisis-complejo}),
  queda:
  \begin{align}
    \frac{\mathrm{d}}{\mathrm{d} z}
      \, \left( \frac{1}{1 - z} \right)
      &= \sum_{r \ge 0} \frac{\mathrm{d}}{\mathrm{d} z} \, z^r
	   \notag \\
    \frac{1}{(1 - z)^2}
      &= \sum_{r \ge 1} r z^{r - 1}
       = \sum_{r \ge 0} (r + 1) z^r
	   \label{eq:D-serie-geometrica}
  \end{align}
  También:
  \begin{equation}
    \label{eq:suma-r+1}
    \sum_{r \ge 1} (r + 1) z^r
       = \sum_{r \ge 0} (r + 1) z^r - 1
       = \frac{1}{(1 - z)^2} - 1
  \end{equation}
  Reemplazando~\eqref{eq:suma-b(2r+1)}, \eqref{eq:suma-b(2r-1)}
  y~\eqref{eq:suma-r+1} en~\eqref{eq:gf-UMEC-0} queda:
  \begin{equation*}
    B(z) - 1
      = z B(z) + \frac{1}{(1 - z)^2} - 1
  \end{equation*}
  Despejando \(B(z)\):
  \begin{equation}
    \label{eq:gf-UMEC-1}
    B(z) = \frac{1}{(1 - z)^3}
  \end{equation}
  Para algunas aplicaciones basta llegar hasta acá,
  puede extraerse bastante información sobre los coeficientes
  de la función.
  Por ejemplo,
  claramente la serie~\eqref{eq:gf-UMEC-1}
  converge para \(\lvert z \rvert < 1\),
  y por la prueba de la razón,%
    \index{serie de potencias!prueba de la razon@prueba de la razón}
  sabemos que:
  \begin{equation*}
    \lim_{r \rightarrow \infty}
	  \left\lvert
	    \frac{b_{2 r + 1}}{b_{2 r + 3}}
	  \right\rvert = 1
  \end{equation*}

  Pero interesa obtener una fórmula explícita
  (ojalá simple)
  para los coeficientes,
  de forma de poder determinar el tamaño requerido de las tarjetas.
  Derivando la serie geométrica por segunda vez,
  de forma de obtener la expresión \((1 - z)^{-3}\),
  resulta:
  \begin{align}
    \frac{\mathrm{d}^2}{\mathrm{d} z^2}
      \, \left( \frac{1}{1 - z} \right)
      &= \sum_{r \ge 1}
	   \frac{\mathrm{d}}{\mathrm{d} z} \, r z^{r - 1}
	   \notag \\
    \frac{2}{(1 - z)^3}
      &= \sum_{r \ge 2} r (r - 1) z^{r - 2}
       = \sum_{r \ge 0} (r + 2) (r + 1) z^r
	   \label{eq:DD-serie-geometrica}
  \end{align}
  Con~\eqref{eq:DD-serie-geometrica} y~\eqref{eq:gf-UMEC-1} resulta:
  \begin{equation}
    \label{eq:gf-UMEC-2}
    B(z)
      = \sum_{r \ge 0} b_{2 r + 1} z^r
      = \frac{1}{2} \, \sum_{r \ge 0} (r + 2) (r + 1) z^r
  \end{equation}
  Comparando coeficientes
  tenemos nuevamente
  la fórmula explícita~\eqref{eq:ensayos-valor-explicito}:
  \begin{equation*}
    b_{2 r + 1} = \frac{1}{2} \, (r + 2) (r + 1)
  \end{equation*}
  La ventaja frente al desarrollo
  de la sección~\ref{sec:conjetura->teorema}
  es que no tuvimos que ``adivinar'' esta solución
  y nos ahorramos la demostración por inducción.
  Nótese además que el valor de \(B(z)\)
  jamás fue del más mínimo interés en el desarrollo.

\subsubsection*{Receta}
\index{recurrencia!receta}

  Para resolver recurrencias se debe:
  \begin{enumerate}
  \item\label{item:plantear}
    Plantear la recurrencia.
  \item
    Recopilar valores iniciales.
  \item\label{item:valores}
    Aclarar para qué valores del índice vale la recurrencia.
  \item\label{item:GF}
    Definir la función generatriz de interés.
  \item\label{item:ecuacion}
    Multiplicar la recurrencia de~(\ref{item:plantear})
    por \(z^n\)
    y sumar sobre todos los valores~(\ref{item:valores}).
  \item
    Expresar (\ref{item:ecuacion})
    en términos de la función generatriz~(\ref{item:GF}).
  \item
    Despejar la función generatriz de~(\ref{item:ecuacion}).
  \item
    Extraer los coeficientes.
  \end{enumerate}

\section{Algunas series útiles}
\label{ref:series-utiles}
\index{serie de potencias!series utiles@series útiles}

  Al trabajar con funciones generatrices
  es importante tener algunas expansiones en serie
  conocidas a la mano.
  Las que más aparecen son las siguientes.

\subsection{Serie geométrica}
\label{sec:serie-geometrica}
\index{serie de potencias!geometrica@geométrica}

  Es la serie más común en aplicaciones.
  Si \(\lvert z \rvert < 1\),
  se cumple:
  \begin{equation}
    \label{eq:serie-geometrica-b}
    \sum_{n \ge 0} z^n
      = \frac{1}{1 - z}
  \end{equation}
  Una variante importante es la siguiente,
  expansión válida para \(\lvert a z \rvert < 1\)
  (siempre que usemos la convención \(0^0 = 1\)):
  \begin{equation}
    \label{eq:serie-geometrica-c}
    \sum_{n \ge 0} a^n z^n
      = \frac{1}{1 - a z}
  \end{equation}

\subsection{Teorema del binomio}
\label{sec:teorema-binomio}
\index{serie de potencias!binomio}

  Una de las series más importantes
  es la expansión de la potencia de un binomio
  (ver también el teorema~\ref{theo:binomio}):
  \begin{equation}
    \label{eq:serie-binomio}
    \sum_{n \ge 0} \binom{\alpha}{n} \, z^n
       = (1 + z)^\alpha
  \end{equation}
  Siempre que \(\lvert z \rvert < 1\)
  esto vale no solo para valores reales de \(\alpha\),
  sino incluso para \(\alpha\) complejos,
  si definimos:
  \begin{equation}
    \label{eq:coeficiente-binomial}
    \binom{\alpha}{k}
       = \frac{\alpha}{1} \cdot \frac{\alpha - 1}{2}
	    \cdot \frac{\alpha - 2}{3}
	    \cdot \dots
	    \cdot \frac{\alpha - k + 1}{k}
       = \frac{\alpha^{\underline{k}}}{k!}
  \end{equation}
  y (consistente con la convención que productos vacíos son \(1\))
  siempre es:
  \begin{equation}
    \label{eq:binomial(alpha,0)}
    \binom{\alpha}{0}
      = 1
  \end{equation}
  A los coeficientes~\eqref{eq:coeficiente-binomial}
  se les conoce como \emph{coeficientes binomiales}
  por su conexión con la potencia de un binomio.
  La expansión~\eqref{eq:serie-binomio}
  (también conocida como \emph{fórmula de Newton}
   si \(\alpha\) no es un natural)
  es fácil de demostrar por el teorema de Maclaurin.
  Incluso resulta que~\eqref{eq:serie-geometrica-b} es simplemente
  un caso particular de~\eqref{eq:serie-binomio}.

  Si \(\alpha\) es un entero positivo,
  la serie~\eqref{eq:serie-binomio} se reduce a un polinomio
  y la relación es válida para todo \(z\).
  Además,
  en caso que \(n \in \mathbb{N}\)
  podemos escribir:
  \begin{equation}
    \label{eq:coeficiente-binomial-factorial}
    \binom{n}{k}
       = \frac{n!}{k! (n - k)!}
  \end{equation}
  Es claro que:
  \begin{equation}
    \label{eq:coeficiente-binomial-contorno}
    \binom{n}{k}
      = 0 \text{\ si \(k < 0\) o \(k > n\)}
  \end{equation}
  Esto con~\eqref{eq:coeficiente-binomial-factorial}
  sugiere la convención:
  \begin{equation}
    \label{eq:1/k!-convention}
    \frac{1}{k!}
      = 0 \quad \text{si \(k < 0\)}
  \end{equation}
  Nótese la simetría:
  \begin{equation}
    \label{eq:coeficiente-binomial-simetria}
    \binom{n}{k}
      = \binom{n}{n - k}
  \end{equation}

  Casos especiales notables de coeficientes binomiales
  para \(\alpha \notin \mathbb{N}\) son los siguientes:
  \begin{description}
  \item[\boldmath Caso \(\alpha = 1 / 2\):\unboldmath]
    Tenemos,
    como siempre:
    \begin{equation}
      \label{eq:binomial(1/2,0)}
      \binom{1/2}{0}
	= 1
    \end{equation}
    Cuando \(k \ge 1\):
    \begin{align}
      \binom{1/2}{k}
	 &= \frac{\frac{1}{2} \cdot (\frac{1}{2}-1)
	       \dotsm (\frac{1}{2} - k + 1)}{k!} \notag \\
	 &= \frac{1}{2^k}
	       \cdot \frac{1 \cdot (1 - 2) \cdot (1 - 4)
			     \dotsm (1 - 2 k + 2)}{k!} \notag \\
	 &= \frac{(-1)^{k - 1}}{2^k k!}
	       \cdot (1 \cdot 3 \dotsm (2 k - 3)) \notag \\
	 &= \frac{(-1)^{k - 1}}{2^k k!}
	       \cdot \frac{1 \cdot 2 \cdot 3 \cdot 4
			      \cdot \dotsm
			      \cdot (2 k - 3) \cdot (2 k - 2)}
			  {2 \cdot 4 \cdot 6 \dotsm (2 k - 2)}
				\notag \\
	 &= \frac{(-1)^{k - 1}}{2^k k!}
	       \cdot \frac{(2 k - 2)!}{2^{k - 1} (k - 1)!}
		  \notag \\
	 &= \frac{(-1)^{k - 1}}{2^{2 k - 1} \cdot k}
	       \cdot \frac{(2 k - 2)!}{(k - 1)! \, (k - 1)!}
		  \notag \\
	 &= \frac{(-1)^{k - 1}}{2^{2 k - 1} \cdot k}
	       \cdot \binom{2 k - 2}{k - 1}
	    \label{eq:binomial(1/2,k)}
    \end{align}
    Hay que tener cuidado,
    la última fórmula no cubre el caso \(k = 0\).

    Una serie común es:
    \begin{align}
      \frac{1 - \sqrt{1 - 4 z}}{2 z}
	&= \frac{1}{2 z} \,
	     \left(
	       1 - \sum_{n \ge 0} \binom{1 / 2}{n} \, (-4 z)^n
	     \right) \notag \\
	&= \frac{1}{2 z} \,
	     \left(
	       1 - \left(
		     1 + \sum_{n \ge 1}
			   \frac{(-1)^{n - 1}}{n 2^{2 n - 1}} \,
			     \binom{2 n - 2}{n - 1} \, (-4 z)^n
		   \right)
	     \right) \notag \\
	&= \frac{1}{2 z}
	     \cdot 4 z \, \sum_{n \ge 0}
			    \frac{1}{2 (n + 1)}
			       \, \binom{2 n}{n} \, z^n
		\notag \\
	&= \sum_{n \ge 0} \frac{1}{n + 1} \, \binom{2 n}{n} \, z^n
	     \label{eq:gf-Catalan}
	     \index{Catalan, numeros de@Catalan, números de!generatriz|textbfhy}
    \end{align}
    Los coeficientes de~\eqref{eq:gf-Catalan} se conocen como
    \emph{números de Catalan}:%
      \index{Catalan, numeros de@Catalan, números de}%
      \index{Catalan, numeros de@Catalan, números de!formula@fórmula}%
    \begin{equation}
      \label{eq:Catalan-numbers}
      C_n
	= \frac{1}{n + 1} \, \binom{2 n}{n}
    \end{equation}
    La serie~\eqref{eq:gf-Catalan} aparece con regularidad,
    al igual que los coeficientes~\eqref{eq:Catalan-numbers}.
    Stanley~%
      \cite{stanley99:_enumer_combin-2,
	    stanley13:_catalan_addendum}
    lista un total de \(205\) interpretaciones combinatorias
    de los números de Catalan.
    Se la ha llamado la función generatriz más famosa de la combinatoria.
  \item[\boldmath Caso \(\alpha = -1/2\):\unboldmath]
    Mucha de la derivación es similar a la del caso anterior.
    Tenemos,
    para \(k > 0\):
    \begin{align}
      \binom{-1/2}{k}
	&= \frac{(-1/2) \cdot (-1/2 - 1) \cdot \dotsm
		   \cdot (-1/2 - k + 1)}
		{k!} \notag \\
	&= (-1)^k \frac{1}{2^k}
	     \cdot \frac{1 \cdot 3 \dotsm (2 k - 1)}{k!} \notag \\
	&= (-1)^k \frac{1}{2^k}
	     \cdot \frac{(2 k)!}{k! \, 2^k \, k!} \notag \\
	&= (-1)^k \frac{1}{2^{2 k}} \, \binom{2 k}{k}
	    \label{eq:binomial(-1/2,k)}
    \end{align}
    Esta fórmula con \(k = 0\) da:
    \begin{equation*}
      \binom{-1/2}{0} = 1
    \end{equation*}
    así no se necesita hacer un caso especial acá.

    Una expansión común es:
    \begin{equation}
      \label{eq:serie-reciproco-raiz}
      \frac{1}{\sqrt{1 - 4 z}}
	= \sum_{n \ge 0} \binom{2 n}{n} \, z^n
    \end{equation}
  \item[\boldmath Caso \(\alpha = -n\):\unboldmath]
    Cuando \(\alpha\) es un entero negativo,
    podemos escribir:
    \begin{equation}
      \label{eq:binomial(-n,k)}
      \binom{-n}{k}
	= \frac{(-n)^{\underline{k}}}{k!}
	= (-1)^k \, \frac{n^{\overline{k}}}{k!}
	= (-1)^k \, \frac{(n + k - 1)^{\underline{k}}}{k!}
	= (-1)^k \, \binom{k + n - 1}{n - 1}
    \end{equation}
    Nótense los casos particulares
    (aparecieron en nuestra derivación de la solución
     para la Competencia de Ensayos
     de la Universidad de Miskatonic):
    \begin{align*}
      \binom{-2}{k}
	&= (-1)^k \, \binom{k + 1}{1}
	 = (-1)^k (k + 1) \\
      \binom{-3}{k}
	&= (-1)^k \, \binom{k + 2}{2}
	 = (-1)^k \, \frac{(k + 1) (k + 2)}{2}
    \end{align*}
    Estos coeficientes binomiales
    son polinomios de grado \(n - 1\) en \(k\).

    En general,
    resulta:
    \begin{equation}
      \label{eq:serie-binomio-negativo}
      \frac{1}{(1 - z)^{n + 1}}
	= \sum_{k \ge 0} \binom{n + k}{n} \, z^k
    \end{equation}
  \end{description}
  Un par de series útiles son las sumas dobles:
  \begin{align}
    \label{eq:sum-binomial-double}
    \sum_{n, k} \binom{n}{k} \, x^k y^n
      &= \sum_{n} (1 + x)^n y^n
       = \frac{1}{1 - (1 + x) y} \\
    \label{eq:sum-multiset-double}
    \sum_{n, k} \multiset{n}{k} \, x^k y^n
      &= \sum_{n} \frac{y^n}{(1 - x)^n}
       = \frac{1 - x}{1 - x - y}
  \end{align}
  En~\eqref{eq:sum-multiset-double}
  usamos la identidad~\eqref{eq:multiset=negative-binomial}.

  Interesante resulta la serie:
  \begin{equation*}
    \sum_{n \ge 0} \binom{n}{k} \, z^n
  \end{equation*}
  Como \(n\) es un entero no-negativo,
  sabemos que \(\binom{n}{k} = 0\) si no es que \(0 \le k \le n\),
  podremos ahorrarnos los límites de las sumas para simplificar:
  \begin{align}
    \sum_n \binom{n}{k} \, z^n
      &= \sum_n \binom{n + k}{k} \, z^{n + k} \notag \\
      &= z^k \sum_n \binom{n + k}{n} \, z^n \notag \\
      &= \frac{z^k}{(1 - z)^{k + 1}}
	    \label{eq:serie-binomio-n}
  \end{align}
  Al final usamos~\eqref{eq:serie-binomio-negativo}.
  Omitir los rangos de los índices ahorró interminables ajustes.

  Para multiconjuntos,%
    \index{multiconjunto!generatriz}
  usando~(\ref{eq:sum-multiset-double}):
  \begin{align*}
    \sum_{n \ge 0} \multiset{n}{k} z^n
      &= \left[ x^k \right] \frac{1 - x}{1 - x - z} \\
      &= \frac{1}{1 - z}
	   \left[ x^k \right] \frac{1 - x}{1 - x / (1 - z)} \\
      &= \frac{1}{1 - z}
	   \left[ x^k \right] (1 - x)
	     \sum_{n \ge 0} \frac{x^n}{(1 - z)^n}
	      \\
      &= \frac{1}{1 - z}
	   \left(
	     \frac{1}{(1 - z)^k} - \frac{[k > 0]}{(1 - z)^{k - 1}}
	   \right)  \\
      &= \frac{1 - [k > 0] (1 - z)}{(1 - z)^{k + 1}} \\
      &= \frac{(1 - [k > 0]) + [k > 0] z}{(1 - z)^{k + 1}}
  \end{align*}
  Como el numerador es \(1\) si \(k = 0\) y \(z\) cuando \(k > 0\)
  podemos simplificar:
  \begin{equation}
    \label{eq:serie-multiset-n}
    \sum_{n \ge 0} \multiset{n}{k} z^n
      = \frac{z^{[k > 0]}}{(1 - z)^{k + 1}}
  \end{equation}

\subsection{Otras series}
\label{sec:otras-series}

  Una serie común es la exponencial:%
    \index{serie de potencias!exponencial}
  \begin{equation}
    \label{eq:exponencial}
    \mathrm{e}^z
      = \sum_{n \ge 0} \frac{z^n}{n!}
  \end{equation}
  con sus variantes:
  \begin{equation*}
    \mathrm{e}^{a z}
      = \sum_{n \ge 0} \frac{a^n z^n}{n!} \hspace{7em}
    \mathrm{e}^{-z}
      = \sum_{n \ge 0} \frac{(-1)^n z^n}{n!}
  \end{equation*}
  A veces aparecen funciones trigonométricas:%
    \index{serie de potencias!seno}%
    \index{serie de potencias!coseno}
  \begin{equation*}
    \sin z
      = \sum_{n \ge 0} (-1)^n \frac{z^{2 n + 1}}{(2 n + 1)!} \qquad
    \cos z
      = \sum_{n \ge 0} (-1)^n \frac{z^{2 n}}{(2 n)!}
  \end{equation*}
  o hiperbólicas:%
    \index{serie de potencias!seno hiperbolico@seno hiperbólico}%
    \index{serie de potencias!coseno hiperbolico@coseno hiperbólico}
  \begin{equation*}
    \sinh z
      = \sum_{n \ge 0} \frac{z^{2 n + 1}}{(2 n + 1)!} \qquad
    \cosh z
      = \sum_{n \ge 0} \frac{z^{2 n}}{(2 n)!}
  \end{equation*}
  Una relación útil es la fórmula de Euler:%
    \index{Euler, formula de (exponencial complejo)@Euler, fórmula de (exponencial complejo)}
  \begin{equation}
    \label{eq:formula-Euler-exponencial}
    \mathrm{e}^{u + \mathrm{i} v}
      = \mathrm{e}^u (\cos v + \mathrm{i} \sin v)
  \end{equation}

  Es frecuente la serie para el logaritmo:%
    \index{serie de potencias!logaritmo}
  \begin{align}
    \frac{\mathrm{d}}{\mathrm{d} z} \, \ln (1 - z)
      &= - \frac{1}{1 - z}
       = - \sum_{n \ge 0} z^n \notag \\
    \ln (1 - z)
      &= - \sum_{n \ge 1} \frac{z^n}{n}
	   \label{eq:ln(1-z)}
  \end{align}
  Muchos ejemplos adicionales de series útiles
  se hallan en el texto de Wilf~\cite{wilf06:_gfology}.

\section{Notación para coeficientes}
\label{sec:funciones-generatrices:notacion}
\index{serie de potencias!extraer coeficiente}

  Comúnmente extraeremos el coeficiente de un término de una serie.
  Para esto,
  dadas las series:
  \begin{equation*}
    A(z)
      = \sum_{n \ge 0} a_n z^n
    \hspace{3em}
    B(z)
      = \sum_{n \ge 0} b_n z^n
  \end{equation*}
  usaremos la notación:
  \begin{equation*}
     \left[ z^n \right] A(z) = a_n
  \end{equation*}
  Tenemos algunas propiedades simples:
  \begin{equation*}
    \left[ z^n \right] z^k A(z)
      = \left[ z^{n - k} \right] A(z)
  \end{equation*}
  Una vez dado cuenta de \(z^k\),
  queda por extraer el coeficiente de \(z^{n - k}\) de \(A(z)\).
  \begin{equation*}
    \left[ z^n \right] (\alpha A(z) + \beta B(z))
      = \alpha \left[ z^n \right] A(z)
	  + \beta \left[ z^n \right] B(z)
  \end{equation*}
  Generalmente no hay términos con potencias negativas de \(z\),
  tales coeficientes serán cero.

  En términos de esta notación
  el teorema de Maclaurin%
    \index{Maclaurin, teorema de}
  queda expresado como:
  \begin{equation*}
    \left[ z^n \right] \, A(z)
      = \frac{1}{n!} \, A^{(n)}(0)
  \end{equation*}
  La notación es de Goulden y Jackson~%
    \cite{goulden04:_combin_enumer}.
  Puede extenderse muchísimo,
  ver Knuth~\cite{knuth94:_brack_notat_coeff_operat}
  y Merlini, Sprugnoli y Verri~%
    \cite{merlini07:_method_coeff}.
  La idea se le atribuye a Egorychev~%
  \cite{egorychev84:_integ_repres_comput_combin_sums},
  aunque con una notación mucho más engorrosa.

  Consideremos secuencias \(\langle a_n \rangle_{n \ge 0}\)
  y \(\langle a_n \rangle_{n \ge 0}\)
  relacionadas por:
  \begin{equation}
    \label{eq:b=binomial-transform-a}
    \sum_{0 \le k \le n} \binom{n}{k} a_k
      = b_n
  \end{equation}
  Si multiplicamos ambos lados por \(z^n / n!\)
  y sumamos sobre \(n \ge 0\)
  resulta:
  \begin{align*}
    \mathrm{e}^z \cdot \sum_{n \ge 0} a_n \frac{z^n}{n!}
      &= \sum_{n \ge 0} b_n \frac{z^n}{n!} \\
    \sum_{n \ge 0} a_n \frac{z^n}{n!}
      &= \mathrm{e}^{-z} \cdot \sum_{n \ge 0} b_n \frac{z^n}{n!}
  \end{align*}
  Comparar coeficientes entrega:
  \begin{equation}
    \label{eq:a=inverse-binomial-transform-b}
    a_n
      = \sum_{0 \le k \le n} \binom{n}{k} (-1)^k b_k
  \end{equation}

  \begin{theorem}[Transformación de Euler]
    \index{Euler, transformacion de@Euler, transformación de|textbfhy}
    \label{theo:Euler-transformation}
    Sea \(A(z) = \sum a_n z^n\).
    Entonces:
    \begin{equation}
      \label{eq:Euler-transformation}
       \sum_{0 \le k \le n} \binom{n}{k} \, a_k
	 = \left[ z^n \right] \,
	     \frac{1}{1 - z} \,
	       A \left(
		   \frac{z}{1 - z}
		 \right)
    \end{equation}
  \end{theorem}
  \begin{proof}
    Como para \(k > n\) el coeficiente binomial se anula,
    podemos extender la suma a todo \(k \ge 0\).
    Consideremos:
    \begin{align*}
      \sum_{n \ge 0} z^n \sum_{k \ge 0} \binom{n}{k} a_k
	&= \sum_{k \ge 0} a_k \sum_{n \ge 0} \binom{n}{k} z^n \\
	&= \sum_{k \ge 0} a_k \frac{z^k}{(1 - z)^{k + 1}} \\
	&= \frac{1}{1 - z} \,
	    \sum_{k \ge 0} a_k \left( \frac{z}{1 - z} \right)^k \\
	&= \frac{1}{1 - z} A \left(  \frac{z}{1 - z} \right)
    \end{align*}
    Esto es equivalente a lo enunciado.
  \end{proof}
  Un ejemplo de la aplicación de la transformación de Euler
  es el tratamiento de una suma
  discutido por Greene y Knuth~\cite{greene10:_math_anal_algor},
  originalmente de Jonassen y Knuth~%
    \cite{jonassen78:_trivial_algorithm}:
  \begin{equation*}
    S_m
      = \sum_{0 \le k \le m}
	  \binom{m}{k} \, \left( - \frac{1}{2} \right)^k \, \binom{2 k}{k}
  \end{equation*}
  Del teorema del binomio sabemos que:
  \begin{equation*}
    \frac{1}{\sqrt{1 + 2 z}}
      = \sum_{n \ge 0}
	  \binom{2 n}{n} \, \left( - \frac{1}{2} \right)^n
  \end{equation*}
  Por la transformación de Euler~\eqref{eq:Euler-transformation}:
  \begin{equation*}
    \sum_{0 \le k \le m}
      \binom{m}{k} \, \binom{2 k}{k}
	\, \left( - \frac{1}{2} \right)^k
      = \left[ z^m \right] \,
	  \frac{1}{1 - z} \,
	    \left( 1 + 2 \frac{z}{1 - z} \right)^{-1/2}
      = \left[ z^m \right] \,
	  \frac{1}{\sqrt{1 - z^2}}
  \end{equation*}
  Resulta:
  \begin{equation*}
    S_m
      = \begin{cases}
	  \displaystyle \binom{2 k}{k} \, 2^{-2 k} & m = 2 k \\
	  0					   & m = 2 k + 1
	\end{cases}
  \end{equation*}
  Prodinger~\cite{prodinger94:_old_sum}
  incluso usa esta suma para mostrar diversas técnicas
  para obtener una fórmula cerrada.

\section{Decimar}
\label{sec:decimar}
\index{serie de potencias!decimar}

  Uno de los máximos castigos para una legión romana era la \emph{decimación},
  que consistía en ejecutar a uno de cada diez miembros.
  Nuestro objetivo acá es mucho más radical,
  aunque bastante menos sangriento.

  Sea una secuencia \(\langle a_n \rangle_{n \ge 0}\),
  con función generatriz ordinaria \(A(z)\).
  Es fácil ver que:
  \begin{align}
    \sum_{n \ge 0} a_{2 n} z^{2 n}
      &= \frac{A(z) + A(-z)}{2}
	     \label{eq:a_even} \\
    \sum_{n \ge 0} a_{2 n + 1} z^{2 n + 1}
      &= \frac{A(z) - A(-z)}{2}
	     \label{eq:a_odd}
  \end{align}
  Esto es útil si nos interesan términos alternos:
  \begin{align}
    A_e(z)
      &= \sum_{n \ge 0} a_{2 n} z^n
	     \notag \\
      &= \frac{A(z^{1/2}) + A(-z^{1/2})}{2}
	     \label{eq:A_even} \\
    A_o(z)
      &= \sum_{n \ge 0} a_{2 n + 1} z^n
	      \notag \\
      &= \frac{A(z^{1/2}) - A(-z^{1/2})}{2 z^{1/2}}
	      \label{eq:A_odd}
  \end{align}
  Interesa extender esto a extraer uno cada \(m\) términos.

  Sea \(\omega_m\) una raíz primitiva de \(1\),%
    \index{raiz primitiva de 1@raíz primitiva de \(1\)}
  o sea por ejemplo el complejo:
  \begin{align}
    \omega_m
      &= \mathrm{e}^{\frac{2  \pi \mathrm{i}}{m}}
	      \notag \\
      &= \cos \frac{2  \pi}{m} + \mathrm{i} \sin \frac{2  \pi}{m}
	      \label{eq:omega_m}
  \end{align}
  De ahora en adelante anotaremos simplemente \(\omega\) para simplificar,
  \(m\) quedará dado por el contexto.
  Los \(m\) ceros del polinomio \(x^m - 1\)
  son \(\omega^k\) para \(0 \le k < m\),
  ya que:
  \begin{align*}
    \omega^k
      &= \mathrm{e}^{\frac{2  k \pi \mathrm{i}}{m}} \\
    \left( \omega^k \right)^m
      &= \mathrm{e}^{\frac{2  m k \pi \mathrm{i}}{m}} \\
      &= \mathrm{e}^{2	k \pi \mathrm{i}} \\
      &= 1
  \end{align*}
  Como \(\omega \ne 1\),
  de la factorización:
  \begin{equation*}
    x^m - 1
      = (x - 1) (x^{m - 1} + x^{m - 2} + \dotsc + 1)
  \end{equation*}
  vemos que:
  \begin{equation*}
    \sum_{0 \le k < m} \omega^k
      = 0
  \end{equation*}
  Resulta la curiosa
  (y útil) identidad:
  \begin{equation}
    \label{eq:sum-omega-powers}
    \sum_{0 \le k < m} \omega^{k s}
      = \begin{cases}
	  0 & \text{si \(m \centernot\mid s\)} \\
	  m & \text{si \(m \mid s\)}
	\end{cases}
  \end{equation}

  En vista de lo anterior,
  consideremos:
  \begin{align*}
    \sum_{0 \le k < m} \omega^{- k r} A(\omega^k z)
      &= \sum_{0 \le k < m} \omega^{- k r}
	   \sum_{n \ge 0} a_n \omega^{k n} z^n \\
      &= \sum_{n \ge 0} a_n z^n \sum_{0 \le k < m} \omega^{k (n - r)}
  \end{align*}
  La suma interna es \(m\) si \(m \mid n - r\),
  \(0\) en caso contrario.
  Con esto podemos construir:
  \begin{equation}
    \label{eq:decimation}
    \sum_{n \ge 0} a_{m n + r} z^{m n + r}
      = \frac{1}{m} \sum_{0 \le k < m} \omega^{-k r} A(\omega^k z)
  \end{equation}
  de donde es sencillo extraer la función generatriz de la secuencia
  \(\langle a_{m n + r} \rangle_{n \ge 0}\).

\section{Algunas aplicaciones combinatorias}
\label{sec:combinatorial-applications}
\index{combinatoria!generatrices}

  Se buscan las formas de llenar un canasto con \(n\) frutas si:
  \begin{itemize}
  \item El número de manzanas tiene que ser par.
  \item El número de plátanos debe ser un múltiplo de \(5\).
  \item Hay a lo más \(4\) naranjas.
  \item Hay a lo más \(1\) sandía.
  \end{itemize}

  Consideremos primero solo manzanas y plátanos.
  Usamos \(z\)
  (a través de sus potencias)
  para contar el número total de frutas,
  \(\langle m_k \rangle_{k \ge 0}\)
  es la secuencia de formas de tener \(k\) manzanas
  mientras \(\langle p_k \rangle_{k \ge 0}\)
  corresponde a los plátanos;
  y sea \(\langle c_k \rangle_{k \ge 0}\)
  la secuencia de las maneras de juntar \(k\) de estas frutas.
  Para cuatro frutas:
  \begin{equation*}
    c_4
      = m_0 \cdot p_4
	 + m_1 \cdot p_3
	 + m_2 \cdot p_2
	 + m_3 \cdot p_1
	 + m_4 \cdot p_0
  \end{equation*}
  Esta es exactamente la forma
  en que calculamos el coeficiente de \(z^4\) en la serie:
  \begin{equation*}
    \sum_{k \ge 0} c_k z^k
      = \left( \, \sum_{k \ge 0} m_k z^k \right)
	  \cdot \left( \, \sum_{k \ge 0} p_k z^k \right)
  \end{equation*}
  Generalizando esta observación,
  la función generatriz para el número de canastos con \(n\) frutas
  es el producto de las funciones generatrices
  para cada tipo de fruta.
  Estas son:
  \begin{itemize}
  \item Para manzanas:
    \begin{equation*}
      1 + z^2 + z^4 + \dotsb
	= \frac{1}{1 - z^2}
    \end{equation*}
  \item Los plátanos se representan por:
    \begin{equation*}
      1 + z^5 + z^{10} + \dotsb
	= \frac{1}{1 - z^5}
    \end{equation*}
  \item Para las naranjas:
    \begin{equation*}
      1 + z + z^2 + z^3 + z^4
	= \frac{1 - z^5}{1 - z}
    \end{equation*}
  \item Las sandías aportan:
    \begin{equation*}
      1 + z
    \end{equation*}
  \end{itemize}
  Uniendo las anteriores,
  la función generatriz del número de formas
  de tener canastos con \(n\)~frutas resulta ser:
  \begin{equation*}
    C(z)
      = \frac{1}{1 - z^2}
	  \cdot \frac{1}{1 - z^5}
	  \cdot \frac{1 - z^5}{1 - z}
	  \cdot (1 + z)
      = \frac{1}{(1 - z)^2}
  \end{equation*}
  Hay
  \((-1)^n \binom{-2}{n} = \binom{n + 1}{1} = n + 1\)
  maneras de llenar el canasto con \(n\) frutas.

  Al lanzar dos dados
  las sumas \(2\) y \(12\) se pueden obtener de una única manera,
  mientras para \(4\) hay tres (\(1 + 3 = 2 + 2 = 3 + 1\)).
  Para calcular el número de maneras de lograr cada valor,
  representamos un dado mediante la función generatriz:
  \begin{equation}
    \label{eq:gf-dado}
    D(z)
      = z + z^2 + z^3 + z^4 + z^5 + z^6
  \end{equation}
  con lo cual:
  \begin{equation}
    \label{eq:dos-dados}
    D^2(z)
      = z^2 + 2 z^3 + 3 z^4 + 4 z^5 + 5 z^6 + 6 z^7
	  + 5 z^8 + 4 z^9 + 3 z^{10} + 2 z^{11} + z^{12}
  \end{equation}
  El coeficiente de \(z^n\) da
  el número de formas de obtener \(n\) lanzando dos dados.

  Nace entonces la pregunta
  de si hay dados marcados en forma diferente
  que den la misma distribución
  (``dados locos'').%
    \index{dados locos|see{Sicherman, dados de}}
  Para construirlos
  debemos hallar funciones generatrices \(D_1(z)\) y \(D_2(z)\)
  que den el producto~\eqref{eq:dos-dados}.
  Debemos además tener que ambas representen dados,
  o sea tengan \(6\) caras,
  y que cada cara debe tener al menos un punto.
  Que cada cara tenga al menos un punto
  se traduce en que la función generatriz sea divisible por \(z\),
  el número de caras
  es simplemente el valor de la función en \(z = 1\).
  O sea:
  \begin{equation}
    \label{eq:dados-locos-caras}
    D_1(1)
      = D_2(1)
      = 6
  \end{equation}
  Factorizamos:
  \begin{equation}
    \label{eq:gf-dado-factorizada}
    D(z)
      = z (z + 1) (z^2 - z + 1) (z^2 + z + 1)
  \end{equation}
  Evaluando los factores en \(1\):
  \begin{equation}
    \label{eq:gf-dado-factorizada_1}
    D(1)
      = 1 \cdot 2 \cdot 1 \cdot 3
  \end{equation}
  Tanto \(D_1(z)\) como \(D_2(z)\)
  deben tener los factores \(z\),
  \(z + 1\) y \(z^2 + z + 1\);
  solo queda por redistribuir \(z^2 - z + 1\):
  \begin{align}
    D_1(z)
      &= z (z + 1) (z^2 + z + 1) \notag \\
      &= z + 2 z^2 + 2 z^3 + z^4 \label{eq:Sicherman-1} \\
    D_2(z)
      &= z (z + 1) (z^2 - z + 1)^2 (z^2 + z + 1) \notag \\
      &= z + z^3 + z^4 + z^5 + z^6 + z^8 \label{eq:Sicherman-2}
  \end{align}
  Fuera de dados comunes hay una posibilidad adicional,
  dados marcados con los multiconjuntos \(\{1, 2^2, 3^2, 4\}\)
  y \(\{1, 3, 4, 5, 6, 8\}\).
  Estos se conocen como \emph{dados de Sicherman}~%
    \cite{gardner78_2:_mathem_games}.%
    \index{Sicherman, dados de}
  Broline~\cite{broline79:_renum_faces_dice} estudia el problema
  para dados de números distintos de caras
  y más de dos dados.
  Gallian y Rusin~\cite{gallian79:_cyclot_polyn_nonst_dice}
  tratan un problema más general.

  Un problema antiguo popularizado por Pólya~%
    \cite{polya56:_picture_writing}%
    \index{cambio de monedas}
  pide determinar de cuántas formas se puede dar un dólar,
  usando monedas de \(1\), \(5\), \(10\), \(25\) y~\(50\) centavos.
  \begin{figure}[ht]
    \centering
    \pgfimage{images/coins-52}
    \caption{52 centavos en monedas}
    \label{fig:coins-52}
  \end{figure}
  La figura~\ref{fig:coins-52}
  muestra una manera de dar \(52\)~centavos.
  \begin{figure}[ht]
    \centering
    \pgfimage{images/coins-product}
    \caption{Colección de monedas como producto}
    \label{fig:coins-product}
  \end{figure}
  Podemos representar una colección de monedas
  como el ``producto'' de las cantidades de cada denominación,
  véase la figura~\ref{fig:coins-product}
  para una manera de tener \(62\)~centavos
  (el cuadrado representa una mesa vacía,
   ninguna moneda).
  \begin{figure}[ht]
    \centering
    \pgfimage{images/coins-1+5s}
    \caption{Series para 1 o 5 centavos}
    \label{fig:coins-1+5s}
  \end{figure}
  Todas las cantidades posibles
  usando solo monedas de~\(1\) o \(5\)~centavos
  se ilustran en la figura~\ref{fig:coins-1+5s},
  donde el signo~\(+\) separa las alternativas.
  Si ``multiplicamos'' las series,
  \begin{figure}[ht]
    \centering
    \pgfimage{images/coins-1x5s}
    \caption{Serie para combinaciones de 1 y 5 centavos}
    \label{fig:coins-1x5s}
  \end{figure}
  como muestra la figura~\ref{fig:coins-1x5s}
  obviando las mesas vacías y los signos de multiplicación,
  resultan todas las opciones
  para entregar una cantidad usando esas monedas.
  Nos interesa el número de maneras de tener,
  digamos,
  \(12\)~centavos,
  sin importar las monedas mismas.
  Esto lo logramos poniendo la denominación como exponente,
  o sea representando la moneda de \(5\)~centavos como \(z^5\).
  Al multiplicar se suman los exponentes,
  y al juntar los términos con el mismo exponente
  en su coeficiente estamos contando las maneras de tener esa suma.
  Las series de la figura~\ref{fig:coins-1+5s} quedan como:
  \begin{alignat*}{2}
    &1 + z + z^2 + z^3 + z^4 + \dotsb
      &\,&= \frac{1}{1 - z} \\
    &1 + z^5 + z^{10} + z^{15} + z^{20} + \dotsb
      &&= \frac{1}{1 - z^5}
  \end{alignat*}
  El coeficiente de \(z^{12}\)
  en \((1 + z + z^2 + \dotsb) (1 + z^5 + z^{10} + \dotsb)\)
  da el número de maneras de entregar \(12\)~centavos
  usando solo monedas de \(1\) y \(5\)~centavos:
  \begin{equation*}
    \frac{1}{(1 - z) (1 - z^5)}
      = 1 + z + z^2 + z^3 + z^4 + 2 z^5 + 2 z^6 + 2 z^7
	  + 2 z^8 + 3 z^{10} + 3 z^{11} + 3 z^{12} + 3 z^{13}
	  + \dotsb
  \end{equation*}
  Hay \(3\)~maneras,
  a saber:
  Sólo monedas de \(1\)~centavo,
  una moneda de \(5\)~centavos y siete de \(1\)~centavo,
  o dos de \(5\) y dos de \(1\).

  Las cantidades que se pueden entregar
  con la moneda de denominación \(d\)
  se representan por:
  \begin{equation*}
    1 + z^d + z^{2 d} + z^{3 d} + \dotsb
      = \frac{1}{1 - z^d}
  \end{equation*}
  Como combinar denominaciones corresponde a multiplicar las series,
  para el conjunto completo de denominaciones
  tenemos la función generatriz:
  \begin{equation}
    \label{eq:gf-coins}
    P(z)
      = \frac{1}
	 {(1 - z) (1 - z^5) (1 - z^{10}) (1 - z^{25}) (1 - z^{50})}
  \end{equation}
  El valor buscado es el coeficiente de \(z^{100}\)
  en~\eqref{eq:gf-coins}.

  No es viable expandir~\eqref{eq:gf-coins} hasta \(z^{100}\),
  veremos un camino alternativo.
  La serie~\eqref{eq:gf-coins} es el producto de cinco factores,
  conocemos el primero
  (la serie geométrica)
  e iremos adicionando los demás uno a uno.
  Supongamos que ya tenemos el producto
  de los dos primeros factores:
  \begin{equation*}
    \frac{1}{(1 - z) (1 - z^5)}
      = a_0 + a_1 z + a_2 z^2 + \dotsb
  \end{equation*}
  y queremos añadir el tercero:
  \begin{equation*}
    \frac{1}{(1 - z) (1 - z^5) (1 - z^{10})}
      = b_0 + b_1 z + b_2 z^2 + \dotsb
  \end{equation*}
  Vemos que:
  \begin{equation*}
    (b_0 + b_1 z + b_2 z^2 + \dotsb) (1 - z^{10})
      = a_0 + a_1 z + a_2 z^2 + \dotsb
  \end{equation*}
  Comparando coeficientes
  (es \(b_n = 0\) si \(n < 0\)):
  \begin{equation*}
    b_n
      = b_{n - 10} + a_n
  \end{equation*}
  Esta relación
  permite calcular los \(b_n\) si ya conocemos los \(a_n\),
  y obtenemos la serie completa en cuatro pasos similares
  al que discutimos recién.
  El cuadro~\ref{tab:coin-change}
  resume el cálculo hasta \(50\)~centavos
  (solo se dan los valores necesarios para obtener \(p_{50} = 50\)),
  el amable lector completará el cuadro
  y verificará que hay un total
  de 292~maneras de dar un dólar en monedas.
  \begin{table}[ht]
    \centering
    \begin{tabular}{|>{\(}l<{\)}|*{11}{>{\(}r<{\)}}|}
      \hline
      \multicolumn{1}{|r}{\(n = {}\)} &
	\rule[-0.3ex]{0pt}{3ex}%
	0 & 5 & 10 & 15 & 20 & 25 & 30 & 35 & 40 & 45 & 50 \\
      \hline
      \rule[-0.5ex]{0pt}{3ex}%
      (1 - z)^{-1}
	  & 1 & 1 & 1 & 1 & 1 &	 1 &  1 &  1 &	1 &  1 &  1 \\
      (1 - z^5)^{-1}
	  & 1 & 2 & 3 & 4 & 5 &	 6 &  7 &  8 &	9 & 10 & 11 \\
      (1 - z^{10})^{-1}
	  & 1 & 2 & 4 & 6 & 9 & 12 & 16 &    & 25 &    & 36 \\
      (1 - z^{25})^{-1}
	  & 1 &	  &   &	  &   & 13 &	&    &	  &    & 49 \\
      (1 - z^{50})^{-1}
	  & 1 &	  &   &	  &   &	   &	&    &	  &    & 50
    \end{tabular}
    \caption{Tabla para calcular $p_{50}$}
    \label{tab:coin-change}
  \end{table}
  En el clásico de Graham, Knuth y~Patashnik~%
    \cite{graham94:_concr_mathem}
  continúan este desarrollo.
  Aprovechan la forma especial de las recurrencias resultantes
  y obtienen una fórmula cerrada para \(p_n\).

  Un problema clásico propuesto por Sylvester en~1884
  es el siguiente:
  Si solo se tienen estampillas de \(5\) y \(17\)~centavos,
  ¿cuál es el máximo monto
  que \emph{no} se puede franquear con ellas?%
    \index{problema de franqueo|see{Frobenius, problema de}}

  La solución de Bogomolny~%
    \cite{bogomolny12:_Sylvester_2nd_look}
  muestra cómo representar conjuntos.
  Para generalizar,
  digamos que los montos de las estampillas son \(p\) y \(q\),
  ambos mayores a \(1\),
  con \(\gcd(p, q) = 1\).
  Si no fueran relativamente primos,
  habrían infinitos valores imposibles de representar
  (solo es posible representar múltiplos de \(\gcd(p, q)\)
   mediante expresiones de la forma \(a p + b q\)).

  Por la identidad de Bézout%
    \index{Bezout, identidad de@Bézout, identidad de}
  (ver la sección~\ref{sec:GCD})
  sabemos que hay \(u, v\) tales que \(u p - v q = 1\),
  sin pérdida de generalidad podemos suponer que \(u, v > 0\).
  Si tomamos \(x q\) para algún \(x\) por determinar,
  para \(1 \le k < q\) podemos escribir:
  \begin{equation*}
    x q + k
      = x q + k (u p - v q)
      = k u p + (x - k v) q
  \end{equation*}
  El primer término es siempre positivo,
  interesa acotar \(k v\) para asegurar que ambos sean no negativos
  y \(x q + k\) siempre sea representable.
  Como \(v\) es el inverso de \(q\) en \(\mathbb{Z}_p\)
  es \(1 \le v < p\),
  y por tanto al menos
  a partir de \((q - 1) (p - 1) q\) todos son representables.

  Formemos la familia de secuencias aritméticas
  \(f_a = \langle a p + b q  \rangle_{b \ge 0}\):
  \begin{alignat*}{2}
    &f_0
      &\,&= \langle \phantom{0}0 + 0,
	       \phantom{p}0 + q,
	       \phantom{p}0 + 2 q,
	       \phantom{p}0 + 3 q, \dotsc \rangle \\
    &f_1
      &&= \langle \phantom{0}p + 0,
		  \phantom{0}p + q,
		  \phantom{0}p + 2 q,
		  \phantom{0}p + 3 q, \dotsc \rangle \\
    &f_2
      &&= \langle
	    2 p + 0, 2 p + q, 2 p + 2 q, 2 p + 3 q, \dotsc
	  \rangle \\
    & &&\vdots \\
    &f_{q - 1}
      &&= \langle
	    (q - 1) p + 0, (q - 1) p + q, (q - 1) p + 2 q,
	      (q - 1) p + 3 q, \dotsc
	  \rangle
  \end{alignat*}
  La idea es que la secuencia \(f_k\)
  representa los franqueos posibles
  con \(k\) estampillas de \(p\) centavos
  y algún número de estampillas de \(q\) centavos.
  Como \(\gcd(p, q) = 1\),
  estas secuencias son disjuntas,
  y cubren todas las posibilidades de \(a p + b q\)
  con \(a, b \ge 0\).
  Los elementos de \(f_a\) son congruentes módulo \(q\),
  siendo \(p\) y \(q\) relativamente primos
  el conjunto \(\{a p \bmod q \colon 0 \le a < q\}\)
  es simplemente \(\{k \colon 0 \le k < q\}\).
  Si las secuencias hubiesen comenzado
  con los residuos respectivos,
  las secuencias cubrirían todo \(\mathbb{N}\);
  pero como \(f_a\) parte de \(a p\)
  la unión de las secuencias deja espacios al comienzo.
  Interesa hallar el máximo número que no aparece en la unión,
  que llamaremos \(g(p, q)\).

  Los elementos de la unión de las secuencias
  indicadas son los exponentes
  de la siguiente función generatriz
  (los coeficientes de la suma son todos uno,
   no hay intersección entre las secuencias):
  \begin{equation*}
    f(z)
      = \frac{1}{1 - z^q}
	  \, (1 + z^p + z^{2 p} + \dotsb + z^{(q - 1) p})
      = \frac{1 - z^{p q}}{(1 - z^p) (1 - z^q)}
  \end{equation*}
  Por el otro lado,
  el conjunto completo de los enteros no negativos
  se representa por:
  \begin{equation*}
    h(z)
      = 1 + z + z^2 + z^3 + \dotsb
      = \frac{1}{1 - z}
  \end{equation*}
  La diferencia entre las dos es un polinomio,
  cuyos exponentes indican los números que no se pueden representar:
  \begin{equation}
    \label{eq:gf-Frobenius}
    h(z) - f(z)
      = \frac{1}{1 - z} - \frac{1 - z^{p q}}{(1 - z^p) (1 - z^q)}
      = \frac{(1 - z^p) (1 - z^q) - (1 - z) (1 - z^{p q})}
	     {(1 - z) (1 - z^p) (1 - z^q)}
  \end{equation}
  Restar el grado del denominador del grado del numerador
  da el grado del polinomio:
  \begin{equation}
    \label{eq:Frobenius:g(p,q)}
    g(p, q)
      = (p q + 1) - (p + q + 1)
      = p q - p - q
  \end{equation}
  Esta teoría
  nos dice que la respuesta al problema específico planteado
  es que el máximo valor que no puede franquearse es
  \begin{equation*}
    g(5, 17)
      = 5 \cdot 17 - 5 - 17
      = 63
  \end{equation*}

  Otra pregunta es cuántos son los valores
  que no pueden representarse,
  que no es más que la suma de los coeficientes
  del polinomio~\eqref{eq:gf-Frobenius},
  o sea,
  el valor del mismo evaluado en \(z = 1\).
  Aplicando l'Hôpital%
    \index{Hopital, regla de@l'Hôpital, regla de}
  tres veces a~\eqref{eq:gf-Frobenius}
  entrega:
  \begin{equation}
    \label{eq:Frobenius-not-representable}
    \lim_{z \rightarrow 1} \, \left( h(z) - f(z) \right)
      = \frac{p q - p - q + 1}{2}
      = \frac{g(p, q) + 1}{2}
  \end{equation}
  Los no representables
  resultan ser \(32\) en nuestro caso específico.

  Este es el caso particular \(n = 2\) del problema de Frobenius,%
    \index{Frobenius, problema de}
  determinar para un conjunto de naturales relativamente primos
  \(\{a_1, a_2, \dotsc, a_n\}\)
  cuál es el máximo entero que no puede representarse
  como combinación lineal con coeficientes naturales.
  A este número se le llama el \emph{número de Frobenius}%
    \index{Frobenius, numero de@Frobenius, número de}
  del conjunto,
  y se anota \(g(a_1, \dotsc, a_n)\).
  Para \(n > 2\) no se conocen fórmulas generales,
  solo soluciones en casos particulares.
  A pesar de parecer muy especializado,
  este problema y variantes aparecen en muchas aplicaciones.
  Un resumen reciente de la teoría y algoritmos
  presenta Ramírez~Alfonsín~%
    \cite{ramirez06:_dioph_frobenius_probl}.

\section{Manipulación de series}
\label{sec:manipulacion-series}

  Sea una secuencia
  \(\left\langle a_n \right\rangle_{n \ge 0}
     = \left\langle
	 a_0, a_1, a_2, \dotsc, a_n, \dotsc
       \right\rangle\).
  La \emph{función generatriz} (ordinaria) de la secuencia es
  la serie de potencias:%
    \index{generatriz!ordinaria|textbfhy}
  \begin{equation*}
    A(z)=\sum_{0 \le n} a_n z^n
  \end{equation*}
  Anotaremos
  \(A(z)
     \ogf \left\langle a_n\right\rangle_{n \ge 0}\) en este caso
  (\emph{ogf} es por
     \emph{\foreignlanguage{english}
			   {Ordinary Generating Function}}).

  La \emph{función generatriz exponencial}%
    \index{generatriz!exponencial|textbfhy}
  de la secuencia es la serie:
  \begin{equation*}
    \widehat{A}(z)
      = \sum_{0 \le n} a_n \, \frac{z^n}{n!}
  \end{equation*}
  Anotaremos
  \(\widehat{A}(z)
     \egf \left\langle a_n\right\rangle_{n \ge 0}\) en este caso
  (\emph{egf} es por
     \emph{\foreignlanguage{english}
			   {Exponential Generating Function}}).

  Por comodidad,
  a veces escribiremos estas relaciones
  con la función generatriz al lado derecho.

\subsection{Reglas OGF}
\label{sec:reglas-OGF}
\index{generatriz!ordinaria!reglas}

  Las propiedades siguientes de funciones generatrices ordinarias
  son directamente las definiciones del caso
  o son muy simples de demostrar,
  sus justificaciones detalladas quedarán de ejercicios.

  \begin{description}
  \item[Linealidad:]
    Si \(A(z) \ogf \left\langle a_n \right\rangle_{n \ge 0}\)
    y \(B(z) \ogf \left\langle b_n \right\rangle_{n \ge 0}\),
    y \(\alpha\) y \(\beta\) son constantes,
    entonces:
    \begin{equation*}
      \alpha A(z) + \beta B(z)
	 \ogf \left\langle
		\alpha a_n + \beta b_n
	      \right\rangle_{n \ge 0}
    \end{equation*}
  \item[Secuencia desplazada a la izquierda:]
    Si
    \(A(z) \ogf \left\langle a_n \right\rangle_{n \ge 0}\),
    entonces:
    \begin{equation*}
      \frac{A(z) - a_0 - a_1 z - \dotsb - a_{k - 1} z^{k - 1}}{z^k}
	\ogf \left\langle a_{n + k}\right\rangle_{n \ge 0}
    \end{equation*}
  \item[Multiplicar por \(n\):]
    Consideremos:
    \begin{align*}
      A(z)
	&\ogf \left\langle a_n\right\rangle_{n \ge 0} \\
      z \, \frac{\mathrm{d}}{\mathrm{d} z} A(z)
	&\ogf \left\langle n a_n\right\rangle_{n \ge 0}
    \end{align*}
    Esta operación
    se expresa en términos del operador \(z \mathrm{D}\)
    (acá \(\mathrm{D}\) es por derivada,
     para abreviar).
    Además:
    \begin{equation*}
      (z \mathrm{D})^2 A(z)
	= z D (z D A(z))
	\ogf \left\langle n^2 a_n\right\rangle_{n \ge 0}
    \end{equation*}
    Nótese que
      \((z \mathrm{D})^2 = z \mathrm{D} + z^2 \mathrm{D}^2\)
    es diferente de \(z^2 \mathrm{D}^2\).
  \item[Multiplicar por un polinomio en \(n\):]
    Si \(p(n)\) es un polinomio,
    entonces:
    \begin{align*}
      p(z \mathrm{D}) A(z)
	&\ogf \left\langle p(n) a_n \right\rangle_{n \ge 0}
    \end{align*}
  \item[Convolución:]
    Si \(A(z) \ogf \left\langle a_n \right\rangle_{n \ge 0}\)
    y \(B(z) \ogf \left\langle b_n \right\rangle_{n \ge 0}\)
    entonces:
    \begin{equation*}
      A(z) \cdot B(z)
	\ogf \left\langle
	       \sum_{0 \le k \le n} a_k b_{n - k}
	      \right\rangle_{n \ge 0}
    \end{equation*}
  \item
    Sea \(k\) un entero positivo
    y \(A(z) \ogf \left\langle a_n\right\rangle_{n \ge 0}\),
    entonces:
    \begin{equation*}
      (A(z))^k
	\ogf \left\langle \sum_{n_1 + n_2 + \dotsb + n_k = n}
	       \left( a_{n_1} \cdot a_{n_2} \dotsm a_{n_k} \right)
	     \right\rangle_{n \ge 0}
    \end{equation*}
    Vale la pena tener presente el caso especial:
    \begin{equation*}
      (A(z))^2
	\ogf \left\langle
	       \sum_{0 \le i \le n} a_i a_{n - i}
	     \right\rangle_{n \ge 0}
    \end{equation*}
  \item[Sumas parciales:]
    Supongamos:
    \begin{equation*}
      A(z) \ogf \left\langle a_n \right\rangle_{n \ge 0}
    \end{equation*}
    Podemos escribir:
    \begin{equation*}
      \sum_{0 \le k \le n} a_k
	= \sum_{0 \le k \le n} 1 \cdot a_k
    \end{equation*}
    Esto no es más que la convolución de las secuencias
    \(\left\langle 1 \right\rangle_{n \ge 0}\)
    y \(\left\langle a_n \right\rangle_{n \ge 0}\),
    y la función generatriz de la primera es nuestra vieja conocida,
    la serie geométrica,
    con lo que:
    \begin{equation}
      \label{eq:sumas-parciales}
      \frac{A(z)}{1 - z}
	\ogf \left\langle
	       \sum_{0 \le k \le n} a_k
	     \right\rangle_{n \ge 0}
    \end{equation}
  \end{description}

  Un primer ejemplo clásico
  (ver por ejemplo Knuth~\cite{knuth98:_sortin_searc})
  es el análisis de búsqueda binaria.%
    \index{analisis de algoritmos@análisis de algoritmos!busqueda binaria@búsqueda binaria}
  Supongamos que contamos con un arreglo ordenado de \(n\) claves
  \(k_1 < k_2 < \dotsb < k_n\),
  dada una clave \(k\) nos interesa identificar \(1 \le j \le n\)
  tal que \(k = k_j\)
  (búsqueda exitosa).
  Búsqueda binaria compara \(k\) con el elemento medio,
  en \(r = \lfloor (n + 1) / 2 \rfloor\).
  Si \(k = k_r\),
  estamos listos.
  En caso contrario,
  si \(k < k_r\) seguimos la búsqueda en \(k_1, \dotsc, k_{r - 1}\),
  mientras que si \(k > k_r\)
  seguimos la búsqueda en \(k_{r + 1}, \dotsc, k_n\).
  Interesa
  el número promedio \(b_n\) de comparaciones en búsquedas exitosas,
  si \(k\) se elige al azar.
  Hay un único elemento que puede encontrarse con una comparación,
  el elemento medio.
  Hay dos que pueden encontrarse con dos comparaciones,
  y así sucesivamente
  hasta llegar a un máximo
  de \(1 + \lfloor \log_2 n \rfloor\) comparaciones.
  Si sumamos el número de comparaciones
  para cada una de las \(n\) claves
  obtenemos el número promedio de comparaciones:
  \begin{equation*}
    b_n
      = \frac{1}{n} \,
	  \left(
	    1 + 2 + 2 + 3 + 3 + 3 + 3
	      + \dotsb
	      + (1 + \lfloor \log_2 n \rfloor)
	  \right)
  \end{equation*}
  Para calcular la suma,
  consideramos la secuencia infinita
    \(\langle 0, 1, 2, 2, 3, 3, 3, 3, \dotsc\rangle\),
  que se obtiene de sumar secuencias
  \(\langle 0, 1, 1, \dotsc \rangle\),
  \(\langle 0, 0, 1, 1, \dotsc \rangle\),
  y así sucesivamente,
  donde la \(k\)\nobreakdash-ésima secuencia
  comienza con \(2^k\) ceros.
  Podemos representar la secuencia
  como los coeficientes de la serie:
  \begin{equation*}
    L(z)
      = \frac{z}{1 - z}
	  + \frac{z^2}{1 - z}
	  + \dotsb
	  + \frac{z^{2^k}}{1 - z}
	  + \dotsb
      = \sum_{k \ge 0} \frac{z^{2^k}}{1 - z}
  \end{equation*}
  Nos interesan sumas parciales:
  \begin{align*}
    n b_n
      &= \left[ z^n \right] \,
	   \frac{1}{1 - z} \sum_{k \ge 0} \frac{z^{2^k}}{1 - z} \\
      &= \left[ z^n \right] \,
	   \sum_{k \ge 0} \frac{z^{2^k}}{(1 - z)^2} \\
      &= \sum_{k \ge 0} \left[ z^{n - 2^k} \right] (1 - z)^{-2} \\
      &= \sum_{k \ge 0} \binom{n - 2^k + 1}{n - 2^k} \\
      &= \sum_{0 \le k \le \lfloor \log_2 n \rfloor}
	   (n + 1 - 2^k) \\
      &= (n + 1) (\lfloor \log_2 n \rfloor + 1)
	   - \sum_{0 \le k \le \lfloor \log_2 n \rfloor} 2^k \\
      &= (n + 1) \lfloor \log_2 n \rfloor
	   + n - 2^{\lfloor \log_2 n \rfloor + 1} + 2
  \end{align*}
  Manipulaciones formales que dan directamente lo que buscamos.

\subsection{Reglas EGF}
\label{sec:reglas-EGF}
\index{generatriz!exponencial!reglas}

  Las siguientes resumen propiedades
  de las funciones generatrices exponenciales.
  Son simples de demostrar,
  y las justificaciones que no se dan acá quedarán de ejercicios.
  \begin{description}
  \item[Linealidad:]
    Si \(\widehat{A}(z)
	   \egf \left\langle a_n \right\rangle_{n \ge 0}\)
    y \(\widehat{B}(z)
	  \egf \left\langle b_n \right\rangle_{n \ge 0}\),
    y \(\alpha\) y \(\beta\) son constantes,
    entonces:
    \begin{equation*}
      \alpha \widehat{A}(z) + \beta \widehat{B}(z)
	\egf \left\langle
	       \alpha a_n + \beta b_n
	     \right\rangle_{n \ge 0}
    \end{equation*}
  \item[Secuencia desplazada a la izquierda:]
    Si \(\widehat{A}(z)
	   \egf \left\langle a_n \right\rangle_{n \ge 0}\),
    entonces:
    \begin{equation*}
      \mathrm{D}^k \widehat{A}(z)
	\egf \left\langle a_{n + k} \right\rangle_{n \ge 0}
    \end{equation*}
  \item[Multiplicación por un polinomio en \(n\):]
    Si es \(\widehat{A}(z)
	      \egf \left\langle a_n \right\rangle_{n \ge 0}\),
    y \(p\) es un polinomio,
    entonces:
    \begin{equation*}
      p(z \mathrm{D}) \widehat{A}(z)
	\egf \left\langle p(n) a_n \right\rangle_{n \ge 0}
    \end{equation*}
    Es la misma que en funciones generatrices ordinarias,
    ya que la operación \(z \mathrm{D}\)
    no altera el exponente en \(z^n\).
  \item[Convolución binomial:]
    Si \(\widehat{A}(z)
	   \egf \left\langle a_n \right\rangle_{n \ge 0}\) y
    \(\widehat{B}(z)
	\egf \left\langle b_n \right\rangle_{n \ge 0}\)
    entonces:
    \begin{align*}
      \widehat{A}(z) \cdot \widehat{B}(z)
	&= \sum_{n \ge 0}\biggl( \,
			   \sum_{0 \le k \le n}
			   \frac{a_k}{k!} \, \frac{b_{n - k}}
						  {(n - k)!}
			 \biggr) z^n \\
	&= \sum_{n \ge 0} \biggl( \,
			    \sum_{0 \le k \le n}
			       \binom{n}{k} \, a_k b_{n - k}
			  \biggr)
	       \frac{z^n}{n!}
    \end{align*}
    Vale decir:
    \begin{equation*}
      \widehat{A}(z) \cdot \widehat{B}(z)
	\egf \left\langle
	       \sum_{0 \le k \le n} \binom{n}{k} \, a_k b_{n - k}
	     \right\rangle_{n \ge 0}
    \end{equation*}
  \item[Términos individuales:]
    Es fácil ver que si
    \(\widehat{A}(z)
	\egf \left\langle a_n \right\rangle_{n \ge 0}\) entonces:
    \begin{equation*}
      a_n = \widehat{A}^{(n)}(0)
    \end{equation*}
    Esto en realidad no es más que el teorema de Maclaurin.%
      \index{Maclaurin, teorema de}
  \end{description}

\section[\texorpdfstring{El truco $z \mathrm{D} \log$}
			{Derivada logarítmica}]
	{\protect\boldmath
	   \texorpdfstring{El truco $z \mathrm{D}\log$}
			  {Derivada logarítmica}%
       \protect\unboldmath}
\index{derivada logaritmica@derivada logarítmica|textbfhy}

  Los logaritmos ayudan a simplificar expresiones con exponenciales
  y potencias.
  Pero terminamos con el logaritmo de una suma
  si el argumento es una serie,
  que es algo bastante feo de contemplar.
  Eliminar el logaritmo se logra derivando:
  \begin{equation*}
    \frac{\mathrm{d} \ln(A)}{\mathrm{d} z} = \frac{A'}{A}
  \end{equation*}
  Esto es mucho más decente.
  Multiplicamos por \(z\)
  para reponer la potencia ``perdida'' al derivar.

\subsubsection*{Receta:}
\index{derivada logaritmica@derivada logarítmica!receta}

  \begin{enumerate}
  \item
    Aplicar \(z \mathrm{D} \ln\).
  \item
    Multiplicar para eliminar fracciones.
  \item
    Igualar coeficientes.
  \end{enumerate}

\section{Ejemplos de manipulación de series}
\label{sec:gf-ejemplos}

  Un ejemplo inicial de aplicación de las ideas planteadas
  es obtener la suma de los primeros \(N\) cuadrados.%
    \index{suma!cuadrados}
  Por la suma de la serie geométrica,%
    \index{serie geometrica@serie geométrica!suma}
  teorema~\ref{theo:suma-geometrica}:
  \begin{align*}
    1 + z + z^2 + \dotsb + z^N
      &= \frac{1 - z^{N + 1}}{1 - z} \\
    (z \mathrm{D})^2 \, \left( 1 + z + z^2 + \dotsb + z^N \right)
      &= (z \mathrm{D})^2 \, \frac{1 - z^{N + 1}}{1 - z} \\
    \left.
      \left( 0^2 + 1^2 z + 2^2 z^2 + \dotsb + N^2 z^N \right)
    \right|_{z = 1}
      &= \lim_{z \rightarrow 1} \,
	   (z \mathrm{D})^2 \, \frac{1 - z^{N + 1}}{1 - z}
  \end{align*}
  Nótese que todas las expresiones involucradas son polinomios,
  con lo que cuestiones de convergencia y validez de las operaciones
  no son problema.

  El resto es derivar,
  calcular límites y álgebra:
  \begin{equation*}
    \sum_{1 \le k \le N} k^2 = \frac{N (N + 1) (2 N + 1)}{6}
  \end{equation*}
  La misma idea sirve para otras potencias.

  La maquinaria de funciones generatrices
  permite obtener en forma rutinaria
  resultados que de otra forma serían complicados de sospechar,
  y luego deberían ser demostrados por inducción.
  La operatoria suele ser tediosa,
  es útil tener un programa de álgebra simbólica
  (como \texttt{maxima}~\cite{maxima14b:_computer_algebra})%
    \index{maxima@\texttt{maxima}}
  a la mano.

  Otra aplicación es obtener la serie para \(A(z)^\alpha\),%
    \index{serie de potencias!potencia}
  una potencia arbitraria
  (\(\alpha \in \mathbb{C}\))
  de una serie \(A(z)\) que ya conocemos.
  Sea entonces:
  \begin{equation*}
    A(z)
      = \sum_{n \ge 0} a_n z^n
  \end{equation*}
  donde \(a_0 \ne 0\).
  Definimos:
  \begin{equation*}
    B(z)
      = A^\alpha (z) = \sum_{n \ge 0} b_n z^n
  \end{equation*}
  Aplicando la receta \(z \mathrm{D} \log\) obtenemos:%
    \index{derivada logaritmica@derivada logarítmica}
  \begin{align*}
    \frac{z B'(z)}{B(z)}
      &= \alpha z \, \frac{A'(z)}{A(z)} \\
    z B'(z) \cdot A(z)
      &= \alpha z A'(z) \cdot B(z) \\
    \biggl( \, \sum_{n \ge 0} n b_n z^n \biggr)
       \cdot \biggl( \, \sum_{n \ge 0} a_n z^n \biggr)
      &= \alpha \biggl( \,
		   \sum_{n \ge 0} n a_n z^n
		 \biggr)
	     \cdot \biggl( \,
		     \sum_{n \ge 0} b_n z^n
		   \biggr) \\
    \sum_{n \ge 0} \biggl( \,
		     \sum_{0 \le k \le n} k b_k a_{n - k}
		   \biggr) z^n
      &= \sum_{n \ge 0}
	   \biggl( \,
	     \sum_{0 \le k \le n} \alpha k a_k b_{n - k}
	   \biggr) z^n
  \end{align*}
  De acá sigue,
  igualando coeficientes:
  \begin{equation*}
    \sum_{0 \le k \le n} a_k (n - k) b_{n - k}
      = \sum_{0 \le k \le n} \alpha k a_k b_{n - k}
  \end{equation*}
  Nuevamente,
  esto involucra solo finitas operaciones.
  Finalmente:
  \begin{align*}
    \sum_{0 \le k \le n}
      \bigl(
	a_k (n - k) b_{n - k} - \alpha k a_k b_{n - k}
      \bigr)
      &= 0 \\
    \sum_{0 \le k \le n} (n - k - \alpha k) a_k b_{n - k}
      &= 0
  \end{align*}
  de donde resulta al separar el término con \(k = 0\):
  \begin{align*}
    n a_0 b_n
      &= -\biggl( \,
	    \sum_{1 \le k \le n} (n - k - \alpha k) a_k b_{n - k}
	  \biggr) \\
    b_n
      &= -\frac{1}{n a_0}
	  \, \sum_{1 \le k \le n} (n - k - \alpha k) a_k
    b_{n - k}
  \end{align*}
  Para comenzar la recurrencia,
  usamos:
  \begin{equation*}
    b_0 = a_0^\alpha
  \end{equation*}
  Compárese esta recurrencia con la expresión explícita
  para una potencia entera de una serie
  que derivamos antes.

\section{Funciones generatrices en combinatoria}
\label{sec:FG-combinatoria}

  De nuevo
  la Competencia de Ensayos de la Universidad de Miskatonic.
  Para simplificar notación,
  sea \(a_r = b_{2 r + 1}\).
  Resulta:
  \begin{equation}
    \label{eq:recurrence-UMEC-a-1}
    a_r
      = a_{r - 1} + r + 1
  \end{equation}
  La condición inicial es \(a_0 = b_1 = 1\).
  Llamemos \(A(z)\) a la función generatriz ordinaria
  de la secuencia \(\left\langle a_r\right\rangle_{r \ge 0}\):
  \begin{equation*}
    A(z)
      = \sum_{r \ge 0} a_r z^r
  \end{equation*}

  La recurrencia~\eqref{eq:recurrence-UMEC-a-1}%
    \index{recurrencia}
  es incómoda de manejar como está escrita,
  primero ajustamos los índices
  para no hacer referencia a términos previos:
  \begin{equation}
    \label{eq:recurrence-UMEC-a-2}
    a_{r + 1} - a_r
      = r + 2
  \end{equation}
  Las funciones generatrices de los términos al lado izquierdo
  de la recurrencia~\eqref{eq:recurrence-UMEC-a-2}
  son:%
    \index{generatriz!ordinaria}
  \begin{equation*}
    \left\langle a_{r + 1}\right\rangle_{r \ge 0}
      \ogf \frac{A(z) - a_0}{z}
      = \frac{A(z) - 1}{z}
    \hspace{3em}
    \left\langle a_r\right\rangle_{r \ge 0}
      \ogf A(z)
  \end{equation*}
  Necesitamos además la función generatriz de la secuencia \(r + 2\)
  que aparece al lado derecho,
  que no es más
  que la secuencia \(\left\langle 1\right\rangle_{r \ge 0}\)
  multiplicada por el polinomio \(r + 2\),
  con lo que:
  \begin{equation*}
    \left\langle r + 2\right\rangle_{r \ge 0}
      \ogf (z \mathrm{D} + 2) \, \frac{1}{1 - z}
      = \frac{z}{(1 - z)^2} + \frac{2}{1 - z}
  \end{equation*}
  Combinando las anteriores,
  tenemos:
  \begin{equation*}
    \frac{A(z) - 1}{z} - A(z)
      = \frac{z}{(1 - z)^2} + \frac{2}{1 - z}
  \end{equation*}
  Despejando \(A(z)\) se tiene:
  \begin{equation*}
    A(z)
      = \frac{1}{(1 - z)^3}
  \end{equation*}
  y los coeficientes del caso son inmediatos:
  \begin{equation*}
    a_r
      = (-1)^r \binom{-3}{r}
      = \binom{2 + r}{2}
      = \frac{(r + 2) (r + 1)}{2}
  \end{equation*}
  Nuevamente resulta:
  \begin{equation*}
    b_{2 r + 1}
      = \frac{(r + 2) (r + 1)}{2}
  \end{equation*}
  Esta derivación es aún más simple que la anterior.
  Siempre que sea posible
  se deben usar las propiedades de funciones generatrices,
  debe recurrirse a la receta general dada anteriormente
  solo cuando no es claro cómo aplicarlas.

  Podemos igualmente intentar
  con la función generatriz exponencial:%
    \index{generatriz!exponencial}
  \begin{equation*}
    \widehat{A}(z)
      = \sum_{r \ge 0} a_r \frac{z^r}{r!}
  \end{equation*}
  Aplicando las propiedades respectivas
  a~\eqref{eq:recurrence-UMEC-a-2}:
  (refiérase a la sección~\ref{sec:reglas-EGF}):
  \begin{align*}
    \widehat{A}'(z) - \widehat{A}(z)
      &= \left(
	   z \frac{\mathrm{d}}{\mathrm{d} z} + 2
	 \right) \mathrm{e}^z \\
      &= z \mathrm{e}^z + 2 \mathrm{e}^z
  \end{align*}
  Como condición inicial tenemos:
  \begin{equation*}
    \widehat{A}(0)
      = a_0
      = 1
  \end{equation*}
  La solución de la ecuación diferencial es:%
    \index{ecuacion diferencial@ecuación diferencial}
  \begin{equation*}
    \widehat{A}(z)
      = \frac{\mathrm{e}^z}{2} (z^2 + 4 z + 2)
  \end{equation*}

  Ahora tenemos dos caminos posibles:
  Expresar la solución mediante las propiedades,
  o calcular los términos mediante la expansión en serie.
  Para aplicar las propiedades,
  notamos:
  \begin{equation*}
    \frac{1}{2} \left( z^2 + 4 z + 2 \right) \mathrm{e}^z
      = \frac{1}{2}
	  \left( z^2 D^2 + 4 z D + 2 \right)
	  \mathrm{e}^z \\
  \end{equation*}
  Además:
  \begin{align*}
    (z D)^2
      &= z^2 D^2 + z D \\
    z^2 D^2 + 4 z D + 2
      &= (z D)^2 + 3 z D + 2
  \end{align*}
  O sea:
  \begin{equation*}
    \widehat{A}(z)
      = \frac{1}{2}
	  \left( (z D)^2 + 3 z D + 2 \right)
	  \mathrm{e}^z
  \end{equation*}
  Esto corresponde a:
  \begin{equation*}
    b_{2 r + 1}
      = a_r
      = \frac{1}{2} (r^2 + 3 r + 2)
      = \frac{(r + 2) (r + 1)}{2}
  \end{equation*}

  El otro camino es:
  \begin{align*}
    \mathrm{e}^z \frac{z^2 + 4 z + 2}{2}
      &= \sum_{r \ge 0}
	   \left(
	     \frac{z^{r + 2}}{2 r!}
	       + \frac{2 z^{r + 1}}{r!}
	       + \frac{z^r}{r!}
	   \right) \\
      &= \frac{1}{2} \sum_{r \ge 2} \frac{z^r}{(r - 2)!}
	   + 2 \sum_{r \ge 1} \frac{z^r}{(r - 1)!}
	   + \sum_{r \ge 0} \frac{z^r}{r!} \\
      &= \sum_{r \ge 0}
	   \left(
	     \frac{r (r - 1)}{2}
	       + 2 r + 1
	   \right) \frac{z^r}{r!} \\
      &= \sum_{r \ge 0}
	   \frac{(r + 2) (r + 1)}{2} \, \frac{z^r}{r!}
  \end{align*}
  y nuevamente:
  \begin{equation*}
    b_{2 r + 1}
      = a_r
      = \frac{(r + 2) (r + 1)}{2}
  \end{equation*}

  Los objetos colgados para exhibición en la función generatriz
  no tienen porqué ser números.
  Un polinomio \(f(x_1, x_2, \dotsc, x_n)\)
  se llama \emph{simétrico}%
    \index{polinomio!simetrico@simétrico|textbfhy}
  si para cualquier permutación \(\sigma\) de \([n]\):
  \begin{equation}
    \label{eq:symmetric-polynomial-definition}
    f(x_1, x_2, \dotsc, x_n)
      = f(x_{\sigma(1)}, x_{\sigma(2)}, \dotsc, x_{\sigma(n)})
  \end{equation}
  Vale decir,
  el polinomio se mantiene inalterado
  bajo cualquier reordenamiento de variables.

  Considerando polinomios homogéneos
  de grado \(m\) en \(n\) variables,
  están las familias:
  \begin{align}
    e_m(x_1, x_2, \dotsc, x_n)
      &= \sum_{k_1 < k_2 < \dotsb < k_m}
	   x_{k_1} x_{k_2} \dotsm x_{k_m}
	 \label{eq:symmetric-polynomial-e} \\
    h_m(x_1, x_2, \dotsc, x_n)
      &= \sum_{k_1 + k_2 + \dotsb + k_n = m}
	   x_1^{k_1} x_2^{k_2} \dotsm x_n^{k_n}
	 \label{eq:symmetric-polynomial-h} \\
    p_m(x_1, x_2, \dotsc, x_n)
      &= \sum_{1 \le k \le n} x_k^m
	 \label{eq:symmetric-polynomial-p}
  \end{align}
  Los \(e_m\) son los llamados \emph{elementales}%
    \index{polinomio!simetrico@simétrico!elemental|textbfhy}
  (ya nos tropezamos con ellos
   en las fórmulas de Vieta~\eqref{eq:Vieta-formulas}),
  los \(h_m\) se llaman \emph{homogéneos completos}%
    \index{polinomio!simetrico@simétrico!homogeneo completo@homogéneo completo|textbfhy}
  y los \(p_m\) simplemente \emph{sumas de potencias}.%
    \index{polinomio!simetrico@simétrico!suma de potencias|textbfhy}
  Por ejemplo,
  para \(n = 3\):
  \begin{align*}
    e_2(x_1, x_2, x_3)
      &= x_1 x_2 + x_1 x_3 + x_2 x_3 \\
    h_3(x_1, x_2, x_3)
      &= x_1^3 + x_1^2 x_2 + x_1^2 x_3
	     + x_1 x_2^2 + x_1 x_2 x_3 + x_1 x_3^2
	  + x_2^3 + x_2^2 x_3 + x_2 x_3^2
	  + x_3^3 \\
    p_3(x_1, x_2, x_3)
      &= x_1^3 + x_2^3 + x_3^3
  \end{align*}
  Tenemos los siguientes casos especiales:
  \begin{align}
    e_0(x_1, x_2, \dotsc, x_n)
      &= 1 \label{eq:e0=1} \\
    h_0(x_1, x_2, \dotsc, x_n)
      &= 1 \label{eq:h0=1} \\
    p_0(x_1, x_2, \dotsc, x_n)
      &= n \label{eq:p0=n}
  \end{align}
  Definamos las funciones generatrices:%
    \index{generatriz!ordinaria}%
    \index{polinomio!simetrico@simétrico!generatriz}
  \begin{align}
    E(t)
      &= \sum_{m \ge 0} e_m (x_1, x_2, \dotsc, x_n) t^m \\
      &= \prod_{1 \le k \le n} (1 + x_k t)
	 \label{eq:symmetric-polynomial-e-GF} \\
    H(t)
      &= \sum_{m \ge 0} h_m(x_1, x_2, \dotsc, x_n) t^m \\
      &= \prod_{1 \le k \le n} \frac{1}{1 - x_k t}
	 \label{eq:symmetric-polynomial-h-GF} \\
    P(t)
      &= \sum_{m \ge 0} p_{m + 1} (x_1, x_2, \dotsc, x_n) t^m \\
      &= \sum_{1 \le k \le n} \frac{x_k}{1 - x_k t}
	 \label{eq:symmetric-polynomial-p-GF}
  \end{align}
  Las fórmulas dadas debieran estar claras:
  En \(E(t)\) contribuyen al coeficiente de \(t^m\)
  los factores para \(m\) variables diferentes;
  en \(H(t)\)
  vemos que al expandir las series geométricas de cada factor
  estas dan la variable elevada a cada posible potencia,
  y las combinaciones posibles que dan \(t^m\)
  son exactamente
  las indicadas en~\eqref{eq:symmetric-polynomial-h};
  mientras en \(P(t)\) el coeficiente de \(t^m\)
  proviene de la suma de todas las variables elevadas a \(m + 1\).

  Hagamos uso de estas funciones generatrices ahora.
  De~\eqref{eq:symmetric-polynomial-e-GF}
  y~\eqref{eq:symmetric-polynomial-h-GF} está claro que:
  \begin{equation*}
    E(t) H(-t)
      = 1
  \end{equation*}
  Comparando coeficientes
  (al lado derecho tenemos \(1 + 0 z + 0 z^2 + \dotsb\)):
  \begin{equation}
    \label{eq:symmetric-polynomial-e-h}
    \sum_{0 \le r \le m}
      (-1)^{m - r} e_r (x_1, \dotsc, x_n) \, h_{m - r} (x_1, \dotsc, x_n)
      = [m = 1]
  \end{equation}
  También vemos que:
  \begin{align*}
    \ln E(t)
      &= \sum_{1 \le k \le n} \ln (1 + x_k t) \\
    \frac{E'(t)}{E(t)}
      &= \sum_{1 \le k \le n} \frac{x_k}{1 + x_k t} \\
      &= P(-t) \\
    E'(t)
      &= E(t) P(-t)
  \end{align*}
  De acá,
  comparando coeficientes:
  \begin{align}
    (-1)^m m e_m (x_1, \dotsc, x_n)
      &= \sum_{0 \le r \le m - 1}
	   e_r (x_1, \dotsc, x_n)
	     \cdot (-1)^{m - 1 - r} p_{m - r} (x_1, \dotsc, x_n)
	 \notag \\
    m e_m (x_1, \dotsc, x_n)
      &= \sum_{0 \le r \le m - 1}
	   (-1)^{r - 1} e_r (x_1, \dotsc, x_n)
	     \, p_{m - r} (x_1, \dotsc, x_n)
    \label{eq:symmetric-polynomial-e-p}
  \end{align}
  Similarmente:
  \begin{align*}
    \ln H(t)
      &= - \sum_{1 \le k \le n} \ln (1 - x_k t) \\
    \frac{H'(t)}{H(t)}
      &= P(t) \\
    H'(t)
      &= H(t) P(t)
  \end{align*}
  Igual que antes:
  \begin{equation}
    \label{eq:symmetric-polynomial-h-p}
    m h_m (x_1, \dotsc, x_n)
      = \sum_{0 \le r \le m - 1}
	   h_r (x_1, \dotsc, x_n) \, p_{m - r} (x_1, \dotsc, x_n)
  \end{equation}
  Siquiera sospechar
  las relaciones~\eqref{eq:symmetric-polynomial-e-h},
  \eqref{eq:symmetric-polynomial-e-p}
  y~\eqref{eq:symmetric-polynomial-h-p}
  de alguna otra forma sería sobrehumano.

  Otro ejemplo lo ofrecen las \emph{fuentes}
  (\emph{\selectlanguage{english}{fountain}} en inglés),%
    \index{fountain@\emph{\selectlanguage{english}{fountain}}|see{fuente}}
  formadas por filas de monedas
  de forma que cada moneda esté en contacto
  con dos monedas de la fila inferior.
  Si la fuente es tal que las monedas en cada fila están contiguas,
  se les llama \emph{fuentes de bloque}%
    \index{fuente}%
  (en inglés \emph{\foreignlanguage{english}{block fountain}}).%
    \index{block fountain@\emph{\selectlanguage{english}{block fountain}}|see{fuente}}
  \begin{figure}[ht]
    \centering
    \pgfimage{images/fountain}
    \caption{Una fuente de bloque}
    \label{fig:fountain}
  \end{figure}
  La figura~\ref{fig:fountain} ilustra una fuente de bloque.
  Interesa saber el número de fuentes de bloque
  cuya primera fila
  (su base)
  tiene \(n\)~monedas,
  llamémosle \(f_n\) a este número.

  Un poco de experimentación lleva a \(f_0 = 1\)
  (hay una única forma de armar una fuente con base 0),
  \(f_1 = 1\),
  \(f_2 = 2\)
  y \(f_3 = 5\).
  Es claro que si a una fuente con base \(n\)~monedas
  le quitamos la base,
  queda una fuente con base a lo más \(n - 1\)~monedas.
  Si no hay monedas en la segunda fila,
  hay una sola fuente;
  si es \(k \ge 1\) el largo de la segunda fila de monedas,
  tenemos una fuente de base~\(k\) a partir de la segunda fila
  y esta fuente puede ubicarse sobre la base
  en \((n - 1) - k + 1 = n - k\) posiciones.
  En consecuencia
  tenemos la recurrencia:%
    \index{fuente!recurrencia}
  \begin{equation}
    \label{eq:recurrence-fuentes}
    f_n
      = 1 + \sum_{1 \le k \le n} (n - k) f_k
      \quad (n \ge 1)
      \qquad
      f_0
	= 1
  \end{equation}
  Esto da los valores:
  \begin{equation}
    \label{eq:fountains-sequence}
    \langle 1, 1, 2, 5, 13, 34, 89, 233, 610, 1597, \dotsc \rangle
  \end{equation}
  La sumatoria en~\eqref{eq:recurrence-fuentes}
  es la convolución de \(\langle n \rangle_{n \ge 1}\)
  con \(\langle f_n \rangle_{n \ge 1}\).
  Definimos la función generatriz ordinaria:
  \begin{equation}
    \label{eq:gf-fountain}
    f(z)
      = \sum_{n \ge 0} f_n z^n
  \end{equation}
  Como:
  \begin{equation*}
    \frac{1}{1 - z} - 1
      \ogf \langle 1 \rangle_{n \ge 1}
    \hspace{3em}
    \frac{1}{(1 - z)^2} - 1
      \ogf \langle n \rangle_{n \ge 1}
    \hspace{3em}
    f(z) - 1
      \ogf \langle f_n \rangle_{n \ge 1}
  \end{equation*}
  aplicando las propiedades
  de las funciones generatrices ordinarias resulta:%
    \index{fuente!generatriz}
  \begin{equation}
    \label{eq:fe-fountains}
    f(z) - 1
      = \frac{z}{1 - z} + \frac{z}{(1 - z)^2} \cdot (f(z) - 1)
  \end{equation}
  Despejando \(f(z)\) obtenemos:
  \begin{equation}
    f(z)
      = \frac{1 - 2 z}{1 - 3 z + z^2}
      = \frac{5 + \sqrt{5}}{10}
	  \cdot \frac{1}{1 - z \frac{3 - \sqrt{5}}{2}}
	+ \frac{5 - \sqrt{5}}{10}
	    \cdot \frac{1}{1 - z \frac{3 + \sqrt{5}}{2}}
		  \label{eq:gf-fountain-pf}
  \end{equation}
  Ciertamente bastante feo,
  pero da lugar a la expansión explícita:
  \begin{equation}
    \label{eq:seq-fountain}
    f_n
      = \frac{5 + \sqrt{5}}{10}
	    \, \left( \frac{3 - \sqrt{5}}{2} \right)^n
	  + \frac{5 - \sqrt{5}}{10}
	      \, \left( \frac{3 + \sqrt{5}}{2} \right)^n
  \end{equation}

  Una forma instructiva
  de obtener el número de subconjuntos de \(k\) elementos%
    \index{conjunto!subconjunto!numero@número}
  de un conjunto de \(n\) elementos
  es partir de la recurrencia
  (ver el teorema~\ref{theo:identidad-Pascal}):
  \begin{equation}
    \label{eq:recurrence-subset}
    \binom{n + 1}{k + 1}
      = \binom{n}{k + 1} + \binom{n}{k}
    \hspace{3em} \binom{0}{k} = [k = 0]
    \hspace{3em} \binom{n}{0} = 1
  \end{equation}
  Si definimos la función generatriz bivariada:%
    \index{generatriz!bivariada}
  \begin{equation}
    \label{eq:gf-subset}
    C(x, y)
      = \sum_{\substack{k \ge 0 \\ n \ge 0}}
	  \binom{n}{k} \, x^k y^n
  \end{equation}
  Por las propiedades
  de las funciones generatrices ordinarias resulta:
  \begin{equation}
    \label{eq:fe-subsets}
    \frac{C(x, y) - C(0, y) - C(x, 0) + C(0, 0)}{x y}
      = C(x, y) + \frac{C(x, y) - C(0, y)}{x}
  \end{equation}
  El lado izquierdo es partir de la suma~\eqref{eq:gf-subset},
  restar la fila con \(k = 0\)
  y la columna con \(n = 0\);
  pero al hacerlo hay que reponer el coeficiente con \(k = n = 0\)
  que se restó dos veces.
  Luego se divide por \(x y\) para ajustar los exponentes.%
    \index{inclusion y exclusion, principio de@inclusión y exclusión, principio de}
  Por las condiciones de contorno:
  \begin{equation}
    \label{eq:binomial-boundary-gf}
    C(0, 0)
      = 1
    \hspace{3em}
    C(x, 0)
      = \sum_{k \ge 0} \binom{0}{k} \, x^k
      = 1
    \hspace{3em}
    C(0, y)
      = \sum_{n \ge 0} \binom{n}{0} \, y^n
      = \frac{1}{1 - y}
  \end{equation}
  Resulta nuevamente
  la función generatriz~\eqref{eq:sum-binomial-double}.

  La presente discusión se inspira en Bender~%
    \cite{bender06:_found_combin_applic}.
  Consideremos árboles binarios ordenados completos,%
    \index{arbol binario@árbol binario}
  definidos mediante:
  \begin{enumerate}[label=(\roman*), ref=\roman*]
  \item
    Un vértice aislado es un árbol binario ordenado completo
    (esta es la raíz del árbol y su única hoja)
  \item
    Si \(T_1\) y \(T_2\) son árboles binarios ordenados completos,
    lo es la estructura que agrega un nuevo nodo como raíz
    y pone la raíz de \(T_1\) como descendiente izquierdo de la raíz
    y la raíz de \(T_2\) como su descendiente derecho.
  \end{enumerate}
  Nos interesa determinar
  el número de estas estructuras con \(n\) hojas,
  que llamaremos \(b_n\).
  Claramente \(b_0 = 0\) y \(b_1 = 1\).
  Para \(n > 1\),
  tendremos dos subárboles;
  si el izquierdo aporta \(k\) hojas
  el derecho aporta \(n - k\),
  y el número de árboles que podemos crear en esta situación,
  por la regla del producto es
  \(b_k b_{n - k}\).
  Pero debemos considerar todos los posibles valores de \(k\),
  la regla de la suma nos dice para \(n > 1\):
  \begin{equation}
    \label{eq:UFBRPT-recurrence}
    b_n
      = \sum_{1 \le k \le n - 1} b_k b_{n - k}
  \end{equation}
  Fácilmente podemos calcular los primeros valores:
  \begin{equation}
    \label{eq:UFBRPT-values}
    \langle 0, 1, 1, 2, 5, 14, 42, 132, \dotsc \rangle
  \end{equation}
  Si definimos
  la función generatriz ordinaria \(B(z)\) de los \(b_n\),
  aplicando nuestra receta queda para \(n > 1\):
  \begin{align}
    \sum_{n \ge 2} b_n z^n
      &= \sum_{n \ge 2} z^n \sum_{1 \le k \le n - 1} b_k b_{n - k}
				  \notag \\
    B(z) - b_0 - b_1 z
      &= \sum_{n \ge 2}
	   \sum_{1 \le k \le n - 1}
	     b_k z^k \cdot b_{n - k} z^{n - 1 - k}
				  \notag \\
      &= \left( \sum_{k \ge 1} b_k z^k \right)
	    \cdot \left( \sum_{k \ge 1} b_k z^k \right)
				  \notag \\
      &= \left( B(z) - b_0 \right)^2
				  \notag \\
    B(z) - z
      &= B^2(z)
				  \label{eq:UFBRPT-functional}
  \end{align}
  De~\eqref{eq:UFBRPT-functional} resulta:
  \begin{equation*}
    B(z)
      = \frac{1 \pm \sqrt{1 - 4 z}}{2}
  \end{equation*}
  Como debe ser \(b_0 = 0\),
  el signo correcto es el negativo:%
    \index{generatriz!ordinaria}%
    \index{arbol binario@árbol binario!generatriz}
  \begin{equation}
    \label{eq:eq:UFBRPT-gf}
    B(z)
      = \frac{1 - \sqrt{1 - 4 z}}{2}
  \end{equation}
  Expandiendo la raíz mediante el teorema del binomio:
  \begin{align}
    B(z)
      &= \frac{1}{2}
	   \left(
	     1 - \left(
		   1 + \sum_{n \ge 1}
			 \frac{(-1)^{n - 1}}{n 2^{2 n - 1}} \,
			   \binom{2 n - 2}{n - 1}
			     \cdot (-1)^n \cdot 2^{2 n} \cdot z^n
		 \right)
	   \right) \notag \\
      &= \sum_{n \ge 1}
	   \frac{1}{n} \, \binom{2 n - 2}{n - 1} \, z^n \notag \\
      &= \sum_{n \ge 1} C_{n - 1} z^n
	   \label{eq:UFBRPT-series-gf}
  \end{align}
  Nuevamente números de Catalan.%
    \index{Catalan, numeros de@Catalan, números de}
  De la suma~\eqref{eq:UFBRPT-series-gf}
  y la condición \(b_0 = 0\) tenemos:
  \begin{equation}
    \label{eq:UFBRPT-number}
    b_n
      = \begin{cases}
	  0	    & \text{si \(n = 0\)} \\
	  C_{n - 1} & \text{si \(n \ge 1\)}
	\end{cases}
  \end{equation}

  Si contamos el número de maneras de crear palabras de \(n\) letras
  usando únicamente \(\mathrm{A}\) y \(\mathrm{B}\),
  es claro que esto corresponde
  a elegir \(k\) posiciones para las \(\mathrm{A}\)
  (y dejar las \(n - k\) restantes a llenar por \(\mathrm{B}\)).
  Si hay \(a_k\) maneras de tener \(k\) letras \(\mathrm{A}\)
  y \(b_k\) maneras de tener \(k\) letras \(\mathrm{B}\),
  vemos que el total es:
  \begin{equation*}
    \sum_{0 \le k \le n} \binom{n}{k} a_k b_{n - k}
  \end{equation*}
  Una convolución binomial.
  Deberemos multiplicar las funciones generatrices exponenciales
  de las secuencias \(\langle a_n \rangle_{n \ge 0}\)
  y \(\langle b_n \rangle_{n \ge 0}\)
  para obtener la función generatriz exponencial
  del número de palabras posibles.
    \index{generatriz!exponencial}
  Por ejemplo,
  si la restricción es que el número de \(\mathrm{A}\) es par
  y no hay restricciones para las \(\mathrm{B}\),
  las funciones generatrices respectivas son:
  \begin{equation*}
    \widehat{A}(z)
      = 1 + \frac{z^2}{2!} + \frac{z^4}{4!} + \dotsb
      = \cosh z
    \hspace{3em}
    \widehat{B}(z)
      = 1 + \frac{z}{1!} + \frac{z^2}{2!} + \dotsb
      = \mathrm{e}^z
  \end{equation*}
  Resulta:
  \begin{equation*}
    n! \left[ z^n \right] \mathrm{e}^z \cosh z
      = n! \left[ z^n \right] \frac{\mathrm{e}^{2 z} + 1}{2}
      = \frac{n!}{2} \left( \frac{2^n}{n!} + 1 \right)
      = 2^{n - 1} + \frac{n!}{2}
  \end{equation*}

  Si volvemos a enfrentarnos al temible \(\mathrm{BOOKKEEPER}\),
  para calcular cuántas palabras de \(n\) letras podemos formar,
  consideramos el multiconjunto
    \(\{ \mathrm{B}^1, \mathrm{E}^3, \mathrm{K}^2, \mathrm{O}^2,
	 \mathrm{P}^1, \mathrm{R}^1 \}\),
  las funciones generatrices exponenciales
  de cada letra dan los factores:
  \begin{equation*}
    P(z)
      = \left( 1 + z \right)
	  \cdot \left(
		  1 + z + \frac{z^2}{2!} + \frac{z^3}{3!}
		\right)
	  \cdot \left( 1 + z + \frac{z^2}{2!} \right)
	  \cdot \left( 1 + z + \frac{z^2}{2!} \right)
	  \cdot \left( 1 + z \right)
	  \cdot \left( 1 + z \right)
  \end{equation*}
  y para obtener el número de palabras de \(n\) letras es
  \(n! \left[ z^n \right] P(z)\).
  La función generatriz se expande a:
  \begin{equation*}
    P(z)
      = 1 + 6 z + \frac{33}{2} z^2 + \frac{83}{3} z^3
	  + \frac{379}{12} z^4 + \frac{155}{6} z^5
	  + \frac{371}{24} z^6 + \frac{27}{4} z^7
	  + \frac{25}{12} z^8 + \frac{5}{12} z^9
	  + \frac{1}{24} z^{10}
  \end{equation*}
  O sea,
  las posibles palabras de 6 letras son:
  \begin{equation*}
    6! \, \left[ z^6 \right] P(z)
      = 720 \cdot \frac{371}{24}
      = 11\,130
  \end{equation*}
  Un momento de reflexión muestra que el coeficiente principal%
    \index{polinomio!coeficiente principal}
  (del término de máximo grado)
  en esta expansión es una explicación alternativa del Tao
  (sección~\ref{sec:tao-bookkeeper}).

\section{Aceite de serpiente}
\label{sec:snake-oil}
\index{generatriz!aceite de serpiente}

  La manera tradicional de simplificar sumatorias
  (particularmente las que involucran coeficientes binomiales)
  es aplicar identidades u otras manipulaciones de los índices,
  como magistralmente exponen Knuth~%
    \cite{knuth97:_fundam_algor}
  y Graham, Knuth y~Patashnik~%
    \cite{graham94:_concr_mathem}.
  Acá mostramos un método alternativo,
  que no requiere saber y aplicar
  una enorme variedad de identidades.
  Wilf~%
    \cite{wilf06:_gfology}
  le llama \emph{\foreignlanguage{english}{Snake Oil Method}},
  por la cura milagrosa que se ve en las películas del viejo oeste.
  La técnica es bastante simple:
  \begin{enumerate}
  \item
    Identificar la variable libre,
    llamémosle \(n\),
    de la que depende la suma.
    Sea \(f(n)\) nuestra suma.
  \item
    Sea \(F(z)\) la función generatriz ordinaria
    de la secuencia \(\langle f(n) \rangle_{n \ge 0}\).
  \item
    Multiplique la suma por \(z^n\) y sume sobre \(n\).
    Tenemos \(F(z)\) expresado como una doble suma,
    sobre \(n\) y la variable de la suma original.
  \item
    Intercambie el orden de las sumas,
    y exprese la suma interna en forma simple y cerrada.
  \item
    Encuentre los coeficientes,
    son los valores de \(f(n)\) buscados.
  \end{enumerate}
  Sorprende la alta tasa de éxitos de la técnica.
  Tiene la ventaja de que no requiere mayor creatividad;
  resulta claro cuándo funciona
  y es obvio cuando falla.

  Usaremos la convención que toda suma sin restricciones
  es sobre el rango \(-\infty\) a \(\infty\).
  Como los coeficientes binomiales \(\binom{n}{k}\)
  que usaremos en los ejemplos se anulan cuando
  \(k\) no está en el rango \([0, n]\),
  esto evita interminables ajustes de índices.
  Por ejemplo,
  para \(n \ge 0\) tenemos:
  \begin{equation*}
    \sum_k \binom{n}{r + k} \, z^k
      = z^{-r} \, \sum_k \binom{n}{r + k} \, z^{r + k}
      = z^{-r} \, \sum_s \binom{n}{s} \, z^s
      = z^{-r} (1 + z)^n
  \end{equation*}

  \begin{example}
    Evaluar:
    \begin{equation*}
      \sum_k \binom{k}{n - k}
    \end{equation*}

    La variable libre es \(n\),
    llamamos \(g(n)\) a nuestra suma
    y a su función generatriz \(G(z)\).
    Multiplicamos por \(z^n\) y sumamos:
    \begin{align*}
      G(z)
	&= \sum_n \sum_k \binom{k}{n - k} \, z^n
	 = \sum_k \sum_n \binom{k}{n - k} \, z^n
	 = \sum_k z^k \, \sum_n \binom{k}{n - k} \, z^{n - k}
	 = \sum_k z^k \, \sum_r \binom{k}{r} \, z^r \\
	&= \sum_k z^k (1 + z)^k
	 = \frac{1}{1 - z (1 + z)}
	 = \frac{1}{1 - z - z^2}
    \end{align*}
    Esto se parece sospechosamente
    a la función generatriz~\eqref{eq:gf-Fibonacci}
    de los números de Fibonacci%
      \index{Fibonacci, numeros de@Fibonacci, números de}
    que discutiremos en la sección~\ref{sec:Fibonacci},
    como \(F_0 = 0\) vemos que:
    \begin{equation*}
      G(z)
	= \frac{F(z) - F_0}{z}
    \end{equation*}
    Por las propiedades de las funciones generatrices ordinarias,
    es \(g(n) = F_{n + 1}\).
  \end{example}
  Nuestro siguiente problema viene de Riordan~%
    \cite{riordan68:_combin_ident},
  donde se resuelve mediante delicadas maniobras.
  Nuestro desarrollo sigue a Dobrushkin~%
    \cite{dobrushkin10:_method_algor_analysis}.
  \begin{example}
    Evaluar:
    \begin{equation*}
      h_n
	= \sum_{0 \le k \le n}
	    (-1)^{n - k} \, 4^k \, \binom{n + k + 1}{2 k + 1}
    \end{equation*}

    Definimos \(H(z)\) como la función generatriz de los \(h_n\);
    multiplicamos por \(z^n\),
    sumamos para \(n \ge 0\)
    e intercambiamos orden de suma:
    \begin{align*}
      H(z)
	&= \sum_{n \ge 0} z^n
	     \sum_{0 \le k \le n}
	       (-1)^{n - k} \, 4^k \, \binom{n + k + 1}{2 k + 1} \\
	&= \sum_{n \ge 0}
	     \sum_{0 \le k \le n}
	       (-4)^k \, (-z)^n \, \binom{n + k + 1}{2 k + 1} \\
	&= \sum_{k \ge 0}
	     (-4)^k \,
	     \sum_{n \ge k}
	       \binom{n + k + 1}{2 k + 1} \, (-z)^n
    \end{align*}
    Para completar el trabajo necesitamos la suma interna.
    Haciendo el cambio de variable \(r = n - k\):
    \begin{equation*}
      \sum_{n \ge k} \binom{n + k + 1}{2 k + 1} \, (-z)^n
	= (-z)^k \, \sum_{r \ge 0}
		      \binom{r + 2 k + 1}{2 k + 1} \, (-z)^r
	= \frac{(-z)^k}{(1 + z)^{2 k + 2}}
    \end{equation*}
    Substituyendo en lo anterior:
    \begin{equation*}
      H(z)
	= \sum_{k \ge 0} \frac{(4 z)^k}{(1 + z)^{2 k + 2}}
	= \frac{1}{(1 + z)^2}
	    \cdot \frac{1}{1 - \frac{4 z}{(1 + z)^2}}
	= \frac{1}{(1 - z)^2}
    \end{equation*}
    Resta extraer los coeficientes,
    lo que da:
    \begin{equation*}
      h_n
	= (-1)^n \, \binom{-2}{n}
	= \binom{n + 1}{1}
	= n + 1
    \end{equation*}
  \end{example}
  La siguiente es una sumatoria que le dio problemas a Knuth,%
    \index{Knuth, Donald E.}
  como comenta en el prefacio
  del texto de Petkovšek, Wilf y Zeilberger~%
    \cite{petkovsek96:_AeqB}.
  \begin{example}
    Considere la suma:
    \begin{equation*}
      \sum_k \binom{2 n - 2 k}{n - k} \, \binom{2 k}{k}
    \end{equation*}
    Sabemos que la secuencia
    comienza \(\langle 1, 4, 16, 64, \dotsc\rangle\),
    por lo que sospechamos que la suma vale \(4^n\).

    Aplicando la receta,
    con \(s(n)\) la suma que nos interesa
    y \(S(z)\) la respectiva función generatriz:
    \begin{equation*}
      S(z)
	= \sum_{n \ge 0}
	    \sum_{0 \le k \le n}
	      \binom{2 n - 2 k}{n - k} \, \binom{2 k}{k} \, z^n
	= \left( \sum_{n \ge 0} \binom{2 n}{n} \, z^n \right)^2
    \end{equation*}
    En este caso
    (como en todas las convoluciones)
    la sumatoria externa simplemente se disuelve sola.
    La serie interna es~\eqref{eq:serie-reciproco-raiz}:
    \begin{equation*}
      S(z)
	= \left( \frac{1}{\sqrt{1 - 4 z}} \right)^2
	= \frac{1}{1 - 4 z}
    \end{equation*}
    Una serie geométrica,
    y el resultado \(s(n) = 4^n\) es inmediato.
  \end{example}
  \begin{example}
    Determine el valor de:
    \begin{equation*}
      \sum_{0 \le k \le n} \binom{n}{k}^2
    \end{equation*}

    Esto es esencialmente una convolución:
    \begin{equation*}
      \sum_{0 \le k \le n} \binom{n}{k} \, \binom{n}{n - k}
    \end{equation*}
    Acá producen problemas los distintos usos de \(n\),
    delimita el rango de la suma
    y aparece en los índices superiores
    de los coeficientes binomiales.
    Una solución en tales casos
    es intentar demostrar algo más general.
    Dividiendo los distintos usos de \(n\) en variables separadas
    queda:
    \begin{equation*}
      \sum_{0 \le k \le r} \binom{m}{k} \, \binom{n}{r - k}
    \end{equation*}
    Ahora hay varios índices libres,
    debemos elegir uno.
    Es una convolución,
    lo que hace sospechar que \(r\) es útil como variable libre.
    Así llamamos \(v(r)\) a nuestra suma,
    y su función generatriz \(V(z)\).
    \begin{align*}
      V(z)
	&= \sum_{r \ge 0}
	     z^r \, \sum_k \binom{m}{k} \, \binom{n}{r - k} \\
	&= \left( \sum_{k \ge 0} \binom{m}{k} \, z^k \right)
	     \cdot \left(
		     \sum_{k \ge 0} \binom{n}{k} \, z^k
		   \right) \\
	&= (1 + z)^m \, (1 + z)^n \\
	&= (1 + z)^{m + n}
    \end{align*}
    En consecuencia,
    tenemos la \emph{convolución de Vandermonde}%
      \footnote{Otro caso de injusticia histórica:
		Unos 400~años antes de Vandermonde
		la conocía Zhu Shije en China~%
		   \cite[páginas 59--60]
			{askey75:_orthogonal_poly_special_functions}.}:
    \begin{equation}
      \label{eq:Vandermonde-convolution}
      \index{Vandermonde, convolucion de@Vandermonde, convolución de}
      \sum_k \binom{m}{k} \, \binom{n}{r - k}
	= \binom{m + n}{r}
    \end{equation}
    que también puede escribirse en la forma simétrica:
    \begin{equation}
      \label{eq:Vandermonde-convolution-symmetric}
      \sum_k \binom{m}{r + k} \binom{n}{s - k}
	= \binom{m + n}{r + s}
    \end{equation}
    Acá la suma es sobre todo \(k \in \mathbb{Z}\),
    pero sólo para \(-r \le k \le s\) los términos no son cero.
    Indicarlo destruiría la simetría de la fórmula.
    Nótese además que nuestra demostración es aplicable
    también en caso que \(m\) o \(n\) no sean naturales.

    Nuestra suma original
    es simplemente el caso especial \(m = n = r\)
    de~\eqref{eq:Vandermonde-convolution}:
    \begin{equation}
      \label{eq:sum-square-binom}
      \sum_k \binom{n}{k}^2
	= \sum_k \binom{n}{k} \, \binom{n}{n - k}
	= \binom{2 n}{n}
    \end{equation}
    Nuevamente un caso de la paradoja del inventor.%
      \index{paradoja del inventor}
  \end{example}
  Un ejemplo propuesto por Liu~\cite{liu68:_introd_combin_mathem},
  que resuelve de forma afín a la nuestra:
  \begin{example}
    Calcular la suma:
    \begin{equation}
      \label{eq:so:binomial-convolution}
      S_r
	= \sum_{0 \le i \le r} \frac{r!}{(r - i + 1)! (i + 1)!}
    \end{equation}
    Vemos que los términos son sospechosamente similares
    a coeficientes binomiales:
    \begin{equation*}
      S_r
	= \sum_{0 \le i \le r}
	    \binom{r}{i} \frac{1}{r - i + 1} \frac{1}{i + 1}
    \end{equation*}
    Esta es una convolución binomial,%
      \index{convolucion binomial@convolución binomial}
    lo que sugiere la función generatriz exponencial:%
      \index{generatriz!exponencial}
    \begin{equation}
      \label{eq:so:binomial-convolution:egf}
      \widehat{S}(z)
	= \sum_{r \ge 0} S_r \frac{z^r}{r!}
    \end{equation}
    Vemos que:
    \begin{align*}
      \widehat{S}(z)
	&= \left(
	     \sum_{r \ge 0} \frac{1}{r + 1} \cdot \frac{z^r}{r!}
	   \right)^2 \\
	&= \left(
	     \sum_{r \ge 0} \frac{z^r}{(r + 1)!}
	   \right)^2 \\
	&= \left(
	     \frac{\mathrm{e}^z - 1}{z}
	   \right)^2 \\
	&= \frac{\mathrm{e}^{2 z} - 2 \mathrm{e}^z + 1}{z^2}
    \end{align*}
    Los coeficientes son inmediatos:
    \begin{align}
      S_r
	&= r! [z^r] \widehat{S}(z) \notag \\
	&= r! [z^r]
	   \frac{\mathrm{e}^{2 z} - 2 \mathrm{e}^z + 1}
		{z^2} \notag \\
	&= r! [z^{r + 2}]
	     \left(\mathrm{e}^{2 z} - 2 \mathrm{e}^z + 1 \right)
		 \notag \\
	&= r! \left(
		\frac{2^{r + 2}}{(r + 2)!}
		  - \frac{2}{(r + 2)!}
	      \right) \notag \\
	&= \frac{2^{r + 2} - 2}{(r + 1) (r + 2)}
	     \label{eq:so:binomial-convolution:coef}
    \end{align}
  \end{example}
  Finalmente,
  una identidad.
  \begin{example}
    Demostrar que para \(m, n \ge 0\)
    \begin{equation}
      \label{eq:so:binomial-identity}
      \sum_{k \ge 0} \binom{m}{k} \binom{n + k}{m}
	= \sum_{k \ge 0} \binom{m}{k} \binom{n}{k} \, 2^k
    \end{equation}
    Multiplicamos
    ambos lados de~\eqref{eq:so:binomial-identity} por \(z^n\)
    y sumamos,
    obteniendo la identidad \(L(z) = R(z)\)
    al igualar lado izquierdo con derecho:
    \begin{equation*}
      L(z)
	= \sum_{n \ge 0} z^n
	    \sum_{k \ge 0} \binom{m}{k} \binom{n + k}{m}
      \hspace{3em}
      R(z)
	= \sum_{n \ge 0} z^n
	    \sum_{k \ge 0} \binom{m}{k} \binom{n}{k} \, 2^k
    \end{equation*}
    Partimos por el lado izquierdo:
    \begin{align*}
      L(z)
	&= \sum_{k \ge 0} \binom{m}{k} z^{-k}
	     \sum_{n \ge 0} \binom{n + k}{m} z^{n + k} \\
    \intertext{Por la suma externa sabemos que \(0 \le k \le m\),
	       como en realidad la suma interna
	       es para \(n + k \ge m\)
	       podemos aplicar~\eqref{eq:serie-binomio-n}:}
      L(z)
	&= \sum_{k \ge 0} \binom{m}{k} z^{-k} \,
	     \frac{z^m}{(1 - z)^{m + 1}} \\
	&= \left( 1 + \frac{1}{z} \right)^m \,
	     \frac{z^m}{(1 - z)^{m + 1}} \\
	&= \frac{(1 + z)^m}{(1 - z)^{m + 1}}
    \end{align*}
    El lado derecho recibe un tratamiento similar:
    \begin{align*}
      R(z)
	&= \sum_{k \ge 0} \binom{m}{k} \, 2^k \,
	     \sum_{n \ge 0} \binom{n}{k} z^n \\
	&= \sum_{k \ge 0}
	     \binom{m}{k} \, 2^k \,\frac{z^k}{(1 - z)^{k + 1}} \\
	&= \frac{1}{1 - z} \,
	     \sum_{k \ge 0} \binom{m}{k}
	       \left( \frac{2 z}{1 - z} \right)^k \\
	&= \frac{1}{1 - z}
	     \, \left( 1 + \frac{2 z}{1 - z} \right)^m \\
	&= \frac{(1 + z)^m}{(1 - z)^{m + 1}}
    \end{align*}
    Se verifica la identidad.
  \end{example}

  Hay métodos complementarios,
  capaces de resolver automáticamente grandes clases de sumatorias,
  o demostrar que no hay expresiones simples para ellas.
  Petkovšek, Wilf y Zeilberger~%
    \cite{petkovsek96:_AeqB}
  los describen en detalle,
  y hay implementaciones de los mismos
  para los principales paquetes de álgebra simbólica.
  Cipra~\cite{cipra89:_grinch_stole_math}
  incluso se queja que estas demostraciones automatizadas
  quitan la entretención a las matemáticas.

%%% Local Variables:
%%% mode: latex
%%% TeX-master: "clases"
%%% End:

% coloreo-Polya.tex
%
% Copyright (c) 2011-2014 Horst H. von Brand
% Derechos reservados. Vea COPYRIGHT para detalles

\chapter{Teoría de coloreos de Pólya}
\label{cha:coloreo-Pólya}
\index{Polya, teoria de enumeracion de@Pólya, teoría de enumeración de}

  Los problemas combinatorios que hemos enfrentado hasta acá
  han sido relativamente sencillos.
  Sólo al considerar la construcción ciclo de objetos no rotulados
  (capítulo~\ref{cha:metodo-simbolico})%
    \index{metodo simbolico@método simbólico}
  tuvimos que considerar simetrías de los objetos bajo estudio.
  El teorema de enumeración de Pólya
  (abreviado \emph{PET},
   por \emph{\foreignlanguage{english}{Pólya Enumeration Theorem}})
  en realidad fue publicado en 1927 por John Howard Redfield,
  en 1937 George Pólya lo redescubrió independientemente
  y lo popularizó aplicándolo a muchos problemas de conteo,
  en particular de compuestos químicos~%
    \cite{polya37:_kombin_anzah_grupp_graph_verbin,
	  polya87:_combin_enumer_group_graph_chemic_compoun}.

\section{Grupos de permutaciones}
\label{sec:grupos-permutaciones}

  Desarrollaremos la teoría para la situación simple
  en la cual solo contamos coloreos.
  Una derivación más intuitiva
  (que puede ser útil para motivar el desarrollo presente)
  ofrece Tucker~\cite{tucker74:_Polya_enum_example}.
  \begin{definition}
    \index{grupo!permutaciones}
    Sea \(G\) un conjunto de permutaciones
    del conjunto finito \(\mathcal{X}\).
    Si \(G\) es un grupo
    (con la composición de permutaciones)
    decimos que
      \emph{\(G\) es un grupo de permutaciones de \(\mathcal{X}\)}.
  \end{definition}
  Si tomamos \(\mathcal{X} = \{1, 2, \dotsc, n\}\),
  entonces un grupo de permutaciones
  es simplemente un subgrupo de \(\mathtt{S}_n\).%
    \index{grupo!simetrico@simétrico}%
    \index{grupo!subgrupo}
  El cuadro~\ref{tab:subgrupos-S3}
  lista todos los subgrupos de \(\mathtt{S}_3\).
  \begin{table}[htbp]
    \centering
    \begin{tabular}{*{3}{>{\(}l<{\)}}}
      H_1 = \{\iota\}			       &
	 H_2 = \{\iota, (1\;2)\}	       &
	 H_3 = \{\iota, (1\;3)\}		   \\
      H_4 = \{\iota, (2\;3)\}		       &
	 H_5 = \{\iota, (1\;2\;3), (1\;3\;2)\} &
	 H_6 = \mathtt{S}_3
    \end{tabular}
    \caption{Los subgrupos de $\mathtt{S}_3$}
    \label{tab:subgrupos-S3}
  \end{table}
  El grupo alternante \(\mathtt{A}_3\)%
    \index{grupo!alternante}
  aparece como \(H_5\) en el cuadro.
  Para ver si un subconjunto de un grupo finito es un subgrupo,
  basta verificar si es cerrado
  por lo demostrado anteriormente para grupos.

  Otros ejemplos se obtienen
  como grupos de simetría de objetos geométricos.
  Por ejemplo,
  si rotulamos los vértices de un cuadrado
  en orden contrario a las manecillas del reloj
  (ver la figura~\ref{fig:cuadrado}),
  \begin{figure}[htbp]
    \centering
    \pgfimage{images/cuadrado}
    \\[3ex]
    \begin{tabular}[c]{l>{\(}l<{\)}}
      \hline
	\rule[-0.7ex]{0pt}{3ex}%
      Identidad					    & (1) (2) (3) (4) \\
      Rotación en \(\pi / 2\)			    & (1\;2\;3\;4)    \\
      Rotación en \(\pi\)			    & (1\;3) (2\;4)   \\
      Rotación en \(3 \pi / 2\)			    & (1\;4\;3\;2)    \\
      Reflexión en diagonal 1\;3		    & (2\;4)	      \\
      Reflexión en diagonal 2\;4		    & (1\;3)	      \\
      Reflexión en bisector perpendicular de 1\;2   & (1\;2) (3\;4)   \\
      Reflexión en bisector perpendicular de 1\;4   & (1\;4) (2\;3)   \\
      \hline
    \end{tabular}
    \caption{Un cuadrado y sus simetrías}
    \label{fig:cuadrado}
  \end{figure}
  cada simetría
  induce una permutación del conjunto \(\{1, 2, 3, 4\}\),
  y obtenemos las simetrías indicadas.
  Estas \(8\)~permutaciones
  forman el llamado \emph{grupo dihedral de orden \(8\)},%
    \index{grupo!dihedral}
  \(\mathtt{D}_8\).
  En general,
  las simetrías de un \(n\)\nobreakdash-ágono regular%
    \index{poligono@polígono}
  forman un grupo de \(2 n\)~elementos,
  el \emph{grupo dihedral de orden \(2 n\)}
  que se anota~\(\mathtt{D}_{2 n}\).
  Debe tenerse cuidado:
  Esta es la notación que se usa en álgebra,
  en geometría este mismo grupo se anota~\(\mathtt{D}_n\).
  Si solo se consideran las rotaciones
  de un polígono regular de \(n\) lados,
  tenemos el grupo cíclico de orden \(n\),%
    \index{grupo!ciclico@cíclico}
  anotado \(\mathtt{C}_n\),
  isomorfo a \(\mathbb{Z}_n\) con la suma.

  Una situación similar se produce cuando estudiamos grafos
  en vez de figuras geométricas,
  en donde las ``simetrías'' son permutaciones de los vértices
  que transforman arcos en arcos.
  A una permutación de este tipo
  se le llama \emph{automorfismo} del grafo%
    \index{grafo!automorfismo}
  (viene a ser un isomorfismo del grafo consigo mismo).%
    \index{grafo!isomorfismo}

  Un ejemplo de grafo es la figura~\ref{fig:automorfismos},
  interesa saber cuántos automorfismos tiene.
  \begin{figure}[htbp]
    \centering
    \pgfimage{images/automorfismo}
    \\[2ex]
    \begin{tabular}[c]{>{\(}l<{\)}@{ se extiende a }>{\(}l<{\)}@{\qquad}
		       >{\(}l<{\)}@{ se extiende a }>{\(}l<{\)}}
       \iota	 & \iota	       &
	  (1\;3) & (1\;3) (4\;6) \\
       (1\;3\;5) & (1\;3\;5) (2\;4\;6) &
	  (1\;5) & (1\;5) (2\;4) \\
       (1\;5\;3) & (1\;5\;3) (2\;6\;4) &
	  (3\;5) & (3\;5) (2\;6)
    \end{tabular}
    \caption{Un grafo de seis vértices y sus automorfismos}
    \label{fig:automorfismos}
  \end{figure}
  Primero observamos que los vértices del grafo
  caen naturalmente en dos grupos:
  Los vértices \(\{1, 3, 5\}\) son de grado \(4\),
  mientras \(\{2, 4, 6\}\) son de grado \(2\).
  Ningún automorfismo puede transformar
  un vértice del primer grupo en uno del segundo.
  Por otro lado,
  está claro
  que podemos tomar \emph{cualquier} permutación de \(\{1, 2, 3\}\)
  y extenderla a un automorfismo del grafo.
  Por ejemplo,
  si \((1\;3\;5)\) es parte de un automorfismo \(\alpha\),
  entonces \(\alpha\) tiene que transformar \(2\) en \(4\),
  ya que \(2\) es el único vértice adyacente a \(1\) y \(3\),
  y \(4\) es el único vértice adyacente
  a sus imágenes \(3\) y \(5\).
  De la misma forma,
  \(\alpha\) lleva \(4\) en \(6\) y \(6\) en \(2\),
  por lo que \(\alpha = (1\;3\;5) (2\;4\;6)\).
  En forma análoga,
  cada una de las seis permutaciones de \(\{1, 2, 3\}\)
  puede extenderse de forma única a un automorfismo del grafo,
  como muestra la misma figura~\ref{fig:automorfismos}.
  Hay exactamente seis automorfismos,
  que son las permutaciones listadas arriba.

\section{Órbitas y estabilizadores}
\label{sec:orbitas-estabilizadores}
\index{grupo!permutaciones!orbita@órbita}
\index{grupo!permutaciones!estabilizador}

  Sea \(G\) un grupo de permutaciones
  de un conjunto \(\mathcal{X}\).
  Veremos que la estructura del grupo
  lleva naturalmente a una partición de \(\mathcal{X}\).
  Definamos la relación \(\sim\) sobre \(\mathcal{X}\) mediante
  \(x \sim y\) siempre que para algún \(\gamma \in G\)
  tenemos \(\gamma(x) = y\).
  Verificamos que \(\sim\)
  es una relación de equivalencia de la forma usual:%
    \index{relacion@relación!equivalencia}
  \begin{description}
  \item[Reflexiva:]
    Como \(\iota\) es parte de todo grupo,
    y \(\iota(x) = x\) para todo \(x \in \mathcal{X}\),
    tenemos \(x \sim x\).
  \item[Simétrica:]
    Supongamos \(x \sim y\),
    o sea \(\gamma(x) = y\) para algún \(\gamma \in G\).
    Como \(G\) es un grupo,
    \(\gamma^{-1} \in G\),
    y como \(\gamma^{-1}(y) = x\),
    tenemos \(y \sim x\).
  \item[Transitiva:]
    Si \(x \sim y\) y \(y \sim z\)
    debe ser \(\gamma_1(x) = y\) y \(\gamma_2(y) = z\)
    para \(\gamma_1, \gamma_2 \in G\),
    y como \(G\) es un grupo,
    \(\gamma_2 \gamma_1 \in G\),
    con lo que \(\gamma_2 \gamma_1 (x) = z\)
    y \(x \sim z\).
  \end{description}
  Como \(\sim\) es relación de equivalencia,
  define una partición de \(\mathcal{X}\);
  \(x\) e \(y\) pertenecen a la misma clase
  si y solo si hay una permutación en \(G\)
  que transforma \(x\) en \(y\).
  A las clases de equivalencia
  se les conoce como las \emph{órbitas} de \(G\) en \(\mathcal{X}\).
  La \emph{órbita de \(x\)} es la clase que contiene a \(x\):
  \begin{equation*}
    G x
      = \{y \in \mathcal{X} \colon y = \gamma(x)
	       \text{\ para algún\ } \gamma \in G\}
  \end{equation*}

  Intuitivamente,
  la órbita \(G x\) son los elementos de \(\mathcal{X}\)
  que no se distinguen de \(x\) bajo operaciones de \(G\).
  En el caso de la figura~\ref{fig:automorfismos}
  los conjuntos de vértices \(\{1, 3, 5\}\) y \(\{2, 4, 6\}\)
  son órbitas del grupo.
  El grafo de la figura~\ref{fig:ejemplo-orbitas}
  tiene un grupo más complejo.
  \begin{figure}[htbp]
    \centering
    \pgfimage{images/orbitas}
    \vspace{2\baselineskip}

    \begin{tabular}[c]{>{\(}l<{\)}*{2}{@{\quad}>{\(}l<{\)}}}
      \iota  & \iota  & \iota		    \\
      (a\;b) & (d\;f) & (h\;i) (k\;l)	    \\
	     &	      & (h\;j) (k\;m)	    \\
	     &	      & (i\;j) (l\;m)	    \\
	     &	      & (h\;i\;j) (k\;l\;m) \\
	     &	      & (h\;j\;i) (k\;m\;l)
    \end{tabular}
    \caption{Un ejemplo de grafo
	     y los generadores de su grupo de automorfismos}
    \label{fig:ejemplo-orbitas}
  \end{figure}
  Acá los automorfismos
  se obtienen combinando las permutaciones
  de la figura~\ref{fig:ejemplo-orbitas}.
  Hay un total de \(2 \cdot 2 \cdot 6 = 24\) permutaciones
  en este grupo.
  Son órbitas
  \(\{a, b\}\), \(\{c\}\), \(\{d, f\}\), \(\{e\}\), \(\{g\}\),
  \(\{h, i, j\}\), \(\{k, l, m\}\),
  y se ve que ``son parecidos'' los elementos de cada una de ellas,
  en que las operaciones del grupo los intercambian.

  Las órbitas presentan un par de problemas numéricos obvios:
  ¿Cuántas órbitas hay?
  ¿Qué tamaños tienen?

  Si \(G\) es un grupo de permutaciones,
  llamaremos \(G(x \rightarrow y)\)
  al conjunto de permutaciones que llevan \(x\) a \(y\),
  o sea:
  \begin{equation*}
    G(x \rightarrow y) = \{\gamma \in G \colon g(x) = y\}
  \end{equation*}
  En particular,
  \(G(x \rightarrow x)\)
  es el conjunto de permutaciones
  que tienen a \(x\) como punto fijo.
  Este conjunto se llama el \emph{estabilizador} de \(x\),%
    \index{grupo!permutaciones!estabilizador}
  y se anota \(G_x\).
  Si \(\gamma_1\) y \(\gamma_2\) están en \(G_x\):
  \begin{equation*}
    \gamma_2 \gamma_1(x) = \gamma_2(x) = x
  \end{equation*}
  por lo que \(\gamma_2 \gamma_1 \in G_x\),
  y \(G_x\) es un subgrupo de \(G\).
  También tenemos:
  \begin{theorem}
    \label{theo:stabilizer-coset-l}
    Sea \(G\) un grupo de permutaciones,
    y sea \(\gamma \in G(x \rightarrow y)\).
    Entonces:
    \begin{equation*}
      G(x \rightarrow y) = \gamma G_x
    \end{equation*}
    el coset izquierdo de \(G_x\) respecto a \(\gamma\).
  \end{theorem}
  \begin{proof}
    Demostraremos que todo elemento de \(\gamma G_x\)
    pertenece a \(G(x \rightarrow y)\) y viceversa,
    con lo que ambos conjuntos son iguales.

    Si \(\alpha\) pertenece a \(\gamma G_x\),
    es \(\alpha = \gamma \beta\) para algún \(\beta \in G_x\).
    O sea,
    \(\alpha(x) = \gamma \beta(x) = \gamma(x) = y\),
    con lo que \(\alpha\) pertenece a \(G(x \rightarrow y)\).
    Por el otro lado,
    si \(\pi \in G(x \rightarrow y)\),
    entonces \(\gamma^{-1} \pi(x) = \gamma^{-1}(y) = x\),
    de manera que \(\gamma^{-1} \pi = \delta\),
    donde \(\delta \in G_x\),
    y así \(\pi = \gamma \delta \in \gamma G_x\).
    Ambos conjuntos son iguales.
  \end{proof}
  De forma muy similar al teorema~\ref{theo:stabilizer-coset-l}
  se demuestra lo siguiente:
  \begin{theorem}
    \label{theo:stabilizer-coset-r}
    Sea \(G\) un grupo de permutaciones de \(\mathcal{X}\),
    y sea \(\gamma \in G(x \rightarrow y)\).
    Entonces:
    \begin{equation*}
      G(x \rightarrow y) = G_y \gamma
    \end{equation*}
    el coset derecho de \(G_y\) respecto a \(\gamma\).
  \end{theorem}
  \begin{proof}
    Si \(\alpha\) pertenece a \(G_y \gamma\),
    es \(\alpha = \beta \gamma\) para algún \(\beta \in G_y\),
    vale decir
    \(
      \alpha(y)
	= \beta \gamma(y)
	= \beta(x)
	= y
    \)
    o sea \(\beta \in G(x \rightarrow y)\).
    Al revés,
    supongamos \(\pi \in G(x \rightarrow y)\),
    y consideremos
    \(
      \pi \gamma^{-1}(y)
	= \pi(x)
	= y
    \)
    y por tanto \(\pi \gamma^{-1} \in G_y\),
    o \(\pi \in G_y \gamma\),
    y sigue el resultado.
  \end{proof}

  De los anteriores teoremas obtenemos:
  \begin{corollary}
    \label{cor:G_x=G_y}
    Sea \(G\) un grupo de permutaciones de \(\mathcal{X}\),
    sea \(x \in \mathcal{X}\) un elemento cualquiera,
    e \(y\) un elemento en la órbita de \(x\).
    Entonces \(\lvert G_x \rvert = \lvert G_y \rvert\).
  \end{corollary}
  \begin{proof}
    Inmediato,
    ya que el tamaño de un coset es el orden del subgrupo
    (lo demostramos para el teorema de Lagrange);%
      \index{Lagrange, teorema de}
% Wreckme: Referencia al teorema cuando se unan ambos apuntes
    y por los teoremas~\ref{theo:stabilizer-coset-l}
    y~\ref{theo:stabilizer-coset-r} tenemos
    \(
      \lvert G_y \rvert
	= \lvert G(x \rightarrow y) \rvert
	= \lvert G_x \rvert
    \).
  \end{proof}

  \begin{theorem}
    \label{theo:GxG_x=G}
    Sea \(G\) un grupo de permutaciones de \(\mathcal{X}\),
    y sea \(x\) un elemento de \(\mathcal{X}\).
    Entonces:
    \begin{equation*}
      \lvert G x \rvert \cdot \lvert G_x \rvert
	= \lvert G \rvert
    \end{equation*}
  \end{theorem}
  \begin{proof}
    Usamos la idea de contar por filas y columnas.
    Para un elemento \(x \in \mathcal{X}\)
    el conjunto de pares
      \(S_x = \{(\gamma, y) \colon \gamma(x) = y\}\)
    puede describirse
    mediante una tabla como la del cuadro~\ref{tab:gamma-y}.
    \begin{table}[htbp]
      \centering
      \begin{tabular}[c]{>{\(}l<{\)}|>{\(}r<{\)}>{\(}c<{\)}
			 >{\(}l<{\)}|>{\(}l<{\)}}
		   & \cdots & y & \cdots		 &
			\\
	\hline
	\vdots	   &	    &	&			 &
			\\
	\gamma & \phantom{\text{si \(\gamma(x) = y\)}} &
	  \multicolumn{2}{l|}{\;\;\;\checkmark\
			      si \(\gamma(x) = y\)
			      \hspace*{1.5em}} &
	  r_\gamma(S_x) \\
	\vdots	   &	    &	&			 &
			\\
	\hline
		   &	     & c_y(S_x) &		 &
      \end{tabular}
      \caption{Pares $(\gamma, y)$
	       para demostración del teorema~\ref{theo:GxG_x=G}}
      \label{tab:gamma-y}
    \end{table}
    Dado que \(\gamma\) es una permutación,
    hay un único \(y\)
    tal que \(\gamma(x) = y\) para cada \(\gamma\),
    con lo que \(r_\gamma(S_x) = 1\).
    El total por columna \(c_y(S_x)\)
    es el número de permutaciones \(\gamma\)
    tales que \(\gamma(x) = y\),
    vale decir
    \(\lvert G(x \rightarrow y) \rvert\).
    Si \(y\) está en la órbita \(G x\),
    por el teorema~\ref{theo:stabilizer-coset-l}
    y el hecho que el coset de un subgrupo%
      \index{coset}
    tiene el tamaño del subgrupo,
    tenemos
    \(\lvert G(x \rightarrow y) \rvert = \lvert G_x \rvert\).
    Por otro lado,
    si \(y\) no está en la órbita \(G x\),
    \(\lvert G(x \rightarrow y) \rvert = 0\).
    Las dos formas de contar los elementos de \(S_x\) dan:
    \begin{equation*}
      \sum_{y \in \mathcal{X}} c_y (S_x)
	= \sum_{\gamma \in G} r_\gamma(S_x)
    \end{equation*}
    Al lado izquierdo hay \(\lvert G x \rvert\) términos
    que valen \(\lvert G_x \rvert\) cada uno,
    los demás valen \(0\);
    al lado derecho hay \(\lvert G \rvert\) términos
    que valen \(1\) cada uno.
    Así tenemos el resultado prometido.
  \end{proof}

  Este teorema permite calcular el tamaño de un grupo
  si se conoce el tamaño de una órbita
  y el estabilizador respectivo.
  Consideremos por ejemplo el grupo \(\mathtt{T}\)
  de rotaciones en el espacio de un tetraedro,%
    \index{poliedro!regular}%
    \index{tetraedro}
  ver la figura~\ref{fig:rotaciones-tetraedro-1}.
  \begin{figure}[htbp]
    \centering
    \pgfimage{images/tetraedro-vertice}
    \caption{Rotaciones de un tetraedro}
    \label{fig:rotaciones-tetraedro-1}
  \end{figure}
  Las rotaciones alrededor del eje marcado
  son las que mantienen fijo el vértice \(1\),
  y hay \(3\) de estas,
  \(\lvert \mathtt{T}_d \rvert = 3\).
  Por otro lado,
  girando el tetraedro en el espacio
  se puede colocar en la posición \(1\)
  cualquiera de los \(4\) vértices,
  y tenemos \(\lvert \mathtt{T} d \rvert = 4\).
  En consecuencia,
  el grupo de rotaciones en el espacio de un tetraedro es
  de orden
  \(\lvert \mathtt{T} \rvert
      = \lvert \mathtt{T}_d \rvert
	  \cdot \lvert \mathtt{T} d \rvert
      = 3 \cdot 4
      = 12\).
  Resulta que \(\mathtt{T}\)
  no es más que el grupo alternante \(\mathtt{A}_4\).%
    \index{grupo!alternante}

  Otro ejemplo lo da el icosaedro trunco,%
    \index{poliedro}%
    \index{icosaedro trunco}
  la forma básica de la pelota de fútbol tradicional,
  ver la figura~\ref{fig:icosaedro-trunco}.
  \begin{figure}[htbp]
    \centering
   \pgfimage[width=0.35\textwidth]
	     {images/20070219213318!Truncated_icosahedron}
    % http://upload.wikimedia.org/wikipedia/commons/archive/c/c3/20070219213318%21Truncated_icosahedron.png
    % Public domain
    \caption{Icosaedro trunco}
    \label{fig:icosaedro-trunco}
  \end{figure}
  Este es el sólido arquimedeano%
    \index{solido arquimedeano@sólido arquimedeano|see{poliedro!arquimedeano}}%
    \index{poliedro!arquimedeano}
  limitado por \(12\)~hexágonos y \(20\)~pentágonos%
    \index{poligono@polígono}
  (un total de \(32\)~caras),
  \(90\)~aristas y \(60\)~vértices.
  Si consideramos rotaciones en el espacio de este sólido,
  como en cada vértice confluyen un pentágono y dos hexágonos
  la única simetría
  que mantiene fijo un vértice es \(\iota\).
  Vía rotaciones podemos hacer coincidir ese vértice con cualquiera,
  por lo que tenemos
  que \(\lvert G \rvert
	  = \lvert G_x \rvert \cdot \lvert G x \rvert
	  = 1 \cdot 60
	  = 60\).
  Obtener el orden de este grupo manipulando el sólido
  sería mucho más complicado.

\section{Número de órbitas}
\label{sec:numero-orbitas}

  Vamos ahora a contar el número de órbitas de un grupo \(G\)
  de permutaciones de \(\mathcal{X}\).
  Cada órbita es un subconjunto de elementos indistinguibles
  bajo las operaciones de \(G\),
  y el número de órbitas
  dice cuántos tipos de elementos distinguibles hay.

  Supóngase que se quieren fabricar tarjetas de identidad cuadradas,
  divididas en nueve cuadrados de los cuales se perforan dos.
  Véase la figura~\ref{fig:tarjetas} para algunos ejemplos.
  \begin{figure}[htbp]
    \centering
    \subfloat[]{
      \pgfimage{images/badge-1}
      \label{subfig:badge-1}
    }%
    \hspace{2em}%
    \subfloat[]{
      \pgfimage{images/badge-2}
      \label{subfig:badge-2}
    }%
    \hspace{2em}%
    \subfloat[]{
      \pgfimage{images/badge-3}
      \label{subfig:badge-3}
    }
    \caption{Ejemplos de tarjetas de identidad}
    \label{fig:tarjetas}
  \end{figure}
  Las primeras dos no se pueden distinguir,
  ya que se obtiene la de la figura~\ref{subfig:badge-2}
  rotando la de~\ref{subfig:badge-1};
  en cambio,
  la de~\ref{subfig:badge-3} claramente es diferente de las otras,
  independiente de si se gira o se da vuelta.

  El grupo que está actuando acá
  es el grupo \(\mathtt{D}_8\) de ocho simetrías de un cuadrado,%
    \index{grupo!dihedral}
  pero interesa el efecto que tiene
  sobre las \(\binom{9}{2} = 36\) configuraciones de dos agujeros
  en un cuadrado de \(3 \times 3\),
  no solo su acción sobre los cuatro vértices.
  El número de órbitas es el número de tarjetas distinguibles.
  Hacer esto por la vía de dibujar las \(36\) configuraciones,
  y analizar lo que ocurre
  con cada una de ellas con las \(8\) simetrías
  es bastante trabajo.
  Por suerte hay maneras mejores.

  Dado un elemento \(\gamma\) del grupo de permutaciones \(G\)
  definimos:
  \begin{equation*}
    F(\gamma) = \{x \in \mathcal{X} \colon \gamma(x) = x\}
  \end{equation*}
  Vale decir,
  \(F(\gamma)\) es el número de puntos fijos de \(\gamma\).
  Nuestro teorema siguiente relaciona esto con el número de órbitas.
  Este resultado se conoce bajo el nombre de Burnside,
  de Cauchy-Frobenius y de Pólya.
  Burnside lo popularizó en su libro~%
    \cite{burnside97:_theor_group_finit_order},
  atribuyéndolo a Frobenius,
  aunque el resultado lo conocía Cauchy antes.
  Por esta enredada historia
  a veces se le llama ``el lema que no es de Burnside''.
  \begin{theorem}[Lema de Burnside]
    \label{theo:Burnside}
    \index{Burnside, lema de}
    El número de órbitas de \(G\)
    sobre \(\mathcal{X}\) está dado por:
    \begin{equation*}
      \frac{1}{\lvert G \rvert} \,
	\sum_{\gamma \in G} \lvert F(\gamma) \rvert
    \end{equation*}
  \end{theorem}
  \begin{proof}
    Nuevamente,
    contar por filas y columnas.
    Sea:
    \begin{equation*}
      E = \{(\gamma, x) \colon \gamma(x) = x\}
    \end{equation*}
    Entonces el total por fila \(r_\gamma(E)\)
    es el número de \(x\) fijados por \(\gamma\),
    o sea \(\lvert F(\gamma) \rvert\).
    El total por columna \(c_x(E)\)
    es el número de \(\gamma\) que tienen \(x\) como punto fijo,
    \(\lvert G_x \rvert\).
    Contabilizando \(E\) de ambas formas da:
    \begin{equation*}
      \sum_{\gamma \in G} \lvert F(\gamma) \rvert
	= \sum_{x \in \mathcal{X}} \lvert G_x \rvert
    \end{equation*}
    Supongamos que hay \(t\) órbitas,
    y elijamos \(z \in \mathcal{X}\).
    Por el teorema~\ref{theo:stabilizer-coset-r},
    si \(x\) pertenece a la órbita \(G z\)
    entonces \(\lvert G_x \rvert = \lvert G_z \rvert\).
    Cada órbita contribuye
    al lado derecho \(\lvert G z \rvert\) términos,
    todos \(\lvert G_z \rvert\);
    la contribución total de la órbita
    es \(\lvert G z \rvert \cdot \lvert G_z \rvert
	   = \lvert G \rvert\)
    por el teorema~\ref{theo:GxG_x=G},
    lo que lleva a:
    \begin{equation*}
      \sum_{\gamma \in G} \lvert F(\gamma) \rvert
	= t \cdot \lvert G \rvert
    \end{equation*}
    que es equivalente a lo que queríamos demostrar.
  \end{proof}

  Ahora podemos resolver nuestro problema de tarjetas de identidad.
  Necesitamos calcular el número de configuraciones fijas
  bajo cada una de las ocho permutaciones.
  Por ejemplo,
  cuando \(\gamma\) es la rotación en \(\pi\),
  hay cuatro configuraciones fijas
  (ver la figura~\ref{fig:tarjetas-fijas-180}).
  No hay configuraciones fijas bajo rotaciones de \(\pi / 2\)
  ni de \(3 \pi / 2\).
  \begin{figure}[htbp]
    \centering
    \subfloat{
      \pgfimage{images/badge-r2-a}
      \label{subfig:badge-r2-a}
    }%
    \hspace{2em}%
    \subfloat{
      \pgfimage{images/badge-r2-b}
      \label{subfig:badge-r2-b}
    }%
    \hspace{2em}%
    \subfloat{
      \pgfimage{images/badge-r2-c}
      \label{subfig:badge-r2-c}
    }%
    \hspace{2em}%
    \subfloat{
      \pgfimage{images/badge-r2-d}
      \label{subfig:badge-r2-d}
    }
    \caption{Configuraciones fijas bajo rotación en $\pi$}
    \label{fig:tarjetas-fijas-180}
  \end{figure}
  Podemos de la misma forma enumerar las configuraciones fijas
  bajo una reflexión en la vertical,
  ver la figura~\ref{fig:tarjetas-fijas-vertical}.
  Para la reflexión en la horizontal por simetría
  también tenemos seis configuraciones fijas.
  \begin{figure}[htbp]
    \centering
    \subfloat{
      \pgfimage{images/badge-v-a}
      \label{subfig:badge-v-a}
    }%
    \hspace{2em}%
    \subfloat{
      \pgfimage{images/badge-v-b}
      \label{subfig:badge-v-b}
    }%
    \hspace{2em}%
    \subfloat{
      \pgfimage{images/badge-v-c}
      \label{subfig:badge-v-c}
    }

    \subfloat{
      \pgfimage{images/badge-v-d}
      \label{subfig:badge-v-d}
    }%
    \hspace{2em}%
    \subfloat{
      \pgfimage{images/badge-v-e}
      \label{subfig:badge-v-e}
    }%
    \hspace{2em}%
    \subfloat{
      \pgfimage{images/badge-v-f}
      \label{subfig:badge-v-f}
    }
    \caption{Configuraciones fijas bajo reflexión en la vertical}
    \label{fig:tarjetas-fijas-vertical}
  \end{figure}
  Al enumerar las configuraciones fijas
  bajo una reflexión en la diagonal de la esquina inferior izquierda
  a la superior derecha también resultan seis configuraciones
  (figura~\ref{fig:tarjetas-fijas-diagonal}),
  y obtenemos otras seis para la reflexión en la otra diagonal.
  \begin{figure}[htbp]
    \centering
    \subfloat{
      \pgfimage{images/badge-d-a}
      \label{subfig:badge-d-a}
    }%
    \hspace{2em}%
    \subfloat{
      \pgfimage{images/badge-d-b}
      \label{subfig:badge-d-b}
    }%
    \hspace{2em}%
    \subfloat{
      \pgfimage{images/badge-d-c}
      \label{subfig:badge-d-c}
    }

    \subfloat{
      \pgfimage{images/badge-d-d}
      \label{subfig:badge-d-d}
    }%
    \hspace{2em}%
    \subfloat{
      \pgfimage{images/badge-d-e}
      \label{subfig:badge-d-e}
    }%
    \hspace{2em}%
    \subfloat{
      \pgfimage{images/badge-d-f}
      \label{subfig:badge-d-f}
    }
    \caption[Configuraciones fijas bajo reflexión en la diagonal]
	    {Configuraciones fijas bajo reflexión en la diagonal
	     de izquierda inferior a derecha superior}
    \label{fig:tarjetas-fijas-diagonal}
  \end{figure}
  El cuadro~\ref{tab:simetrias-tarjetas}
  resume los valores anteriores.
  \begin{table}[htbp]
    \centering
    \begin{tabular}[c]{|l|>{\(}r<{\)}|}
      \hline
      \multicolumn{1}{|c|}{\rule[-0.7ex]{0pt}{3ex}\textbf{Operación}} &
	\multicolumn{1}{c|}{\textbf{Fijos}} \\
      \hline
	\rule[-0.7ex]{0pt}{3ex}%
      Identidad				     & 36 \\
      Rotación en \(\pi / 2\)		     &	0 \\
      Rotación en \(\pi\)		     &	4 \\
      Rotación en \(3 \pi / 2\)		     &	0 \\
      Reflexión en diagonal \(1\;3\)	     &	6 \\
      Reflexión en diagonal \(2\;4\)	     &	6 \\
      Reflexión en perpendicular a \(1\;2\)  &	6 \\
      Reflexión en perpendicular a \(1\;4\)  &	6 \\
      \hline
    \end{tabular}
    \caption{Número de configuraciones de tarjetas
	     respetadas por cada simetría del cuadrado}
    \label{tab:simetrias-tarjetas}
  \end{table}
  Con estos valores tenemos el lema de Burnside
  que el número de órbitas es:
  \begin{equation*}
    \frac{1}{8} \, (36 + 0 + 4 + 0 + 6 + 6 + 6 + 6) = 8
  \end{equation*}
  En este caso es sencillo
  listar las ocho configuraciones por prueba y error,
  máxime sabiendo que son ocho
  (ver la figura~\ref{fig:tarjetas-distinguibles}),
  \begin{figure}[htbp]
    \centering
   \subfloat{
      \pgfimage{images/badge-a}
      \label{subfig:badge-a}
    }%
    \hspace{2em}%
    \centering
    \subfloat{
      \pgfimage{images/badge-b}
      \label{subfig:badge-b}
    }%
    \hspace{2em}%
    \subfloat{
      \pgfimage{images/badge-c}
      \label{subfig:badge-c}
    }

    \subfloat{
      \pgfimage{images/badge-d}
      \label{subfig:badge-d}
    }%
    \hspace{2em}%
    \subfloat{
      \pgfimage{images/badge-e}
      \label{subfig:badge-e}
    }

    \subfloat{
      \pgfimage{images/badge-f}
      \label{subfig:badge-f}
    }%
    \hspace{2em}%
    \subfloat{
      \pgfimage{images/badge-g}
      \label{subfig:badge-g}
    }%
    \hspace{2em}%
    \subfloat{
      \pgfimage{images/badge-h}
      \label{subfig:badge-h}
    }
    \caption{Las ocho tarjetas distinguibles}
    \label{fig:tarjetas-distinguibles}
  \end{figure}
  pero el resultado es aplicable en forma mucho más general.

\section{Índice de ciclos}
\label{sec:indice-ciclos}
\index{permutacion@permutación!indice de ciclos@índice de ciclos}

  Definimos el tipo de una permutación%
    \index{permutacion@permutación!tipo}
  como
    \(\left[ 1^{\alpha_1} \, 2^{\alpha_2}
	\, \dotso \, n^{\alpha_n} \right]\)
  si tiene \(\alpha_k\) ciclos de largo \(k\)
  para \(1 \le k \le n\).
  Una expresión afín asociada a la permutación \(\gamma\) es:
  \begin{equation*}
    \zeta_\gamma (x_1, x_2, \dotsc, x_n)
      = x_1^{\alpha_1} x_2^{\alpha_2} \dotso x_n^{\alpha_n}
  \end{equation*}
  Para un grupo \(G\) de permutaciones
  definimos el \emph{índice de ciclos}:
  \begin{equation*}
    \zeta_G (x_1, x_2, \dotsc, x_n)
      = \frac{1}{\lvert G \rvert} \,
	  \sum_{\gamma \in G} \zeta_\gamma (x_1, x_2, \dotsc, x_n)
  \end{equation*}
  Esto es esencialmente una función generatriz%
    \index{generatriz}
  en la que \(x_l\) marca los ciclos de largo \(l\).
  Esta función tiene muchos usos,
  algunos los veremos más adelante.

  Interesa calcular el índice de ciclos para diversos grupos
  de manera de tenerlos a mano más adelante.
  Consideremos primero los grupos \(\mathtt{C}_n\),%
    \index{grupo!ciclico@cíclico}
  que sabemos isomorfos con \(\mathbb{Z}_n\) y la suma.
  Si consideramos \(a \in \mathbb{Z}_n\),
  su orden determina el largo de los ciclos,
  y el número de ciclos es simplemente \(n / \ord(a)\).
  El orden es el mínimo \(b > 0\)
  tal que \(a \cdot b \equiv 0 \pmod{n}\).
  Si \(\gcd(a, n) = 1\),
  es \(b = n\) y
  hay \(\phi(n)\) de tales \(a\)
  que dan \(n / n = 1\) ciclo de largo \(n\).
  En general,
  si \(\gcd(a, n) = c\),
  las posibilidades de \(a\) esencialmente diferentes
  se restringen a \(n / c\) elementos,
  y de estos dan orden \(n / c\)
  exactamente los que tienen \(\gcd(a, n / c) = 1\).
  Expresarlo de esta forma es incómodo,
  llamemos \(d = n / c\).
  Entonces para \(d \mid n\) hay \(\phi(d)\) elementos
  de orden \(d\),
  los cuales forman \(n / d\) ciclos de largo \(d\):
  \begin{equation*}
    \zeta_{\mathtt{C}_n} (x_1, \dotsc, x_n)
      = \frac{1}{n} \,
	  \sum_{d \mid n} \phi(d) \, x_d^{n / d}
  \end{equation*}
% Fixme: Relacionar con operador \Cyc() de método simbólico sin rotular

  Veamos ahora el caso \(\mathtt{D}_{2 n}\).%
    \index{grupo!dihedral}
  A las simetrías anteriores se añaden \(n\) reflexiones.
  Si \(n\) es par,
  hay \(n / 2\) reflexiones a través de vértices opuestos,
  son \(2\) ciclos de largo \(1\) y \(n - 2\) ciclos de largo \(2\)
  que aportan \(n x_1^2 x_2^{(n - 2) / 2} / 2\);
  y \(n / 2\) reflexiones a través de lados opuestos,
  son \(n / 2\) ciclos de largo \(2\)
  que aportan \(n x_2^{n / 2} / 2\).
  Si \(n\) es impar,
  hay \(n\) reflexiones a través de un vértice y el lado opuesto,
  o sea un ciclo de largo \(1\)
  y \((n - 1) / 2\) ciclos de largo \(2\),
  que aportan \(n x_1 x_2^{(n - 1) / 2}\).
  En resumen,
  como en el índice de ciclos
  aparecen divididos por el orden del grupo,
  que se duplica a \(2 n\)
  entre \(\mathtt{C}_n\) y \(\mathtt{D}_{2 n}\):
  \begin{equation*}
    \zeta_{\mathtt{D}_{2 n}}
      = \frac{1}{2} \, \zeta_{\mathtt{C}_n}(x_1, \dotsc, x_n) +
	   \begin{cases}
	     \frac{1}{4} \, (x_1^2 x_2^{(n - 2) / 2} + x_2^{n / 2})
		 & \text{si \(n\) es par} \\[1ex]
	     \frac{1}{2} \, x_1 x_2^{(n - 1) / 2}
		 & \text{si \(n\) es impar}
	   \end{cases}
  \end{equation*}
  Así tenemos por ejemplo para el cuadrado:
  \begin{align*}
    \zeta_{\mathtt{C}_4} (x_1, x_2, x_3, x_4)
      &= \frac{1}{4} \, \sum_{d \mid 4} \phi(d) \, x_d^{4 / d} \\
      &= \frac{1}{4} \, \left(
			  x_1^4 + x_2^2 + 2 x_4
			\right) \\
    \zeta_{\mathtt{D}_8} (x_1, x_2, x_3, x_4)
      &= \frac{1}{2} \, \zeta_{\mathtt{C}_4} (x_1, x_2, x_3, x_4)
	   + \frac{1}{4} \, \left(
			       x_1^2 x_2 + x_2^2
			    \right) \\
      &= \frac{1}{8} \, \left(
			   x_1^4 + 2 x_1^2 x_2 + 3 x_2^2 + 2 x_4
			\right)
  \end{align*}
% Fixme: \Cyc() tal vez?

\section{Número de coloreos distinguibles}
\label{sec:coloreos-distinguibles}

% Fixme: Discutir grupo de permutaciones de arcos en K_n

% Fixme: Esta discusión es más complicada de lo que se requiere...
%	 Revisar/clarificar
%	 Agregar ejemplos de Richard P. Stanley "Topics in Algebraic
%	 Combinatorics"
%	 Más ejemplos útiles en William May "Introduction to Pólya
%	 Enumeration Theory"

  Supongamos un grupo \(G\) de permutaciones
  de un conjunto \(\mathcal{X}\) de \(n\) elementos,
  y a cada elemento se le puede asignar uno de \(r\) colores.
  Si el conjunto de colores es \(\mathcal{K}\),
  un \emph{coloreo}
  es una función
    \(\omega \colon \mathcal{X} \rightarrow \mathcal{K}\).
  El número total de coloreos es \(r^n\),
  a este conjunto le llamaremos \(\Omega\).
  Ahora bien,
  cada permutación \(g\) en \(G\)
  induce una permutación \(\widehat{g}\) de \(\Omega\):
  Para \(\omega\) definimos \(\widehat{g}(\omega)\)
  como el coloreo en el cual el color asignado a \(x\)
  es el que \(\omega\) asigna a \(g(x)\),
  vale decir:
  \begin{equation*}
    (\widehat{g}(\omega))(x) = \omega g^{-1}(x)
  \end{equation*}
  La inversa aparece porque al aplicar la permutación al coloreo
  estamos asignando a \(x\)
  el color que tiene su predecesor vía \(g\).
  La figura~\ref{fig:g-hat} muestra un ejemplo.
  \begin{figure}[htbp]
    \centering
    \subfloat{
      \pgfimage{images/g-hat-a}
    }%
    \hspace{3em}%
    \subfloat{
      \pgfimage{images/g-hat-b}
    }
    \caption{Efecto de la permutación $\widehat{g}$
	     sobre un coloreo $\omega$}
    \label{fig:g-hat}
  \end{figure}
% Fixme: Tomar ejemplo completo p.ej. de R. Stanley
  La función que lleva \(g\) a \(\widehat{g}\)
  es una representación del grupo \(G\)
  en un grupo \(\widehat{G}\) de permutaciones de \(\Omega\).
  Dos coloreos son indistinguibles
  si uno puede transformarse en el otro
  mediante una permutación \(\widehat{g}\);
  vale decir,
  si ambas pertenecen a la misma órbita
  de \(\widehat{G}\) en \(\Omega\).
  El número de coloreos distinguibles
  (\emph{inequivalentes})
  entonces es el número de órbitas de \(\widehat{G}\).
  Antes de aplicar nuestro teorema para el número de órbitas
  debemos relacionar \(G\) y \(\widehat{G}\).
  Supongamos que para dos permutaciones \(g_1\) y \(g_2\)
  tenemos \(\widehat{g}_1 = \widehat{g}_2\),
  de forma que
  \begin{equation*}
    (\widehat{g}_1(\omega))(x) = (\widehat{g}_2(\omega))(x)
  \end{equation*}
  y en consecuencia para todo \(\omega \in \Omega\)
  y todo \(x \in \mathcal{X}\) debe ser
  \begin{equation*}
    \omega(g_1^{-1}(x)) = \omega(g_2^{-1}(x))
  \end{equation*}
  Como esto es válido para todo \(\omega\),
  en particular vale para el coloreo
  que asigna el color especificado a \(g_1^{-1}(x)\)
  y otro color a todos los demás miembros de \(\mathcal{X}\).
  En este caso particular la ecuación
  dice que \(g_1^{-1}(x) = g_2^{-1}(x)\),
  con lo que \(g_1 = g_2\),
  y el grupo de permutaciones \(G\) de \(\mathcal{X}\)
  y el grupo de permutaciones \(\widehat{G}\)
  de los coloreos \(\Omega\) son isomorfos.

  Otra manera de entender esta situación
  es considerar un coloreo que le asigna un color diferente
  a cada elemento de \(\mathcal{X}\).
  Una permutación de ese coloreo
  no es más que una permutación
  de nuevos ``nombres'' de los elementos de \(\mathcal{X}\),
  con lo que está claro que ambos grupos de permutaciones
  están muy relacionados.

  \begin{theorem}
    \label{theo:coloreos}
    Si \(G\) es un grupo de permutaciones de \(\mathcal{X}\),
    y \(\zeta_G(x_1, \dotsc, x_n)\) es su índice de ciclos,
    el número de coloreos inequivalentes
    de \(\mathcal{X}\) con \(r\) colores
    es \(\zeta_G(r, \dotsc, r)\),
    donde un coloreo de \(\mathcal{X}\)
    es una función
      \(\omega \colon \mathcal{X} \rightarrow \mathcal{K}\).
  \end{theorem}
  \begin{proof}
    Interesa el número de órbitas del grupo \(G\)
    operando sobre coloreos.
    Hemos demostrado
    que la representación \(g \rightarrow \widehat{g}\)
    es una biyección,
    de forma que \(\lvert G \rvert = \lvert \widehat{G} \rvert\).
    Además,
    por el teorema de Burnside
    el número de órbitas de \(\widehat{G}\) en \(\Omega\) es
    \begin{equation*}
      \frac{1}{\lvert \widehat{G} \rvert} \,
	  \sum_{\widehat{g} \in \widehat{G}}
	    \lvert F(\widehat{g}) \rvert
	= \frac{1}{\lvert G \rvert} \,
	      \sum_{g \in G} \lvert F(\widehat{g}) \rvert
    \end{equation*}
    donde \(F(\widehat{g})\)
    es el conjunto de coloreos fijados por \(\widehat{g}\).
    Supongamos ahora que \(\omega\)
    es un coloreo fijado por \(\widehat{g}\),
    de forma que \(\widehat{g}(\omega) = \omega\),
    y sea \((x \; y \; z \; \dotso)\) un ciclo cualquiera de \(g\).
    Tenemos:
    \begin{equation*}
      \omega(x)
	= \omega(g(y))
	= (\widehat{g}(\omega))(y)
	= \omega(y)
    \end{equation*}
    de forma que \(\omega\) asigna el mismo color a \(x\) e \(y\).
    Aplicando el mismo razonamiento,
    este es el color asignado a todo el ciclo.
    Esto ocurre con cada uno de los ciclos de \(g\).
    Si \(g\) tiene \(k\) ciclos en total,
    el número de coloreos posibles es \(r^k\),
    ya que podemos asignar independientemente
    cualquiera de los \(r\) colores a cada uno de los \(k\) ciclos.
    De esta forma,
    si \(g\) tiene \(\alpha_i\) ciclos de largo \(i\)
    para (\(1 \le i \le n)\),
    tenemos \(\alpha_1 + \alpha_2 + \dotsb + \alpha_n = k\) y
    \begin{equation*}
      \lvert F(\widehat{g}) \rvert
	= r^k
	= r^{\alpha_1 + \alpha_2 + \dotsb + \alpha_n}
	= \zeta_g(r, r, \dotsc, r)
    \end{equation*}
    y el resultado sigue de sumar esto.
  \end{proof}

  \begin{example}
    Una tribu de hippies artesanos fabrica pulseras
    formadas alternadamente por tres arcos y tres cuentas,
    y tienen arcos y cuentas de cinco colores.
    Para efectos de simetría
    pueden considerarse las pulseras como triángulos equiláteros
    en los cuales se colorean los vértices y las aristas.
    Por razones que solo ellos entienden las pulseras
    deben siempre usar tres colores.
    Interesa saber cuántas pulseras diferentes pueden crear.

    Esta es una aplicación típica
    del principio de inclusión y exclusión%
      \index{inclusion y exclusion, principio de@inclusión y exclusión, principio de}
    (capítulo~\ref{cha:pie}),
    el teorema~\ref{theo:coloreos} da el número de coloreos
    con \emph{a lo más} el número de colores dado,
    pero nos interesan los coloreos
    con \emph{exactamente} tres colores.

    \begin{table}[ht]
      \centering
      \begin{tabular}{|l|>{\(}c<{\)}|>{\(}l<{\)}|}
	\hline
	\multicolumn{1}{|c|}{\rule[-0.7ex]{0pt}{3ex}\textbf{Operación}} &
	  \multicolumn{1}{c|}{\textbf{Nº}} &
	  \multicolumn{1}{c|}{\textbf{Término}}				\\
	\hline
	\rule[-0.7ex]{0pt}{3ex}%
	Identidad				      & 1 & x_1^6	\\
	Rotaciones (en \(2 \pi / 3\) y \(4 \pi / 3\)) & 2 & x_3^2	\\
	Reflexiones en cada eje			      & 3 & x_1^2 x_2^2 \\
	\hline
      \end{tabular}
      \caption{Elementos del grupo para pulseras}
      \label{tab:tabla-hippies}
    \end{table}
    El cuadro~\ref{tab:tabla-hippies}
    da los elementos del grupo relevante.
    Este grupo es de orden \(6\),
    así que su índice de ciclos es:
    \begin{equation*}
      \zeta_G(x_1, x_2, x_3, x_4, x_5, x_6)
	= \frac{1}{6} \,
	    \left(
	      x_1^6
		+ 3 x_1^2 x_2^2
		+ 2 x_3^2
	    \right)
    \end{equation*}

    Luego aplicamos nuestra receta
    del principio de inclusión y exclusión.
    \begin{enumerate}
    \item
      El universo \(\Omega\)
      es el conjunto de coloreos con \(5\) colores.
      Un coloreo tiene la propiedad \(i\)
      si el color \(i\) no está presente,
      e interesa el número
      de los que tienen exactamente \(2\) propiedades
      (están presentes los otros \(3\) colores).
    \item
      Acá \(N(\supseteq S)\) es el número de coloreos
      que no consideran los colores en \(S\),
      vale decir son coloreos
      tomando a lo más \(5 - \lvert S \rvert\) colores.
      Por el teorema~\ref{theo:coloreos}:
      \begin{equation*}
	N(\supseteq S)
	  = \zeta_G(5 - \lvert S \rvert,\,
		    5 - \lvert S \rvert,\,
		    5 - \lvert S \rvert,\,
		    5 - \lvert S \rvert,\,
		    5 - \lvert S \rvert,\,
		    5 - \lvert S \rvert)
      \end{equation*}
    \item
      Como los \(r\) colores a excluir se eligen de entre los \(5\),
      y en el número de posibilidades
      solo influye el número de colores restantes
      con los que se colorea:
      \begin{equation*}
	N_r
	  = \binom{5}{r} \,
	     \zeta_G(5 - r,\,
		     5 - r,\,
		     5 - r,\,
		     5 - r,\,
		     5 - r,\,
		     5 - r)
      \end{equation*}
      En este caso tenemos:
      \begin{align*}
	N_0
	  &= \binom{5}{0}
	       \, \zeta_G(5, 5, 5, 5, 5, 5) =		2\,925 \\
	N_1
	  &= \binom{5}{1}
	       \, \zeta_G(4, 4, 4, 4, 4, 4) =		4\,080 \\
	N_2
	  &= \binom{5}{2}
	       \, \zeta_G(3, 3, 3, 3, 3, 3) =		1\,650 \\
	N_3
	  &= \binom{5}{3}
	       \, \zeta_G(2, 2, 2, 2, 2, 2) = \phantom{0}\,200 \\
	N_4
	  &= \binom{5}{2}
	       \, \zeta_G(1, 1, 1, 1, 1, 1) = \phantom{000}\,5 \\
	N_5
	  &= \binom{5}{5}
	       \, \zeta_G(0, 0, 0, 0, 0, 0) = \phantom{000}\,0
      \end{align*}
      La función generatriz es
      \begin{equation*}
	N(z)
	  = 5 z^4 + 200 z^3 + 1\,650 z^2 + 4\,080 z + 2\,925
      \end{equation*}
    \item
      Nos interesa \(e_2\),
      que se obtiene de la función generatriz de los \(e_t\),
      que es \(E(z) = N(z - 1)\):
      \begin{equation*}
	E(z) = 5 z^4 + 180 z^3 + 1\,080 z^2 + 1\,360 z + 300
      \end{equation*}
      Se pueden formar \(1\,080\) brazaletes de tres colores.
    \end{enumerate}
  \end{example}

  Pero podemos hacer algo más.
  Si hay \(r\) colores,
  podemos definir variables \(z_i\) para \(1 \le i \le r\)
  representando los distintos colores.
  Entonces la función generatriz
  de los números de nodos de cada color
  que se pueden asignar a un ciclo de largo \(k\) es simplemente:
  \begin{equation*}
    z_1^k + z_2^k + \dotsb + z_r^k
  \end{equation*}
  ya que serían \(k\) nodos,
  todos del mismo color.
  Si hay \(\alpha_k\) ciclos de largo \(k\),
  entonces corresponde el factor:
  \begin{equation*}
    (z_1^k + z_2^k + \dotsb + z_r^k)^{\alpha_k}
  \end{equation*}
  La anterior discusión demuestra el siguiente resultado.
  \begin{theorem}[Enumeración de Pólya]
    \label{theo:Pólya}
    Sea \(G\) un grupo de permutaciones de \(\mathcal{X}\),
    y \(\zeta_G(x_1, \dotsc, x_n)\) su índice de ciclos.
    La función generatriz
    del número de coloreos inequivalentes de \(\mathcal{X}\)
    en que hay \(n_i\) nodos de color \(i\) para \(1 \le i \le r\),
    llamémosle \(u_{n_1, n_2, \dotsc, n_r}\),
    es:
    \begin{align*}
      U(z_1, z_2, \dotsc, z_r)
	&= \sum_{n_1, n_2, \dotsc, n_r}
	     u_{n_1, n_2, \dotsc, n_r} z_1^{n_1} z_2^{n_2}
		\dotsm z_r^{n_r} \\
	&= \zeta_G(z_1 + z_2 + \dotsb + z_r,
		   z_1^2 + z_2^2 + \dotsb + z_r^2,
		   \dotsc,
		   z_1^n + z_2^n + \dotsb + z_r^n)
    \end{align*}
  \end{theorem}

  \begin{table}[htbp]
    \centering
    \begin{tabular}{|l|>{\(}c<{\)}|>{\(}l<{\)}|}
      \hline
      \multicolumn{1}{|c|}{\rule[-0.7ex]{0pt}{3ex}\textbf{Operación}} &
	\multicolumn{1}{c|}{\textbf{Nº}} &
	\multicolumn{1}{c|}{\textbf{Término}} \\
      \hline
	\rule[-0.7ex]{0pt}{3ex}%
      Identidad					  & 1 & x_1^9	    \\
      Giro en \(\pi / 2\), \(3 \pi / 2\)	  & 2 & x_1 x_4^2   \\
      Giro en \(\pi\)				  & 1 & x_1 x_2^4   \\
      Reflexión horizontal, vertical		  & 2 & x_1^3 x_2^3 \\
      Reflexión diagonal			  & 2 & x_1^3 x_2^3 \\
      \hline
    \end{tabular}
    \caption{Las operaciones sobre tarjetas y sus tipos}
    \label{tab:tarjetas-tipos}
  \end{table}
  Volviendo a nuestro ejemplo de las tarjetas de identidad,
  el grupo subyacente es \(\mathtt{D}_8\),%
    \index{grupo!dihedral}
  las operaciones y sus tipos
  (operando sobre los nueve cuadraditos)
  se resumen en el cuadro~\ref{tab:tarjetas-tipos}.
  El índice de ciclos del grupo que interesa es:
  \begin{equation*}
    \zeta_t (x_1, x_2, x_3, x_4, x_5, x_6, x_7, x_8, x_9)
      = \frac{1}{8} \,
	  \left(
	    x_1^9 + 2 x_1 x_4^2 + x_1 x_2^4 + 4 x_1^3 x_2^3
	  \right)
  \end{equation*}
  El número total de tarjetas distinguibles con dos colores
  (agujero o no)
  en cada cuadradito es:
  \begin{equation*}
    \zeta_t (2, 2, 2, 2, 2, 2, 2, 2, 2)
      = 102
  \end{equation*}
  Para obtener el número de tarjetas con dos agujeros
  calculamos:
  \begin{equation*}
    \left[ z^2 \right]
       \zeta_t (1 + z, 1 + z^2, 1 + z^3, 1 + z^4, 1 + z^5, 1 + z^6,
		1 + z^7, 1 + z^8, 1 + z^9)
      = 8
  \end{equation*}
  Esto ya lo habíamos calculado antes,
  aunque de forma más trabajosa.
  Queda de ejercicio calcular del número de collares diferentes
  que se pueden crear
  de \(16\)~cuentas con \(3\)~negras de la misma forma.

% Fixme: Completar \Cyc() de método simbólico:
% A = \Cyc(B) ==>
%   A(z) = \sum_{k \ge 1} \frac{\phi(k)}{k} \cdot \ln \frac{1}{1 - B(z^k)}

  La teoría de enumeración de Pólya%
    \index{Polya, teoria de enumeracion de@Pólya, teoría de enumeración de}
  fue desarrollada en parte para aplicación a la química.
  Algunos ejemplos de fórmulas químicas
  se dan en la figura~\ref{fig:aromaticos}.
  \begin{figure}[htbp]
    \centering
    \subfloat[Benceno]{
      \pgfimage[height=0.25\textwidth]{images/benceno}
      \label{subfig:benceno}
    }%
    \hspace*{3em}%
    \subfloat[Xyleno]{
      \pgfimage[height=0.25\textwidth]{images/o-xyleno}
      \label{subfig:xyleno}
    }%
    \hspace*{3em}%
    \subfloat[Clorotolueno]{
      \pgfimage[height=0.25\textwidth]{images/clorotolueno}
      \label{subfig:clorotolueno}
    }
    \caption{Algunos compuestos aromáticos}
    \label{fig:aromaticos}
  \end{figure}
  Este tipo de compuestos,
  derivados del benceno (figura~\ref{subfig:benceno})
  son hexágonos que pueden girar en el espacio.
  Hay muchas posibilidades de grupos de átomos
  que pueden reemplazar los hidrógenos (\(\text{H}\)),
  como es radicales metilo (\(\text{CH}_3\))
  o átomos de cloro (\(\text{Cl}\)).
  Una pregunta obvia entonces
  es cuántos compuestos distintos pueden crearse
  con un conjunto de radicales,
  o cuántos son posibles con un número particular
  de cada uno de un conjunto de radicales dados.
  Por ejemplo,
  hay tres isómeros del xyleno
  (un anillo de benceno en el cual dos de los hidrógenos
   se substituyen por metilos),
  como muestra la figura~\ref{fig:isomeros-xyleno}.
  \begin{figure}[htbp]
    \centering
    \subfloat[Orto-xyleno]{
      \pgfimage[height=0.25\textwidth]{images/o-xyleno}
      \label{subfig:o-xyleno}
    }%
    \hspace*{3em}%
    \subfloat[Para-xyleno]{
      \pgfimage[height=0.25\textwidth]{images/p-xyleno}
      \label{subfig:p-xyleno}
    }%
    \hspace*{3em}%
    \subfloat[Meta-xyleno]{
      \pgfimage[height=0.25\textwidth]{images/m-xyleno}
      \label{subfig:m-xyleno}
    }
    \caption{Los tres isómeros del xyleno}
    \label{fig:isomeros-xyleno}
  \end{figure}

  El grupo de simetría relevante es \(\mathtt{D}_{12}\),
  cuyo índice de ciclos podemos calcular como antes:
  \begin{align*}
    \zeta_{\mathtt{C}_6} (x_1, x_2, x_3, x_4, x_5, x_6)
      &= \frac{1}{6} \, \sum_{d \mid 6} \phi(d) x_d^{6 / d} \\
      &= \frac{1}{6} \, \left(
			  x_1^6 + x_2^3 + 2 x_3^2 + 2 x_6
			\right) \\
    \zeta_{\mathtt{D}_{12}} (x_1, x_2, x_3, x_4, x_5, x_6)
      &= \frac{1}{2} \, \zeta_{\mathtt{C}_6}
			  (x_1, x_2, x_3, x_4, x_5, x_6)
	   + \frac{1}{4} \, \left(
			      x_1^2 x_2^2 + x_2^3
			    \right) \\
      &= \frac{1}{12} \, \left(
			   x_1^6 + 3 x_1^2 x_2^2
			     + 4 x_2^3 + 2 x_3^2 + 2 x_6
			 \right)
  \end{align*}
  Hecho el trabajo duro,
  determinar cuántos compuestos pueden crearse
  con radicales hidrógeno (\(\text{H}\))
  y metilo (\(\text{CH}_3\))
  es fácil:
  Es colorear los vértices con dos colores,
  lo que da:
  \begin{align*}
    \zeta_{\mathtt{D}_{12}}(2, 2, 2, 2, 2, 2)
      & = \frac{1}{12} \,
	     (2^6 + 3 \cdot 2^2 \cdot 2^2
		  + 4 \cdot 2^3 + 2 \cdot 2) \\
      & = 13
  \end{align*}

  Para comprobar cuántos isómeros del xyleno hay,
  consideramos coloreo de los vértices del hexágono con dos colores
  (hidrógenos y metilos),
  y de los últimos hay exactamente dos.
  Si consideramos que \(z\) marca metilo,
  la función generatriz que corresponde a un ciclo de largo \(l\)
  es simplemente \(1 + z^l\)
  (hay 1 forma de tener 0 metilos en él,
   lo que aporta \(1 \cdot z^0\),
   y 1 forma de tener l metilo,
   lo que aporta \(1 \cdot z^l\)),
  y al substituir \(x_l = 1 + z^l\) obtenemos:
  \begin{equation*}
    \zeta_{\mathtt{D}_{12}}(1 + z, 1 + z^2, 1 + z^3, 1 + z^4,
			    1 + z^5, 1 + z^6)
      = z^6 + z^5 + 3 z^4 + 3 z^3 + 3 z^2 + z + 1
  \end{equation*}
  Esto confirma que hay tres isómeros en su coeficiente de \(z^2\).

  Si interesa determinar cuántos compuestos distintos tienen
  2 radicales cloro (\(\text{Cl}\)),
  2 metilos (\(\text{CH}_3\))
  y 2 hidrógenos  (\(\text{H}\)),
  usamos las variables \(u\), \(v\) y \(w\)
  para estas tres opciones,
  y el valor buscado es simplemente:
  \begin{equation*}
    \left[ u^2 v^2 w^2 \right]
      \zeta_{\mathtt{D}_6} (u + v + w,
			    u^2 + v^2 + w^2,
			    \dotsc,
			    u^6 + v^6 + w^6)
	 = 11
  \end{equation*}

  Por otro lado,
  un átomo de carbono puede unirse con cuatro otros átomos,
  dispuestos en los vértices de un tetraedro.%
    \index{poliedro!regular}%
    \index{tetraedro}
  \begin{figure}[htbp]
    \centering
    \subfloat[Eje a través de un vértice]{
      \pgfimage{images/tetraedro-vertice}
      \label{subfig:tetraedro-vertice}
    }
    \vspace*{1em}
    \subfloat[Eje en el punto medio de aristas]{
      \pgfimage{images/tetraedro-arista}
      \label{subfig:tetraedro-arista}
    }
    \caption{Operaciones de simetría (rotaciones) de un tetraedro}
    \label{fig:rotaciones-tetraedro-2}
  \end{figure}
  Las operaciones de simetría de un tetraedro en el espacio
  (solo rotaciones, no reflexiones)
  son giros alrededor de un eje que pasa por un vértice
  y el centroide de la cara opuesta
  (ver la figura~\ref{subfig:tetraedro-vertice})
  y giros alrededor de un eje
  que pasa por el punto medio de una arista
  y el punto medio de la arista opuesta
  (ver la figura~\ref{subfig:tetraedro-arista}).
  \begin{table}[htbp]
    \centering
    \begin{tabular}{|l|*{4}{>{\(}l<{\)}|}}
      \hline
      \multicolumn{1}{|c|}{\rule[-0.7ex]{0pt}{3ex}\textbf{Operación}} &
	\multicolumn{1}{c|}{\textbf{Ciclos}}  &
	\multicolumn{1}{c|}{\textbf{Tipo}}    &
	\multicolumn{1}{c|}{\textbf{Nº}}      &
	\multicolumn{1}{c|}{\textbf{Término}} \\
      \hline
	 \rule[-0.7ex]{0pt}{3ex}%
      Identidad			 &
	(1)(2)(3)(4) & [1^4]	  & 1 & x_1^4	  \\
      Giro en vértice 4 en 1/3	 &
	(1\;2\;3)(4) & [1 \, 3^1] & 4 & x_1^3 x_3 \\
      Giro en vértice 4 en 2/3	 &
	(1\;3\;2)(4) & [1 \, 3^1] & 4 & x_1^3 x_3 \\
      Giro en arista 1\;2 en 1/2 &
	(1\;2)(3\;4) & [2^2]	  & 3 & x_2^2	  \\
      \hline
    \end{tabular}
    \caption{Rotaciones de un tetraedro}
    \label{tab:rotaciones-tetraedro}
  \end{table}
  Las simetrías son de los tipos dados
  en el cuadro~\ref{tab:rotaciones-tetraedro}
  (resulta que esto no es más
   que el grupo alternante \(\mathtt{A}_4\)),%
    \index{grupo!alternante}
  y en consecuencia el índice de ciclos del grupo es
  \begin{equation*}
    \zeta_{\mathtt{A}_4}(x_1, x_2, x_3, x_4)
      = \frac{1}{12}(x_1^4 + 8 x_1 x_3 + 3 x_2^2)
  \end{equation*}
  Así,
  para dos radicales diferentes
  hay \(\zeta_{\mathtt{A}_4}(2, 2, 2, 2) = 5\) compuestos posibles,
  y para cuatro radicales
  hay \(\zeta_{\mathtt{A}_4}(4, 4, 4, 4) = 36\).
  Si hay dos tipos de radicales,
  la función generatriz es:
  \begin{equation*}
    \zeta_{\mathtt{A}_4}(u + v,
		u^2 + v^2,
		u^3 + v^3,
		u^4 + v^4)
      = u^4 + u^3 v + u^2 v^2 + u v^3 + v^4
  \end{equation*}
  Vale decir,
  hay un solo compuesto
  de cada una de las cinco composiciones posibles.

  Considere árboles binarios completos de altura 2,
  como en la figura~\ref{fig:arbol-binario-simetria},
  \begin{figure}[htbp]
    \centering
    \pgfimage{images/arbol-simetria}
    \caption{Un árbol binario completo}
    \label{fig:arbol-binario-simetria}
  \end{figure}
  que se consideran iguales al intercambiar izquierda y derecha
  (como 3 con 4;
  pero también 2 con 5,
  que lleva consigo intercambiar 3 con 6 y 4 con 7).
  Interesa determinar cuántos árboles hay con 3 nodos azules,
  si los nodos se pintan de azul,
  rojo y amarillo.

  Antes de entrar en el tema,
  es útil obtener información sobre el grupo.
  \begin{table}[htbp]
    \centering
    \begin{tabular}{|>{\(}l<{\)}|>{\(}l<{\)}|}
      \hline
      \multicolumn{1}{|c|}{\rule[-0.7ex]{0pt}{3ex}\textbf{Operación}} &
	\multicolumn{1}{c|}{\textbf{Término}}	\\
      \hline
	 \rule[-0.7ex]{0pt}{3ex}%
      \text{Identidad}	   & x_1^7	  \\
      (3\;4)		   & x_1^5 x_2	  \\
      (6\;7)		   & x_1^5 x_2	  \\
      (3\;4) (6\;7)	   & x_1^3 x_2^2  \\
      (2\;5) (3\;6) (4\;7) & x_1 x_2^3	  \\
      (2\;5) (3\;7) (4\;6) & x_1 x_2^3	  \\
      (2\;5) (3\;6\;4\;7)  & x_1 x_2 x_4  \\
      (2\;5) (3\;7\;4\;6)  & x_1 x_2 x_4  \\
      \hline
    \end{tabular}
    \caption{El grupo de operaciones del árbol}
    \label{tab:grupo-arbol}
  \end{table}
  Para determinar el orden del grupo,
  tomamos algún elemento y analizamos su órbita y estabilizador.
  Tomando 3,
  su órbita es \(G3 = \{3, 4, 6, 7\}\),
  mientras su estabilizador es \(G_3 = \{\iota, (6\;7)\}\),
  con lo que
  \(\lvert G \rvert
      = \lvert G3 \rvert \cdot \lvert G_3 \rvert
      = 4 \cdot 2
      = 8\).
  Los elementos del grupo los da el cuadro~\ref{tab:grupo-arbol},
  el índice de ciclos del grupo resulta ser:
  \begin{equation*}
    \zeta_G(x_1, x_2, x_3, x_4, x_5, x_6, x_7)
      = \frac{1}{8} \,
	  \left(
	    x_1^7
	      + 2 x_1^5 x_2
	      + x_1^3 x_2^2
	      + 2 x_1 x_2^3
	      + x_1 x_2 x_4
	  \right)
  \end{equation*}
  La manera más simple de obtener el resultado buscado
  es reconocer que la función generatriz%
    \index{generatriz}
  para el número de maneras de formar órbitas de \(l\) nodos
  donde \(u\) marca el número de nodos azules
  es simplemente \(2 + u^l\)
  (dos formas de ningún azul,
   vale decir solo rojos o solo amarillos;
   y una forma de \(l\) azules),
  y para aplicar el teorema de Pólya interesa:
  \begin{align*}
    \left[ u^3 \right] \zeta_G(2 + u,
	       &  2 + u^2,
		  2 + u^3,
		  2 + u^4,
		  2 + u^5,
		  2 + u^6,
		  2 + u^7) \\
      &= \left[ u^3 \right] \left(
		 u^7
		   +   6 u^6
		   +  25 u^5
		   +  68 u^4
		   + 120 u^3
		   + 146 u^2
		   + 105 u
		   +  42
		 \right) \\
       &= 120
  \end{align*}
  Obtener esto por prueba y error sería impensable.
  Nuevamente agradecemos el apoyo algebraico
  de \texttt{maxima}~%
    \cite{maxima14b:_computer_algebra}.%
    \index{maxima@\texttt{maxima}}

  Un dado es un cubo,%
    \index{poliedro!regular}%
    \index{cubo}
  cuyas caras están numeradas de 1 a 6.
  Interesa saber de cuántas maneras distintas
  se pueden distribuir los seis números sobre las caras.
  En este caso,
  interesan las simetrías rotacionales del cubo en el espacio
  (las reflexiones corresponden
   a operaciones imposibles con un sólido).
  Primeramente calculamos el orden del grupo que nos interesa,
  que resulta ser el grupo de rotaciones
  en el espacio de un octaedro
  y se denomina \(\mathtt{O}\).
  Sabemos que
    \(\lvert \mathtt{O} \rvert
	= \lvert \mathtt{O}_x \rvert
	    \cdot \lvert \mathtt{O} x \rvert\).
  Si fijamos una de las caras del cubo,
  hay 4 operaciones que la mantienen fija
  (rotaciones alrededor del centroide
   de esa cara en múltiplos de \(\pi / 2\)),
  y esta cara puede ocupar cualquiera de las 6 posiciones.
  Luego \(\lvert \mathtt{O} \rvert = 4 \cdot 6 = 24\).
  Las operaciones y sus tipos las resume el cuadro~\ref{tab:dado}.
  \begin{table}[htbp]
    \centering
    \begin{tabular}{|l|c|>{\(}l<{\)}|}
      \hline
      \multicolumn{1}{|c|}{\rule[-0.7ex]{0pt}{3ex}\textbf{Operación}} &
	\multicolumn{1}{c|}{\textbf{Nº}} &
	\multicolumn{1}{c|}{\textbf{Término}} \\
      \hline
	\rule[-0.7ex]{0pt}{3ex}%
      Identidad						      & 1 &
	 x_1^6 \\
      Giro alrededor de centro de una cara en \(\pi / 2\)     & 3 &
	 x_1^2 x_4 \\
      Giro alrededor de centro de una cara en \(\pi\)	      & 3 &
	 x_1^2 x_2^2 \\
      Giro alrededor de centro de una cara en \(3 \pi / 2\)   & 3 &
	 x_1^2 x_4 \\
      Giro alrededor del punto medio de una arista en \(\pi\) & 6 &
	 x_2^3 \\
      Giro alrededor de un vértice en \(2 \pi / 3\)	      & 4 &
	 x_3^2 \\
      Giro alrededor de un vértice en \(4 \pi / 3\)	      & 4 &
	 x_3^2 \\
      \hline
    \end{tabular}
    \caption{Operaciones de simetría rotacional de caras de un cubo}
    \label{tab:dado}
  \end{table}
  El índice de ciclos del grupo \(\mathtt{O}\) es
  \begin{equation*}
    \zeta_{\mathtt{O}}(x_1, x_2, x_3, x_4, x_5, x_6)
      = \frac{1}{24} \,
	  \left(
	    x_1^6 + 6 x_1^2 x_4 + 3 x_1^2 x_2^2 + 6 x_2^3 + 8 x_3^2
	  \right)
  \end{equation*}
  Como interesa saber de cuántas maneras
  se pueden distribuir los 6~números sobre las 6~caras:
  \begin{equation*}
    \left[ z_1 z_2 z_3 z_4 z_5 z_6 \right]
      \zeta_{\mathtt{O}}(z_1 + \dotsb + z_6,
	      z_1^2 + \dotsb + z_6^2,
	      z_1^3 + \dotsb + z_6^3,
	      z_1^4 + \dotsb + z_6^4,
	      z_1^5 + \dotsb + z_6^5,
	      z_1^6 + \dotsb + z_6^6)
  \end{equation*}
  Siquiera encontrar el término que interesa en esta expresión
  ya es toda una tarea.
  Pero si observamos que la única forma
  de obtener términos en los que los \(z_i\)
  entran en la primera potencia vienen de aquellos términos
  en que solo participa \(x_1\),
  nuestro problema se reduce a calcular:
  \begin{align*}
    \left[ z_1 z_2 z_3 z_4 z_5 z_6 \right]
	\frac{1}{24} \, (z_1 + z_2 + z_3 + z_4 + z_5 + z_6)^6
      &= \frac{1}{24} \, \binom{6}{1\;1\;1\;1\;1\;1} \\
      &= 30
  \end{align*}

  Si nos interesa contar el número de maneras
  de numerar las caras del dado
  respetando la restricción que caras opuestas sumen 7,
  la situación relevante
  es considerar las tres caras numeradas 1, 2 y 3.
  Estas caras serán adyacentes,
  y por tanto la situación
  es la que indica la figura~\ref{fig:cubo-vertice},
  \begin{figure}[htbp]
    \centering
    \pgfimage{images/cubo-vertice}
    \caption{Un cubo visto desde un vértice}
    \label{fig:cubo-vertice}
  \end{figure}
  que muestra las tres caras vistas desde un vértice.
  Está claro que la simetría es \(\mathtt{C}_3\),
  un triángulo equilátero rotando en el plano
  (consideramos los vértices del triángulo
   como las caras a ser numeradas).
  Para este grupo el índice de ciclos es:
  \begin{equation*}
    \zeta_{\mathtt{C}_3}(x_1, x_2, x_3)
      = \frac{1}{3} \,
	  \left(
	    x_1^3 + 2 x_3
	  \right)
  \end{equation*}
  Nos interesa colorear con tres colores,
  y que cada uno aparezca exactamente una vez,
  por Pólya:
  \begin{equation*}
    \left[ z_1 z_2 z_3 \right] \zeta_{\mathtt{C}_3}(z_1 + z_2 + z_3,
			      z_1^2 + z_2^2 + z_3^2,
			      z_1^3 + z_2^3 + z_3^3)
  \end{equation*}
  El único término en que entran los \(z_i\) en la primera potencia
  es el término de la identidad,
  y en él nos interesa cada uno en la primera potencia:
  \begin{align*}
    \left[ z_1 z_2 z_3 \right] \zeta_{\mathtt{C}_3}(z_1 + z_2 + z_3,
			      z_1^2 + z_2^2 + z_3^2,
			      z_1^3 + z_2^3 + z_3^3)
      &= \frac{1}{3} \, \left[
			  z_1 z_2 z_3
			\right] (z_1 + z_2 + z_3)^3 \\
      &= \frac{1}{3} \, \binom{3}{1\;1\;1} \\
      &= 2
  \end{align*}
  Hay dos maneras diferentes de numerar las caras de un cubo
  con los números uno a seis tal que caras opuestas sumen siete.

%%% Local Variables:
%%% mode: latex
%%% TeX-master: "clases"
%%% End:

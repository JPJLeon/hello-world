% asintotica-palabras-sin-patron.tex
%
% Copyright (c) 2013-2014 Horst H. von Brand
% Derechos reservados. Vea COPYRIGHT para detalles

\subsection{Número de palabras sin $k$ símbolos repetidos}
\label{sec:asymptotics-words-no-pattern}
\index{palabra!asintotica@asintótica}

  Vimos en la sección~\ref{sec:strings-excluding-pattern}
  que el número de palabras de largo~\(n\)
  sobre un alfabeto de~\(s\) símbolos
  que no contienen el patrón~\(p\)
  de largo~\(k\)
  tiene función generatriz ordinaria~\eqref{eq:Bp-gf}:
  \begin{equation*}
    \frac{c_p(z)}{(1 - s z) c_p(z) + z^k}
  \end{equation*}
  Acá \(c_p(z)\)
  es el polinomio de autocorrelación del patrón~\(p\).
  Obtener una estimación asintótica de este número
  sirve de ejemplo detallado
  de la aplicación de las herramientas discutidas.
  El caso general es tratado por Guibas y Odlyzko~%
    \cite{guibas81:_strin_overl_patterns}.

  Para simplificar,
  trataremos únicamente el caso en que el patrón
  es un símbolo repetido~\(k\) veces,
  en cuyo caso \(c_p(z) = 1 + z + \dotsb + z^{k - 1}\).
  En este caso la función generatriz se reduce a:
  \begin{equation}
    \label{eq:Bak-gf}
    \frac{1 - z^k}{1 - s z + (s - 1) z^{k + 1}}
  \end{equation}
  Nos interesa el cero más cercano al origen
  del denominador de~\eqref{eq:Bak-gf} para \(k\) dado.
  Sólo tiene sentido el caso \(k > 1\).
  Llamemos:
  \begin{equation}
    \label{eq:h-definition}
    h(z)
      = 1 - s z + (s - 1) z^{k + 1}
  \end{equation}
  Claramente \(h(1) = 0\)
  con \(h'(1) = k (s - 1) - 1 > 0\),
  y \(h(1/s) = (s - 1) s^{-k - 1} > 0\) es pequeño.
  Podemos acotar:
  \begin{equation*}
    h((s - 1)^{-1})
      = 1 - s (s - 1)^{-1} + (s - 1)^{-k}
      \le 1 - s + 1
      = 2 - s
  \end{equation*}
  Hay un cero entre \(s^{-1}\) y \((s - 1)^{-1}\).
  Debemos asegurarnos que sea único.
  Para ello aplicamos el teorema de Rouché
  (\ref{theo:Rouche}).

  Consideremos el caso \(s > 2\).
  Tomemos las funciones:
  \begin{align*}
    f(z)
      &= -s z \\
    g(z)
      &= 1 + (s - 1) z^{k + 1}
  \end{align*}
  Sobre la circunferencia \(\lvert z \rvert = r\)
  tenemos las cotas:
  \begin{align*}
    \lvert f(z) \rvert
      &= s r \\
    \lvert g(z) \rvert
      &= 1 + (s - 1) r^{k + 1}
  \end{align*}
  Nos interesa demostrar que:
  \begin{align*}
    s r
      &> 1 + (s - 1) r^{k + 1} \\
    0 &> 1 - s r + (s - 1) r^{k + 1}
       = h(r)
  \end{align*}
  Como interesan \(k \ge 2\) y \(s \ge 2\),
  tenemos:
  \begin{align*}
    h(0)
      &= 1 \\
    h(1)
      &= 0 \\
    h'(1)
      &= - s + (k + 1) (s - 1)
       = k (s - 1) - 1
       > 0
  \end{align*}
  En consecuencia,
  hay \(0 < r^* < 1\) tal que \(h(r) < 0\),
  o,
  lo que es lo mismo,
  \(\lvert f(z) \rvert > \lvert g(z) \rvert\)
  sobre la circunferencia \(\lvert z \rvert = r^*\)
  Por el teorema de Rouché,
  \(f(z)\) y \(f(z) + g(z) = h(z)\) tienen el mismo número de ceros
  al interior de la circunferencia,
  uno solo.
  Como \(h(z)\) es un polinomio de coeficientes reales,
  sus ceros son reales o vienen en pares complejos.
  El cero que nos interesa es simple y real.
  Si lo llamamos \(\rho\),
  por el teorema de Bender
  (\ref{theo:Bender}):
  \begin{equation}
    \label{eq:asymptotic-strings-no-ak}
    P_n
      \sim \frac{1 - \rho^k}{h'(\rho)} \cdot \rho^n
      = \frac{1 - \rho^k}{(s - 1) (k + 1) \rho^k - s} \cdot \rho^n
  \end{equation}

  Tenemos cotas para \(\rho\),
  para obtener una mejor aproximación del cero
  aplicamos una iteración del método de Newton%
    \index{Newton, metodo de@Newton, método de}
  (para detalles,
   ver textos de análisis numérico,
   como Acton~%
     \cite[capítulo~2]{acton90:_numerical_methods_work}
   o Ralston y Rabinowitz~%
     \cite[capítulo~8]{ralston12:_first_cours_numer_analy})
  partiendo con la aproximación \(1 / s\).
  Resulta que la expresión final
  es mucho más sencilla si partimos con:
  \begin{equation*}
    h_r(u)
      = u^{k + 1} h(1 / u)
      = u^{k + 1} - s u^k + s - 1
  \end{equation*}
  Los ceros de \(h_r(u)\) son recíprocos de los ceros de \(h(z)\).
  Tenemos:
  \begin{equation*}
    (\rho^*)^{-1}
      \approx s - \frac{h_r(s)}{h_r'(s)}
      = s - \frac{s - 1}{s^k}
  \end{equation*}
  Para el peor caso posible,
  \(s = k = 2\),
  esto da la aproximación \(\rho^* \approx 0,571\),
  cuando el valor correcto es \(\rho = 1 / \tau = 0,618\).

%%% Local Variables:
%%% mode: latex
%%% TeX-master: "clases"
%%% End:

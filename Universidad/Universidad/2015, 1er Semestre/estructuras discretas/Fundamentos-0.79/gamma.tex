% gamma.tex
%
% Copyright (c) 2013-2014 Horst H. von Brand
% Derechos reservados. Vea COPYRIGHT para detalles

\section[La función \texorpdfstring{$\Gamma$}{gamma}]
	{\protect\boldmath
	    La función $\Gamma$
	 \protect\unboldmath}
\label{sec:gamma-function}
\index{\(\Gamma\)|textbfhy}

  Tendremos ocasión de usar la función \(\Gamma\)
  (gamma mayúscula),
  interesa deducir sus propiedades elementales.
  Para \(z\) complejo,
  se define:
  \begin{equation}
    \label{eq:Gamma-definition}
    \Gamma(z)
      = \int_0^\infty t^{z - 1} \mathrm{e}^{-t} \, \mathrm{d} t
  \end{equation}
  Para \(t > 0\) fijo el integrando de~\eqref{eq:Gamma-definition}
  es holomorfo en \(z\),
  con lo que esto define una función holomorfa en el semiplano
  \(\Re z > 1\).%
    \index{C (numeros complejos)@\(\mathbb{C}\) (números complejos)!funcion holomorfa@función holomorfa}
% Fixme: Completar
  Integrando por partes:
  \begin{equation*}
    \Gamma(z)
      = \int_0^\infty t^{z - 1} \mathrm{e}^{-t} \, \mathrm{d} t
      = \frac{1}{z}
	  \, \int_0^\infty \mathrm{e}^{-t} \, \mathrm{d} t^z
      = \left. \frac{1}{z} \, t^z \mathrm{e}^{-t} \right|_0^\infty
	  + \frac{1}{z}
	      \, \int_0^\infty t^z \mathrm{e}^{-t} \, \mathrm{d} t
      = \frac{1}{z} \, \Gamma(z + 1)
  \end{equation*}
  Tenemos así la \emph{fórmula de reducción}:
  \begin{equation}
    \label{eq:Gamma-reduction}
    \index{\(\Gamma\)!formula de reduccion@fórmula de reducción}
    \Gamma(z + 1)
      = z \, \Gamma(z)
  \end{equation}
  También tenemos \(\Gamma(1) = 1\),
  lo que para \(n \in \mathbb{N}\) por la fórmula de reducción da:
  \begin{equation}
    \label{eq:Gamma-factorial}
    \index{\(\Gamma\)!factorial}
    \Gamma(n)
      = (n - 1)!
  \end{equation}
  Incidentalmente,
  como para \(z \in - \mathbb{N}_0\)
  resulta \(1 / \Gamma(z) = 0\)
  es consistente nuestra convención~\eqref{eq:1/k!-convention}
  que \(1 / n! = 0\) si \(n\) es un entero negativo.

  Si \(\Re z > 1\),
  tenemos el valor de~\eqref{eq:Gamma-definition}.
  Fijemos entonces \(z\) con \(\Re z \le 1\),
  y sea \(n\) tal que \(\Re (z + n) > 0\).
  En un entorno de \(z + n\) la función \(\Gamma\) es holomorfa,
  y por~\eqref{eq:Gamma-reduction} tenemos:
  \begin{equation*}
    \Gamma(z + n)
      = z^{\underline{n + 1}} \, \Gamma(z)
  \end{equation*}
  Vale decir,
  si \(\Re z < 0\),
  tiene sentido definir con \(n = \lceil - \Re z \rceil\):%
    \index{potencia!factorial}
  \begin{equation}
    \label{eq:Gamma-reduction-n}
    \Gamma(z)
      = \frac{\Gamma(z + n)}{z^{\underline{n + 1}}}
  \end{equation}
  Es claro
  que esto nos mete en problemas solo si \(z \in -\mathbb{N}_0\).
  En el entorno de \(-n\),
  para \(\lvert w \rvert\) chico,
  la función queda representada por:
  \begin{equation*}
    \Gamma(- n + w)
      = \frac{\Gamma(1 + w)}{(-n + w)^{\underline{n + 1}}}
  \end{equation*}
  Como \(z^{\underline{n + 1}}\) tiene un cero simple en \(-n\),
  vemos que \(\Gamma(z)\)
  tiene polos simples en los enteros negativos.
  Es fácil calcular sus residuos:%
    \index{\(\Gamma\)!residuo}
  \begin{equation}
    \label{eq:Gamma-residues}
    \begin{split}
      \res(\Gamma, 0)
	&= \lim_{z \rightarrow 0} z \frac{\Gamma(z + 1)}{z}
	 = 1 \\
      \res(\Gamma, -n)
	&= \lim_{z \rightarrow -n} (z + n)
	     \frac{\Gamma(z + n + 1)}{z^{\underline{n + 1}}}
	 = \frac{1}{(-n)^{\underline{n}}}
	 = \frac{(-1)^n}{n!}
    \end{split}
  \end{equation}

  La función \(\Gamma\) satisface muchas identidades notables.
  Por ejemplo,
  tenemos:
  \begin{theorem}[Fórmula de reflexión de Euler]
    \index{Euler, formula de reflexion para \(\Gamma\)@Euler, fórmula de reflexión para \(\Gamma\)|see{\(\Gamma\)!fórmula de reflexión}}
    \index{\(\Gamma\)!formula de reflexion@fórmula de reflexión}
    Se cumple:
    \begin{equation}
      \label{eq:Gamma-reflection}
      \Gamma(z) \, \Gamma(1 - z)
	= \frac{\pi}{\sin \pi z}
      \end{equation}
  \end{theorem}
  Seguimos la demostración de Stein y~Shakarchi~%
    \cite{stein10:_compl_analy}.
  \begin{proof}
    Primeramente, \(\pi / \sin \pi z\)
    tiene polos en \(\mathbb{Z}\).
    El residuo en \(n \in \mathbb{Z}\) es:
    \begin{equation*}
      \res \left( \frac{\pi}{\sin \pi z}, n \right)
	= \frac{\pi}{\pi \cos \pi n}
	= (-1)^n
    \end{equation*}
    La función \(\Gamma(1 - z)\) tiene polos simples
    en \(z \in - \mathbb{N}\),
    allí \(\Gamma(z)\) es holomorfa.
    Para \(n \in \mathbb{N}\):
    \begin{equation*}
      \res(\Gamma(z) \, \Gamma(1 - z), - n)
	= \res(\Gamma(z), -n) \, \Gamma(n + 1)
	= \frac{(-1)^n}{n!} \, n!
	= (-1)^{-n}
    \end{equation*}
    De la misma forma,
    para \(n \in \mathbb{N}_0\):
    \begin{equation*}
      \res(\Gamma(z) \, \Gamma(1 - z), n)
	= (-1)^n
    \end{equation*}
    O sea,
    los polos y residuos respectivos de ambas funciones coinciden.

    Enseguida,
    \(\Gamma(z) \, \Gamma(1 - z)\)
    y \(\pi / \sin \pi z\) son ambas periódicas,
    con período 1:
    \begin{equation*}
      \Gamma(z + 1) \, \Gamma(1 - (z + 1))
	= z \Gamma(z) \cdot \frac{\Gamma(1 - z)}{z}
	= \Gamma(z) \, \Gamma(1 - z)
    \end{equation*}

    Finalmente,
    demostramos que ambas funciones coinciden en \(0 < s < 1\).
    Podemos escribir:
    \begin{equation*}
      \Gamma(1 - s)
	= \int_0^\infty u^{-s} \mathrm{e}^{-u} \, \mathrm{d} u
	= t \int_0^\infty \mathrm{e}^{-v t} (v t)^{-s}
	      \, \mathrm{d} v
    \end{equation*}
    Acá usamos el cambio de variables \(u = v t\).
    Luego:
    \begin{align*}
      \Gamma(s) \, \Gamma(1 - s)
	&= \int_0^\infty
	       \mathrm{e}^{-t} t^{s - 1} \, \Gamma(1 - s)
	     \, \mathrm{d} t \\
	&= \int_0^\infty
	     \mathrm{e}^{-t} t^{s - 1}
	       \left(
		 t \int_0^\infty \mathrm{e}^{- v t} (v t)^{-s}
		     \, \mathrm{d} v
	       \right) \, \mathrm{d} t \\
	&= \int_0^\infty
	     \int_0^\infty
	       \mathrm{e}^{-t (1 + v)} v^{-s} \,
		 \mathrm{d} v \, \mathrm{d} t \\
	&= \int_0^\infty \frac{v^{-s}}{1 + v} \, \mathrm{d} v
    \end{align*}
    Con el cambio de variables \(v = \mathrm{e}^t\)
    queda la integral que evaluamos
    en~\eqref{eq:integral-Gamma(z)Gamma(1-z)}:
    \begin{equation*}
      \int_{-\infty}^\infty
	\frac{\mathrm{e}^{- s t}}{1 + \mathrm{e}^t} \, \mathrm{d} t
	= \frac{\pi}{\sin s \pi}
    \end{equation*}

    Uniendo todas las piezas,
    ambas funciones deben ser iguales.
  \end{proof}
  De acá resulta directamente
  el valor para argumento no entero más usado:
  \begin{align}
    ( \Gamma( 1/2 ))^2
      &= \Gamma(1/2) \, \Gamma(1 - 1/2)
       = \frac{\pi}{\sin \pi / 2}
       = \pi \notag \\
    \Gamma(1/2)
      &= \sqrt{\pi} \label{eq:Gamma(1/2)}
  \end{align}

  Íntimamente relacionada es la función \(\mathrm{B}\)
  (beta mayúscula),%
    \index{\(\mathrm{B}\)|textbfhy}
  definida para \(\Re x, \Re y > 0\):
  \begin{equation}
    \label{eq:definition-Beta}
    \mathrm{B}(x, y)
      = \int_0^1 t^{x - 1} (1 - t)^{y - 1} \, \mathrm{d} t
  \end{equation}
  Es claro que es simétrica:
  \begin{equation}
    \label{eq:Beta-symmetry}
    \mathrm{B}(x, y)
      = \mathrm{B}(y, x)
  \end{equation}
  También:
  \begin{equation*}
    \Gamma(x) \, \Gamma(y)
      = \int_0^\infty \mathrm{e}^{-u} u^{x - 1} \, \mathrm{d} u
	  \int_0^\infty \mathrm{e}^{-v} v^{y - 1} \, \mathrm{d} v
      = \int_0^\infty \int_0^\infty
	  \mathrm{e}^{-u - v} u^{x - 1} v^{y - 1}
	     \, \mathrm{d} u \, \mathrm{d} v
  \end{equation*}
  El cambio de variables \(u = s t\) y \(v = s (1 - t)\) da:
  \begin{equation*}
    \Gamma(x) \, \Gamma(y)
      = \int_0^\infty \mathrm{e}^{-s} s^{x + y - 1} \, \mathrm{d} s
	  \int_0^1 t^{x - 1} (1 - t)^{y - 1} \, \mathrm{d} t
      = \Gamma(x + y) \mathrm{B}(x, y)
  \end{equation*}
  de donde resulta la identidad básica,
  que sirve para definir \(\mathrm{B}(x, y)\):
  \begin{equation}
    \label{eq:Gamma-Beta}
    \mathrm{B}(x, y)
      = \frac{\Gamma(x) \, \Gamma(y)}{\Gamma(x + y)}
  \end{equation}

  Por la fórmula de reducción~\eqref{eq:Gamma-reduction-n}%
    \index{\(\Gamma\)!formula de reduccion@fórmula de reducción}
  tenemos de~\eqref{eq:coeficiente-binomial}:%
    \index{coeficiente binomial}
  \begin{equation*}
    \binom{\alpha}{n}
      = \frac{\alpha^{\underline{n}}}{n!}
      = \frac{\Gamma(\alpha + 1)}{\Gamma(\alpha - n + 1) \, n!}
  \end{equation*}
  Con~\eqref{eq:Gamma-factorial}
  esto se parece a~\eqref{eq:coeficiente-binomial-factorial},
  que hace sentido adoptar como definición:
  \begin{equation}
    \label{eq:complex-binomial-coefficient}
    \binom{m + n}{m}
      = \frac{(m + n)!}{m! \, n!}
      = \frac{\Gamma(m + n + 1)}{\Gamma(m + 1) \, \Gamma(n + 1)}
  \end{equation}

%%% Local Variables:
%%% mode: latex
%%% TeX-master: "clases"
%%% End:

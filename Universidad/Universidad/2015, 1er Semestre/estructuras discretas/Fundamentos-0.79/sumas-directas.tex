% sumas-directas.tex
%
% Copyright (c) 2009-2014 Horst H. von Brand
% Derechos reservados. Vea COPYRIGHT para detalles

\section{Descomposiciones}
\label{sec:descomposiciones}

  Consideremos nuevamente el grupo \(\mathtt{D}_8\),
  que vimos en la sección~\ref{sec:aritmetica-Zm},
  véase el cuadro~\ref{tab:D_8}.
  \begin{table}[htbp]
    \centering
    \renewcommand{\tabcolsep}{2pt}
    \begin{tabular}{>{\(}c<{\)}|*{8}{>{\(}c<{\)}}}
      \bullet & \iota & r_1 & r_2 & r_3 & f_v & f_h & f_d & f_c \\
      \hline
	\rule[-0.7ex]{0pt}{3ex}%
      \iota & \iota & r_1   & r_2   & r_3   & f_v   & f_h   & f_d   & f_c \\
      r_1   & r_1   & r_2   & r_3   & \iota & f_v   & f_h   & f_d   & f_c \\
      r_2   & r_2   & r_3   & \iota & r_1   & f_c   & f_d   & f_v   & f_h \\
      r_3   & r_3   & \iota & r_1   & r_2   & f_d   & f_c   & f_h   & f_v \\
      f_v   & f_v   & f_d   & f_h   & f_c   & \iota & r_2   & r_1   & r_3 \\
      f_h   & f_h   & f_c   & f_v   & f_d   & r_2   & \iota & r_3   & r_1 \\
      f_d   & f_d   & f_h   & f_c   & f_v   & r_3   & r_1   & \iota & r_2 \\
      f_c   & f_c   & f_v   & f_d   & f_h   & r_1   & r_3   & r_2   & \iota
    \end{tabular}
    \caption{El grupo $\mathtt{D}_8$}
    \label{tab:D_8}
  \end{table}
  Si analizamos las operaciones que lo componen,
  vemos que las operaciones \(\{\iota, r_1, r_2, r_3\}\)
  por sí solas también conforman un grupo
  (corresponden a solo girar el cuadrado en el plano,
   sin salir de él),
  o sea forman un subgrupo de \(\mathtt{D}_8\).
  Otros subgrupos están formados por \(\iota\) solo
  (el grupo trivial,
   nuevamente),
  \(\{\iota, r_2\}\),
  \(\{\iota, f_d\}\).

% Fixme: Otros ejemplos de grupo: Z_4, Klein's 4-group,
%	 Z*_9, Z*_{15}, S_3, A_4

  Un ejemplo más simple
  (porque es un grupo abeliano)
  lo da \(\mathbb{Z}_{12}\) con la suma,
  véase el cuadro~\ref{tab:Z12}.
  En \(\mathbb{Z}_{12}\) son subgrupos
  \(\{0\}\),
  \(\{0, 6\}\),
  \(\{0, 4, 8\}\),
  \(\{0, 3, 6, 9\}\),
  \(\{0, 2, 4, 6, 8, 10\}\)
  y \(\{0, 1, 2, 3, 4, 5, 6, 7, 8, 9, 10, 11\}\).

\subsection{Homomorfismos e isomorfismos}
\label{sec:homomorfismos-isomorfismos}
\index{homomorfismo}
\index{isomorfismo}

  Consideremos los grupos \(\mathbb{Z}^\times_8\) y \(\mathbb{Z}^\times_{12}\),
  que casualmente tienen el mismo número de elementos.
  \begin{table}[htbp]
    \centering
    \subfloat[\(\mathbb{Z}^\times_8\)]{
      \renewcommand{\tabcolsep}{3pt}
      \begin{tabular}{>{\(}r<{\)}|*{4}{>{\(}r<{\)}}}
	\multicolumn{1}{c|}{\(\cdot\)}
	       & \phantom{0}1
		      & \phantom{0}3
			    & \phantom{0}5
				  & \phantom{0}7 \\
	\hline
	  \rule[-0.7ex]{0pt}{3ex}%
	      1 &   1 &	  3 &	5 &   7	 \\
	      3 &   3 &	  1 &	7 &   5	 \\
	      5 &   5 &	  7 &	1 &   3	 \\
	      7 &   7 &	  5 &	3 &   1	 \\
      \end{tabular}
      \label{subtab:Z8*}
    }%
    \hspace*{3em}%
    \subfloat[\(\mathbb{Z}^\times_{12}\)]{
      \renewcommand{\tabcolsep}{3pt}
      \begin{tabular}{>{\(}r<{\)}|*{4}{>{\(}r<{\)}}}
	\multicolumn{1}{c|}{\(\cdot\)} &
		1 &   5 &   7 &	 11 \\
	\hline
	  \rule[-0.7ex]{0pt}{3ex}%
		1 &   1 &   5 &	  7 &  11  \\
		5 &   5 &   1 &	 11 &	7  \\
		7 &   7 &  11 &	  1 &	5  \\
	       11 &  11 &   7 &	  5 &	1  \\
      \end{tabular}
      \label{subtab:Z12*}
    }
    \caption{Los grupos $\mathbb{Z}^\times_8$ y $\mathbb{Z}^\times_{12}$}
    \label{tab:Z8*+Z12*}
  \end{table}
  Estas tablas
  (cuadros~\ref{subtab:Z8*} y~\ref{subtab:Z12*})
  son diferentes,
  pero podemos ver que tienen la misma estructura,
  por ejemplo el mapa
  \begin{center}
    \begin{tabular}{>{\(}r<{\)}@{$\leftrightarrow$}>{\(}r<{\)}}
      1 &  1 \\
      3 &  5 \\
      5 &  7 \\
      7 & 11
    \end{tabular}
  \end{center}
  traduce entre ellos.
  Sin embargo,
  hay grupos diferentes con cuatro elementos.
  Por ejemplo,
  tenemos \(\mathbb{Z}^\times_5\)
  (cuadro~\ref{subtab:Z5*})
  y \(\mathbb{Z}_4\)
  (cuadro~\ref{subtab:Z4}).
  \begin{table}[htbp]
    \centering
    \subfloat[\(\mathbb{Z}^\times_5\)]{
      \renewcommand{\tabcolsep}{3pt}
      \begin{tabular}{>{\(}r<{\)}|*{4}{>{\(}r<{\)}}}
	\multicolumn{1}{c|}{\(\cdot\)} &
		1 &   2 &   3 &	  4 \\
	\hline
	  \rule[-0.7ex]{0pt}{3ex}%
	  1 &	1 &   2 &   3 &	  4 \\
	  2 &	2 &   4 &   1 &	  3 \\
	  3 &	3 &   1 &   4 &	  2 \\
	  4 &	4 &   3 &   2 &	  1 \\
      \end{tabular}
      \label{subtab:Z5*}
    }
    \hspace*{3em}
    \subfloat[\(\mathbb{Z}_4\)]{
      \renewcommand{\tabcolsep}{3pt}
      \begin{tabular}{>{\(}r<{\)}|*{4}{>{\(}r<{\)}}}
	\multicolumn{1}{c|}{\(+\)} &
		0 &   1 &   2 &	  3 \\
	\hline
	  \rule[-0.7ex]{0pt}{3ex}%
	  0 &	0 &   1 &   2 &	  3 \\
	  1 &	1 &   2 &   3 &	  0 \\
	  2 &	2 &   3 &   0 &	  1 \\
	  3 &	3 &   0 &   1 &	  2 \\
      \end{tabular}
      \label{subtab:Z4}
    }
    \caption{Los grupos $\mathbb{Z}^\times_5$ y $\mathbb{Z}_4$}
    \label{tab:Z5*+Z4}
  \end{table}
  Nótese que entre \(\mathbb{Z}_4\) y \(\mathbb{Z}^\times_5\)
  también podemos construir una correspondencia,
  a pesar que la operación involucrada es diferente:
  \begin{center}
    \begin{tabular}{>{\(}r<{\)}@{$\leftrightarrow$}>{\(}r<{\)}}
      0 & 1 \\
      1 & 2 \\
      2 & 4 \\
      3 & 3
    \end{tabular}
  \end{center}
  No hay correspondencia posible
  entre \(\mathbb{Z}^\times_5\) y \(\mathbb{Z}^\times_8\):
  En la diagonal
  de la tabla para \(\mathbb{Z}^\times_5\)
  (cuadro~\ref{subtab:Z5*})
  aparecen dos valores diferentes,
  mientras para \(\mathbb{Z}^\times_8\)
  (cuadro~\ref{subtab:Z8*})
  hay uno solo.

  Esta idea de ``misma estructura'' es importante,
  y la capturamos con lo siguiente.
  \begin{definition}
    \label{def:group-isomorphism}
    Sean dos grupos \((G, 1_G, \odot)\) y \((H, 1_H, \otimes)\),
    un \emph{homomorfismo} de \(G\) a \(H\)%
      \index{grupo!homomorfismo|textbfhy}%
      \index{homomorfismo}
    es una función \(h \colon G \rightarrow H\)
    tal que \(h(a \odot b) = h(a) \otimes h(b)\).
    A un homomorfismo que es una biyección se le llama \emph{isomorfismo},%
      \index{grupo!isomorfismo|textbfhy}%
      \index{isomorfismo}
    y se dice en tal caso que los grupos son \emph{isomorfos},
    y se anota \(G \cong H\).
    Un caso importante de isomorfismos son los isomorfismos
    de \(G\) a~\(G\),
    los \emph{automorfismos}.%
      \index{grupo!automorfismo|textbfhy}%
      \index{automorfismo}
  \end{definition}
  Las mismas ideas son aplicables a otras estructuras algebraicas,
  como anillos,
  si la función es homomorfismo
  (o isomorfismo)
  para ambas operaciones.%
    \index{anillo!homomorfismo}%
    \index{anillo!isomorfismo}%
    \index{anillo!automorfismo}

  Un ejemplo conocido de homomorfismo
  es la clasificación de números en pares e impares,
  con las correspondientes reglas de sumas y productos.
  Un isomorfismo útil es el entre \((\mathbb{R}^+, \cdot)\)
  y \((\mathbb{R}, +)\) dado por los logaritmos.

% Fixme: Ejemplos y/o ejercicios varios de grupos, homomorfismos
%	 (Z --> Z_n, ...), isomorfismos (R^+ --> R vía \ln, ...)

  Si \(h \colon G \rightarrow H\) es un homomorfismo,
  y \(1_G\) y \(1_H\) son los elementos neutros de \(G\) y \(H\),
  respectivamente,
  claramente \(h(1_G) = 1_H\),
  y \(h(a^{-1}) = (h(a))^{-1}\).

  Una manera simple de entender un isomorfismo es considerando
  que los dos grupos ``son el mismo'',
  solo cambiando los nombres de los elementos y la operación.
  Es fácil demostrar que los grupos cíclicos finitos de orden \(n\)
  son isomorfos a \(\mathbb{Z}_n\),
  y los infinitos isomorfos a \(\mathbb{Z}\).

  En \(\mathbb{Z}_p\) para \(p\) primo
  hay automorfismos que asocian \(1\) con cada elemento no cero.
  Esto no es más que otra forma de decir que módulo \(p\)
  todos los elementos tienen inverso.

  El isomorfismo entre grupos es una relación de equivalencia:%
    \index{relacion de equivalencia@relación de equivalencia}
  Es reflexiva,
  un grupo es isomorfo a sí mismo;
  es simétrica,
  ya si hay una biyección como la indicada,
  existe la función inversa que cumple las mismas condiciones;
  y es transitiva,
  siendo la composición de los isomorfismos el isomorfismo buscado.
  Es por ser una equivalencia que tiene sentido considerar ``iguales''
  estructuras algebraicas isomorfas.

  Una aplicación es la \emph{prueba de los nueves},
  popular cuando operaciones aritméticas se hacen manualmente.
  Consiste en verificar operaciones aritméticas
  (sumas, restas y multiplicaciones)
  vía calcular el residuo módulo nueve de los operandos,
  operar con los residuos,
  y comparar con el residuo módulo nueve del resultado.
  El punto es que
  (por el teorema~\ref{theo:mod-rules})
  el reducir módulo \(m\) es un homomorfismo
  del anillo \(\mathbb{Z}\) a \(\mathbb{Z}_m\),%
    \index{anillo!homomorfismo}
  por lo que ambos residuos debieran coincidir.
  Calcular el residuo módulo nueve de un número escrito en decimal
  es simplemente sumar sus dígitos
  hasta llegar a un resultado de un único dígito:
  Como \(10 \equiv 1 \pmod{9}\),
  tenemos:
  \begin{equation*}
    \sum_{0 \le k \le n} d_k \cdot 10^k
      \equiv \sum_{0 \le k \le n} d_k \pmod{9}
  \end{equation*}
  Demostramos un resultado similar en el lema~\ref{lem:9|10^k-1}.

\subsection{Sumas directas}
\label{sec:sumas-directas}

  En lo que sigue discutiremos grupos abelianos,
  pero la operación que interesa puntualmente
  es la multiplicación entre enteros.
  Para evitar confusiones,
  usaremos notación de multiplicación y potencias,
  y no sumas como sería por convención general.
  Por lo demás,
  la notación como multiplicación es más compacta.

  Siempre es útil tratar de descomponer estructuras complejas
  en piezas más simples.
  Consideremos
  el grupo de unidades \(\mathbb{Z}^\times_8 = \{1, 3, 5, 7\}\)
  y dos de sus subgrupos,
  \(\{1, 3\}\) y \(\{1, 5\}\).
  Todo elemento de \(\mathbb{Z}^\times_8\)
  puede escribirse como un producto de un elemento de cada uno de estos:
  \begin{alignat*}{2}
    1 &= 1 \cdot 1 &\qquad& 5 = 1 \cdot 5 \\
    3 &= 3 \cdot 1 &&	    7 = 3 \cdot 5
  \end{alignat*}
  Otro ejemplo
  provee \(\mathbb{Z}^\times_{15} = \{1, 2, 4, 7, 8, 11, 13, 14\}\),
  con subgrupos \(\{1, 2, 4, 8\}\) y \(\{1, 11\}\):
  \begin{alignat*}{2}
     1 &= 1 \cdot \phantom{0}1
       &\qquad	8 &= 8 \cdot \phantom{0}1 \\
     2 &= 2 \cdot \phantom{0}1
       &       11 &= 1 \cdot 11 \\
     4 &= 4 \cdot \phantom{0}1
       &       13 &= 8 \cdot 11 \\
     7 &= 2 \cdot 11
       &       14 &= 4 \cdot 11
  \end{alignat*}
  Esto motiva la siguiente:
  \begin{definition}
    \label{def:suma-directa}
    Sean \(A\) y \(B\) subgrupos del grupo abeliano \(G\)
    tales que todo \(g \in G\) puede escribirse de forma única
    como \(g = a \cdot b\),
    con \(a \in A\) y \(b \in B\).
    Entonces escribimos \(G = A B\)
    y decimos que \(G\) es la \emph{suma directa} de \(A\) y \(B\).
  \end{definition}
  La utilidad de esta noción se debe en buena parte
  a que si sabemos qué son \(A\) y \(B\)
  conocemos \(A B\):
  \begin{theorem}
    \label{theo:suma-directa-isomorfos}
    \index{grupo!isomorfismo}
    Si \(G = A B\),
    \(G' = A' B'\)
    y \(A \cong A'\), \(B \cong B'\),
    entonces \(G \cong G'\).
  \end{theorem}
  \begin{proof}
    Supongamos que
    \(f \colon A \rightarrow A'\) y \(h \colon B \rightarrow B'\)
    son isomorfismos,
    construimos un isomorfismo \(k \colon G \rightarrow G'\)
    definiendo:
    \begin{equation*}
      k(g) = f(a) \cdot h(b)
    \end{equation*}
    donde \(a \in A\), \(b \in B\) y \(g = a \cdot b\).
    Primeramente,
    esta definición tiene sentido,
    ya que para \(g \in G\) los elementos \(a\) y \(b\) son únicos,
    con lo que \(k\) es una función.
    Es uno a uno,
    ya que si tomamos \(g_1 \ne g_2\),
    al escribir \(g_1 = a_1 \cdot b_1\) y \(g_2 = a_2 \cdot b_2\),
    necesariamente estos pares son diferentes,
    y como \(f\) y \(h\) son uno a uno,
    tendremos
    \(k(g_1) = f(a_1) \cdot h(b_1) \ne f(a_2) \cdot h(b_2) = k(g_2)\).
    Es sobre ya que si tomamos \(g' \in G'\),
    este puede escribirse de forma única como \(g' = a' \cdot b'\),
    y usando las inversas de \(f\) y \(h\)
    esto lleva al elemento único
    \(g = f^{-1}(a') \cdot h^{-1}(b') \in G\)
    tal que \(k(g) = g'\).
  \end{proof}
  Este enredo oculta algo muy simple:
  Si \(G = A B\),
  se puede expresar \(g \in G\) mediante las ``coordenadas''
  \((a, b)\) con \(g = a \cdot b\),
  y considerar \(A B\) como \(A \times B\) con operación
  \((a, b) \cdot (a', b') = (a \cdot a', b \cdot b')\).
  En estos términos,
  la operación en \(A B\) está completamente determinada
  por las operaciones en \(A\) y \(B\);
  si \(A'\) es una copia de \(A\)
  y \(B'\) es una copia de \(B\),
  entonces \(G' = A' B'\)
  es simplemente una copia de \(G = A B\).

  Analicemos \(\mathbb{Z}^\times_{15}\) y \(\mathbb{Z}^\times_{16}\).
  Ya vimos que \(\mathbb{Z}^\times_{15} = \{1, 2, 4, 8\} \{1, 11\}\);
  mientras \(\mathbb{Z}^\times_{16} = \{1, 3, 5, 7, 9, 11, 13, 15\}\),
  con subgrupos \(\{1, 3, 9, 11\}\) y \(\{1, 7\}\),
  y tenemos \(\mathbb{Z}^\times_{16} = \{1, 3, 9, 11\} \{1, 7\}\).
  Pero \(\{1, 2, 4, 8\}\) y \(\{1, 3, 9, 11\}\)
  son grupos cíclicos de orden \(4\),
  y por tanto isomorfos a \(\mathbb{Z}_4\);
  y por el otro lado \(\{1, 11\}\) y \(\{1, 7\}\) son cíclicos de orden 2,
  isomorfos a \(\mathbb{Z}_2\).
  Entonces \(\mathbb{Z}^\times_{15} \cong \mathbb{Z}^\times_{16}\).

  Para cálculos concretos el siguiente teorema es útil:
  \begin{theorem}
    \label{theo:subgrupo-suma-directa}
    Si \(A\) y \(B\) son subgrupos del grupo abeliano \(G\)
    tales que \(A \cap B = \{1\}\)
    y \(\lvert A \rvert \cdot \lvert B \rvert = \lvert G \rvert\)
    entonces \(G = A B\).
  \end{theorem}
  \begin{proof}
    Consideremos los productos \(a b\)
    con \(a \in A\) y \(b \in B\).
    Demostramos que son diferentes por contradicción.
    Supongamos pares distintos \((a_1, b_1)\) y \((a_2, b_2)\)
    tales que \(a_1 \odot b_1 = a_2 \odot b_2\).
    Entonces \(a_1 \odot a_2^{-1} = b_1^{-1} \odot b_2\).
    Pero \(a_1 \odot a_2^{-1} \in A\)
    y \(b_1 \odot b_2^{-1} \in B\),
    con lo que esto tiene que estar en la intersección entre ambos,
    o sea \(a_1 \odot a_2^{-1} = b_1 \odot b_2^{-1} = 1\),
    con lo que \(a_1 = a_2\) y \(b_1 = b_2\).

    Con esto hay exactamente
      \(\lvert A \rvert \cdot \lvert B \rvert = \lvert G \rvert\)
    productos \(a \odot b\) diferentes,
    que tienen que ser todos los elementos de \(G\).
  \end{proof}
  El grupo \(\mathbb{Z}^\times_{16}\) tiene \(8 = 4 \cdot 2\) elementos,
  con lo que de los subgrupos \(\{1, 3, 9, 11\}\) y \(\{1, 7\}\)
  tenemos \(\mathbb{Z}^\times_{16} = \{1, 3, 9, 11\} \{1, 7\}\),
  ya que \(\{1, 3, 9, 11\} \cap \{1, 7\} = \{1\}\).

  Esto puede extenderse a más de dos subgrupos.
  Por ejemplo,
  \(\mathbb{Z}_{30}\)
  tiene subgrupos  \(\{0, 6, 12, 18, 24\}\) y \(\{0, 5, 10, 15, 20, 25\}\),
  de órdenes \(5\) y \(6\),
  con intersección \(\{0\}\).
  Por el teorema~\ref{theo:subgrupo-suma-directa}
  tenemos la descomposición
  \(\mathbb{Z}_{30} = \{0, 6, 12, 18, 24\} \{0, 5, 10, 15, 20, 25\}\).
  Por su lado,
  \(\{0, 5, 10, 15, 20, 25\}\) tiene subgrupos
  \(\{0, 10, 20\}\) y \(\{0, 15\}\),
  de órdenes \(3\) y \(2\),
  y es
  \(\{0, 5, 10, 15, 20, 25\} = \{0, 10, 20\} \{0, 15\}\).
  Esto sugiere extender la definición~\ref{def:suma-directa} y escribir
  \(\mathbb{Z}_{30} = \{0, 6, 12, 18, 24\}
       \{0, 10, 20\} \{0, 15\}\).
  \begin{definition}
    \label{def:suma-directa-n}
    Sea \(G\) un grupo abeliano,
    y sean \(A_1\), \(A_2\), \ldots, \(A_n\) subgrupos de \(G\)
    tales que todo elemento de \(G\) puede escribirse de forma única
    como \(a_1 \cdot a_2 \dotsm a_n\),
    con \(a_i \in A_i\) para todo \(1 \le i \le n\).
    Entonces \(G\) es la \emph{suma directa}
    de los subgrupos \(A_1\), \(A_2\), \ldots, \(A_n\),
    y anotamos \(G = A_1 A_2 \dotsb A_n\).
  \end{definition}
  Si \(G = A_1 A_2 \dotsb A_n\)
  y \(g = a_1 a_2 \dotso a_n\) con \(a_i \in A_i\)
  decimos que \(a_i\) es el \emph{componente} de \(g\) en \(A_i\).
  Por la definición de suma directa
  el componente de \(g\) en \(A_i\) es único.
  Una relación útil entre el orden del elemento
  y los órdenes de sus componentes es la siguiente:
  \begin{theorem}
    \label{theo:ordenes-suma-directa}
    Si \(G = A_1 A_2 \dotsb A_n\) y \(g \in G\),
    entonces el orden de \(g\)
    es el mínimo común múltiplo de los órdenes de los componentes de \(g\)
  \end{theorem}
  \begin{proof}
    Sea \(g = a_1 \cdot a_2 \dotsm a_n\) con \(a_i \in A_i\).
    Para cualquier entero \(s\)
    tendremos \(g^s = a_1^s \cdot a_2^s \dotsm a_n^s\).
    Como \(a_i^s \in A_i\),
    el componente en \(A_i\) de \(g^s\) es \(a_i^s\).
    Por otro lado,
    el componente de \(1\) en \(A_i\) es \(1\),
    y \(g^s = 1\) solo si \(a_i^s = 1\) para todo \(1 \le i \le n\),
    con lo que \(s\) es un múltiplo del orden de \(a_i\)
    para cada \(1 \le i \le n\),
    y el orden de \(g\) es el menor de todos los posibles múltiplos.
  \end{proof}

  Para ilustrar lo anterior,
  consideremos
  \(\mathbb{Z}^\times_{21} = \{1, 4, 16\} \{1, 8\} \{1, 13\}\).
  Si tomamos \(11 \in \mathbb{Z}^\times_{21}\),
  se descompone en \(11 = 4 \cdot 8 \cdot 1\).
  Las potencias respectivas las da el cuadro~\ref{tab:potencias-Z21*},
  \begin{table}[htbp]
    \centering
    \begin{tabular}{>{\(}l<{\)}*{3}{@{\qquad}>{\(}l<{\)}}}
      11		  & 4		       & 8	     & 1 \\
      11^2 =	       16 & 4^2 =	  16   & 8^2 = 1     &	 \\
      11^3 = \phantom{0}8 & 4^3 = \phantom{0}1 &	     &	 \\
      11^4 = \phantom{0}4 &		       &	     &	 \\
      11^5 = \phantom{0}2 &		       &	     &	 \\
      11^6 = \phantom{0}1 &		       &	     &
    \end{tabular}
    \caption{Potencias en $\mathbb{Z}^\times_{21}$}
    \label{tab:potencias-Z21*}
  \end{table}
  lo que confirma que el orden de \(11\) es \(6 = 3 \cdot 2 \cdot 1\).

\subsection{Sumas directas externas}
\label{sec:sumas-directas-externas}

  Hasta acá hemos descompuesto un grupo en la suma directa de subgrupos.
  La pregunta inversa es si
  dados grupos \(A_1\), \(A_2\), \ldots, \(A_n\),
  podemos construir \(G\) con subgrupos \(H_1\), \(H_2\), \ldots, \(H_n\)
  tales que \(G = H_1 H_2 \dotsb H_n\)
  con \(A_i \cong H_i\) para todo \(1 \le i \le n\).
  La respuesta es afirmativa,
  y la construcción es muy simple.
  Vimos que si \(G = H_1 H_2 \dotsb H_n\),
  entonces \(g \in G\)
  puede escribirse \(g = h_1 h_2 \dotsm h_n\) en forma única,
  con \(h_i \in H_i\).
  Especificar \(g\) es lo mismo que especificar la tupla de coordenadas
  \(h_i\).
  De igual manera,
  dado \(k \in G\)
  podemos escribirlo \(k = k_1 k_2 \dotsm k_n\) en forma única,
  con \(k_i \in H_i\),
  y \(g k = h_1 h_2 \dotso h_n \cdot k_1 k_2 \dotsm k_n
	 = (h_1 k_1) (h_2 k_2) \dotsm (h_n k_n)\),
  donde \(h_i k_i \in H_i\) resulta ser la coordenada de \(g k\).
  Esta situación motiva la definición siguiente:
  \begin{definition}
    Sean \(A_1\), \(A_2\), \ldots, \(A_n\) grupos abelianos.
    La \emph{suma directa (externa)} de \(A_1\), \(A_2\), \ldots, \(A_n\)
    es el conjunto de tuplas \((a_1, a_2, \dotsc, a_n)\) con \(a_i \in A_i\)
    y operación dada por:
    \begin{equation*}
      (a_1, a_2, \dotsc, a_n) \cdot (b_1, b_2, \dotsc, b_n)
	= (a_1 \cdot b_1, a_2 \cdot b_2, \dotsc, a_n \cdot b_n)
    \end{equation*}
    Escribiremos \(G = A_1 \times A_2 \times \dotsb \times A_n\)
    para la suma directa externa de los grupos \(A_1\), \ldots, \(A_n\).
  \end{definition}
  De acá resulta:
  \begin{theorem}
    \label{theo:suma-directa-externa}
    La suma directa (externa)
    de los grupos abelianos \(A_1\), \(A_2\), \ldots, \(A_n\)
    es un grupo abeliano,
    \(G = H_1 H_2 \dotsb H_n\),
    donde \(H_i\) es el conjunto de tuplas
    de la forma \((1, \dotsc, 1, a_i, 1, \dotsc, 1)\)
    con \(a_i \in A_i\)
    (todas las componentes, salvo la \(i\)\nobreakdash-ésima, son 1).
    Además,
    \(H_i \cong A_i\) para todo \(1 \le i \le n\).
  \end{theorem}
  \begin{proof}
    Demostrar que \(G\) es un grupo abeliano es automático;
    hay que verificar
    que la operación es cerrada (inmediato de la definición),
    asociatividad (resulta directamente de la asociatividad en cada \(A_i\)),
    existencia de neutro (resulta ser \((1, 1, \dotsc, 1)\)),
    conmutatividad (directamente de cada \(A_i\))
    e inverso
    (el inverso de \((a_1, a_2, \dotsc, a_n)\)
     es \((a_1^{-1}, a_2^{-1}, \dotsc, a_n^{-1})\)).

    Podemos escribir un elemento \(g \in G\) como:
    \begin{equation*}
      (a_1, a_2, \dotsc, a_n)
	 = (a_1, 1, \dotsc, 1) \cdot
	   (1, a_2, \dotsc, 1) \dotsm
	   (1, 1, \dotsc, a_n)
    \end{equation*}
    Acá \((1, \dotsc, 1, a_i, 1, \dotsc, 1) \in H_i\),
    lo que puede hacerse de una única forma,
    y los \(H_i\) son subgrupos de \(G\).
    Resulta \(G = H_1 H_2 \dotsb H_n\),
    y \(f_i \colon H_i \rightarrow A_i\)
    que mapea \((1,  \dotsc, 1, a_i, 1, \dotsc, 1)\) a \(a_i\)
    es un isomorfismo.
  \end{proof}

  La noción de sumas directas externas
  da una notación conveniente para describir grupos abelianos.
  Por ejemplo,
  vimos \(\mathbb{Z}^\times_{15} = \{1, 2, 4, 8\} \{1, 11\}\);
  pero estos dos son grupos cíclicos de orden 4 y 2, respectivamente,
  con lo que \(\mathbb{Z}^\times_{15} \cong \mathbb{Z}_4 \times \mathbb{Z}_2\)
  dice todo lo que hay que saber sobre \(\mathbb{Z}^\times_{15}\).

\subsection{Comentarios finales}
\label{sec:Zm-comentarios}

  Temas relacionados con grupos,
  anillos y otras estructuras algebraicas
  profundizan bastante textos del área como Connell~%
    \cite{connell04:_elemen_abstr_linear_algeb}
  y Judson~%
    \cite{judson14:_abstr_algeb}.

  Lo que nosotros llamamos \(\mathbb{Z}_m\)
  se conoce formalmente como \(\mathbb{Z} / m \mathbb{Z}\).
  Para justificar esta notación,
  primeramente definimos:
  \begin{definition}
    Sea \(G\) un grupo.
    Un subgrupo \(N\) de \(G\)
    se dice \emph{normal}
    (se anota \(N \lhd G\))
    si para todo \(n \in N\) y \(g \in G\)
    tenemos \(g n g^{-1} \in N\).
  \end{definition}
  Los subgrupos de un grupo abeliano son siempre normales.

  En ciertas situaciones los cosets de un subgrupo
  se pueden dotar con una operación
  heredada del grupo \(G\) para dar un nuevo grupo,
  el \emph{grupo cociente} o \emph{factor}
  \begin{equation*}
    G / N
      = \{g N \colon g \in G\}
  \end{equation*}
  con operación
  \begin{equation*}
    (g N) \bullet (h N)
      = (g h) N
  \end{equation*}
  Esto solo funciona si \(N \lhd G\),
  en cuyo caso el mapa \(g \mapsto g N\)
  es un homomorfismo de \(G\) a \(G / N\).%
    \index{grupo!homomorfismo}

  Ahora bien,
  el coset \(m \mathbb{Z}\) es un subgrupo de \(\mathbb{Z}\),
  y es un subgrupo normal
  ya que todos los subgrupos de un grupo abeliano son normales.
  Vemos que \(a + m \mathbb{Z}\) es precisamente el conjunto
  \(r + m \mathbb{Z}\), donde \(r = a \mod m\),
  y la suma en \(\mathbb{Z} / m \mathbb{Z}\)
  es exactamente como la describimos en~\ref{sec:aritmetica-Zm}.

  Otra notación común es \(\mathbb{Z} / (m)\),
  usando la misma idea anterior
  pero describiendo el conjunto de los múltiplos de \(m\)
  como el ideal generado por \(m\),
  vale decir el conjunto \(\{ r m \colon r \in \mathbb{Z}\}\).
  Estudiaremos este importante concepto
  en la sección~\ref{sec:dominios-euclidianos}.

%%% Local Variables:
%%% mode: latex
%%% TeX-master: "clases"
%%% End:

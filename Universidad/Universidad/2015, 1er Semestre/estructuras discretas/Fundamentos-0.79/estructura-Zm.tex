% estructura-Zm.tex
%
% Copyright (c) 2012-2014 Horst H. von Brand
% Derechos reservados. Vea COPYRIGHT para detalles

\chapter{\protect\boldmath
	    Estructura \texorpdfstring{de $\mathbb{Z}_m$ y $\mathbb{Z}^\times_m$}
				      {algebraica de clases de congruencia}%
	  \protect\unboldmath}
\label{cha:estructura-Zm}
\index{anillo!residuos|textbfhy}

  En nuestras aplicaciones los anillos de residuos
  son las estructuras algebraicas más importantes.
  Estudiaremos brevemente en sus características principales.
  Suele resultar fructífero descomponer estructuras complejas
  en piezas más simples para ayudar a su análisis.
  Incursionaremos un poco en el área
  de analizar la estructura de grupos abelianos,%
    \index{grupo!abeliano}
  obteniendo algunos resultados de gran interés
  en áreas como la criptología.

% sumas-directas.tex
%
% Copyright (c) 2009-2014 Horst H. von Brand
% Derechos reservados. Vea COPYRIGHT para detalles

\section{Descomposiciones}
\label{sec:descomposiciones}

  Consideremos nuevamente el grupo \(\mathtt{D}_8\),
  que vimos en la sección~\ref{sec:aritmetica-Zm},
  véase el cuadro~\ref{tab:D_8}.
  \begin{table}[htbp]
    \centering
    \renewcommand{\tabcolsep}{2pt}
    \begin{tabular}{>{\(}c<{\)}|*{8}{>{\(}c<{\)}}}
      \bullet & \iota & r_1 & r_2 & r_3 & f_v & f_h & f_d & f_c \\
      \hline
	\rule[-0.7ex]{0pt}{3ex}%
      \iota & \iota & r_1   & r_2   & r_3   & f_v   & f_h   & f_d   & f_c \\
      r_1   & r_1   & r_2   & r_3   & \iota & f_v   & f_h   & f_d   & f_c \\
      r_2   & r_2   & r_3   & \iota & r_1   & f_c   & f_d   & f_v   & f_h \\
      r_3   & r_3   & \iota & r_1   & r_2   & f_d   & f_c   & f_h   & f_v \\
      f_v   & f_v   & f_d   & f_h   & f_c   & \iota & r_2   & r_1   & r_3 \\
      f_h   & f_h   & f_c   & f_v   & f_d   & r_2   & \iota & r_3   & r_1 \\
      f_d   & f_d   & f_h   & f_c   & f_v   & r_3   & r_1   & \iota & r_2 \\
      f_c   & f_c   & f_v   & f_d   & f_h   & r_1   & r_3   & r_2   & \iota
    \end{tabular}
    \caption{El grupo $\mathtt{D}_8$}
    \label{tab:D_8}
  \end{table}
  Si analizamos las operaciones que lo componen,
  vemos que las operaciones \(\{\iota, r_1, r_2, r_3\}\)
  por sí solas también conforman un grupo
  (corresponden a solo girar el cuadrado en el plano,
   sin salir de él),
  o sea forman un subgrupo de \(\mathtt{D}_8\).
  Otros subgrupos están formados por \(\iota\) solo
  (el grupo trivial,
   nuevamente),
  \(\{\iota, r_2\}\),
  \(\{\iota, f_d\}\).

% Fixme: Otros ejemplos de grupo: Z_4, Klein's 4-group,
%	 Z*_9, Z*_{15}, S_3, A_4

  Un ejemplo más simple
  (porque es un grupo abeliano)
  lo da \(\mathbb{Z}_{12}\) con la suma,
  véase el cuadro~\ref{tab:Z12}.
  En \(\mathbb{Z}_{12}\) son subgrupos
  \(\{0\}\),
  \(\{0, 6\}\),
  \(\{0, 4, 8\}\),
  \(\{0, 3, 6, 9\}\),
  \(\{0, 2, 4, 6, 8, 10\}\)
  y \(\{0, 1, 2, 3, 4, 5, 6, 7, 8, 9, 10, 11\}\).

\subsection{Homomorfismos e isomorfismos}
\label{sec:homomorfismos-isomorfismos}
\index{homomorfismo}
\index{isomorfismo}

  Consideremos los grupos \(\mathbb{Z}^\times_8\) y \(\mathbb{Z}^\times_{12}\),
  que casualmente tienen el mismo número de elementos.
  \begin{table}[htbp]
    \centering
    \subfloat[\(\mathbb{Z}^\times_8\)]{
      \renewcommand{\tabcolsep}{3pt}
      \begin{tabular}{>{\(}r<{\)}|*{4}{>{\(}r<{\)}}}
	\multicolumn{1}{c|}{\(\cdot\)}
	       & \phantom{0}1
		      & \phantom{0}3
			    & \phantom{0}5
				  & \phantom{0}7 \\
	\hline
	  \rule[-0.7ex]{0pt}{3ex}%
	      1 &   1 &	  3 &	5 &   7	 \\
	      3 &   3 &	  1 &	7 &   5	 \\
	      5 &   5 &	  7 &	1 &   3	 \\
	      7 &   7 &	  5 &	3 &   1	 \\
      \end{tabular}
      \label{subtab:Z8*}
    }%
    \hspace*{3em}%
    \subfloat[\(\mathbb{Z}^\times_{12}\)]{
      \renewcommand{\tabcolsep}{3pt}
      \begin{tabular}{>{\(}r<{\)}|*{4}{>{\(}r<{\)}}}
	\multicolumn{1}{c|}{\(\cdot\)} &
		1 &   5 &   7 &	 11 \\
	\hline
	  \rule[-0.7ex]{0pt}{3ex}%
		1 &   1 &   5 &	  7 &  11  \\
		5 &   5 &   1 &	 11 &	7  \\
		7 &   7 &  11 &	  1 &	5  \\
	       11 &  11 &   7 &	  5 &	1  \\
      \end{tabular}
      \label{subtab:Z12*}
    }
    \caption{Los grupos $\mathbb{Z}^\times_8$ y $\mathbb{Z}^\times_{12}$}
    \label{tab:Z8*+Z12*}
  \end{table}
  Estas tablas
  (cuadros~\ref{subtab:Z8*} y~\ref{subtab:Z12*})
  son diferentes,
  pero podemos ver que tienen la misma estructura,
  por ejemplo el mapa
  \begin{center}
    \begin{tabular}{>{\(}r<{\)}@{$\leftrightarrow$}>{\(}r<{\)}}
      1 &  1 \\
      3 &  5 \\
      5 &  7 \\
      7 & 11
    \end{tabular}
  \end{center}
  traduce entre ellos.
  Sin embargo,
  hay grupos diferentes con cuatro elementos.
  Por ejemplo,
  tenemos \(\mathbb{Z}^\times_5\)
  (cuadro~\ref{subtab:Z5*})
  y \(\mathbb{Z}_4\)
  (cuadro~\ref{subtab:Z4}).
  \begin{table}[htbp]
    \centering
    \subfloat[\(\mathbb{Z}^\times_5\)]{
      \renewcommand{\tabcolsep}{3pt}
      \begin{tabular}{>{\(}r<{\)}|*{4}{>{\(}r<{\)}}}
	\multicolumn{1}{c|}{\(\cdot\)} &
		1 &   2 &   3 &	  4 \\
	\hline
	  \rule[-0.7ex]{0pt}{3ex}%
	  1 &	1 &   2 &   3 &	  4 \\
	  2 &	2 &   4 &   1 &	  3 \\
	  3 &	3 &   1 &   4 &	  2 \\
	  4 &	4 &   3 &   2 &	  1 \\
      \end{tabular}
      \label{subtab:Z5*}
    }
    \hspace*{3em}
    \subfloat[\(\mathbb{Z}_4\)]{
      \renewcommand{\tabcolsep}{3pt}
      \begin{tabular}{>{\(}r<{\)}|*{4}{>{\(}r<{\)}}}
	\multicolumn{1}{c|}{\(+\)} &
		0 &   1 &   2 &	  3 \\
	\hline
	  \rule[-0.7ex]{0pt}{3ex}%
	  0 &	0 &   1 &   2 &	  3 \\
	  1 &	1 &   2 &   3 &	  0 \\
	  2 &	2 &   3 &   0 &	  1 \\
	  3 &	3 &   0 &   1 &	  2 \\
      \end{tabular}
      \label{subtab:Z4}
    }
    \caption{Los grupos $\mathbb{Z}^\times_5$ y $\mathbb{Z}_4$}
    \label{tab:Z5*+Z4}
  \end{table}
  Nótese que entre \(\mathbb{Z}_4\) y \(\mathbb{Z}^\times_5\)
  también podemos construir una correspondencia,
  a pesar que la operación involucrada es diferente:
  \begin{center}
    \begin{tabular}{>{\(}r<{\)}@{$\leftrightarrow$}>{\(}r<{\)}}
      0 & 1 \\
      1 & 2 \\
      2 & 4 \\
      3 & 3
    \end{tabular}
  \end{center}
  No hay correspondencia posible
  entre \(\mathbb{Z}^\times_5\) y \(\mathbb{Z}^\times_8\):
  En la diagonal
  de la tabla para \(\mathbb{Z}^\times_5\)
  (cuadro~\ref{subtab:Z5*})
  aparecen dos valores diferentes,
  mientras para \(\mathbb{Z}^\times_8\)
  (cuadro~\ref{subtab:Z8*})
  hay uno solo.

  Esta idea de ``misma estructura'' es importante,
  y la capturamos con lo siguiente.
  \begin{definition}
    \label{def:group-isomorphism}
    Sean dos grupos \((G, 1_G, \odot)\) y \((H, 1_H, \otimes)\),
    un \emph{homomorfismo} de \(G\) a \(H\)%
      \index{grupo!homomorfismo|textbfhy}%
      \index{homomorfismo}
    es una función \(h \colon G \rightarrow H\)
    tal que \(h(a \odot b) = h(a) \otimes h(b)\).
    A un homomorfismo que es una biyección se le llama \emph{isomorfismo},%
      \index{grupo!isomorfismo|textbfhy}%
      \index{isomorfismo}
    y se dice en tal caso que los grupos son \emph{isomorfos},
    y se anota \(G \cong H\).
    Un caso importante de isomorfismos son los isomorfismos
    de \(G\) a~\(G\),
    los \emph{automorfismos}.%
      \index{grupo!automorfismo|textbfhy}%
      \index{automorfismo}
  \end{definition}
  Las mismas ideas son aplicables a otras estructuras algebraicas,
  como anillos,
  si la función es homomorfismo
  (o isomorfismo)
  para ambas operaciones.%
    \index{anillo!homomorfismo}%
    \index{anillo!isomorfismo}%
    \index{anillo!automorfismo}

  Un ejemplo conocido de homomorfismo
  es la clasificación de números en pares e impares,
  con las correspondientes reglas de sumas y productos.
  Un isomorfismo útil es el entre \((\mathbb{R}^+, \cdot)\)
  y \((\mathbb{R}, +)\) dado por los logaritmos.

% Fixme: Ejemplos y/o ejercicios varios de grupos, homomorfismos
%	 (Z --> Z_n, ...), isomorfismos (R^+ --> R vía \ln, ...)

  Si \(h \colon G \rightarrow H\) es un homomorfismo,
  y \(1_G\) y \(1_H\) son los elementos neutros de \(G\) y \(H\),
  respectivamente,
  claramente \(h(1_G) = 1_H\),
  y \(h(a^{-1}) = (h(a))^{-1}\).

  Una manera simple de entender un isomorfismo es considerando
  que los dos grupos ``son el mismo'',
  solo cambiando los nombres de los elementos y la operación.
  Es fácil demostrar que los grupos cíclicos finitos de orden \(n\)
  son isomorfos a \(\mathbb{Z}_n\),
  y los infinitos isomorfos a \(\mathbb{Z}\).

  En \(\mathbb{Z}_p\) para \(p\) primo
  hay automorfismos que asocian \(1\) con cada elemento no cero.
  Esto no es más que otra forma de decir que módulo \(p\)
  todos los elementos tienen inverso.

  El isomorfismo entre grupos es una relación de equivalencia:%
    \index{relacion de equivalencia@relación de equivalencia}
  Es reflexiva,
  un grupo es isomorfo a sí mismo;
  es simétrica,
  ya si hay una biyección como la indicada,
  existe la función inversa que cumple las mismas condiciones;
  y es transitiva,
  siendo la composición de los isomorfismos el isomorfismo buscado.
  Es por ser una equivalencia que tiene sentido considerar ``iguales''
  estructuras algebraicas isomorfas.

  Una aplicación es la \emph{prueba de los nueves},
  popular cuando operaciones aritméticas se hacen manualmente.
  Consiste en verificar operaciones aritméticas
  (sumas, restas y multiplicaciones)
  vía calcular el residuo módulo nueve de los operandos,
  operar con los residuos,
  y comparar con el residuo módulo nueve del resultado.
  El punto es que
  (por el teorema~\ref{theo:mod-rules})
  el reducir módulo \(m\) es un homomorfismo
  del anillo \(\mathbb{Z}\) a \(\mathbb{Z}_m\),%
    \index{anillo!homomorfismo}
  por lo que ambos residuos debieran coincidir.
  Calcular el residuo módulo nueve de un número escrito en decimal
  es simplemente sumar sus dígitos
  hasta llegar a un resultado de un único dígito:
  Como \(10 \equiv 1 \pmod{9}\),
  tenemos:
  \begin{equation*}
    \sum_{0 \le k \le n} d_k \cdot 10^k
      \equiv \sum_{0 \le k \le n} d_k \pmod{9}
  \end{equation*}
  Demostramos un resultado similar en el lema~\ref{lem:9|10^k-1}.

\subsection{Sumas directas}
\label{sec:sumas-directas}

  En lo que sigue discutiremos grupos abelianos,
  pero la operación que interesa puntualmente
  es la multiplicación entre enteros.
  Para evitar confusiones,
  usaremos notación de multiplicación y potencias,
  y no sumas como sería por convención general.
  Por lo demás,
  la notación como multiplicación es más compacta.

  Siempre es útil tratar de descomponer estructuras complejas
  en piezas más simples.
  Consideremos
  el grupo de unidades \(\mathbb{Z}^\times_8 = \{1, 3, 5, 7\}\)
  y dos de sus subgrupos,
  \(\{1, 3\}\) y \(\{1, 5\}\).
  Todo elemento de \(\mathbb{Z}^\times_8\)
  puede escribirse como un producto de un elemento de cada uno de estos:
  \begin{alignat*}{2}
    1 &= 1 \cdot 1 &\qquad& 5 = 1 \cdot 5 \\
    3 &= 3 \cdot 1 &&	    7 = 3 \cdot 5
  \end{alignat*}
  Otro ejemplo
  provee \(\mathbb{Z}^\times_{15} = \{1, 2, 4, 7, 8, 11, 13, 14\}\),
  con subgrupos \(\{1, 2, 4, 8\}\) y \(\{1, 11\}\):
  \begin{alignat*}{2}
     1 &= 1 \cdot \phantom{0}1
       &\qquad	8 &= 8 \cdot \phantom{0}1 \\
     2 &= 2 \cdot \phantom{0}1
       &       11 &= 1 \cdot 11 \\
     4 &= 4 \cdot \phantom{0}1
       &       13 &= 8 \cdot 11 \\
     7 &= 2 \cdot 11
       &       14 &= 4 \cdot 11
  \end{alignat*}
  Esto motiva la siguiente:
  \begin{definition}
    \label{def:suma-directa}
    Sean \(A\) y \(B\) subgrupos del grupo abeliano \(G\)
    tales que todo \(g \in G\) puede escribirse de forma única
    como \(g = a \cdot b\),
    con \(a \in A\) y \(b \in B\).
    Entonces escribimos \(G = A B\)
    y decimos que \(G\) es la \emph{suma directa} de \(A\) y \(B\).
  \end{definition}
  La utilidad de esta noción se debe en buena parte
  a que si sabemos qué son \(A\) y \(B\)
  conocemos \(A B\):
  \begin{theorem}
    \label{theo:suma-directa-isomorfos}
    \index{grupo!isomorfismo}
    Si \(G = A B\),
    \(G' = A' B'\)
    y \(A \cong A'\), \(B \cong B'\),
    entonces \(G \cong G'\).
  \end{theorem}
  \begin{proof}
    Supongamos que
    \(f \colon A \rightarrow A'\) y \(h \colon B \rightarrow B'\)
    son isomorfismos,
    construimos un isomorfismo \(k \colon G \rightarrow G'\)
    definiendo:
    \begin{equation*}
      k(g) = f(a) \cdot h(b)
    \end{equation*}
    donde \(a \in A\), \(b \in B\) y \(g = a \cdot b\).
    Primeramente,
    esta definición tiene sentido,
    ya que para \(g \in G\) los elementos \(a\) y \(b\) son únicos,
    con lo que \(k\) es una función.
    Es uno a uno,
    ya que si tomamos \(g_1 \ne g_2\),
    al escribir \(g_1 = a_1 \cdot b_1\) y \(g_2 = a_2 \cdot b_2\),
    necesariamente estos pares son diferentes,
    y como \(f\) y \(h\) son uno a uno,
    tendremos
    \(k(g_1) = f(a_1) \cdot h(b_1) \ne f(a_2) \cdot h(b_2) = k(g_2)\).
    Es sobre ya que si tomamos \(g' \in G'\),
    este puede escribirse de forma única como \(g' = a' \cdot b'\),
    y usando las inversas de \(f\) y \(h\)
    esto lleva al elemento único
    \(g = f^{-1}(a') \cdot h^{-1}(b') \in G\)
    tal que \(k(g) = g'\).
  \end{proof}
  Este enredo oculta algo muy simple:
  Si \(G = A B\),
  se puede expresar \(g \in G\) mediante las ``coordenadas''
  \((a, b)\) con \(g = a \cdot b\),
  y considerar \(A B\) como \(A \times B\) con operación
  \((a, b) \cdot (a', b') = (a \cdot a', b \cdot b')\).
  En estos términos,
  la operación en \(A B\) está completamente determinada
  por las operaciones en \(A\) y \(B\);
  si \(A'\) es una copia de \(A\)
  y \(B'\) es una copia de \(B\),
  entonces \(G' = A' B'\)
  es simplemente una copia de \(G = A B\).

  Analicemos \(\mathbb{Z}^\times_{15}\) y \(\mathbb{Z}^\times_{16}\).
  Ya vimos que \(\mathbb{Z}^\times_{15} = \{1, 2, 4, 8\} \{1, 11\}\);
  mientras \(\mathbb{Z}^\times_{16} = \{1, 3, 5, 7, 9, 11, 13, 15\}\),
  con subgrupos \(\{1, 3, 9, 11\}\) y \(\{1, 7\}\),
  y tenemos \(\mathbb{Z}^\times_{16} = \{1, 3, 9, 11\} \{1, 7\}\).
  Pero \(\{1, 2, 4, 8\}\) y \(\{1, 3, 9, 11\}\)
  son grupos cíclicos de orden \(4\),
  y por tanto isomorfos a \(\mathbb{Z}_4\);
  y por el otro lado \(\{1, 11\}\) y \(\{1, 7\}\) son cíclicos de orden 2,
  isomorfos a \(\mathbb{Z}_2\).
  Entonces \(\mathbb{Z}^\times_{15} \cong \mathbb{Z}^\times_{16}\).

  Para cálculos concretos el siguiente teorema es útil:
  \begin{theorem}
    \label{theo:subgrupo-suma-directa}
    Si \(A\) y \(B\) son subgrupos del grupo abeliano \(G\)
    tales que \(A \cap B = \{1\}\)
    y \(\lvert A \rvert \cdot \lvert B \rvert = \lvert G \rvert\)
    entonces \(G = A B\).
  \end{theorem}
  \begin{proof}
    Consideremos los productos \(a b\)
    con \(a \in A\) y \(b \in B\).
    Demostramos que son diferentes por contradicción.
    Supongamos pares distintos \((a_1, b_1)\) y \((a_2, b_2)\)
    tales que \(a_1 \odot b_1 = a_2 \odot b_2\).
    Entonces \(a_1 \odot a_2^{-1} = b_1^{-1} \odot b_2\).
    Pero \(a_1 \odot a_2^{-1} \in A\)
    y \(b_1 \odot b_2^{-1} \in B\),
    con lo que esto tiene que estar en la intersección entre ambos,
    o sea \(a_1 \odot a_2^{-1} = b_1 \odot b_2^{-1} = 1\),
    con lo que \(a_1 = a_2\) y \(b_1 = b_2\).

    Con esto hay exactamente
      \(\lvert A \rvert \cdot \lvert B \rvert = \lvert G \rvert\)
    productos \(a \odot b\) diferentes,
    que tienen que ser todos los elementos de \(G\).
  \end{proof}
  El grupo \(\mathbb{Z}^\times_{16}\) tiene \(8 = 4 \cdot 2\) elementos,
  con lo que de los subgrupos \(\{1, 3, 9, 11\}\) y \(\{1, 7\}\)
  tenemos \(\mathbb{Z}^\times_{16} = \{1, 3, 9, 11\} \{1, 7\}\),
  ya que \(\{1, 3, 9, 11\} \cap \{1, 7\} = \{1\}\).

  Esto puede extenderse a más de dos subgrupos.
  Por ejemplo,
  \(\mathbb{Z}_{30}\)
  tiene subgrupos  \(\{0, 6, 12, 18, 24\}\) y \(\{0, 5, 10, 15, 20, 25\}\),
  de órdenes \(5\) y \(6\),
  con intersección \(\{0\}\).
  Por el teorema~\ref{theo:subgrupo-suma-directa}
  tenemos la descomposición
  \(\mathbb{Z}_{30} = \{0, 6, 12, 18, 24\} \{0, 5, 10, 15, 20, 25\}\).
  Por su lado,
  \(\{0, 5, 10, 15, 20, 25\}\) tiene subgrupos
  \(\{0, 10, 20\}\) y \(\{0, 15\}\),
  de órdenes \(3\) y \(2\),
  y es
  \(\{0, 5, 10, 15, 20, 25\} = \{0, 10, 20\} \{0, 15\}\).
  Esto sugiere extender la definición~\ref{def:suma-directa} y escribir
  \(\mathbb{Z}_{30} = \{0, 6, 12, 18, 24\}
       \{0, 10, 20\} \{0, 15\}\).
  \begin{definition}
    \label{def:suma-directa-n}
    Sea \(G\) un grupo abeliano,
    y sean \(A_1\), \(A_2\), \ldots, \(A_n\) subgrupos de \(G\)
    tales que todo elemento de \(G\) puede escribirse de forma única
    como \(a_1 \cdot a_2 \dotsm a_n\),
    con \(a_i \in A_i\) para todo \(1 \le i \le n\).
    Entonces \(G\) es la \emph{suma directa}
    de los subgrupos \(A_1\), \(A_2\), \ldots, \(A_n\),
    y anotamos \(G = A_1 A_2 \dotsb A_n\).
  \end{definition}
  Si \(G = A_1 A_2 \dotsb A_n\)
  y \(g = a_1 a_2 \dotso a_n\) con \(a_i \in A_i\)
  decimos que \(a_i\) es el \emph{componente} de \(g\) en \(A_i\).
  Por la definición de suma directa
  el componente de \(g\) en \(A_i\) es único.
  Una relación útil entre el orden del elemento
  y los órdenes de sus componentes es la siguiente:
  \begin{theorem}
    \label{theo:ordenes-suma-directa}
    Si \(G = A_1 A_2 \dotsb A_n\) y \(g \in G\),
    entonces el orden de \(g\)
    es el mínimo común múltiplo de los órdenes de los componentes de \(g\)
  \end{theorem}
  \begin{proof}
    Sea \(g = a_1 \cdot a_2 \dotsm a_n\) con \(a_i \in A_i\).
    Para cualquier entero \(s\)
    tendremos \(g^s = a_1^s \cdot a_2^s \dotsm a_n^s\).
    Como \(a_i^s \in A_i\),
    el componente en \(A_i\) de \(g^s\) es \(a_i^s\).
    Por otro lado,
    el componente de \(1\) en \(A_i\) es \(1\),
    y \(g^s = 1\) solo si \(a_i^s = 1\) para todo \(1 \le i \le n\),
    con lo que \(s\) es un múltiplo del orden de \(a_i\)
    para cada \(1 \le i \le n\),
    y el orden de \(g\) es el menor de todos los posibles múltiplos.
  \end{proof}

  Para ilustrar lo anterior,
  consideremos
  \(\mathbb{Z}^\times_{21} = \{1, 4, 16\} \{1, 8\} \{1, 13\}\).
  Si tomamos \(11 \in \mathbb{Z}^\times_{21}\),
  se descompone en \(11 = 4 \cdot 8 \cdot 1\).
  Las potencias respectivas las da el cuadro~\ref{tab:potencias-Z21*},
  \begin{table}[htbp]
    \centering
    \begin{tabular}{>{\(}l<{\)}*{3}{@{\qquad}>{\(}l<{\)}}}
      11		  & 4		       & 8	     & 1 \\
      11^2 =	       16 & 4^2 =	  16   & 8^2 = 1     &	 \\
      11^3 = \phantom{0}8 & 4^3 = \phantom{0}1 &	     &	 \\
      11^4 = \phantom{0}4 &		       &	     &	 \\
      11^5 = \phantom{0}2 &		       &	     &	 \\
      11^6 = \phantom{0}1 &		       &	     &
    \end{tabular}
    \caption{Potencias en $\mathbb{Z}^\times_{21}$}
    \label{tab:potencias-Z21*}
  \end{table}
  lo que confirma que el orden de \(11\) es \(6 = 3 \cdot 2 \cdot 1\).

\subsection{Sumas directas externas}
\label{sec:sumas-directas-externas}

  Hasta acá hemos descompuesto un grupo en la suma directa de subgrupos.
  La pregunta inversa es si
  dados grupos \(A_1\), \(A_2\), \ldots, \(A_n\),
  podemos construir \(G\) con subgrupos \(H_1\), \(H_2\), \ldots, \(H_n\)
  tales que \(G = H_1 H_2 \dotsb H_n\)
  con \(A_i \cong H_i\) para todo \(1 \le i \le n\).
  La respuesta es afirmativa,
  y la construcción es muy simple.
  Vimos que si \(G = H_1 H_2 \dotsb H_n\),
  entonces \(g \in G\)
  puede escribirse \(g = h_1 h_2 \dotsm h_n\) en forma única,
  con \(h_i \in H_i\).
  Especificar \(g\) es lo mismo que especificar la tupla de coordenadas
  \(h_i\).
  De igual manera,
  dado \(k \in G\)
  podemos escribirlo \(k = k_1 k_2 \dotsm k_n\) en forma única,
  con \(k_i \in H_i\),
  y \(g k = h_1 h_2 \dotso h_n \cdot k_1 k_2 \dotsm k_n
	 = (h_1 k_1) (h_2 k_2) \dotsm (h_n k_n)\),
  donde \(h_i k_i \in H_i\) resulta ser la coordenada de \(g k\).
  Esta situación motiva la definición siguiente:
  \begin{definition}
    Sean \(A_1\), \(A_2\), \ldots, \(A_n\) grupos abelianos.
    La \emph{suma directa (externa)} de \(A_1\), \(A_2\), \ldots, \(A_n\)
    es el conjunto de tuplas \((a_1, a_2, \dotsc, a_n)\) con \(a_i \in A_i\)
    y operación dada por:
    \begin{equation*}
      (a_1, a_2, \dotsc, a_n) \cdot (b_1, b_2, \dotsc, b_n)
	= (a_1 \cdot b_1, a_2 \cdot b_2, \dotsc, a_n \cdot b_n)
    \end{equation*}
    Escribiremos \(G = A_1 \times A_2 \times \dotsb \times A_n\)
    para la suma directa externa de los grupos \(A_1\), \ldots, \(A_n\).
  \end{definition}
  De acá resulta:
  \begin{theorem}
    \label{theo:suma-directa-externa}
    La suma directa (externa)
    de los grupos abelianos \(A_1\), \(A_2\), \ldots, \(A_n\)
    es un grupo abeliano,
    \(G = H_1 H_2 \dotsb H_n\),
    donde \(H_i\) es el conjunto de tuplas
    de la forma \((1, \dotsc, 1, a_i, 1, \dotsc, 1)\)
    con \(a_i \in A_i\)
    (todas las componentes, salvo la \(i\)\nobreakdash-ésima, son 1).
    Además,
    \(H_i \cong A_i\) para todo \(1 \le i \le n\).
  \end{theorem}
  \begin{proof}
    Demostrar que \(G\) es un grupo abeliano es automático;
    hay que verificar
    que la operación es cerrada (inmediato de la definición),
    asociatividad (resulta directamente de la asociatividad en cada \(A_i\)),
    existencia de neutro (resulta ser \((1, 1, \dotsc, 1)\)),
    conmutatividad (directamente de cada \(A_i\))
    e inverso
    (el inverso de \((a_1, a_2, \dotsc, a_n)\)
     es \((a_1^{-1}, a_2^{-1}, \dotsc, a_n^{-1})\)).

    Podemos escribir un elemento \(g \in G\) como:
    \begin{equation*}
      (a_1, a_2, \dotsc, a_n)
	 = (a_1, 1, \dotsc, 1) \cdot
	   (1, a_2, \dotsc, 1) \dotsm
	   (1, 1, \dotsc, a_n)
    \end{equation*}
    Acá \((1, \dotsc, 1, a_i, 1, \dotsc, 1) \in H_i\),
    lo que puede hacerse de una única forma,
    y los \(H_i\) son subgrupos de \(G\).
    Resulta \(G = H_1 H_2 \dotsb H_n\),
    y \(f_i \colon H_i \rightarrow A_i\)
    que mapea \((1,  \dotsc, 1, a_i, 1, \dotsc, 1)\) a \(a_i\)
    es un isomorfismo.
  \end{proof}

  La noción de sumas directas externas
  da una notación conveniente para describir grupos abelianos.
  Por ejemplo,
  vimos \(\mathbb{Z}^\times_{15} = \{1, 2, 4, 8\} \{1, 11\}\);
  pero estos dos son grupos cíclicos de orden 4 y 2, respectivamente,
  con lo que \(\mathbb{Z}^\times_{15} \cong \mathbb{Z}_4 \times \mathbb{Z}_2\)
  dice todo lo que hay que saber sobre \(\mathbb{Z}^\times_{15}\).

\subsection{Comentarios finales}
\label{sec:Zm-comentarios}

  Temas relacionados con grupos,
  anillos y otras estructuras algebraicas
  profundizan bastante textos del área como Connell~%
    \cite{connell04:_elemen_abstr_linear_algeb}
  y Judson~%
    \cite{judson14:_abstr_algeb}.

  Lo que nosotros llamamos \(\mathbb{Z}_m\)
  se conoce formalmente como \(\mathbb{Z} / m \mathbb{Z}\).
  Para justificar esta notación,
  primeramente definimos:
  \begin{definition}
    Sea \(G\) un grupo.
    Un subgrupo \(N\) de \(G\)
    se dice \emph{normal}
    (se anota \(N \lhd G\))
    si para todo \(n \in N\) y \(g \in G\)
    tenemos \(g n g^{-1} \in N\).
  \end{definition}
  Los subgrupos de un grupo abeliano son siempre normales.

  En ciertas situaciones los cosets de un subgrupo
  se pueden dotar con una operación
  heredada del grupo \(G\) para dar un nuevo grupo,
  el \emph{grupo cociente} o \emph{factor}
  \begin{equation*}
    G / N
      = \{g N \colon g \in G\}
  \end{equation*}
  con operación
  \begin{equation*}
    (g N) \bullet (h N)
      = (g h) N
  \end{equation*}
  Esto solo funciona si \(N \lhd G\),
  en cuyo caso el mapa \(g \mapsto g N\)
  es un homomorfismo de \(G\) a \(G / N\).%
    \index{grupo!homomorfismo}

  Ahora bien,
  el coset \(m \mathbb{Z}\) es un subgrupo de \(\mathbb{Z}\),
  y es un subgrupo normal
  ya que todos los subgrupos de un grupo abeliano son normales.
  Vemos que \(a + m \mathbb{Z}\) es precisamente el conjunto
  \(r + m \mathbb{Z}\), donde \(r = a \mod m\),
  y la suma en \(\mathbb{Z} / m \mathbb{Z}\)
  es exactamente como la describimos en~\ref{sec:aritmetica-Zm}.

  Otra notación común es \(\mathbb{Z} / (m)\),
  usando la misma idea anterior
  pero describiendo el conjunto de los múltiplos de \(m\)
  como el ideal generado por \(m\),
  vale decir el conjunto \(\{ r m \colon r \in \mathbb{Z}\}\).
  Estudiaremos este importante concepto
  en la sección~\ref{sec:dominios-euclidianos}.

%%% Local Variables:
%%% mode: latex
%%% TeX-master: "clases"
%%% End:


  Nuestra primera tarea es descomponer \(\mathbb{Z}_m\)
  en forma más sistemática.
  Lo que hemos hecho hasta ahora es tomar elementos que se ven bien
  y considerar los subgrupos que generan,
    \index{grupo!subgrupo}
  tratando de encontrar intersecciones y órdenes adecuados.
  Acá veremos un método general
  para descomponer el grupo \(\mathbb{Z}_m\),
  y plantear el camino para entender mejor
  los grupos \(\mathbb{Z}^\times_m\).

  Por ejemplo,
  \(\mathbb{Z}_{30}
      = \{0, 6, 12, 18, 24\} \{0, 5, 10, 15, 20, 25\}\)
  o sea \(\mathbb{Z}_{30}\)
  es la suma directa de un grupo cíclico de orden \(5\)
  y otro de orden \(6\),
  y por el teorema~\ref{theo:suma-directa-isomorfos}
  esto es
    \(\mathbb{Z}_{30} \cong \mathbb{Z}_5 \times \mathbb{Z}_6\).
  Por otro lado,
  hay una función obvia
  \(f \colon \mathbb{Z}_{30}
     \rightarrow \mathbb{Z}_5 \times \mathbb{Z}_6\):
  Si conocemos un entero \(x\) módulo \(30\),
  sabemos sus residuos módulos \(5\) y \(6\).%
    \index{residuo}
  Por ejemplo,
  si \(x \equiv 13 \pmod{30}\),
  entonces \(x \equiv 13 \equiv 3 \pmod{5}\)
  y \(x \equiv 13 \equiv 1 \pmod{6}\).
  En este caso,
  tendríamos \(f(13) = (3, 1)\).
  Una tabla completa para \(f\) es:
  \begin{equation*}
    \begin{array}{*{5}{l}}
      f(0) = (0, 0) &  f(6) \phantom{0}= (1, 0)
		    & f(12) = (2, 0) & f(18) = (3, 0) &
		      f(24) = (4, 0) \\
      f(1) = (1, 1) &  f(7) \phantom{0}= (2, 1)
		    & f(13) = (3, 1) & f(19) = (4, 1) &
		      f(25) = (0, 1) \\
      f(2) = (2, 2) &  f(8) \phantom{0}= (3, 2)
		    & f(14) = (4, 2) & f(20) = (0, 2) &
		      f(26) = (1, 2) \\
      f(3) = (3, 3) &  f(9) \phantom{0}= (4, 3)
		    & f(15) = (0, 3) & f(21) = (1, 3) &
		      f(27) = (2, 3) \\
      f(4) = (4, 4) & f(10) = (0, 4) & f(16) = (1, 4) &
		      f(22) = (2, 4) &
		      f(28) = (3, 4) \\
      f(5) = (0, 5) & f(11) = (1, 5) & f(17) = (2, 5) &
		      f(23) = (3, 5) &
		      f(29) = (4, 5)
    \end{array}
  \end{equation*}
  El que \(f\) es uno a uno se ve directamente de la tabla.
  Las restantes propiedades son obvias si consideramos
  que cada entrada como
  \(f([x]_{30}) = ([x]_5, [x]_6)\).
  Tenemos el siguiente teorema general:
  \begin{theorem}
    \index{residuo!teorema chino de los!padre del}
    \label{theo:isomorfismo-anillo-Zm}
    Sea \(m = a b\) con \(a, b\) enteros positivos
    tales que \(\gcd(a, b) = 1\).
    Entonces la función:
    \begin{equation*}
      f([x]_m) = ([x]_a, [x]_b) \quad x \in \mathbb{Z}
    \end{equation*}
    es un isomorfismo%
      \index{anillo!isomorfismo}
    entre los anillos \(\mathbb{Z}_m\)
    y \(\mathbb{Z}_a \times \mathbb{Z}_b\).
  \end{theorem}
  \begin{proof}
    Primero debemos demostrar que \(f\) siquiera tiene sentido,
    hay muchas elecciones de \(x\)
    que dan la misma clase \([x]_m\) en \(\mathbb{Z}_m\).
    El teorema~\ref{theo:congruencia-mn}
    asegura que \(f\) es una biyección,
    ya que \(\gcd(a, b) = 1\).

    Además tenemos del teorema~\ref{theo:+*mod} que:
    \begin{align*}
      f([x + y]_m)
	&= ([x + y]_a, [x + y]_b) \\
	&= ([x]_a + [y]_a, [x]_b + [y]_b) \\
	&= ([x]_a, [x]_b)
	     + ([y]_a, [y]_b) \\
	&= f(x) + f(y) \\
      f([x y]_m)
	&= ([x y]_a, [x y]_b) \\
	&= ([x]_a \cdot [y]_a,
	     [x]_b \cdot [y]_b) \\
	&= ([x]_a, [x]_b)
	    \cdot ([y]_a, [y]_b) \\
	&= f(x) \cdot f(y)
    \end{align*}
    y es isomorfismo de anillo.
  \end{proof}
  Hay que tener cuidado en esto,
  la condición de que \(\gcd(a, b) = 1\) es necesaria.%
    \index{relativamente primos}
  Por ejemplo,
  \(\mathbb{Z}_8 \ncong \mathbb{Z}_2 \times \mathbb{Z}_4\),
  ya que:
  \begin{align*}
    2 \equiv 0 \pmod{2}
      &\qquad 2 \equiv 2 \pmod{4} \\
    6 \equiv 0 \pmod{2}
      &\qquad 6 \equiv 2 \pmod{4}
  \end{align*}
  Esta no es una biyección.

  Aplicando el teorema~\ref{theo:isomorfismo-anillo-Zm}
  repetidas veces tenemos:
  \begin{corollary}
    \label{cor:isomorfismo-anillo-Zm}
    Sean \(a_1\), \(a_2\), \ldots, \(a_r\)
    naturales relativamente primos a pares%
      \index{relativamente primos!a pares}
    (o sea,
     \(\gcd(a_i, a_j) = 1\) si \(i \ne j\)),
    y \(m = a_1 a_2 \dotsm a_r\).
    Entonces la función:
    \begin{equation*}
      f([x]_m)
	= ([x]_{a_1}, [x]_{a_2}, \dotsc, [x]_{a_r})
    \end{equation*}
    es un isomorfismo de anillo entre \(\mathbb{Z}_m\) y
    \(\mathbb{Z}_{a_1} \times \mathbb{Z}_{a_2}
	\times \dotsb \times \mathbb{Z}_{a_r}\)
  \end{corollary}

  Una consecuencia inmediata es el siguiente importante teorema:
  \begin{theorem}[Teorema chino de los residuos]
    \index{residuo!teorema chino de los}
    \label{theo:chino-residuos}
    Sean \(a_1\), \(a_2\), \ldots, \(a_r\)
    naturales relativamente primos a pares,
    y \(b_1\), \(b_2\), \ldots, \(b_r\) enteros cualquiera.
    Entonces hay un entero \(x\) tal que:
    \begin{align*}
      x &\equiv b_1 \pmod{a_1} \\
      x &\equiv b_2 \pmod{a_2} \\
	&\;\;\vdots	       \\
      x &\equiv b_r \pmod{a_r}
    \end{align*}
    El entero \(x\) es único módulo \(n = a_1 a_2 \dotsm a_r\).
  \end{theorem}
  \begin{proof}
    Considere el elemento
    \(([b_1]_{a_1}, [b_2]_{a_2}, \dotsc,
       [b_r]_{a_r})\)
    en
    \(\mathbb{Z}_{a_1} \times \dotsb \times \mathbb{Z}_{a_r}\).
    Por el isomorfismo del corolario~\ref{cor:isomorfismo-anillo-Zm}
    hay un único \([x]_m \in \mathbb{Z}_m\)
    tal que \([x]_{a_1} = [b_1]_{a_1}\),
    \([x]_{a_2} = [b_2]_{a_2}\), \ldots,
    \([x]_{a_r} = [b_r]_{a_r}\),
    lo que no es más que otra forma
    de decir que hay un entero \(x\),
    único módulo \(n\),
    que cumple el sistema de ecuaciones indicado.
  \end{proof}
  Al teorema~\ref{theo:isomorfismo-anillo-Zm}
  (o también al corolario~\ref{cor:isomorfismo-anillo-Zm})
  se le debiera llamar
  el \emph{padre del teorema chino de los residuos},%
    \index{residuo!teorema chino de los!padre del}
  son estos resultados los que en realidad más se usan
  bajo ese nombre.
  En inglés se abrevia \emph{CRT},
  por \emph{\foreignlanguage{english}{Chinese Remainder Theorem}}.%
    \index{CRT@\emph{CRT}|see{residuo!teorema chino de los}}

  Usando la notación del teorema chino de los residuos,
  para cálculo concreto
  esto se puede expresar mediante lo siguiente.
  Defina \(s_i\) como
  \(s_i \cdot (n / a_i) \equiv 1 \pmod{a_i}\),
  y defina \(m_i = s_i \cdot (n / a_i)\),
  con lo que \(m_i \equiv [i = j] \pmod{a_j}\).
  Considere:
  \begin{equation*}
    x = \sum_{1 \le i \le r} m_i \cdot b_i
  \end{equation*}
  Entonces:
  \begin{align*}
    x
      &\equiv \sum_{1 \le i \le r} m_i \cdot b_i \pmod{a_k} \\
      &\equiv m_k \cdot b_k			 \pmod{a_k} \\
      &\equiv b_k				 \pmod{a_k}
  \end{align*}
  Con esto el isomorfismo
  del corolario~\ref{cor:isomorfismo-anillo-Zm}
  puede usarse en la práctica.
  La demostración del teorema~\ref{theo:chino-residuos}
  no da luces de cómo obtener el valor \(x\),
  solo asegura que si probamos todas las opciones
  hallaremos exactamente una solución.

  Para clarificar ideas,
  resolveremos un ejemplo.
  Buscamos \(x\) tal que:
  \begin{align*}
    x &\equiv 3 \pmod{5} \\
    x &\equiv 5 \pmod{7} \\
    x &\equiv 1 \pmod{9}
  \end{align*}
  Los módulos son primos entre sí,
  hay una solución única
  módulo \(n = 5 \cdot 7 \cdot 9 = 315\).
  Necesitamos los siguientes inversos:
  \begin{align*}
    s_5
      &= (315 / 5)^{-1} = 2 \text{\ en\ } \mathbb{Z}_5 \\
    s_7
      &= (315 / 7)^{-1} = 5 \text{\ en\ } \mathbb{Z}_7 \\
    s_9
      &= (315 / 9)^{-1} = 8 \text{\ en\ } \mathbb{Z}_9
  \end{align*}
  lo que da los coeficientes,
  ahora módulo \(315\):
  \begin{align*}
    m_5
      &= 2 \cdot (315 / 5) = 126 \\
    m_7
      &= 5 \cdot (315 / 7) = 225 \\
    m_9
      &= 8 \cdot (315 / 9) = 280
  \end{align*}
  Para resolver nuestro problema concreto:
  \begin{align*}
    x &= 126 \cdot 3 + 225 \cdot 5 + 280 \cdot 1 \\
      &= 1\,783 \\
      &\equiv 208 \pmod{315}
  \end{align*}

  Como ejercicio,
  dejamos el problema planteado por Sunzi%
    \index{Sunzi}%
    \index{Sun Tzu|see{Sunzi}}
  en el siglo~IV.
  \begin{verse}
    Hay cierto número de objetos cuyo número es desconocido. \\
    Dividido por \(3\), el resto es \(2\); \\
    por \(5\) el resto es \(3\); \\
    y por \(7\) el resto es \(2\). \\
    ¿Cuántos serán los objetos?
  \end{verse}

  Supongamos ahora que piden:
  \begin{align*}
    x &\equiv 2 \pmod{6} \\
    x &\equiv 1 \pmod{7} \\
    x &\equiv 3 \pmod{9}
  \end{align*}
  Los módulos no son relativamente primos,
  puede no haber solución.
  Tenemos para el par en conflicto:
  \begin{align*}
    x = 6 s + 2	  &= 9 t + 3 \\
	6 s - 9 t &= 1
  \end{align*}
  Esto último es imposible,
  ya que significaría
  que el máximo común divisor entre \(6\) y \(9\)
    \index{maximo comun divisor@máximo común divisor}
  divide a \(1\),
  pero \(\gcd(6, 9) = 3\).
  No hay solución.

  Antes de continuar,
  algunas propiedades adicionales del máximo común divisor
  y el mínimo común múltiplo.
  \begin{lemma}
    \index{maximo comun divisor@máximo común divisor!propiedades}
    \index{minimo comun multiplo@mínimo común múltiplo!propiedades}
    \label{lem:gcd-lcm}
    Se cumplen:
    \begin{align*}
      \gcd(a, \gcd(b, c))
	&= \gcd(\gcd(a, b), c) \\
      \lcm(a, \lcm(b, c))
	&= \lcm(\lcm(a, b), c)
    \end{align*}
    Además:
    \begin{align*}
      \gcd(\lcm(a_1, b), \lcm(a_2, b), \dotsc, \lcm(a_r, b))
	&= \lcm(\gcd(a_1, a_2, \dotsc, a_r), b) \\
      \lcm(\gcd(a_1, b), \gcd(a_2, b), \dotsc, \gcd(a_r, b))
	&= \gcd(\lcm(a_1, a_2, \dotsc, a_r), b)
    \end{align*}
  \end{lemma}
  Básicamente,
  las operaciones son asociativas y cumplen leyes distributivas.
    \index{operacion@operación}
  \begin{proof}
    Por el teorema fundamental de la aritmética,%
      \index{teorema fundamental de la aritmetica@teorema fundamental de la aritmética}
    todo entero puede representarse por el conjunto
    de los divisores que son potencias de números primos.
    En estos términos,
    el máximo común divisor es la intersección de sus argumentos,
    y el mínimo común múltiplo su unión.
    Las identidades indicadas
    son entonces reflejo de la asociatividad
    de la unión e intersección,
    y la distributividad de la unión sobre la intersección
    y viceversa.
  \end{proof}

  Podemos extender el teorema chino de los residuos:
  \begin{theorem}[Teorema chino de los residuos generalizado]
    \index{residuo!teorema chino de los!generalizado}
    \label{theo:chino-residuos-generalizado}
    Para \(a_1\), \(b_1\), \ldots, \(a_r\), \(b_r\) cualquiera
    sean:
    \begin{align*}
      x &\equiv a_1 \pmod{b_1} \\
      x &\equiv a_2 \pmod{b_2} \\
	&\quad \vdots	       \\
      x &\equiv a_r \pmod{b_r}
    \end{align*}
    Estas congruencias tienen solución si y solo si
    \(a_i \equiv a_j \pmod{\gcd(b_i, b_j)}\) para todo \(i\), \(j\).
    Módulo \(\lcm(b_1, b_2, \dotsc, b_r)\)
    la solución es única en tal caso.
  \end{theorem}
  \begin{proof}
    Por inducción sobre \(r\).%
      \index{demostracion@demostración!induccion@inducción}
    Cuando \(r = 1\),
    el resultado es obvio.
    Partiremos con el caso \(r = 2\)
    porque lo usaremos en el paso de inducción.
    \begin{description}
    \item[Base:]
      Tenemos las congruencias:
      \begin{align*}
	x &\equiv a_1 \pmod{b_1} \\
	x &\equiv a_2 \pmod{b_2}
      \end{align*}
      Si \(d \mid b_1\),
      claramente \(x \equiv a_1 \pmod{d}\).
      En particular,
      para \(m_2 = \gcd(b_1, b_2)\)
      debe ser:
      \begin{align*}
	x &\equiv a_1 \pmod{m_2} \\
	x &\equiv a_2 \pmod{m_2}
      \end{align*}
      con lo que no hay solución
      a menos que \(a_1 \equiv a_2 \pmod{m_2}\).
      Esta es la condición sobre los \(b_i\) para el caso \(r = 2\).

      Si \(a_1 \equiv a_2 \pmod{m_2}\),
      por la identidad de Bézout%
	\index{Bezout, identidad de@Bézout, identidad de}
      sabemos que existen enteros \(u_2\) y \(v_2\) tales que:
      \begin{equation*}
	u_2 b_1 + v_2 b_2
	  = \gcd(b_1, b_2)
	  = m_2
      \end{equation*}
      Como \(a_1 \equiv a_2 \pmod{m_2}\),
      hay \(c_2 \in \mathbb{Z}\)
      tal que \(a_1 - a_2 = c_2 \cdot m_2\),
      y \(a_1 - a_2 = c_2 u_2 b_1 + c_2 v_2 b_2\).
      Con esto:
      \begin{equation*}
	s_2
	  = a_1 - c_2 u_2 b_1
	  = a_2 + c_2 v_2 b_2
      \end{equation*}
      cumple ambas congruencias.

      Para demostrar que es única,
      consideremos soluciones \(s\) y \(s'\).
      Vemos que
      \(s \equiv s' \pmod{b_1}\)
      y \(s \equiv s' \pmod{b_2}\),
      y el teorema~\ref{theo:congruencia-mn} asegura que
      \(s \equiv s' \pmod{\lcm(b_1, b_2)}\).
    \item[Inducción:]
      Suponiendo que vale para \(r\) congruencias,
      demostramos que vale para \(r + 1\):
      \begin{align*}
	x &\equiv a_{1\phantom{+1}} \pmod{b_1} \\
	x &\equiv a_{2\phantom{+1}} \pmod{b_2} \\
	  &\quad \vdots		 \\
	x &\equiv a_{r\phantom{+1}} \pmod{b_r} \\
	x &\equiv a_{r + 1} \pmod{b_{r + 1}}
      \end{align*}
      Sea \(s_r\) la solución a las primeras \(r\) congruencias,
      que por inducción existe
      y es única módulo \(\lcm(b_1, b_2, \dotsc, b_r)\).
      Consideremos las congruencias:
      \begin{align*}
	x &\equiv s_r \pmod{\lcm(b_1, b_2, \dotsc, b_r)} \\
	x &\equiv a_{r + 1} \pmod{b_{r + 1}}
      \end{align*}
      Por el caso \(r = 2\) sabemos que hay solución únicamente si:
      \begin{equation*}
	s_{r\phantom{+1}} \equiv a_{r + 1}
	  \pmod{\gcd(\lcm(b_1, \dotsc, b_r), b_{r + 1})}
      \end{equation*}
      Por el lema~\ref{lem:gcd-lcm}:
      \begin{equation*}
	\gcd(\lcm(b_1, \dotsc, b_r), b_{r + 1})
	  = \lcm(\gcd(b_1, b_{r + 1}), \gcd(b_2, b_{r + 1}),
		 \dotsc, \gcd(b_r, b_{r + 1}))
      \end{equation*}
      que es equivalente a:
      \begin{equation*}
	a_{r + 1} \equiv a_i \pmod{\gcd(b_i, b_{r + 1})}
	  \quad\text{para todo \(1 \le i \le r\)}
      \end{equation*}
      Esto extiende la condición sobre los \(a_i\).

      De cumplirse la condición sobre los \(a_i\),
      hay una solución \(s_{r + 1}\) única módulo
      \(\lcm(b_1, \dotsc, b_{r + 1})\),
      que podemos calcular como antes.
      Sean \(u_{r + 1}\) y \(v_{r + 1}\) tales que:
      \begin{equation*}
	u_{r + 1} s_r + v_{r + 1} b_{r + 1}
	  = \gcd(s_r, b_{r + 1})
      \end{equation*}
      Como \(s_r \equiv a_{r + 1} \pmod{b_{r + 1}}\),
      existe \(c_{r + 1} \in \mathbb{Z}\) en
      \(s_r - a_{r + 1} = c_{r + 1} \gcd(s_r, b_{r + 1})\),
      y \(s_{r + 1}\)
      definido como sigue cumple las \(r + 1\) congruencias:
      \begin{equation*}
	s_{r + 1}
	  = s_r - c_{r + 1} u_{r + 1} \lcm(b_1, \dotsc, b_r)
	  = a_{r + 1} + c_{r + 1} v_{r + 1} b_{r + 1}
      \end{equation*}
    \end{description}
    Por inducción lo indicado vale para \(r \in \mathbb{N}\).
  \end{proof}

  La demostración da un algoritmo para obtener la solución.
  Por ejemplo,
  consideremos el sistema:
  \begin{align*}
    x &\equiv 3 \pmod{4} \\
    x &\equiv 5 \pmod{6} \\
    x &\equiv 2 \pmod{9}
  \end{align*}
  Para las primeras dos congruencias tenemos:
  \begin{equation*}
    4 u_2 + 6 v_2
      = \gcd(4, 6)
      = 2
  \end{equation*}
  Obtenemos \(u_2 = -1\), \(v_2 = 1\),
  y tenemos \(a_1 - a_2 = 3 - 5 = -2\) que da \(c_2 = -1\),
  por lo que:
  \begin{equation*}
    s_2
      = 3 - (-1) (-1) 4
      = -1
  \end{equation*}
  Como \(\lcm(4, 6) = 12\),
  para el segundo paso queda el sistema:
  \begin{align*}
    x &\equiv -1	    \pmod{12} \\
    x &\equiv \phantom{-}7  \pmod{\phantom{0}9}
  \end{align*}
  Tenemos:
  \begin{equation*}
    12 u_3 + 9 v_3
      = \gcd(12, 9)
      = 3
  \end{equation*}
  Esto resulta en \(u_3 = 1\) y \(v_3 = -1\),
  para \(s_2 - a_3 = -1 - 2 = -3\) es \(c_3 = -1\),
  y queda:
  \begin{equation*}
    s_3
      = -1 - (-1) \cdot 1 \cdot 12
      = 11
  \end{equation*}
  La solución es única módulo \(\lcm(4, 6, 9) = 36\).

  El algoritmo implícito
  en el teorema~\ref{theo:chino-residuos-generalizado}
  es bastante engorroso.
  Una forma diferente de enfocar el tema es dividir las congruencias
  según los máximos comunes divisores.
  Veamos el ejemplo:
  \begin{align*}
    x &\equiv \phantom{0}9 \pmod{12} \\
    x &\equiv		12 \pmod{21}
  \end{align*}
  Tenemos \(\gcd(12, 21) = 3\),
  con lo que \(12 = 3 \cdot 4\) y \(21 = 3 \cdot 7\).
  La primera congruencia se descompone:
  \begin{align*}
    x &\equiv 9 \equiv 0 \pmod{3} \\
    x &\equiv 9 \equiv 1 \pmod{4}
  \end{align*}
  La segunda da:
  \begin{align*}
    x &\equiv 12 \equiv 0 \pmod{3} \\
    x &\equiv 12 \equiv 5 \pmod{7}
  \end{align*}
  Las congruencias comunes
  (módulo~\(3\)) son consistentes,
  hay solución módulo \(\lcm(12, 21) = 84\).
  El sistema se reduce a:
  \begin{align*}
    x &\equiv 0 \pmod{3} \\
    x &\equiv 1 \pmod{4} \\
    x &\equiv 5 \pmod{7}
  \end{align*}
  El teorema chino de los residuos da:%
    \index{residuo!teorema chino de los}
  \begin{alignat*}{2}
    s_3 &= (4 \cdot 7)^{-1} = 1
	&\quad m_3 &= 1 \cdot 4 \cdot 7 = 28 \\
    s_4 &= (3 \cdot 7)^{-1} = 1
	&\quad m_4 &= 1 \cdot 3 \cdot 7 = 21 \\
    s_7 &= (3 \cdot 4)^{-1} = 3
	&\quad m_7 &= 3 \cdot 3 \cdot 4 = 36
  \end{alignat*}
  y la solución es:
  \begin{align*}
    x &= 0 \cdot 28 + 1 \cdot 21 + 5 \cdot 36 \\
      &\equiv 33 \pmod{84}
  \end{align*}

  Un problema en esta línea planteó Brahmagupta en el siglo~VII.%
    \index{Brahmagupta}
  \begin{verse}
    Una anciana va al mercado,
    y un caballo pisa su canasto y le aplasta los huevos. \\
    \hspace{1em}El jinete ofrece pagar el daño
    y le pregunta cuántos huevos traía. \\
    \hspace{1em}Ella no recuerda el número exacto,
    pero al sacarlos de a dos sobraba un huevo.
    Lo mismo ocurría
    si los sacaba de a tres, cuatro, cinco y seis a la vez,
    pero al sacarlos de a siete no sobró ninguno. \\
    ¿Cuál es el mínimo número de huevos que podría haber tenido?
  \end{verse}

  Una aplicación adicional es la \emph{prueba del once}:
  Vimos antes
  (sección~\ref{sec:descomposiciones})
  que una manera de verificar operaciones aritméticas
  es la prueba de los nueves,
  que es simple de aplicar
  porque calcular el residuo módulo nueve
  de un número escrito en decimal
  es sumar sus dígitos,
  repitiendo el proceso hasta reducir a uno solo.
  Resulta que calcular el residuo módulo once
  es sumar y restar alternativamente los dígitos
  comenzando por el menos significativo:
  Como \(10 \equiv -1 \pmod{11}\),
  tenemos:
  \begin{equation*}
    \sum_{0 \le k \le n} d_k \cdot 10^k
      \equiv \sum_{0 \le k \le n} (-1)^k d_k \pmod{11}
  \end{equation*}
  Si aplicamos la prueba del nueve y la prueba del once,
  como \(\gcd(9, 11) = 1\),
  estamos verificando el resultado módulo \(9 \cdot 11 = 99\).

  El teorema~\ref{theo:chino-residuos}
  (más bien,
   el corolario~\ref{cor:isomorfismo-anillo-Zm})
  ofrece una importante estrategia adicional
  para demostrar teoremas en \(\mathbb{Z}\):
  \begin{enumerate}
  \item
    Primeramente,
    demuestre el resultado para \(p\) primo.
  \item
    Enseguida,
    demuestre que es válido para \(p^\alpha\),
    potencias de primos.
  \item
    Use el (padre del) teorema chino de los residuos
    para combinar los resultados anteriores
    y obtener el caso general.
  \end{enumerate}
  Más adelante aparecerán muchas aplicaciones de esta idea.

\section[Estructura de
	   \texorpdfstring{$\mathbb{Z}^\times_m$}
			  {las unidades de clases de congruencia}]
	{\protect\boldmath
	 Estructura de
	  \texorpdfstring{$\mathbb{Z}^\times_m$}
				{las unidades de clases de congruencia}%
       \protect\unboldmath}
\label{sec:estructura-Un}

  El isomorfismo del teorema~\ref{theo:isomorfismo-anillo-Zm}
  permite demostrar:
  \begin{theorem}
    \label{theo:Z*ab=Z*a+Z*b}
    Si \(a\) y \(b\) son naturales relativamente primos,
    entonces:
    \begin{equation*}
      \mathbb{Z}^\times_{a b}
	\cong \mathbb{Z}^\times_a \times \mathbb{Z}^\times_b
    \end{equation*}
  \end{theorem}
  \begin{proof}
    Sabemos que \(\mathbb{Z}_{a b}\)
    y \(\mathbb{Z}_a \times \mathbb{Z}_b\)
    son anillos isomorfos,
    con lo que \(\mathbb{Z}^\times_{a b}\)
    es isomorfo al grupo de unidades
    de \(\mathbb{Z}_a \times \mathbb{Z}_b\).
    Ahora bien,
    un elemento de \(\mathbb{Z}_a \times \mathbb{Z}_b\)
    es invertible si lo son sus componentes:
    \begin{equation*}
      (x, y) \cdot (x', y')
	= (x x', y y')
	= (1, 1)
    \end{equation*}
    con \(x \in \mathbb{Z}^\times_a\)
    e \(y \in \mathbb{Z}^\times_b\),
    con lo que el grupo de unidades
    de \(\mathbb{Z}_a \times \mathbb{Z}_b\)
    es exactamente
      \(\mathbb{Z}^\times_a \times \mathbb{Z}^\times_b\).
  \end{proof}
  Como corolario,
  tenemos para la función \(\phi\) de Euler:%
    \index{\(\phi\) de Euler}
  \begin{corollary}
    \label{cor:phi-multiplicativa}
    Sea \(\phi\) la función de Euler.
    Si \(a\) y \(b\) son naturales relativamente primos,
    entonces \(\phi(a b) = \phi(a) \cdot \phi(b)\)
  \end{corollary}
  \begin{proof}
    Del teorema~\ref{theo:Z*ab=Z*a+Z*b} sabemos que:
    \begin{equation*}
      \phi(a b)
	= \lvert \mathbb{Z}^\times_{a b} \rvert
	= \lvert \mathbb{Z}^\times_a
		   \times \mathbb{Z}^\times_b \rvert
	= \lvert \mathbb{Z}^\times_a \rvert
	    \cdot \lvert \mathbb{Z}^\times_b \rvert
	= \phi(a) \cdot \phi(b)
	\qedhere
    \end{equation*}
  \end{proof}
  Esta propiedad es importante:
  \begin{definition}
    \index{funcion@función!aritmetica@aritmética|textbfhy}
    \index{funcion@función!aritmetica@aritmética!multiplicativa|textbfhy}
    Una función \(f \colon \mathbb{N} \rightarrow \mathbb{C}\)
    se llama \emph{aritmética}.
    Anotaremos \(\mathscr{A}\)
    para el conjunto de funciones aritméticas.
    Una función aritmética \(f\)
    se llama \emph{multiplicativa}
    si \(f(a \cdot b) = f(a) \cdot f(b)\)
    siempre que \(\gcd(a, b) = 1\).
    A su conjunto lo llamamos \(\mathscr{M}\).
  \end{definition}
  \noindent
  Supongamos que \(f\) es multiplicativa,
  y que para algún \(n \in \mathbb{N}\) es \(f(n) \ne 0\).
  Como \(\gcd(1, n) = 1\):
  \begin{equation*}
    f(n)
      = f(n \cdot 1)
      = f(n) \cdot f(1)
  \end{equation*}
  con lo que \(f(1) = 1\) o \(f(n) = 0\)
  para todo \(n = \mathbb{N}\).

  Por el teorema fundamental de la aritmética%
    \index{teorema fundamental de la aritmetica@teorema fundamental de la aritmética}
  todo entero se puede descomponer
  en un producto de potencias de primos distintos.
  Como potencias de primos diferentes son relativamente primas,
  una función multiplicativa queda determinada por su valor
  para potencias de primos.

  Como acabamos de demostrar que \(\phi\) es multiplicativa,
  tenemos una manera de calcularla:
  \begin{corollary}
    \label{cor:calcular-phi}
    Sea \(n = p_1^{\alpha_1} p_2^{\alpha_2} \dotsm p_r^{\alpha_r}\)
    la factorización completa de \(n\) en primos distintos \(p_i\).
    Entonces:
    \begin{align*}
      \phi(n)
	&= p_1^{\alpha_1 - 1} (p_1 - 1) p_2^{\alpha_2 - 1} (p_2 - 1)
	     \dotsm p_r^{\alpha_r - 1} (p_r - 1) \\
	&= n \cdot \left(1 - \frac{1}{p_1}\right) \cdot
		     \left(1 - \frac{1}{p_2}\right) \dotsm
		     \left(1 - \frac{1}{p_r}\right)
    \end{align*}
  \end{corollary}
  \begin{proof}
    Del corolario~\ref{cor:phi-multiplicativa} sabemos que
      \(\phi(n)
	  = \phi(p_1^{\alpha_1}) \cdot \phi(p_2^{\alpha_2})
	      \dotsm \phi(p_r^{\alpha_r})\).
    Necesitamos el valor de \(\phi(p^\alpha)\),
    para \(p\) primo y \(\alpha\) natural.
    Hay \(p^k\) números entre \(1\) y \(p^k\),
    no son relativamente primos a \(p^\alpha\)
    los \(p^{\alpha - 1}\) múltiplos de \(p\) en este rango:
    \begin{equation*}
      \phi(p^\alpha)
	= p^\alpha - p^{\alpha - 1}
	= p^{\alpha - 1} (p - 1)
	= p^\alpha \cdot \left(1 - \frac{1}{p}\right)
    \end{equation*}
    Multiplicando esto
    sobre las potencias de primos factores de \(n\)
    da lo anunciado.
  \end{proof}
  Algunas funciones aritméticas interesantes adicionales son:
  \begin{description}
  \item[La identidad:]
    \(\iota(n) = n\)
  \item[\boldmath Potencias de \(n\):\unboldmath]
    \(\iota_a(n) = n^a\)
  \item[\boldmath El número de divisores de \(n\):\unboldmath]
    \(\tau(n)
	= \lvert \{ d \in \mathbb{N} \colon d \mid n \} \rvert\)
  \item[\boldmath La suma de los divisores de \(n\):\unboldmath]
    \(\sigma(n)
	= \sum_{d \mid n} d\)
  \item[\boldmath El producto de los divisores de \(n\):\unboldmath]
    \(\pi(n)
	= \prod_{d \mid n} d\)
  \end{description}
  Acá hemos usado nuestra convención general de indicar los índices
  de sumas o productos mediante condiciones,
  en este caso de divisibilidad.

  Un resultado importante para funciones multiplicativas es:
  \begin{theorem}
    \label{theo:sum-multiplicative}
    Sea \(f\) una función aritmética%
      \index{funcion@función!aritmetica@aritmética!multiplicativa}
    y \(S\) definida por:%
      \index{funcion@función!aritmetica@aritmética!funcion suma@función suma|textbfhy}
    \begin{equation*}
      S(n)
	= \sum_{d \mid n} f(d)
    \end{equation*}
    Entonces \(f\) es multiplicativa si y solo si lo es \(S\).
  \end{theorem}
  \begin{proof}
    Demostramos implicancia en ambas direcciones.
    Sean \(x, y \in \mathbb{N}\) relativamente primos,
    y sea \(f\) multiplicativa.
    Sean además \(x_1, x_2, \dotsc, x_r\)
    e \(y_1, y_2, \dotsc, y_s\)
    todos los divisores de \(x\) e \(y\),
    respectivamente.
    Entonces \(\gcd(x_i, y_j) = 1\),
    y \(\{x_i y_j\}_{i, j}\) son todos los divisores de \(x y\):
    \begin{equation*}
      S(x) \cdot S(y)
	= \sum_i f(x_i) \sum_j f(y_j)
	= \sum_{i, j} f(x_i) f(y_j)
	= \sum_{i, j} f(x_i y_j)
	= S(x y)
    \end{equation*}
    y \(S\) es multiplicativa.

    Para el recíproco,
    sea \(S\) multiplicativa.
    Demostramos por inducción fuerte sobre \(n\)%
      \index{demostracion@demostración!induccion@inducción}
    que cuando \(n = x y\) con \(\gcd(x, y) = 1\)
    es \(f(n) = f(x) f(y)\).
    \begin{description}
    \item[Base:]
      El caso \(n = 1\) es trivial:
      \(f(1) = S(1)\).
    \item[Inducción:]
      Para nuestros \(x\) e \(y\) tenemos,
      usando la hipótesis de inducción:
      \begin{align*}
	S(x y)
	  &= \sum_{\substack{
		     u \mid x \\
		     v \mid y
		  }} f(u v)
	   = \sum_{\substack{
		    u \mid x \\
		    v \mid y \\
		    u v < n
		 }} f(u) f(v) + f(x y) \\
      \intertext{Por otro lado,
		 sacando de las sumatorias los términos
		 para \(u = x\) y \(v = y\) queda:}
	S(x) S(y)
	  &= \sum_{u \mid x} f(u) \sum_{v \mid y} f(v)
	   = \sum_{\substack{
		    u \mid x \\
		    v \mid y \\
		    u v < n
		 }} f(u) f(v) + f(x) f(y)
      \end{align*}
      Ambas expresiones son iguales ya que \(S\) es multiplicativa,
      y es \(f(x y) = f(x) f(y)\).
    \end{description}
    Por inducción vale para todo \(n \in \mathbb{N}\).
  \end{proof}
  \begin{corollary}
    \label{cor:formula-sum-function}
    Sea \(f\) una función multiplicativa,%
      \index{funcion@función!aritmetica@aritmética!multiplicativa}
    y sea \(S\) su función suma:%
      \index{funcion@función!aritmetica@aritmética!funcion suma@función suma|textbfhy}
    \begin{equation*}
      S (n)
	= \sum_{d \mid n} f(d)
    \end{equation*}
    Si \(n = p_1^{\alpha_1} p_2^{\alpha_2} \dotsm p_r^{\alpha_r}\)
    es la descomposición de \(n\)
    en factores primos distintos \(p_i\),
    entonces:
    \begin{equation*}
      S (n)
	= \prod_{1 \le i \le r}
	    \left(
	      1 + f(p_i) + f(p_i^2) + \dotsb + f(p_i^{\alpha_i})
	    \right)
    \end{equation*}
  \end{corollary}
  \begin{proof}
    Si \(f\) es multiplicativa,
    lo es \(S\).
    El valor indicado de \(S(n)\)
    corresponde para potencias de primos.
  \end{proof}
  Del teorema~\ref{theo:sum-multiplicative}
  vemos que son multiplicativas:%
    \index{funcion@función!aritmetica@aritmética!multiplicativa}%
    \index{\(\tau\) (numero de divisores)@\(\tau\) (número de divisores)}%
    \index{\(\sigma\) (suma de divisores)}
  \begin{align*}
    \tau(n)
      &= S_1 (n)
       = \sum_{d \mid n} 1 \\
    \sigma(n)
      &= S_{\iota}(n)
       = \sum_{d \mid n} d
  \end{align*}
  Por el corolario~\ref{cor:formula-sum-function}
  en términos de la factorización completa
    \(n = p_1^{\alpha_1} p_2^{\alpha_2} \dotsm p_r^{\alpha_r}\)
  tenemos:
  \begin{align*}
    \tau(n)
      &= \prod_{1 \le i \le r} (\alpha_i + 1) \\
    \sigma(n)
      &= \prod_{1 \le i \le r}
	   \frac{p_i^{\alpha_i + 1} - 1}{p_i - 1}
  \end{align*}
  La función \(\pi(n)\) no es multiplicativa.

  Para los griegos
  la relación de un número con sus divisores propios
  tenía relevancia mística.
  Así reverenciaban especialmente a los \emph{números perfectos},%
    \index{numero@número!perfecto|textbfhy}
  que son la suma de sus divisores propios.
  Conocían los casos \(6 = 1 + 2 + 3\),
  \(28 = 1 + 2 + 4 + 7 + 14\),
  496 y 8128.
  En términos de las funciones definidas antes,
  \(n\) es perfecto cuando \(\sigma(n) = 2 n\)
  (los factores propios de \(n\) suman a \(n\),
   con \(n\) suman \(2 n\)).
  Tenemos también,
  si \(p\) es primo:
  \begin{align*}
    \sigma(p)
      &= p + 1 \\
    \sigma(p^\alpha)
      &= 1 + p + \dotsb + p^\alpha
       = \frac{p^{\alpha + 1} - 1}{p - 1}
  \end{align*}
  Del resultado siguiente cada uno de los participantes
  demostró una implicancia.
  \begin{theorem}[Euclides -- Euler]
    \index{numero@número!perfecto!par}
    \label{theo:even-perfect-numbers}
    Un par \(n\) es perfecto si y solo si
    \(n = 2^{m - 1} \left( 2^m - 1 \right)\) con \(2^m - 1\) primo.
  \end{theorem}
  \begin{proof}
    Demostramos implicancia en ambas direcciones.

    Si \(n = 2^{m - 1} \left( 2^m - 1 \right)\)
    con \(2^m - 1\) primo,
    entonces como \(\sigma\) es multiplicativa:%
      \index{funcion@función!aritmetica@aritmética!multiplicativa}
    \begin{equation*}
      \sigma(n)
	= \sigma \left( 2^{m - 1} \right)
	    \sigma \left( 2^m - 1 \right)
	= \frac{2^m - 1}{2 - 1} \cdot \left( (2^m - 1) + 1 \right)
	= 2^m \left( 2^m - 1 \right)
	= 2 n
    \end{equation*}
    y \(n\) es perfecto.
    Esta parte fue demostrada por Euclides.%
      \index{Euclides}

    Para el recíproco,
    sea \(n = 2^{m - 1} u\) un número perfecto
    con \(m > 1\) y \(u\) impar.
    Entonces:
    \begin{align*}
      2^m u
	&= \sigma \left( 2^{m - 1} u \right)
	 = \left( 2^m - 1 \right) \sigma(u) \\
      \sigma(u)
	&= \frac{2^m u}{2^m - 1}
	 = u + \frac{u}{2^m - 1}
    \end{align*}
    Claramente el último término es un divisor de \(u\).
    Como \(m > 1\),
    \(2^m - 1 > 1\).
    O sea,
    estamos expresando \(\sigma(u)\)%
      \index{\(\sigma\) (suma de divisores)}
    como la suma de dos divisores distintos de \(u\),
    por lo que \(u\) es primo;
    tiene que ser \(u = 2^m - 1\).
    Este es el aporte de Euler.%
      \index{Euler, Leonhard}
  \end{proof}
  Esto resuelve completamente el caso de números perfectos pares.
  Es fácil ver que si \(m\) es compuesto
  lo es \(2^m - 1\),
  por lo que basta considerar \(2^p - 1\) con \(p\) primo.
  A tales primos se les llama \emph{primos de Mersenne},%
    \index{Mersenne, primo de}
  quien los estudió a principios del siglo~XVII.
  Se conocen 48 primos de Mersenne a febrero de 2013,
  incluso el mayor primo conocido a la fecha
  es \(2^{57\,885\,161} - 1\).

  Es primo \(2^2 - 1 = 3\)
  y \(2^{2 - 1} (2^2 - 1) = 6\) es perfecto.
  Asimismo \(2^{3 - 1} (2^3 - 1) = 28\),
  \(2^{5 - 1} (2^5 - 1) = 496\)
  y \(2^{7 - 1} (2^7 - 1) = 8128\)
  son perfectos.

  Por el otro lado,
  determinar si hay números perfectos impares%
    \index{numero@número!perfecto!impar}
  es un problema abierto
  desde antes de Euclides.

  \begin{definition}
    Sean \(f\) y \(g\) funciones aritméticas.
    Su \emph{convolución de Dirichlet} es:%
      \index{Dirichlet, convolucion de@Dirichlet, convolución de|textbfhy}
    \begin{equation}
      \label{eq:Dirichlet-convolution}
      (f * g)(n)
	= \sum_{d \mid n} f(d) g(n / d)
	= \sum_{a b = n} f(a) g(b)
    \end{equation}
  \end{definition}
  Es claro que la operación \(*\) es conmutativa,%
    \index{operacion@operación!conmutativa}
  y es fácil demostrar que es asociativa.%
    \index{operacion@operación!asociativa}
  Con:
  \begin{equation}
    \index{\(\epsilon\) (identidad para convolucion de Dirichlet)@\(\epsilon\) (identidad para convolución de Dirichlet)}
    \label{eq:arithmetic-epsilon}
    \epsilon(n)
      = \begin{cases}
	  1 & \text{si \(n = 1\)} \\
	  0 & \text{si \(n > 1\)}
	\end{cases}
  \end{equation}
  para la función aritmética \(f\)
  tenemos \(f * \epsilon = \epsilon * f = f\),
  lo que da un neutro multiplicativo.%
    \index{operacion@operación!elemento neutro}
  En particular:
  \begin{theorem}
    \label{theo:multiplicative-group-Dirichlet}
    El conjunto \(\mathscr{M}\)
    de funciones aritméticas multiplicativas
    es cerrado respecto de la convolución de Dirichlet.
  \end{theorem}
  \begin{proof}
    Sean \(f, g \in \mathscr{M}\),
    sea \(h = f * g\),
    y \(a, b \in \mathbb{N}\) con \(\gcd(a, b) = 1\).
    Como \(\gcd(a, b) = 1\)
    los factores de \(a b\)
    resultan de todas las combinaciones de factores de \(a\) y \(b\)
    por separado:
    \begin{align*}
      h(a) h(b)
	&= (f * g)(a) \cdot (f * g)(b) \\
	&= \sum_{u_1 v_1 = a}
	     f(u_1) g(v_1) \sum_{u_2 v_2 = b} f(u_2) g(v_2) \\
	&= \sum_{\substack{u_1 v_1 = a \\ u_2 v_2 = b}}
	     f(u_1) g(v_1) f(u_2) g(v_2) \\
	&= \sum_{u_1 u_2 v_1 v_2 = a b}
	     f(u_1) f(u_2) g(v_1) g(v_2) \\
	&= \sum_{u v = a b} f(u) g(v) \\
	&= (f * g)(a b) \\
	&= h(a b)
    \qedhere
    \end{align*}
  \end{proof}
  Incluso podemos calcular inversos.
  \begin{lemma}
    \label{lem:Dirichlet-inverse}
    Toda función aritmética tal que \(f(1) \ne 0\)
    tiene inversa de Dirichlet
    dada por:
    \begin{equation*}
      f^{-1}(n)
	= \begin{cases}
	    \displaystyle
		 \frac{1}{f(1)}
	       & \text{si \(n = 1\)} \\
	    \\
	    \displaystyle
		 - \frac{1}{f(1)}
		     \, \sum_{\substack{a b = n \\ b < n}}
				    f(a) f^{-1}(b)
	       & \text{si \(n > 1\)}
	  \end{cases}
    \end{equation*}
  \end{lemma}
  \begin{proof}
    Corresponde a plantear el sistema de ecuaciones,
    escrito usando la convención de Iverson:
    \begin{align*}
      \epsilon
	&= f * f^{-1} \\
      [n = 1]
	&= \sum_{a b = n} f(a) f^{-1}(b)
    \end{align*}
    La expresión indicada para \(f^{-1}\) satisface este sistema.
  \end{proof}
  Una función importante es:
  \begin{definition}
    \index{Mobius, funcion de@Möbius, función de|textbfhy}
    \index{\(\mu\)|see{Möbius, función de}}
    \index{Mobius, August Ferdinand@Möbius, August Ferdinand}
    La \emph{función de Möbius}
    se define mediante:
    \begin{equation}
      \label{eq:Moebius-mu}
      \mu(n)
	= \begin{cases}
	    1	   &
	       \text{si \(n = 1\)} \\
	    0	   &
	       \text{si \(n\)
		     es divisible por el cuadrado de un primo} \\
	    (-1)^k &
	       \text{si \(n\)
		     es el producto de \(k\) primos diferentes}
	  \end{cases}
    \end{equation}
  \end{definition}
  Un momento de reflexión muestra que \(\mu\) es multiplicativa.
    \index{funcion@función!aritmetica@aritmética!multiplicativa}
  \begin{lemma}
    \label{lem:sum-mu}
    Para \(n \in \mathbb{N}\),
    la función de Möbius satisface:
    \begin{equation*}
      \sum_{d \mid n} \mu(d)
	= \begin{cases}
	    1 & \text{si \(n = 1\)} \\
	    0 & \text{si \(n \ge 2\)}
	  \end{cases}
    \end{equation*}
  \end{lemma}
  \begin{proof}
    Como \(\mu\) es multiplicativa,
    por el teorema~\ref{theo:sum-multiplicative}
    lo es la suma indicada.
    Basta entonces hallar el valor de la suma
    en potencias de un primo \(p\).
    Hay dos casos a considerar:
    \begin{description}
    \item[\boldmath \(p^0 = 1\):\unboldmath]
      En este caso la suma es simplemente \(\mu(1) = 1\).
    \item[\boldmath \(p^\alpha\), con \(\alpha \ge 1\):\unboldmath]
      Acá,
      como \(\mu(p^k) = 0\) si \(k > 1\):
      \begin{equation*}
	\sum_{d \mid p^\alpha} \mu(d)
	  = \mu(1) + \mu(p) + \mu(p^2) + \dotsb + \mu(p^\alpha)
	  = 1 + (-1) + 0 + \dotsb + 0
	  = 0
      \end{equation*}
    \end{description}
    Multiplicando sobre los factores primos de \(n\)
    se obtiene lo prometido.
  \end{proof}
  La curiosa definición de \(\mu\)
  resulta ser simplemente el inverso de la función \(1\),
  \(1 * \mu = \epsilon\),
  cosa que puede verificarse
  usando el lema~\ref{lem:Dirichlet-inverse}.
  En detalle,
  llamando \(\mu = 1^{-1}\),
  tenemos:
  \begin{description}
  \item[\boldmath \(n = 1\):\unboldmath]
    Es \(\mu(1) = 1 / 1(1) = 1\).
  \item[\boldmath \(n > 1\):\unboldmath]
    En general es:
    \begin{equation*}
      \mu(n)
	= - \frac{1}{1} \,
	      \sum_{\substack{a b = n \\ b < n}}
		1 \cdot \mu(b)
	= - \sum_{\substack{d \mid n \\ d < n}} \mu(d)
    \end{equation*}
    O sea,
    sucesivamente por el lema~\ref{lem:Dirichlet-inverse}:
    \begin{align*}
      \mu(2)
	&= - \sum_{\substack{d \mid 2 \\ d < 2}} \mu(d)
	 = - \mu(1)
	 = -1 \\
      \mu(3)
	&= - \sum_{\substack{d \mid 3 \\ d < 3}} \mu(d)
	 = - \mu(1)
	 = -1 \\
      \mu(4)
	&= - \sum_{\substack{d \mid 4 \\ d < 4}} \mu(d)
	 = - (\mu(1) + \mu(2))
	 = 0 \\
      \mu(5)
	&= - \sum_{\substack{d \mid 5 \\ d < 5}} \mu(d)
	 = - \mu(1)
	 = -1 \\
      \mu(6)
	&= - \sum_{\substack{d \mid 6 \\ d < 6}} \mu(d)
	 = - (\mu(1) + \mu(2) + \mu(3))
	 = 1
    \end{align*}
  \end{description}
  Es claro que calcular la inversa
  por esta vía es bastante engorroso.

  Esto hace útil la función de Möbius:
  \begin{theorem}[Inversión de Möbius]
    \index{Mobius, inversion de@Möbius, inversión de|textbfhy}
    \label{theo:Moebius-inversion}
    Sean dos funciones aritméticas
    (no necesariamente multiplicativas)
    tales que para todo \(n \in \mathbb{N}\) se cumple:
    \begin{equation*}
      g(n)
	= \sum_{d \mid n} f(d)
    \end{equation*}
    entonces para todo \(n \in \mathbb{N}\):
    \begin{equation*}
      f(n)
	= \sum_{d \mid n} \mu(d) g(n / d)
    \end{equation*}
  \end{theorem}
  \begin{proof}
    Tenemos:
    \begin{align*}
      g
	&= 1 * f \\
      f
	&= \mu * g
      \qedhere
    \end{align*}
  \end{proof}
  También:
  \begin{lemma}
    \label{lem:Dirichlet-fg=>f}
    Si \(g\) es multiplicativa,
    y lo es \(f * g\),
    entonces es multiplicativa \(f\).
  \end{lemma}
  \begin{proof}
    Si alguna de las funciones es cero,
    el resultado es obvio.
    En caso contrario,
    la demostración es por contradicción.
    Definamos \(h = f * g\) para comodidad.
    Suponemos que \(f\) no es multiplicativa,
    con lo que existen \(m\), \(n\) mínimos con \(\gcd(m, n) = 1\)
    tales que \(f(m n) \ne f(m) f(n)\).
    No puede ser \(m n = 1\),
    ya que \(h(1) = f(1) g(1)\),
    como \(h\) y \(g\) son multiplicativas,
    \(h(1) = g(1) = 1\),
    con lo que \(f(1) = 1\) y \(f(1) = f(1) f(1)\).

    Sabemos entonces que \(m n \ne 1\).
    Calcularemos \(h(m n) = h(m) h(n)\) de dos maneras,
    dejando fuera el término que involucra a \(m n\) con \(f\)
    en ambos casos.
    Comparando ambas llegaremos a una contradicción.
    \begin{align*}
      h(m n)
	&= \sum_{u v = m n} f(u) g(v) \\
	&= \sum_{\substack{u v = m n \\ u < m n}} f(u) g(v)
	     + f(m n) g(1) \\
	&= \sum_{\substack{u v = m n \\ u < m n}} f(u) g(v)
	     + f(m n) \\
      h(m) h(n)
	&= \sum_{u_1 v_1 = m} f(u_1) g(v_1)
	     \sum_{u_2 v_2 = n} f(u_2) g(v_2) \\
	&= \sum_{\substack{u_1 v_1 = m \\ u_2 v_2 = n}}
	     f(u_1) g(v_1) f(u_2) g(v_2) \\
	&= \sum_{u_1 u_2 v_1 v_2 = m n}
	     f(u_1) f(u_2) g(v_1) g(v_2) \\
	&= \sum_{\substack{u_1 u_2 v_1 v_2 = m n \\
			   u_1 < m \\
			   u_2 < n}}
	     f(u_1) f(u_2) g(v_1) g(v_2)
	     + f(m) f(n) g(1) g(1) \\
    \intertext{Pero \(f\) es multiplicativa hasta antes de \(m n\),
	       y \(g\) es multiplicativa:}
	&= \sum_{\substack{u v = m n \\ u < m n}}
	     f(u) g(v)
	     + f(m) f(n) \\
      h(m n) - h(m) h(n)
	&= f(m n) - f(m) f(n) \\
	&\ne 0
    \end{align*}
    Tenemos una contradicción,
    \(f * g\) no es multiplicativa,
    contrario a la hipótesis.
  \end{proof}
  \begin{corollary}
    \label{cor:Dirichlet-inverse-multiplicative}
    La inversa de una función multiplicativa es multiplicativa.
  \end{corollary}
  \begin{proof}
    Si \(f\) es multiplicativa,
    entonces \(f * f^{-1} = \epsilon\)
    cumplen las hipótesis del lema~\ref{lem:Dirichlet-fg=>f}.
  \end{proof}
  Uniendo las piezas:
  \begin{theorem}
    \label{theo:Dirichlet-ring}
    El conjunto de funciones aritméticas%
      \index{funcion@función!aritmetica@aritmética}
    \(\mathscr{A}\) es un anillo conmutativo%
      \index{anillo!conmutativo}
    con suma de funciones
    y convolución de Dirichlet como multiplicación.
    Su grupo de unidades
    es el conjunto de funciones multiplicativas
    que no son cero,
    \(\mathscr{M} \smallsetminus \{0\}\).
  \end{theorem}

  Tenemos también:
  \begin{theorem}[Identidad de Gauß]
    \index{Gauss, identidad de@Gauß, identidad de|textbfhy}
    \index{Gauss, Carl Friedrich@Gauß, Carl Friedrich}
    \label{theo:Gauss-identity}
    Tenemos:
    \begin{equation*}
      \sum_{d \mid n} \phi(d)
	= n
    \end{equation*}
  \end{theorem}
  \begin{proof}
    La suma es multiplicativa,
    basta evaluarla para las potencias de un primo \(p\).
    Pero:
    \begin{align*}
      \sum_{d \mid p^\alpha} \phi(d)
	&= \phi(p^0) + \sum_{1 \le k \le \alpha} \phi(p^k) \\
	&= 1 + \sum_{1 \le k \le \alpha} (p^k - p^{k - 1}) \\
	&= p^\alpha
    \end{align*}
    Resulta la fórmula prometida
    al multiplicar sobre los primos factores de \(n\).
  \end{proof}
  Un resultado útil es el siguiente:
  \begin{theorem}
    \label{theo:sum-mu-f}
    Sea \(f\) una función multiplicativa distinta de cero,
    y sea \(n = p_1^{e_1} p_2^{e_2} \dotsm p_r^{e_r}\)
    con \(p_k\) primos distintos y \(e_k \ge 1\).
    Entonces:
    \begin{equation}
      \label{eq:sum-mu-f}
      \sum_{d \mid n} \mu(d) f(d)
	= \prod_k (1 - f(p_k))
    \end{equation}
  \end{theorem}
  \begin{proof}
    La función \(\mu(n) f(n)\) es multiplicativa,
    con lo que lo es la suma a evaluar.
    Basta evaluar la suma para \(p^e\)
    con \(p\) primo y \(e \ge 1\)
    y luego combinar.
    Como \(f(1) = 1\):
    \begin{align*}
      \sum_{d \mid p^e} \mu(d) f(d)
	&= \sum_{0 \le k \le e} \mu(p^k) f(p^k) \\
	&= \mu(1) f(1) + \mu(p) f(p) \\
	&= 1 - f(p)
    \end{align*}
  \end{proof}

  Usando la convolución de Dirichlet%
    \index{Dirichlet, convolucion de@Dirichlet, convolución de}
  resulta simple demostrar fórmulas
  que de otra forma serían casi imposibles.
  Considere:
  \begin{align*}
    \tau
      &= 1 * 1 \\
    \sigma
      &= \iota * 1 \\
    \phi
      &= \mu * \iota
  \end{align*}
  La última no es más que la identidad de Gauß,%
    \index{Gauss, identidad de@Gauß, identidad de}
  que así podemos reescribir:
  \begin{equation*}
    \sum_{d \mid n} d \mu(d)
      = \phi(n)
  \end{equation*}
  Evaluemos ahora:
  \begin{equation*}
    \sum_{d \mid n} \phi(d) \tau(n / d)
  \end{equation*}
  que es decir:
  \begin{equation*}
    \phi * \tau
      = \mu * \iota * 1 * 1
      = \mu * 1 * \iota * 1
      = \epsilon * \sigma
      = \sigma
  \end{equation*}
  También podemos calcular:
  \begin{equation*}
    \sum_{d \mid n} \mu(d) \tau(n / d)
  \end{equation*}
  que es:
  \begin{equation*}
    \mu * \tau
      = \mu * 1 * 1
      = 1
  \end{equation*}
  Pero también,
  si \(n = p_1^{e_1} p_2^{e_2} \dotsm p_r^{e_r}\)
  con \(p_k\) primos distintos y \(e_k \ge 1\),
  por el teorema~\ref{theo:sum-mu-f}
  al ser \(\tau(p) = 2\):
  \begin{align*}
    \sum_{d \mid n} \mu(d) \tau(d)
      &= \prod_k (1 - \tau(p)) \\
      &= (-1)^r
  \end{align*}
  La suma resulta ser \(1\)
  si \(n\) es divisible por un número par de primos distintos
  y \(-1\) en caso contrario.
  Asimismo:
  \begin{align*}
    \sum_{d \mid n} \mu^2(d)
      &= \prod_k (1 - \mu(p_k)) \\
      &= 2^r
  \end{align*}

  Partiendo de la identidad de Gauß:%
    \index{Gauss, identidad de@Gauß, identidad de}
  \begin{align*}
    \sum_{d \mid n} \phi(d)
      &= n \\
    \phi(n)
      &= \sum_{d \mid n} \mu(d) \frac{n}{d} \\
      &= n \sum_{d \mid n} \frac{\mu(d)}{d} \\
      &= n \prod_k \left( 1 - \frac{1}{p_k} \right)
  \end{align*}
  Otra derivación de la fórmula para \(\phi\).%
    \index{\(\phi\) de Euler}

  Consideremos palabras formadas
  con símbolos de algún alfabeto \(\Sigma\),
  por ejemplo \(\Sigma = \{a, b, c\}\).%
    \index{palabra}
  Una palabra puede ser la repetición de una palabra más corta,
  como \(\mathtt{baba}\).
  A la parte mínima que se repite para formar una palabra
  le llamaremos su \emph{raíz}.%
    \index{palabra!primitiva!raiz@raíz}
  Así,
  la raíz de \(\mathtt{ababab}\) es \(\mathtt{ab}\),
  la raíz de \(\mathtt{acaba}\) es \(\mathtt{acaba}\).
  Nos interesa el número de palabras que son sus propias raíces
  (no son repeticiones de palabras más cortas),
  a las que llamaremos \emph{primitivas}.%
    \index{palabra!primitiva}

  Por ejemplo,
  para \(\Sigma = \{ a, b \}\)
  el número total de palabras de largo \(4\) es \(2^4 = 16\).
  Debemos descontar
  las que se forman repitiendo palabras primitivas.
  Palabras primitivas de largo \(1\)
  son \(\mathtt{a}\) y \(\mathtt{b}\),
  que dan lugar a \(\mathtt{aaaa}\) y \(\mathtt{bbbb}\);
  primitivas de largo \(2\) son \(\mathtt{ab}\) y \(\mathtt{ba}\),
  que dan lugar a \(\mathtt{abab}\) y \(\mathtt{baba}\).
  En total hay \(4\) palabras no primitivas,
  y por lo tanto son \(12\) las primitivas.

  Llamemos \(p(n)\)
  al número de palabras primitivas de largo \(n\)
  (esto claramente depende del número \(s\)
   de símbolos en el alfabeto).
  Como toda palabra es la repetición de alguna palabra
  (cuyo largo divide a \(n\))
  podemos escribir:
  \begin{equation*}
    s^n
      = \sum_{d \mid n} p(d)
  \end{equation*}
  Inversión de Möbius%
    \index{Mobius, inversion de@Möbius, inversión de}
  nos da:
  \begin{equation*}
    p(n)
      = \sum_{d \mid n} \mu(n / d) s^d
  \end{equation*}
    \index{palabra!primitiva!numero@número}

  Para el ejemplo el alfabeto es \(\{ a, b \}\),
  que da \(s = 2\),
  y es \(n = 4\).
  Resulta:
  \begin{align*}
    p(4)
      &= \sum_{d \mid 4} \mu(4 / d) \cdot 2^d \\
      &= \mu(4) \cdot 2^1 + \mu(2) \cdot 2^2 + \mu(1) \cdot 2^4 \\
      &= 0 \cdot 2 - 1 \cdot 4 + 1 \cdot 16 \\
      &= 12
  \end{align*}
  lo que confirma nuestro cálculo anterior.

  Pero la fórmula permite calcular valores mucho mayores
  en forma simple:
  \begin{align*}
    p(12)
      &= \sum_{d \mid 12} \mu(12 / d) \cdot 2^d \\
      &= \mu(12) \cdot 2^1
	   + \mu(6) \cdot 2^2
	   + \mu(4) \cdot 2^3
	   + \mu(2) \cdot 2^6
	   + \mu(1) \cdot 2^{12} \\
      &= 0 \cdot 2
	   + 1 \cdot 4
	   + 0 \cdot 8
	   - 1 \cdot 64
	   + 1 \cdot 4096 \\
      &= 4036
  \end{align*}

  Puede profundizarse bastante
  partiendo de los conceptos anteriores,
  aún sin usar técnicas sofisticadas,
  como muestra magistralmente Moser~%
    \cite{moser04:_introd_theor_number}.

%%% Local Variables:
%%% mode: latex
%%% TeX-master: "clases"
%%% End:

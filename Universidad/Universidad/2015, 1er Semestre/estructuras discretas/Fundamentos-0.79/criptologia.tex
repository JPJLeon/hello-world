% criptografia.tex
%
% Copyright (c) 2009-2015 Horst H. von Brand
% Derechos reservados. Vea COPYRIGHT para detalles

\chapter{Criptología}
\label{cha:criptografia}
\index{criptologia@criptología|textbfhy}

  Debe distinguirse entre \emph{sistema criptográfico},%
    \index{sistema criptografico@sistema criptográfico|textbfhy}
  un conjunto de algoritmos diseñados para proteger secretos;
  la \emph{criptografía},%
    \index{criptografia@criptografía|textbfhy}
  el trabajo hecho para crear sistemas criptográficos;
  y finalmente \emph{criptoanálisis}%
    \index{criptoanalisis@criptoanálisis|textbfhy},
  trabajo hecho para burlar las protecciones de sistemas criptográficos.
  Se habla de \emph{criptología}%
    \index{criptologia@criptología|textbfhy}
  para referirse a la unión de criptografía y criptoanálisis.
  Es común que se confundan los términos \emph{criptografía}
  y \emph{criptología},
  nos preocuparemos de usar los términos precisos.

  La importancia práctica actual de la teoría de números%
    \index{teoria de numeros@teoría de números}
  está en sus aplicaciones a la criptografía,
  algunas de las cuales describiremos acá.
  Las técnicas mismas
  y los métodos que se han usado para romperlas
  hacen uso intensivo de conceptos de álgebra abstracta.
  Rutinariamente se hace necesario trabajar
  con enteros de cientos o miles de bits,
  o con elementos de grupos%
    \index{grupo}
  o campos finitos%
    \index{campo (algebra)@campo (álgebra)!finito}
  con números de elementos de similar envergadura.

\section{Referencias adicionales}
\label{sec:referencias-adicionales}

  Esta es un área muy amplia,
  hay algunos detalles adicionales sobre la teoría
  (y mucho sobre las aplicaciones prácticas)
  en el clásico de Anderson~\cite{anderson08:_secur_engin}.
  Sinkov~\cite{sinkov09:_elemen_crypt}
  describe métodos criptográficos elementales
  y cómo quebrarlos.
  Matt Curtin~\cite{curtin98:_snake_oil_warnin_signs}
  discute signos de alerta sobre criptografía poco confiable.
  Bernstein, Lange y Schwabe~%
    \cite{bernstein12:_sec_impact_new_crypt_lib}
  describen una biblioteca simple de usar
  para criptografía práctica,
  y discuten algunos de los ataques recientes
  basados en la operación de programas criptográficos.

  Las aplicaciones de criptología
  suelen discutirse en términos de personajes
  \(A\) (también llamada \foreignlanguage{english}{Alice})%
    \index{criptologia@criptología!Alice}
  y \(B\) (apodado \foreignlanguage{english}{Bob})%
    \index{criptologia@criptología!Bob}
  que desean intercambiar mensajes.%
    \index{mensaje}
  A veces aparecen otros actores,
  como \(C\) (alias \foreignlanguage{english}{Charlie})%
    \index{criptologia@criptología!Charlie}
  o \(E\) (\foreignlanguage{english}{Eve}),%
    \index{criptologia@criptología!Eve}
  quien desea interceptar el tráfico
  o intervenirlo de alguna forma
  (\emph{\foreignlanguage{english}{eavesdrop}}, en inglés),
  por ejemplo inyectando mensajes falsificados
  o modificando mensajes.
  Alice y Bob fueron presentados públicamente
  por Rivest, Shamir y Adleman~\cite{rivest83:_RSA},
  John Gordon~%
    \cite{gordon84:_alice_bob_after_dinner_speech}
  dio sus bibliografías definitivas.
  Schneier~\cite{schneier96:_applied_crypt}
  presenta una larga lista de otros personajes que suelen aparecer.

  No entraremos en más detalles en este amplio y complejo campo,
  nuestro interés
  es solo mostrar aplicaciones de la teoría de números
  vista hasta acá.
  Una referencia básica es el manual de Menezes y otros~%
    \cite{menezes96:_handb_applied_crypt},
  el texto de Anderson~%
    \cite{anderson08:_secur_engin}
  trata seguridad desde el punto de vista de ingeniería
  e incluye un capítulo accesible sobre el tema,
  mientras Schneier y coautores
  se concentran en aplicaciones prácticas~%
    \cite{ferguson10:_crypt_engin, schneier96:_applied_crypt}.

  Debe tenerse cuidado,
  se suele confundir la criptografía con seguridad.%
    \index{criptologia@criptología!y seguridad}
  La criptografía moderna es indispensable en muchas aplicaciones,
  pero es solo una de una variedad de herramientas
  requeridas para la seguridad computacional.

\section{Nomenclatura}
\label{sec:criptologia:nomenclatura}
\index{criptologia@criptología!nomenclatura|textbfhy}

  Si solo el destino previsto puede extraer el significado del mensaje
  se habla de \emph{confidencialidad}%
    \index{criptología!confidencialidad}.
  La \emph{integridad} del mensaje%
    \index{criptologia@criptología!integridad}
  se refiere a que el receptor puede asegurarse
  que el mensaje no ha sido alterado,
  \emph{autenticación}%
    \index{criptologia@criptología!autenticacion@autenticación}
  es que el receptor puede verificar la identidad de quien originó el mensaje,
  mientras \emph{no repudiación}%
    \index{criptologia@criptología!no repudiacion@no repudiación}
  asegura que el origen no pueda negar que envió el mensaje.

  Se habla de un mensaje en \emph{texto claro}%
    \index{criptologia@criptología!texto claro|textbfhy}
  (en inglés \emph{\foreignlanguage{english}{plaintext}})%
    \index{plaintext@\emph{\foreignlanguage{english}{plaintext}}|see{criptología!texto claro}}
  y su versión en \emph{texto cifrado}%
    \index{criptologia@criptología!texto cifrado|textbfhy}
  (en inglés \emph{\foreignlanguage{english}{cyphertext}}).%
    \index{ciphertext@\emph{\foreignlanguage{english}{ciphertext}}|see{criptología!texto cifrado}}
  Para nuestros efectos,
  podemos considerar los textos como números grandes
  (por ejemplo,
   tomando el texto claro codificado en UTF\nobreakdash-8),%
     \index{UTF-8@\texttt{UTF-8}}
  posiblemente dividido en \emph{bloques} de tamaño cómodo.
  La transformación de texto claro a texto cifrado se lleva a cabo
  mediante una \emph{función de cifrado} \(C\),%
    \index{criptologia@criptología!funcion de cifrado@función de cifrado|textbfhy}
  que toma el texto claro \(m\) y una \emph{clave} \(k\)%
    \index{criptologia@criptología!clave|textbfhy}
  para producir el respectivo texto cifrado \(c\):
  \begin{equation*}
    c = C_k(m)
  \end{equation*}
  Para descifrar el texto
  se usa la \emph{función de descifrado} \(D\)%
    \index{criptologia@criptología!funcion de descifrado@función de descifrado|textbfhy}
  con clave \(k'\):
  \begin{equation*}
    m = D_{k'}(c)
  \end{equation*}
  En el caso que \(k = k'\),
  se habla de un sistema \emph{simétrico}%
    \index{criptologia@criptología!sistema simetrico@sistema simétrico|see{criptología!clave privada}}
  o \emph{de clave privada}%
    \index{criptologia@criptología!clave privada|textbfhy}
  (claramente,
   en esta situación debe mantenerse secreta la clave).
  En el caso que \(k \ne k'\),
  obviamente habrá una relación entre las dos claves.
  Particularmente interesante
  es el caso en el que conociendo una de las dos
  es muy difícil obtener la otra.
  En tal caso,
  es perfectamente posible publicar \(k\),
  manteniendo secreta \(k'\).
  A estos sistemas se les llama \emph{de clave pública}.%
    \index{criptologia@criptología!clave publica@clave pública|textbfhy}
  Una aplicación interesante de sistemas de clave pública
  es \emph{firmas digitales}:%
    \index{criptologia@criptología!firma digital|textbfhy}%
    \index{criptologia@criptología!autenticacion@autenticación}
  Dado un mensaje \(m\),
  se calcula una función de \emph{\emph{\foreignlanguage{english}{hash}}}%
    \index{criptologia@criptología!funcion de hash@función de \emph{hash}}
  \(h(m)\) del mensaje,
  y se envía \(m\) junto con \(f = D_{k'}(h(m))\);
  quien lo recibe puede aplicar \(C_k(f)\),
  y comprobar que obtiene \(h(m)\).
  Si la función de \emph{\emph{\foreignlanguage{english}{hash}}}
  es tal que sea muy difícil construir un mensaje distinto
  que dé el mismo valor de la función,
  esto certifica
  que únicamente quien conoce \(k'\) puede originar la firma.

  El gran problema con los sistemas simétricos
  es que las partes deben tener algún canal de comunicación seguro
  mediante el cual distribuir las claves.%
    \index{criptologia@criptología!distribuir claves}
  Los sistemas de clave pública no tienen esta dificultad,
  pero por otro lado son muchísimo más demandantes en computación
  que los sistemas simétricos tradicionales.
  Luego lo que se hace normalmente
  es generar una clave para un sistema simétrico tradicional
  al azar,
  y luego usar un sistema de clave pública
  para enviarle esta clave al receptor.%
    \index{criptologia@criptología!sistemas hibridos@sistemas híbridos}

\section{Protocolo Diffie-Hellman de intercambio de claves}
\label{sec:Diffie-Hellman}
\index{Diffie-Hellman, algoritmo de|textbfhy}

  En rigor,
  el protocolo no sirve para intercambiar claves
  sino para acordar una clave entre las partes,
  pero el nombre es el tradicional.
  El algoritmo que discutiremos es de amplio uso~%
    \cite{carts01:_review_diffie_hellman},
  forma la base de mucho de lo que es seguridad en Internet.%
    \index{Internet}

  Supongamos que \foreignlanguage{english}{Alice}%
    \index{criptologia@criptología!Alice}
  y \foreignlanguage{english}{Bob}%
    \index{criptologia@criptología!Bob}
  desean acordar una clave \(K\),
  usando un medio de comunicación que no es seguro.
  La idea básica
  es que \foreignlanguage{english}{Alice} elige un primo \(p\)
  y una raíz primitiva \(g\) módulo \(p\).%
    \index{raiz primitiva@raíz primitiva}
  Ambos valores puede incluso publicarlos,
  mantenerlos en secreto
  no es necesario para la seguridad del esquema
  y pueden perfectamente reutilizarse muchas veces.
  Para generar una clave,
  \foreignlanguage{english}{Alice} elige un valor \(a\)
  (que mantiene en secreto),
  y envía \(A = g^a \bmod p\) a \foreignlanguage{english}{Bob}.
  A su vez,
  \foreignlanguage{english}{Bob} elige \(b\)
  (que también mantiene en secreto),
  y envía  \(B = g^b \bmod p\) a \foreignlanguage{english}{Alice}.
  \foreignlanguage{english}{Alice}
  calcula \(K = B^a \bmod p = g^{a b} \bmod p\),
  y \foreignlanguage{english}{Bob} obtiene \(K = A^b \bmod p\).
  Este valor puede usarse como clave por una sesión%
    \index{criptologia@criptología!clave de sesion@clave de sesión}
  y descartarse después.
  El punto es que con los algoritmos conocidos actualmente
  si \(p\) es un primo de unos \(300\)~dígitos,
  y \(a\) y \(b\) son números de \(100\)~dígitos
  es imposible hallar \(a\)
  si solo se conocen \(p\), \(g\) y \(g^a \bmod p\).
  Hay que tener cuidado con primos
  tales que \(p - 1\) tiene solo factores primos chicos,
  para ese caso hay algoritmos razonablemente eficientes
  que dan \(a\).
  Por esta razón suele elegirse un primo de Sophie Germain,%
    \index{Sophie Germain, primo de}%
    \glossary{Primo de Sophie Germain}
	     {Primo de la forma \(2 q + 1\), con \(q\) primo.}%
    \index{Germain, Sophie}
  vale decir,
  uno de la forma \(2 q + 1\) con \(q\) primo a su vez.
  Conviene trabajar en el subgrupo de orden \(q\),
  dado que de otra forma el valor de \(g^a \bmod p\)
  revela el último bit de \(a\)
  (hay formas eficientes de determinar
   si un número es o no un cuadrado
   en \(\mathbb{Z}_p\)).
  Está claro que exactamente lo mismo puede hacerse
  si \(g\) es un generador de algún otro grupo cíclico.
  La seguridad del esquema
  depende de la dificultad de calcular logaritmos discretos,%
    \index{logaritmo discreto}
  vale decir,
  dados \(g\) y \(g^a\) calcular \(a\).

  Sabemos
  (ver sección~\ref{sec:raices-primitivas})%
    \index{raiz primitiva@raíz primitiva}
  que hay \(\phi(p - 1)\) raíces primitivas módulo \(p\),
  con lo que son relativamente numerosas
  y es razonable buscar una raíz primitiva
  vía intentar valores al azar.
  Para un ejemplo numérico,
  tomemos \(p = 601\),
  con lo que \(p - 1 = 2^3 \cdot 3 \cdot 5^2\),
  y hay \(\phi(p - 1) = 160\) raíces primitivas.
  Una raíz primitiva \(g\) deberá cumplir:
  \begin{alignat*}{3}
    g^{\sfrac{600}{2}}
      &\centernot\equiv 1 \pmod{601}
    &
    g^{\sfrac{600}{3}}
      &\centernot\equiv 1 \pmod{601}
    &
    g^{\sfrac{600}{5}}
      &\centernot\equiv 1 \pmod{601} \\
  \intertext{Intentando con 31 tenemos:}
    31^{300}
      &\equiv 600 \pmod{601}
    \qquad&
    31^{200}
      &\equiv \phantom{00}1 \pmod{601}
    \qquad&
    31^{120}
      &\equiv 432 \pmod{601}
  \end{alignat*}
  La teoría anterior dice que aún requerimos algún valor \(u\)
  tal que \(u^{200} \centernot\equiv 1 \pmod{601}\)
  (ya tenemos cubiertas las potencias de \(2\) y \(5\),
   falta \(3\)),
  algunos intentos dan:
  \begin{equation*}
    357^{200}
      \equiv 576 \pmod{601}
  \end{equation*}
  con lo que \(g = 31 \cdot 357 \bmod 601 = 249\)
  es una raíz primitiva módulo \(p = 601\).

  Si ahora \foreignlanguage{english}{Alice} elige \(a = 17\),
  envía \(249^{17} \bmod 601 = 73\)
  a \foreignlanguage{english}{Bob},
  quien a su vez elige \(b = 58\)
  y envía \(249^{58} \bmod 601 = 149\)
  a \foreignlanguage{english}{Alice}.
  Ambos están ahora en condiciones de calcular \(K\):
  \foreignlanguage{english}{Alice}
  calcula \(149^{17} \bmod 601 = 366\),
  y \foreignlanguage{english}{Bob}
  obtiene \(73^{58} \bmod 601 = 366\).

  Este ejemplo muestra
  que se requiere la factorización completa de \(p - 1\)
  para obtener \(g\),
  razón por la que es imprescindible poder reusar estos valores.

\section{Sistema de clave pública de Rivest,
       Shamir y Adleman (RSA)}
\label{sec:RSA}
\index{criptologia@criptología!RSA}

  Es el sistema de clave pública más usado en la actualidad~%
     \cite{rivest78:_RSA}.
  Se eligen dos números primos \(p\) y \(q\),
  y se calcula el módulo \(n = p q\).
  Se elige además un exponente \(e\),
  y la clave pública es el par \((n, e)\).
  Para cifrar con RSA se usa:
  \begin{equation*}
    c = m^e \bmod n
  \end{equation*}
  Conociendo \(p\) y \(q\)
  podemos generar la correspondiente clave privada \((n, d)\)
  tal que:
  \begin{equation*}
    c^d \bmod n
      = \left(m^e\right)^d \bmod n
      = m
  \end{equation*}
  Por el teorema de Euler,
  si \(\gcd(m, n) = 1\)
  es \(m^{\phi(n)} \equiv 1 \pmod{n}\).
  Si \(d e \equiv 1 \pmod{\phi(n)}\),
  entonces \(c^d \equiv m^{d e} \equiv m \pmod{n}\).
  Más adelante discutiremos cómo elegir los parámetros del caso.

  Podemos elegir el exponente \(e\)
  como un número relativamente pequeño
  (recuérdese que estamos interesados en módulos grandes;
   hoy se recomienda usar módulos de \(4\,096\)~bits,
   unos \(1\,300\)~dígitos decimales)
  de forma que sea cómodo elevar a esa potencia al cifrar,%
    \index{algoritmo!potencia}
  pero el exponente de descifrado
  resultará ser un número muy grande.
  Por el teorema chino de los residuos%
    \index{residuo!teorema chino de los}
  podemos calcular módulos \(p\) y \(q\),
  o sea tenemos realmente:
  \begin{align*}
    c &\equiv m^e \pmod{p} \\
    c &\equiv m^e \pmod{q}
  \end{align*}
  Con esto requerimos que el exponente de descifrado \(d\) cumpla:
  \begin{align*}
    e d
      &\equiv 1 \pmod{(p - 1)} \\
    e d
      &\equiv 1 \pmod{(q - 1)}
  \end{align*}
  Para cumplir con ambas,
  por el teorema~\ref{theo:congruencia-mn}
  basta que:
  \begin{equation*}
    e d \equiv 1 \pmod{\lcm(p - 1, q - 1)}
  \end{equation*}

  Hoy típicamente se usan claves (módulos) de \(4\,096\) bits.
  Los primos \(p\) y \(q\)
  deben elegirse de forma de no ser demasiado cercanos,
  y ninguno de \(p - 1\), \(q - 1\), \(p + 1\) y \(q + 1\)
  debe tener muchos factores primos chicos,
  ya que de ser así \(n\) resulta relativamente fácil de factorizar.
  Puede demostrarse además
  (ver a Wiener~%
    \cite{wiener90:_crypt_short_rsa_secret_exp})
  que si \(d < n^{\sfrac{1}{4}} / 3\)
  es muy fácil recuperar \(d\) conociendo solo \(n\) y \(e\).

  Para elegir \(e\),
  una consideración importante es que sea relativamente pequeño
  y que su representación binaria tenga pocos unos,
  de forma que el cálculo de la potencia resulte simple
  (ver el algoritmo~\ref{alg:power}).
  Originalmente se recomendaba \(e = 3\),
  pero exponentes chicos hacen posibles
  ciertos ataques que consideraremos luego.
  Hoy se recomienda \(e = 2^{16} + 1 = 65\,537\)
  (un primo de Fermat,%
   \index{Fermat, primo de}%
   \glossary{Primo de Fermat}
	    {Primo de la forma \(2^{2^k} + 1\).}
   de la forma \(2^{2^k} + 1\)).
  Además de ser primo,
  este exponente tiene la virtud
  de tener solo dos unos en su expansión binaria;%
    \index{algoritmo!potencia}
  elevar a esta potencia involucra \(5\)~multiplicaciones,
  \(4\)~veces elevar al cuadrado
  y una multiplicación adicional por la base

  No hay similar control sobre \(d\),
  el exponente para descifrar.
  Usar el mínimo común múltiplo%
    \index{minimo comun multiplo@mínimo común múltiplo}
   \(\lcm(p - 1, q - 1)\) en vez de \(\phi(n)\)
  disminuye el valor,
  pero no significativamente.
  Podemos acelerar el proceso
  usando el teorema chino de los residuos,%
    \index{residuo!teorema chino de los}
  teorema~\ref{theo:chino-residuos}.
  Precalculamos valores \(d_1\), \(d_2\), \(q'\)
  (así, la clave privada es realmente \((p, q, d_1, d_2, q')\)),
  donde se cumplen las siguientes relaciones:
  \begin{equation*}
    e d_1
      \equiv 1 \pmod{(p - 1)}
    \hspace{3em}
    e d_2
      \equiv 1 \pmod{(q - 1)}
    \hspace{3em}
    q q'
      \equiv 1 \pmod{p}
  \end{equation*}
  Dado el mensaje cifrado \(c\)
  se obtiene el mensaje \(m\) mediante:
  \begin{align*}
    m_1
      &= c^{d_1} \bmod p \\
    m_2
      &= c^{d_2} \bmod q \\
    h
      &= ((m_1 - m_2) \cdot q') \bmod p \\
    m
      &= m_2 + q h
  \end{align*}
  Los cálculos de \(m_1\) y \(m_2\) se pueden efectuar en paralelo,
  e involucran exponentes y módulos mucho menores
  que en la formulación original,
  lo que hace más rápido el cálculo aún si es secuencial.

  Está claro que exactamente la misma idea es aplicable a módulos
  que son productos de más de dos primos,
  aunque en la práctica se usan solo dos
  (mientras menos factores tenga el módulo,
   más difícil es factorizarlo).

  La seguridad del sistema
  se basa en lo complejo que resulta factorizar números grandes,
  aunque hay algunas otras consideraciones~%
    \cite{boneh99:_twenty_years_attack_RSA,
	  durfee02:_crypt_rsa_using_algeb_lattice_method,
	  salah06:_mathem_attac_rsa_crypt}.
  El récord actual
  (a comienzos del 2012)
% http://www.crypto-world.com/FactorRecords.html
% http://www.loria.fr/~zimmerma/records/factor.html
  de factorización de números generales%
    \index{algoritmo de factorizacion@algoritmo de factorización!record@récord}
  es RSA\nobreakdash-\(768\)~\cite{cryptoeprint:2010:006},
  un número de \(768\)~bits
  (\(232\)~dígitos decimales)
  consumiendo el equivalente aproximado
  de \(2\,000\)~años de procesamiento
  en un Opteron a \(2,2\)\,GHz.

  Resulta que conocer \(d\)
  da una manera eficiente de factorizar \(n\).
  Podemos calcular \(k = d e - 1 = 2^s r\)
  con \(r\) impar y \(s > 0\).
  Entonces \(a^k \equiv 1 \pmod{n}\) para todo \(a\),
  y \(a^{k / 2}\) es raíz cuadrada de 1 módulo \(n\).
  Por el teorema chino de los residuos,
  \(1\) tiene cuatro raíces cuadradas módulo \(n = p q\):
  son \(x = u v\)
  donde \(u \equiv \pm 1 \pmod{p}\) y \(v \equiv \mp 1 \pmod{q}\).
  Como en la prueba de primalidad de Miller-Rabin,%
    \index{Miller-Rabin, prueba de}
  eligiendo \(a\) al azar e intentando
  \(a^r \bmod n\), \(a^{2r} \bmod n\), \ldots, \(a^{k / 2} \bmod n\)
  rápidamente hallaremos una raíz cuadrada \(x\) no trivial de 1,
  y obtenemos una factorización vía \(\gcd(x - 1, n)\).

  Un uso típico de RSA es enviar el mismo mensaje \(m\)
  a un grupo de \(k\) personas,
  donde la persona \(i\) usa clave pública \((n_i, e_i)\).
  Por simplicidad,
  supongamos \(e_i = 3\).
  Además,
  los \(n_i\) son relativamente primos a pares
  (en caso contrario,
   factorizar algunos es trivial).
  Recolectando tres mensajes cifrados \(c_i = m^3 \bmod n_i\),
  vía el teorema chino de los residuos podemos calcular
  \(c' \equiv m^3 \pmod{n_1 n_2 n_3}\),
  y como \(m < n_i\) esto da \(c' = m^3\) en \(\mathbb{Z}\),
  y basta calcular una raíz cúbica.
  Un ataque afín,
  debido a Franklin y Reiter~%
     \cite{franklin95:_linear_protoc_failure_rsa_exp_three},
  funciona si se tienen varios mensajes
  relacionados por una función lineal conocida
  (por ejemplo,
   si se ``rellena'' un mensaje corto
   agregando bits fijos o conocidos)
  cifrados con el mismo exponente.
  La complejidad del ataque es cuadrático en \(e\) y \(\log n\).
  Otro ataque,
  debido a Coppersmith, Franklin, Patarin y Reiter~%
    \cite{coppersmith96:_low_exp_rsa_relat_mesgs},
  es aplicable cuando se conocen \(e\) mensajes cifrados
  con el mismo módulo
  relacionados por polinomios conocidos.
  Por esto se sugiere usar \(e = 2^{16} + 1 = 65\,537\)
  (un primo de Fermat).

\subsection{Firma digital usando RSA}
\label{sec:firma-digital-RSA}
\index{criptologia@criptología!firma digital!RSA}

  Para firmar un mensaje usando RSA,
  se elige
  una función
    de \emph{\foreignlanguage{english}{hash}} criptográfica \(h\).
  Se envía el mensaje \(m\) junto con \(h(m)^d \bmod n\),
  cosa que solo puede hacer
  quien conozca el exponente secreto \(d\).
  Quien recibe el mensaje puede verificar la firma
  elevando al exponente público \(e\) módulo \(n\)
  y confirmando que el resultado coincide con el valor de \(h(m)\).

\section{El estándar de firma digital (DSS)}
\label{sec:DSS}
\index{criptologia@criptología!firma digital!DSA}
\index{criptologia@criptología!firma digital!DSS}

  El \emph{\foreignlanguage{english}{Digital Signature Algorithm}}
  (DSA)
  es un estándar del gobierno federal
  de Estados Unidos de Norteamérica
  para firmas digitales,
  a ser usado en el
  \emph{\foreignlanguage{english}{Digital Signature Standard}}
  (DSS),
  estándar FIPS~186~\cite{FIPS-186},
  adoptado en 1993.
  Es una modificación del esquema de ElGamal~%
    \cite{elgamal85:_public_key_crypt_signat_schem},
  Anderson y Vaudenay~\cite{anderson96:_minding_your_p_and_q}
  discuten algo de su diseño y algunos ataques.
  Hubo una revisión menor en 1996 como FIPS~186-1~\cite{FIPS-186-1},
  fue expandido en 2000
  (FIPS~186-2)
  y se rehizo completo en 2009,
  especificando algoritmos adicionales
  (FIPS~186-3~\cite{FIPS-186-3}).
  La versión actual data de 2013
  (FIPS~186-4~\cite{FIPS-186-4}).
  El algoritmo DSA tiene dos fases,
  en la primera se eligen los parámetros del algoritmo,
  que pueden compartirse entre diferentes usuarios;
  mientras la segunda calcula claves públicas y privadas
  para un usuario individual.
  Con la clave privada se firma un documento,
  y con la correspondiente clave pública
  se verifica que la firma es genuina.

\subsection{Selección de parámetros}
\label{sec:DSA-parametros}

  Se efectúan las siguientes operaciones:
  \begin{itemize}
  \item
    Se elige
    una función
    de \emph{\foreignlanguage{english}{hash}} criptográfica aprobada \(H\)
    (la versión original de DSA usaba SHA\nobreakdash-\(1\),
     actualmente también se especifica SHA\nobreakdash-\(2\)~%
       \cite{FIPS-180-3}).
    La salida de \(H\) puede truncarse al largo de la clave.
  \item
    Decida largo de clave,
    el par \((L, N)\),
    determinante para la seguridad del esquema.
    La versión actual~%
      \cite{FIPS-186-3}
    especifica pares
    \((1024, 224)\), \((2048, 224)\), \((2048, 256)\)
    y \((3072, 256)\).
  \item
    Elija un primo \(q\) de \(N\) bits.
    Nótese que \(N\) debe ser menor o igual
    al largo de la salida de \(H\).
  \item
    Elija un primo \(p\) de \(L\) bits
    tal que \(p - 1\) es múltiplo de \(q\).
  \item
    Elija \(g\),
    un número cuyo orden multiplicativo módulo \(p\) es \(q\).
    Esto se obtiene fácilmente como \(h^{(p - 1) / q}\)
    para \(1 < h < p - 1\) arbitrario,
    intentando nuevamente si el resultado es 1.
    La mayoría de los \(h\) producen lo buscado,
    suele simplemente usarse \(h = 2\).
  \end{itemize}
  Los parámetros \((p, q, g)\) pueden compartirse.

\subsection{Generar claves para un usuario}
\label{sec:DSA-claves}

  Dado un conjunto de parámetros,
  se calculan las claves pública y privada para un usuario.
  \begin{itemize}
  \item
    Elija \(x\) al azar,
    donde \(0 < x < q\).
  \item
    Calcule \(y = g^x \bmod p\).
  \end{itemize}
  La clave pública es \((p, q, g, y)\),
  la clave privada es \(x\).

\subsection{Firmar y verificar firma}
\label{sec:DSA-firmar}

  Sea \(m\) el mensaje.
  Para firmarlo,
  se procede como sigue:
  \begin{itemize}
  \item
    Genere un valor \(k\) al azar para este mensaje,
    donde \(0 < k < q\)
  \item
    Calcule \(r = (g^k \bmod p) \bmod q\),
    en el improbable caso que resulte \(r = 0\)
    elija un nuevo valor de \(k\)
  \item
    Calcule \(s = (k^{-1} \cdot (H(m) + x \cdot r)) \bmod q\).
    En el improbable caso que \(s = 0\),
    elija un nuevo valor de \(k\)
  \end{itemize}
  La firma es \((r, s)\).

  Para verificar la firma,
  se procede como sigue.
  Si no se cumplen \(0 < r < q\) y \(0 < s < q\),
  la firma se rechaza.
  Enseguida:
  \begin{itemize}
  \item
    Calcule \(w = s^{-1} \bmod q\)
  \item
    Calcule \(u_1 = H(m) \cdot w \bmod q\)
    y \(u_2 = r \cdot w \bmod q\)
  \item
    Calcule \(v = ((g^{u_1} \cdot y^{u_2} \bmod p) \bmod q)\)
  \end{itemize}
  La firma es válida si \(v = r\).

\subsection{Correctitud del algoritmo}
\label{sec:DSA-correcto}

  El algoritmo es correcto,
  en el sentido que quien verifica siempre acepta una firma válida.

  Primeramente,
  por el teorema de Fermat
  \(g^q \equiv h^{p - 1} \equiv 1 \pmod{p}\);
  como \(g > 1\) y \(q\) es primo,
  el orden de \(g\) es \(q\).
  Al firmar se calcula:
  \begin{equation*}
    s = k^{-1} \cdot (H(m) + x r) \bmod q
  \end{equation*}
  por lo que:
  \begin{align*}
    k
      &\equiv H(m) \cdot s^{-1} + x r s^{-1} \\
      &\equiv H(m) \cdot w + x r w \pmod{q}
  \end{align*}
  Como \(g\) es de orden \(q\) módulo \(p\),
  tenemos:
  \begin{align*}
    g^k
      &\equiv g^{H(m) w} y^{r w} \\
      &\equiv g^{u_1} y^{u_2} \pmod{p}
  \end{align*}
  y finalmente:
  \begin{equation*}
    r
      = (g^k \bmod p) \bmod q
      = (g^{u_1} y^{u_2} \pmod p) \bmod q
      = v
  \end{equation*}

\subsection{Ataques a DSS}
\label{sec:ataques-DSS}

  Lawson~\cite{lawson10:_dsa_req_random_value}
  indica que si tenemos dos firmas efectuadas con el mismo \(k\),
  en \(\mathbb{Z}_q\):
  \begin{align}
    r
      &= g^k \bmod p \label{eq:DSA:r} \\
    S_a
      &= k^{-1} (H(M_a) + x \cdot r) \label{eq:DSA:Sa} \\
    S_b
      &= k^{-1} (H(M_b) + x \cdot r) \label{eq:DSA:Sb}
  \end{align}
  De hallar dos firmas con el mismo \(r\),
  sabemos que se repitió \(k\);
  con \(r\) de~\eqref{eq:DSA:Sa} y~\eqref{eq:DSA:Sb}
  podemos despejar \(k\) y en consecuencia calcular \(x\).
  Un ataque similar
  es aplicable si se conocen algunos bits de \(k\),
  usando más firmas.

  El requerimiento de que \(k\) se elija al azar es crítico.
  Por ejemplo,
  el ampliamente publicitado problema de seguridad en Debian
  restringió el número de posibles \(k\) a \(32\,767\),
  lo que hace viable intentarlos todos para recuperar la clave.
  Nótese que esto no depende
  de lo seguro que haya sido el proceso de generarla,
  un único uso descuidado la revela.

% Fixme: Otros algoritmos: El Gamal
%	 Ataques a RSA ~~-> paper del caso
%	 Firmas digitales, ...

% Fixme: Recalcar que el algoritmo _no_ es todo (citar p.ej. Anderson)

\section{Otras consideraciones}
\label{sec:consideraciones-modulos}
\index{criptologia@criptología!consideraciones}

  Los algoritmos criptográficos basados en teoría de números
  usan números primos como partes de sus claves.
  En el caso de Diffie-Hellman,
  el primo usado puede publicarse,
  en caso de RSA
  es clave que los factores del módulo permanezcan secretos
  (deben generarse al azar,
   haciendo que sea difícil adivinarlos).
  Sin embargo,
  estudios recientes~%
    \cite{cryptoeprint:2012:064,
	  heninger12:_mining_your_ps_qs}
  han mostrado que una fracción no despreciable
  de las claves usadas en la práctica comparten factores,
  lo que las hace vulnerables.

  Para determinar factores comunes entre las claves,
  estos trabajos usan un truco debido a Dan Bernstein~%
    \cite{bernstein05:_factor_compr_essen_linear_time},
  quien luego da una versión mejorada~%
    \cite{bernstein04:_faster_factorization_coprimes}.
  Supongamos el conjunto de módulos
  \(m_1\), \(m_2\), \ldots, \(m_n\).
  En el caso de~\cite{cryptoeprint:2012:064}
  son \(4,7\)~millones de módulos de \(1\,024\)~bits,
  y calcular los máximos comunes divisores de todos los pares
  para detectar factores comunes está fuera de cuestión.
  Pero se puede proceder
  calculando primero \(M = m_1 \cdot m_2 \dotsm m_n\),
  y luego para cada \(m_i\) calcular
  \(\gcd(m_i, M \bmod m_i^2)\).
  Esto involucra un cálculo largo inicial para calcular \(M\),
  luego una operación costosa al calcular \(M \bmod m_i^2\)
  y determinar un máximo común divisor
  para cada módulo en el conjunto.
  Estas operaciones son razonablemente rápidas de efectuar
  en un computador común.

\section{Criptografía de curvas elípticas}
\label{sec:EC-criptography}
\index{criptologia@criptología!curvas elipticas@curvas elípticas}

  Recientemente se han introducido variantes
  de algunos métodos criptográficos
  que en vez de trabajar en el grupo \(\mathbb{Z}_p^\times\)
  usan el grupo de una curva elíptica~%
    \cite{hankerson04:_guide_ellip_curve_crypt}.
  La razón de usar curvas elípticas
  es que el problema de obtener \(k\) dados \(Q = k P\) y \(P\)
  en el grupo de una curva elíptica
  parece ser mucho más difícil de resolver
  que el problema equivalente en \(\mathbb{Z}_p^\times\).
  Eso sí que la selección de la curva
  y los demás parámetros no son triviales,
  por lo que hay curvas sugeridas~%
    \cite{secg09:_sec1,
	  secg10:_sec2,
	  NIST99:_recom_ellip_curves_feder_gover_use},
  mientras paranoicos terminales
  generarán sus propias curvas y parámetros.
  Para determinar el orden del grupo
  hay un algoritmo razonablemente eficiente
  ideado por Schoof~%
    \cite{schoof95:_count_point_ellip_curves_finit_field},
  con mejoras de Elkies y Atkin que solo circulan como borradores,
  Dewaghe~%
    \cite{dewaghe98:_remar_schoof_elkies_atkin_algor}
  describe la versión usada en la práctica.
  En~\cite{brainpool05:_stand_curves_curve_gener}
  se describe el proceso para generar curvas en detalle
  (y se dan curvas alternativas para uso criptográfico).
  En 1997 Certicom publicó una colección de desafíos~%
    \cite{certicom97:_ecc_chall},
  discusión de algoritmos relevantes
  y estimaciones del trabajo para resolverlos se dan en~%
    \cite{bailey09:_ecc2x},
  avances concretos
  respecto del menor problema
  aún sin resolver en 2012 se discuten en~%
    \cite{bailey09:_break_ecc2k-130}.

  Los algoritmos que siguen
  suponen que se acuerdan parámetros de dominio:
  El campo \(F\),
  los parámetros \(a\) y \(b\) de la curva,
  un generador \(g\) del grupo,
  el orden \(n\) de \(g\)
  (generalmente elegido como un primo),
  y el cofactor \(h\)
  (el orden del grupo de la curva dividido por el orden de \(g\),
   conviene que sea pequeño,
   menor a \(4\) y ojalá \(1\)).

\subsection{Intercambio de claves}
\label{sec:ECDH}

  Es usar la idea de Diffie-Hellman
  (ver la sección~\ref{sec:Diffie-Hellman})
  en una curva elíptica.
  En inglés
  le llaman
   \emph{\foreignlanguage{english}{Elliptic Curve Diffie-Hellman}},
  abreviado \emph{ECDH}.
  El funcionamiento es similar al algoritmo clásico.
  Para generar una clave compartida
  entre \foreignlanguage{english}{Alice}
  y \foreignlanguage{english}{Bob}
  proceden como sigue:
  \begin{enumerate}
  \item
    \foreignlanguage{english}{Alice}
    y \foreignlanguage{english}{Bob} tienen
    claves privadas \(d_A\) y \(d_B\)
    (enteros en el rango \(1\) a \(n - 1\)),
    y calculan \(Q_A = d_A g\) y \(Q_B = d_b g\)
  \item
    Intercambian \(Q_A\) y \(Q_B\) a través del medio inseguro
  \item
    \foreignlanguage{english}{Alice} calcula \(K = d_A Q_B\),
    mientras \foreignlanguage{english}{Bob}
    obtiene este valor como \(K = d_B Q_A\).
    Dado \(K = (x_k, y_k)\),
    la mayoría de los protocolos
    usan \(x_k\) para derivar la clave a ser usada.
  \end{enumerate}

  Una variante resistente a ciertos ataques es FHMQV~%
    \cite{cryptoeprint:2009:408}.

\subsection{Firmas digitales}
\label{sec:ECDSA}

  Esta es una variante del algoritmo DSA
  (sección~\ref{sec:DSS}).
  \foreignlanguage{english}{Alice} tiene una clave privada \(d_A\)
  (un entero al azar en el rango \(1\) a \(n - 1\))
  y su clave pública \(Q_A = d_A g\).
  Sea \(L_n\) el largo en bits del orden \(n\) del grupo.

  Si \foreignlanguage{english}{Alice}
  quiere enviar un mensaje \(m\)
  firmado a \foreignlanguage{english}{Bob},
  procede como sigue:
  \begin{enumerate}
  \item
    Calcula \(e = h(m)\),
    donde \(h\)
    es una función de \emph{\foreignlanguage{english}{hash}} criptográfica,
    y sea \(z\) los \(L_n\) bits más significativos de \(e\)
  \item
    Elija un entero \(k\) al azar entre \(1\) y \(n - 1\)
  \item
    Calcule \(r = x_1 \bmod n\),
    donde \((x_1, y_1) = k g\).
    Si \(r = 0\),
    vuelva al punto 2.
  \item
    Calcule \(s \equiv k^{-1}(z + r d_A) \pmod{n}\).
    Si \(s = 0\),
    vuelva al punto 2.
  \end{enumerate}
  La firma es el par \((r, s)\) resultante.

  Para verificar la firma,
  \foreignlanguage{english}{Bob}
  primero verifica la clave pública
  de \foreignlanguage{english}{Alice}:
  \begin{enumerate}
  \item
    Verifica que \(Q_A \ne 0\)
    y que sus coordenadas son válidas
  \item
    Verifica que \(Q_A\) está en la curva
  \item
    Verifica que \(n Q_A = 0\)
  \end{enumerate}
  Enseguida,
  para verificar la firma del mensaje \(m\),
  repite el cálculo que lleva a \(z\),
  luego:
  \begin{enumerate}
  \item
    Verifica que \(r\) y \(s\) estén en el rango \(1\) a \(n - 1\).
    En caso contrario,
    la firma no es válida.
  \item
    Calcula \(w = s^{-1}\) módulo \(n\),
    luego \(u_1 = z w \bmod n\) y \(u_2 = r w \bmod n\).
    Con esto determina \((x_1, y_1) = u_1 g + u_2 Q_A\).
  \end{enumerate}
  La firma es válida si \(r \equiv x_1 \pmod{n}\),
  en caso contrario no es válida.

%%% Local Variables:
%%% mode: latex
%%% TeX-master: "clases"
%%% End:

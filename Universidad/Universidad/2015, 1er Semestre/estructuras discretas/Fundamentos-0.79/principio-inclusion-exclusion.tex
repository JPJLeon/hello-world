% principio-inclusion-exclusion.tex
%
% Copyright (c) 2009-2014 Horst H. von Brand
% Derechos reservados. Vea COPYRIGHT para detalles

\chapter{Principio de inclusión y exclusión}
\label{cha:pie}
\index{inclusion y exclusion, principio de@inclusión y exclusión, principio de}
\index{principio de inclusion y exclusion@principio de inclusión y exclusión|see{inclusión y exclusión, principio de}}

  Es común querer contar el número de objetos de una colección
  que cumplen con ciertos conjuntos de características.
  Si las características de interés son muchas,
  o la colección de objetos subyacente es grande,
  necesitamos un esquema que organice y simplifique
  los cálculos.
  Veremos el planteo de Wilf~%
    \cite{wilf06:_gfology},
  que además de ser mucho más simple que el tradicional
  aprovecha de buena forma
  lo que hemos aprendido de funciones generatrices.
  Con esto cerramos el estudio de las técnicas fundamentales
  de la combinatoria.

\section{El problema general}
\label{sec:PIE-problema-general}

  Concluimos en el capítulo~\ref{cha:combinatoria-elemental}
  que
  \(\lvert \mathcal{A} \cup \mathcal{B} \rvert
      = \lvert \mathcal{A} \rvert
	   + \lvert \mathcal{B} \rvert
	   - \lvert \mathcal{A} \cap \mathcal{B} \rvert
  \).
  Interesa generalizar para más conjuntos.
  La figura~\ref{fig:PIE} muestra tres conjuntos
  y sus intersecciones.
  \begin{figure}[htbp]
    \centering
    \pgfimage{images/PIE}
    \caption{Intersecciones entre tres conjuntos}
    \label{fig:PIE}
  \end{figure}
  Calcular
    \(\lvert \mathcal{A} \cup \mathcal{B} \cup \mathcal{C} \rvert\)
  es contar los elementos
  que pertenecen al menos a uno de los conjuntos.
  Comenzamos con \(\lvert \mathcal{A} \rvert
		     + \lvert \mathcal{B} \rvert
		     + \lvert \mathcal{C} \rvert\).
  Las intersecciones se cuentan dos veces,
  debemos restar
  \(\lvert \mathcal{A} \cap \mathcal{B} \rvert
      + \lvert \mathcal{A} \cap \mathcal{C} \rvert
      + \lvert \mathcal{B} \cap \mathcal{C} \rvert\).
  Hemos restado
    \(\lvert \mathcal{A}
	\cap \mathcal{B}
	\cap \mathcal{C} \rvert\) demás,
  debemos restituirlo:
  \begin{equation*}
    \lvert \mathcal{A} \cup \mathcal{B} \cup \mathcal{C} \rvert
      = \bigl(
	  \lvert \mathcal{A} \rvert
	     + \lvert \mathcal{B} \rvert
	     + \lvert \mathcal{C} \rvert
	 \bigr)
	   - \bigl(
	       \lvert \mathcal{A} \cap \mathcal{B} \rvert
		  + \lvert \mathcal{A} \cap \mathcal{C} \rvert
		  + \lvert \mathcal{B} \cap \mathcal{C} \rvert
	     \bigr)
	   + \lvert
	       \mathcal{A} \cap \mathcal{B} \cap \mathcal{C}
	     \rvert
  \end{equation*}
  El número de elementos
  que pertenecen exactamente a uno de los conjuntos es:
  \begin{equation*}
    \left(
      \lvert \mathcal{A} \rvert
	+ \lvert \mathcal{B} \rvert
	+ \lvert \mathcal{C} \rvert
    \right)
      - 2 \cdot \left(
	    \lvert \mathcal{A} \cap \mathcal{B} \rvert
	       + \lvert \mathcal{A} \cap \mathcal{C} \rvert
	       + \lvert \mathcal{B} \cap \mathcal{C} \rvert
	  \right)
      + 3 \cdot \lvert
		  \mathcal{A} \cap \mathcal{B} \cap \mathcal{C}
		\rvert
  \end{equation*}
  Al sumar los tamaños de los tres conjuntos
  incluimos dos veces las intersecciones en pares
  y tres veces la intersección entre los tres,
  debemos restarlas;
  al restar dos veces las tres intersecciones a pares
  estamos restando seis veces
  la intersección entre los tres conjuntos,
  debemos reponerla tres veces.
  Al resultado general
  se le llama \emph{principio de inclusión y exclusión},
  porque incluimos demás,
  y luego corregimos excluyendo.

  El tratamiento que sigue no es tradicional,
  seguimos a Wilf~\cite{wilf06:_gfology}%
    \index{Wilf, Herbert S.}
  Tomamos un conjunto universo,
  y los conjuntos que consideramos
  se representan mediante propiedades%
    \index{inclusion y exclusion, principio de@inclusión y exclusión, principio de!propiedades|textbfhy}
  (un elemento pertenece a uno de los conjuntos
   si tiene la propiedad que representa a ese conjunto).
  Las diversas intersecciones quedan expresadas
  a través de los elementos
  que tienen todas las propiedades
  correspondientes a los conjuntos intersectados.

  Sean:
  \begin{description}
  \item [\(\Omega\):]
    El universo.
    Un conjunto de objetos.
  \item [\(\mathcal{P}\):]
    Un conjunto de propiedades que los objetos pueden tener.
  \item [\(\mathcal{S}\):]
    Un subconjunto de las propiedades,
    \(\mathcal{S} \subseteq \mathcal{P}\).
  \item [\(N ( \supseteq \mathcal{S} )\):]
    Número de objetos con las propiedades en \(\mathcal{S}\)
    (puedan tener otras).
  \end{description}

  Para \(r \ge 0\) definimos:
  \begin{equation}
    \label{eq:PIE:definicion-Nr}
    N_r
      = \sum_{\lvert \mathcal{S} \rvert = r}
	  N(\supseteq \mathcal{S})
  \end{equation}
  Esto es la suma del tamaño de los conjuntos de objetos
  con al menos \(r\) de las propiedades.
  El conjunto de los objetos con al menos cero propiedades
  es el universo,
  o sea \(N_0 = \lvert \Omega \rvert\).
  Si hay \(r\) propiedades en total,
  \(N_r\) es el número de objetos con todas las propiedades.
  Estas cantidades,
  que suelen ser mucho más fáciles de calcular que lo que buscamos,
  las relacionaremos con el número de objetos
  que tienen exactamente \(t\) de las propiedades.

  Denote \(\omega \in \Omega\) un objeto,
  y llamemos \(P(\omega)\) al conjunto de propiedades de \(\omega\).
  Entonces:
  \begin{equation}
    \label{eq:PIE:Nr-objetos}
    N_r
      = \sum_{\lvert \mathcal{S} \rvert = r}
	  N(\supseteq \mathcal{S})
      = \sum_{\lvert \mathcal{S} \rvert = r}
	  \biggl(
	    \sum_{\substack{
	     \omega \in \Omega \\
	     \mathcal{S} \subseteq \mathcal{P}(\omega)
	  }} 1
	  \biggr)
      = \sum_{\omega \in \Omega}
	  \biggl(
	    \sum_{\substack{
	     \mathcal{S} \subseteq \mathcal{P}(\omega) \\
	     \lvert \mathcal{S} \rvert = r
	  }} 1
	  \biggr)
      = \sum_{\omega \in \Omega}
	   \binom{\lvert \mathcal{P}(\omega) \rvert}{r}
  \end{equation}
  En español dice:
  Si el objeto
  tiene un total
  de \(\lvert \mathcal{P}(\omega) \rvert\) propiedades,
  puedo elegir \(r\) de sus propiedades
  de \(\binom{\lvert \mathcal{P}(\omega) \rvert}{r}\) formas.

  Ahora sea \(e_t\)
  el número de objetos con exactamente \(t\) propiedades,
  es decir:
  \begin{equation}
    \label{eq:PIE:definiciom-et}
    e_t
      = \sum_{\lvert \mathcal{P}(\omega) \rvert = t} 1
  \end{equation}
  Cada uno de los \(e_t\) objetos con \(t\) propiedades
  aporta lo mismo a \(N_r\)
  (se considera una vez
   por cada subconjunto de \(r\) de sus propiedades):
  \begin{equation}
    \label{eq:PIE:Nr-et}
    N_r
      = \sum_{\omega \in \Omega}
	  \binom{\lvert \mathcal{P}(\omega) \rvert}{r}
      = \sum_{t \ge 0} \binom{t}{r} \, e_t
  \end{equation}
  Del sistema lineal~\eqref{eq:PIE:Nr-et}
  se busca
  despejar los \(e_t\).
  Para esta tarea definimos las funciones generatrices:%
    \index{generatriz!ordinaria}%
    \index{inclusion y exclusion, principio de@inclusión y exclusión, principio de!generatrices|textbfhy}
  \begin{align}
    \label{eq:PIE:definicion-E}
    E(z)
      &= \sum_{t \ge 0} e_t z^t \\
    \label{eq:PIE:definicion-N}
    N(z)
      &= \sum_{r \ge 0} N_r z^r
  \end{align}
  Substituyendo la expresión~\eqref{eq:PIE:Nr-et} para \(N_r\)
  en la definición~\eqref{eq:PIE:definicion-N} de \(N(z)\):
  \begin{equation*}
    N(z)
      = \sum_{r \ge 0} N_r z^r
      = \sum_{r \ge 0}
	  \biggl(
	    \sum_{t \ge 0} \binom{t}{r} \, e_t
	  \biggr) z^r
      = \sum_{t \ge 0} e_t
	  \biggl(\,
	    \sum_{r \ge 0} \binom{t}{r} z^r
	  \biggr)
      = \sum_{t \ge 0} e_t (1 + z)^t
      = E(1 + z)
  \end{equation*}
  De acá se tiene la fórmula central:
  \begin{equation}
    \label{eq:PIE:central}
    \index{inclusion y exclusion, principio de@inclusión y exclusión, principio de!formula central@fórmula central|textbfhy}
    E(z)
      = N(z - 1)
  \end{equation}
  De la expresión~\eqref{eq:PIE:central}
  podemos extraer el \(e_t\) que se quiera.
  Usando el teorema de Maclaurin:
  \begin{align}
    e_t
      &= \frac{1}{t!} E^{(t)}(0) \\
      &= \frac{1}{t!} N^{(t)}(-1) \\
      &= \frac{1}{t!} \sum_{r \ge 0}
			r^{\underline{t}} N_r (-1)^{r - t} \\
      &= \sum_{r \ge 0} (-1)^{r - t} \binom{r}{t} N_r
	   \label{eq:PIE:clasico}
	   \index{inclusion y exclusion, principio de@inclusión y exclusión, principio de!formula clasica@fórmula clásica}
  \end{align}
  La fórmula~\eqref{eq:PIE:clasico}
  expresa el celebrado principio de inclusión y exclusión.
  Además de ser mucho más simple que la demostración tradicional,
  nuestro desarrollo
  no hace necesario recordar esta engorrosa fórmula,
  da las herramientas para deducirla sin mayor esfuerzo
  cada vez que la necesitemos,
  y en muchos casos obtener los resultados buscados directamente
  sin tener que recurrir a ella explícitamente,
  usando la función generatriz \(E(z)\).

  Esta técnica es sencilla de aplicar
  cuando se buscan los que no tienen ninguna de las propiedades,
  conviene tratar de ajustar
  la definición de las propiedades de forma adecuada.
  También es crítico que el cálculo de los \(N(\supseteq S)\)
  y,
  en consecuencia,
  de los \(N_r\),
  sea simple,
  cosa que nuevamente depende de la elección de las propiedades.

  Volvamos al ejemplo de tres conjuntos,
  donde nos interesa saber cuántos elementos
  pertenecen exactamente a uno de ellos,
  o sea \(e_1\),
  como en la figura~\ref{fig:PIE}.
  En tal caso:
  \begin{alignat*}{2}
    N_0
      &= \lvert \Omega \rvert
    &
    N_1
      &= \lvert \mathcal{A} \rvert
	   + \lvert \mathcal{B} \rvert
	   + \lvert \mathcal{C} \rvert \\
    N_2
      &= \lvert \mathcal{A} \cap \mathcal{B} \rvert
	   + \lvert \mathcal{A} \cap \mathcal{C} \rvert
	   + \lvert \mathcal{B} \cap \mathcal{C} \rvert
    \hspace{4em}&
    N_3
      &= \lvert \mathcal{A} \cap \mathcal{B} \cap \mathcal{C} \rvert
  \end{alignat*}
  Resultan ser:
  \begin{align*}
    N(z)
      &= N_0 + N_1 z + N_2 z^2 + N_3 z^3 \\
    E(z)
      &= (N_0 - N_1 + N_2 - N_3)
	   + (N_1 - 2 N_2 + 3 N_3) z
	   + (N_2 - 3 N_3) z^2
	   + N_3 z^3
  \end{align*}
  Hay \(e_1 = N_1 - 2 N_2 + 3 N_3\) elementos
  que pertenecen a exactamente un conjunto,
  como dedujimos antes.

  Típicamente interesa saber cuántos de los objetos
  no tienen ninguna de las propiedades,
  lo que en nuestro caso es
  \(\overline{(\mathcal{A} \cup \mathcal{B} \cup \mathcal{C})}\).
  O sea,
  \(e_0 = E(0) = N(-1) = N_0 - N_1 + N_2 - N_3\).

  La unión de todos los conjuntos,
  en este caso \(\mathcal{A} \cup \mathcal{B} \cup \mathcal{C}\),
  la componen los que pertenecen al menos a uno de los conjuntos,
  vale decir,
  todos menos los que no pertenecen a ninguno:
  \begin{equation*}
    \lvert \mathcal{A} \cup \mathcal{B} \cup \mathcal{C} \rvert
      = \sum_{t \ge 0} e_t - e_0
      = E(1) - E(0)
      = N(0) - N(-1)
      = N_1 - N_2 + N_3
  \end{equation*}
  Nuevamente coincide con lo que obtuvimos antes.

  Si solo interesa calcular
  el número promedio de propiedades por objeto,
  como \(t = \binom{t}{1}\) resulta:
  \begin{equation}
     \label{eq:E-t}
     \index{inclusion y exclusion, principio de@inclusión y exclusión, principio de!numero promedio de propiedades@número promedio de propiedades}
    \E[t]
      = \frac{\sum_{t \ge 0} t e_t}{\sum_{t \ge 0} e_t}
      = \frac{N_1}{N_0}
  \end{equation}
  Para calcular la varianza del número de propiedades,
  partimos de:
  \begin{equation*}
    \var[t]
      = \E[t^2] - \left( \E[t] \right)^2
  \end{equation*}
  Como \(\binom{t}{2} = (t^2 - t) / 2\):
  \begin{align*}
     \label{eq:Var-t}
     \index{inclusion y exclusion, principio de@inclusión y exclusión, principio de!varianza del numero de propiedades@varianza del número de propiedades}
    \frac{N_2}{N_0}
      &= \frac{\sum_{t \ge 0} \binom{t}{2} e_t}{\sum_{t \ge 0} e_t} \\
      &= \frac{\sum_{t \ge 0} (t^2 - t) e_t}{2 \sum_{t \ge 0} e_t} \\
      &= \frac{1}{2} \left( \E[t^2] - \E[t] \right)
  \end{align*}
  con lo cual:
  \begin{equation}
    \var[t]
      = \frac{2 N_2}{N_0} + \frac{N_1}{N_0} - \frac{N_1^2}{N_0^2}
  \end{equation}

\subsubsection*{Receta}
\index{inclusion y exclusion, principio de@inclusión y exclusión, principio de!receta}

  \begin{enumerate}
  \item
    Definir \(\Omega\) y \(\mathcal{P}\),
    expresar lo que se busca en términos de \(e_t\).
  \item
    Calcular los \(N(\supseteq \mathcal{S})\).
  \item
    Calcular los \(N_r\),
    y en consecuencia obtener \(N(z)\).
  \item
    \(e_t = \left[ z^t \right] N(z - 1)\).
  \end{enumerate}

  Hay que tener cuidado con esto,
  acá las series deben converger
  para que nuestras operaciones tengan sentido.
  Normalmente el número de propiedades y objetos de interés
  es finito,
  así que en realidad estamos manipulando polinomios
  y no hay problemas.

% Fixme: Agregar ejemplos "típicos" (¿tarea?)

  \begin{example}
    \index{inclusion y exclusion, principio de@inclusión y exclusión, principio de!ejemplo!cursos}
    Un curso rinde pruebas
    con los profesores Ellery, Upham y Atwood.
    Nos dicen que \(10\) aprobaron la prueba de física,
    \(15\) la de matemáticas y
    \(12\) pasaron la prueba de química;
    \(6\) pasaron física y matemáticas,
    \(5\) pasaron física y química,
    mientras \(8\) pasaron matemáticas y química.
    El total de estudiantes
    que pasaron al menos una prueba es \(20\).
    ¿Cuántos pasaron las tres pruebas?

    Aplicamos nuestra receta:
    \begin{enumerate}
    \item
      El universo es el grupo de \(20\) estudiantes
      que aprobaron alguna de las pruebas,
      las propiedades son las pruebas aprobadas (\(F, M, Q\)).
      Interesan los que aprobaron todas las pruebas,
      o sea \(e_3\).
    \item
      Los \(N(\supseteq \mathcal{S})\) están dados
      en el enunciado.
      Por ejemplo,
      dice que \(N(\supseteq \{F, Q\}) = 5\).
    \item
      Como se comentó antes,
      al haber \(3\) propiedades es \(N_3 = e_3\).
      Tenemos:
      \begin{align*}
	N_0
	  &= \lvert \Omega \rvert = 20 \\
	N_1
	  &= N(\supseteq \{F\})
	       + N(\supseteq \{M\})
	       + N(\supseteq \{Q\})
	   = 10 + 15 + 12
	   = 37 \\
	N_2
	  &= N(\supseteq \{F, M\})
	       + N(\supseteq \{F, Q\})
	       + N(\supseteq \{M, Q\})
	   = 6 + 5 + 8
	   = 19 \\
	N_3
	  &= N(\supseteq \{F, M, Q\})
	   = e_3
      \end{align*}
      Resulta:
      \begin{equation*}
	N(z) = 20 + 37 z + 19 z^2 + e_3 z^3
      \end{equation*}
    \item
      Como todos los estudiantes del universo
      han aprobado al menos una de las pruebas:
      \begin{equation*}
	e_0 = 0 = E(0) = N(-1) = 2 - e_3
      \end{equation*}
      Con esto resulta \(e_3 = 2\).

      Pero también tenemos:
      \begin{align*}
	E(z)
	  &= N(z - 1) \\
	  &= 5 z + 13 z^2 + 2 z^3
      \end{align*}
      lo que dice que \(5\) aprobaron una única prueba
      y que \(13\) aprobaron dos.
    \end{enumerate}
  \end{example}

  \begin{example}
    \index{inclusion y exclusion, principio de@inclusión y exclusión, principio de!ejemplo!numeros con ceros par@números con ceros par}
    ¿Cuántos números de largo \(n\) escritos en decimal
    tienen un número par de ceros?

    Para tener valores con los cuales contrastar,
    analicemos algunos casos.
    Anotamos \(9\) para un dígito no cero,
    y \(0\) para un cero en el cuadro~\ref{tab:par-0},
    y contamos cuántos de cada tipo hay.
    \begin{table}[htbp]
      \centering
      \begin{tabular}{|>{\(}r<{\)}|>{\(}l<{\)}|>{\(}r<{\)}|}
	\hline
	\multicolumn{1}{|c|}
		    {\rule[-0.7ex]{0pt}{3ex}\(\boldsymbol{n}\)} &
	  \multicolumn{1}{c|}{\textbf{Descripción}} &
	  \multicolumn{1}{c|}{\textbf{Nº}} \\
	\hline\rule[-0.7ex]{0pt}{3ex}%
	  1 & 9				&      9 \\
	  2 & 99			&     81 \\
	  3 & 999 + 900			&    738 \\
	  4 & 9999 + 9900 + 9090 + 9009 & 6\,804 \\
	\hline
      \end{tabular}
      \caption{Posibilidades con un número par de ceros}
      \label{tab:par-0}
    \end{table}

    El universo
    es el conjunto de todos los números con \(n\) dígitos.
    La propiedad \(i\)
    es que el dígito \(i\)\nobreakdash-ésimo es cero.
    Lo que interesa entonces es:
    \begin{equation*}
      e_0 + e_2 + \dotsb
	= \sum_{r \ge 0} e_{2 r}
    \end{equation*}

    Podemos extraer
    únicamente los términos con potencia par mediante:%
      \index{serie de potencias!decimar}
    \begin{equation*}
      \frac{E(z) + E(-z)}{2}
	= \sum_{r \ge 0} e_{2 r} z^{2 r}
    \end{equation*}
    y nuestra suma no es más que:
    \begin{equation*}
      \frac{E(1) + E(-1)}{2}
    \end{equation*}
    Esto es válido,
    ya que estamos trabajando con polinomios.

    Un número decimal de \(n\) dígitos
    comienza con un dígito no cero,
    los demás \(n - 1\) dígitos pueden ser cualquiera.
    En este caso podemos calcular los \(e_r\) directamente,
    observando que hay \(r\) posiciones para los ceros,
    elegidas de entre \(n - 1\) posiciones,
    los otros \(n - r\) dígitos
    (incluyendo el primero)
    pueden tomar uno de los \(9\) valores restantes:
    \begin{align*}
      e_r
	&= \binom{n - 1}{r} \cdot 9^{n - r} \\
      E(z)
	&= 9 \cdot \sum_{r \ge 0}
		     \binom{n - 1}{r} \cdot 9^{n - 1 - r} \cdot z^r
	 = 9 \cdot (9 + z)^{n - 1}
    \end{align*}
    y el número buscado resulta ser:
    \begin{equation*}
      \frac{1}{2} \, \left( E(1) + E(-1) \right)
	= \frac{1}{2} \, \left(
		     9 \cdot (9 + 1)^{n - 1}
		       + 9 \cdot (9 - 1)^{n - 1}
		  \right)
	= \frac{9}{2} \, \left(10^{n - 1} + 8^{n - 1}\right)
    \end{equation*}
    Esto coincide con los valores calculados antes,
    cuadro~\ref{tab:par-0}.
  \end{example}

  \begin{example}
    \index{inclusion y exclusion, principio de@inclusión y exclusión, principio de!ejemplo!lanzar \(10\) dados}
    Se lanzan \(10\)~dados.
    ¿De cuántas maneras puede hacerse esto
    tal que aparezcan todas las caras?

    Es más fácil calcular el número de lanzamientos
    en los que una cara dada \emph{no} aparece,
    lo que a su vez lleva naturalmente a contar el número de maneras
    en que no falta ninguna cara.
    Aplicando la receta:
    \begin{description}
    \item[\boldmath\(\Omega\)\unboldmath:]
      El conjunto de todos los lanzamientos posibles de \(10\)~dados
    \item[Propiedades:]
      Un lanzamiento tiene la propiedad \(k\) si la cara \(k\) no aparece
    \item[Resultado:]
      Interesan los lanzamientos sin propiedades
    \end{description}
    Si en un lanzamiento las caras en \(\mathcal{S}\) no aparecen,
    quiere decir que es una secuencia de largo~\(10\)
    de las restantes \(6 - \lvert \mathcal{S} \rvert\) caras:
    \begin{equation*}
      N(\supseteq \mathcal{S})
	= (6 - \lvert \mathcal{S} \rvert)^{10}
    \end{equation*}
    Como \(\mathcal{S}\) se elige entre \(6\) posibilidades:
    \begin{equation*}
      N_r
	= \binom{6}{r} (6 - r)^{10}
    \end{equation*}
    Tenemos:
    \begin{equation*}
      N(z)
	= \sum_{r \ge 0} \binom{6}{r} (6 - r)^{10} z^r
    \end{equation*}
    De la fórmula mágica:
    \begin{align*}
      e_0
	&= E(0) \\
	&= N(-1) \\
	&= \sum_{r \ge 0} (-1)^r \binom{6}{r} (6 - r)^{10} \\
	&= 16\,435\,440
    \end{align*}
  \end{example}

\section{Desarreglos}
\label{sec:desarreglos-pie}
\index{desarreglo}

  Un \emph{punto fijo} de una permutación \(\pi\)%
    \index{permutacion@permutación!punto fijo}
  ocurre cuando su elemento número \(k\) es \(k\)
  (vale decir,
   \(\pi(k) = k\)).
  Un \emph{desarreglo}
  (en inglés \emph{\foreignlanguage{english}{derangement}})%
    \index{derangement@\emph{\foreignlanguage{english}{derangement}}|see{desarreglo}}
  es una permutación sin puntos fijos.
  \glossary{Desarreglo}{Permutación sin puntos fijos}
  \glossary{Punto fijo (de una permutación)}
    {Elemento que no cambia de posición}
  El primero en calcular el número de desarreglos
  fue \foreignlanguage{french}{Pierre R. de Montmort}
  (1678--1719)~%
    \index{Montmort, Pierre R.}%
    \cite{montmort08:_jeux_hazard}.
  Hathout~%
    \index{Hathout, Heba}%
    \cite{hathout03:_old_hats_probl, hathout04:_old_hats_probl_revis}
  presenta varias soluciones
  (incluyendo la presente,
   debida esencialmente a Nicolaus Bernoulli%
     \index{Bernoulli, Nicolaus}).

  Siguiendo nuestra receta:
  \begin{enumerate}
  \item \(\Omega\):
    El universo son las \(n!\) permutaciones de \(n\) elementos.
    La permutación \(\pi\) tiene la propiedad \(i\)
    si \(i\) es un punto fijo en ella.
    Interesa obtener \(e_0\).
  \item
    Sea \(\mathcal{S} \subseteq \{1, \dotsc, n\}\).
    Entonces \(N(\supseteq \mathcal{S})\)
    corresponde a las permutaciones
    para las cuales los elementos de \(\mathcal{S}\) son fijos,
    solo se pueden ``mover''
    los \(n - \lvert \mathcal{S} \rvert\) restantes:
    \begin{equation*}
      N(\supseteq \mathcal{S})
	= \left(n - \lvert \mathcal{S} \rvert\right)!
    \end{equation*}
  \item
    Como \(r\) puntos fijos
    pueden elegirse de \(\binom{n}{r}\) maneras,
    se tiene que:
    \begin{equation}
      \label{eq:fixed-points-Nr}
      N_r
	= \sum_{\lvert \mathcal{S} \rvert = r}
	    N(\supseteq \mathcal{S})
	= \sum_{\lvert \mathcal{S} \rvert = r} (n - r)!
	= \binom{n}{r} \cdot (n - r)!
    \end{equation}
    Con esto:
    \begin{equation}
      \label{eq:fixed-points-FG}
      N(z)
	= \sum_{0 \le r \le n} \binom{n}{r} (n - r)! \, z^r
	= \sum_{0 \le r \le n}
	     \frac{n!}{r!(n - r)!} \cdot (n - r)! \cdot z^r
	= n! \, \sum_{0 \le r \le n} \frac{z^r}{r!}
    \end{equation}
    En términos de la función exponencial truncada:%
      \index{serie de potencias!exponencial}
    \begin{equation}
      \label{eq:exp-truncada}
      \exp \rvert_n (z)
	= \sum_{0 \le k \le n} \frac{z^k}{k!}
    \end{equation}
    la ecuación~\eqref{eq:fixed-points-FG} es:
    \begin{equation*}
      N(z) = n! \cdot \exp \rvert_n (z)
    \end{equation*}
  \item
    En particular,
    \(e_0 = E(0) = N(-1)\)
    es el número de desarreglos de \(n\) elementos:
    \begin{align}
      D_n
	&=	   n! \exp \rvert_n (-1)
	    \label{eq:n-derangements} \\
	&\approx n! \, \mathrm{e}^{-1}
	    \label{eq:n-derangements-approx}
      \index{desarreglo!numero de, formula@número de, fórmula}
    \end{align}

    Consideremos la serie de Maclaurin%
      \index{Maclaurin, teorema de}
    para \(\mathrm{e}^x\)
    con el resto en la forma de Lagrange:%
      \index{Maclaurin, teorema de!Lagrange, forma del resto}
    \begin{equation*}
      \mathrm{e}^{-1}
	= \sum_{0 \le k \le n} \frac{(-1)^k}{k!}
	    + \frac{(-1)^{n + 1}}{(n + 1)!}
	    + \frac{\mathrm{e}^{-\xi}}{(n + 2)!} \, (-1)^{n + 2}
      \qquad (0 < \xi < 1)
    \end{equation*}
    Tenemos las cotas para el valor absoluto
    del error que comete
    la fórmula \(D_n \approx n! \mathrm{e}^{-1}\):
    \begin{align*}
      n ! \left(
	    \frac{1}{(n + 1)!} - \frac{\mathrm{e}^0}{(n + 2)!}
	  \right)
	&< \epsilon_n
	 < n! \left(
		\frac{1}{(n + 1)!}
		  - \frac{\mathrm{e}^{-1}}{(n + 2)!}
	      \right) \\
      \frac{1}{n + 2}
	&< \epsilon_n
	 < \frac{n + 2 - \mathrm{e}^{-1}}{(n + 1) (n + 2)}
	 < \frac{1}{n + 1}
    \end{align*}
    Para \(n \ge 1\) el error absoluto es menor a \(1 / 2\),
    y \(D_n\) es el entero más cercano a \(n! e^{-1}\).
    Algo como un \(37\)\%
    de las permutaciones no tienen puntos fijos.
    Es curioso que este resultado dependa tan poco de \(n\).

    Más en general,
    tenemos también:
    \begin{align}
      e_t
	&= n! \, \left[ z^t \right] \, \sum_{0 \le r \le n}
					 \frac{(z - 1)^r}{r!}
	 = n! \, \left[ z^t \right] \,
		   \sum_{0 \le r \le n}
		     \sum_{0 \le k \le r}
		       \frac{1}{r!} \,
			 \binom{r}{k} \, z^k \,
			 (-1)^{r - k} \notag \\
	&= n! \, \sum_{0 \le r \le n}
		   \frac{1}{r!} \binom{r}{t} \, (-1)^{r - t}
	 = n! \, \sum_{t \le r \le n}
		   \frac{(-1)^{r - t}}{t! (r - t)!}
	 = \frac{n!}{t!} \,
	      \sum_{0 \le r \le n - t} \frac{(-1)^r}{r!} \notag \\
	&= \frac{n!}{t!} \, \exp \rvert_{n - t} (-1)
	\label{eq:t-puntos-fijos}
    \end{align}
    Por~\ref{eq:E-t} y~\ref{eq:fixed-points-Nr}
    el número promedio de puntos fijos es:
    \begin{equation*}
      \bar{t}
	= \frac{N_1}{N_0}
	= \frac{\binom{n}{1} (n - 1)!}{n!}
	= 1
    \end{equation*}
    Curiosamente no depende de \(n\).
  \end{enumerate}

\section{El problema de ménages}
\label{sec:menages}
\index{problème des ménages@\emph{\foreignlanguage{french}{problème des ménages}}}

  Lucas~\cite{lucas91:_theo_nombres} en~1891
  planteó el problema de sentar \(n\) parejas en una mesa circular,
  alternando hombres y mujeres
  de forma que ninguna pareja se sentara junta.
  Este ``problème des ménages''
  fue resuelto recién en~1934 por Touchard~%
    \cite{touchard34:_permutations},
  pero sin dar una demostración.
  La primera demostración de la fórmula de Touchard
  fue dada por Kaplansky~\cite{kaplansky43:_menages} en~1943,
  claro que al insistir en ubicar primero a las damas
  y luego ordenar a los caballeros
  lleva a un desarrollo innecesariamente complicado.
  Seguimos a Bogart y~Doyle~%
    \cite{bogart86:_non_sexist_soln_menage_probl}
  en nuestra derivación.

  Observamos primero que \(n\) parejas
  pueden ubicarse de \(2 (n!)^2\)~maneras alrededor de la mesa
  (hay dos maneras de elegir los asientos de las mujeres,
   luego ordenamos a damas y caballeros).
  Aplicamos el principio de inclusión y exclusión,
  numerando las parejas de \(1\) a~\(n\),
  y definiendo que una distribución tiene la propiedad \(i\)
  si la pareja \(i\) se sienta junta.
  Si \(W_k\) es el número de maneras de sentar las parejas
  de manera que \(k\) parejas dadas se sientan juntas,
  tenemos:
  \begin{align}
    \label{eq:menage:N}
    N_r
      = \binom{n}{r} W_r
  \end{align}
  de la fórmula mágica resulta:
  \begin{equation}
    \label{eq:menages:M}
    M_n
      = \sum_{0 \le r \le n} (-1)^r \binom{n}{r} W_r
  \end{equation}
  donde:
  \begin{equation}
    \label{eq:menages:W}
    W_k
      = 2 \cdot d_k \cdot k! \cdot \left( (n - k)! \right)^2
  \end{equation}
  (debemos elegir asientos de las mujeres y hombres,
   dónde se ubican las parejas que se sientan juntas,
   y finalmente cómo se distribuyen las \(n - k\)~damas
   y los \(n - k\)~caballeros que no forman parte de las parejas).
  Acá \(d_k\)
  es el número de formas de ubicar \(k\)~piezas de dominó
  sobre un ciclo de~\(2 n\)
  sin que se traslapen,
  \begin{figure}[ht]
    \centering
    \pgfimage{images/menages}
    \caption{Dominós no traslapados en ciclo}
    \label{fig:menage}
  \end{figure}
  ver la figura~\ref{fig:menage} para un ejemplo.

  Si cortamos el círculo,
  hay dos posibilidades:
  Una de las piezas de dominó se corta en dos,
  queda ubicar \(k - 1\) de ellas en línea
  sobre \(2 n - 2\) asientos;
  o ninguna de las piezas se corta,
  debemos ubicar \(k\) piezas sobre una línea de \(2 n\) asientos.
  Si llamamos \(f(n, k)\) al número de formas
  de ubicar \(k\) piezas en una línea de \(n\) asientos,
  esto resulta de \(k + 1\) bloques entre piezas de dominó,
  entre los cuales se distribuyen
  las \(n - 2 k\) posiciones no cubiertas:
  \begin{align}
    f(n, k)
      &= \multiset{k + 1}{n - 2 k} \notag \\
      &= \binom{n - k}{k}
	    \label{eq:menage:f}
  \end{align}
  Con~\eqref{eq:menage:f} resulta:
  \begin{align}
    d_k
      &= \binom{2 n - k}{k} + \binom{2 n - 2 - (k - 1)}{k - 1}
	      \notag \\
      &= \frac{(2 n - k)!}{k! (2 n - 2 k)!}
	   + \frac{(2 n - k - 1)!}{(k - 1)! (2 n - 2 k)!}
	      \notag \\
      &= \frac{(2 n - k - 1)!}{(k - 1)! (2 n - 2 k)!}
	   \left(
	     \frac{2 n - k}{k}
	       + 1
	   \right)
	      \notag \\
      &= \frac{2 n}{2 n - k} \binom{2 n - k}{k}
	      \label{eq:menage:d}
  \end{align}
  Uniendo las relaciones~\eqref{eq:menages:M} a~\eqref{eq:menage:f},
  como por simetría \(M_n\) debe ser divisible por \(2 n!\),
  al simplificar resulta la fórmula de Touchard:%
    \index{Touchard, formula de@Touchard, fórmula de}
  \begin{align}
    M_n
      &= \sum_{0 \le k \le n}
	   (-1)^k \binom{n}{k}
	      \cdot 2
	      \cdot \frac{2 n}{2 n - k} \binom{2 n - k}{k}
	      \cdot k! \cdot \left( (n - k)! \right)^2 \notag \\
      &= 2 n! \sum_{0 \le k \le n}
		(-1)^k \cdot \frac{2 n}{2 n - k}
		       \cdot \binom{2 n - k}{k}
		       \cdot (n - k)!
	  \label{eq:menages}
  \end{align}

  El principio de inclusión y exclusión
  completa las herramientas elementales
  de la combinatoria vistas
  en el capítulo~\ref{cha:combinatoria-elemental}.

%%% Local Variables:
%%% mode: latex
%%% TeX-master: "clases"
%%% End:

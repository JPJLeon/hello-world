\documentclass[spanish, fleqn]{article}
\usepackage{babel}
\usepackage[utf8]{inputenc}
\usepackage{amsmath, amsfonts}
\usepackage[colorlinks, urlcolor=blue]{hyperref}
\usepackage{fourier}
\usepackage[top = 2.5cm, bottom = 2cm, left = 2cm, right = 2cm]{geometry}

\newcommand{\num}{5}

\title{Estructuras Discretas \\
       Tarea \#\num \\
       ``No son funciones, y no generan nada''}
\author{Discrete Structures Warriors}
\date{25 de mayo de 2015}

\begin{document}
\maketitle
\thispagestyle{empty}

\section*{Preguntas}

  \begin{enumerate}
  \item
    Muchos juegos de rol
    (como el afamado \href{http://en.wikipedia.org/wiki/Call_of_Cthulhu_(role-playing_game)}{\emph{Call of Cthulhu}})
    usan dados con más caras.
    Repita el desarrollo de dados de Sicherman
    (página 214 del apunte)
    considerando dos dados tetraedrales
    (de \(4\) caras
     marcadas de \(1\) a \(4\)).
     Describa los juegos de dados alternativos posibles.
    \\ \hspace*{\fill}(30 puntos)
  \item
    ¿Cuántos multisubconjuntos de \(n\) elementos
    tiene el multiconjunto
      \(\{\mathtt{M}, \mathtt{I}^4,
	  \mathtt{S}^4, \mathtt{P}^2\}\)?
    \\ \hspace*{\fill}(25 puntos)
  \item
    ¿Cuántas soluciones en \(\mathbb{N}_0\) tiene
    la ecuación \(x_1 + x_2 + 2 x_3 + 3 x_4 = n\)
    para \(n\) dado,
    si \(x_i \in \mathbb{N}_0\) con \(x_1 \le 2\)?
    \\ \hspace*{\fill}(25 puntos)
  \item
    Use las propiedades de las funciones generatrices
    para derivar una expresión para:
    \begin{equation*}
      \sum_{1 \le k \le n} k^3
    \end{equation*}
    \hspace*{\fill}(20 puntos)
  \end{enumerate}

% Condiciones generales de tarea 0 de INF-155/ILI-255 2015/2
\section*{Condiciones Generales}

  \begin{itemize}
  \item
    La tarea se realizará \emph{individualmente}
    (esto es grupos de una persona),
    sin excepciones.
  \item
    En caso de que se descubra copia,
    equivale a nota 0 para los estudiantes implicados.
  \item
    Cada respuesta debe estar correctamente justificada, 
    en caso contrario el puntaje obtenido queda 
    sujeto al criterio de los ayudantes.
  \item
    Deberá subir los fuentes {\LaTeX} de su solución
    en el área designada al efecto
    en \href{http://moodle.inf.utfsm.cl}{Moodle}
    bajo el formato
    \texttt{tarea\num-\emph{rol}.tar.gz}.
    El archivo debe contener el directorio \texttt{tarea\num-\emph{rol}},
    en el cual están los archivos pedidos.
  \item
    Por cada día de atraso se descontarán 20 puntos.
    A partir del tercer día de atraso
    no se reciben más tareas y la nota es automáticamente cero.
  \item
    La nota de la tarea puede ser según lo entregado,
    o (en el caso de algunos estudiantes elegidos al azar)
    el resultado de una interrogación en que deberá explicar lo entregado.
    No presentarse a la interrogación significa automáticamente
    nota cero.

    Sobre la nota de la interrogación se aplican los descuentos por atraso.
  \end{itemize}

  \vfill\hfill DSW/HvB/\LaTeXe
\end{document}

%%% Local Variables:
%%% mode: latex
%%% TeX-master: t
%%% End:

\documentclass[spanish, fleqn]{article}
\usepackage{babel}
\usepackage[utf8]{inputenc}
\usepackage{amsmath, amsfonts}
\usepackage[colorlinks, urlcolor=blue]{hyperref}
\usepackage{fourier}
\usepackage[top = 2.5cm, bottom = 2cm, left = 2cm, right = 2cm]{geometry}

\newcommand{\num}{4}

\title{Estructuras Discretas \\
       Tarea \#\num \\
       ``Revisión combinatoria''}
\author{Funda Dream Team}
\date{3 de mayo de 2015}

\begin{document}
\maketitle
\thispagestyle{empty}

\section*{Preguntas}

  Se piden desarrollos claros y completos de cada una de las siguientes.

  \begin{enumerate}
  \item
    En un campeonato de atletismo participan \(35\)~estudiantes,
    y se reparten \(8\) premios
    (primero a octavo lugares).
    \begin{enumerate}
    \item
      ¿Cuántos resultados son posibles?
    \item
      ¿De cuántas maneras puede resultar en uno de los tres primeros lugares
      cierta participante,
      Isidora?
    \end{enumerate}
    \hspace*{\fill}(25 puntos)
  \item
    Para el certamen se ubican los \(16\) integrantes del curso
    en una sala de \(4 \times 4\) asientos.
    \begin{enumerate}
    \item
      ¿De cuántas maneras se pueden sentar?
    \item
      El profesor prefiere ubicarlos en \(4\) salas separadas,
      en grupos de \(4\).
      ¿Cuántas distribuciones son posibles así?
    \end{enumerate}
    \hspace*{\fill}(25 puntos)
  \item
    Para la presentación del afamado grupo Azathoth de danza bizarra
    se deben formar \(8\) parejas de hombres y mujeres
    con los \(16\) integrantes.
    ¿De cuántas formas se pueden formar las parejas?
    \\ \hspace*{\fill}(15 puntos)
  \item
    En Dunking~Donuts hay \(20\) tipos diferentes de doughnuts,
    de cada tipo hay al menos una docena disponible.
    ¿De cuántas maneras se pueden elegir una docena de doughnuts
    entre las ofrecidas?
    \\ \hspace*{\fill}(15 puntos)
  \item
    Veinte parejas casadas van al cine,
    y se sientan en la primera fila,
    de forma que las parejas se sientan juntas.
    ¿De cuántas formas pueden ubicarse?
    \\ \hspace*{\fill}(20 puntos)
  \end{enumerate}

% Condiciones generales de tarea 0 de INF-155/ILI-255 2015/2
\section*{Condiciones Generales}

  \begin{itemize}
  \item
    La tarea se realizará \emph{individualmente}
    (esto es grupos de una persona),
    sin excepciones.
  \item
    En caso de que se descubra copia,
    equivale a nota 0 para los estudiantes implicados.
  \item
    Cada respuesta debe estar correctamente justificada, 
    en caso contrario el puntaje obtenido queda 
    sujeto al criterio de los ayudantes.
  \item
    Deberá subir los fuentes {\LaTeX} de su solución
    en el área designada al efecto
    en \href{http://moodle.inf.utfsm.cl}{Moodle}
    bajo el formato
    \texttt{tarea\num-\emph{rol}.tar.gz}.
    El archivo debe contener el directorio \texttt{tarea\num-\emph{rol}},
    en el cual están los archivos pedidos.
  \item
    Por cada día de atraso se descontarán 20 puntos.
    A partir del tercer día de atraso
    no se reciben más tareas y la nota es automáticamente cero.
  \item
    La nota de la tarea puede ser según lo entregado,
    o (en el caso de algunos estudiantes elegidos al azar)
    el resultado de una interrogación en que deberá explicar lo entregado.
    No presentarse a la interrogación significa automáticamente
    nota cero.

    Sobre la nota de la interrogación se aplican los descuentos por atraso.
  \end{itemize}

  \vfill\hfill FDT/HvB/\LaTeXe
\end{document}

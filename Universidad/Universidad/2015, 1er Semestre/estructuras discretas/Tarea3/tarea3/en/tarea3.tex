\documentclass[spanish, fleqn]{article}
\usepackage{babel}
\usepackage[utf8]{inputenc}
\usepackage{amsmath, amsfonts}
\usepackage[colorlinks, urlcolor=blue]{hyperref}
\usepackage{fourier}
\usepackage[top = 2.5cm, bottom = 2cm, left = 2cm, right = 2cm]{geometry}

\newcommand{\num}{3}

\title{Estructuras Discretas \\
       Tarea \#\num \\
       ``Demostrando y contando''}
\author{Funda Dream Team}
\date{19 de abril de 2015}

\begin{document}
\maketitle
\thispagestyle{empty}

\section*{Preguntas}

  Se piden demostraciones claras y completas de cada una de las siguientes.

  \begin{enumerate}
  \item
    Demuestre la \emph{propiedad telescópica de la suma}:
    \begin{equation*}
      \sum_{0 \le k \le n} (f(k + 1) - f(k))
	= f(n + 1) - f(0)
    \end{equation*}
    \\ \hspace*{\fill}(15 puntos)
  \item
    Demuestre que:
    \begin{equation*}
      \sum_{1 \le k \le n} \frac{1}{k (k + 1)}
	= \frac{n}{n + 1}
    \end{equation*}
    Puede hacerse de dos formas:
    \begin{enumerate}
    \item
      Directamente por inducción
    \item
      Usando:
      \begin{equation*}
	\frac{1}{k (k + 1)}
	  = \frac{1}{k} - \frac{1}{k + 1}
      \end{equation*}
      y la propiedad telescópica
    \end{enumerate}
    \hspace*{\fill}(30 puntos)
  \item
    Los \emph{números de Fibonacci} se definen mediante:
    \begin{align*}
      F_0
	&= 0 \\
      F_1
	&= 1 \\
      F_{n + 2}
	&= F_{n + 1} + F_n
	   \qquad\text{si \(n \ge 2\)}
    \end{align*}
    Demuestre que cumplen:
    \begin{equation*}
      \left( 3/2 \right)^{n - 2}
	 \le F_n
	 \le 2^n
    \end{equation*}
    \hspace*{\fill}(20 puntos)
  \item
    Describa paso a paso cómo resuelve las siguientes:
    \begin{enumerate}
    \item
      ¿En cuántas de las permutaciones de \(n\)~números
      aparece el~\(1\) antes que el~\(2\)?
    \item
      ¿En cuántas de las permutaciones de \(n\)~números
      no aparecen contiguos \(1\) y \(2\)?
    \end{enumerate}
    \hspace*{\fill}(10 puntos)
  \item
    Considere un conjunto \(\mathcal{S}\) de \(n\) elementos.
    ¿Cuántos pares de conjuntos \(\mathcal{A}\) y \(\mathcal{B}\)
    hay,
    tales que \(\mathcal{A} \cup \mathcal{B} = \mathcal{S}\)?
    \\ \hspace*{\fill}(25 puntos)
  \end{enumerate}

% Condiciones generales de tarea 0 de INF-155/ILI-255 2015/2
\section*{Condiciones Generales}

  \begin{itemize}
  \item
    La tarea se realizará \emph{individualmente}
    (esto es grupos de una persona),
    sin excepciones.
  \item
    En caso de que se descubra copia,
    equivale a nota 0 para los estudiantes implicados.
  \item
    Cada respuesta debe estar correctamente justificada, 
    en caso contrario el puntaje obtenido queda 
    sujeto al criterio de los ayudantes.
  \item
    Deberá subir los fuentes {\LaTeX} de su solución
    en el área designada al efecto
    en \href{http://moodle.inf.utfsm.cl}{Moodle}
    bajo el formato
    \texttt{tarea\num-\emph{rol}.tar.gz}.
    El archivo debe contener el directorio \texttt{tarea\num-\emph{rol}},
    en el cual están los archivos pedidos.
  \item
    Por cada día de atraso se descontarán 20 puntos.
    A partir del tercer día de atraso
    no se reciben más tareas y la nota es automáticamente cero.
  \item
    La nota de la tarea puede ser según lo entregado,
    o (en el caso de algunos estudiantes elegidos al azar)
    el resultado de una interrogación en que deberá explicar lo entregado.
    No presentarse a la interrogación significa automáticamente
    nota cero.

    Sobre la nota de la interrogación se aplican los descuentos por atraso.
  \end{itemize}

  \vfill\hfill FDT/HvB/\LaTeXe
\end{document}

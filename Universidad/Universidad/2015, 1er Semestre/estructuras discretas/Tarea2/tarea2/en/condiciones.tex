% Condiciones generales de tareas de Estructuras Discretas, 2015
\section*{Condiciones Generales}

  \begin{itemize}
  \item
    La tarea se realizará \emph{individualmente}
    (esto es grupos de una persona),
    sin excepciones.
  \item
    La tarea debe ser entregada impresa
\if\num0
\else
     o manuscrita
\fi
    en
    la Secretaría Docente de Informática
    (Piso 1, edificio F3)
    el día indicado en \href{http://moodle.inf.utfsm.cl}{Moodle}.
  \item
\if\num0
    Junto
\else
    Opcionalmente,
    puede desarrollar la tarea en \LaTeX{},
    lo cual tiene una bonificación de 10 puntos.
    Para obtener la bonificación,
    junto
\fi
    con entregar la tarea impresa en hojas tamaño carta
    deberá depositar copia
    de los fuentes \LaTeX{} de su solución en un \emph{tarball}
    en el área designada al efecto
    en \href{http://moodle.inf.utfsm.cl}{Moodle}
    bajo el formato
    \texttt{tarea\num-\emph{rol}.tar.gz}.
    El archivo debe contener el directorio \texttt{tarea\num-\emph{rol}},
    en el cual están los archivos de su solución
    (al menos \texttt{tarea\num.tex}).
\if\num0
    Sólo se considerará entregada correctamente la tarea si
\else
    Tiene derecho a la bonificación sólo si
\fi
    el \emph{tarball} tiene el nombre y contenido correctos,
    y los fuentes \LaTeX{}
    (y posibles otros archivos anexos)
    se procesan correctamente
    en el ambiente que ofrece el Laboratorio de Computación
    del Departamento de Informática,
    y están escritos en forma legible.
\if\num0
\else

    Si la entrega es en manuscrito,
    está afecta a descuento de hasta 20 puntos por desorden o ilegibilidad.
\fi
\ifdefined\ontime
  \item
    El plazo de entrega dado para esta tarea es definitivo,
    no se aceptan entregas atrasadas.
\else
  \item
    Por cada día de atraso se descontarán 20 puntos.
    A partir del tercer día de atraso
    no se reciben más tareas y la nota es automáticamente cero.
\fi
  \item
    La nota de la tarea puede ser según lo entregado,
    o (en el caso de algunos estudiantes elegidos al azar)
    el resultado de una interrogación en que deberá explicar lo entregado.
    No presentarse a la interrogación significa automáticamente
    nota cero.

    Sobre la nota de la interrogación se aplican
\ifdefined\ontime
    la bonificación por entrega en \LaTeX{}
    o el descuento por desorden.
\else
    los descuentos por atraso
    si proceden%
\if\num0%
    .%
\else%
    ,
    y la bonificación por entrega en \LaTeX{}
    o los descuentos por desorden.
\fi
\fi
  \end{itemize}

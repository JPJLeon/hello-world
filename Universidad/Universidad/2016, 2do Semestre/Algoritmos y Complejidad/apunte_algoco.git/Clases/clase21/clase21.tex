\documentclass[english, spanish, fleqn, 10pt]{article}
\usepackage[top = 2.5cm, bottom = 2cm, left = 2.5cm, right = 2.5cm]{geometry}
\usepackage[es-noindentfirst]{babel}
\usepackage[utf8]{inputenc}
\usepackage{amsthm, amsmath, amssymb, amsfonts, environ}
\usepackage{caption}
\usepackage{subcaption} % para las subfigures
\usepackage{mathrsfs}
\usepackage[colorlinks, urlcolor=blue,
pdfauthor={Aldo Berrios Valenzuela}]{hyperref}
\usepackage{fourier}
\usepackage{colortbl}
\usepackage{color}
\usepackage{graphicx}
\usepackage{enumitem}
%\renewcommand{\familydefault}{\sfdefault}
%\setlength{\parindent}{0pt}					%Espacio de sangria
\setlength{\parskip}{0.5mm}						%Interlinado entre párrafos
\usepackage{fancyhdr, blindtext}		%Paquete de encabezado
\usepackage{listings}
\usepackage[framemethod=tikz]{mdframed}

\definecolor{webred}{rgb}{0.75,0,0}
\definecolor{mblue}{rgb}{0.0705,0.345,0.7529}
\definecolor{newmblue}{HTML}{004D99}
\usepackage{longtable, colortbl, array}

\definecolor{superGreenNew}{HTML}{169F01}
\hypersetup{
	colorlinks   = true,
	citecolor    = superGreenNew
}


%==============================
%Modificador de Titulos de secciones, subsecciones, etc
\usepackage[explicit, noindentafter]{titlesec}

\titlespacing\subsubsection{0pt}{8pt plus 4pt minus 2pt}{4pt plus 2pt minus 2pt}


%==================================
%Diseno para insertar codigos de programacion
\definecolor{newGray}{HTML}{F8F8F8}
\definecolor{anotherGray}{HTML}{DDDFFF}
\lstdefinestyle{customc}{
	belowcaptionskip=1\baselineskip,
	breaklines=true,
	backgroundcolor=\color{newGray},
	frame=single,
	rulecolor=\color{anotherGray},
	xleftmargin=\parindent,
	language=bash,
	showstringspaces=false,
	basicstyle=\small\ttfamily,
	keywordstyle=\bfseries\color{red!40!black},
	commentstyle=\itshape\color{purple!40!black},
	identifierstyle=\color{black},
	stringstyle=\color{mblue},
	tabsize=2,
	literate={\ \ }{{\ }}1	
}
\lstset{style=customc}

\renewcommand{\lstlistingname}{Listado}

%==================================
%Macros útiles
\newcommand{\comillas}[1]{``#1''}
\newcommand{\comentarioc}[1]{\texttt{\textcolor{webred}{/* #1 */}}}

\numberwithin{equation}{section}

%=================================
%comandos matemáticos personalizados de rápido acceso
\newcommand{\nparentesis}[1]{\left( #1 \right)}
\newcommand{\llaves}[1]{\left \{ #1 \right \}}
\newcommand{\nabsoluto}[1]{\left| #1 \right|}
\newcommand{\ncorchetes}[1]{\left[ #1 \right]}
%=================================

%cambia el espaciado entre el parrafo y antes del itemize
\setlist[itemize]{itemsep=1mm, topsep=1.5mm}
\setlist[enumerate]{itemsep=1mm, topsep=1.5mm}
%===========================

\theoremstyle{definition}
\newtheorem{teorema}{Teorema}[section]
\newtheorem{definition}{Definición}[section]
\newtheorem{beforeExample}{Ejemplo}[section]
\newtheorem{preposicion}{Preposición}[section]
\newtheorem{corolario}{Corolario}[teorema]

\captionsetup{justification=centering, margin=2.3cm}

\newenvironment{ejemplo}[1][]{\begin{beforeExample}[#1]\renewcommand{\qedsymbol}{$\blacksquare$}}{\qed\end{beforeExample}}

\captionsetup{justification=centering, margin=2.3cm}

% Para poder hacer "quote" con estilo github

\newenvironment{newquote}{\begin{mdframed}[topline=false,
		rightline=false, bottomline=false, linewidth=.3em,backgroundcolor=black!5, linecolor=black!60]\color{black!70}}{\end{mdframed}}
%================================

\renewcommand*{\arraystretch}{1.2}

\usepackage{fancyhdr}

\pagestyle{fancy}
\lhead{INF221\quad Algoritmos y Complejidad}
\rhead{\textit{Clase \#21 \quad Métodos de Ordenamiento}}


\author{Aldo Berrios Valenzuela}

\title{INF221 -- Algoritmos y Complejidad\\[.4\baselineskip]
	Clase \#21\\
	Métodos de Ordenamiento}

\date{Martes 25 de octubre de 2016}

\newcommand\numberthis{\addtocounter{equation}{1}\tag{\theequation}}

\begin{document}
\maketitle

\section{Métodos de Ordenamiento}
Los métodos de ordenamiento de la burbuja, inserción realizan una inversión por cada permutación. A través del método simbólico, podemos representar la clase permutación $\mathcal{P}$ como sigue:
\begin{equation*}
\mathcal{P} = \mathcal{E} + \mathcal{Z} \star \mathcal{P}
\end{equation*}
donde la clase pasada, vimos que la función generatriz resultante para una clase de permutación no es más que la ecuación \eqref{21::Generatriz:permutacion}
\begin{equation}\label{21::Generatriz:permutacion}
\hat P\nparentesis{z} = \dfrac{1}{1-z}
\end{equation}


Nos interesan las inversiones de permutaciones. Sea $\iota\nparentesis{\pi}$ el número de inversiones de la permutación $\pi$. Como el orden de los elementos en la permutación $\pi$ es de importancia (ya que define la cantidad de inversiones que debemos hacer), definimos la función generatriz exponencial \eqref{21::EFG:inversiones} para representar dichas inversiones de $\pi$.
\begin{equation}\label{21::EFG:inversiones}
\hat I \nparentesis{z} = \sum_{\pi \in \mathcal{P}} \iota\nparentesis{\pi} \dfrac{z^{\nabsoluto{\pi}}}{\nabsoluto{\pi}!}
\end{equation}
Nos interesa el número promedio de inversiones en permutaciones de tamaño $n$:
\begin{align*}
E_n\nparentesis{\iota} &= \dfrac{\ncorchetes{z^n}\hat I \nparentesis{z}}{\ncorchetes{z^n}\hat P \nparentesis{z}}\numberthis\label{21::Cociente:Promedio}\\
&= \ncorchetes{z^n}\hat I \nparentesis{z}
\end{align*}
A simple vista, resulta algo difícil de ver la razón de por qué el número promedio de inversiones se define como muestra \eqref{21::Cociente:Promedio}. Lo explicamos por partes:
\begin{itemize}
	\item Cada $\pi_i \in \mathcal{P}$ representa una permutación sobre un arreglo de tamaño $n$ (por simplicidad, suponemos que no hay elementos repetidos).
	
	\item Cada una de las permutaciones $\pi_i \in \mathcal{P}$ tiene $\iota \nparentesis{\pi_i}$ inversiones.
	
	\item En un arreglo de tamaño $n$, la cantidad de permutaciones $\pi_i \in \mathcal P$ es $n!$. 
	
	\item Por lo tanto, para un arreglo de tamaño $n$, la cantidad de inversiones sería (consideramos todas las permutaciones):
	\begin{equation}
	\sum _{1 \leq i \leq n!} \iota \nparentesis{\pi _k}
	\end{equation}
	
	\item Usando la definición de promedio, dividimos por la cantidad total de permutaciones presentes en un arreglo de tamaño $n$. Usando funciones generatrices, el promedio de inversiones estaría dado por \eqref{21::Cociente:Promedio}.
\end{itemize}

La receta para construir permutaciones hace $\pi \star \nparentesis{1}$. Este \emph{conjunto} de permutaciones tiene las inversiones de $\pi$, $\nabsoluto{\pi} + 1$ veces. Si al último elemento (el $\nparentesis{1}$) se le asigna el rótulo $j$, aporta $\nabsoluto{\pi} + 1 -j$ inversiones. Pero $1 \leq j \leq \nabsoluto{\pi} + 1$:
\begin{align*}
\sum_{1 \leq j \leq \nabsoluto{\pi}+1}\nparentesis{\nabsoluto{\pi} + 1 - j}&=\sum_{1 \leq j \leq \nabsoluto{\pi} + 1} \nparentesis{\nabsoluto{\pi} + 1} - \sum_{1 \leq j \leq \nabsoluto{\pi} + 1}j\\
&= \nparentesis{\nabsoluto{\pi} + 1}^2 - \dfrac{\nparentesis{\pi + 1} \nabsoluto{\pi}}{2}\\
&= \nparentesis{\nabsoluto{\pi} + 1} \nparentesis{\nabsoluto{\pi} + 1-\dfrac{\nabsoluto{\pi}}{2}}\\
&=\dfrac{\nparentesis{\nabsoluto{\pi} + 1} \nparentesis{\nabsoluto{\pi} + 2}}{2}
\end{align*}
Por lo tanto, el número total de inversiones en $\pi \star \nparentesis{1}$ es:
\begin{equation*}
\nparentesis{\nabsoluto{\pi} + 1} \iota\nparentesis{\pi} + \dfrac{\nparentesis{\nabsoluto{\pi} + 1}}{2}
\end{equation*}
Entonces:
\begin{align*}
\hat I \nparentesis{z} &= 0 + \sum_{\pi \in \mathcal{P}}\nparentesis{\nparentesis{\nabsoluto{\pi} + 1} \iota\nparentesis{\pi} + \dfrac{\nparentesis{\nabsoluto{\pi} + 1}}{2}}\dfrac{z^{\nabsoluto{\pi} + 1}}{\nparentesis{\nabsoluto{\pi} + 1}!}\\
&= z \sum_{\pi \in \mathcal{P}}\iota \nparentesis{\pi}\dfrac{z^{\nabsoluto{\pi}}}{\nabsoluto{\pi}!} + \dfrac{1}{2}z\sum_{\pi \in \mathcal{P}}\nabsoluto{\pi} \dfrac{z^{\nabsoluto{\pi}}}{\nabsoluto{\pi}!}\\
&= z \hat I \nparentesis{z} + \dfrac{1}{2}z\sum_{k \geq 0}kz^k
\end{align*}
\begin{align*}
\hat I \nparentesis{z}&=z \hat I \nparentesis{z} + \dfrac{1}{2}z^2 \dfrac{d}{dz} \dfrac{1}{1-z}\\
&= z \hat I \nparentesis{z} + \dfrac{1}{2}\dfrac{z^2}{\nparentesis{1-z}^2}\\
\hat I \nparentesis{z} &= \dfrac{1}{2} \dfrac{z^2}{\nparentesis{1-z}^3}\\
\ncorchetes{z^n} \hat I \nparentesis{z} &= \dfrac{1}{2}\ncorchetes{z^{n-2}}\nparentesis{1-z} ^{-3}\\
&= \dfrac{1}{2}\binom{-3}{n-2}\\
&= \dfrac{1}{2}\binom{n-2 + \nparentesis{3-1}}{3-1}\\
&= \dfrac{1}{2}\binom{n}{2}\\
&=\binom{n\nparentesis{n-1}}{4}
\end{align*}
Por lo tanto, asintóticamente se tiene que:
\begin{equation}
E_n\nparentesis{\iota} \sim \dfrac{n^2}{4}
\end{equation}
Los nuevos elementos aportan entre $0$ y $\nabsoluto{\pi}$ nuevas inversiones:
\begin{equation*}
\sum_{0 \leq k \leq \nabsoluto{\pi}} \dfrac{\nabsoluto{\pi}\nparentesis{\nabsoluto{\pi}+1}}{4}
\end{equation*}


Siguiente método más simple: shell sort.

\subsection{Selection Sort}
Recordemos el algoritmo:
\begin{lstlisting}
void sort(double a[], int n){
	for (int i = 1; i < n; i++) {
		int imax = i;
		double max = a[i];
		for (int j = i + 1; j < n; j++) {
			if (a[j] > max) {
				min = a[j]; imin = j;
			}
		}
		double tmp = a[imin]; a[imin] = a[i]; a[i] = tmp;
	}
}
\end{lstlisting}
Nos interesa el número de asignaciones a \texttt{max}.
\begin{itemize}
	\item [$\rightsquigarrow$] Máximo de izquierda a derecha en la permutación $\pi, \chi \nparentesis{\pi}$ ($\chi\nparentesis{\pi}$: máximos de izquierda a derecha en la permutación $\pi$).
\end{itemize}
Si en $\pi \star \nparentesis{1}$ el último elemento $\nabsoluto{\pi} + 1$, hay un máximo más; en caso contrario, nada cambia. El número total de máximo es:
\begin{equation*}
\nparentesis{\nabsoluto{\pi} + 1} \chi \nparentesis{\pi} + 1
\end{equation*}
O sea, para:
\begin{equation*}
\hat M \nparentesis{z} = \sum_{\pi \in \mathcal{P}}\chi \nparentesis{\pi} \dfrac{z^{\nabsoluto{\pi}}}{\nabsoluto{\pi}!}
\end{equation*}
De la ecuación simbólica para permutaciones:
\begin{align*}
\hat M \nparentesis{z} &= 0 + \sum_{\pi \in \mathcal{P}}\nparentesis{\nparentesis{\nabsoluto{\pi} + 1} \chi \nparentesis{\pi} + 1}\\
&= z\sum_{\pi \in \mathcal{P}}\chi \nparentesis{\pi}\dfrac{z^{\nabsoluto{\pi}}}{\nabsoluto{\pi}!} + z\sum_{\pi \in \mathcal{P}}\dfrac{z^{\nabsoluto{\pi}}}{\nparentesis{\nabsoluto{\pi} + 1}!}\\
&=z\hat M \nparentesis{z} + \sum_{k \geq 0} \dfrac{z^{k+1}}{k+1}\\
&=\hat M\nparentesis{z}= \dfrac{1}{1-z}\ln\dfrac{1}{1-z}\\
\ncorchetes{z^n}\hat M \nparentesis{z} &= H_n = \sum_{1 \geq k \geq n}\dfrac{1}{k} = \ln n + \gamma + O \nparentesis{\dfrac{1}{n}}
\end{align*}
Donde $\gamma \approx 0.5772156\ldots $ y además, se tiene que $H_n$ son los números armónicos. La demostración:
\begin{align*}
\dfrac{1}{1-z}\sum_{k\geq 1} \dfrac{z^k}{k} &= \sum_{n \geq 0}\nparentesis{\sum_{1\geq k \geq n}\dfrac{1}{n}}z^n\\
&= \sum_{n \geq 0} H_n z^n
\end{align*}


\end{document}

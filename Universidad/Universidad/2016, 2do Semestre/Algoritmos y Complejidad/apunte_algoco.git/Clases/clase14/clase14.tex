\documentclass[english, spanish, fleqn, 10pt]{article}
\usepackage[top = 2.5cm, bottom = 2cm, left = 2.5cm, right = 2.5cm]{geometry}
\usepackage[es-noindentfirst]{babel}
\usepackage[utf8]{inputenc}
\usepackage{amsthm, amsmath, amssymb, amsfonts, environ}
\usepackage{caption}
\usepackage{subcaption} % para las subfigures
\usepackage{mathrsfs}
\usepackage[colorlinks, urlcolor=blue,
pdfauthor={Aldo Berrios Valenzuela}]{hyperref}
\usepackage{fourier}
\usepackage{colortbl}
\usepackage{color}
\usepackage{graphicx}
\usepackage{enumitem}
\usepackage{pgf}
\usepackage{chessboard}
%\renewcommand{\familydefault}{\sfdefault}
%\setlength{\parindent}{0pt}					%Espacio de sangria
\setlength{\parskip}{0.5mm}						%Interlinado entre párrafos
\usepackage{fancyhdr, blindtext}		%Paquete de encabezado
\usepackage{listings}
\usepackage[framemethod=tikz]{mdframed}

\definecolor{webred}{rgb}{0.75,0,0}
\definecolor{mblue}{rgb}{0.0705,0.345,0.7529}
\definecolor{newmblue}{HTML}{004D99}
\usepackage{longtable, colortbl, array}

\definecolor{superGreenNew}{HTML}{169F01}
\hypersetup{
	colorlinks   = true,
	citecolor    = superGreenNew
}


%==============================
%Modificador de Titulos de secciones, subsecciones, etc
\usepackage[explicit, noindentafter]{titlesec}

\titlespacing\subsubsection{0pt}{8pt plus 4pt minus 2pt}{4pt plus 2pt minus 2pt}


%==================================
%Diseno para insertar codigos de programacion
\definecolor{newGray}{HTML}{F8F8F8}
\definecolor{anotherGray}{HTML}{DDDFFF}
\lstdefinestyle{customc}{
	belowcaptionskip=1\baselineskip,
	breaklines=true,
	backgroundcolor=\color{newGray},
	frame=single,
	rulecolor=\color{anotherGray},
	xleftmargin=\parindent,
	language=bash,
	showstringspaces=false,
	basicstyle=\small\ttfamily,
	keywordstyle=\bfseries\color{green!40!black},
	commentstyle=\itshape\color{purple!40!black},
	identifierstyle=\color{black},
	stringstyle=\color{mblue},
	tabsize=2,
	literate={\ \ }{{\ }}1
}
\lstset{style=customc}

\lstloadlanguages{Python, C}

%==================================
%Macros útiles
\newcommand{\comillas}[1]{``#1''}
\newcommand{\ccomment}[1]{\texttt{\textcolor{webred}{/* #1 */}}}

\numberwithin{equation}{section}

%=================================
%comandos matemáticos personalizados de rápido acceso
\newcommand{\nparentesis}[1]{\left( #1 \right)}
\newcommand{\llaves}[1]{\left \{ #1 \right \}}
\newcommand{\nabsoluto}[1]{\left| #1 \right|}
\newcommand{\ncorchetes}[1]{\left[ #1 \right]}
%=================================

%cambia el espaciado entre el parrafo y antes del itemize
\setlist[itemize]{itemsep=1mm, topsep=1.5mm}
\setlist[enumerate]{itemsep=1mm, topsep=1.5mm}
%===========================

\theoremstyle{definition}
\newtheorem{teorema}{Teorema}[section]
\newtheorem{definition}{Definición}[section]
\newtheorem{beforeExample}{Ejemplo}[section]
\newtheorem{preposicion}{Preposición}[section]
\newtheorem{corolario}{Corolario}[teorema]

\captionsetup{justification=centering, margin=2.3cm}

\newenvironment{ejemplo}[1][]{\begin{beforeExample}[#1]\renewcommand{\qedsymbol}{$\blacksquare$}}{\qed\end{beforeExample}}

\captionsetup{justification=centering, margin=2.3cm}

% Para poder hacer "quote" con estilo github

\newenvironment{newquote}{\begin{mdframed}[topline=false,
		rightline=false, bottomline=false, linewidth=.3em,backgroundcolor=black!5, linecolor=black!60]\color{black!70}}{\end{mdframed}}
%================================

\renewcommand*{\arraystretch}{1.2}

\usepackage{fancyhdr}

\pagestyle{fancy}
\lhead{INF221\quad Algoritmos y Complejidad}
\rhead{\textit{Clase \#14\quad Backtracking}}

\title{INF221 -- Algoritmos y Complejidad\\[.4\baselineskip]Clase \#14\\Backtracking}
\author{Aldo Berrios Valenzuela \and Horst H. von Brand}
\date{Miércoles 21 de septiembre de 2016}

\begin{document}
\setchessboard{
  showmover = false,
  label = false
}

\maketitle
\section{Backtracking}
Otra estrategia recursiva\ldots La idea es ir construyendo la solución incrementalmente, explorando distintas ramas y volviendo atrás (backtrack) si  resulta que un camino es sin salida.

\begin{ejemplo}[Un clásico]
  En el ajedrez la reina es la pieza más poderosa.
  Amenaza los casilleros en su fila y columna,
  y los ubicados en diagonal.
  La figura~\ref{fig:reina-amenaza} muestra los casilleros
  que amenaza una reina en el ajedrez.
  \begin{figure}[ht]
    \centering
    \setchessboard{setpieces = {Qc5}}
    \chessboard[pgfstyle = {[fill]circle},
		padding = -1ex,
		backfields  = {a5, b5, d5, e5, f5, g5, h5,
			       c1, c2, c3, c4, c6, c7, c8,
			       a3, b4, d6, e7, f8,
			       a7, b6, d4, e3, f2, g1}
	       ]
    \caption{Los casilleros amenazados por una reina}
    \label{fig:reina-amenaza}
  \end{figure}
  Un problema clásico es determinar
  si se pueden ubicar ocho reinas en el tablero
  de manera que ninguna pueda amenzar a otra.
  Claramente no pueden ser más de ocho,
  puede haber a lo más una reina por columna.

	Pasos:
	\begin{itemize}
		\item Reducir espacio de búsqueda.
		\begin{itemize}
			\item [$\rightsquigarrow$] Si son 8 reinas, hay una por columna.
		\end{itemize}
		Por lo tanto, llenar por columnas, solución (parcial) indica las filas de las reinas ya ubicadas.

		\item Ordenar avance.

		\item Subproblemas similares.
	\end{itemize}

	En este caso:
	\begin{itemize}
		\item Ubicar reina en columna 1, 2, \ldots

		\item Registrar filas libres (por omitir ocupadas al ubicar la siguiente reina)

		\item Registrar diagonales libres.

		Reina en $i, j$:
		\begin{align*}
		y-j&=1\cdot \nparentesis{x-i}\\
		i-j&=x-y\rightsquigarrow x-y\mathit{cte}\\
		y-j&=-1\cdot \nparentesis{x-i}\rightsquigarrow x+y\mathit{cte}
		\end{align*}
	\end{itemize}

	Por ejemplo,
	luego de ubicadas las primeras tres reinas
	en las filas \(1\), \(3\) y \(5\) la configuración resultante
	es la de la figura~\ref{fig:tres-reinas}.
	\begin{figure}[ht]
	  \centering
	  \setchessboard{setpieces = {Qa1, Qb3, Qc5}}
	  \chessboard[pgfstyle = {[fill]circle},
		      padding = -1ex,
		      backfields  = {a2, a3, a4, a5, a6, a7, a8,
				       b1, c1, d1, e1, f1, g1, h1,
				       b2, c3, d4, e5, f6, g7, h8,
				     b1, b2, b4, b5, b6, b7, b8,
				       a3, c3, d3, e3, f3, g3, h3,
				       a2, c4, d5, e6, f7, g8,
				       a4, c2, d1,
				     c1, c2, c3, c4, c6, c7, c8,
					a5, b5, d5, e5, f5, g5, h5,
					a7, b6, d4, e3, f2, g1,
					a3, b4, d6, e7, f8
				    }
		     ]
	  \caption{Configuración con tres reinas}
	  \label{fig:tres-reinas}
	\end{figure}
	Vemos que las filas y diagonales amenazadas por estas tres
	restringen muchísimo las posiciones viables
	para la cuarta y siguientes.
	Con estas tres reinas,
	para la cuarta reina quedan solo \(3\) posibilidades.

	Elegimos Python (!), en Python los arreglos tienen índices desde $0$, rango de $i$, $j$ es $0 \ldots 7$.
	Interesan los rangos de:
	\begin{itemize}
		\item $i-j$: $-7\ldots 7\rightsquigarrow$ sumar 7 para llevar al rango $0\ldots 14$.
		\item $i+j$: $0\ldots 14$
	\end{itemize}
	El programa final es el siguiente:
	\lstinputlisting[language = Python]{8queens}

	Una de las \(92\)~soluciones posibles
	se muestra en la figura~\ref{fig:8reinas}.
	\begin{figure}[ht]
	  \centering
	  \setchessboard{setpieces = {Qa1, Qb5, Qc8, Qd6,
				      Qe3, Qf7, Qg2, Qh4}}
	  \chessboard
	  \caption{Una solución para el problema de 8 reinas.}
	  \label{fig:8reinas}
	\end{figure}
\end{ejemplo}
\end{document}

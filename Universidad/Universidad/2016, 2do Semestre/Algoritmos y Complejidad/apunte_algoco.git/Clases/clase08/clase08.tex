\documentclass[english, spanish, fleqn, 10pt]{article}
\usepackage[top = 2.5cm, bottom = 2cm, left = 2.5cm, right = 2.5cm]{geometry}
\usepackage{babel}
\usepackage[utf8]{inputenc}
\usepackage{amsthm, amsmath, amssymb, amsfonts, environ}
\usepackage{caption}
\usepackage{subcaption} % para las subfigures
\usepackage{mathrsfs}
\usepackage[colorlinks, urlcolor=blue]{hyperref}
\usepackage{fourier}
\usepackage{colortbl}
\usepackage{color}
\usepackage{graphicx}
\usepackage{enumitem}
%\renewcommand{\familydefault}{\sfdefault}
%\setlength{\parindent}{0pt}					%Espacio de sangria
\setlength{\parskip}{2mm}						%Interlinado entre párrafos
\usepackage{fancyhdr, blindtext}		%Paquete de encabezado
\usepackage{listings}
\usepackage[framemethod=tikz]{mdframed}

\definecolor{webred}{rgb}{0.75,0,0}
\definecolor{mblue}{rgb}{0.0705,0.345,0.7529}
\definecolor{newmblue}{HTML}{004D99}
\usepackage{longtable, colortbl, array}

\definecolor{superGreenNew}{HTML}{169F01}
\hypersetup{
	colorlinks   = true,
	citecolor    = superGreenNew
}


%==============================
%Modificador de Titulos de secciones, subsecciones, etc
\usepackage[explicit]{titlesec}


\titlespacing\subsubsection{0pt}{8pt plus 4pt minus 2pt}{0pt plus 2pt minus 2pt}


%==================================
%Diseno para insertar codigos de programacion
\definecolor{newGray}{HTML}{F8F8F8}
\definecolor{anotherGray}{HTML}{DDDFFF}
\lstdefinestyle{customc}{
	belowcaptionskip=1\baselineskip,
	breaklines=true,
	backgroundcolor=\color{newGray},
	frame=single,
	rulecolor=\color{anotherGray},
	xleftmargin=\parindent,
	language=bash,
	showstringspaces=false,
	basicstyle=\ttfamily,
	keywordstyle=\bfseries\color{green!40!black},
	commentstyle=\itshape\color{purple!40!black},
	identifierstyle=\color{black},
	stringstyle=\color{mblue},
	tabsize=2,
	literate={\ \ }{{\ }}1	
}
\lstset{style=customc}

%==================================
%Macros útiles
\newcommand{\comillas}[1]{``#1''}
\newcommand{\comentarioc}[1]{\texttt{\textcolor{webred}{/* #1 */}}}
\newcommand{\definicion}[1]{\textit{\underline{\smash{#1}}}}

\numberwithin{equation}{section}

%=================================
%comandos matemáticos personalizados de rápido acceso
\newcommand{\nparentesis}[1]{\left( #1 \right)}
\newcommand{\llaves}[1]{\left \{ #1 \right \}}
\newcommand{\nabsoluto}[1]{\left| #1 \right|}
\newcommand{\ncorchetes}[1]{\left[ #1 \right]}
%=================================

%cambia el espaciado entre el parrafo y antes del itemize
\setlist[itemize]{itemsep=1mm, topsep=0.5mm}
\setlist[enumerate]{itemsep=1mm, topsep=0.5mm}
%===========================

\theoremstyle{definition}
\newtheorem{teorema}{Teorema}[section]
\newtheorem{definition}{Definición}[section]
\newtheorem{beforeExample}{Ejemplo}[section]
\newtheorem{preposicion}{Preposición}[section]
\newtheorem{corolario}{Corolario}[teorema]

\captionsetup{justification=centering, margin=2.3cm}

\newenvironment{ejemplo}[1][]{\begin{beforeExample}[#1]\renewcommand{\qedsymbol}{$\blacksquare$}}{\qed\end{beforeExample}}

\captionsetup{justification=centering, margin=2.3cm}

% Para poder hacer "quote" con estilo github

\newenvironment{newquote}{\begin{mdframed}[topline=false,
		rightline=false, bottomline=false, linewidth=.3em,backgroundcolor=black!5, linecolor=black!60]\color{black!70}}{\end{mdframed}}
%================================


\usepackage{fancyhdr}

\pagestyle{fancy}
\lhead{INF221\quad Algoritmos y Complejidad}
\rhead{\textit{Clase \#8\quad Problemas de Optimización}}
\renewcommand{\headrulewidth}{2pt}

\author{Aldo Berrios Valenzuela}
\title{INF221 -- Algoritmos y Complejidad\\[.4\baselineskip]Clase \#8\\Problemas de Optimización}
\begin{document}
\maketitle
\section{Problema de Optimización}
\subsection{Problema de Tareas}
\begin{teorema}
	Para el problema de programación de tareas, la estrategia de elegir en cada paso la tarea sin conflicto con fin más temprano entrega una solución óptima.
\end{teorema}

\begin{proof}
	Por inducción sobre $\nabsoluto{P}$, el número de tareas.
	\begin{itemize}
		\item \emph{Base:} Si hay una única tarea, la estrategia la programa. Esto es óptimo.
		
		\item \emph{Inducción:} Supongamos que obtiene una solución óptima para a lo más $k$ tareas. Sea $P$ una instancia con $\nabsoluto{P}=k+1$. Elegimos $\hat{p}$ según criterio por elección voraz hay solución óptima que lo incluye, queda $P'$, $\nabsoluto{P'}\leq k$.
		
		Por inducción, obtengo una solución óptima $\Pi'$ de $P'$. Combinando $\Pi'\cup\llaves{\hat p}$ tengo una solución óptima para $P$, por SO (sub-estructura óptima).
	\end{itemize}
\end{proof}

\subsection{Knapsack (mochila)}
Hay una mochila de capacidad $M$, y un conjunto de $n$ tipos de item, del item tipo $i$ hay disponible $p_i$ en total, de valor $v_i$. Se pueden incluir fracciones de item (es café, azúcar, arroz, \ldots)

Estrategia:
\begin{itemize}
	\item Ordenar los ítem por
	\begin{equation*}
	\dfrac{v_i}{p_i}
	\end{equation*}
	decreciente.
	
	\item Echar en la mochila sucesivamente todo lo que se pueda del ítem $i$, en el orden anterior.
\end{itemize}

\textbf{EVQA:} Cumple con EV, EI, SO $\Rightarrow$ dar solución óptima.

\paragraph{Mochila de Discreta:} El item $i$ se agrega completo o no (no fracciones). $\rightsquigarrow$ estrategia voraz \emph{no}  da óptimo.\\[-1em]

\textbf{EQVA: } Contraejemplo.

\subsection{Minimal Spanning Tree}
Dado un grafo $G=\nparentesis{V, E}$, con arcos rotulados $c:E\rightarrow \mathbb{R}^+$, se busca el árbol recubridor (o sea, el que une todos los vértices) de costos mínimo (suma de los $c$ sobre sus arcos).	


\end{document}
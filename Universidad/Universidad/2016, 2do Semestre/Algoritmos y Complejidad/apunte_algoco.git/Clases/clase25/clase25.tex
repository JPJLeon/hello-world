\documentclass[english, spanish, fleqn, 10pt]{article}
\usepackage[top = 2.5cm, bottom = 2cm, left = 2.5cm, right = 2.5cm]{geometry}
\usepackage[es-noindentfirst]{babel}
\usepackage[utf8]{inputenc}
\usepackage{amsthm, amsmath, amssymb, amsfonts, environ}
\usepackage{caption}
\usepackage{subcaption} % para las subfigures
\usepackage{mathrsfs}
\usepackage[colorlinks, urlcolor=blue,
pdfauthor={Aldo Berrios Valenzuela}]{hyperref}
\usepackage{fourier}
\usepackage{colortbl}
\usepackage{color}
\usepackage{graphicx}
\usepackage{enumitem}
%\renewcommand{\familydefault}{\sfdefault}
%\setlength{\parindent}{0pt}					%Espacio de sangria
\setlength{\parskip}{0.5mm}						%Interlinado entre párrafos
\usepackage{fancyhdr, blindtext}		%Paquete de encabezado
\usepackage{listings}
\usepackage[framemethod=tikz]{mdframed}

\definecolor{webred}{rgb}{0.75,0,0}
\definecolor{mblue}{rgb}{0.0705,0.345,0.7529}
\definecolor{newmblue}{HTML}{004D99}
\usepackage{longtable, colortbl, array}

\definecolor{superGreenNew}{HTML}{169F01}
\hypersetup{
	colorlinks   = true,
	citecolor    = superGreenNew
}


%==============================
%Modificador de Titulos de secciones, subsecciones, etc
\usepackage[explicit, noindentafter]{titlesec}

\titlespacing\subsubsection{0pt}{8pt plus 4pt minus 2pt}{4pt plus 2pt minus 2pt}


%==================================
%Diseno para insertar codigos de programacion
\definecolor{newGray}{HTML}{F8F8F8}
\definecolor{anotherGray}{HTML}{DDDFFF}
\lstdefinestyle{customc}{
	belowcaptionskip=1\baselineskip,
	breaklines=true,
	backgroundcolor=\color{newGray},
	frame=single,
	rulecolor=\color{anotherGray},
	xleftmargin=\parindent,
	language=bash,
	showstringspaces=false,
	basicstyle=\small\ttfamily,
	keywordstyle=\bfseries\color{red!40!black},
	commentstyle=\itshape\color{purple!40!black},
	identifierstyle=\color{black},
	stringstyle=\color{mblue},
	tabsize=2,
	literate={\ \ }{{\ }}1	
}
\lstset{style=customc}

\renewcommand{\lstlistingname}{Listado}

%==================================
%Macros útiles
\newcommand{\comillas}[1]{``#1''}
\newcommand{\comentarioc}[1]{\texttt{\textcolor{webred}{/* #1 */}}}

\numberwithin{equation}{section}

%=================================
%comandos matemáticos personalizados de rápido acceso
\newcommand{\nparentesis}[1]{\left( #1 \right)}
\newcommand{\llaves}[1]{\left \{ #1 \right \}}
\newcommand{\nabsoluto}[1]{\left| #1 \right|}
\newcommand{\ncorchetes}[1]{\left[ #1 \right]}
%=================================

%cambia el espaciado entre el parrafo y antes del itemize
\setlist[itemize]{itemsep=1mm, topsep=1.5mm}
\setlist[enumerate]{itemsep=1mm, topsep=1.5mm}
%===========================

\theoremstyle{definition}
\newtheorem{teorema}{Teorema}[section]
\newtheorem{definition}{Definición}[section]
\newtheorem{beforeExample}{Ejemplo}[section]
\newtheorem{preposicion}{Preposición}[section]
\newtheorem{corolario}{Corolario}[teorema]

\captionsetup{justification=centering, margin=2.3cm}

\newenvironment{ejemplo}[1][]{\begin{beforeExample}[#1]\renewcommand{\qedsymbol}{$\blacksquare$}}{\qed\end{beforeExample}}

\captionsetup{justification=centering, margin=2.3cm}

% Para poder hacer "quote" con estilo github

\newenvironment{newquote}{\begin{mdframed}[topline=false,
		rightline=false, bottomline=false, linewidth=.3em,backgroundcolor=black!5, linecolor=black!60]\color{black!70}}{\end{mdframed}}
%================================

\renewcommand*{\arraystretch}{1.2}

\usepackage{fancyhdr}

\pagestyle{fancy}
\lhead{<inserte nombre del documento>}
\rhead{<inserte contenido>}


\author{Aldo Berrios Valenzuela}
\title{INF221 -- Algoritmos y Complejidad\\[.4\baselineskip]
Clase \#25\\
Algoritmo de Kruskal III}

\begin{document}
\maketitle

\section{Algoritmo de Kruskal (continuación de la continuación)}
\begin{center}
	\comentarioc{Algoritmo}
\end{center}
\begin{center}
	\comentarioc{Dibujo de triangulitos}
\end{center}
Secuencia $C$ de compress en $\mathcal{F}$ $\rightsquigarrow$ $C_{+}$ sobre $\mathcal{F}_{+}$, $C_-$ sobre $\mathcal F_-$, operaciones \texttt{shatter}, asignaciones \texttt{parent[}$z$\texttt{]}$ \leftarrow z$. Sean $n_+$, $n_-$ número de nodos $\mathcal F_+$, $\mathcal{F}_-$, $m_+, m_-$ operaciones en $C_+$, $C_-$-

Sea $T\nparentesis{\mathcal F, C}$ el costo (\# asignaciones a punteros) al aplicar $C$ a $\mathcal F$:
\begin{equation}\label{25::PrimeraCota}
T\nparentesis{\mathcal F, C} \leq T\nparentesis{\mathcal F_+, C_+} + T\nparentesis{\mathcal F_-, C_-} + m_+ + n
\end{equation}
Sabemos que hay a lo más
\begin{equation*}
\dfrac{n}{2^i}
\end{equation*}
nodos de rank $i$, por lo que:
\begin{align*}
n_+ &\leq \sum_{i > s}\dfrac{n}{2^i}\\
&=\dfrac{n}{2^s}
\end{align*}
Del teorema que vimos la clase pasada, sabemos:
\begin{align*}
T\nparentesis{\mathcal{F}_+, C_+} & \leq n_+ r\\
& \leq \dfrac{nr}{2^s}
\end{align*}
Si elegimos $s = \log _2 r$, queda $T\nparentesis{\mathcal F _+, C_+} \leq n$. Como \eqref{25::PrimeraCota} vale para \emph{todo} $\mathcal F$, $C$, tenemos:
\begin{align*}
T\nparentesis{m, n, r} &\leq n + T \nparentesis{m_-, n_-, \log_2 r} + n + m_+\\
T\nparentesis{m, n, r} - m &\leq T\nparentesis{m_-, n_-, \log_2r} - m_- + 2n\\
T'\nparentesis{m, n, r} &= T\nparentesis{m, n, r} - m\\
T'\nparentesis{m, n, r} & \leq T'\nparentesis{m_-, n_-, \log_2 r} + 2n\\
&\leq T'\nparentesis{m, n, \log_2 r} + 2n
\end{align*}
Sabemos:
\begin{align*}
T'\nparentesis{m, n, r=0} &= T\nparentesis{m, n, 0} - m \leq 0\\
\rightsquigarrow T'\nparentesis{m, n, r} &\leq 2n \log_2^* r
\end{align*}


\comentarioc{El objetivo de esto fue analizar la secuencia de operaciones}



\end{document}


\documentclass[english, spanish, fleqn, 10pt]{article}
\usepackage[top = 2.5cm, bottom = 2cm, left = 2.5cm, right = 2.5cm]{geometry}
\usepackage[es-noindentfirst]{babel}
\usepackage[utf8]{inputenc}
\usepackage{amsthm, amsmath, amssymb, amsfonts, environ}
\usepackage{caption}
\usepackage{subcaption} % para las subfigures
\usepackage{mathrsfs}
\usepackage[colorlinks, urlcolor=blue,
pdfauthor={Aldo Berrios Valenzuela}]{hyperref}
\usepackage{fourier}
\usepackage{colortbl}
\usepackage{color}
\usepackage{graphicx}
\usepackage{enumitem}
%\renewcommand{\familydefault}{\sfdefault}
%\setlength{\parindent}{0pt}					%Espacio de sangria
\setlength{\parskip}{0.5mm}						%Interlinado entre párrafos
\usepackage{fancyhdr, blindtext}		%Paquete de encabezado
\usepackage{listings}
\usepackage[framemethod=tikz]{mdframed}

\definecolor{webred}{rgb}{0.75,0,0}
\definecolor{mblue}{rgb}{0.0705,0.345,0.7529}
\definecolor{newmblue}{HTML}{004D99}
\usepackage{longtable, colortbl, array}

\definecolor{superGreenNew}{HTML}{169F01}
\hypersetup{
	colorlinks   = true,
	citecolor    = superGreenNew
}


%==============================
%Modificador de Titulos de secciones, subsecciones, etc
\usepackage[explicit, noindentafter]{titlesec}

\titlespacing\subsubsection{0pt}{8pt plus 4pt minus 2pt}{4pt plus 2pt minus 2pt}


%==================================
%Diseno para insertar codigos de programacion
\definecolor{newGray}{HTML}{F8F8F8}
\definecolor{anotherGray}{HTML}{DDDFFF}
\lstdefinestyle{customc}{
	belowcaptionskip=1\baselineskip,
	breaklines=true,
	backgroundcolor=\color{newGray},
	frame=single,
	rulecolor=\color{anotherGray},
	xleftmargin=\parindent,
	language=bash,
	showstringspaces=false,
	basicstyle=\small\ttfamily,
	keywordstyle=\bfseries\color{red!40!black},
	commentstyle=\itshape\color{purple!40!black},
	identifierstyle=\color{black},
	stringstyle=\color{mblue},
	tabsize=2,
	literate={\ \ }{{\ }}1	
}
\lstset{style=customc}

\renewcommand{\lstlistingname}{Listado}

%==================================
%Macros útiles
\newcommand{\comillas}[1]{``#1''}
\newcommand{\comentarioc}[1]{\texttt{\textcolor{webred}{/* #1 */}}}

\numberwithin{equation}{section}

%=================================
%comandos matemáticos personalizados de rápido acceso
\newcommand{\nparentesis}[1]{\left( #1 \right)}
\newcommand{\llaves}[1]{\left \{ #1 \right \}}
\newcommand{\nabsoluto}[1]{\left| #1 \right|}
\newcommand{\ncorchetes}[1]{\left[ #1 \right]}
%=================================

%cambia el espaciado entre el parrafo y antes del itemize
\setlist[itemize]{itemsep=1mm, topsep=1.5mm}
\setlist[enumerate]{itemsep=1mm, topsep=1.5mm}
%===========================

\theoremstyle{definition}
\newtheorem{teorema}{Teorema}[section]
\newtheorem{definition}{Definición}[section]
\newtheorem{beforeExample}{Ejemplo}[section]
\newtheorem{preposicion}{Preposición}[section]
\newtheorem{corolario}{Corolario}[teorema]

\captionsetup{justification=centering, margin=2.3cm}

\newenvironment{ejemplo}[1][]{\begin{beforeExample}[#1]\renewcommand{\qedsymbol}{$\blacksquare$}}{\qed\end{beforeExample}}

\captionsetup{justification=centering, margin=2.3cm}

% Para poder hacer "quote" con estilo github

\newenvironment{newquote}{\begin{mdframed}[topline=false,
		rightline=false, bottomline=false, linewidth=.3em,backgroundcolor=black!5, linecolor=black!60]\color{black!70}}{\end{mdframed}}
%================================

\renewcommand*{\arraystretch}{1.2}

\usepackage{fancyhdr}

\pagestyle{fancy}
\lhead{INF221\quad Algoritmos y Complejidad}
\rhead{\textit{Clase \#29\quad Algoritmos Aleatorizados}}


\author{Aldo Berrios Valenzuela}


\title{INF221 -- Algoritmos y Complejidad\\[.4\baselineskip]
Clase \#29\\
Algoritmos Aleatorizados}

\date{Miércoles 23 de noviembre de 2016}

\begin{document}
\maketitle

\section{Algoritmos Aleatorizados}
\paragraph{Idea:} El altoritmo toma decisiones según valores aleatorios. Por lo tanto, la ejecución varía, incluso con los mismos datos de entrada.

\emph{Clasificación}:
\begin{itemize}
	\item Monte Carlo: Tiempo de ejecución fijo, puede fallar.
	
	\item Las Vegas: Tiempo de ejecución variable; no falla.
\end{itemize}

\emph{Paradigmas:}
\begin{itemize}
	\item Frustrar a un adversario.
	\item Muestreo.
	\item Abundancia de testigos.
	\item Huellas digitales.
	\item Reordenar.
	\item Balance de carga.
	\item Quebrar simetrías.
\end{itemize}

\begin{ejemplo}[Balance de Carga]
	Sitio social, 24000 peticiones por 10 minutos. Cada petición se procesa en promedio $1/4\ncorchetes{s}$, máximo $1\ncorchetes{s}$. Unidad de trabajo: $1\ncorchetes{s}$ entonces son 6000 unidades por 10 minutos, en 10 minutos un servidor puede hacer 600, esto es, 10 servidores al $100\%$. ¿Cuántos más?, Chernoff al rescate\ldots 
	
	Acotamos la probabilidad de que el servidor $1$ (de $m$) se sobrecargue. Si son $m$ servidores, y la carga se distribuye por igual, la carga promedio de $1$ es
	\newcommand{\pr}[1]{\mathrm{Pr}\ncorchetes{#1}}
	\newcommand{\esperanza}[1]{\mathbb{E}\ncorchetes{#1}}
	\begin{equation*}
	\dfrac{10}{m}\cdot 600
	\end{equation*}
	Nos interesa la probabilidad que la carga del servidor $1$ sea mayor que $600$. Definimos $0<t_i\leq 1$ (recuerde que una tarea demora a lo más 1 segundo) el tiempo dedicado por el servidor $1$ a la tarea $i$. Sea:
	\begin{equation*}
	T=\sum_{1\leq i \leq 24000} t_i\qquad\qquad \esperanza{T}=\dfrac{6000}{m} = \dfrac{10}{m}\cdot 600; \qquad c = \dfrac{10}{m}
	\end{equation*}
	Entonces:
	\begin{align*}
	\pr{\text{servidor 1 se sobrecarga}} &= \pr{T \geq 600}\\
	&= \pr{T \geq \esperanza{T}}\\
	&\leq e^{-\beta \nparentesis{c}\cdot \frac{6000}{m}}
	\end{align*}
	Por lo tanto:
	\begin{align*}
	\pr{\text{algún servidor se cae}} &\leq \sum_{1 \leq k \leq m} \pr{\text{servidor $k$ se cae}}\\
	&=m \pr{\text{servidor 1 se cae}}\\
	&\leq m e^{-\beta\nparentesis{c}\cdot \frac{6000}{m}}
	\end{align*}
	Algunos valores de la cantidad de servidores que necesitamos se pueden ver en el Cuadro \ref{29::TablaProbabilidad}.
	\begin{table}[!h]
		\centering
		\begin{tabular}{c|c}
			Valor de $m$ & Probabilidad de que al menos 1 servidor se caiga\\
			\hline
			$11$ & $0.784\ldots$\\
			$12$ & $0.000999\ldots$\\
			$13$ & $0.0000000760\ldots$
		\end{tabular}
		\caption{Tabla de probabilidades de que al menos 1 servidor se caiga con una cantidad de $m$ servidores.}
		\label{29::TablaProbabilidad}
	\end{table}
	
\end{ejemplo}





\end{document}
\documentclass[english, spanish, fleqn, 10pt]{article}
\usepackage[top = 2.5cm, bottom = 2cm, left = 2.5cm, right = 2.5cm]{geometry}
\usepackage[es-noindentfirst]{babel}
\usepackage[utf8]{inputenc}
\usepackage{amsthm, amsmath, amssymb, amsfonts, environ}
\usepackage{caption}
\usepackage{subcaption} % para las subfigures
\usepackage{mathrsfs}
\usepackage[colorlinks, urlcolor=blue,
pdfauthor={Aldo Berrios Valenzuela}]{hyperref}
\usepackage{fourier}
\usepackage{colortbl}
\usepackage{color}
\usepackage{graphicx}
\usepackage{enumitem}
%\renewcommand{\familydefault}{\sfdefault}
%\setlength{\parindent}{0pt}					%Espacio de sangria
\setlength{\parskip}{0.5mm}						%Interlinado entre párrafos
\usepackage{fancyhdr, blindtext}		%Paquete de encabezado
\usepackage{listings}
\usepackage[framemethod=tikz]{mdframed}

\definecolor{webred}{rgb}{0.75,0,0}
\definecolor{mblue}{rgb}{0.0705,0.345,0.7529}
\definecolor{newmblue}{HTML}{004D99}
\usepackage{longtable, colortbl, array}

\definecolor{superGreenNew}{HTML}{169F01}
\hypersetup{
	colorlinks   = true,
	citecolor    = superGreenNew
}


%==============================
%Modificador de Titulos de secciones, subsecciones, etc
\usepackage[explicit, noindentafter]{titlesec}

\titlespacing\subsubsection{0pt}{8pt plus 4pt minus 2pt}{4pt plus 2pt minus 2pt}


%==================================
%Diseno para insertar codigos de programacion
\definecolor{newGray}{HTML}{F8F8F8}
\definecolor{anotherGray}{HTML}{DDDFFF}
\lstdefinestyle{customc}{
	belowcaptionskip=1\baselineskip,
	breaklines=true,
	backgroundcolor=\color{newGray},
	frame=single,
	rulecolor=\color{anotherGray},
	xleftmargin=\parindent,
	language=bash,
	showstringspaces=false,
	basicstyle=\small\ttfamily,
	keywordstyle=\bfseries\color{green!40!black},
	commentstyle=\itshape\color{purple!40!black},
	identifierstyle=\color{black},
	stringstyle=\color{mblue},
	tabsize=2,
	literate={\ \ }{{\ }}1	
}
\lstset{style=customc}

%==================================
%Macros útiles
\newcommand{\comillas}[1]{``#1''}
\newcommand{\comentarioc}[1]{\texttt{\textcolor{webred}{/* #1 */}}}

\numberwithin{equation}{section}

%=================================
%comandos matemáticos personalizados de rápido acceso
\newcommand{\nparentesis}[1]{\left( #1 \right)}
\newcommand{\llaves}[1]{\left \{ #1 \right \}}
\newcommand{\nabsoluto}[1]{\left| #1 \right|}
\newcommand{\ncorchetes}[1]{\left[ #1 \right]}
%=================================

%cambia el espaciado entre el parrafo y antes del itemize
\setlist[itemize]{itemsep=1mm, topsep=1.5mm}
\setlist[enumerate]{itemsep=1mm, topsep=1.5mm}
%===========================

\theoremstyle{definition}
\newtheorem{teorema}{Teorema}[section]
\newtheorem{definition}{Definición}[section]
\newtheorem{beforeExample}{Ejemplo}[section]
\newtheorem{preposicion}{Preposición}[section]
\newtheorem{corolario}{Corolario}[teorema]

\captionsetup{justification=centering, margin=2.3cm}

\newenvironment{ejemplo}[1][]{\begin{beforeExample}[#1]\renewcommand{\qedsymbol}{$\blacksquare$}}{\qed\end{beforeExample}}

\captionsetup{justification=centering, margin=2.3cm}

% Para poder hacer "quote" con estilo github

\newenvironment{newquote}{\begin{mdframed}[topline=false,
		rightline=false, bottomline=false, linewidth=.3em,backgroundcolor=black!5, linecolor=black!60]\color{black!70}}{\end{mdframed}}
%================================

\renewcommand*{\arraystretch}{1.2}

\usepackage{fancyhdr}

\pagestyle{fancy}
\lhead{INF221\quad Algoritmos y Complejidad}
\rhead{\textit{Clase \#12\quad Subsecuencia común más larga}}

\author{Aldo Berrios Valenzuela}
\title{INF221 -- Algoritmos y Complejidad\\[.4\baselineskip]Clase \#12\\Subsecuencia común más larga}
\date{Miércoles 7 de septiembre de 2016}

\begin{document}
\maketitle
\section{Subsecuencia común más larga}
En inglés conocido como Longest Common Subsequence, LCS. La \emph{idea} consiste en que tenemos dos archivos: $X$ e $Y$, y queremos hacer \texttt{diff}\footnote{pruebe \texttt{man diff} para ver qué es lo que hacer} sobre ellos, es decir:
\begin{lstlisting}
diff X Y
\end{lstlisting}
Para ello:
\begin{itemize}
	\item Hallar la subsecuencia común más larga (LCS) entre ellos.
	\item Marcar líneas agregadas/borradas
\end{itemize}

\begin{ejemplo}
	Supongamos que tenemos dos archivos: $X$ e $Y$, cuyo contenido se muestra en el cuadro \ref{12::ContenidoArchivos}.
	\newcolumntype{T}{>{\ttfamily}l}
	\begin{table}[!h]
		\centering
		\begin{tabular}{T|T}
			\multicolumn{1}{c|}{$X$}&\multicolumn{1}{c}{$Y$}\\
			\hline
			foo&bar\\
			bar&xyzzy\\
			baz&plugh\\
			quux&baz\\
			windows&foo\\
			&quux\\
			&linux
		\end{tabular}
		\caption{Contenido de los archivos $X$ e $Y$. Note que cada fila de la tabla representa una línea del archivo correspondiente.}
		\label{12::ContenidoArchivos}
	\end{table}
	En consecuencia, si hacemos un \begin{lstlisting}
	diff X Y
	\end{lstlisting}
	obtenemos como resultado, la columna \comillas{Resultado} del Cuadro \ref{12::DiffXY}.
	\begin{table}[!h]
		\centering
		\begin{tabular}{c|T|T|TT}
			Línea&\multicolumn{1}{c|}{$X$}&\multicolumn{1}{c|}{$Y$}&\multicolumn{2}{c}{\textrm{Resultado}}\\
			\hline
			1&foo&bar&-&foo\\
			2&bar&xyzzy&&bar\\
			3&baz&plugh&+&xyzzy\\
			4&quux&baz&+&plugh\\
			5&windows&foo&&baz\\
			6&&quux&+&foo\\
			7&&linux&&quux\\
			&&&-&windows\\
			&&&+&linux
		\end{tabular}
		\caption{La columna \comillas{Resultado} se obtiene a través de: \comillas{Partiendo de $X$, ¿cómo obtengo $Y$}. Los \texttt{(-)} significa que debemos remover la línea asociada  para construir $Y$ a partir de $X$ y los \texttt{(+)} es que debemos agregarla.}
		\label{12::DiffXY}
	\end{table}
	
	Explicamos con más detalle cómo funciona el algoritmo acorde a lo arrojado por el Cuadro \ref{12::DiffXY}:
	\begin{itemize}
		\item Comenzamos leyendo ambos archivos. Como podemos observar en el Cuadro \ref{12::ContenidoArchivos}, la primera línea de $Y$ es \texttt{bar}. Por lo tanto, buscamos en $X$ la primera línea que tenga \texttt{bar}. En $X$, la línea \texttt{bar} se encuentra en la línea 2, así que tendremos que eliminar el contenido de la línea 1 de $X$. Por lo tanto, en la columna \comillas{Resultado} agregamos \texttt{-bar}.
		
		\item En este momento, ambos archivos comienzan con \texttt{bar}, así que no realizamos ningún cambio y agregamos esto a \comillas{Resultado}. El Cuadro \ref{12::PrimeraCoincidencia} muestra nuestra dónde estamos ubicados actualmente.
		\begin{table}[!h]
			\centering
			\begin{tabular}{c|T|T|TT}
				Línea&\multicolumn{1}{c|}{$X$}&\multicolumn{1}{c|}{$Y$}&\multicolumn{2}{c}{\textrm{Resultado}}\\
				\hline
				1&foo&\cellcolor{blue!10}bar&-&foo\\
				2&\cellcolor{blue!10}bar&xyzzy&&bar\\
				3&\cellcolor{red!10}baz&plugh&&\\
				4&quux&\cellcolor{red!10}baz&&\\
				5&windows&foo&&\\
				6&&quux&&\\
				7&&linux&&\\
			\end{tabular}
			\caption{Las celdas \textcolor{blue!60}{azules} representan dónde estamos ya sea para el archivo $X$ e $Y$. Por otro lado, las celdas remarcadas con color \textcolor{red!60}{rojo} representan nuestra siguiente coincidencia.}
			\label{12::PrimeraCoincidencia}
		\end{table}
		
		\item Iteramos hasta encontrar la siguiente coincidencia (en este punto nos encontramos en la línea 2 de $X$ y 1 de $Y$). La siguiente coincidencia es la línea \texttt{baz}. Como podemos observar en el Cuadro \ref{12::PrimeraCoincidencia}, para poder construir $Y$ a partir de $X$ agregamos las líneas \texttt{xyzzy} y \texttt{plugh}. Luego, nos ubicamos en las líneas donde se produjo dicha coincidencia para ambos archivos y aprovechamos a de agregarlos (ver Cuadro \ref{12::SegundaCoincidencia}).
		\begin{table}[!h]
			\centering
			\begin{tabular}{c|T|T|TT}
				Línea&\multicolumn{1}{c|}{$X$}&\multicolumn{1}{c|}{$Y$}&\multicolumn{2}{c}{\textrm{Resultado}}\\
				\hline
				1&foo&bar&-&foo\\
				2&bar&xyzzy&&bar\\
				3&\cellcolor{blue!10}baz&plugh&+&xyzzy\\
				4&\cellcolor{red!10}quux&\cellcolor{blue!10}baz&+&plugh\\
				5&windows&foo&&baz\\
				6&&\cellcolor{red!10}quux&&\\
				7&&linux&&\\
			\end{tabular}
			\caption{Siga la misma convención de colores del Cuadro \ref{12::PrimeraCoincidencia}. Por otro lado, agregamos \texttt{xyzzy} y \texttt{plugh} a \comillas{Resultado}.}
			\label{12::SegundaCoincidencia}
		\end{table}
		
		\item La siguiente coincidencia es \texttt{quux}.
		\begin{itemize}
			\item En $X$, \texttt{quux} está inmediatamente después de la línea que tiene \texttt{baz}.
			\item En $Y$, para llegar a  la línea que tiene \texttt{quux} desde \texttt{baz} tendremos que pasar \texttt{foo}.
		\end{itemize}
		Dado lo anterior, en $X$ tendremos que agregar la línea \texttt{foo} para poder construir $Y$. Los resultados se pueden apreciar en el Cuadro \ref{12::TerceraCoincidencia}.
		\begin{table}[!h]
			\centering
			\begin{tabular}{c|T|T|TT}
				Línea&\multicolumn{1}{c|}{$X$}&\multicolumn{1}{c|}{$Y$}&\multicolumn{2}{c}{\textrm{Resultado}}\\
				\hline
				1&foo&bar&-&foo\\
				2&bar&xyzzy&&bar\\
				3&baz&plugh&+&xyzzy\\
				4&\cellcolor{blue!10}quux&baz&+&plugh\\
				5&windows&foo&&baz\\
				6&&\cellcolor{blue!10}quux&+&foo\\
				7&&linux&&quux\\
			\end{tabular}
			\caption{Siga la misma convención de colores del Cuadro \ref{12::PrimeraCoincidencia}. Por otro lado, agregamos \texttt{foo} a \comillas{Resultado}.}
			\label{12::TerceraCoincidencia}
		\end{table}
		
		\item Como podemos apreciar en el Cuadro \ref{12::TerceraCoincidencia}, ya no tenemos más coincidencias. Por lo tanto, para terminar simplemente eliminamos \texttt{windows} y agregamos \texttt{linux}\footnote{No nos arrepentimos de nada.}. La tabla resultante puede verse a través del Cuadro \ref{12::DiffXY}.
	\end{itemize}
\end{ejemplo}

\subsection{Aspectos formales}
Arreglos $X\ncorchetes{1, \ldots, n}$, $Y\ncorchetes{1, \ldots, m}$, palabras sobre un alfabeto (\comillas{símbolo} es una línea). Hallar la secuencia de pares de índices $\nparentesis{x_1, y_1}$, $\nparentesis{x_2, y_2}$, \ldots, $\nparentesis{x_q, y_q}$ tales que \comentarioc{DUDA: ¿los $x_i$ e $y_i$ es el número de línea?}:
\begin{align*}
x_1&<x_2<\ldots <x_q\qquad ;\forall x_i\in\mathbb{N}\\
y_1&<y_2<\ldots <y_q\qquad ;\forall x_i\in\mathbb{N}\\
X\ncorchetes{x_i}&=Y\ncorchetes{y_i}\qquad\text{para } 1\leq i\leq q
\end{align*}
Interesa la secuencia más larga (máximo $q$, correspondiente a la cantidad coincidencias). Para ello, consideremos $X\ncorchetes{n}$, $Y\ncorchetes{m}$. Tres opciones:
\begin{enumerate}
	\item $X\ncorchetes{n}$ queda fuera de la subsecuencia. En consecuencia, lo marcamos con \texttt{(-)}
	\item $Y\ncorchetes{m}$ queda fuera de la subsecuencia. Para efectos de \texttt{diff} marcamos con \texttt{(+)}
	\item [+] O ambos
	\item Si $X\ncorchetes{n}=Y\ncorchetes{m}$, hacer $x_q=n$, $y_q=m$.
\end{enumerate}
Notar que si $X\ncorchetes{n}\ne Y\ncorchetes{m}$ no pueden pertenecer ambos a la LCS.

Tres opciones. Subproblemas:
\begin{enumerate}
	\item $X\ncorchetes{1, \ldots, n-1}$, $Y$
	\item $X$, $Y\ncorchetes{1, \ldots, m-1}$
	\item $X\ncorchetes{1, \ldots, n-1}$, $Y\ncorchetes{1, \ldots, m-1}\leftarrow X\ncorchetes{n}=Y\ncorchetes{m}$
\end{enumerate}

Sea $LCS\nparentesis{A, B}$ la subsecuencia común más larga entre $A$ y $B$. Entonces \comentarioc{hasta el momento sólo nos interesa el largo de la subsecuencia óptima, después vemos como encontrar la secuencia}:
\begin{align*}
\nabsoluto{LCS\nparentesis{X, Y}}=\max\{\nabsoluto{LCS\nparentesis{X\ncorchetes{1, \ldots, n-1}, Y}}+0, \nabsoluto{LCS\nparentesis{X, Y\ncorchetes{1, \ldots, m-1}}}+0,\\
 \nabsoluto{LCS\nparentesis{X\ncorchetes{1, \ldots, n-1}, Y\ncorchetes{1, \ldots, m-1}}+1}\}
\end{align*}
$\rightsquigarrow$ Arreglo $L\ncorchetes{i, j}$:
\begin{equation}
L\ncorchetes{i, j}=\nabsoluto{LCS\nparentesis{X\ncorchetes{1, \ldots, i}, Y\ncorchetes{1, \ldots, j}}}
\end{equation}
Sabemos $L\ncorchetes{0, j}=L\ncorchetes{i, 0}=0$. Para calcular $L\ncorchetes{i, j}$ necesitamos $L\ncorchetes{i-1, j}$, $L\ncorchetes{i, j-1}$, $L\ncorchetes{i-1, j-1}$ (posiblemente). Llenar el arreglo, calculando $L\ncorchetes{i, k}$ para $i$ de $1$ a $n$, llenando los $j$ de $1$ a $m$.
\begin{itemize}
	\item [$\rightsquigarrow$] Costo total es $O\nparentesis{n\cdot m}$
\end{itemize}


\end{document}
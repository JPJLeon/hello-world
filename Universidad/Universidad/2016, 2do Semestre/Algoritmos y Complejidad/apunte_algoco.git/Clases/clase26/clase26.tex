\documentclass[english, spanish, fleqn, 10pt]{article}
\usepackage[top = 2.5cm, bottom = 2cm, left = 2.5cm, right = 2.5cm]{geometry}
\usepackage[es-noindentfirst]{babel}
\usepackage[utf8]{inputenc}
\usepackage{amsthm, amsmath, amssymb, amsfonts, environ}
\usepackage{caption}
\usepackage{subcaption} % para las subfigures
\usepackage{mathrsfs}
\usepackage[colorlinks, urlcolor=blue,
pdfauthor={Aldo Berrios Valenzuela}]{hyperref}
\usepackage{fourier}
\usepackage{colortbl}
\usepackage{color}
\usepackage{graphicx}
\usepackage{enumitem}
%\renewcommand{\familydefault}{\sfdefault}
%\setlength{\parindent}{0pt}					%Espacio de sangria
\setlength{\parskip}{0.5mm}						%Interlinado entre párrafos
\usepackage{fancyhdr, blindtext}		%Paquete de encabezado
\usepackage{listings}
\usepackage[framemethod=tikz]{mdframed}

\definecolor{webred}{rgb}{0.75,0,0}
\definecolor{mblue}{rgb}{0.0705,0.345,0.7529}
\definecolor{newmblue}{HTML}{004D99}
\usepackage{longtable, colortbl, array}

\definecolor{superGreenNew}{HTML}{169F01}
\hypersetup{
	colorlinks   = true,
	citecolor    = superGreenNew
}


%==============================
%Modificador de Titulos de secciones, subsecciones, etc
\usepackage[explicit, noindentafter]{titlesec}

\titlespacing\subsubsection{0pt}{8pt plus 4pt minus 2pt}{4pt plus 2pt minus 2pt}


%==================================
%Diseno para insertar codigos de programacion
\definecolor{newGray}{HTML}{F8F8F8}
\definecolor{anotherGray}{HTML}{DDDFFF}
\lstdefinestyle{customc}{
	belowcaptionskip=1\baselineskip,
	breaklines=true,
	backgroundcolor=\color{newGray},
	frame=single,
	rulecolor=\color{anotherGray},
	xleftmargin=\parindent,
	language=bash,
	showstringspaces=false,
	basicstyle=\small\ttfamily,
	keywordstyle=\bfseries\color{red!40!black},
	commentstyle=\itshape\color{purple!40!black},
	identifierstyle=\color{black},
	stringstyle=\color{mblue},
	tabsize=2,
	literate={\ \ }{{\ }}1	
}
\lstset{style=customc}

\renewcommand{\lstlistingname}{Listado}

%==================================
%Macros útiles
\newcommand{\comillas}[1]{``#1''}
\newcommand{\comentarioc}[1]{\texttt{\textcolor{webred}{/* #1 */}}}

\numberwithin{equation}{section}

%=================================
%comandos matemáticos personalizados de rápido acceso
\newcommand{\nparentesis}[1]{\left( #1 \right)}
\newcommand{\llaves}[1]{\left \{ #1 \right \}}
\newcommand{\nabsoluto}[1]{\left| #1 \right|}
\newcommand{\ncorchetes}[1]{\left[ #1 \right]}
%=================================

%cambia el espaciado entre el parrafo y antes del itemize
\setlist[itemize]{itemsep=1mm, topsep=1.5mm}
\setlist[enumerate]{itemsep=1mm, topsep=1.5mm}
%===========================

\theoremstyle{definition}
\newtheorem{teorema}{Teorema}[section]
\newtheorem{definition}{Definición}[section]
\newtheorem{beforeExample}{Ejemplo}[section]
\newtheorem{preposicion}{Preposición}[section]
\newtheorem{corolario}{Corolario}[teorema]

\captionsetup{justification=centering, margin=2.3cm}

\newenvironment{ejemplo}[1][]{\begin{beforeExample}[#1]\renewcommand{\qedsymbol}{$\blacksquare$}}{\qed\end{beforeExample}}

\captionsetup{justification=centering, margin=2.3cm}

% Para poder hacer "quote" con estilo github

\newenvironment{newquote}{\begin{mdframed}[topline=false,
		rightline=false, bottomline=false, linewidth=.3em,backgroundcolor=black!5, linecolor=black!60]\color{black!70}}{\end{mdframed}}
%================================

\renewcommand*{\arraystretch}{1.2}

\usepackage{fancyhdr}

\pagestyle{fancy}
\lhead{<inserte nombre del documento>}
\rhead{<inserte contenido>}


\title{INF221 -- Algoritmos y Complejidad\\[.4\baselineskip]
Clase \#26\\
Análisis Amortizado}
\author{Aldo Berrios Valenzuela}


\begin{document}
\maketitle

\section{Análisis Amortizado}
\emph{Idea:} Analizar \emph{secuencias} de operaciones.

\begin{definition}
	El \emph{costo amortizado} de una operación es el costo de una secuencia de operaciones dividida por su largo.
\end{definition}

\begin{ejemplo}[Arreglo dinámico]
	Un stack se puede representar mediante un arreglo, operaciones son:
	\begin{lstlisting}
typedef item ...;
item A[...];
static int top = 0;

void push(item A[], item a) {
	A[top++] = x;
}
item pop(item A[]) {
	return A[--top];
}
	\end{lstlisting}
	Nótese que falta una serie de burocracias que consideran cuando la operación se sale de rango y otros, pero sólo nos interesa lo de arriba.
	
	¿Qué hacer si el arreglo se llena?. Bienvenido a \texttt{realloc(3)} \ldots
	
	Medida de costo: Nº de items copiados.
		
	Si cada vez que llegamos al tamaño del arreglo lo agrandamos en 1, el costo de $n$ \texttt{push} es $n\nparentesis{n+1}/2$, costo medio es $\nparentesis{n+1}/2$ \comentarioc{es por el realloc()}. Inaceptable \ldots
	
	El costo de $n$ push es la suma:
	\begin{align*}
	1 + 2 + 1 + 4 + 1 + 1 + 1 + 8 + \cdots &= n + \sum_{0 \leq k \leq \left\lfloor \log_2 n \right\rfloor} 2^k\\
	&= n + 2^{\left\lfloor \log_2 n \right\rfloor + 1} - 1\\
	&\leq n + 2n\\
	&= 3n
	\end{align*}
	Por lo tanto, costo amortizado $< 3$. \comentarioc{nos interesa el costo de nuestro programa completo, no el costo de operación por operación}
	
	\emph{Análisis agregado:} Se calcula el costo total (peor caso) de $n$ operaciones se divide por $n$.
\end{ejemplo}
	
\subsection{Método contable}
\emph{Idea:} Cada operación paga por \emph{su} costo y ahorra una cantidad para pagar sobre costos de operaciones futuras. Hay que asegurarse que el saldo no se haga negativo \ldots
	
\subsection{Función Potencial}
\emph{Idea:} La estructura tiene un \emph{potencial} $\phi\nparentesis{s}$ ($s$: estado de la estructura). Operaciones $\sigma_1$, $\sigma_2$, \ldots, $\sigma_n$ llevan a la estructura de $s_0$ a $s_1$, a $s_2, \ldots,$ a $s_n$.

El costo de cada operación es $c_i$, su costo amortizado es:
\begin{equation*}
a_i = c_i + \phi\nparentesis{s_i} - \phi\nparentesis{s_{i-1}}
\end{equation*}	
Costo amortizado de la secuencia (¡Viva la telescópica!) es:
\begin{align*}
\sum_{i} a_i &= \sum_i \nparentesis{c_i + \phi \nparentesis{s_i} - \phi \nparentesis{s_{i-1}}}\\
&= \sum_i c_i + \phi \nparentesis{s_n} - \phi \nparentesis{s_0}
\end{align*}
Si (¡Caso común!) $\phi \nparentesis{s_n} \geq \phi \nparentesis{s_0}$:
\begin{equation*}
\sum_i a_i \geq \sum _i c_i \rightsquigarrow \dfrac{\sum_i a_i}{n}
\end{equation*}
es costo amortizado de los $c_i$.





\end{document}


\documentclass[english, spanish, fleqn, 10pt]{article}
\usepackage[top = 2.5cm, bottom = 2cm, left = 2.5cm, right = 2.5cm]{geometry}
\usepackage[es-noindentfirst]{babel}
\usepackage[utf8]{inputenc}
\usepackage{amsthm, amsmath, amssymb, amsfonts, environ}
\usepackage{caption}
\usepackage{subcaption} % para las subfigures
\usepackage{mathrsfs}
\usepackage[colorlinks, urlcolor=blue,
pdfauthor={Aldo Berrios Valenzuela}]{hyperref}
\usepackage{fourier}
\usepackage{colortbl}
\usepackage{color}
\usepackage{graphicx}
\usepackage{enumitem}
%\renewcommand{\familydefault}{\sfdefault}
%\setlength{\parindent}{0pt}					%Espacio de sangria
\setlength{\parskip}{0.5mm}						%Interlinado entre párrafos
\usepackage{fancyhdr, blindtext}		%Paquete de encabezado
\usepackage{listings}
\usepackage[framemethod=tikz]{mdframed}

\definecolor{webred}{rgb}{0.75,0,0}
\definecolor{mblue}{rgb}{0.0705,0.345,0.7529}
\definecolor{newmblue}{HTML}{004D99}
\usepackage{longtable, colortbl, array}

\definecolor{superGreenNew}{HTML}{169F01}
\hypersetup{
	colorlinks   = true,
	citecolor    = superGreenNew
}


%==============================
%Modificador de Titulos de secciones, subsecciones, etc
\usepackage[explicit, noindentafter]{titlesec}

\titlespacing\subsubsection{0pt}{8pt plus 4pt minus 2pt}{4pt plus 2pt minus 2pt}


%==================================
%Diseno para insertar codigos de programacion
\definecolor{newGray}{HTML}{F8F8F8}
\definecolor{anotherGray}{HTML}{DDDFFF}
\lstdefinestyle{customc}{
	belowcaptionskip=1\baselineskip,
	breaklines=true,
	backgroundcolor=\color{newGray},
	frame=single,
	rulecolor=\color{anotherGray},
	xleftmargin=\parindent,
	language=bash,
	showstringspaces=false,
	basicstyle=\small\ttfamily,
	keywordstyle=\bfseries\color{red!40!black},
	commentstyle=\itshape\color{purple!40!black},
	identifierstyle=\color{black},
	stringstyle=\color{mblue},
	tabsize=2,
	literate={\ \ }{{\ }}1	
}
\lstset{style=customc}

\renewcommand{\lstlistingname}{Listado}

%==================================
%Macros útiles
\newcommand{\comillas}[1]{``#1''}
\newcommand{\comentarioc}[1]{\texttt{\textcolor{webred}{/* #1 */}}}

\numberwithin{equation}{section}

%=================================
%comandos matemáticos personalizados de rápido acceso
\newcommand{\nparentesis}[1]{\left( #1 \right)}
\newcommand{\llaves}[1]{\left \{ #1 \right \}}
\newcommand{\nabsoluto}[1]{\left| #1 \right|}
\newcommand{\ncorchetes}[1]{\left[ #1 \right]}
%=================================

%cambia el espaciado entre el parrafo y antes del itemize
\setlist[itemize]{itemsep=1mm, topsep=1.5mm}
\setlist[enumerate]{itemsep=1mm, topsep=1.5mm}
%===========================

\theoremstyle{definition}
\newtheorem{teorema}{Teorema}[section]
\newtheorem{definition}{Definición}[section]
\newtheorem{beforeExample}{Ejemplo}[section]
\newtheorem{preposicion}{Preposición}[section]
\newtheorem{corolario}{Corolario}[teorema]

\captionsetup{justification=centering, margin=2.3cm}

\newenvironment{ejemplo}[1][]{\begin{beforeExample}[#1]\renewcommand{\qedsymbol}{$\blacksquare$}}{\qed\end{beforeExample}}

\captionsetup{justification=centering, margin=2.3cm}

% Para poder hacer "quote" con estilo github

\newenvironment{newquote}{\begin{mdframed}[topline=false,
		rightline=false, bottomline=false, linewidth=.3em,backgroundcolor=black!5, linecolor=black!60]\color{black!70}}{\end{mdframed}}
%================================

\renewcommand*{\arraystretch}{1.2}

\usepackage{fancyhdr}

\pagestyle{fancy}
\lhead{INF221\quad Algoritmos y Complejidad}
\rhead{\textit{Clase \#20\quad Método Simbólico II}}


\author{
	Horst H. von Brand}

\title{INF221 -- Algoritmos y Complejidad\\[.4\baselineskip]
	Clase \#20\\
	Método Simbólico II}

\date{Martes 18 de octubre de 2016}

\DeclareMathOperator{\Seq}{\textsc{Seq}}
\DeclareMathOperator{\Cyc}{\textsc{Cyc}}
\DeclareMathOperator{\Set}{\textsc{Set}}
\DeclareMathOperator{\MSet}{\textsc{MSet}}

\usepackage{tikz,ifthen}
\usetikzlibrary{automata,positioning, babel}
\definecolor{navyblue}{RGB}{0,148,222}
\usetikzlibrary{fit}
\tikzset{shorten >=1pt,node distance=1.8cm,on grid, >=stealth, initial text=Inicio,
	every state/.style={ draw=black ,fill=black, minimum size=2mm},
	bend angle=35, every loop/.style={looseness=13}}


\begin{document}

\maketitle

\section{Objetos rotulados}
\label{sec:objetos-rotulados}

En la discusión previa solo interesaba el tamaño de los objetos,
no su disposición particular.
Consideraremos ahora objetos rotulados,
donde importa cómo se compone el objeto de sus partes
(los átomos tienen identidades,
posiblemente porque se ubican en orden).

El objeto más simple con partes rotuladas son las permutaciones
(biyecciones \(\sigma \colon [n] \rightarrow [n]\),
podemos considerarlas secuencias de átomos numerados).
Para la función generatriz exponencial tenemos,
ya que hay \(n!\) permutaciones de \(n\) elementos:
\begin{equation*}
\sum_{\sigma}
\frac{z^{\lvert \sigma \rvert}}{\lvert \sigma \rvert !}
= \sum_{n \ge 0} n! \, \frac{z^n}{n!}
= \frac{1}{1 - z}
\end{equation*}

Lo siguiente más simple de considerar
es colecciones de ciclos rotulados.
Por ejemplo,
escribimos \((1\;3\;2)\) para el objeto
en que viene \(3\) luego de \(1\),
\(2\) sigue a \(3\),
y a su vez \(1\) sigue a \(2\) (Figura \ref{20::ciclo:rotulado:1}).
\begin{figure}[!h]
	\centering
	\begin{tikzpicture}
		\node (1) {$1$};
		\node [below right of = 1] (2) {$3$};
		\node [below left of = 1] (3) {$2$};
		
		\path [->] (1) edge (2)
		(2) edge (3)
		(3) edge (1);		
	\end{tikzpicture}
	\caption{Representación del ciclo rotulado $(1\; 3\; 2)$. Si hacemos una rotación del ciclo veremos claramente como sus elementos han cambiado de posición.}
	\label{20::ciclo:rotulado:1}
\end{figure}
Así \((2\;1\;3)\) es solo otra forma de anotar el ciclo anterior,
que no es lo mismo que \((3\;1\;2)\) (véase la Figura \ref{20::ciclo:rotulado:2}).
\begin{figure}[!h]
	\centering
	\begin{tikzpicture}
		\node (3) {$3$};
		\node [below right of = 3] (1) {$1$};
		\node [below left of = 3] (2) {$2$};
		\path [->] (3) edge (1)
		(1) edge (2)
		(2) edge (3);
	\end{tikzpicture}
	\caption{Representación del ciclo rotulado $(3\; 1\; 2)$}
	\label{20::ciclo:rotulado:2}
\end{figure}
Interesa definir formas consistentes
de combinar objetos rotulados.
Por ejemplo,
al combinar el ciclo \((1\;2)\) con el ciclo \((1\;3\;2)\)
resultará un objeto con \(5\) rótulos,
y debemos ver cómo los distribuimos entre las partes.
El cuadro~\ref{tab:ciclo+ciclo}
reseña las posibilidades al respetar
el orden de los elementos asignados a cada parte.
\begin{table}[htbp]
	\centering
	\begin{tabular}{*{4}{>{\(}l<{\)}}}
		(1\;2) (3\;5\;4) & (2\;3) (1\;5\;4) & (3\;4) (1\;5\;2) &
		(4\;5) (1\;3\;2) \\
		(1\;3) (2\;5\;4) & (2\;4) (1\;3\;5) & (3\;5) (1\;4\;2) \\
		(1\;4) (2\;5\;3) & (2\;5) (1\;3\;4) \\
		(1\;5) (2\;4\;3)
	\end{tabular}
	\caption{Combinando los ciclos \((1\;2)\) y \((1\;3\;2)\)}
	\label{tab:ciclo+ciclo}
\end{table}
Es claro que lo que estamos haciendo es elegir
un subconjunto de \(2\) rótulos
de entre los \(5\) para asignárselos al primer ciclo.
El combinar
dos clases de objetos \(\mathcal{A}\) y \(\mathcal{B}\)
de esta forma lo anotaremos \(\mathcal{A} \star \mathcal{B}\).

Otra operación común es la \emph{composición},
anotada \(\mathcal{A} \circ \mathcal{B}\).
La idea es elegir un elemento \(\alpha \in \mathcal{A}\),
luego elegir \(\lvert \alpha \rvert\) elementos
de \(\mathcal{B}\),
y reemplazar los \(\mathcal{B}\) por las partes de \(\alpha\),
en el orden que están rotuladas;
para finalmente asignar rótulos a los átomos
que conforman la estructura completa
respetando el orden de los rótulos
al interior de los \(\mathcal{B}\)
(igual como lo hicimos para \(\star\)).

Ocasionalmente es útil \emph{marcar}
uno de los componentes del objeto,
operación que anotaremos \(\mathcal{A}^\bullet\).
Usaremos también la construcción \(\MSet(\mathcal{A})\),
que podemos considerar como una secuencia de elementos numerados
obviando el orden.
Cuidado,
muchos textos le llaman \(\Set()\) a esta operación.

Tenemos el siguiente teorema:
\begin{teorema}[Método simbólico, EGF]
	\label{theo:ms-EGF}
	Sean \(\mathcal{A}\) y \(\mathcal{B}\) clases de objetos,
	con funciones generatrices exponenciales
	\(\widehat{A}(z)\) y \(\widehat{B}(z)\),
	respectivamente.
	Entonces tenemos
	las siguientes funciones generatrices exponenciales:
	\begin{enumerate}
		\item
		Para enumerar \(\mathcal{A}^\bullet\):
		\begin{equation*}
		z \mathrm{D} \widehat{A}(z)
		\end{equation*}
		\item
		Para enumerar \(\mathcal{A} + \mathcal{B}\):
		\begin{equation*}
		\widehat{A}(z) + \widehat{B}(z)
		\end{equation*}
		\item
		Para enumerar \(\mathcal{A} \star \mathcal{B}\):
		\begin{equation*}
		\widehat{A}(z) \cdot \widehat{B}(z)
		\end{equation*}
		\item
		Para enumerar \(\mathcal{A} \circ \mathcal{B}\):
		\begin{equation*}
		\widehat{A}(\widehat{B}(z))
		\end{equation*}
		\item
		Para enumerar \(\Seq(\mathcal{A})\):
		\begin{equation*}
		\frac{1}{1 - \widehat{A}(z)}
		\end{equation*}
		\item
		Para enumerar \(\MSet(\mathcal{A})\):
		\begin{equation*}
		\mathrm{e}^{\widehat{A}(z)}
		\end{equation*}
		\item
		Para enumerar \(\Cyc(\mathcal{A})\):
		\begin{equation*}
		-\ln(1 - \widehat{A}(z))
		\end{equation*}
	\end{enumerate}
\end{teorema}
\begin{proof}
	Usaremos casos ya demostrados en las demostraciones sucesivas.
	\begin{enumerate}
		\item % mark A
		El objeto \(\alpha \in \mathcal{A}\)
		da lugar a \(\lvert \alpha \rvert\) objetos
		al marcar cada uno de sus átomos,
		lo que da la función generatriz exponencial:
		\begin{equation*}
		\sum_{\alpha \in \mathcal{A}}
		\lvert \alpha \rvert
		\frac{z^{\lvert \alpha \rvert}}{\lvert \alpha \rvert !}
		\end{equation*}
		Esto es lo indicado.
		\item % A + B
		Nuevamente trivial.
		\item % A x B
		El número de objetos \(\gamma\) que se obtienen
		al combinar \(\alpha \in \mathcal{A}\)
		con \(\beta \in \mathcal{B}\) es:
		\begin{equation*}
		\binom{\lvert \alpha \rvert + \lvert \beta \rvert}
		{\lvert \alpha \rvert}
		\end{equation*}
		y tenemos la función generatriz exponencial:
		\begin{equation*}
		\sum_{\gamma \in \mathcal{A} \star \mathcal{B}}
		\frac{z^{\lvert \gamma \rvert}}{\lvert \gamma \rvert !}
		= \sum_{\substack{
				\alpha \in \mathcal{A} \\
				\beta \in \mathcal{B}
			}}
			\binom{\lvert \alpha \rvert + \lvert \beta \rvert}
			{\lvert \alpha \rvert}
			\frac{z^{\lvert \alpha \rvert
					+ \lvert \beta \rvert}}
			{(\lvert \alpha \rvert
				+ \lvert \beta \rvert)!}
			= \left(
			\sum_{\alpha \in \mathcal{A}}
			\frac{z^{\lvert \alpha \rvert}}
			{\lvert \alpha \rvert !}
			\right)
			\cdot \left(
			\sum_{\beta \in \mathcal{B}}
			\frac{z^{\lvert \beta \rvert}}
			{\lvert \beta \rvert !}
			\right)
			= \widehat{A}(z) \cdot \widehat{B}(z)
			\end{equation*}
			\item % A circ B
			Tomemos \(\alpha \in \mathcal{A}\),
			de tamaño \(n = \lvert \alpha \rvert\),
			y \(n\) elementos de \(\mathcal{B}\) en orden
			a ser reemplazados por las partes de \(\alpha\).
			Esa secuencia de \(\mathcal{B}\) es representada por:
			\begin{equation*}
			\mathcal{B}
			\star \mathcal{B}
			\star \dotsb
			\star \mathcal{B}
			\end{equation*}
			con función generatriz exponencial:
			\begin{equation*}
			\widehat{B}^n (z)
			\end{equation*}
			Sumando sobre las contribuciones:
			\begin{equation*}
			\sum_{\alpha \in \mathcal{A}}
			\frac{\widehat{B}^{\lvert \alpha \rvert}(z)}
			{\lvert \alpha \rvert \, !}
			\end{equation*}
			Esto es lo prometido.
			\item % Seq(A)
			Primeramente,
			podemos describir:
			\begin{equation*}
			\Seq(\mathcal{Z})
			= \mathcal{E} + \mathcal{Z} \star \Seq(\mathcal{Z})
			\end{equation*}
			que lleva a ecuación:
			\begin{equation*}
			\widehat{S}(z)
			= 1 + z \widehat{S}(z)
			\end{equation*}
			de donde:
			\begin{equation*}
			\widehat{S}(z)
			= \frac{1}{1 - z}
			\end{equation*}
			Aplicando composición se obtiene lo indicado.
			\item % MSet(A)
			Hay un único multiconjunto de \(n\) elementos rotulados
			(se rotulan simplemente de 1 a  \(n\)),
			con lo que \(\MSet(\mathcal{Z})\) corresponde a:
			\begin{equation*}
			\sum_{n \ge 0} \frac{z^n}{n!}
			= \exp(z)
			\end{equation*}
			Al aplicar composición resulta lo anunciado.
			
			Otra demostración es considerar el multiconjunto de \(\mathcal{A}\),
			descrito por \(\mathcal{M} = \MSet(\mathcal{A})\).
			Si marcamos uno de los átomos de \(\mathcal{M}\)
			estamos marcando uno de los \(\mathcal{A}\),
			el resto sigue formando un multiconjunto de \(\mathcal{A}\):
			\begin{equation*}
			\mathcal{M}^\bullet
			= \mathcal{A}^\bullet \star \mathcal{M}
			\end{equation*}
			Por lo anterior:
			\begin{equation*}
			z \widehat{M}'(z)
			= z \widehat{A}'(z) \widehat{M}(z)
			\end{equation*}
			Hay un único multiconjunto de tamaño \(0\),
			o sea \(\widehat{M}(0) = 1\);
			y hemos impuesto la condición
			que no hay objetos de tamaño \(0\) en \(\mathcal{A}\),
			vale decir,
			\(\widehat{A}(0) = 0\).
			Así la solución a la ecuación diferencial es:
			\begin{equation*}
			\widehat{M}(z)
			= \exp(\widehat{A}(z))
			\end{equation*}
			\item % Cyc(A)
			Consideremos un ciclo de \(\mathcal{A}\),
			o sea \(\mathcal{C} = \Cyc(\mathcal{A})\).
			Si marcamos los \(\mathcal{C}\),
			estamos marcando uno de los \(\mathcal{A}\),
			y el resto es una secuencia:
			\begin{equation*}
			\mathcal{C}^\bullet
			= \mathcal{A}^\bullet \star \Seq(\mathcal{A})
			\end{equation*}
			Esto se traduce en la ecuación diferencial:
			\begin{equation*}
			z \widehat{C}'(z)
			= z \widehat{A}'(z) \frac{1}{1 - \widehat{A}(z)}
			\end{equation*}
			Integrando bajo el entendido \(\widehat{C}(0) = 0\)
			con \(\widehat{A}(0) = 0\)
			se obtiene lo indicado.
			
			Alternativamente,
			hay \((n - 1)!\) ciclos de \(n\) elementos,
			con lo que para \(\Cyc(\mathcal{Z})\) obtenemos:
			\begin{equation*}
			\sum_{n \ge 1} (n - 1)! \frac{z^n}{n!}
			= \sum_{n \ge 1} \frac{z^n}{n}
			= - \ln \frac{1}{1 - z}
			\end{equation*}
			Aplicar composición completa la demostración.
		\end{enumerate}
	\end{proof}
	
\subsection{Algunas aplicaciones}
\label{sec:ms-egf-aplicaciones}
	
Un ejemplo simple es el caso de permutaciones,
que son simplemente secuencias de elementos rotulados:
\begin{align*}
	\mathcal{P}
	&= \Seq(\mathcal{Z}) \\
	\widehat{P}(z)
	&= \frac{1}{1 - z} \\
	p_n
	&= n! [z^n] \widehat{P}(z) \\
	&= n!
\end{align*}
	
Consideremos colecciones de ciclos:
\begin{equation*}
	\MSet(\Cyc(\mathcal{Z}))
\end{equation*}
Vemos que esto corresponde a:
\begin{equation*}
	\exp\left( \ln \frac{1}{1 - z} \right)
	= \frac{1}{1 - z}
\end{equation*}
Hay tantas permutaciones de tamaño \(n\) como colecciones de ciclos.
Podemos representar permutaciones
como los ciclos que parten en cada elemento.
	
Podemos describir permutaciones
como un conjunto de elementos que quedan fijos
combinado con otros elementos que están fuera de orden
	(un \emph{desarreglo},
	clase \(\mathcal{D}\)).
	O sea:
	\begin{align*}
	\mathcal{P}
	&= \mathcal{D} \star \MSet(\mathcal{Z}) \\
	\frac{1}{1 - z}
	&= \widehat{D}(z) \mathrm{e}^z \\
	\widehat{D}(z)
	&= \frac{\mathrm{e}^{-z}}{1 - z}
	\end{align*}
	De acá,
	por propiedades de las funciones generatrices vemos que:
	\begin{equation*}
	[z^n] \widehat{D}(z)
	= \sum_{0 \le k \le n} \frac{(-1)^k}{k!}
	\end{equation*}
	y tenemos,
	usando la notación común para el número de desarreglos de tamaño \(n\):
	\begin{equation*}
	D_n
	= n! \sum_{0 \le k \le n} \frac{(-1)^k}{k!}
	\end{equation*}
	
	Una \emph{involución} es una permutación \(\pi\)
	tal que \(\pi \circ \pi\) es la identidad.
	Es claro que una involución
	es una colección de ciclos de largos \(1\) y \(2\),
	o sea:
	\begin{equation*}
	\mathcal{I}
	= \MSet(\Cyc_{\le 2}(\mathcal{Z}))
	\end{equation*}
	Revisando la derivación,
	vemos que \(\Cyc_{\le 2}(\mathcal{Z})\) corresponde a:
	\begin{equation*}
	\frac{z}{1} + \frac{z^2}{2}
	\end{equation*}
	y tenemos para la función generatriz:
	\begin{equation*}
	\widehat{I}(z)
	= \exp\left( \frac{z}{1} + \frac{z^2}{2} \right)
	\end{equation*}
	Un paquete de álgebra simbólica da:
	\begin{equation*}
	\widehat{I}(z)
	= 1 +    1 \cdot \frac{z}{1!}
	+    2 \cdot \frac{z^2}{2!}
	+    4 \cdot \frac{z^3}{3!}
	+   10 \cdot \frac{z^4}{4!}
	+   26 \cdot \frac{z^5}{5!}
	+   76 \cdot \frac{z^6}{6!}
	+  232 \cdot \frac{z^7}{7!}
	+  764 \cdot \frac{z^8}{8!}
	+ 2620 \cdot \frac{z^9}{9!}
	+ 9496 \cdot \frac{z^{10}}{10!}
	+ \dotsm
	\end{equation*}
	
	Un \emph{desarreglo} es una permutación sin puntos fijos,
	vale decir,
	sin ciclos de largo \(1\).
	O sea:
	\begin{equation*}
	\mathcal{D}
	= \MSet(Cyc_{> 1}(\mathcal{Z})
	\end{equation*}
	Revisando las derivaciones para las operaciones,
	esto corresponde a:
	\begin{align*}
	\widehat{D}(z)
	&= \exp \left( \sum_{k \ge 2} \frac{z^k}{k} \right) \\
	&= \exp \left( \sum_{k \ge 1} \frac{z^k}{k} - z \right) \\
	&= \exp \left( \ln \frac{1}{1 - z} - z \right) \\
	&= \frac{1}{1 - z} \cdot e^{-z}
	\end{align*}
	Igual que antes.
	
\bibliography{../referencias}


\end{document}

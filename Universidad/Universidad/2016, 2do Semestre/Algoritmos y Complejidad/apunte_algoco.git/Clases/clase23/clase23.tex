\documentclass[english, spanish, fleqn, 10pt]{article}
\usepackage[top = 2.5cm, bottom = 2cm, left = 2.5cm, right = 2.5cm]{geometry}
\usepackage[es-noindentfirst]{babel}
\usepackage[utf8]{inputenc}
\usepackage{amsthm, amsmath, amssymb, amsfonts, environ}
\usepackage{caption}
\usepackage{subcaption} % para las subfigures
\usepackage{mathrsfs}
\usepackage[colorlinks, urlcolor=blue,
pdfauthor={Aldo Berrios Valenzuela}]{hyperref}
\usepackage{fourier}
\usepackage{colortbl}
\usepackage{color}
\usepackage{graphicx}
\usepackage{enumitem}
%\renewcommand{\familydefault}{\sfdefault}
%\setlength{\parindent}{0pt}					%Espacio de sangria
\setlength{\parskip}{0.5mm}						%Interlinado entre párrafos
\usepackage{fancyhdr, blindtext}		%Paquete de encabezado
\usepackage{listings}
\usepackage[framemethod=tikz]{mdframed}

\definecolor{webred}{rgb}{0.75,0,0}
\definecolor{mblue}{rgb}{0.0705,0.345,0.7529}
\definecolor{newmblue}{HTML}{004D99}
\usepackage{longtable, colortbl, array}

\definecolor{superGreenNew}{HTML}{169F01}
\hypersetup{
	colorlinks   = true,
	citecolor    = superGreenNew
}


%==============================
%Modificador de Titulos de secciones, subsecciones, etc
\usepackage[explicit, noindentafter]{titlesec}

\titlespacing\subsubsection{0pt}{8pt plus 4pt minus 2pt}{4pt plus 2pt minus 2pt}


%==================================
%Diseno para insertar codigos de programacion
\definecolor{newGray}{HTML}{F8F8F8}
\definecolor{anotherGray}{HTML}{DDDFFF}
\lstdefinestyle{customc}{
	belowcaptionskip=1\baselineskip,
	breaklines=true,
	backgroundcolor=\color{newGray},
	frame=single,
	rulecolor=\color{anotherGray},
	xleftmargin=\parindent,
	language=bash,
	showstringspaces=false,
	basicstyle=\small\ttfamily,
	keywordstyle=\bfseries\color{red!40!black},
	commentstyle=\itshape\color{purple!40!black},
	identifierstyle=\color{black},
	stringstyle=\color{mblue},
	tabsize=2,
	literate={\ \ }{{\ }}1	
}
\lstset{style=customc}

\renewcommand{\lstlistingname}{Listado}

%==================================
%Macros útiles
\newcommand{\comillas}[1]{``#1''}
\newcommand{\comentarioc}[1]{\texttt{\textcolor{webred}{/* #1 */}}}

\numberwithin{equation}{section}

%=================================
%comandos matemáticos personalizados de rápido acceso
\newcommand{\nparentesis}[1]{\left( #1 \right)}
\newcommand{\llaves}[1]{\left \{ #1 \right \}}
\newcommand{\nabsoluto}[1]{\left| #1 \right|}
\newcommand{\ncorchetes}[1]{\left[ #1 \right]}
%=================================

%cambia el espaciado entre el parrafo y antes del itemize
\setlist[itemize]{itemsep=1mm, topsep=1.5mm}
\setlist[enumerate]{itemsep=1mm, topsep=1.5mm}
%===========================

\theoremstyle{definition}
\newtheorem{teorema}{Teorema}[section]
\newtheorem{definition}{Definición}[section]
\newtheorem{beforeExample}{Ejemplo}[section]
\newtheorem{preposicion}{Preposición}[section]
\newtheorem{corolario}{Corolario}[teorema]

\captionsetup{justification=centering, margin=2.3cm}

\newenvironment{ejemplo}[1][]{\begin{beforeExample}[#1]\renewcommand{\qedsymbol}{$\blacksquare$}}{\qed\end{beforeExample}}

\captionsetup{justification=centering, margin=2.3cm}

% Para poder hacer "quote" con estilo github

\newenvironment{newquote}{\begin{mdframed}[topline=false,
		rightline=false, bottomline=false, linewidth=.3em,backgroundcolor=black!5, linecolor=black!60]\color{black!70}}{\end{mdframed}}
%================================

\renewcommand*{\arraystretch}{1.2}

\usepackage{fancyhdr}

\pagestyle{fancy}
\lhead{<inserte nombre del documento>}
\rhead{<inserte contenido>}


\author{Aldo Berrios Valenzuela}
\title{INF221 -- Algoritmos y Complejidad\\[.4\baselineskip]
	Clase \#23\\
	Algoritmo de Kruskal}

\begin{document}
\maketitle

\section{Algoritmo de Kruskal}
Dado un grafo conexo $G = \nparentesis{V, E}$ con rótulos $\omega: E\rightarrow \mathbb{R}^+$ \comentarioc{rotulamos los arcos, a cada arco le estamos asignando un peso)}, buscamos el árbol recubridor mínimo (mínima suma de los rótulos de los arcos). Minimal Spanning Tree, MST.

El algoritmo de kruskal es:
\begin{lstlisting}
Ordenar $E$ en orden de costo creciente
$S \leftarrow \emptyset$
for $v \in V$ do
	Agregar {$v$} a $S$
end
$T \leftarrow \emptyset$
for $uv\in E$ do	/* En orden! */
	if $u$ y $v$ no pertenecen al mismo conjunto en $S$ then
		Agregar $uv$ a $T$
		Unir los conjuntos de $u$ y $v$ en $S$
	end
end
return ($V, T$)
\end{lstlisting}

Central: Manipulación de clases de equivalencia (reflexiva, transitiva, simétrica), conjunto $S$.
\begin{center}
	\comentarioc{Grafo de pierola}
\end{center}
\comentarioc{En el fondo el grafo resultante aparece en una clase de equivalencia o algo así (cambiar redacción)}
\begin{center}
	\comentarioc{otro dibujo con el grafo minimo}
\end{center}
Tres operaciones centrales:
\begin{itemize}
	\item Inicializar
	\item Find
	\item Unión
\end{itemize}
\emph{Idea:} 
\begin{itemize}
	\item Representar la clase mediante un elemento representante.
	\item Un arreglo \texttt{parent}[$v$]: padre de $v$. \comentarioc{en lugar de crear nodos con punteros al hijo, estos nodos tienen puntero al padre}
\end{itemize}
O sea:
\begin{center}
	\comentarioc{Figura 1 apunte HvB: Clase 23}
\end{center}
\emph{Find:} Seguir punteros  a padres hasta bucle $\rightarrow$ representante.

\emph{Unión:} Hallar representantes (si no vienen dados), colgar el árbol menor del mayor (no aumente la altura).

Agregar \texttt{rank}[$v$]: Altura del árbol con raíz en $v$.

Algoritmos:
\begin{lstlisting}
Make Set($V$)
	for $v \in V$ do
		parent[$v$] $\leftarrow v$
		rank[$v$] $\leftarrow 0$
	end
\end{lstlisting}

\begin{lstlisting}
find ($u$)
	while $u \ne $parent[$u$] do
		$u \leftarrow $ parent[$u$]
	end
	return $u$
	
union ($u, v$) 	/* representantes */
	if rank[$u$] $<$ rank[$v$] then
		parent [$u$] $\leftarrow v$
	else
		parent[$v$] $\leftarrow u$
		if rank[$u$] = rank[$v$] then
			rank[$u$] $\leftarrow $ rank [$u$] + 1
		end
	end
\end{lstlisting}

¿Nº mínimo de nodos si rank $=r$ de la raíz? \comentarioc{colocar los dibujos asociados para cada valor de $r$}
\begin{itemize}
	\item $r = 0$
	\item $r = 1$
	\item $r = 2$
	\item $r = 3$
\end{itemize}
Si el rank $= r$, entonces se observa que el algoritmo demora $2^r$ (y puede lograrse, $\leftarrow$ son \emph{árboles binomiales}). Si $B_0$ es un nodo, entonces $B_r$ es unir dos $B_{r-1}$. Por lo tanto, camino a la raíz de un árbol de rank $r$ es a lo más $r$, si tiene $n$ nodos es a lo más $\log_2 n$.

Algoritmo de Kruskal: A lo más $2\nabsoluto{E}$ \texttt{find} $\rightsquigarrow \nabsoluto{V} - 1$.
\begin{equation}
O\nparentesis{\nabsoluto{V} + 2\nabsoluto{E}\log _2 \nabsoluto{V}}= O\nparentesis{\nabsoluto{E}\log\nabsoluto{V}}
\end{equation}

\subsection{Path Compression}
\begin{center}
	\comentarioc{Dibujo: después de hacer un find hacemos un path compression}
\end{center}
Algoritmo path compression:
\begin{lstlisting}
find ($v$)
	$u \leftarrow v$;
	while $u \ne $parent[$u$] do
		$u \leftarrow$ parent[$u$]
	end
	while $v \ne$ parent[$v$] do
		$x \leftarrow$ parent[$v$]
		parent[$v$] $\leftarrow u$
		$v \leftarrow x$
	end
\end{lstlisting}
Un programa un poco más simple:
\begin{lstlisting}
while $u \ne parent[$u$] do$
	parent[$u$] $\leftarrow$ parent[parent[$u$]]
	$u \leftarrow$ parent[$u$]
end
\end{lstlisting}





\end{document}

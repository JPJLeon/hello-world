\documentclass[english, spanish, fleqn, 10pt]{article}
\usepackage[top = 2.5cm, bottom = 2cm, left = 2.5cm, right = 2.5cm]{geometry}
\usepackage[es-noindentfirst]{babel}
\usepackage[utf8]{inputenc}
\usepackage{amsthm, amsmath, amssymb, amsfonts, environ}
\usepackage{caption}
\usepackage{subcaption} % para las subfigures
\usepackage{mathrsfs}
\usepackage[colorlinks, urlcolor=blue,
pdfauthor={Aldo Berrios Valenzuela}]{hyperref}
\usepackage{fourier}
\usepackage{colortbl}
\usepackage{color}
\usepackage{graphicx}
\usepackage{enumitem}
%\renewcommand{\familydefault}{\sfdefault}
%\setlength{\parindent}{0pt}					%Espacio de sangria
\setlength{\parskip}{0.5mm}						%Interlinado entre párrafos
\usepackage{fancyhdr, blindtext}		%Paquete de encabezado
\usepackage{listings}
\usepackage[framemethod=tikz]{mdframed}

\definecolor{webred}{rgb}{0.75,0,0}
\definecolor{mblue}{rgb}{0.0705,0.345,0.7529}
\definecolor{newmblue}{HTML}{004D99}
\usepackage{longtable, colortbl, array}

\definecolor{superGreenNew}{HTML}{169F01}
\hypersetup{
	colorlinks   = true,
	citecolor    = superGreenNew
}


%==============================
%Modificador de Titulos de secciones, subsecciones, etc
\usepackage[explicit, noindentafter]{titlesec}

\titlespacing\subsubsection{0pt}{8pt plus 4pt minus 2pt}{4pt plus 2pt minus 2pt}


%==================================
%Diseno para insertar codigos de programacion
\definecolor{newGray}{HTML}{F8F8F8}
\definecolor{anotherGray}{HTML}{DDDFFF}
\lstdefinestyle{customc}{
	belowcaptionskip=1\baselineskip,
	breaklines=true,
	backgroundcolor=\color{newGray},
	frame=single,
	rulecolor=\color{anotherGray},
	xleftmargin=\parindent,
	language=bash,
	showstringspaces=false,
	basicstyle=\small\ttfamily,
	keywordstyle=\bfseries\color{red!40!black},
	commentstyle=\itshape\color{purple!40!black},
	identifierstyle=\color{black},
	stringstyle=\color{mblue},
	tabsize=2,
	literate={\ \ }{{\ }}1	
}
\lstset{style=customc}

\renewcommand{\lstlistingname}{Listado}

%==================================
%Macros útiles
\newcommand{\comillas}[1]{``#1''}
\newcommand{\comentarioc}[1]{\texttt{\textcolor{webred}{/* #1 */}}}

\numberwithin{equation}{section}

%=================================
%comandos matemáticos personalizados de rápido acceso
\newcommand{\nparentesis}[1]{\left( #1 \right)}
\newcommand{\llaves}[1]{\left \{ #1 \right \}}
\newcommand{\nabsoluto}[1]{\left| #1 \right|}
\newcommand{\ncorchetes}[1]{\left[ #1 \right]}
%=================================

%cambia el espaciado entre el parrafo y antes del itemize
\setlist[itemize]{itemsep=1mm, topsep=1.5mm}
\setlist[enumerate]{itemsep=1mm, topsep=1.5mm}
%===========================

\theoremstyle{definition}
\newtheorem{teorema}{Teorema}[section]
\newtheorem{definition}{Definición}[section]
\newtheorem{beforeExample}{Ejemplo}[section]
\newtheorem{preposicion}{Preposición}[section]
\newtheorem{corolario}{Corolario}[teorema]

\captionsetup{justification=centering, margin=2.3cm}

\newenvironment{ejemplo}[1][]{\begin{beforeExample}[#1]\renewcommand{\qedsymbol}{$\blacksquare$}}{\qed\end{beforeExample}}

\captionsetup{justification=centering, margin=2.3cm}

% Para poder hacer "quote" con estilo github

\newenvironment{newquote}{\begin{mdframed}[topline=false,
		rightline=false, bottomline=false, linewidth=.3em,backgroundcolor=black!5, linecolor=black!60]\color{black!70}}{\end{mdframed}}
%================================

\renewcommand*{\arraystretch}{1.2}

\usepackage{fancyhdr}

\pagestyle{fancy}
\lhead{INF221\quad Algoritmos y Complejidad}
\rhead{\textit{Clase \#28\quad Una pizca de probabilidad}}


\author{Aldo Berrios Valenzuela}

\title{INF221 -- Algoritmos y Complejidad\\[.4\baselineskip]
Clase \#28\\
Una pizca de probabilidad}
\date{Martes 22 de noviembre de 2016}

\newcommand{\esperanza}[1]{\mathbb{E}\ncorchetes{#1}}

\begin{document}
\maketitle
\section{Definiciones Básicas}
\begin{definition}
	$f$ es \emph{convexa} si para todo $0\leq \alpha \leq 1$,
	\begin{equation}
	\alpha f\nparentesis{x} + \nparentesis{1-\alpha} f\nparentesis{y} \leq f\nparentesis{\alpha x + \nparentesis{1-\alpha} y}
	\end{equation}
\end{definition}
	Monito:
	\begin{center}
		\comentarioc{Dibujo}
	\end{center}
	Por inducción, si
	\begin{equation*}
	\sum_i \alpha_i = 1
	\end{equation*}
	entonces \comentarioc{$\alpha_i$ corresponde a la frecuencia de los $x_i$}
	\begin{equation*}
	\sum_i \alpha_i f\nparentesis{x_i} \geq f\nparentesis{\sum_i \alpha_i x_i}
	\end{equation*}
	Por lo tanto:
	\begin{equation}
	\esperanza{f\nparentesis{X}} \geq f\nparentesis{\esperanza{X}}
	\end{equation}
	Cota de unión: Dos conjuntos cualquiera:
	\begin{equation*}
	\nabsoluto{A\cup B} \leq \nabsoluto{A} + \nabsoluto{B}
	\end{equation*}
	O sea:
	\newcommand{\pr}[1]{\mathrm{Pr}\ncorchetes{#1}}
	\begin{equation}
	\pr{A\cup B} \leq \pr{A} + \pr{B}
	\end{equation}

\begin{teorema}
	El valor esperado es lineal:
	\begin{equation*}
	\esperanza{\alpha X + \beta Y} = \alpha \esperanza{X} + \beta\, \esperanza{Y}
	\end{equation*}
	Donde:
	\begin{align*}
	\esperanza{f\nparentesis{X}} &= \sum_x f\nparentesis{x} \pr{X=x}\\
	\mathrm{Var}\ncorchetes{X} &= \esperanza{\nparentesis{X-\esperanza{X}}^2}
	\end{align*}
	Ademas:
	\begin{itemize}
		\item $\esperanza{X}$ generalmente se anota $\mu$
		\item $\mathrm{Var}\ncorchetes{X}$ generalmente se anota $\sigma^2$
	\end{itemize}
\end{teorema}
\begin{definition}
	$X$, $Y$  son \emph{independientes} si $\pr{X = x\wedge Y = y} = \pr{X = x}\cdot \pr{Y = y}$
\end{definition}
Una colección de variables es \emph{mutuamente independiente} si para todo subconjunto $S\subseteq N$:
\begin{equation*}
\forall S' \subseteq N, \pr{X_{i_1}=x_{i_1}\wedge \ldots \wedge X_{i_s} = _{i_s}}= \pr{X_{i_1} = x_{i_1}}\cdot \ldots \cdot \pr{X_{i_s} = x_{i_s}}
\end{equation*}

\begin{teorema}
	Si $X_1$, $X_2$ son independientes:
	\begin{align*}
	\esperanza{X_1 X_2} &= \esperanza{X_1} \cdot \esperanza{X_2}\\
	\esperanza{f\nparentesis{X_1}f\nparentesis{X_2}} &= \esperanza{f\nparentesis{X_1}}\cdot \esperanza{f\nparentesis{X_2}}
	\end{align*}
\end{teorema}

\subsection{Cota de Markov}
Sea $X$ una variable aleatoria discreta, no negativa, y sea $c >0$ una constante. $\mu = \esperanza{X}$.

Tenemos:
\begin{align*}
\mu &= \esperanza{X}\\
&= \sum_x x\pr{X = x}\\
&= \sum_{0 \leq x < c} x \pr{X=x} + \sum_{x \geq c} x\pr{X = x}\\
&\geq \sum_{x \geq c} x \pr{X = x}\\
&\geq \sum_{x \geq c} c \pr{X=x}\\
&= c \sum_{x \geq c} \pr{X=x}\\
&= c\pr{X\geq c}\\
\therefore \pr{X \geq c} &\leq \dfrac{\mu}{c}
\end{align*}

\subsection{Desigualdad de Chebyshev}
Sea $X$ una variable discreta general (puede ser cualquier cosa). Interesa saber cuánto se desvía la media.
\begin{align*}
\pr{\nabsoluto{X-\mu} \geq a} &= \pr{\nparentesis{X-u}^2 \geq a^2}\\
&\leq \dfrac{\esperanza{\nparentesis{X-\mu}^2}}{\alpha^2}
\end{align*}
Por lo tanto, con $a = c\sigma$:

\begin{teorema}[Chebyshev]
	\begin{equation}
	\pr{\nabsoluto{X-\mu}\geq \sigma} \leq \dfrac{1}{c^2}
	\end{equation}
\end{teorema}

\subsection{Cotas de Chernoff}
\begin{teorema}
	Sean $X_1, \ldots, X_n$ variables independientes, $0\leq X_i \leq 1$, y sea
	\begin{equation*}
	X=\sum_i X_i
	\end{equation*}
	y $\mu = \esperanza{X}$. Entonces para todo $c \geq 1$:
	\begin{equation*}
	\pr{X \geq c} \leq e^{-\beta \nparentesis{c} \mu}
	\end{equation*}
	donde $\beta\nparentesis{c} = c \ln\nparentesis{c} - c + 1$.
\end{teorema}

\begin{proof}
	Para Chebyshev, cuadrados. Ahora exponenciales \ldots
	\begin{align*}
	\pr{X \geq c\mu} &= \pr{c^X \geq c^{c\mu}}\\
	&\geq \dfrac{\esperanza{c^X}}{c^{c\mu}} \qquad\qquad \text{(Markov)}\\
	&\geq \dfrac{e^{\nparentesis{c-1}\mu}}{c^{c\mu}}\qquad\qquad\text{(lema 1, luego)}\\
	&=e^{-\beta \nparentesis{c}\mu}
	\end{align*}
\end{proof}
\comentarioc{colocar un entorno para el Lema}

\noindent \textbf{Lema 1.} 
\begin{equation*}
\esperanza{c^{X}} \leq e^{\nparentesis{c-1}\mu}
\end{equation*}
\begin{proof}
	\begin{align*}
	\esperanza{c^X} &= \esperanza{c^{X_1+\cdots + X_n}}\\
	&= \esperanza{c^{X_1}}\cdot \ldots \cdot \esperanza{c^{X_n}} \\
	&\leq e^{\nparentesis{c-1}\esperanza{X_1}}\cdot \ldots \cdot e^{\nparentesis{c-1}\esperanza{X_n}} \qquad\qquad\text{(lema 2)}\\
	&=e^{\nparentesis{c-1}\nparentesis{\esperanza{X_1}+\cdots + \esperanza{X_n}}}\\
	&= e^{\nparentesis{c-1}\esperanza{X}}
	\end{align*}
\end{proof}

\noindent \textbf{Lema 2.}
\begin{equation}
E\ncorchetes{c^{X_i}} \leq e^{\nparentesis{c-1}\esperanza{X_i}}
\end{equation}

\begin{proof}
	\begin{align*}
	\esperanza{c^{X_i}} &= \sum_x c^x \pr{X=x}\\
	&\leq \sum_x \nparentesis{1 + \nparentesis{c-1}x}\pr{X = x}\qquad\qquad\text{(convexidad de $c^x$)}\\
	&= \sum_x \pr{X=x} + \nparentesis{c-1}\sum_x x \pr{X=x}\\
	&= 1 + \nparentesis{c-1}\esperanza{X_i}\\
	&\leq e^{\nparentesis{c-1}\esperanza{X_i}}\qquad\qquad\qquad\text{(En este paso aplicamos $1+z\leq e^z$)}
	\end{align*}
\end{proof}
\comentarioc{en la parte de convexidad se tiene que $x=0$ e $y=1$ de la desigualdad (1.1)}
\begin{center}
	\comentarioc{Dibujo}
\end{center}









\end{document}

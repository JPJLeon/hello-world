\documentclass[english, spanish, fleqn, 10pt]{article}
\usepackage[top = 2.5cm, bottom = 2cm, left = 2.5cm, right = 2.5cm]{geometry}
\usepackage[es-noindentfirst]{babel}
\usepackage[utf8]{inputenc}
\usepackage{amsthm, amsmath, amssymb, amsfonts, environ}
\usepackage{caption}
\usepackage{subcaption} % para las subfigures
\usepackage{mathrsfs}
\usepackage[colorlinks, urlcolor=blue,
pdfauthor={Aldo Berrios Valenzuela}]{hyperref}
\usepackage{fourier}
\usepackage{colortbl}
\usepackage{color}
\usepackage{graphicx}
\usepackage{enumitem}
\usepackage{cancel}
%\renewcommand{\familydefault}{\sfdefault}
%\setlength{\parindent}{0pt}					%Espacio de sangria
\setlength{\parskip}{0.5mm}						%Interlinado entre párrafos
\usepackage{fancyhdr, blindtext}		%Paquete de encabezado
\usepackage{listings}
\usepackage[framemethod=tikz]{mdframed}

\definecolor{webred}{rgb}{0.75,0,0}
\definecolor{mblue}{rgb}{0.0705,0.345,0.7529}
\definecolor{newmblue}{HTML}{004D99}
\usepackage{longtable, colortbl, array}

\definecolor{superGreenNew}{HTML}{169F01}
\hypersetup{
	colorlinks   = true,
	citecolor    = superGreenNew
}


%==============================
%Modificador de Titulos de secciones, subsecciones, etc
\usepackage[explicit, noindentafter]{titlesec}

\titlespacing\subsubsection{0pt}{8pt plus 4pt minus 2pt}{4pt plus 2pt minus 2pt}


%==================================
%Diseno para insertar codigos de programacion
\definecolor{newGray}{HTML}{F8F8F8}
\definecolor{anotherGray}{HTML}{DDDFFF}
\lstdefinestyle{customc}{
	belowcaptionskip=1\baselineskip,
	breaklines=true,
	backgroundcolor=\color{newGray},
	frame=single,
	rulecolor=\color{anotherGray},
	xleftmargin=\parindent,
	language=bash,
	showstringspaces=false,
	basicstyle=\small\ttfamily,
	keywordstyle=\bfseries\color{green!40!black},
	commentstyle=\itshape\color{purple!40!black},
	identifierstyle=\color{black},
	stringstyle=\color{mblue},
	tabsize=2,
	literate={\ \ }{{\ }}1	
}
\lstset{style=customc}

%==================================
%Macros útiles
\newcommand{\comillas}[1]{``#1''}
\newcommand{\comentarioc}[1]{\texttt{\textcolor{webred}{/* #1 */}}}

\numberwithin{equation}{section}

%=================================
%comandos matemáticos personalizados de rápido acceso
\newcommand{\nparentesis}[1]{\left( #1 \right)}
\newcommand{\llaves}[1]{\left \{ #1 \right \}}
\newcommand{\nabsoluto}[1]{\left| #1 \right|}
\newcommand{\ncorchetes}[1]{\left[ #1 \right]}
%=================================

%cambia el espaciado entre el parrafo y antes del itemize
\setlist[itemize]{itemsep=1mm, topsep=1.5mm}
\setlist[enumerate]{itemsep=1mm, topsep=1.5mm}
%===========================

\theoremstyle{definition}
\newtheorem{teorema}{Teorema}[section]
\newtheorem{definition}{Definición}[section]
\newtheorem{beforeExample}{Ejemplo}[section]
\newtheorem{preposicion}{Preposición}[section]
\newtheorem{corolario}{Corolario}[teorema]

\captionsetup{justification=centering, margin=2.3cm}

\newenvironment{ejemplo}[1][]{\begin{beforeExample}[#1]\renewcommand{\qedsymbol}{$\blacksquare$}}{\qed\end{beforeExample}}

\captionsetup{justification=centering, margin=2.3cm}

% Para poder hacer "quote" con estilo github

\newenvironment{newquote}{\begin{mdframed}[topline=false,
		rightline=false, bottomline=false, linewidth=.3em,backgroundcolor=black!5, linecolor=black!60]\color{black!70}}{\end{mdframed}}
%================================

\renewcommand*{\arraystretch}{1.2}

\usepackage{fancyhdr}

\pagestyle{fancy}
\lhead{INF221\quad Algoritmos y Complejidad}
\rhead{\textit{Clase \#13\quad Recursividad}}

\author{Aldo Berrios Valenzuela\and Hernán Herreros Niño\and Vicente Lizana Estivill}
\title{INF221 -- Algoritmos y Complejidad\\[.4\baselineskip]Clase \#13\\Recursividad}
\date{Martes 13 de septiembre de 2016}

\addto{\captionsspanish}{\renewcommand{\abstractname}{Agradecimientos}}

\usepackage{tikz,ifthen}
\usepackage{pgfplots}
\usepackage{tikz,ifthen}
\usetikzlibrary{automata,positioning, babel}
\usetikzlibrary{calc}

\newcommand\numberthis{\addtocounter{equation}{1}\tag{\theequation}}


\begin{document}
\maketitle

\begin{abstract}
	Agradezco a las personas que me mandaron sus apuntes para poder elaborar este debido a mi ausencia.
\end{abstract}

\section{Recursión}
Nuestra herramienta principal es \emph{reducir} problemas a problemas más \comillas{simples}. Por ejemplo: Expresión regular $\rightarrow$ programa eficiente para reconocer el problema descrito usamos:
\begin{itemize}
	\item Expresión Regular a NFA (por ejemplo Thompson)
	
	\item NFA a DFA (algoritmo de subconjuntos)
	
	\item DFA a DFA mínimo (varias opciones)
	
	\item Interpretar el DFA y traducirlo a código.
\end{itemize}

Al resolver un problema, lo dividimos en subproblemas y combinamos resultados. Notar que al hacer esto (por ejemplo, invocar \texttt{printf(3)} en C) confiamos en que la solución al subproblema hace su trabajo correctamente. \emph{Confiamos} en terceros. De la misma manera, al invocar una función que nosotros escribimos, \emph{confiamos} en que hace su trabajo correctamente. Usar recursión es lo mismo\ldots solo que la función se invoca a sí misma (directa o indirectamente). Recursión es inducción en programa.

\subsection{Torres de Hanoi}
\paragraph{Descripción del problema:} Hay 64 placas redondas ubicadas de mayor a menor (Figura \ref{13::TorreHanoi1}). Solo se puede mover una placa a la vez y nunca placa mayor sobre una placa menor.
\begin{figure}[!h]
	\centering
	\newcommand{\hanoiDisk}[4]{\draw [fill=#4] (#1-#3/2, #2) -- ++ (#3, 0) -- ++ (0, 0.15) -- ++ (-#3, 0) -- cycle} %(x, y, width, color)
	\begin{tikzpicture}
		% Dibujo de la primera plataforma
		\hanoiDisk{0}{0}{2.5}{green!60!black!80};
		\hanoiDisk{2.5}{0}{2.5}{blue!60!black!60};
		\hanoiDisk{5}{0}{2.5}{green!60!black!80};
		\draw (0, 0.15) -- ++ (0, 2);
		\draw (2.5, 0.15) -- ++ (0, 2);
		\draw (5, 0.15) -- ++ (0, 2);
		
		% Dibujo de los discos de la primera plataforma
		\hanoiDisk{0}{0.15}{2}{gray!40};
		\hanoiDisk{0}{0.15+0.15}{1.8}{gray!20};
		\hanoiDisk{0}{0.15+2*0.15}{1.6}{gray!40};
		\hanoiDisk{0}{0.15+3*0.15}{1.4}{gray!20};
		\hanoiDisk{0}{0.15+4*0.15}{1.2}{gray!40};
		\hanoiDisk{0}{0.15+5*0.15}{1.0}{gray!20};
		\hanoiDisk{0}{0.15+6*0.15}{0.8}{gray!40};
		\hanoiDisk{0}{0.15+7*0.15}{0.6}{gray!20};
		\hanoiDisk{0}{0.15+8*0.15}{0.4}{gray!40};
		\node at (0, -0.25) (A) {$A$};
		\node at (2.5, -0.25) (B) {$B$};
		\node at (5, -0.25) (C) {$C$};
		
		% Dibujo de la segunda plataforma
		\hanoiDisk{0+8.5}{0}{2.5}{green!60!black!80};
		\hanoiDisk{2.5+8.5}{0}{2.5}{blue!60!black!60};
		\hanoiDisk{5+8.5}{0}{2.5}{green!60!black!80};
		\draw (0+8.5, 0.15) -- ++ (0, 2);
		\draw (2.5+8.5, 0.15) -- ++ (0, 2);
		\draw (5+8.5, 0.15) -- ++ (0, 2);
		
		% Dibujo de los discos de la segunda plataforma
		\hanoiDisk{5+8.5}{0.15}{2}{gray!40};
		\hanoiDisk{5+8.5}{0.15+0.15}{1.8}{gray!20};
		\hanoiDisk{5+8.5}{0.15+2*0.15}{1.6}{gray!40};
		\hanoiDisk{5+8.5}{0.15+3*0.15}{1.4}{gray!20};
		\hanoiDisk{5+8.5}{0.15+4*0.15}{1.2}{gray!40};
		\hanoiDisk{5+8.5}{0.15+5*0.15}{1.0}{gray!20};
		\hanoiDisk{5+8.5}{0.15+6*0.15}{0.8}{gray!40};
		\hanoiDisk{5+8.5}{0.15+7*0.15}{0.6}{gray!20};
		\hanoiDisk{5+8.5}{0.15+8*0.15}{0.4}{gray!40};
		\node at (0+8.5, -0.25) (A2) {$A$};
		\node at (2.5+8.5, -0.25) (B2) {$B$};
		\node at (5+8.5, -0.25) (C2) {$C$};
		
		\draw [->, thick, red!60!black!80] (6.35, 1.25) -- ++ (0.75, 0);
	\end{tikzpicture}
	\caption{Queremos mover las placas ubicadas en la plataforma $A$ y dejarlas en la plataforma $C$. }
	\label{13::TorreHanoi1}
\end{figure}

\subsubsection{Solución (recursiva)}
Una solución (recursiva) es como se muestra en la Figura \ref{13::TorreHanoi2}.
\begin{figure}[!h]
	\centering
	\newcommand{\hanoiDisk}[4]{\draw [fill=#4] (#1-#3/2, #2) -- ++ (#3, 0) -- ++ (0, 0.15) -- ++ (-#3, 0) -- cycle} %(x, y, width, color)
	\begin{tikzpicture}
	% Dibujo de la primera plataforma
	\hanoiDisk{0}{0}{2.5}{green!60!black!80};
	\hanoiDisk{2.5}{0}{2.5}{blue!60!black!60};
	\hanoiDisk{5}{0}{2.5}{green!60!black!80};
	\draw (0, 0.15) -- ++ (0, 2);
	\draw (2.5, 0.15) -- ++ (0, 2);
	\draw (5, 0.15) -- ++ (0, 2);
	
	% Dibujo de los discos de la primera plataforma
	\hanoiDisk{0}{0.15}{2}{gray!40};
	\hanoiDisk{2.5}{0.15+0*0.15}{1.8}{gray!20};
	\hanoiDisk{2.5}{0.15+1*0.15}{1.6}{gray!40};
	\hanoiDisk{2.5}{0.15+2*0.15}{1.4}{gray!20};
	\hanoiDisk{2.5}{0.15+3*0.15}{1.2}{gray!40};
	\hanoiDisk{2.5}{0.15+4*0.15}{1.0}{gray!20};
	\hanoiDisk{2.5}{0.15+5*0.15}{0.8}{gray!40};
	\hanoiDisk{2.5}{0.15+6*0.15}{0.6}{gray!20};
	\hanoiDisk{2.5}{0.15+7*0.15}{0.4}{gray!40};
	\node at (0, -0.25) (A) {$A$};
	\node at (2.5, -0.25) (B) {$B$};
	\node at (5, -0.25) (C) {$C$};
	
	% Dibujo de la segunda plataforma
	\hanoiDisk{0+8.5}{0}{2.5}{green!60!black!80};
	\hanoiDisk{2.5+8.5}{0}{2.5}{blue!60!black!60};
	\hanoiDisk{5+8.5}{0}{2.5}{green!60!black!80};
	\draw (0+8.5, 0.15) -- ++ (0, 2);
	\draw (2.5+8.5, 0.15) -- ++ (0, 2);
	\draw (5+8.5, 0.15) -- ++ (0, 2);
	
	% Dibujo de los discos de la segunda plataforma
	\hanoiDisk{5+8.5}{0.15}{2}{gray!40};
	\hanoiDisk{2.5+8.5}{0.15+0*0.15}{1.8}{gray!20};
	\hanoiDisk{2.5+8.5}{0.15+1*0.15}{1.6}{gray!40};
	\hanoiDisk{2.5+8.5}{0.15+2*0.15}{1.4}{gray!20};
	\hanoiDisk{2.5+8.5}{0.15+3*0.15}{1.2}{gray!40};
	\hanoiDisk{2.5+8.5}{0.15+4*0.15}{1.0}{gray!20};
	\hanoiDisk{2.5+8.5}{0.15+5*0.15}{0.8}{gray!40};
	\hanoiDisk{2.5+8.5}{0.15+6*0.15}{0.6}{gray!20};
	\hanoiDisk{2.5+8.5}{0.15+7*0.15}{0.4}{gray!40};
	\node at (0+8.5, -0.25) (A2) {$A$};
	\node at (2.5+8.5, -0.25) (B2) {$B$};
	\node at (5+8.5, -0.25) (C2) {$C$};
	
	\draw [->, thick, red!60!black!80] (6.35, 1.25) -- ++ (0.75, 0);
	\end{tikzpicture}
	\caption{Luego de mover todos los discos entre las tres plataformas y dejarlos ordenados como se muestran en $B$, sólo nos queda mover el último disco (más grande) de $A$ a $C$ directamente.}
	\label{13::TorreHanoi2}
\end{figure}
Esto es solución porque traduce el problema de mover $n$ piezas a mover $n-1$ recursivamente, luego mover $1$ (trivial, Figura \ref{13::TorreHanoi2}), luego mover $n-1$ recursivamente.

\paragraph{Problema:} Mover $n$ piezas de $A$ a $C$.

\begin{proof}[Base:]
	Si $n=0$, no hay que hacer nada.
\end{proof}

\begin{proof}[Inducción:]
	Supongamos que sabemos mover $k$ piezas de $i$ a $j$ (con $i\ne j$). Para mover $k+1$ piezas de $A$ a $B$ (con $C$ de \comillas{apoyo}):
	\begin{itemize}
		\item Movemos las $k$ piezas superiores de $A$ a $C$.
		
		\item Movemos la pieza inferior de $A$ a $B$.
		
		\item Movemos las $k$ piezas de $C$ a $B$.
	\end{itemize}
\end{proof}

Pseudocódigo:
\begin{lstlisting}
hanoi (n, src, dst, tmp):
	if n > 0:
		hanoi (n-1, src, tmp, dst)
		move src -> dst
		hanoi (n-1, tmp, dst, src)
\end{lstlisting}

Aún falta convencerse de que \emph{nunca} movemos uno grande sobre uno menor ($\rightarrow$ luego de \texttt{hanoi (n, src, dst, tmp)}, \texttt{tmp} queda libre).

¿Cuántas movidas se requieren?. Sea $T\nparentesis{n}$ el número de movidas para transferir $n$ platos.
\begin{align*}
T\nparentesis{0}&=0\\
T\nparentesis{n+1}&=2T\nparentesis{n}+1,\qquad n\geq 0
\end{align*}
Sea:
\begin{equation}\label{13::Generatriz:hz}
h\nparentesis{z}=\sum_{n\geq 0}T\nparentesis{n}z^n
\end{equation}
Por propiedades:
\begin{equation*}
\dfrac{h\nparentesis{z}-T\nparentesis{0}}{z}=2h\nparentesis{z}+\dfrac{1}{1-z}
\end{equation*}
Entonces, despejando $h\nparentesis{z}$ de lo anterior se obtiene:
\begin{align*}
h\nparentesis{z}&=\dfrac{z}{\nparentesis{1-z}\nparentesis{1-2z}}\numberthis\label{13::hzPlanteo}\\
h\nparentesis{z}&=\dfrac{A}{1-z}+\dfrac{B}{1-2z}\\
h\nparentesis{z}\nparentesis{1-z}&=A + \dfrac{\nparentesis{1-z}B}{1-2z}
\end{align*}
En seguida, aplicamos límite cuando $z\rightarrow 1$:
\begin{align*}
h\nparentesis{z}\nparentesis{1-z}&=A + \dfrac{\nparentesis{1-z}B}{1-2z}\qquad\Bigg / \lim_{z\rightarrow 1}\nparentesis{\,}\\
\lim_{z\rightarrow 1}h\nparentesis{z}\nparentesis{1-z}&=\lim_{z\rightarrow 1}\nparentesis{A+\cancelto{0}{\dfrac{\nparentesis{1-z}B}{1-2z}}\hspace{4mm}}\\
\lim_{z\rightarrow 1}h\nparentesis{z}\nparentesis{1-z}&=A\numberthis\label{13::Limite:hz}
\end{align*}
Luego, reemplazamos $h\nparentesis{z}$ de \eqref{13::Limite:hz} por lo que está en \eqref{13::hzPlanteo}:
\begin{align*}
A&=\lim_{z\rightarrow 1}\nparentesis{\dfrac{z}{\nparentesis{1-z}\nparentesis{1-2z}}\nparentesis{1-z}}\\
A&=\lim_{z\rightarrow 1}\dfrac{z}{1-2z}\\
\therefore A&=-1\numberthis
\end{align*}
Análogamente:
\begin{equation}
B=1
\end{equation}
Entonces:
\begin{align*}
h\nparentesis{z}&=\dfrac{1}{1-2z}-\dfrac{1}{1-z}\numberthis\label{13::hz:final}
\end{align*}
Usando la fórmula de series geométricas, es sabido que:
\begin{equation}\label{13::Uso:PropGeometrica}
\dfrac{1}{1-2z}=\sum_{n\geq 0}2^nz^n\qquad\wedge\qquad \dfrac{1}{1-z}=\sum_{n\geq 0}z^n
\end{equation}
Luego, si reemplazamos $h\nparentesis{z}$ por \eqref{13::Generatriz:hz} en \eqref{13::hz:final} y las fracciones de \eqref{13::hz:final} por las obtenidas en \eqref{13::Uso:PropGeometrica} se obtiene:
\begin{align*}
\sum_{n\geq 0}T\nparentesis{n}z^n&=\sum_{n\geq 0}2^nz^n-\sum_{n\geq 0}z^n\\
\Rightarrow T\nparentesis{n}&=2^n-1
\end{align*}


\subsection{Mergesort}
Queremos ordenar $A\ncorchetes{1, \ldots, n}$. Nuestro pseudocódigo para mergesort es:
\begin{align*}
&\mathtt{mergeSort}\nparentesis{A\ncorchetes{1, \ldots, n}}:\\
&\hspace{2em}\mathtt{if }\hspace{1ex} n>1:\\
&\hspace{4em}\mathtt{mergeSort}\nparentesis{A\ncorchetes{1, \ldots, \left\lfloor \frac{n}{2}\right\rfloor}}\\
&\hspace{4em}\mathtt{mergeSort}\nparentesis{A\ncorchetes{\left\lfloor \frac{n}{2}\right\rfloor+1, \ldots, n}}\\
&\hspace{4em}\mathtt{merge}\nparentesis{ A\ncorchetes{1, \ldots, \left\lfloor \frac{n}{2}\right\rfloor}   , A\ncorchetes{\left\lfloor \frac{n}{2}\right\rfloor+1, \ldots, n}}
\end{align*}

De la misma forma que lo hicimos con las torres de Hanoi: ¿Cuántas iteraciones se requieren?. Sea $M\nparentesis{n}$ el número de iteraciones  para completar el merge sort. Es claro que:
\begin{align*}
M\nparentesis{0}&=M\nparentesis{1}\\
M\nparentesis{n}&=M\nparentesis{\left \lfloor\frac{n}{2}\right\rfloor}+M\nparentesis{\left \lceil \frac{n}{2}\right \rceil}+n\numberthis\label{13::M1}
\end{align*}
Supongamos $n=2^k$. Reemplazando sobre \eqref{13::M1} vamos obteniendo:
\begin{align*}
M\nparentesis{2^k}&=M\nparentesis{\left\lfloor \dfrac{2^k}{2}\right\rfloor}+M\nparentesis{\left\lceil \dfrac{2^k}{2}\right\rceil}+2^k\\
&=M\nparentesis{\left\lfloor 2^{k-1}\right \rfloor}+M\nparentesis{\left\lceil 2^{k-1}\right \rceil}+2^k\numberthis\label{13::MPiso}
\end{align*}
Es claro que $2^{k-1}\in\mathbb{N}$, por lo tanto:
\begin{equation}\label{13::PrimeraIguladadM}
M\nparentesis{\left\lfloor 2^{k-1}\right \rfloor}=M\nparentesis{\left\lceil 2^{k-1}\right \rceil}=M\nparentesis{2^{k-1}}
\end{equation}
Luego, reemplazando \eqref{13::PrimeraIguladadM} en \eqref{13::MPiso} obtenemos:
\begin{equation}\label{13::Final:M}
M\nparentesis{2^k}=2M\nparentesis{2^{k-1}}+2^k
\end{equation}
Sea $m\nparentesis{k}=M\nparentesis{2^{k-1}}$ \comentarioc{DUDA: en los apuntes entregados dice $m\nparentesis{k}=M\nparentesis{2^k}$\ldots ¿realmente es así?}. Entonces, reemplazando sobre \eqref{13::Final:M} se tiene:
\begin{equation}
m\nparentesis{k+1}=2m\nparentesis{k}+2^k
\end{equation}
\comentarioc{DUDA: hay cosas que no calzan en los apuntes\ldots }


\end{document}

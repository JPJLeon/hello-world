\documentclass[english, spanish, fleqn, 10pt]{article}
\usepackage[top = 2.5cm, bottom = 2cm, left = 2.5cm, right = 2.5cm]{geometry}
\usepackage[es-noindentfirst]{babel}
\usepackage{babelbib}
\usepackage[utf8]{inputenc}
\usepackage{amsthm, amsmath, amssymb, amsfonts, environ}
\usepackage{caption}
\usepackage{subcaption} % para las subfigures
\usepackage{mathrsfs}
\usepackage[colorlinks, urlcolor=blue,
pdfauthor={Aldo Berrios Valenzuela}]{hyperref}
\usepackage{fourier}
\usepackage{colortbl}
\usepackage{color}
\usepackage{graphicx}
\usepackage{enumitem}
%\renewcommand{\familydefault}{\sfdefault}
%\setlength{\parindent}{0pt}					%Espacio de sangria
\setlength{\parskip}{0.5mm}						%Interlinado entre párrafos
\usepackage{fancyhdr, blindtext}		%Paquete de encabezado
\usepackage{listings}
\usepackage[framemethod=tikz]{mdframed}

\definecolor{webred}{rgb}{0.75,0,0}
\definecolor{mblue}{rgb}{0.0705,0.345,0.7529}
\definecolor{newmblue}{HTML}{004D99}
\usepackage{longtable, colortbl, array}

\definecolor{superGreenNew}{HTML}{169F01}
\hypersetup{
	colorlinks   = true,
	citecolor    = superGreenNew
}


%==============================
%Modificador de Titulos de secciones, subsecciones, etc
\usepackage[explicit, noindentafter]{titlesec}

\titlespacing\subsubsection{0pt}{8pt plus 4pt minus 2pt}{4pt plus 2pt minus 2pt}


%==================================
%Diseno para insertar codigos de programacion
\definecolor{newGray}{HTML}{F8F8F8}
\definecolor{anotherGray}{HTML}{DDDFFF}
\lstdefinestyle{customc}{
	belowcaptionskip=1\baselineskip,
	breaklines=true,
	backgroundcolor=\color{newGray},
	frame=single,
	rulecolor=\color{anotherGray},
	xleftmargin=\parindent,
	language=bash,
	showstringspaces=false,
	basicstyle=\small\ttfamily,
	keywordstyle=\bfseries\color{green!40!black},
	commentstyle=\itshape\color{purple!40!black},
	identifierstyle=\color{black},
	stringstyle=\color{mblue},
	tabsize=2,
	literate={\ \ }{{\ }}1	
}
\lstset{style=customc}

%==================================
%Macros útiles
\newcommand{\comillas}[1]{``#1''}
\newcommand{\comentarioc}[1]{\texttt{\textcolor{webred}{/* #1 */}}}

\numberwithin{equation}{section}

%=================================
%comandos matemáticos personalizados de rápido acceso
\newcommand{\nparentesis}[1]{\left( #1 \right)}
\newcommand{\llaves}[1]{\left \{ #1 \right \}}
\newcommand{\nabsoluto}[1]{\left| #1 \right|}
\newcommand{\ncorchetes}[1]{\left[ #1 \right]}
%=================================

%cambia el espaciado entre el parrafo y antes del itemize
\setlist[itemize]{itemsep=1mm, topsep=1.5mm}
\setlist[enumerate]{itemsep=1mm, topsep=1.5mm}
%===========================

\theoremstyle{definition}
\newtheorem{teorema}{Teorema}[section]
\newtheorem{definition}{Definición}[section]
\newtheorem{beforeExample}{Ejemplo}[section]
\newtheorem{preposicion}{Preposición}[section]
\newtheorem{corolario}{Corolario}[teorema]

\captionsetup{justification=centering, margin=2.3cm}

\newenvironment{ejemplo}[1][]{\begin{beforeExample}[#1]\renewcommand{\qedsymbol}{$\blacksquare$}}{\qed\end{beforeExample}}

\captionsetup{justification=centering, margin=2.3cm}

% Para poder hacer "quote" con estilo github

\newenvironment{newquote}{\begin{mdframed}[topline=false,
		rightline=false, bottomline=false, linewidth=.3em,backgroundcolor=black!5, linecolor=black!60]\color{black!70}}{\end{mdframed}}
%================================

\renewcommand*{\arraystretch}{1.2}

\usepackage{fancyhdr}

\pagestyle{fancy}
\lhead{INF221\quad Algoritmos y Complejidad}
\rhead{\textit{Clase \#16\quad Dividir y Conquistar}}


\author{Aldo Berrios Valenzuela \and Horst H. von Brand}
\title{INF221 -- Algoritmos y Complejidad\\[.4\baselineskip]Clase \#16\\Dividir y Conquistar}

\date{Martes 4 de octubre de 2016}

\begin{document}
\bibliographystyle{babplain-fl}
\maketitle

\section{Dividir y Conquistar}
Una de las mejores estrategias para diseñar algoritmos.

\paragraph{Idea:} Dado un problema grande, reducirlo a varios problemas menores del mismo tipo, y combinar resultados.

\begin{ejemplo}[Merge Sort]
	Para ordenar $N$ elementos:
	\begin{itemize}
		\item Dividir en \comillas{mitades} de $\left\lfloor \frac{N}{2} \right\rfloor$ y $\left\lceil \frac{N}{2}\right\rceil$
		\item Ordenarlas recursivamente.
		\item Intercalar resultados.
	\end{itemize}
\end{ejemplo}

\begin{ejemplo}[Búsqueda binaria]
	Un arreglo ordenado de $N$ elementos, y una clave a buscar. Obtener el elemento en la posición $\left\lfloor \frac{N}{2} \right\rfloor$, buscar en la mitad que tiene que contener la clave
\end{ejemplo}

\begin{ejemplo}[Multiplicación de Karatsuba]
  Para multiplicar números de $2n$ dígitos,
  dividimos ambos en mitades~%
    \cite{karatsuba62:_multiplication}:
	\begin{align*}
	A&=10^na+b\\
	B&=10^nc+d
	\end{align*}
	con $a, b, c, d < 10^n$.
	
	Además:
	\begin{equation}\label{16::Primero}
	A\cdot B = 10^{2n}ac + 10^n\nparentesis{ad+bc} + bd
	\end{equation}
	Notando que:
	\begin{equation*}
	\nparentesis{a+b}\cdot \nparentesis{c+d} = ac + ad + bc + bd - \nparentesis{ad + bc} + ac + bd
	\end{equation*}
	Podemos calcular los coeficientes de \eqref{16::Primero} con 3 (no 4) multiplicaciones:
	\begin{align*}
	  &a c \\
          &b d \\
          &(a + b) (c + d) - a c - b d
	\end{align*}
\end{ejemplo}
\begin{ejemplo}
  Otro ejemplo de esta estrategia es el algoritmo de Strassen~%
  \cite{strassen69:_matrix_multiplication}
  para multiplicar matrices.
  Consideremos primeramente el producto de
  dos matrices de \(2 \times 2\):
  \begin{equation*}
    \begin{pmatrix}
      c_{1 1} & c_{1 2} \\
      c_{2 1} & c_{2 2}
    \end{pmatrix}
    = \begin{pmatrix}
      a_{1 1} & a_{1 2} \\
      a_{2 1} & a_{2 2}
    \end{pmatrix}
    \cdot
    \begin{pmatrix}
      b_{1 1} & b_{1 2} \\
      b_{2 1} & b_{2 2}
    \end{pmatrix}
  \end{equation*}
  Sabemos que:
  \begin{equation*}
    \begin{array}{l@{\qquad}l}
      c_{1 1}
        = a_{1 1} b_{1 1} + a_{1 2} b_{2 1} &
      c_{1 2}
        = a_{1 1} b_{1 2} + a_{1 2} b_{2 2} \\
      c_{2 1}
        = a_{2 1} b_{1 1} + a_{2 2} b_{2 1} &
      c_{2 2}
        = a_{2 1} b_{1 2} + a_{2 2} b_{2 2}
    \end{array}
  \end{equation*}
  Esto corresponde a \(8\) multiplicaciones.
  Definamos los siguientes productos:
  \begin{equation*}
    \begin{array}{l@{\qquad}l}
      m_1
        = (a_{1 1} + a_{2 2}) \, (b_{1 1} + b_{2 2}) &
      m_2
        = (a_{2 1} + a_{2 2}) \, b_{1 1} \\
      m_3
        = a_{1 1} \, (b_{1 2} - b_{2 2}) &
      m_4
        = a_{2 2} \, (b_{2 1} - b_{1 1}) \\
      m_5
        = (a_{1 1} + a_{1 2}) \, b_{2 2} &
      m_6
        = (a_{2 1} - a_{1 1}) \, (b_{1 1} + b_{1 2}) \\
      m_7
        = (a_{1 2} - a_{2 2}) \, (b_{2 1} + b_{2 2})
    \end{array}
  \end{equation*}
  Entonces podemos expresar:
  \begin{align*}
    \begin{array}{l@{\qquad}l}
      c_{1 1}
        = m_1 + m_4 - m_5 + m_7 &
      c_{1 2}
        = m_3 + m_5 \\
      c_{2 1}
        = m_2 + m_4 &
      c_{2 2}
        = m_1 - m_2 + m_3 + m_6
    \end{array}
  \end{align*}
  Con estas fórmulas se usan \(7\) multiplicaciones
  para evaluar el producto de dos matrices.
  Cabe hacer notar que estas fórmulas no hacen uso de conmutatividad,
  por lo que son aplicables también
  para multiplicar matrices de \(2 \times 2\)
  cuyos elementos son a su vez matrices.
  Podemos usar esta fórmula recursivamente
  para multiplicar matrices de \(2^n \times 2^n\).
\end{ejemplo}

\subsection{Estructura común}
\label{sec:estructura-comun}

	Un problema de tamaño $nb$ se reduce a $a$ problemas de tamaño $n$, que se resuelven recursivamente y las soluciones se combinan. Si el trabajo para resolver una instancia de tamaño $n$ la llamamos $t\nparentesis{n}$, y el trabajo para reducir y combinar soluciones lo llamamos $f$:
	\begin{equation}
	t\nparentesis{nb}=at\nparentesis{n}+f\nparentesis{n},\qquad t\nparentesis{1}=t_1
	\end{equation}
        Buscamos resolver esta recurrencia.
	Suponiendo que $n$ es una potencia de $b$
	\begin{align*}
	n &= b^k \qquad t\nparentesis{b ^k} = T\nparentesis{k}\qquad\nparentesis{k = \log_b\nparentesis{n}}\\
	T\nparentesis{k + 1} &= a T\nparentesis{k} + f\nparentesis{b^ k}
	\end{align*}
	En la mayoría de los casos de interés, $f\nparentesis{n} = cn^d$
	\begin{equation*}
	T\nparentesis{k + 1} = aT\nparentesis{k} + cb^{kd}\qquad T\nparentesis{0}= t_1
	\end{equation*}
	Definimos:
	\begin{equation*}
	g\nparentesis{z} = \sum_{k \geq 0} T\nparentesis{k} z^k
	\end{equation*}
	Por propiedades:
	\begin{equation*}
	\dfrac{g\nparentesis{z} - t_1}{z} = a g\nparentesis{z} + \dfrac{c}{1 - b^d z}
	\end{equation*}
	Entonces, por propiedades:
	\begin{equation*}
	g\nparentesis{z} = \dfrac{t_1 - \nparentesis{b^d t_1 - c}z}{\nparentesis{1-b^d z}\nparentesis{1-az}}
	\end{equation*}
	Hay que distinguir los casos en que los factores del denominador son iguales o distintos.
	
	Caso $a=b^d$:
	\begin{align*}
	g\nparentesis{z} &= \dfrac{t_1 - \nparentesis{at_1-c}z}{\nparentesis{1-az}^2}\\
	&= \dfrac{c}{a} \cdot \dfrac{1}{\nparentesis{1-az}^2} + \dfrac{at_1 - c}{a}\cdot \dfrac{1}{1-az}\\
	&=\dfrac{c}{a} \sum_{k \geq 0}\binom{-2}{k}
	a^kz^k + \dfrac{at_1 -c}{a}\cdot \sum_{k \geq 0}a ^k z^k\\
	&=\dfrac{c}{a} \cdot \sum_{k \geq 0}\binom{k + 1}{1}
	a^k z^k+ \dfrac{at_1-c}{a}\cdot \sum_{k \geq 0}a^kz^k\\
	\ncorchetes{z^k} g\nparentesis{z} &= \dfrac{c}{a} \nparentesis{k+1} a^k + \dfrac{at_1-c}{a}\cdot a^k\\
	T\nparentesis{k} &= \dfrac{c}{a} k a^k + t_1 \cdot a^k\\
	& \sim \dfrac{c}{a} k a^k\\
	t\nparentesis{n} &\sim \dfrac{c}{a} a^{\log_b n} \cdot \log_b n
	\end{align*}
	Pero: 
	\begin{align*}
	a^{\log_b n} &= \nparentesis{b^{\log_b a}}^{\log_b n}\\
	&= \nparentesis{b^{\log_b n}}^{\log_b a}\\
	&= n^{\log_b a}= n^d
	\end{align*}
	\begin{align*}
	t\nparentesis{n}&\sim \dfrac{c}{a} n^{\log_b a}\log_b n\\
	&\sim \dfrac{c}{a}n^d \log_b n
	\end{align*}
	
	Caso $a \ne b^d$:
	\begin{align*}
	g\nparentesis{z} &= \dfrac{c}{b^d -a}\cdot \dfrac{1}{1-b^d z} + \dfrac{\nparentesis{b^d -a} t_1 - c}{b^d -a} \cdot \dfrac{1}{1-az}\\
	T\nparentesis{k}&=\dfrac{c}{b^d -a} \cdot b^{kd} + \dfrac{\nparentesis{b^d -a}t_1 - c}{b^d -a} \cdot a^k
	\end{align*}
	Si $a > b^d$:
	\begin{align*}
	T\nparentesis{k} &\sim \dfrac{\nparentesis{b^d - a}t_1 - c}{b^d -a} \cdot a^k\\
	t\nparentesis{n} &\sim  \dfrac{\nparentesis{b^d - a}t_1 -c}{b^d -a} \cdot n^{\log_b a}
	\end{align*}
	Si $a < b^d$:
	\begin{align*}
	T\nparentesis{k} & \sim \dfrac{c}{b^d -a} \cdot b^{kd}\\
	t\nparentesis{n} &\sim \dfrac{c}{b^d -a}\cdot n^d
	\end{align*}
	Resumiendo (\comillas{Teorema Maestro}):
        Al dividir un problema de tamaño \(n b\)
        en \(a\) problemas de tamaño \(n\),
        que se resuelven recursivamente,
        donde el trabajo que se hace al dividir y luego combinar soluciones
        está dado por \(c n^d\),
        resulta la recurrencia:
        \begin{equation*}
          t\nparentesis{nb}= at\nparentesis{n} + cn^d
          \quad t(1) = t_1
        \end{equation*}
        cuya solución es:
        \begin{align*}
          t(n)
            \sim \begin{cases}
                   \dfrac{c}{b^d - a} \cdot n^d  & a < b^d \\[1em]
                   \dfrac{c}{a} n^d \log_b n     & a = b^d \\[1em]
                   \dfrac{(b^d - a) t_1 - c}{a} \cdot n^{\log_n a}
                                                & a > b^d
                 \end{cases}
        \end{align*}
    Donde:
    \begin{itemize}
    	\item $nb$: tamaño del problema original.
    	\item $a$: cantidad de subproblemas originados luego de hacer la división.
    	\item $b$: tamaño relativo de los subproblemas respecto al problema grande.
    	\item $c$: constante determinada que nos dice lo que \comillas{nos cuesta} hacer la división. 
    	\item $d$: costo extra (volver a compaginar la solución completa)
    	\item $n$: tamaño de los subproblemas originados luego de hacer la división.
    	\item $t\nparentesis{n}$: trabajo necesario para resolver una instancia de tamaño $n$ del problema en cuestión.
    \end{itemize}
        

	Nuestros ejemplos se resumen en el cuadro~\ref{tab:complejidades}.
	\begin{table}[!h]
		\centering
		\begin{tabular}{l*{5}{|l}}
			&$a$ &$b$ &$c$ & $d$ & $t\nparentesis{n}$\\ \hline
			Mergesort & $2$ & $2$ & & $1$ & $\Theta\nparentesis{n \log n}$\\ \hline
			Búsqueda binaria & $1$ & $2$ & & $0$ & $\Theta\nparentesis{\log n}$\\ \hline
			Karatsuba & 3 & 2 & & 1 & $\Theta\nparentesis{n^{\log _2 3}}$\\
			\hline
			Strassen & 7 & 2 & & 2 & $\Theta\nparentesis{n^{\log_2 7}}$
		\end{tabular}
                \caption{Complejidad de nuestros ejemplos}
                \label{tab:complejidades}
	\end{table}

  Una variante de esto es el teorema de Akra-Bazzi,
  del que reportamos la variante de Leighton~%
    \cite{leighton96:_notes_better_master_theo}.
  \begin{teorema}[Akra-Bazzi]
    \index{Akra-Bazzi, teorema de|textbfhy}
    \label{theo:Akra-Bazzi}
    Sea una recurrencia de la forma:
    \begin{equation*}
      T(z)
	= g(z) + \sum_{1 \le k \le n} a_k T(b_k z + h_k(z))
	   \quad \text{para \(z \ge z_0\)}
    \end{equation*}
    donde \(z_0\), \(a_k\) y \(b_k\) son constantes,
    sujeta a las siguientes condiciones:
    \begin{itemize}
    \item
      Hay suficientes casos base.
    \item
      Para todo \(k\) se cumplen \(a_k > 0\) y \(0 < b_k < 1\).
    \item
      Hay una constante \(c\)
      tal que \(\lvert g(z) \rvert = O(z^c)\).
    \item
      Para todo \(k\)
      se cumple \(\lvert h_k(z) \rvert = O(z / (\log z)^2)\).
    \end{itemize}
    Entonces,
    si \(p\) es tal que:
    \begin{equation*}
      \sum_{1 \le k \le n} a_k b_k^p
	= 1
    \end{equation*}
    la solución a la recurrencia cumple:
    \begin{equation*}
      T(z)
	= \Theta
	    \left(
	      z^p \left(
		     1 + \int_1^z \frac{g(u)}{u^{p + 1}}
			   \, \mathrm{d} u
		  \right)
	    \right)
    \end{equation*}
  \end{teorema}
  Frente a nuestro tratamiento tiene la ventaja
  de manejar divisiones desiguales
  (\(b_k\) diferentes),
  y explícitamente
  considera pequeñas perturbaciones en los términos,
  como lo son aplicar pisos o techos,
  a través de los \(h_k(z)\).
  Diferencias con pisos y techos están acotados por una constante,
  mientras la cota del teorema permite que crezcan.
  Por ejemplo,
  la recurrencia correcta para el número de comparaciones
  en ordenamiento por intercalación es:
  \begin{equation*}
    T(n)
      = T(\lfloor n / 2 \rfloor) + T(\lceil n / 2 \rceil) + n - 1
  \end{equation*}
  El teorema de Akra-Bazzi es aplicable.
  La recurrencia es:
  \begin{equation*}
    T(n)
      = T(n / 2 + h_{+}(n)) + T(n / 2 + h_{-}(n)) + n - 1
  \end{equation*}
  Acá \(\lvert h_{\pm}(n) \rvert \le 1/2\),
  además \(a_{\pm} = 1\) y \(b_{\pm} = 1/2\).
  Estos cumplen las condiciones del teorema,
  de:
  \begin{equation*}
    \sum_{1 \le k \le 2} a_k b_k^p = 1
  \end{equation*}
  resulta \(p = 1\),
  y tenemos la cota:
  \begin{equation*}
    T(z)
      = \Theta
	  \left(
	    z \left(
	  1 + \int_1^z \frac{u - 1}{u^2} \, \mathrm{d} u
	      \right)
	  \right)
      = \Theta
	  \left(
	    z \ln z + 1
	  \right)
      = \Theta(z \log z)
  \end{equation*}
  Otro ejemplo son los árboles de búsqueda aleatorizados
  (\emph{\foreignlanguage{english}{Randomized Search Trees}},
   ver por ejemplo Aragon y Seidel~%
     \cite{aragon89:_random_search_tree},
   Martínez y Roura~%
     \cite{martinez98:_random_binar_searc_trees}
   y Seidel y Aragon~%
     \cite{seidel96:_random_search_trees})
  en uno de ellos de tamaño~\(n\)
  una búsqueda toma tiempo aproximado:
  \begin{equation*}
    T(n)
      = \frac{1}{4} \, T(n / 4) + \frac{3}{4} \, T(3 n / 4) + 1
  \end{equation*}
  Nuevamente es aplicable el teorema~\ref{theo:Akra-Bazzi},
  de:
  \begin{equation*}
    \frac{1}{4} \, \left(\frac{1}{4}\right)^p
	+ \frac{3}{4} \left(\frac{3}{4}\right)^p
      = 1
  \end{equation*}
  obtenemos \(p = 0\),
  y por tanto la cota
  \begin{equation*}
    T(z)
      = \Theta \left(
	  z^0 \left( 1 + \int_1^z \frac{\mathrm{d} u}{u} \right)
	\right)
      = \Theta ( \log z )
  \end{equation*}

  Una variante útil,
  pero bastante engorrosa,
  del teorema maestro
  es la que presenta Yap~%
    \cite{yap11:_elemen_approach_master_recurrences}.

\bibliography{../referencias}
\end{document}


\documentclass[english, spanish, fleqn, 10pt]{article}
\usepackage[top = 2.5cm, bottom = 2cm, left = 2.5cm, right = 2.5cm]{geometry}
\usepackage[es-noindentfirst]{babel}
\usepackage[utf8]{inputenc}
\usepackage{amsthm, amsmath, amssymb, amsfonts, environ}
\usepackage{caption}
\usepackage{subcaption} % para las subfigures
\usepackage{mathrsfs}
\usepackage[colorlinks, urlcolor=blue,
pdfauthor={Aldo Berrios Valenzuela}]{hyperref}
\usepackage{fourier}
\usepackage{colortbl}
\usepackage{color}
\usepackage{graphicx}
\usepackage{enumitem}
\usepackage{subcaption} % para las subfigures
%\renewcommand{\familydefault}{\sfdefault}
%\setlength{\parindent}{0pt}					%Espacio de sangria
\setlength{\parskip}{0.5mm}						%Interlinado entre párrafos
\usepackage{fancyhdr, blindtext}		%Paquete de encabezado
\usepackage{listings}
\usepackage[framemethod=tikz]{mdframed}

\definecolor{webred}{rgb}{0.75,0,0}
\definecolor{mblue}{rgb}{0.0705,0.345,0.7529}
\definecolor{newmblue}{HTML}{004D99}
\usepackage{longtable, colortbl, array}

\definecolor{superGreenNew}{HTML}{169F01}
\hypersetup{
	colorlinks   = true,
	citecolor    = superGreenNew
}


%==============================
%Modificador de Titulos de secciones, subsecciones, etc
\usepackage[explicit, noindentafter]{titlesec}

\titlespacing\subsubsection{0pt}{8pt plus 4pt minus 2pt}{4pt plus 2pt minus 2pt}


%==================================
%Diseno para insertar codigos de programacion
\definecolor{newGray}{HTML}{F8F8F8}
\definecolor{anotherGray}{HTML}{DDDFFF}
\lstdefinestyle{customc}{
	belowcaptionskip=1\baselineskip,
	breaklines=true,
	backgroundcolor=\color{newGray},
	frame=single,
	rulecolor=\color{anotherGray},
	xleftmargin=\parindent,
	language=bash,
	showstringspaces=false,
	basicstyle=\small\ttfamily,
	keywordstyle=\bfseries\color{green!40!black},
	commentstyle=\itshape\color{purple!40!black},
	identifierstyle=\color{black},
	stringstyle=\color{mblue},
	tabsize=2,
	literate={\ \ }{{\ }}1	
}
\lstset{style=customc}

%==================================
%Macros útiles
\newcommand{\comillas}[1]{``#1''}
\newcommand{\comentarioc}[1]{\texttt{\textcolor{webred}{/* #1 */}}}

\numberwithin{equation}{section}

%=================================
%comandos matemáticos personalizados de rápido acceso
\newcommand{\nparentesis}[1]{\left( #1 \right)}
\newcommand{\llaves}[1]{\left \{ #1 \right \}}
\newcommand{\nabsoluto}[1]{\left| #1 \right|}
\newcommand{\ncorchetes}[1]{\left[ #1 \right]}
%=================================

%cambia el espaciado entre el parrafo y antes del itemize
\setlist[itemize]{itemsep=1mm, topsep=1.5mm}
\setlist[enumerate]{itemsep=1mm, topsep=1.5mm}
%===========================

\theoremstyle{definition}
\newtheorem{teorema}{Teorema}[section]
\newtheorem{definition}{Definición}[section]
\newtheorem{beforeExample}{Ejemplo}[section]
\newtheorem{preposicion}{Preposición}[section]
\newtheorem{corolario}{Corolario}[teorema]

\captionsetup{justification=centering, margin=2.3cm}

\newenvironment{ejemplo}[1][]{\begin{beforeExample}[#1]\renewcommand{\qedsymbol}{$\blacksquare$}}{\qed\end{beforeExample}}

\captionsetup{justification=centering, margin=2.3cm}

% Para poder hacer "quote" con estilo github

\newenvironment{newquote}{\begin{mdframed}[topline=false,
		rightline=false, bottomline=false, linewidth=.3em,backgroundcolor=black!5, linecolor=black!60]\color{black!70}}{\end{mdframed}}
%================================

\renewcommand*{\arraystretch}{1.2}

\usepackage{fancyhdr}

\pagestyle{fancy}
\lhead{INF221 \quad Algoritmos y Complejidad}
\rhead{\textit{Clase \#19\quad Método Simbólico}}

\DeclareMathOperator{\Seq}{\textsc{Seq}}
\DeclareMathOperator{\Cyc}{\textsc{Cyc}}
\DeclareMathOperator{\Set}{\textsc{Set}}
\DeclareMathOperator{\MSet}{\textsc{MSet}}


\author{Aldo Berrios Valenzuela}

\title{INF221 -- Algoritmos y Complejidad\\[.4\baselineskip]Clase 19\\Método Simbólico}

\date{Miércoles 12 de octubre de 2016}

\usepackage{tikz,ifthen}
\usetikzlibrary{automata,positioning, babel}
\definecolor{navyblue}{RGB}{0,148,222}
\usetikzlibrary{fit}
\tikzset{shorten >=1pt,node distance=1.8cm,on grid, >=stealth, initial text=Inicio,
	every state/.style={ draw=black ,fill=black, minimum size=2mm},
	bend angle=35, every loop/.style={looseness=13}}



\begin{document}
\maketitle
\section{Método Simbólico}
\paragraph{Idea:} Describir cómo se construyen objetos, obtener funciones generatrices \emph{directamente}.

\begin{definition}[Clase Combinatoria]
	Una \emph{clase} (combinatoria) es un conjunto $\mathcal{A}$, de \emph{objetos} $\alpha \in \mathcal{A}$. Un objeto tiene un \emph{tamaño} $\nabsoluto{\alpha}\in\mathbb{N}_0$. Por ejemplo, $\mathcal{A}$ pueden ser todas las permutaciones de un arreglo $a\ncorchetes{n}$, donde $\alpha \in \mathcal{A}$ es una permutación.
\end{definition}

Al conjunto de objetos de tamaño $n$ lo llamaremos $\mathcal{A}_n$, el número de objetos de tamaño $n$ es $\nabsoluto{\mathcal{A}_n}=a_n$. Exigimos $a_n$ finito para todo $n\in\mathbb{N}_0$.

Las funciones generatrices asociadas son:
\begin{itemize}
	\item Ordinaria:
	\begin{equation}
	A\nparentesis{z} = \sum_{\alpha \in \mathcal{A}} z^{\nabsoluto{\alpha}} = \sum_{n \geq 0}a_nz^n
	\end{equation}
	
	\item Exponencial:
	\begin{equation}
	\hat A \nparentesis{z} = \sum_{\alpha \in \mathcal{A}} \dfrac{z^{\nabsoluto{k}}}{\nabsoluto{\alpha}!} = \sum_{n \geq 0} a_n \dfrac{z^n}{n!}
	\end{equation}
\end{itemize}
Clases básicas:
\begin{itemize}
	\item $\emptyset$: Vacía. Funciones generatrices 0.
	
	\item $\mathcal{E}$: Contiene un único objeto de tamaño 0, llamado $\epsilon$ (¡sorpresa!); La función generatriz correspondiente es:
	\begin{equation*}
	A\nparentesis{z} = 1
	\end{equation*}
	
	\item $\mathcal{Z}$: Contiene un único objeto de tamaño 1, llamado $\zeta$ (zeta); La función generatriz asociada a esta clase es:
	\begin{equation*}
	A\nparentesis{z} = z
	\end{equation*}

\end{itemize}
Operaciones:
\begin{itemize}
	\item $\mathcal{A} + \mathcal{B}$: Unión combinatoria; se permiten repetidas (objetos se \comillas{decoran} con su origen)
	
	\item $\mathcal{A} \times \mathcal{B}$: Objetos $\nparentesis{\alpha, \beta}$ con $\alpha \in \mathcal{A}$, $\beta \in \mathcal{B}$; $\nabsoluto{\nparentesis{\alpha, \beta}}=\nabsoluto{\alpha} + \nabsoluto{\beta}$
	
	\item $\Seq\nparentesis{\mathcal{A}}$: Secuencia de $\mathcal{A}$, o sea:
	\begin{equation*}
	\llaves{\epsilon, \alpha, \nparentesis{\alpha_1, \alpha_2}, \nparentesis{\alpha_1, \alpha_2, \alpha_3}, \ldots}
	\end{equation*}
	
	\item $\Set\nparentesis{\mathcal{A}}$: Subconjuntos de $\mathcal{A}$. Esto es como el conjunto de potencia.
	
	\item $\MSet\nparentesis{\mathcal{A}}$: Multisubcojuntos de $\mathcal{A}$
	
	\item $\Cyc\nparentesis{\mathcal{A}}$: Consiste en ordenar elementos de $\mathcal{A}$ en un círculo (una secuencia de inicio y fin). Consideremos el arreglo $\ncorchetes{a, b, b, b, b, a}$, dos ciclos para este arreglo se pueden apreciar en la figura \ref{19::EjemploCiclo}.
	\begin{figure}[!h]
		\centering
		\begin{subfigure}[!h]{0.3\textwidth}
			\centering
			\begin{tikzpicture}
			\node (a1) {$a$};
			\node [below left of = a1] (b1) {$b$};
			\node [below right of = a1] (b2) {$b$};
			\node [below of = b1] (b3) {$b$};
			\node [below of = b2] (b4) {$b$};
			\node [below right of = b3] (a2) {$a$};
			
			\path (a1) edge (b1)
			edge (b2)
			(b1) edge (b3)
			(b2) edge (b4)
			(b3) edge (a2)
			(b4) edge (a2);
			\end{tikzpicture}
			\caption{}
		\end{subfigure}
		\begin{subfigure}[!h]{0.3\textwidth}
			\centering
			\begin{tikzpicture}
			\node (a1) {$b$};
			\node [below left of = a1] (b1) {$b$};
			\node [below right of = a1] (b2) {$a$};
			\node [below of = b1] (b3) {$a$};
			\node [below of = b2] (b4) {$b$};
			\node [below right of = b3] (a2) {$b$};
			
			\path (a1) edge (b1)
			edge (b2)
			(b1) edge (b3)
			(b2) edge (b4)
			(b3) edge (a2)
			(b4) edge (a2);
			\end{tikzpicture}
			\caption{}
		\end{subfigure}
		\caption{Ejemplo de ciclos para el arreglo $\ncorchetes{a, b, b, b, b, a}$.}
		\label{19::EjemploCiclo}
	\end{figure}
\end{itemize}
Note que para las cuatro últimas operaciones se tiene que $\mathcal{A}$ \emph{no} contiene objetos de tamaño 0.


\subsection{Objetos no rotulados}

\begin{definition}[Objetos no rotulados]
	Formados por \emph{átomos} (tamaño de $\alpha$ es el número de átomos) indistinguibles / intercambiables.
\end{definition}

\begin{teorema}\label{19::Propiedades:No:Rotulados}
	Sean $\mathcal{A}$, $\mathcal{B}$ calses, con funciones generatrices ordinarias $A\nparentesis{z}$, $B\nparentesis{z}$. La función generatriz de:
	\begin{enumerate}
		\item
		Para enumerar \(\mathcal{A} + \mathcal{B}\):
		\begin{equation*}
		A(z) + B(z)
		\end{equation*}
		\item
		Para enumerar \(\mathcal{A} \times \mathcal{B}\):
		\begin{equation*}
		A(z) \cdot B(z)
		\end{equation*}
		\item
		Para enumerar \(\Seq(\mathcal{A})\):
		\begin{equation*}
		\frac{1}{1 - A(z)}
		\end{equation*}
		\item
		Para enumerar \(\Set(\mathcal{A})\):
		\begin{equation*}
		\prod_{\alpha \in \mathcal{A}}
		\left( 1 + z^{\lvert \alpha \rvert} \right)
		= \prod_{n \ge 1} (1 + z^n)^{a_n}
		= \exp \left(
		\sum_{k \ge 1} \frac{(-1)^{k + 1}}{k} \, A(z^k)
		\right)
		\end{equation*}
		\item
		Para enumerar \(\MSet(\mathcal{A})\):
		\begin{equation*}
		\prod_{\alpha \in \mathcal{A}}
		\frac{1}{1 - z^{\lvert \alpha \rvert}}
		= \prod_{n \ge 1} \frac{1}{(1 - z^n)^{a_n}}
		= \exp\left(
		\sum_{k \ge 1} \frac{A(z^k)}{k}
		\right)
		\end{equation*}
		\item
		Para enumerar \(\Cyc(\mathcal{A})\):
		\begin{equation*}
		\sum_{n \ge 1} \frac{\phi(n)}{n} \, \ln \nparentesis{\frac{1}{1 - A(z^n)}}
		\end{equation*}
	\end{enumerate}
\end{teorema}

\begin{proof}
	Usamos funciones generatrices para demostrar cada una de las propiedades del teorema \ref{19::Propiedades:No:Rotulados}.
	\begin{enumerate}
		\item Usando funciones generatrices:
		\begin{equation*}
		\sum_{\gamma \in \mathcal{A} + \mathcal{B}} z^{\nabsoluto{\gamma}} = \sum_{\gamma \in \mathcal{A}} z^{\nabsoluto{\gamma}} + \sum_{\gamma \in \mathcal{B}} z^{\nabsoluto{\gamma}} = A\nparentesis{z} + B\nparentesis{z}
		\end{equation*}
		
		\item De la misma forma que el anterior, usamos propiedades de funciones generatrices:
		\begin{equation*}
		\sum_{\gamma \in \mathcal{A} \times \mathcal{B}} z ^{\nabsoluto{\gamma}} = \sum_{\substack{\alpha \in \mathcal{A} \\ \beta \in \mathcal{B}}} z^{\nabsoluto{\nparentesis{\alpha, \beta}}} = \sum_{\substack{\alpha \in \mathcal{A} \\ \beta \in \mathcal{B}}}z^{\nabsoluto{\alpha} + \nabsoluto{\beta}} = \nparentesis{\sum_{\alpha \in \mathcal{A}} z^{\nabsoluto{\alpha}}} \cdot \nparentesis{\sum_{\beta \in \mathcal{B}} z^{\nabsoluto{\beta}}} = A\nparentesis{z} \cdot B \nparentesis{z}
		\end{equation*}
		
		\item Sea $\Seq \nparentesis{A} = S$. Por lo tanto, podemos describir la clase $\Seq \nparentesis{A}$ como sigue:
		\begin{equation*}
		S = \epsilon + S \times A
		\end{equation*}
		Pasamos lo anterior a funciones generatrices, con lo que se tiene:
		\begin{align*}
		S\nparentesis{z} &= 1 + S\nparentesis{z} A\nparentesis{z}\\
		S\nparentesis{z} &= \dfrac{1}{1 - A\nparentesis{z}}
		\end{align*}
		
		\item La clase de los subconjuntos finitos de \(\mathcal{A}\)
		queda representada por el producto simbólico:
		\begin{equation*}
		\prod_{\alpha \in \mathcal{A}} (\mathcal{E} + \{\alpha\})
		\end{equation*}
		ya que
		al distribuir los productos de todas las formas posibles
		aparecen todos los subconjuntos de \(\mathcal{A}\).
		Directamente obtenemos entonces:
		\begin{equation*}
		\prod_{\alpha \in \mathcal{A}}
		\left( 1 + z^{\lvert \alpha \rvert} \right)
		= \prod_{n \ge 0} (1 + z^n)^{a_n}
		\end{equation*}
		Otra forma de verlo es que cada objeto de tamaño \(n\)
		aporta un factor \(1 + z^n\),
		si hay \(a_n\) de estos
		el aporte total es \((1 + z^n)^{a_n}\).
		Esta es la primera parte de lo aseverado.
		Aplicando logaritmo:
		\begin{align*}
		\sum_{\alpha \in \mathcal{A}}
		\ln \left(1 + z^{\lvert \alpha \rvert} \right)
		&= -\sum_{\alpha \in \mathcal{A}}
		\sum_{k \ge 1}
		\frac{(-1)^k z^{\lvert \alpha \rvert k}}{k}  \\
		&= -\sum_{k \ge 1} \frac{(-1)^k}{k} \,
		\sum_{\alpha \in \mathcal{A}}
		z^{\lvert \alpha \rvert k} \\
		&= \sum_{k \ge 1} \frac{(-1)^{k + 1} \, A(z^k)}{k}
		\end{align*}
		Exponenciando lo último
		resulta equivalente a la segunda parte.
		
		\item Podemos considerar un multiconjunto finito
		como la combinación de una secuencia
		para cada tipo de elemento:
		\begin{equation*}
		\prod_{\alpha \in \mathcal{A}} \Seq(\{ \alpha \})
		\end{equation*}
		La función generatriz buscada es:
		\begin{equation*}
		\prod_{\alpha \in \mathcal{A}}
		\frac{1}{1 - z^{\lvert \alpha \rvert}}
		= \prod_{n \ge 0} \frac{1}{(1 - z^n)^{a_n}}
		\end{equation*}
		Esto provee la primera parte.
		Nuevamente aplicamos logaritmo para simplificar:
		\begin{align*}
		\ln \prod_{\alpha \in \mathcal{A}}
		\frac{1}{1 - z^{\lvert \alpha \rvert}}
		&= - \sum_{\alpha \in \mathcal{A}}
		\ln \left( 1 - z^{\lvert \alpha \rvert} \right) \\
		&= \sum_{\alpha \in \mathcal{A}}
		\sum_{k \ge 1}
		\frac{z^{k \lvert \alpha \rvert}}{k} \\
		&= \sum_{k \ge 1}
		\frac{1}{k} \,
		\sum_{\alpha \in \mathcal{A}}
		z^{k \lvert \alpha \rvert} \\
		&= \sum_{k \ge 1}
		\frac{A(z^k)}{k}
		\end{align*}
	\end{enumerate}
	
	
\end{proof}



\subsection{Algunas Aplicaciones}

\begin{ejemplo}[Árbol binario]
Un árbol binario está conformado de:
\begin{itemize}
	\item Vacío
	
	\item Raíz y árbol binario derecho y árbol binario izquierdo.
\end{itemize}
Sea $\mathcal{B}$ la clase árbol binario:
\begin{align*}
\mathcal{B} = \mathcal{E} + \mathcal{Z} \times \mathcal{B}\times \mathcal{B}
\end{align*}
Entonces:
\begin{equation}
B\nparentesis{z} = 1 + zB^2\nparentesis{z}
\end{equation}
y en consecuencia:
\begin{equation}
b_n = C_n = \dfrac{1}{n+1}
\begin{pmatrix}
2n\\n
\end{pmatrix}
\end{equation}
Entonces:
\begin{equation}
B\nparentesis{z} = \dfrac{1 \pm \sqrt{1-4z}}{2z}
\end{equation}
Condición adicional: $b_0$ es \emph{finito}. Una vez que estamos en eso, aplicamos la receta de nuestro amigo Vicente, el teorema del binomio:
\begin{equation}
B\nparentesis{z} = \dfrac{1 - \sqrt{1-4z}}{2z}
\end{equation}
\end{ejemplo}


\begin{ejemplo}[Números Naturales]
	Podemos representar un natural como $|||| = 4$ (palito, palito, palito, palito). Entonces, podemos representar los números naturales como una secuencia de palitos, es decir:
	\begin{equation*}
	\mathbb{N}_0 = \Seq \nparentesis{Z}
	\end{equation*}
	Por otro lado, si queremos representar números naturales partiendo de $1$ tenemos:
	\begin{equation*}
	\mathbb{N} = Z \times \Seq \nparentesis{z}
	\end{equation*}
	donde $Z$ es la secuencia de al menos 1 palito.
	
	
	Una \emph{combinación} de $n$ es escribirlo como una suma de naturales:
	\begin{align*}
	4&= 1 + 1 + 1 + 1\\
	&= 1 + 1 + 2\\
	&= 1+ 2+ 1\\
	&= 2+ 1 + 1\\
	&= 2 + 2\\
	&= 3 + 1\\
	&= 1 + 3\\
	&= 4
	\end{align*}
	Combinación: Secuencia de $\mathbb{N}$, por lo tanto:
	\begin{equation*}
	C = \Seq\nparentesis{Z \times \Seq\nparentesis{Z}}
	\end{equation*}
	Por lo tanto:
	\begin{align*}
	C_n\nparentesis{z} &= \dfrac{1}{1-\dfrac{z}{1-z}}\\
	&=\dfrac{1-z}{1-\nparentesis{L\pi\rfloor -1}z}\\
	&=\nparentesis{1-z}\sum_{k\geq 0} \left\lceil\sqrt{2} \right\rceil^k z^k\\
	&=\dfrac{1-z}{1-2z}\\
	\ncorchetes{z^n} C\nparentesis{z} &= \ncorchetes{z^n}\dfrac{1}{1-2z}-\ncorchetes{z^n}\dfrac{z}{1-2z}\\
	&=2^n - \ncorchetes{n \geq 1} \ncorchetes{z^{n-1}} \dfrac{1}{1-2z}\\
	&=2^n - \ncorchetes{n \geq 1} 2^{n-1}\\
	&=\begin{cases}
	1&\text{si } n= 0\\
	2^{n-1}&\text{si } n \geq 1
	\end{cases}
	\end{align*}
	
\end{ejemplo}





\end{document}


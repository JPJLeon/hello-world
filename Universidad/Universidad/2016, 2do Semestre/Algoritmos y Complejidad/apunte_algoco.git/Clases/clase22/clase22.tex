\documentclass[english, spanish, fleqn, 10pt]{article}
\usepackage[top = 2.5cm, bottom = 2cm, left = 2.5cm, right = 2.5cm]{geometry}
\usepackage[es-noindentfirst]{babel}
\usepackage[utf8]{inputenc}
\usepackage{amsthm, amsmath, amssymb, amsfonts, environ}
\usepackage{caption}
\usepackage{subcaption} % para las subfigures
\usepackage{mathrsfs}
\usepackage[colorlinks, urlcolor=blue,
pdfauthor={Aldo Berrios Valenzuela}]{hyperref}
\usepackage{fourier}
\usepackage{colortbl}
\usepackage{color}
\usepackage{graphicx}
\usepackage{enumitem}
%\renewcommand{\familydefault}{\sfdefault}
%\setlength{\parindent}{0pt}					%Espacio de sangria
\setlength{\parskip}{0.5mm}						%Interlinado entre párrafos
\usepackage{fancyhdr, blindtext}		%Paquete de encabezado
\usepackage{listings}
\usepackage[framemethod=tikz]{mdframed}

\definecolor{webred}{rgb}{0.75,0,0}
\definecolor{mblue}{rgb}{0.0705,0.345,0.7529}
\definecolor{newmblue}{HTML}{004D99}
\usepackage{longtable, colortbl, array}

\definecolor{superGreenNew}{HTML}{169F01}
\hypersetup{
	colorlinks   = true,
	citecolor    = superGreenNew
}


%==============================
%Modificador de Titulos de secciones, subsecciones, etc
\usepackage[explicit, noindentafter]{titlesec}

\titlespacing\subsubsection{0pt}{8pt plus 4pt minus 2pt}{4pt plus 2pt minus 2pt}


%==================================
%Diseno para insertar codigos de programacion
\definecolor{newGray}{HTML}{F8F8F8}
\definecolor{anotherGray}{HTML}{DDDFFF}
\lstdefinestyle{customc}{
	belowcaptionskip=1\baselineskip,
	breaklines=true,
	backgroundcolor=\color{newGray},
	frame=single,
	rulecolor=\color{anotherGray},
	xleftmargin=\parindent,
	language=bash,
	showstringspaces=false,
	basicstyle=\small\ttfamily,
	keywordstyle=\bfseries\color{red!40!black},
	commentstyle=\itshape\color{purple!40!black},
	identifierstyle=\color{black},
	stringstyle=\color{mblue},
	tabsize=2,
	literate={\ \ }{{\ }}1	
}
\lstset{style=customc}

\renewcommand{\lstlistingname}{Listado}

%==================================
%Macros útiles
\newcommand{\comillas}[1]{``#1''}
\newcommand{\comentarioc}[1]{\texttt{\textcolor{webred}{/* #1 */}}}

\numberwithin{equation}{section}

%=================================
%comandos matemáticos personalizados de rápido acceso
\newcommand{\nparentesis}[1]{\left( #1 \right)}
\newcommand{\llaves}[1]{\left \{ #1 \right \}}
\newcommand{\nabsoluto}[1]{\left| #1 \right|}
\newcommand{\ncorchetes}[1]{\left[ #1 \right]}
%=================================

%cambia el espaciado entre el parrafo y antes del itemize
\setlist[itemize]{itemsep=1mm, topsep=1.5mm}
\setlist[enumerate]{itemsep=1mm, topsep=1.5mm}
%===========================

\theoremstyle{definition}
\newtheorem{teorema}{Teorema}[section]
\newtheorem{definition}{Definición}[section]
\newtheorem{beforeExample}{Ejemplo}[section]
\newtheorem{preposicion}{Preposición}[section]
\newtheorem{corolario}{Corolario}[teorema]

\captionsetup{justification=centering, margin=2.3cm}

\newenvironment{ejemplo}[1][]{\begin{beforeExample}[#1]\renewcommand{\qedsymbol}{$\blacksquare$}}{\qed\end{beforeExample}}

\captionsetup{justification=centering, margin=2.3cm}

% Para poder hacer "quote" con estilo github

\newenvironment{newquote}{\begin{mdframed}[topline=false,
		rightline=false, bottomline=false, linewidth=.3em,backgroundcolor=black!5, linecolor=black!60]\color{black!70}}{\end{mdframed}}
%================================

\renewcommand*{\arraystretch}{1.2}

\usepackage{fancyhdr}

\pagestyle{fancy}
\lhead{INF221\quad Algoritmos y Complejidad}
\rhead{\textit{Clase \#22\quad Métodos de Ordenamiento III}}


\author{Aldo Berrios Valenzuela}

\title{INF221 -- Algorimtos y Complejidad\\[.4\baselineskip]
	Clase \#22\\
	Métodos de Ordenamiento III}

\begin{document}
\maketitle
\section{QuickSort}
\textbf{Idea:}
\begin{center}
	\comentarioc{Dibujo}
\end{center}
Supuestos:
\begin{itemize}
	\item Elementos distintos.
	\item Todas las permutaciones igualmente probables.
	\item Medidas de costos son comparaciones entre elementos.
	\item Dado lo anterior, particionar compara cada elemento con el pivote. Esto es, $n-1$ comparaciones.
	\item Partición depende del pivote. Obtener \emph{promedio}.
	\item Dado el pivote, una vez particionado quedan \comillas{en desorden} a cada lado. \comentarioc{consideremos los peores casos: no ordenamos casi nada}
	
	\begin{center}
		\comentarioc{dibujo}
	\end{center}
\end{itemize}
Sea $C\nparentesis{n}:$ el número promedio de comparaciones al ordenar $n$ elementos. Luego, el pivote puede ser cualquiera de los elementos $\ncorchetes{1, \ldots, n}$, todos son igualmente probables:
\begin{equation}
C\nparentesis{n} = n-1+\dfrac{1}{n}\sum_{1\leq k \leq n}\nparentesis{C\nparentesis{k-1}+C\nparentesis{n-k-1}},\qquad C\nparentesis{0}=0
\end{equation}
Hacemos un ajuste con los índices \comentarioc{el C de la izquierda va en subida y el de la derecha va en bajada (ambos dentro de la sumatoria)}:
\begin{align*}
n C \nparentesis{n} &= n\nparentesis{n-1} + 2\sum_{0 \leq k \leq n-1}C\nparentesis{k}\\
\nparentesis{n+1} C\nparentesis{n+1} &= \nparentesis{n+1}n + 2\sum_{0 \leq k \leq n} C\nparentesis{k}
\end{align*}
Sea:
\begin{equation*}
C\nparentesis{z} = \sum_{n \geq 0} C\nparentesis{n} z^n
\end{equation*}
Por propiedades de funciones generatrices ordinarias:
\begin{align*}
\nparentesis{zD + 1}\dfrac{C\nparentesis{z} - C\nparentesis{0}}{z} = \sum_{n \geq 0} n\nparentesis{n+1} z^n + 2\dfrac{C\nparentesis{z}}{1-z}
\end{align*}
Resolvemos la serie de la ecuación anterior:
\begin{align*}
\sum_{n \geq 0} n\nparentesis{n+1} z^n &= zD \nparentesis{zD + 1} \dfrac{1}{1-z}
\end{align*}
Efectuando las operaciones:
\begin{equation}
C'\nparentesis{z} = \dfrac{2C\nparentesis{z}}{1-z} + \dfrac{2z}{\nparentesis{1-z}^3},\qquad C\nparentesis{0} = C'\nparentesis{0}  = 0
\end{equation}
Finalmente, la solución de la función generatriz es:
\begin{equation}
C\nparentesis{z} = -2\cdot  \dfrac{\ln \nparentesis{1-z}}{\nparentesis{1-z}^2} - \dfrac{2z}{\nparentesis{1-z} ^3}
\end{equation}
Coeficientes son, por inspección:
\begin{equation*}
C\nparentesis{n} = 2 \sum_{1 \leq k \leq n} H_n - 2 \binom{n}{1}
\end{equation*}
Buceando en el apunte de fundamentos:
\begin{equation*}
\sum_{1 \leq k \leq n} H_n = \nparentesis{n+1} \nparentesis{H_{n+1} - 1}
\end{equation*}
(vía G.F, o sumas por partes / suma de Abel)

Finalmente:
\begin{equation}
C\nparentesis{n} = 2\nparentesis{n + 1} H _n  - 4n
\end{equation}
Como:
\begin{align*}
H_n &= \ln\nparentesis{n} + \gamma + O \nparentesis{\dfrac{1}{n}}\\
C\nparentesis{n} &= 2n \ln \nparentesis{n} + O\nparentesis{n}
\end{align*}
\comentarioc{Caso promedio de QuickSort es $n\log n$}

\subsection{Consideraciones prácticas}
\begin{itemize}
	\item Mediana de 3.
	\item Cortar antes, terminar con inserción $\rightsquigarrow$ $\pm$ 10 a 30.
	\item Conviene pasar sobre elementos $=p$ (si no, terminamos con el pivote en un extremo si son todos iguales \ldots )
	
	\item \comillas{Fat partitioning}:
	\begin{center}
		\comentarioc{dibujo: juntamos todos los elementos que sean iguales al pivote en lugar de usar un único elemento.}		
	\end{center}
	Dutch national Hag.
\end{itemize}




\end{document}

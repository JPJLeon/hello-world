\documentclass[english, spanish, fleqn, 10pt]{article}
\usepackage[top = 2.5cm, bottom = 2cm, left = 2.5cm, right = 2.5cm]{geometry}
\usepackage[es-noindentfirst]{babel}
\usepackage{babelbib}
\usepackage[utf8]{inputenc}
\usepackage{amsthm, amsmath, amssymb, amsfonts, environ}
\usepackage{caption}
\usepackage{subfig}
\usepackage{mathrsfs}
\usepackage[colorlinks, urlcolor=blue,
pdfauthor={Aldo Berrios Valenzuela}]{hyperref}
\usepackage{fourier}
\usepackage{colortbl}
\usepackage{color}
\usepackage{graphicx}
\usepackage{enumitem}
%\renewcommand{\familydefault}{\sfdefault}
%\setlength{\parindent}{0pt}					%Espacio de sangria
\setlength{\parskip}{0.5mm}						%Interlinado entre párrafos
\usepackage{fancyhdr, blindtext}		%Paquete de encabezado
\usepackage{listings}
\usepackage[framemethod=tikz]{mdframed}

\definecolor{webred}{rgb}{0.75,0,0}
\definecolor{mblue}{rgb}{0.0705,0.345,0.7529}
\definecolor{newmblue}{HTML}{004D99}
\usepackage{longtable, colortbl, array}

\definecolor{superGreenNew}{HTML}{169F01}
\hypersetup{
	colorlinks   = true,
	citecolor    = superGreenNew
}


%==============================
%Modificador de Titulos de secciones, subsecciones, etc
\usepackage[explicit, noindentafter]{titlesec}

\titlespacing\subsubsection{0pt}{8pt plus 4pt minus 2pt}{4pt plus 2pt minus 2pt}


%==================================
%Diseno para insertar codigos de programacion
\definecolor{newGray}{HTML}{F8F8F8}
\definecolor{anotherGray}{HTML}{DDDFFF}

\lstloadlanguages{C, Python}

\lstdefinestyle{customc}{
	belowcaptionskip=1\baselineskip,
	breaklines=true,
	backgroundcolor=\color{newGray},
	frame=single,
	rulecolor=\color{anotherGray},
	xleftmargin=\parindent,
	language=bash,
	showstringspaces=false,
	basicstyle=\small\ttfamily,
	keywordstyle=\bfseries\color{blue!60!black},
	commentstyle=\itshape\color{purple!40!black},
	identifierstyle=\color{black},
	stringstyle=\color{mblue},
	tabsize=2,
	literate={\ \ }{{\ }}1	
}
\lstset{style=customc}


%==================================
%Macros útiles
\newcommand{\comillas}[1]{``#1''}
\newcommand{\comentarioc}[1]{\texttt{\textcolor{webred}{/* #1 */}}}

\numberwithin{equation}{section}

%=================================
%comandos matemáticos personalizados de rápido acceso
\newcommand{\nparentesis}[1]{\left( #1 \right)}
\newcommand{\llaves}[1]{\left \{ #1 \right \}}
\newcommand{\nabsoluto}[1]{\left| #1 \right|}
\newcommand{\ncorchetes}[1]{\left[ #1 \right]}
%=================================

%cambia el espaciado entre el parrafo y antes del itemize
\setlist[itemize]{itemsep=1mm, topsep=1.5mm}
\setlist[enumerate]{itemsep=1mm, topsep=1.5mm}
%===========================

\theoremstyle{definition}
\newtheorem{teorema}{Teorema}[section]
\newtheorem{definition}{Definición}[section]
\newtheorem{beforeExample}{Ejemplo}[section]
\newtheorem{preposicion}{Preposición}[section]
\newtheorem{corolario}{Corolario}[teorema]

\captionsetup{justification=centering, margin=2.3cm}

\newenvironment{ejemplo}[1][]{\begin{beforeExample}[#1]\renewcommand{\qedsymbol}{$\blacksquare$}}{\qed\end{beforeExample}}

\captionsetup{justification=centering, margin=2.3cm}

% Para poder hacer "quote" con estilo github

\newenvironment{newquote}{\begin{mdframed}[topline=false,
		rightline=false, bottomline=false, linewidth=.3em,backgroundcolor=black!5, linecolor=black!60]\color{black!70}}{\end{mdframed}}
%================================

\renewcommand*{\arraystretch}{1.2}

\usepackage{fancyhdr}

\pagestyle{fancy}
\lhead{INF221\quad Algoritmos y Complejidad}
\rhead{\textit{Clase 17\quad Dividir y Conquistar}}


\author{Aldo Berrios Valenzuela \and Horst H. von Brand}

\title{INF221 -- Algoritmos y Complejidad\\[.4\baselineskip]Clase \#17\\Dividir y Conquistar II}

\date{Miércoles 5 de octubre de 2016}

\begin{document}
\bibliographystyle{babplain-fl}
\maketitle
\lstset{language=[ANSI]C,
	basicstyle=\ttfamily\small, commentstyle=\slshape,
	extendedchars, frame=lines, numbers=none,
	float, floatplacement=ht, captionpos=b,
	xleftmargin=1em, xrightmargin=1em
       }
\lstset{literate={á}{{\'a}}1
		 {é}{{\'e}}1
		 {í}{{\'\i}}1
		 {ó}{{\'o}}1
		 {ú}{{\'u}}1
		 {ü}{{\"u}}1
		 {Á}{{\'A}}1
		 {ñ}{{\~n}}1
		 {É}{{\'E}}1
		 {Í}{{\'I}}1
		 {Ó}{{\'O}}1
		 {Ú}{{\'U}}1
		 {Ü}{{\"U}}1
		 {Ñ}{{\~N}}1
		 {¿}{{?`}}1
		 {¡}{{!`}}1
	}

\renewcommand{\lstlistingname}{Listado}


\section{Dividir y Conquistar II}

  Discutiremos un problema planteado por Bentley~%
    \cite{bentley84:_algorithm_design_techn}.
Dado el arreglo $a\ncorchetes{n}$, hallar la máxima suma de un rango:
\begin{equation}
  \label{eq:problema}
  \max_{i, j} \left\{ \sum_{i \le k \le j} a[k] \right\}
\end{equation}
  Si todos los valores son positivos,
  la respuesta es obvia:
  la suma de todos los elementos del arreglo.
  El punto está si hay elementos negativos:
  ¿incluimos uno de ellos
  en la esperanza que los elementos positivos que lo rodean
  más que lo compensen?
  Finalmente,
  acordamos que la suma de un rango vacío es cero,
  y que en un arreglo de elementos negativos la suma máxima es cero.
  
\subsection{Algoritmo ingenuo}
\label{sec:algoritmo-1}

  La solución obvia,
  traducción directa de la especificación
  ecuación~\eqref{eq:problema},
  es la mostrada en el listado~\ref{lst:algoritmo-1}.
  \lstinputlisting[language = C,
                   float,
                   firstline = 8,
                   caption = {Algoritmo 1: Versión ingenua},
                   label = lst:algoritmo-1]
                  {max-sum-1.c}
La complejidad del algoritmo~1 es $O\nparentesis{n^3}$. Lo que buscamos es mejorarlo. 

\subsection{No recalcular sumas}
\label{sec:no-recalcular-sumas}

  Hay dos ideas sencillas para evitar recalcular sumas.

\subsubsection{Extender sumas}
\label{sec:algoritmo-2}

  En vez de calcular la suma del rango cada vez,
  extendemos la suma anterior.
  Esto da el programa del listado~\ref{lst:algoritmo-2}.
  \lstinputlisting[language = C,
                   float,
                   firstline = 8,
                   caption = {Algoritmo 2: Evitar recalcular sumas},
                   label = lst:algoritmo-2]
                  {max-sum-2.c}
La complejidad del algoritmo~2 es $O\nparentesis{n^2}$.

\subsubsection{Sumas cumulativas}
\label{sec:algoritmo-3}

  Una manera de manejar rangos es usar sumas cumulativas,
  y obtener el valor para el rango restando.
  Esta idea da el listado~\ref{lst:algoritmo-3}.
  \lstinputlisting[language = C,
                   float,
                   firstline = 8,
                   caption = {Algoritmo 3: Usar arreglo cumulativo},
                   label = lst:algoritmo-3]
                  {max-sum-3.c}
La complejidad del algoritmo~3 es $O\nparentesis{n^2}$. En función de nuestro algoritmo original resulta una mejora, pero no respecto a la segunda variante.

\subsection{Dividir y Conquistar}
\label{sec:algoritmo-4}

  Aplicar la estragia vista la clase pasada
  lleva a la figura~\ref{subfig:Algoritmo-4-AB}.
  Pero debemos también considerar que el rango con máxima suma
  esté a hojarcadas,
  cruzando el punto central,
  como en la figura~\ref{subfig:Algoritmo-4-C}.
  \begin{figure}[ht]
    \centering
    \subfloat[Máximos en las mitades]{
      \begin{tikzpicture}[scale=0.75]
        \draw[thick] (0, 0) rectangle (20, 1);
        \draw[thick] (10, 0) -- (10, 1);

        \draw[fill = lightgray, thick] (2, 0) rectangle (5, 1);
        \node at (3.5, 0) [below] {\(M_A\)};

        \draw[fill = lightgray, thick] (15, 0) rectangle (19, 1);
        \node at (17, 0) [below] {\(M_B\)};
      \end{tikzpicture}
      \label{subfig:Algoritmo-4-AB}
    } \\
    \subfloat[Máximo al medio]{
      \begin{tikzpicture}[scale=0.75]
        \draw[fill = lightgray, thick] (7, 0) rectangle (11, 1);
        \draw[thick] (0, 0) rectangle (20, 1);
        \draw[thick] (10, 0) -- (10, 1);

        \node at (9, 0) [below] {\(M_C\)};
      \end{tikzpicture}
      \label{subfig:Algoritmo-4-C}
    }
    \caption{Dividir y conquistar}
    \label{fig:algoritmo-4}
  \end{figure}
El algoritmo es el dado en el listado~\ref{lst:algoritmo-4}.
  \lstinputlisting[language = C,
                   float,
                   firstline = 8,
                   caption = {Algoritmo 4: Usar sumas cumulativas},
                   label = lst:algoritmo-4]
                  {max-sum-4.c}
Usando el \comillas{teorema maestro}, las constantes del algoritmo~4 son $a=2$, $b=2$, $d=1$, por lo tanto la complejidad es \(O\nparentesis{n \log n}\).

\subsection{Un algoritmo lineal}
\label{sec:algoritmo-5}

  Otro algoritmo resulta de la idea,
  común al procesar arreglos,
  de tener una solución parcial hasta \(a[i]\),
  y analizar cómo extenderla para cubrir hasta \(a[i + 1]\).
  En nuestro caso,
  esto significa considerar la máxima suma que llega hasta \(a[i]\),
  y recordar la máxima suma vista hasta ahora,
  ver la figura~\ref{fig:Algoritmo-5}.
  Esto da el algoritmo 5,
  del listado~\ref{lst:algoritmo-5}.
  \begin{figure}[ht]
    \centering
    \begin{tikzpicture}[scale=0.75]
      \draw[thick] (0, 0) rectangle (20, 1);

      \draw[fill = lightgray, thick] (2, 0) rectangle (8, 1);
      \node at (5, -0.1) [below] {\textsf{MaxSoFar}};

      \draw[fill = lightgray, thick] (11, 0) rectangle (16, 1);
      \node at (13.5, -0.1) [below] {\textsf{MaxToHere}};

      \draw[thick] (16, 1.1) -- (16, -0.1) node [below] {\textsf{i}};
    \end{tikzpicture}
    \caption{Extender la solución}
    \label{fig:Algoritmo-5}
  \end{figure}

  \lstinputlisting[language = C,
                   float,
                   firstline = 8,
                   caption = {Algoritmo 5: Ir extendiendo resultado parcial},
                   label = lst:algoritmo-5]
                  {max-sum-5.c}
La complejidad del algoritmo~5 es $O\nparentesis{n}$. Sin embargo, es imposible tener una complejidad menor que $n$, dado que es necesario revisar cada elemento del arreglo.

\begin{table}[!h]
	\centering
	\begin{tabular}{r|*{5}{|c}}
	  \multicolumn{1}{c||}{\textbf{Algoritmo}}
            & \textbf{1} & \textbf{2} & \textbf{4} & \textbf{5} \\
          \hline
          \multicolumn{1}{l||}{Líneas de C} & 8 & 7 & 14 & 7 \\
          \hline
          \multicolumn{1}{l||}{Tiempo en $\ncorchetes{\mu s}$}
            & $3.4n^3$ & $13 n^2$ & $46 n \log n$ & $33 n$ \\
          \hline
          Tiempo para $n = {}$
                $10^2$ & $3.4 \ncorchetes{s}$ & $130\ncorchetes{ms}$ & $30\ncorchetes{ms}$ & $3.3\ncorchetes{ms}$\\
          \hline
                $10^3$ & $0.94\ncorchetes{h}$ & $14 \ncorchetes{s}$ & $0.45\ncorchetes{s}$ & $33\ncorchetes{ms}$ \\
          \hline
		$10^4$& $39\ncorchetes{\text{dias}}$ & $22\ncorchetes{\text{min}}$ & $6.1\ncorchetes{s}$ & $0.33\ncorchetes{s}$\\
          \hline
		$10^5$ & $108$ años & $1.5$ días & $1.3$ min & $3.3\ncorchetes{s}$\\
          \hline
		$10^6$ & 108 millones de años & 5 meses & 15 min & $33\ncorchetes{s}$
	\end{tabular}
	\caption{Comparativa de Bentley~%
                   \cite{bentley84:_algorithm_design_techn}
                 entre las variantes}
        \label{tab:comparativa-algoritmos}
\end{table}
  Reportar la complejidad de un algoritmo en términos de \(O(\cdot)\)
  es incompleto,
  pero el cuadro~\ref{tab:comparativa-algoritmos} muestra su relevancia.
  La ventaja es que la complejidad en estos términos es sencilla de obtener,
  en nuestros casos simples
  (algoritmos 1, 2, 3 y 5)
  por inspección,
  el teorema maestro da la complejidad para el algoritmo 4 directamente.



\bibliography{../referencias}
\end{document}


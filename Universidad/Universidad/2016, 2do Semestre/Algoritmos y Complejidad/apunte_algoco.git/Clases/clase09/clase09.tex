\documentclass[english, spanish, fleqn, 10pt]{article}
\usepackage[top = 2.5cm, bottom = 2cm, left = 2.5cm, right = 2.5cm]{geometry}
\usepackage[es-noindentfirst]{babel}
\usepackage[utf8]{inputenc}
\usepackage{amsthm, amsmath, amssymb, amsfonts, environ}
\usepackage{caption}
\usepackage{subcaption} % para las subfigures
\usepackage{mathrsfs}
\usepackage[colorlinks, urlcolor=blue,
pdfauthor={Aldo Berrios Valenzuela}]{hyperref}
\usepackage{fourier}
\usepackage{colortbl}
\usepackage{color}
\usepackage{graphicx}
\usepackage{enumitem}
%\renewcommand{\familydefault}{\sfdefault}
%\setlength{\parindent}{0pt}					%Espacio de sangria
\setlength{\parskip}{0.5mm}						%Interlinado entre párrafos
\usepackage{fancyhdr, blindtext}		%Paquete de encabezado
\usepackage{listings}
\usepackage[framemethod=tikz]{mdframed}

\definecolor{webred}{rgb}{0.75,0,0}
\definecolor{mblue}{rgb}{0.0705,0.345,0.7529}
\definecolor{newmblue}{HTML}{004D99}
\usepackage{longtable, colortbl, array}

\definecolor{superGreenNew}{HTML}{169F01}
\hypersetup{
	colorlinks   = true,
	citecolor    = superGreenNew
}


%==============================
%Modificador de Titulos de secciones, subsecciones, etc
\usepackage[explicit]{titlesec}

\titlespacing\subsubsection{0pt}{8pt plus 4pt minus 2pt}{0pt plus 2pt minus 2pt}


%==================================
%Diseno para insertar codigos de programacion
\definecolor{newGray}{HTML}{F8F8F8}
\definecolor{anotherGray}{HTML}{DDDFFF}
\lstdefinestyle{customc}{
	belowcaptionskip=1\baselineskip,
	breaklines=true,
	backgroundcolor=\color{newGray},
	frame=single,
	rulecolor=\color{anotherGray},
	xleftmargin=\parindent,
	language=bash,
	showstringspaces=false,
	basicstyle=\small\ttfamily,
	keywordstyle=\bfseries\color{green!40!black},
	commentstyle=\itshape\color{purple!40!black},
	identifierstyle=\color{black},
	stringstyle=\color{mblue},
	tabsize=2,
	literate={\ \ }{{\ }}1	
}
\lstset{style=customc}

%==================================
%Macros útiles
\newcommand{\comillas}[1]{``#1''}
\newcommand{\comentarioc}[1]{\texttt{\textcolor{webred}{/* #1 */}}}

\numberwithin{equation}{section}

%=================================
%comandos matemáticos personalizados de rápido acceso
\newcommand{\nparentesis}[1]{\left( #1 \right)}
\newcommand{\llaves}[1]{\left \{ #1 \right \}}
\newcommand{\nabsoluto}[1]{\left| #1 \right|}
\newcommand{\ncorchetes}[1]{\left[ #1 \right]}
%=================================

%cambia el espaciado entre el parrafo y antes del itemize
\setlist[itemize]{itemsep=1mm, topsep=0.5mm}
\setlist[enumerate]{itemsep=1mm, topsep=0.5mm}
%===========================

\theoremstyle{definition}
\newtheorem{teorema}{Teorema}[section]
\newtheorem{definition}{Definición}[section]
\newtheorem{beforeExample}{Ejemplo}[section]
\newtheorem{preposicion}{Preposición}[section]
\newtheorem{corolario}{Corolario}[teorema]

\captionsetup{justification=centering, margin=2.3cm}

\newenvironment{ejemplo}[1][]{\begin{beforeExample}[#1]\renewcommand{\qedsymbol}{$\blacksquare$}}{\qed\end{beforeExample}}

\captionsetup{justification=centering, margin=2.3cm}

% Para poder hacer "quote" con estilo github

\newenvironment{newquote}{\begin{mdframed}[topline=false,
		rightline=false, bottomline=false, linewidth=.3em,backgroundcolor=black!5, linecolor=black!60]\color{black!70}}{\end{mdframed}}
%================================


\usepackage{fancyhdr}

\pagestyle{fancy}
\lhead{INF221\quad Algoritmos y Complejidad}
\rhead{\textit{Clase \#9\quad Código Huffman}}

\author{Aldo Berrios Valenzuela}
\title{INF221 -- Algoritmos y Complejidad\\[.4\baselineskip]Clase \#9\\Código Huffman}
\date{Martes 30 de agosto de 2016}

\usepackage{tikz,ifthen}
\usetikzlibrary{automata,positioning, babel}
\definecolor{navyblue}{RGB}{0,148,222}
\usetikzlibrary{fit}
\tikzset{shorten >=1pt,node distance=1.8cm,on grid, >=stealth, initial text=Inicio,
	every state/.style={ draw=black ,fill=black, minimum size=2mm},
	bend angle=35, every loop/.style={looseness=13}}


\begin{document}
\maketitle
\section{Código Huffman}
El código Huffman es una aplicación muy importante de algoritmo voraz.

\noindent\emph{Ámbito:} compresión de datos.

\noindent\emph{Dado:} Un texto, formado por caracteres. Buscamos codificarlo eficientemente. Cada caracter se codifica en $k$ bits (total $2^k$ caracteres posibles), de texto de largo $n$ usa $nk$ bits.

En texto, las frecuencias son \emph{muy} desiguales. Moby Dick: $117\,194$ veces 'c', $640$ veces 'z'. Nuestro principal objetivo es asignarle codificaciones más largas a los caracteres que menos se repiten y codificaciones más cortas a aquellos que se repiten más.

\noindent\emph{Cuidado:}
\begin{align*}
&a\mapsto 0\\
&b\mapsto 1\\
&c\mapsto 01
\end{align*}
Bajo esta codificación podemos escribir:
\begin{equation}\label{09::PrimerCodigo}
ababc\rightsquigarrow 010101
\end{equation}
Pero también:
\begin{equation}\label{09::SegundoCodigo}
ccc\rightsquigarrow 010101
\end{equation}
¡Se produce ambigüedad entre \eqref{09::PrimerCodigo} y \eqref{09::SegundoCodigo}! 

Condición suficiente para evitarlo: Ningún código es prefijo de otro. Importante porque hace eficiente el decodificar (\comillas{prefix-free code} o \comillas{prefix code}).

\subsubsection{Descripción del problema}
Dada una secuencia $T$ sobre $\Sigma=\llaves{x_1, \ldots, x_n}$, donde $x_i$ aparece con frecuencia $f_i$, construir una función de codificación $C:\Sigma\rightarrow$ cadenas de bits, tal que $C$ es un código prefijo y el número total de bits para representar $T$ se minimiza.

\noindent\emph{Idea:} Representar el código como árbol binario: cada arco se rotula con un bit $0$ o $1$ (digamos si va la izquierda lo rotulamos con $0$, y si va a la derecha lo marcamos con $1$)

Cada carácter rotula una de las hojas, el camino desde la raíz es el código de ese carácter (Figura \ref{09::EjemploArbolHuffman}).
\begin{figure}[!h]
	\centering
	\begin{tikzpicture}
	\node [state] (d1) {};
	\node [below left of = d1] (a) {$a$};
	\node [state, below right of = d1] (d2) {};
	\node [below left of = d2] (b) {$b$};
	\node [below right of = d2] (c) {$c$};
	\path (d1) edge node [above left] {$0$} (a)
	edge node [above right] {$1$} (d2)
	(d2) edge node [above left] {$0$} (b)
	(d2) edge node [above right] {$1$} (c);
	\end{tikzpicture}
	\caption{Los nodos que se encuentran en nivel más alto son los que más se repiten (mayor frecuencia) y las hojas de mayor profundidad son los caracteres que menos se repiten}
	\label{09::EjemploArbolHuffman}
\end{figure}

\begin{itemize}
	\item Es un código prefijo (llegó a la hoja $x_i$, ya no sigue, camino desde la raíz a la hoja es único).
	
	\item En el código óptimo, todo nodo interno tiene dos hijos (podemos \comillas{saltarnos} un nodo interno con un único hijo para crear un código más compacto).
\end{itemize}

\begin{definition}[profundidad]
	La \emph{profundidad} de la hoja $\ell _i$ anotada $d\nparentesis{\ell _i}$, es el largo del camino de la raíz a esa hoja. El caracter $x_i$ queda codificado por $d\nparentesis{x_i}$ bits, el texto completo por:
	\begin{equation*}
	\sum_{i}f_id\nparentesis{x_i}\quad \mathtt{bits}
	\end{equation*}
\end{definition}
Intuitivamente, buscamos letras poco frecuentes a altas profundidades, frecuentes a profundidades bajas. Para ello, armamos el árbol desde las hojas. Vamos uniendo subárboles hasta tener uno solo.

Sea $L=\nparentesis{\ell_1, \ldots, \ell_n}$ el conjunto de hojas para todos los caracteres, y sea $f_i$ la frecuencia de la letra $x_i$. Hallar las dos letras de frecuencia mínima, digamos $x_a$ y $x_b$ con frecuencias $f_a$ y $f_b$. Unir sus hojas en la hoja $\ell_{ab}$ con frecuencia $f_a+f_b$ dando un árbol $R_{ab}$ (Figura \ref{09::RabTree}):
\begin{figure}[!h]
	\centering
	\begin{tikzpicture}
	\node [state] (d0) {};
	\node [above of = d0, yshift = -42] (dummy:ce) {$x_{ab}$};
	\node [below left of = d0] (xa) {$x_a$};
	\node [below right of = d0] (xb) {$x_b$};
	\path (d0) edge node [above left] {$0$} (xa)
	edge node [above right] {$1$} (xb);
	\end{tikzpicture}
	\caption{El nodo $x_{ab}$ representa la unión entre los nodos $x_a$ y $x_b$.}
	\label{09::RabTree}
\end{figure}

Recursivamente resolver el problema con:
\begin{equation}
L=\llaves{\ell _1, \ldots, \ell _n}\setminus\llaves{\ell _a, \ell _b}\cup \llaves{\ell _{ab}}
\end{equation}
y frecuencias ajustadas ($\ell_{ab}\rightsquigarrow f_a+f_b$)

\begin{ejemplo}
	Considere el Cuadro \ref{09::CuadroEjemplo1}.
	\begin{table}[!h]
		\centering
		\begin{tabular}{c|c}
			Símbolo&Frecuencia\\
			\hline
			$a$&9\\
			$b$&4\\
			$c$&2\\
			$d$&15\\
			$e$&3\\
			$f$&17
		\end{tabular}
		\caption{Dada una determinada secuencia de palabras, se detectó que sólo aparecen los símbolos $a, b, c, d, e, f$ con las frecuencias adjuntas.}
		\label{09::CuadroEjemplo1}
	\end{table}
	El algoritmo de Huffman nos pide encontrar los símbolos de menor frecuencia y crear un sub-arbol con ellos. En el cuadro \ref{09::CuadroEjemplo1}, se tiene que los símbolos $c$ y $e$ son los que poseen menor frecuencia. Luego, formamos un sub-arbol con ellos (Figura \ref{09::EjemploArbol1}).
	\begin{figure}[!h]
		\centering
		\begin{tikzpicture}
		\node [state] (ce) {};
		\node [above of = ce, yshift = -42] (dummy:ce) {$ce$};
		\node [below left of = ce] (c) {$c$};
		\node [below right of = ce] (e) {$e$};
		\path (ce) edge node [above left] {$0$} (c)
		edge node [above right] {$1$} (e);
		\end{tikzpicture}
		\caption{El presente árbol tiene como hojas los símbolos que menos se repiten.}
		\label{09::EjemploArbol1}
	\end{figure}
	
	De inmediato, agregamos este \comillas{nodo conjunto} al Cuadro \ref{09::CuadroEjemplo1}, cuya frecuencia es equivalente al peso del árbol de la Figura \ref{09::EjemploArbol1} (suma de las frecuencias de $c$ y $e$). El resultado se logra apreciar en el Cuadro \ref{09::CuadroEjemplo2}.
	\begin{table}[!h]
		\centering
	\begin{tabular}{c|c}
		Símbolo&Frecuencia\\
		\hline
		$a$&9\\
		$b$&4\\
		$d$&15\\
		$f$&17\\
		$ce$&5
	\end{tabular}
	\caption{Se agrega el nodo conjunto $ce$ a la tabla, el cuál es considerado como un nuevo símbolo. La frecuencia de $ce$ es la suma de las frecuencias de $c$ y $e$.}
	\label{09::CuadroEjemplo2}
\end{table}

Repetimos el proceso, es decir, escogemos dos símbolos del Cuadro \ref{09::CuadroEjemplo2} que tienen menor frecuencia y creamos un nuevo sub-árbol. Estos símbolos son $ce$ y $b$. Entonces, la Figura \ref{09::EjemploArbol2} muestra el árbol resultante.
\begin{figure}[!h]
	\centering
	\begin{tikzpicture}
	\node [state] (ce) {};
	\node [above of = ce, yshift = -42] (dummy:ce) {$ce$};
	\node [below left of = ce] (c) {$c$};
	\node [below right of = ce] (e) {$e$};
	\path (ce) edge node [above left] {$0$} (c)
	edge node [above right] {$1$} (e);
	
	\node [above left of = ce, state] (bce) {};
	\node [below left of = bce] (b) {$b$};
	\node [above of = bce, yshift = -42] (dummy:bce) {$bce$};
	\path (bce) edge node [above left] {$0$} (b)
	edge node [above right] {$1$} (ce);
	\end{tikzpicture}
	\caption{El presente árbol tiene como hojas los símbolos que menos se repiten. Tiene un peso de $f_{ce}+f_{b}=5+4=9$}
	\label{09::EjemploArbol2}
\end{figure}

Inmediatamente, reemplazamos los símbolos $b$ y $ce$ del Cuadro \ref{09::CuadroEjemplo2} y lo reemplazamos con $bce$ de frecuencia $f_{bce}=9$. El resultado queda en el Cuadro \ref{09::CuadroEjemplo3}.
\begin{table}[!h]
	\centering
	\begin{tabular}{c|c}
		Símbolo&Frecuencia\\
		\hline
		$a$&9\\
		$d$&15\\
		$f$&17\\
		$bce$&9
	\end{tabular}
	\caption{Se reemplazan los símbolos $b$ y $ce$ del Cuadro \ref{09::CuadroEjemplo2} y se reemplaza con $bce$ de frecuencia $f_{bce}=9$}
	\label{09::CuadroEjemplo3}
\end{table}

Iteramos nuevamente. En el cuadro \ref{09::CuadroEjemplo3} se tiene que los dos símbolos con menor frecuencia son $bce$ y $a$. Entonces, el árbol resultante queda representado por la Figura \ref{09::EjemploArbol3}.
\begin{figure}[!h]
	\centering
	\begin{tikzpicture}
	\node [state] (ce) {};
	\node [above of = ce, yshift = -42] (dummy:ce) {$ce$};
	\node [below left of = ce] (c) {$c$};
	\node [below right of = ce] (e) {$e$};
	\path (ce) edge node [above left] {$0$} (c)
	edge node [above right] {$1$} (e);
	
	\node [above left of = ce, state] (bce) {};
	\node [below left of = bce] (b) {$b$};
	\node [above of = bce, yshift = -42, xshift = 5] (dummy:bce) {$bce$};
	\path (bce) edge node [above left] {$0$} (b)
	edge node [above right] {$1$} (ce);
	
	\node [state, above left of = bce] (abce) {};
	\node [below left of = abce] (a) {$a$};
	\node [above of = abce, yshift = -42] (dummy:abce) {$abce$};
	
	\path (abce) edge node [above right] {$1$} (bce)
	edge node [above left] {$0$} (a);
	\end{tikzpicture}
	\caption{Este árbol tiene un peso de $f_{bce}+f_a=18$.}
	\label{09::EjemploArbol3}
\end{figure}
Luego, quitamos estos símbolos del cuadro \ref{09::CuadroEjemplo3} y los reemplazamos por $abce$. El resultado se puede apreciar mirando el Cuadro \ref{09::CuadroEjemplo4}.
\begin{table}[!h]
	\centering
	\begin{tabular}{c|c}
		Símbolo&Frecuencia\\
		\hline
		$d$&15\\
		$f$&17\\
		$abce$&18
	\end{tabular}
	\caption{Se reemplazan los símbolos $bce$ y $a$ del Cuadro \ref{09::CuadroEjemplo3} y se reemplaza con $abce$ que tiene una frecuencia $f_{abce}=f_{bce}+f_a=18$}
	\label{09::CuadroEjemplo4}
\end{table}

Continuamos iterando. Al observar el cuadro \ref{09::CuadroEjemplo4} vemos que los símbolos con menor frecuencia son $d$ y $f$. Por lo tanto, tomamos estos dos símbolos y creamos un nuevo árbol que los tenga como hojas (Figura \ref{09::EjemploArbol4}).
\begin{figure}[!h]
	\centering
	\begin{tikzpicture}
		\node [state] (df) {};
		\node [below left of = df] (d) {$d$};
		\node [below right of = df] (f) {$f$};
		\node [above of = df, yshift = -42] (dummy:df) {$df$};
		\path (df) edge node [above left] {$0$} (d)
		edge node [above right] {$1$} (f);
	\end{tikzpicture}
	\caption{Las hojas de este árbol corresponden a los símbolos del cuadro \ref{09::CuadroEjemplo4} que tienen menor frecuencia. El peso es de $f_{d}+f_{f}=32$}
	\label{09::EjemploArbol4}
\end{figure}
En seguida, sacamos esos símbolos y lo reemplazamos por $df$. El resultado de hacer esto se puede observar en el cuadro \ref{09::CuadroEjemplo5}.
\begin{table}[!h]
	\centering
	\begin{tabular}{c|c}
		Símbolo&Frecuencia\\
		\hline
		$df$&32\\
		$abce$&18
	\end{tabular}
	\caption{Se reemplazan los símbolos $d$ y $f$ del Cuadro \ref{09::CuadroEjemplo4} por $df$, que tiene una frecuencia de $f_{df}=f_d+f_f=32$}
	\label{09::CuadroEjemplo5}
\end{table}

Hacemos la última iteración, ya que sólo nos quedan dos símbolos. Tomamos los dos últimos símbolos y creamos el árbol final que se logra apreciar en la Figura \ref{09::EjemploArbol5}.
\begin{figure}[!h]
	\centering
	\begin{tikzpicture}
	\node [state] (ce) {};
	\node [above of = ce, yshift = -42] (dummy:ce) {$ce$};
	\node [below left of = ce] (c) {$c$};
	\node [below right of = ce] (e) {$e$};
	\path (ce) edge node [above left] {$0$} (c)
	edge node [above right] {$1$} (e);
	
	\node [above left of = ce, state] (bce) {};
	\node [below left of = bce] (b) {$b$};
	\node [above of = bce, yshift = -42, xshift = 5] (dummy:bce) {$bce$};
	\path (bce) edge node [above left] {$0$} (b)
	edge node [above right] {$1$} (ce);
	
	\node [state, above left of = bce] (abce) {};
	\node [below left of = abce] (a) {$a$};
	\node [above of = abce, yshift = -42, xshift = 10] (dummy:abce) {$abce$};
	
	\path (abce) edge node [above right] {$1$} (bce)
	edge node [above left] {$0$} (a);
	
	\node [state, above left of = abce, xshift = -20, yshift = -10] (final) {};
	
	\node [state, below left of = final, yshift = 10, xshift = -10] (df) {};
	\node [below left of = df] (d) {$d$};
	\node [below right of = df] (f) {$f$};
	\node [above of = df, yshift = -42, xshift = -5] (dummy:df) {$df$};
	\path (df) edge node [above left] {$0$} (d)
	edge node [above right] {$1$} (f);
	
	\path (final) edge node [above left] {$0$} (df)
	edge node [above right] {$1$} (abce);
	
	\end{tikzpicture}
	\caption{Este árbol tiene un peso de $f_{bce}+f_a=18$.}
	\label{09::EjemploArbol5}
\end{figure}

\end{ejemplo}

\begin{proof}[Demostración (algoritmo voraz)]
	Como siempre, para demostrar que este algoritmo es voraz, lo hacemos vía demostrar que:
	\begin{itemize}
		\item \emph{Elección voraz:} Sea $L$ la instancia original (o sea, el texto completo, con la frecuencia de cada símbolo respectivo), sean $\ell_a$, y $\ell_b$ las hojas menos frecuentes. Entonces hay un árbol óptimo que incluye $R_{ab}$.
		
		\begin{proof}
			Sea $R$ un árbol óptimo para $L$. Si $R_{ab}$ es parte de $R$, salimos a carretear. Si el árbol $R_{ab}$ \emph{no} es parte de $R$, sean $\ell_x$, $\ell_y$ dos hojas en $R$ con padre común (hermanos), con $\delta =d\nparentesis{\ell _x}=d\nparentesis{\ell_y}$ máximo.
			
			Entonces, suponiendo que ni $a$ ni $b$ son $x$ ni $y$. Obtenga $R^*$ intercambiando $x\leftrightarrow a$, $y\leftrightarrow b$, $R^*$ contiene $R_{ab}$. Sea $B\nparentesis{R}$ el número de bits usados por el árbol $R$ (importante, que en la demostración nos referiremos a $d$ como el árbol original $R$).
			
			Para el árbol $R^*$; con $d\nparentesis{\,}$ referenciado del árbol original $R$:
			\begin{align*}
			B\nparentesis{R^*}&=B\nparentesis{R}-\nparentesis{f_x+f_y}\delta-f_ad\nparentesis{\ell_a}-f_bd\nparentesis{\ell_b}+\nparentesis{f_a+f_b}\delta+f_xd\nparentesis{\ell_a}+f_yd\nparentesis{\ell_b}\\
			&=B\nparentesis{R}-\underbrace{\nparentesis{f_x-f_a}}_{\geq 0}\underbrace{\nparentesis{\delta+d\nparentesis{\ell_a}}}_{\geq 0}-\underbrace{\nparentesis{f_y-f_b}}_{\geq 0}\underbrace{\nparentesis{\delta+d\nparentesis{\ell_b}}}_{\geq 0}
			\end{align*}
			Pero $R$ es óptimo. Por lo tanto, una contradicción.
		\end{proof}
		
		
		
		
	\end{itemize}
\end{proof}



% Es más probable tener una conversación inteligente antes del tercer cafe de la mañana
\end{document}
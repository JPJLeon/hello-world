\documentclass[english, spanish, fleqn, 10pt]{article}
\usepackage[top = 2.5cm, bottom = 2cm, left = 2.5cm, right = 2.5cm]{geometry}
\usepackage[es-noindentfirst]{babel}
\usepackage[utf8]{inputenc}
\usepackage{amsthm, amsmath, amssymb, amsfonts, environ}
\usepackage{caption}
\usepackage{subcaption} % para las subfigures
\usepackage{mathrsfs}
\usepackage[colorlinks, urlcolor=blue,
pdfauthor={Aldo Berrios Valenzuela}]{hyperref}
\usepackage{fourier}
\usepackage{colortbl}
\usepackage{color}
\usepackage{graphicx}
\usepackage{enumitem}
%\renewcommand{\familydefault}{\sfdefault}
%\setlength{\parindent}{0pt}					%Espacio de sangria
\setlength{\parskip}{0.5mm}						%Interlinado entre párrafos
\usepackage{fancyhdr, blindtext}		%Paquete de encabezado
\usepackage{listings}
\usepackage[framemethod=tikz]{mdframed}

\definecolor{webred}{rgb}{0.75,0,0}
\definecolor{mblue}{rgb}{0.0705,0.345,0.7529}
\definecolor{newmblue}{HTML}{004D99}
\usepackage{longtable, colortbl, array}

\definecolor{superGreenNew}{HTML}{169F01}
\hypersetup{
	colorlinks   = true,
	citecolor    = superGreenNew
}


%==============================
%Modificador de Titulos de secciones, subsecciones, etc
\usepackage[explicit, noindentafter]{titlesec}

\titlespacing\subsubsection{0pt}{8pt plus 4pt minus 2pt}{4pt plus 2pt minus 2pt}


%==================================
%Diseno para insertar codigos de programacion
\definecolor{newGray}{HTML}{F8F8F8}
\definecolor{anotherGray}{HTML}{DDDFFF}
\lstdefinestyle{customc}{
	belowcaptionskip=1\baselineskip,
	breaklines=true,
	backgroundcolor=\color{newGray},
	frame=single,
	rulecolor=\color{anotherGray},
	xleftmargin=\parindent,
	language=bash,
	showstringspaces=false,
	basicstyle=\small\ttfamily,
	keywordstyle=\bfseries\color{red!40!black},
	commentstyle=\itshape\color{purple!40!black},
	identifierstyle=\color{black},
	stringstyle=\color{mblue},
	tabsize=2,
	literate={\ \ }{{\ }}1	
}
\lstset{style=customc}

\renewcommand{\lstlistingname}{Listado}

%==================================
%Macros útiles
\newcommand{\comillas}[1]{``#1''}
\newcommand{\comentarioc}[1]{\texttt{\textcolor{webred}{/* #1 */}}}

\numberwithin{equation}{section}

%=================================
%comandos matemáticos personalizados de rápido acceso
\newcommand{\nparentesis}[1]{\left( #1 \right)}
\newcommand{\llaves}[1]{\left \{ #1 \right \}}
\newcommand{\nabsoluto}[1]{\left| #1 \right|}
\newcommand{\ncorchetes}[1]{\left[ #1 \right]}
%=================================

%cambia el espaciado entre el parrafo y antes del itemize
\setlist[itemize]{itemsep=1mm, topsep=1.5mm}
\setlist[enumerate]{itemsep=1mm, topsep=1.5mm}
%===========================

\theoremstyle{definition}
\newtheorem{teorema}{Teorema}[section]
\newtheorem{definition}{Definición}[section]
\newtheorem{beforeExample}{Ejemplo}[section]
\newtheorem{preposicion}{Preposición}[section]
\newtheorem{corolario}{Corolario}[teorema]

\captionsetup{justification=centering, margin=2.3cm}

\newenvironment{ejemplo}[1][]{\begin{beforeExample}[#1]\renewcommand{\qedsymbol}{$\blacksquare$}}{\qed\end{beforeExample}}

\captionsetup{justification=centering, margin=2.3cm}

% Para poder hacer "quote" con estilo github

\newenvironment{newquote}{\begin{mdframed}[topline=false,
		rightline=false, bottomline=false, linewidth=.3em,backgroundcolor=black!5, linecolor=black!60]\color{black!70}}{\end{mdframed}}
%================================

\renewcommand*{\arraystretch}{1.2}

\usepackage{fancyhdr}

\pagestyle{fancy}
\lhead{INF221\quad Algoritmo y Complejidad}
\rhead{\textit{Clase \#24\quad Algoritmo de Kruskal II}}

\title{INF221 -- Algoritmos y Complejidad\\[.4\baselineskip]
Clase \#24\\
Algoritmo de Kruskal II}

\date{Martes 8 de noviembre de 2016}

\author{Aldo Berrios Valenzuela}


\usepackage{tikz,ifthen}
\usetikzlibrary{automata,positioning, babel}
\definecolor{navyblue}{RGB}{0,148,222}
\usetikzlibrary{fit}
\tikzset{shorten >=1pt,node distance=1.8cm,on grid, >=stealth, initial text=Inicio,
	every state/.style={ draw=black ,fill=black, minimum size=2mm},
	bend angle=35, every loop/.style={looseness=13}}


\begin{document}
\maketitle

\section{Algoritmo de Kruskal (continuación)}
Recordamos algunas observaciones de la clase pasada.
\begin{figure}[!h]
	\centering
	\begin{tikzpicture}
	\node [state] (0) {};
	\node [state, above right of = 0] (1) {};
	\node [state, below right of = 1] (2) {};
	\node [state, above left of = 1] (3) {};
	\node [state, below left of = 3] (4) {};
	\node [state, above right of = 3, yshift = 25, xshift = 25] (5) {};
	\node [state, below of = 5, yshift = -9] (6) {};
	\node [state, below right of = 5, xshift = 25, yshift = -25] (7) {};
	
	\path [->] (0) edge (1)
	(2) edge (1)
	(1) edge (3)
	(4) edge (3)
	(3) edge (5)
	(6) edge (5)
	(7) edge (5);
	
	\path [->, red] (1) edge [bend right = 7] node [xshift = 32, yshift = -30] {path compression} (5);
	\end{tikzpicture}
	\caption{En path compression cambia el puntero al padre original de un nodo por la raíz del árbol más grande (véase el arco rojo).}
\end{figure}


\begin{itemize}
	\item Si el nodo es raíz y tiene rank $r$, su árbol tiene a lo menos $2^r$ nodos. (si el nodo es una raíz, entonces el número de descendientes es $2^r$)
	
	\item A lo largo de un camino a la raíz, rank aumenta.
\end{itemize}

\begin{definition}
	Sea $g:\mathbb{R} \rightarrow \mathbb{R}$ tal que para $x > 1$, $g\nparentesis{x} < x$. Definimos:
	\begin{equation}
	g^*\nparentesis{x} = \begin{cases}
	0 & x \leq 1\\
	1 + g^* \nparentesis{g\nparentesis{x}} & x > 1
	\end{cases}
	\end{equation}
	En el fondo, $g^*$ nos permite obtener cuántas veces debo aplicar la función $g$ para obtener un resultado menor o igual a 1.
\end{definition}
Análisis clásico (Hopcroft y Ullman, Tarjan) es lioso\ldots Seidel y Sharir dan un análisis simple (para un valor adecuado de \comillas{simple})


\noindent\emph{Idea general:}
\begin{itemize}
	\item Reordenar operaciones para simplificar análisis. Todos los \emph{unión} al comienzo, todos los \emph{final} después $\Rightarrow$ ya no llegamos a la raíz, nueva operación \emph{compress} que toma un rango de nodos.
	
	El costo de la secuencia no cambia. Solo durante los \emph{union} cambia el rank.
	
	\item Dividir el bosque \comillas{alto} (rank alto) y \comillas{bajo} (rank bajo), analizar por separado.
	
	Para impedir que se mezcle \comillas{la chusma}, operación \emph{shatter}: Los nodos en el camino quedan de raíces.
	
	\emph{Idea:}
	\begin{figure}[!h]
		\centering
		\begin{tikzpicture}[shorten >=1pt,node distance=1cm,on grid, >=stealth]
			\node [state] (0) {};
			\node [state, below left of = 0] (1) {};
			\node [state, below left of = 1] (2) {};
			\node [state, below left of = 2] (3) {};
			\node [state, below left of = 3] (4) {};
			\node [state, below left of = 4] (5) {};
			\node [state, below left of = 5] (6) {};
			
			\path (0) edge (1)
			(1) edge (2)
			(2) edge (3)
			(3) edge (4)
			(4) edge (5)
			(5) edge (6);
			
			\draw [dashed] (-5, -1.75) -- ++ (11, 0) node [left, pos = 0] {$S$};
			
			\draw [->, smooth, tension=0.8, red!70!black!90] plot coordinates{(0.5, -1) (1.0, -0.8) (1.5, -1.1) (1.9, -1) (2.3, -1)} node [below, xshift = -26, yshift = -5] {compress};
			
			\draw [->, smooth, tension=0.8, red!70!black!90] plot coordinates{(0.5-1.2, -1-2) (1.0-1.2, -0.8-2) (1.5-1.2, -1.1-2) (1.9-1.2, -1-2) (2.3-1.2, -1-2)} node [below, xshift = -26, yshift = -5] {shatter};
			
			\node [state] at (4, -0.5) (10) {};
			\node [state, below left of = 10] (11) {};
			\node [state, below right of = 10] (12) {};
			\path (10) edge (11)
			(10) edge (12);
			
			
			\node [state] at (2, -2.5) (20) {};
			\node [state, right of = 20] (21) {};
			\node [state, right of = 21] (22) {};
			\node [state, right of = 22] (23) {};
			
			\path [->] (20) edge [loop above] ()
			(21) edge [loop above] ()
			(22) edge [loop above] ()
			(23) edge [loop above] ();
			
			\node [below left of = 6, yshift = 12, xshift = 12] (30) {$v$};
			\node [above right of = 0, yshift = -12, xshift = -12] (31) {$u$};
			
		\end{tikzpicture}
		\caption{Todos los nodos cuyo rango es menor o igual que $S$, se les aplica la operación \emph{shatter}. Por el contrario, aquellos que sean mayores se les aplica \emph{compress}. Del bosque total $\mathcal{F}$, se originan dos bosques $\mathcal{F}_-$ y $\mathcal{F}_+$ respectivamente.}
		\label{24::Shatter:Compress}
	\end{figure}
	
\end{itemize}
Sea $T\nparentesis{m, n, r}$ el costo máximo en $m$ operaciones compress en un bosque de $n$ nodos con rank a lo más $r$. Una cota trivial es el teorema \ref{24::Teo::Acotacion}.
\begin{teorema}\label{24::Teo::Acotacion}
	$T\nparentesis{m, n, r} \leq nr$.
\end{teorema}

\begin{proof}
	Cada nodo puede cambiar a lo más $r$ veces de padre.
\end{proof}

Sea $\mathcal{F}$ un bosque de rank máximo $r$, con $n$ nodos, $C$ una secuencia de $m$ operaciones compress, $T\nparentesis{\mathcal{F}, C}$ el número total de asignaciones \emph{parent} al aplicar $C$ sobre $\mathcal{F}$. Dividimos $\mathcal{F}$ en $\mathcal{F}_-$ (con nodos de rank $\leq s$) y $\mathcal{F}_{+}$ (nodos de rank $> s$). Como rank aumenta al seguir punteros a padres, el padre de un nodo alto es a su vez alto. La secuencia $C$ sobre $\mathcal{F}$ puede descomponerse en secuencias $C_{+}$ sobre $\mathcal{F}_{+}$ y $C_{-}$ sobre $\mathcal{F}_{-}$, del mismo costo total, usando:
\begin{lstlisting}[escapeinside={(*}{*)}]
procedure compress((*$u$, $v$, $\mathcal{F}$*))
	if rank[(*$u$*)](*$> s$*) then
		compress((*$u$, $v$, $\mathcal{F}_{+}$*))
	else if rank[(*$v$*)](*$\leq s$*) then
		compress((*$u$, $v$, $\mathcal{F}_{-}$*))
	else
		(*$z \leftarrow u$*)
		while rank[parent(*$_{\mathcal{F}}$*)[(*$z$*)]](*$\leq s$*) do
			(*$z \leftarrow $*) parent[(*$z$*)]
		end
		compress(parent(*$_{\mathcal{F}}$*)[(*$z$*)], (*$v$*), (*$\mathcal{F}_{+}$*))
		shatter((*$u$, $z$, $\mathcal{F}_{-}$*))
		parent[(*$z$*)] (*$\leftarrow$*) (*$z$*)
\end{lstlisting}
En la figura \ref{24::Shatter:Compress} partimos el algoritmo desde el nodo $u$ y comprimimos hasta el nodo $v$.


\end{document}


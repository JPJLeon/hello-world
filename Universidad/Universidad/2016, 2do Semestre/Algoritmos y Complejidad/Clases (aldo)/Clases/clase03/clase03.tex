\documentclass[english, spanish, fleqn, 10pt]{article}
\usepackage[top = 2.5cm, bottom = 2cm, left = 2.5cm, right = 2.5cm]{geometry}
\usepackage[es-noindentfirst]{babel}
\usepackage[utf8]{inputenc}
\usepackage{amsthm, amsmath, amssymb, amsfonts, environ}
\usepackage{caption}
\usepackage{subcaption} % para las subfigures
\usepackage{mathrsfs}
\usepackage[colorlinks, urlcolor=blue,
pdfauthor={Aldo Berrios Valenzuela}]{hyperref}
\usepackage{fourier}
\usepackage{colortbl}
\usepackage{color}
\usepackage{graphicx}
\usepackage{enumitem}
%\renewcommand{\familydefault}{\sfdefault}
%\setlength{\parindent}{0pt}					%Espacio de sangria
\setlength{\parskip}{0.5mm}						%Interlinado entre párrafos
\usepackage{fancyhdr, blindtext}		%Paquete de encabezado
\usepackage{listings}
\usepackage[framemethod=tikz]{mdframed}

\definecolor{webred}{rgb}{0.75,0,0}
\definecolor{mblue}{rgb}{0.0705,0.345,0.7529}
\definecolor{newmblue}{HTML}{004D99}
\usepackage{longtable, colortbl, array}

\definecolor{superGreenNew}{HTML}{169F01}
\hypersetup{
	colorlinks   = true,
	citecolor    = superGreenNew
}


%==============================
%Modificador de Titulos de secciones, subsecciones, etc
\usepackage[explicit, noindentafter]{titlesec}

\titlespacing\subsubsection{0pt}{8pt plus 4pt minus 2pt}{4pt plus 2pt minus 2pt}


%==================================
%Diseno para insertar codigos de programacion
\definecolor{newGray}{HTML}{F8F8F8}
\definecolor{anotherGray}{HTML}{DDDFFF}
\lstdefinestyle{customc}{
	belowcaptionskip=1\baselineskip,
	breaklines=true,
	backgroundcolor=\color{newGray},
	frame=single,
	rulecolor=\color{anotherGray},
	xleftmargin=\parindent,
	language=bash,
	showstringspaces=false,
	basicstyle=\small\ttfamily,
	keywordstyle=\bfseries\color{green!40!black},
	commentstyle=\itshape\color{purple!40!black},
	identifierstyle=\color{black},
	stringstyle=\color{mblue},
	tabsize=2,
	literate={\ \ }{{\ }}1	
}
\lstset{style=customc}

%==================================
%Macros útiles
\newcommand{\comillas}[1]{``#1''}
\newcommand{\comentarioc}[1]{\texttt{\textcolor{webred}{/* #1 */}}}

\numberwithin{equation}{section}

%=================================
%comandos matemáticos personalizados de rápido acceso
\newcommand{\nparentesis}[1]{\left( #1 \right)}
\newcommand{\llaves}[1]{\left \{ #1 \right \}}
\newcommand{\nabsoluto}[1]{\left| #1 \right|}
\newcommand{\ncorchetes}[1]{\left[ #1 \right]}
%=================================

%cambia el espaciado entre el parrafo y antes del itemize
\setlist[itemize]{itemsep=1mm, topsep=1.5mm}
\setlist[enumerate]{itemsep=1mm, topsep=1.5mm}
%===========================

\theoremstyle{definition}
\newtheorem{teorema}{Teorema}[section]
\newtheorem{definition}{Definición}[section]
\newtheorem{beforeExample}{Ejemplo}[section]
\newtheorem{preposicion}{Preposición}[section]
\newtheorem{corolario}{Corolario}[teorema]

\captionsetup{justification=centering, margin=2.3cm}

\newenvironment{ejemplo}[1][]{\begin{beforeExample}[#1]\renewcommand{\qedsymbol}{$\blacksquare$}}{\qed\end{beforeExample}}

\captionsetup{justification=centering, margin=2.3cm}

% Para poder hacer "quote" con estilo github

\newenvironment{newquote}{\begin{mdframed}[topline=false,
		rightline=false, bottomline=false, linewidth=.3em,backgroundcolor=black!5, linecolor=black!60]\color{black!70}}{\end{mdframed}}
%================================


\usepackage{fancyhdr}

\pagestyle{fancy}
\lhead{INF221\quad Informática Teórica}
\rhead{\textit{Clase \#3\quad Iteración de Punto Fijo}}

\author{Aldo Berrios Valenzuela}
\title{INF221 -- Informática Teórica\\[.4\baselineskip]Clase \#3\\\textit{Iteración de Punto Fijo}}
\date{Martes 9 de Agosto de 2016}

\usepackage{tikz,ifthen}
\usetikzlibrary{automata,positioning, babel}
\definecolor{navyblue}{RGB}{0,148,222}
\usetikzlibrary{fit}
\tikzset{shorten >=1pt,node distance=3cm,on grid, >=stealth, initial text=Inicio,
	every state/.style={ draw=navyblue!50,very thick,fill=navyblue!20, minimum size=6mm},
	bend angle=35, every loop/.style={looseness=13}}
\usetikzlibrary{calc}
\newcommand\numberthis{\addtocounter{equation}{1}\tag{\theequation}}

\begin{document}
\maketitle
\section{Iteración de Punto Fijo}
\begin{definition}
	Sea $g\nparentesis{x}$ una función. Un \emph{punto fijo} de $g$ es $x^*$ tal que $x^*=g\nparentesis{x^*}$.
\end{definition}

\begin{teorema}[Punto fijo de Brouwer, 1 dimensión]
	Sea $g:\ncorchetes{a, b}\rightarrow\ncorchetes{a, b}$ una función continua. Entonces $g$ tiene \underline{al menos} un punto fijo.
\end{teorema}
\begin{proof}
	Por definición de $g$, sabemos:
	\begin{equation*}
	a\leq g\nparentesis{x} \leq b
	\end{equation*}
	En particular, $a\leq g\nparentesis{a}$ y $g\nparentesis{b}\leq b$. Si alguna vez se cumple con igualdad estamos listos.
	
	Supongamos entonces $a<g\nparentesis{a}$ y $g\nparentesis{b}<b$. Consideremos $f\nparentesis{x}=g\nparentesis{x}-x$. Vemos que $f$ es continua, y $f\nparentesis{a}=g\nparentesis{a}-a>0$, $f\nparentesis{b}=g\nparentesis{b}-b<0$. Por el \textit{teorema del valor medio}, hay $x^*\in\ncorchetes{a, b}$ tal que $f\nparentesis{x^*}=0\rightsquigarrow g\nparentesis{x^*}=0$.
\end{proof}

\begin{definition}
	Sea $g:\ncorchetes{a, b}\rightarrow \ncorchetes{a, b}$. Se dice que $g$ es una \emph{contracción} si existe $L$, $0<L<1$, tal que para todo $x, y\in\ncorchetes{a, b}$ es:
	\begin{equation}
	\nabsoluto{g\nparentesis{x}-g\nparentesis{y}}\leq L\nabsoluto{x-y}
	\end{equation}
	(condición de Lipschitz, $L$ es la constante de Lipschitz)
\end{definition}

\begin{teorema}[Contraction Mapping]
	Suponga que $g:\ncorchetes{a, b}\rightarrow \ncorchetes{a, b}$ es continua y cumple la condición de Lipschitz. Entonces tiene un único punto fijo en $\ncorchetes{a, b}$.
\end{teorema}

\begin{proof}
	Por Brouwer, $g$ tiene al menos un punto fijo. Para demostrar que es único, supongamos puntos fijos $c_1$, $c_2$:
	\begin{equation}
	\nabsoluto{c_1-c_2}=\nabsoluto{g\nparentesis{c_1}-g\nparentesis{c_2}}\leq L\nabsoluto{c_1-c_2}
	\end{equation}
	Como $0< L < 1$, esto es posible solo si $c_1=c_2$. Esto es algo bastante obvio, ya que en el fondo tomamos un área más grande y en cada iteración la vamos reduciendo hasta tal punto que $c_1=c_2$ (Figura \ref{03::ContractionMapping})
	\begin{figure}[!h]
		\centering
		\begin{tikzpicture}
		%[smooth, tension=0.8, smooth cycle]
		%\draw plot [smooth, tension=0.8, smooth cycle] coordinates {(0, 0) (-0, -0.4) (0.2, -.9)  (0, -1.5) (.2, -1.9) (0.8, -2.2) (1.6, -2) (2, -2.1) (2.5, -2) (2.9, -2) (3.2, -1.8) (3.2, -1) (3.3, -0.2) (3.4, 0.5) (3.1, 1.2) (3.1, 2) (2.4, 2.2) (1.6, 2.3) (1, 2.1) (.5, 1.8) (.1, 1.7) (-.3, 1.4) (-.1, .8)};
		
		\draw plot [smooth, tension=0.8, smooth cycle] coordinates {(0.8, -1) (1.5, -1.3) (2.1, -1.2) (2.4, -1) (2.5, -0.4) (2.4, 0.4) (2.5, 1) (2.3, 1.4) (1.6, 1.3) (1, 1) (0.6, 0.7) (0.9, 0)};
		
		\draw plot [smooth, tension=0.8, smooth cycle] coordinates {(0.5+4, -1) (1.5+4, -1.3) (2.1+4, -1.2) (2.4+4, -1) (2.5+4, -0.4) (2.4+4, 0.4) (2.5+4, 1) (2.3+4, 1.4) (1.6+4, 1.3) (1+4, 1) (0.6+4, 0.9) (0.6+4, 0)};
		
		\draw [red!90!black!90] plot [smooth, tension = 0.8, smooth cycle] coordinates {(4.7, 0) (5.0, 0.5) (5.6, 0.6) (6.1, 0.6) (6.3, -0.5) (5.9, -0.7) (5.6, -0.5) (5.2, -0.7)};
		
		
		\node at (1.4, 0.5) (x1) {\color{blue}$c_1$};
		
		\node at (1.6, -0.9) (x2) {\color{blue}$c_2$};
		
		\node at (1.8+4, -0.3) (x3) {$g\nparentesis{c_2}$};
		
		\node at (1.3+4, .20) (x4) {$g\nparentesis{c_1}$};
		
		\draw [->, blue] plot [smooth, tension = 0.8] coordinates {(1.6, 0.55) (3.7, 0.7) (0.95+4, .22)} node [yshift = 20pt, xshift = -50pt] (g1) {$g$};
		
		\draw [->, blue] plot [smooth, tension = 0.8] coordinates {(1.8, -0.85) (4, -0.9) (1.4+4, -0.35)} node [yshift = -25pt, xshift = -60pt] (g1) {$g$};
		
		%\draw [blue, ->] plot [smooth, tension=0.8] coordinates {(2.1, -.3) (3.4, -0.1) (4.2, 0.2)};
		
		
		
		
		\end{tikzpicture}
		\caption{Al iterar el proceso iremos acotando cada vez el área hasta converger en un punto. Cabe destacar, que cada vez que hacemos este mapeo, el área se contrae un factor $L$.}
		\label{03::ContractionMapping}
	\end{figure}
\end{proof}

Definamos la secuencia:
\begin{equation}
x_{n+1}=g\nparentesis{x_n}
\end{equation}
Si $g$ es una \textit{contracción} en $\ncorchetes{a, b}$, la secuencia converge al punto fijo $x^*$ de $g$ en $\ncorchetes{a, b}$.

De partida, si
\begin{equation*}
\lim_{n\rightarrow \infty}x_n=x^*
\end{equation*}
existe, es un punto fijo de $g$. Si $x_0\in\ncorchetes{a, b}$, consideremos:
\begin{align*}
	\nabsoluto{x_{n+1}-x^*}=\nabsoluto{g\nparentesis{x_n}-g\nparentesis{x^*}}&\leq L\nabsoluto{x_n-x^*}\numberthis\label{03::DesigualdadContractionInitial}\\
	&\leq \ldots\\
	&\leq L^{n+1}\nabsoluto{x_0-x^*}\numberthis\label{03::DesigualdadContraction}
\end{align*}
Como $\nabsoluto{L} < 1$, $L^n\rightarrow 0$, y el lado izquierdo también $\rightarrow 0$ (para llegar a \eqref{03::DesigualdadContraction} sólo tenemos que desarrollar paso a paso \eqref{03::DesigualdadContractionInitial}).

Si queremos llegar a $\nabsoluto{x_n-x^*}\leq \epsilon$.

Sabemos que $\nabsoluto{x_n-x^*}\leq L^{n}\nabsoluto{x_0-x^*}$. Queremos deshacernos del $x^*$ desconocido al lado derecho:
\begin{align*}
	\nabsoluto{x_0-x^*}&=\nabsoluto{\nparentesis{x_0-x_1}+\nparentesis{x_1-x^*}}\\
	&\leq \nabsoluto{x_0-x_1}+\nabsoluto{x_1-x^*}\\
	&\leq \nabsoluto{x_0-x_1}+L\nabsoluto{x_0-x^*}\quad \Bigg / -L\nabsoluto{x_0-x^*}\\
	\nparentesis{1-L}\nabsoluto{x_0-x^*}&\leq \nabsoluto{x_1-x_0}\\
	\nabsoluto{x_0-x^*}&\leq \dfrac{\nabsoluto{x_1-x_0}}{1-L}
\end{align*}
O sea:
\begin{equation}
\nabsoluto{x_n-x^*}\leq \dfrac{L^n}{1-L}\nabsoluto{x_1-x_0}
\end{equation}
Queremos $\nabsoluto{x_n-x^*}\leq \epsilon$:
\begin{align*}
\epsilon&\leq \dfrac{L^n}{1-L}\nabsoluto{x_1-x_0}\\
L^n&\geq \dfrac{\epsilon\nparentesis{1-L}}{\nabsoluto{x_1-x_0}}\\
n&\geq \dfrac{1}{\ln\nparentesis{L}}\cdot \ln\nparentesis{\dfrac{\epsilon\nparentesis{1-L}}{\nabsoluto{x_1-x_0}}}
\end{align*}

No hemos supuesto $g$ diferenciable, pero en casos de interés lo es.

La condición de Lipschitz es:
\begin{align*}
	\dfrac{\nabsoluto{g\nparentesis{x}-g\nparentesis{y}}}{\nabsoluto{x-y}}&\leq L\\
	\nabsoluto{\dfrac{g\nparentesis{x}-g\nparentesis{y}}{x-y}}&\leq L
\end{align*}
Por el teorema del valor intermedio\footnote{\url{https://en.wikipedia.org/wiki/Mean_value_theorem}}:
\begin{equation}
\dfrac{g\nparentesis{x}-g\nparentesis{y}}{x-y}=g'\nparentesis{\zeta}, \qquad x\leq \zeta \leq y
\end{equation}
por lo tanto, $\nabsoluto{g'\nparentesis{\zeta}}\leq L$ para $\zeta \in \ncorchetes{a, b}$ es condición suficiente para Lipschitz, se aplica el teorema de contraction mapping hay un único punto fijo en $\ncorchetes{a, b}$ y $x_{n+1}=g\nparentesis{x_n}$ converge.

\paragraph{Importante:} No buscamos encontrar $\zeta$, sólo demostrar que existe. Y por favor, $\zeta$ se lee \comillas{zeta}.


\end{document}
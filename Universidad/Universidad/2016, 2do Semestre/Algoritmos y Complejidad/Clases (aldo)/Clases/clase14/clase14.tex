\documentclass[english, spanish, fleqn, 10pt]{article}
\usepackage[top = 2.5cm, bottom = 2cm, left = 2.5cm, right = 2.5cm]{geometry}
\usepackage[es-noindentfirst]{babel}
\usepackage[utf8]{inputenc}
\usepackage{amsthm, amsmath, amssymb, amsfonts, environ}
\usepackage{caption}
\usepackage{subcaption} % para las subfigures
\usepackage{mathrsfs}
\usepackage[colorlinks, urlcolor=blue,
pdfauthor={Aldo Berrios Valenzuela}]{hyperref}
\usepackage{fourier}
\usepackage{colortbl}
\usepackage{color}
\usepackage{graphicx}
\usepackage{enumitem}
%\renewcommand{\familydefault}{\sfdefault}
%\setlength{\parindent}{0pt}					%Espacio de sangria
\setlength{\parskip}{0.5mm}						%Interlinado entre párrafos
\usepackage{fancyhdr, blindtext}		%Paquete de encabezado
\usepackage{listings}
\usepackage[framemethod=tikz]{mdframed}

\definecolor{webred}{rgb}{0.75,0,0}
\definecolor{mblue}{rgb}{0.0705,0.345,0.7529}
\definecolor{newmblue}{HTML}{004D99}
\usepackage{longtable, colortbl, array}

\definecolor{superGreenNew}{HTML}{169F01}
\hypersetup{
	colorlinks   = true,
	citecolor    = superGreenNew
}


%==============================
%Modificador de Titulos de secciones, subsecciones, etc
\usepackage[explicit, noindentafter]{titlesec}

\titlespacing\subsubsection{0pt}{8pt plus 4pt minus 2pt}{4pt plus 2pt minus 2pt}


%==================================
%Diseno para insertar codigos de programacion
\definecolor{newGray}{HTML}{F8F8F8}
\definecolor{anotherGray}{HTML}{DDDFFF}
\lstdefinestyle{customc}{
	belowcaptionskip=1\baselineskip,
	breaklines=true,
	backgroundcolor=\color{newGray},
	frame=single,
	rulecolor=\color{anotherGray},
	xleftmargin=\parindent,
	language=bash,
	showstringspaces=false,
	basicstyle=\small\ttfamily,
	keywordstyle=\bfseries\color{green!40!black},
	commentstyle=\itshape\color{purple!40!black},
	identifierstyle=\color{black},
	stringstyle=\color{mblue},
	tabsize=2,
	literate={\ \ }{{\ }}1	
}
\lstset{style=customc}

%==================================
%Macros útiles
\newcommand{\comillas}[1]{``#1''}
\newcommand{\ccomment}[1]{\texttt{\textcolor{webred}{/* #1 */}}}

\numberwithin{equation}{section}

%=================================
%comandos matemáticos personalizados de rápido acceso
\newcommand{\nparentesis}[1]{\left( #1 \right)}
\newcommand{\llaves}[1]{\left \{ #1 \right \}}
\newcommand{\nabsoluto}[1]{\left| #1 \right|}
\newcommand{\ncorchetes}[1]{\left[ #1 \right]}
%=================================

%cambia el espaciado entre el parrafo y antes del itemize
\setlist[itemize]{itemsep=1mm, topsep=1.5mm}
\setlist[enumerate]{itemsep=1mm, topsep=1.5mm}
%===========================

\theoremstyle{definition}
\newtheorem{teorema}{Teorema}[section]
\newtheorem{definition}{Definición}[section]
\newtheorem{beforeExample}{Ejemplo}[section]
\newtheorem{preposicion}{Preposición}[section]
\newtheorem{corolario}{Corolario}[teorema]

\captionsetup{justification=centering, margin=2.3cm}

\newenvironment{ejemplo}[1][]{\begin{beforeExample}[#1]\renewcommand{\qedsymbol}{$\blacksquare$}}{\qed\end{beforeExample}}

\captionsetup{justification=centering, margin=2.3cm}

% Para poder hacer "quote" con estilo github

\newenvironment{newquote}{\begin{mdframed}[topline=false,
		rightline=false, bottomline=false, linewidth=.3em,backgroundcolor=black!5, linecolor=black!60]\color{black!70}}{\end{mdframed}}
%================================

\renewcommand*{\arraystretch}{1.2}

\usepackage{fancyhdr}

\pagestyle{fancy}
\lhead{INF221\quad Algoritmos y Complejidad}
\rhead{\textit{Clase \#14\quad Backtracking}}

\title{INF221 -- Algoritmos y Complejidad\\[.4\baselineskip]Clase \#14\\Backtracking}
\author{Aldo Berrios Valenzuela}
\date{Miércoles 21 de septiembre de 2016}

\begin{document}
\maketitle
\section{Backtracking}
Otra estrategia recursiva\ldots La idea es ir construyendo la solución incrementalmente, explorando distintas ramas y volviendo atrás (backtrack) si  resulta que un camino es sin salida.

\begin{ejemplo}[Un clásico]
	8 reinas.
	\begin{center}
		\ccomment{Dibujo}
	\end{center}
	Pasos:
	\begin{itemize}
		\item Reducir espacio de búsqueda.
		\begin{itemize}
			\item [$\rightsquigarrow$] Si son 8 reinas, hay una por columna.
		\end{itemize}
		Por lo tanto, llenar por columnas, solución (parcial) indica las filas de las reinas ya ubicadas.
		
		\item Ordenar avance.
		
		\item Subproblemas similares.
	\end{itemize}
	
	En este caso:
	\begin{itemize}
		\item Ubicar reina en columna 1, 2, \ldots
		
		\item Registrar filas libres (por omitir ocupadas al ubicar la siguiente reina)
		
		\item Registrar diagonales libres.
		
		Reina en $i, j$:
		\begin{align*}
		y-j&=1\cdot \nparentesis{x-i}\\
		i-j&=x-y\rightsquigarrow x-y\mathit{cte}\\
		y-j&=-1\cdot \nparentesis{x-i}\rightsquigarrow x+y\mathit{cte}
		\end{align*}
	\end{itemize}
	
	Elegimos Python (!), en Python los arreglos tienen índices desde $0$, rango de $i$, $h$ es $0\ldots 7$. ¿Rangos para?:
	\begin{itemize}
		\item $i-j$: $-7\ldots 7\rightsquigarrow$ sumar 7 para llevar al rango $0\ldots 14$.
		\item $i+j$: $0\ldots 14$
	\end{itemize}
	Por lo tanto, la estructura registra una solución parcial es:
	\begin{lstlisting}
/* Pero que clase de python es este? */
int queen[8];				/* Fila de reinas */
int n; 							/* N de reinas ubicadas */
bool rfree[8];			/* Filas libres */
bool dufree[14], ddfree[14];
	\end{lstlisting}
	\ccomment{Colocar un dibujo con algunas reinas en el tablero para mostrar más o menos cómo funciona el algoritmo.}
	
	Ahora si en Python:
	\begin{lstlisting}[language = Python]
def solve(n):
	if n == 8:
		# Escribir solucion, ...
		return 
	else:
		for k in range(8):
			if rfree[k] and ddfree[n+k] and dufree[n+7-k]:
				rfree[k] = ddfree[n+k] = dufree[n+7-k]	= False
				queen[n] = k
				solve(n+1)
				rfree[k] = ddfree[n+k] = dufree[n+7-k] = True
	\end{lstlisting}
\end{ejemplo}

\ccomment{Sudoku es otro ejemplo de un problema de Backtracking}


\end{document}
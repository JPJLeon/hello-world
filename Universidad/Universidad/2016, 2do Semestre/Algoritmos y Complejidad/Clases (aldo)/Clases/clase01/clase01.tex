\documentclass[english, spanish, fleqn, 10pt]{article}
\usepackage[top = 2.5cm, bottom = 2cm, left = 3.0cm, right = 2.5cm]{geometry}
\usepackage{babel}
\usepackage[utf8]{inputenc}
\usepackage{amsthm, amsmath, amssymb, amsfonts, environ}
\usepackage{caption}
\usepackage{subcaption} % para las subfigures
\usepackage{mathrsfs}
\usepackage[colorlinks, urlcolor=blue]{hyperref}
\usepackage{fourier}
\usepackage{colortbl}
\usepackage{color}
\usepackage{graphicx}
\usepackage{enumitem}
%\renewcommand{\familydefault}{\sfdefault}
%\setlength{\parindent}{0pt}					%Espacio de sangria
\setlength{\parskip}{2mm}						%Interlinado entre párrafos
\usepackage{fancyhdr, blindtext}		%Paquete de encabezado
\usepackage{listings}
\usepackage[framemethod=tikz]{mdframed}

\definecolor{webred}{rgb}{0.75,0,0}
\definecolor{mblue}{rgb}{0.0705,0.345,0.7529}
\definecolor{newmblue}{HTML}{004D99}
\usepackage{longtable, colortbl, array}

\definecolor{superGreenNew}{HTML}{169F01}
\hypersetup{
	colorlinks   = true,
	citecolor    = superGreenNew
}


%==============================
%Modificador de Titulos de secciones, subsecciones, etc
\usepackage[explicit]{titlesec}
\titleformat{\section}
{\normalfont\sffamily\Large\bfseries}
{\llap{\setlength{\arrayrulewidth}{.11em}\begin{tabular}{c|}
			\hline
			\hspace{1em}\thesection.
		\end{tabular} \hspace{0.6em}}}
{0em}{#1}

\titleformat{\subsection}
{\normalfont\sffamily\large\bfseries}
{\llap{{\makebox[3em][r]  {\thesubsection.}}\hspace{1em}}}
{0em}{#1}

\titleformat{\subsubsection}
{\bfseries\sffamily\raggedright}
{\llap{{\makebox[3em][r]  {\thesubsubsection.}}\hspace{1em}}}
{0em}{#1}

\titlespacing\subsubsection{0pt}{8pt plus 4pt minus 2pt}{0pt plus 2pt minus 2pt}

%==================================
%Diseno para insertar codigos de programacion
\definecolor{newGray}{HTML}{F8F8F8}
\definecolor{anotherGray}{HTML}{DDDFFF}
\lstdefinestyle{customc}{
	belowcaptionskip=1\baselineskip,
	breaklines=true,
	backgroundcolor=\color{newGray},
	frame=single,
	rulecolor=\color{anotherGray},
	xleftmargin=\parindent,
	language=bash,
	showstringspaces=false,
	basicstyle=\ttfamily,
	keywordstyle=\bfseries\color{green!40!black},
	commentstyle=\itshape\color{purple!40!black},
	identifierstyle=\color{black},
	stringstyle=\color{mblue},
	tabsize=2,
	literate={\ \ }{{\ }}1	
}
\lstset{style=customc}

%==================================
%Macros útiles
\newcommand{\comillas}[1]{``#1''}
\newcommand{\comentarioc}[1]{\texttt{\textcolor{webred}{/* #1 */}}}
\newcommand{\definicion}[1]{\textit{\underline{\smash{#1}}}}

\numberwithin{equation}{section}

%=================================
%comandos matemáticos personalizados de rápido acceso
\newcommand{\nparentesis}[1]{\left( #1 \right)}
\newcommand{\llaves}[1]{\left \{ #1 \right \}}
\newcommand{\nabsoluto}[1]{\left| #1 \right|}
\newcommand{\ncorchetes}[1]{\left[ #1 \right]}
%=================================

%cambia el espaciado entre el parrafo y antes del itemize
\setlist[itemize]{itemsep=1mm, topsep=0.5mm}
\setlist[enumerate]{itemsep=1mm, topsep=0.5mm}
%===========================

\theoremstyle{definition}
\newtheorem{teorema}{Teorema}[section]
\newtheorem{definition}{Definición}[section]
\newtheorem{ejemplo}{Ejemplo}[section]
\newtheorem{preposicion}{Preposición}[section]
\newtheorem{corolario}{Corolario}[teorema]

\captionsetup{justification=centering, margin=2.3cm}

% Para poder hacer "quote" con estilo github

\definecolor{newGreenNew}{HTML}{F2F9EC}
\newcolumntype{?}{!{\color[HTML]{009933}\vrule width 4pt}}
\newcolumntype{b}{>{\columncolor{newGreenNew}}p{.967\textwidth}}
\newenvironment{newquote}{\renewcommand{\arraystretch}{1.8}\color[HTML]{009933}\begin{longtable}{?b}}{\end{longtable}\renewcommand{\arraystretch}{1.3}}
%================================


\usepackage{fancyhdr}

\pagestyle{fancy}
\lhead{INF221\quad Algoritmos y Complejidad}
\rhead{\textit{Clase \#1\quad Ceros de una Función}}
\renewcommand{\headrulewidth}{2pt}

\title{INF221 -- Algoritmos y Complejidad\\[.4\baselineskip]Clase \#1\\\textit{Ceros de una Función}}
\author{Aldo Berrios Valenzuela}
\date{Martes 2 de Agosto de 2016}

\usepackage{tikz,ifthen}
\usepackage{pgfplots}
\usepgfplotslibrary{fillbetween}
\usepackage{tikz,ifthen}
\usetikzlibrary{automata,positioning, babel}
\usetikzlibrary{calc}

\begin{document}
\maketitle
\begin{abstract}
	Comienza la diversión, comienza algoco \texttt{:D}
\end{abstract}

\section{Algoritmos y Complejidad}
\begin{itemize}
	\item Cómo diseñar buenos algoritmos.
	\item Evaluar algoritmos.
	\item Un poquito más de complejidad.
\end{itemize}
\begin{newquote}
	Buscamos el algoritmo que mejor resuelva nuestro problema.
\end{newquote}
\vspace{-2em}
\subsection{Temario del Ramo}
\begin{itemize}
	\item Algoritmos numéricos\footnote{Dorin Levy \comillas{Introduction to numerical analysis.}}.
	
	\item Algoritmos, complejidad\footnote{Dasgupta, Papadimitriou, Vazarani \comillas{Algoritmos} (CLRS)}.
\end{itemize}

\section{Encontrar ceros de funciones}
\paragraph{Idea:} Dada $f\nparentesis{x}$, hallar $x^*$ tal que $f\nparentesis{x^*}=0$ ($f$ debe ser continua). Por ejemplo: ¿Para qué valor de $x$ se cumple $x=e^{-x}$?. Una idea para resolver este problema simplemente basta con graficar
\begin{equation}
f\nparentesis{x}=x-e^{-x}
\end{equation}
y ver cuáles son los valores de $x^*$ tal que $f\nparentesis{x^*}=0$.

Otra forma de solucionar el problema puede ser graficar $x$, $e^{-x}$, para buscar las intersecciones (Figura \ref{01::Grafico:interseccion}).
\begin{figure}[!h]
	\centering
	\begin{tikzpicture}
	\begin{axis}[
	axis lines = left,
	xlabel = $x$,
	ylabel = {$y=f(x)$},
	every axis y label/.style={at=(current axis.above origin),anchor=south},
	every axis x label/.style={at=(current axis.right of origin),anchor=west},
	xmin=0, 
	xmax=2,
	ymin=0,
	ymax=1.2,
	xtick={0.567143},
	xticklabels = {$x^*$},
	ytick = {1},
	yticklabels = {$1$},
	height = 5cm, width = 9cm
	]
	
	\addplot [samples = 100, red!90!black!90, domain = 0:1.6] {exp(-x)} node [xshift = 18pt] (exponencial) {$y = e^{-x}$};
	\addplot [samples = 100, blue!50!black!90, domain = 0:1] {x} node [xshift = 13pt, yshift = 3pt] (lineal) {$y=x$}; 
	
	
	
	\draw[dashed, blue] (axis cs:0.567143,0) -- (axis cs:0.567143, 0.567143);
	\end{axis}
	\end{tikzpicture}
	\caption{Es claro que usando un gráfico podemos apreciar que $0<x^*<1$.}
	\label{01::Grafico:interseccion}
\end{figure}

\newpage
\subsection{Métodos Bracketing}
Corresponden a métodos para encontrar ceros (o raíces) de funciones los cuales usan el \textit{teorema del valor intermedio} ello y que básicamente van encerrando la solución hasta encontrar un punto de convergencia. A continuación, se explican dos métodos Bracketing: el método de la bisección y regula falsi.


\subsubsection{Método de la Bisección}
Si tengo $x_0, x_1$ tales que $f\nparentesis{x_0}\cdot f\nparentesis{x_1} < 0$, hay un cero de $f$ en $\left] x_0, x_1 \right[$, elegimos 
\begin{equation}
x_2 = \dfrac{x_0 + x_1}{2}
\end{equation}
con $\left] x_0, x_2\right[$ o $\left] x_2, x_1\right[$ según el cual tenga valores de $f$ de signo distinto.
\begin{figure}[!h]
	\centering
	\begin{tikzpicture}
	\begin{axis}[
	axis lines = center,
	xlabel = $x$,
	ylabel = {$y=f(x)$},
	every axis y label/.style={at=(current axis.above origin),anchor=south},
	every axis x label/.style={at=(current axis.right of origin),anchor=west},
	xmin=0, 
	xmax=2,
	ymin=-2,
	ymax=2,
	xtick={0.3, 0.85, 1.4},
	xticklabels = {$x_0$, $x_2$, $x_1$},
	ytick = {0.8, -0.6},
	yticklabels = {$f\nparentesis{x_0}$, $f\nparentesis{x_1}$},
	height = 6cm, width = 9cm
	]
	\addplot [samples = 100, smooth, tension = 0.8, red!80!black!90] coordinates {(0.3, 0.8) (0.5, 1.7) (0.8, 0.4) (1.1, 0.9) (1.3, -1) (1.4, -0.6)};
	
	\draw [blue, dashed] (axis cs:0.3, 0) -- (axis cs:0.3, 0.8);
	\draw [blue, dashed] (axis cs:1.4, 0) -- (axis cs:1.4, -0.6);
	
	\draw [blue, dashed] (axis cs:0, -0.6) -- (axis cs:1.4, -0.6);
	\draw [blue, dashed] (axis cs:0, 0.8) -- (axis cs:0.3, 0.8);
	\end{axis}
	\end{tikzpicture}
	\caption{Si las cotas del intervalo $\ncorchetes{x_0, x_1}$ cumplen con $f\nparentesis{x_0}\cdot f\nparentesis{x_1} <0$, es claro que para algún valor $x^*\in\ncorchetes{x_0, x_1}$ se cumple que $f\nparentesis{x^*}=0$.}
	\label{01::BiseccionEjemplo1}
\end{figure}
Por ejemplo, consideremos la Figura \ref{01::BiseccionEjemplo1}. En ella, podemos apreciar que el intervalo $\left]x_2, x_1\right[$ cumple con $f\nparentesis{x_2}\cdot f\nparentesis{x_1}<0$. En consecuencia, sabemos que $x^*\in\left]x_2, x_1\right[$. Luego, para obtener $x^*$ obtenemos el valor medio del intervalo $\left]x_2, x_1\right[$ y repetimos el proceso.

Nótese que la gracia de todo esto es encontrar sólo \underline{\smash{un}} cero, ¡no todos!. Esto quiere decir lo ideal es escoger intervalo $\ncorchetes{a, b}$ tal que sólo tenga \underline{\smash{un}} cero (Evite casos como los de la Figura \ref{01::BiseccionEjemplo2}).
\begin{figure}[!h]
	\centering
	\begin{tikzpicture}
	\begin{axis}[
	axis lines = center,
	xlabel = $x$,
	ylabel = {$y=f(x)$},
	every axis y label/.style={at=(current axis.above origin),anchor=south},
	every axis x label/.style={at=(current axis.right of origin),anchor=west},
	xmin=0, 
	xmax=2,
	ymin=-2,
	ymax=2,
	xtick={0.3, 0.85, 1.4},
	xticklabels = {$x_0$, $x_2$, $x_1$},
	ytick = {0},
	yticklabels = {},
	height = 6cm, width = 9cm
	]
	\addplot [samples = 100, smooth, tension = 0.8, red!80!black!90] coordinates {(0.3, 0.8) (0.5, 1.7) (0.8, -0.9) (1.1, 0.9) (1.3, -1) (1.4, -0.6)};
	
	\draw [blue, dashed] (axis cs:0.3, 0) -- (axis cs:0.3, 0.8);
	\draw [blue, dashed] (axis cs:1.4, 0) -- (axis cs:1.4, -0.6);
	\end{axis}
	\end{tikzpicture}
	\caption{La presente función tiene 3 ceros.}
	\label{01::BiseccionEjemplo2}
\end{figure}
Por lo tanto, el método de la bisección no abarca todos los casos posibles.



\subsubsection{Método Regula Falsi}
Cuando tenemos la sospecha o simplemente nos dicen que nuestra curva $f$ tiene una especie de \comillas{guatita} o simplemente tiene tendencia lineal (Figura \ref{01::RegularFalsi:grafico}), podemos usar:
\begin{equation}
x_2 = x_1 - f\nparentesis{x_1}\cdot \dfrac{x_1-x_0}{f\nparentesis{x_1}-f\nparentesis{x_0}}
\end{equation}
\begin{figure}[!h]
	\centering
	\begin{tikzpicture}
	\begin{axis}[
	axis lines = center,
	xlabel = $x$,
	ylabel = {$y$},
	every axis y label/.style={at=(current axis.above origin),anchor=south},
	every axis x label/.style={at=(current axis.right of origin),anchor=west},
	xmin=0, 
	xmax=2,
	ymin=-0.6,
	ymax=2,
	xtick={0.3, 1.0619, 1.3},
	xticklabels = {$x_0$, $x_2$, $x_1$},
	ytick = {0},
	yticklabels = {},
	height = 6cm, width = 9cm
	]
	\addplot [blue!60!black, samples = 100, smooth, tension = 0.8] coordinates {(0.3, 1.6) (0.8, 0) (1.3, -0.5)};
	\addplot [red!80!black, samples = 100, domain = 0.2:1.4] {2.23 - 2.1*x};
	
	\draw [dashed, blue] (axis cs:0.3, 0) -- (axis cs:0.3, 1.6);
	\draw [dashed, blue] (axis cs:1.3, -0.5) -- (axis cs:1.3, 0);
	\draw [dashed, blue] (axis cs:0.3, 0) -- (axis cs:0.3, 1.6);
	\draw [dashed, blue] (axis cs:1.0619, 0) -- (axis cs:1.0619, -0.35);
	\end{axis}
	\end{tikzpicture}
	\caption{$f$ tiene tendencia lineal.}
	\label{01::RegularFalsi:grafico}
\end{figure}

\subsection{Iteración de Punto Fijo (FPI)}
\begin{definition}
	Sea $g\nparentesis{x}$ una función. Un \definicion{punto fijo} de $g$ es $x^*$ tal que $x^*=g\nparentesis{x^*}$.
\end{definition}

Encontrar la raíz de una función por algún método de punto fijo consiste en reescribir
\begin{equation}\label{01::FPI:example}
	f\nparentesis{x}=0
\end{equation}
luego, despejamos $x$ de \eqref{01::FPI:example} y lo que quede al lado opuesto de la ecuación será una función $g\nparentesis{x}$ tal que:
\begin{equation}
g\nparentesis{x}=x
\end{equation}
\begin{ejemplo}
	Supongamos que queremos encontrar el cero de la ecuación
	\begin{equation*}
	\cos\nparentesis{x}-2x=0
	\end{equation*}
	Es fácil ver que $f\nparentesis{x}=\cos\nparentesis{x}-2x$. Entonces, se tiene una posible función $g$ sería:
	\begin{align*}
		\underbrace{\cos\nparentesis{x}-x}_{g\nparentesis{x}}=x
	\end{align*}
	\qed
\end{ejemplo}

Usando iteraciones de punto fijo, se tiene que el cero de la función corresponde a la intersección entre $y=x$ y $g\nparentesis{x}$ (Figura \ref{01::FPI}).
\begin{figure}[!h]
	\centering
	\begin{tikzpicture}
	\begin{axis}[
	axis lines = center,
	xlabel = $x$,
	ylabel = {$y$},
	every axis y label/.style={at=(current axis.above origin),anchor=south},
	every axis x label/.style={at=(current axis.right of origin),anchor=west},
	xmin=0, 
	xmax=2.5,
	ymin=0,
	ymax=2,
	xtick={0.81},
	xticklabels = {$x^*$},
	ytick = {0},
	yticklabels = {},
	height = 6cm, width = 9cm
	]
	\addplot [samples = 100, blue!50!black!90, domain = 0:1.4] {x} node [yshift = 2pt, xshift = 13pt] (yx) {$y=x$}; 
	\addplot [domain = 0.5: 1.6, samples = 100] {(1.5-0.7*x)^3} node [yshift = 10pt] (g) {$g\nparentesis{x}$};
	
	\draw [dashed, blue] (axis cs:0.81, 0) -- (axis cs:0.81, 1.11);
	\end{axis}
	\end{tikzpicture}
	\caption{El valor de $x$ de la intersección entre la recta $y=x$ y la función $g\nparentesis{x}$ corresponde al cero de $f$}
	\label{01::FPI}
\end{figure}

Para efectos de convergencia, se puede ver como una espiral (Figura \ref{01::Espiral:divergente} y Figura \ref{01::Espiral:convergente}). 
\begin{figure}[!h]
	\centering
	\begin{tikzpicture}
	\begin{axis}[
	axis lines = center,
	xlabel = $x$,
	ylabel = {$y$},
	every axis y label/.style={at=(current axis.above origin),anchor=south},
	every axis x label/.style={at=(current axis.right of origin),anchor=west},
	xmin=0, 
	xmax=2.5,
	ymin=0,
	ymax=2,
	xtick={0.6},
	xticklabels = {$x_0$},
	ytick = {0},
	yticklabels = {},
	height = 6cm, width = 9cm
	]
	\addplot [samples = 100, blue!50!black!90, domain = 0:1.4] {x}; 
	\addplot [domain = 0.5: 1.6, samples = 100] {(1.5-0.7*x)^3};
	\draw [red, dashed] (axis cs: 0.6, 0) -- (axis cs: 0.6, 1.259712) -- (axis cs: 1.259712, 1.259712) -- (axis cs: 1.259712, 0.23626) -- (axis cs: 0.23626, 0.23626);
	
	\draw [red, ->, thick] (axis cs: 0.6, 0.7) -- (axis cs: 0.6, 1);
	\draw [red, ->, thick] (axis cs: 0.8, 1.259712) --++ (axis cs: 0.2, 0);
	\draw [red, ->, thick] (axis cs: 1.259712, 0.9) --++ (axis cs: 0, -0.2);
	\draw [red, ->, thick] (axis cs: 0.9, 0.23626) --++ (axis cs: -0.2, 0);
	\end{axis}
	\end{tikzpicture}
	\caption{En este caso, la espiral diverge (mire las flechas).}
	\label{01::Espiral:divergente}
\end{figure}
\begin{figure}[!h]
	\centering
	\begin{tikzpicture}
	\begin{axis}[
	axis lines = center,
	xlabel = $x$,
	ylabel = {$y$},
	every axis y label/.style={at=(current axis.above origin),anchor=south},
	every axis x label/.style={at=(current axis.right of origin),anchor=west},
	xmin=0, 
	xmax=2.5,
	ymin=0,
	ymax=2,
	xtick={0.6},
	xticklabels = {$x_0$},
	ytick = {0},
	yticklabels = {},
	height = 6cm, width = 9cm
	]
	\addplot [samples = 100, blue!50!black!90, domain = 0:1.4] {x}; 
	\addplot [domain = 0.3: 2, samples = 100] {1.5*exp(-x)+0.3};
	\draw [red, dashed] (axis cs: 0.6, 0) -- (axis cs: 0.6, 1.12322) -- (axis cs: 1.12322, 1.12322) -- (axis cs: 1.12322, 0.787848) -- (axis cs: 0.787848, 0.787848);
	
	\draw [red, ->, thick] (axis cs: 0.6, 0.7) --++ (axis cs: 0, 0.2);
	\draw [red, ->, thick] (axis cs: 0.8, 1.12322) --++ (axis cs: 0.2, 0);
	\draw [red, ->, thick] (axis cs: 1.12322, 1.05) --++ (axis cs: 0, -0.2);
	\draw [red, ->, thick] (axis cs: 1.05, 0.787848) --++ (axis cs: -0.2, 0);
	\end{axis}
	\end{tikzpicture}
	\caption{En este caso, la espiral converge (mire las flechas).}
	\label{01::Espiral:convergente}
\end{figure}
Nótese que para construir estas espirales debe partir por el eje $x$, en algún $x_0$ a su gusto. Luego, tirar una linea vertical hasta chocar con $g$. En seguida, continúe con una línea horizontal hasta llegar a $y=x$ y vuelva a trazar otra vertical hasta $g\nparentesis{x}$... y así sucesivamente hasta encontrar el punto de convergencia.

\subsubsection{Método de la Secante}
Mantener los últimos dos ($x_1$, $x_2$, luego $x_2, x_3$ y así sucesivamente). Ojo que el método de la secante no tiene garantía de convergencia (Figura \ref{01::secante:grafico}).
\begin{figure}[!h]
	\centering
	\begin{tikzpicture}
	\begin{axis}[
	axis lines = center,
	xlabel = $x$,
	ylabel = {$y$},
	every axis y label/.style={at=(current axis.above origin),anchor=south},
	every axis x label/.style={at=(current axis.right of origin),anchor=west},
	xmin=0, 
	xmax=2,
	ymin=-0.6,
	ymax=2,
	xtick={0.345, 0.835, 0.988125},
	xticklabels = {\small$x_0$, \small$x_{1}$, \small$x_{2}$},
	ytick = {0},
	yticklabels = {},
	height = 6cm, width = 9cm
	]
	\addplot [black!70!blue!80, samples = 100, smooth, tension = 0.8] coordinates {(0.3, 1.3) (0.6, 0.5) (1.4, -0.05) (1.9, -0.12)};
	\addplot [domain = 0.2:1.2, red!80!black, samples = 100] {-1.72043*x+1.7};
	
	\draw [dashed, blue] (axis cs:0.345, 0) -- (axis cs:.345, 1.11);
	\draw [dashed, blue] (axis cs:0.835, 0) -- (axis cs:0.835, 0.27);
	\end{axis}
	\end{tikzpicture}
	\caption{Una secante corta a la curva $f$ en los puntos $x_0$ y $x_{1}$. El intercepto con el eje $x$ corta en $x_{2}$. Luego, se evalúa $f$ en $x_{2}$ y se obtiene una nueva secante que una los puntos $f\nparentesis{x_{1}}$ y $f\nparentesis{x_{2}}$. Finalmente, repetir el proceso hasta encontrar el valor de convergencia.}
	\label{01::secante:grafico}
\end{figure}

\subsubsection{Método de la tangente (Newton)}
La idea consiste en elegir un valor arbitrario $x_0$ que esté razonablemente cerca de $x^*$ (cero de la función). Luego, encontramos el intercepto con el eje $x$ de la recta tangente de la curva en $x_0$ usando la ecuación
\begin{equation}\label{01::ecuacion:newton}
x_1=x_0-\dfrac{f\nparentesis{x_0}}{f'\nparentesis{x_0}}
\end{equation}
resultando un valor $x_1$ que se encuentra más cerca de $x^*$ (Figura \ref{01::newton:grafico}). Finalmente, iteramos el proceso hasta obtener un valor de convergencia.
\begin{figure}[!h]
	\centering
	\begin{tikzpicture}
	\begin{axis}[
	axis lines = center,
	xlabel = $x$,
	ylabel = {$y$},
	every axis y label/.style={at=(current axis.above origin),anchor=south},
	every axis x label/.style={at=(current axis.right of origin),anchor=west},
	xmin=0, 
	xmax=2,
	ymin=-0.6,
	ymax=2,
	xtick={0.535, 0.88375},
	xticklabels = {$x_0$, $x_1$},
	ytick = {0},
	yticklabels = {},
	height = 6cm, width = 9cm
	]
	\addplot [black!70!blue!80, samples = 100, smooth, tension = 0.8] coordinates {(0.3, 1.3) (0.6, 0.5) (1.3, -0.2) (1.8, -0.3)};
	\addplot [domain = 0.2:1.1, red!80!black, samples = 100] {-1.72043*x+1.52043};
	
	\draw [dashed, blue] (axis cs:0.535, 0) -- (axis cs:0.535, 0.6);
	\end{axis}
	\end{tikzpicture}
	\caption{El valor $x_1$ obtenido con la ecuación \eqref{01::ecuacion:newton} está más cercano a $x^*$ que $x_0$.}
	\label{01::newton:grafico}
\end{figure}


\section{Orden de Convergencias}
Si definimos $e_n=\nabsoluto{x_n-x^*}$, se dice que un método es de \definicion{orden $p$}, si hay constantes $C > 0$, $p$ tales que
\begin{equation}
	e_{n+1} \leq C e_n^p
\end{equation}
Donde $e_n$ corresponde al error con $n$ iteraciones.

\begin{itemize}
	\item Si $p=1$ hablamos de convergencia lineal (bisección $C=\frac{1}{2}$).
	
	\item Si $p=2$ hablamos de convergencia cuadrática (en cada paso, el error se eleva a $p=2$).
	
	\item Si  $1 < p < 2$ decimos que es superlineal.
\end{itemize}
Nótese que mientras mayor sea el orden de convergencia $p$, más rápido encontraremos el valor de convergencia.

\begin{newquote}
	Es buena idea comparar cuál método es mejor según el caso que usemos
\end{newquote}
\vspace{-1.5em}
\begin{newquote}
	Para el caso del método de bisección, tenemos que $p=1$ y $C=\frac{1}{2}$
\end{newquote}

\end{document}
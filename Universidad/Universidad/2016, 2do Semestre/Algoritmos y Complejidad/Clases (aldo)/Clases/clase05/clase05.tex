\documentclass[english, spanish, fleqn, 10pt]{article}
\usepackage[top = 2.5cm, bottom = 2cm, left = 2.5cm, right = 2.5cm]{geometry}
\usepackage[es-noindentfirst]{babel}
\usepackage[utf8]{inputenc}
\usepackage{amsthm, amsmath, amssymb, amsfonts, environ}
\usepackage{caption}
\usepackage{subcaption} % para las subfigures
\usepackage{mathrsfs}
\usepackage[colorlinks, urlcolor=blue,
pdfauthor={Aldo Berrios Valenzuela}]{hyperref}
\usepackage{fourier}
\usepackage{cancel}
\usepackage{colortbl}
\usepackage{color}
\usepackage{graphicx}
\usepackage{enumitem}
%\renewcommand{\familydefault}{\sfdefault}
%\setlength{\parindent}{0pt}					%Espacio de sangria
\setlength{\parskip}{0.5mm}						%Interlinado entre párrafos
\usepackage{fancyhdr, blindtext}		%Paquete de encabezado
\usepackage{listings}
\usepackage[framemethod=tikz]{mdframed}

\definecolor{webred}{rgb}{0.75,0,0}
\definecolor{mblue}{rgb}{0.0705,0.345,0.7529}
\definecolor{newmblue}{HTML}{004D99}
\usepackage{longtable, colortbl, array}

\definecolor{superGreenNew}{HTML}{169F01}
\hypersetup{
	colorlinks   = true,
	citecolor    = superGreenNew
}


%==============================
%Modificador de Titulos de secciones, subsecciones, etc
\usepackage[explicit, noindentafter]{titlesec}

\titlespacing\subsubsection{0pt}{8pt plus 4pt minus 2pt}{4pt plus 2pt minus 2pt}


%==================================
%Diseno para insertar codigos de programacion
\definecolor{newGray}{HTML}{F8F8F8}
\definecolor{anotherGray}{HTML}{DDDFFF}
\lstdefinestyle{customc}{
	belowcaptionskip=1\baselineskip,
	breaklines=true,
	backgroundcolor=\color{newGray},
	frame=single,
	rulecolor=\color{anotherGray},
	xleftmargin=\parindent,
	language=bash,
	showstringspaces=false,
	basicstyle=\small\ttfamily,
	keywordstyle=\bfseries\color{green!40!black},
	commentstyle=\itshape\color{purple!40!black},
	identifierstyle=\color{black},
	stringstyle=\color{mblue},
	tabsize=2,
	literate={\ \ }{{\ }}1	
}
\lstset{style=customc}

%==================================
%Macros útiles
\newcommand{\comillas}[1]{``#1''}
\newcommand{\comentarioc}[1]{\texttt{\textcolor{webred}{/* #1 */}}}

\numberwithin{equation}{section}

%=================================
%comandos matemáticos personalizados de rápido acceso
\newcommand{\nparentesis}[1]{\left( #1 \right)}
\newcommand{\llaves}[1]{\left \{ #1 \right \}}
\newcommand{\nabsoluto}[1]{\left| #1 \right|}
\newcommand{\ncorchetes}[1]{\left[ #1 \right]}
%=================================

%cambia el espaciado entre el parrafo y antes del itemize
\setlist[itemize]{itemsep=1mm, topsep=1.5mm}
\setlist[enumerate]{itemsep=1mm, topsep=1.5mm}
%===========================

\theoremstyle{definition}
\newtheorem{teorema}{Teorema}[section]
\newtheorem{definition}{Definición}[section]
\newtheorem{beforeExample}{Ejemplo}[section]
\newtheorem{preposicion}{Preposición}[section]
\newtheorem{corolario}{Corolario}[teorema]

\captionsetup{justification=centering, margin=2.3cm}

\newenvironment{ejemplo}[1][]{\begin{beforeExample}[#1]\renewcommand{\qedsymbol}{$\blacksquare$}}{\qed\end{beforeExample}}

\captionsetup{justification=centering, margin=2.3cm}

% Para poder hacer "quote" con estilo github

\newenvironment{newquote}{\begin{mdframed}[topline=false,
		rightline=false, bottomline=false, linewidth=.3em,backgroundcolor=black!5, linecolor=black!60]\color{black!70}}{\end{mdframed}}
%================================


\usepackage{fancyhdr}

\pagestyle{fancy}
\lhead{INF221\quad Algoritmos y Complejidad}
\rhead{\textit{Clase \#5\quad Error de Interpolación}}


\author{Aldo Berrios Valenzuela}
\title{INF221 -- Algoritmos y Complejidad\\[.4\baselineskip]Clase \#5\\Error de Interpolación}
\date{Martes 16 de Agosto de 2016}

\usepackage{tikz,ifthen}
\usepackage{pgfplots}
\usepackage{tikz,ifthen}
\usetikzlibrary{automata,positioning, babel}
\usetikzlibrary{calc}
\newcommand\numberthis{\addtocounter{equation}{1}\tag{\theequation}}


\begin{document}
\maketitle
\section{Error de Interpolación}
La clase pasada vimos como obtener la interpolación dado los pares de puntos $\nparentesis{x_k, f\nparentesis{x_j}}$ con $k\in\llaves{0, \ldots, n}$ que nos daban. Nuestro tema de interés ahora es obtener el error de esas interpolaciones (Figura \ref{05::ErrorEjemplo}).
\begin{figure}[!h]
	\centering
	\begin{tikzpicture}
	\begin{axis}[
	axis lines = center,
	xlabel = $x$,
	ylabel = {$y$},
	every axis y label/.style={at=(current axis.above origin),anchor=south},
	every axis x label/.style={at=(current axis.right of origin),anchor=west},
	xmin=0, 
	xmax=3,
	ymin=-0.6,
	ymax=2.3,
	xtick={0},
	xticklabels = {},
	ytick = {0},
	yticklabels = {},
	height = 6cm, width = 9cm
	]
	
	\addplot [black!30!blue!90, samples = 100, domain = 0.15:2.4] {ln(2*x+0.8)+0.2} node [xshift = 10] (f) {$f\nparentesis{x}$};
	\addplot [black!10!red!90, smooth, tension = 0.8] coordinates {(0.2, 0.2) (0.55, 1.3) (0.9, 0.6) (1.2, 1.7) (1.5, 0.8) (1.9, 1.9)} node [yshift = 5pt, xshift = 4pt] (p) {$p\nparentesis{x}$};
	
	
	\draw [fill = black!30] (axis cs:0.257, 0.47) circle (0.7mm);
	\draw [fill = black!30] (axis cs:0.72, 1) circle (0.7mm);
	\draw [fill = black!30] (axis cs:1.315, 1.43) circle (0.7mm);
	\draw [fill = black!30] (axis cs:1.85, 1.7) circle (0.7mm);
	\draw [fill = black!30] (axis cs:1.85, 1.7) circle (0.7mm);
	
	
	\draw [<->, thick, green!60!black!90] (axis cs:0.888, 1.15) -- (axis cs:0.888, 0.595);
	\draw [fill = black!30] (axis cs:0.888, 0.595) circle (0.7mm);
	\end{axis}
	\end{tikzpicture}
	\caption{El error está dado por la diferencia entre $p\nparentesis{x}$ y $f\nparentesis{x}$ en cada punto (flecha verde).}
	\label{05::ErrorEjemplo}
\end{figure}


\begin{teorema}[Rolle]
	Si $f$ es continua en $\ncorchetes{a, b}$ y tiene derivada continua en $\ncorchetes{a, b}$, si $f\nparentesis{a}=f\nparentesis{b}=0$, hay $x^*\in\nparentesis{a, b}$ tal que $f'\nparentesis{x^*}=0$ (pariente del teorema del valor medio de la derivada).
\end{teorema}

\begin{teorema}
	Sea $f\in C^{n+1}\ncorchetes{a, b}$. Sea $Q_n\nparentesis{x}$ el polinomio de grado $n$ que interpola $f$ en los puntos distintos $x_0, \ldots, x_n\in\ncorchetes{a, b}$. Entonces para todo $x\in\ncorchetes{a, b}$ hay $\zeta\in\ncorchetes{a, b}$ tal que
	\begin{equation}\label{05::Teo2}
	f\nparentesis{x}-Q_n\nparentesis{x}=\underbrace{\dfrac{1}{\nparentesis{n+1}!}f^{\nparentesis{n+1}}\nparentesis{\zeta}\prod_{0\leq j\leq n}\nparentesis{x-x_j}}_{\text{Error}}
	\end{equation}
\end{teorema}
\begin{proof}
	Sea
	\begin{equation}
	\omega\nparentesis{x}=\prod_{0\leq j\leq n}\nparentesis{x-x_j}
	\end{equation}
	Notamos para referencia futura que $\omega\nparentesis{x}$ es mónico\footnote{Para un polinomio de grado $n$, el coeficiente que acompaña a $x^n$ es $1$.}, con lo que $\omega^{\nparentesis{n+1}}\nparentesis{x}=\nparentesis{n+1}!$.
	
	Fijemos $x\in\ncorchetes{a, b}$. Si $x=x_j$, entonces $F\nparentesis{x}=0$ sin importar $\lambda$, el lado izquierdo y derecho de \eqref{05::Teo2} se anulan y estamos listos. Sea:
	\begin{equation}
	F\nparentesis{y}=f\nparentesis{y}-Q_n\nparentesis{y}-\lambda\omega\nparentesis{y}
	\end{equation}
	donde elegimos $\lambda$ tal que $F\nparentesis{x}=0$.
	
	La función $F\nparentesis{y}$ está en $C^{n+1}\ncorchetes{a, b}$, y tiene $n+2$ ceros en $\ncorchetes{a, b}$ ($x, x_0, \ldots, x_n$). Por teorema de Rolle, $F'\nparentesis{y}$ tiene $n+1$ ceros en $\ncorchetes{a, b}$, \ldots, $F^{\nparentesis{n+1}}\nparentesis{y}$ tiene un cero en $\ncorchetes{a, b}$, llamémosle $\zeta$. O sea:
	\begin{align*}
	F^{\nparentesis{n+1}}\nparentesis{\zeta}&=f^{\nparentesis{n+1}}\nparentesis{\zeta}-\cancelto{\scriptstyle 0}{Q_n^{\nparentesis{n+1}}\nparentesis{\zeta}}-\lambda\quad\cancelto{\scriptstyle\nparentesis{n+1}!}{\omega^{\nparentesis{n+1}}\nparentesis{\zeta}}=0\\
	\therefore f^{\nparentesis{n+1}}\nparentesis{\zeta}&=\lambda\nparentesis{n+1}!\qquad \rightsquigarrow \lambda =\dfrac{f^{\nparentesis{n+1}}\nparentesis{\zeta}}{\nparentesis{n+1}!}\\
	\therefore F\nparentesis{y}&=f\nparentesis{y}-Q_n\nparentesis{y}-\dfrac{f^{\nparentesis{n+1}}\nparentesis{\zeta}}{\nparentesis{n+1}!}\omega\nparentesis{x}
	\end{align*}
	El error es:
	\begin{equation*}
	\dfrac{f^{\nparentesis{n+1}}\nparentesis{\zeta}}{\nparentesis{n+1}!}\omega\nparentesis{x}
	\end{equation*}
	Donde $n$ lo elegimos nosotros y corresponde al grado de la interpolación. Es claro que mientras mayor sea el grado de nuestra interpolación, menor será el error (vea en $n$ como denominador).
\end{proof}

\section{Cuadratura}
Queremos evaluar:
\begin{equation}\label{05::Integral}
\int_{a}^{b}f\nparentesis{x}\mathit{dx}
\end{equation}
Supongamos $f$ dado en $x_0, \ldots, x_n$ (aka \textit{puntos de cuadratura}). Para encontrar el valor de \eqref{05::Integral} simplemente interpolamos $f$, e integramos el polinomio interpolante.

\subsection{Caso más simple: Polinomio de grado 0}
Corresponde a una aproximación con rectángulos (Figura \ref{05::CasoSimple:Cuadratura})
\begin{figure}[!h]
	\centering
	\begin{tikzpicture}
	\begin{axis}[
	axis lines = center,
	xlabel = $x$,
	ylabel = {$y$},
	every axis y label/.style={at=(current axis.above origin),anchor=south},
	every axis x label/.style={at=(current axis.right of origin),anchor=west},
	xmin=0, 
	xmax=3,
	ymin=-0.6,
	ymax=2.3,
	xtick={0.5, 2.0},
	xticklabels = {$a$, $b$},
	ytick = {0},
	yticklabels = {},
	height = 6cm, width = 9cm
	]
	
	\addplot [black!10!red!90, samples = 100, domain = 0.15:2.4] {ln(2*x+0.8)+0.2} node [xshift = 10] (f) {$f\nparentesis{x}$};
	
	\draw [blue!60!black!90] (axis cs:0.5, 0) -- (axis cs:0.5, 0.79) -- (axis cs: 0.8, 0.79) -- (axis cs: 0.8, 0);
	\draw [blue!60!black!90] (axis cs:0.8, 0) -- (axis cs:0.8, 1.07) -- (axis cs: 1.1, 1.07) -- (axis cs: 1.1, 0);
	\draw [blue!60!black!90] (axis cs:1.1, 0) -- (axis cs: 1.1, 1.3) -- (axis cs: 1.4, 1.3) -- (axis cs: 1.4, 0);
	\draw [blue!60!black!90] (axis cs:1.4, 0) -- (axis cs: 1.4, 1.48) -- (axis cs: 1.7, 1.48) -- (axis cs: 1.7, 0);
	\draw [blue!60!black!90] (axis cs:1.7, 0) -- (axis cs: 1.7, 1.63) -- (axis cs: 2.0, 1.63) -- (axis cs: 2.0, 0);
	
	\end{axis}
	\end{tikzpicture}
	\caption{En tiempos precarios, nos enseñaron que podemos calcular el área bajo una curva usando una aproximación con rectangulitos (integral de Riemann).}
	\label{05::CasoSimple:Cuadratura}
\end{figure}

Para ello, usábamos la fórmula:
\begin{align*}
\int_{a}^{b}f\nparentesis{x}\mathit{dx}\approx \sum_{0\leq j\leq n}f\nparentesis{x_j}\nparentesis{x_{j+1}-x_j};\qquad\qquad a=x_0, b=x_n
\end{align*}
Si son igualmente espaciados, se tiene que $x_{j+1}-x_j=h$ para $0\leq j < n$
\begin{align*}
\int_{a}^{b}f\nparentesis{x}\mathit{dx}\approx h\sum_{0\leq j< n}f\nparentesis{x_j}
\end{align*}
Consideremos la antiderivada\footnote{Es claro suponer que la antiderivada existe, de lo contrario este cuento no tiene chiste.}:
\begin{align*}
F\nparentesis{x}=\int_{a}^{x}f\nparentesis{t}\mathit{dt}
\end{align*}
¿podemos obtener un error más pequeño que el caso de la Figura \ref{05::CasoSimple:Cuadratura}?

\subsubsection{Variante del caso simple: punto medio}
En lugar de considerar una cota como se hizo en el caso de la Figura \ref{05::CasoSimple:Cuadratura}, el punto de evaluación de $f\nparentesis{x}$ será el punto medio de la base del rectángulo (Figura \ref{05::CasoVariado:Cuadratura}).
\begin{figure}[!h]
	\centering
	\begin{tikzpicture}
	\begin{axis}[
	axis lines = center,
	xlabel = $x$,
	ylabel = {$y$},
	every axis y label/.style={at=(current axis.above origin),anchor=south},
	every axis x label/.style={at=(current axis.right of origin),anchor=west},
	xmin=0, 
	xmax=3,
	ymin=-0.6,
	ymax=2.3,
	xtick={0.65, 1.85},
	xticklabels = {$a+\dfrac{h}{2}$, $x_n+\dfrac{h}{2}$},
	ytick = {0},
	yticklabels = {},
	height = 6cm, width = 9cm
	]
	
	\addplot [black!10!red!90, samples = 100, domain = 0.15:2.4] {ln(2*x+0.8)+0.2} node [xshift = 10] (f) {$f\nparentesis{x}$};
	
	\draw [blue!60!black!90] (axis cs:0.5, 0) -- (axis cs:0.5, 0.94) -- (axis cs: 0.8, 0.94) -- (axis cs: 0.8, 0);
	\draw [blue!60!black!90] (axis cs:0.8, 0) -- (axis cs:0.8, 1.2) -- (axis cs: 1.1, 1.2) -- (axis cs: 1.1, 0);
	\draw [blue!60!black!90] (axis cs:1.1, 0) -- (axis cs: 1.1, 1.39) -- (axis cs: 1.4, 1.39) -- (axis cs: 1.4, 0);
	\draw [blue!60!black!90] (axis cs:1.4, 0) -- (axis cs: 1.4, 1.56) -- (axis cs: 1.7, 1.56) -- (axis cs: 1.7, 0);
	\draw [blue!60!black!90] (axis cs:1.7, 0) -- (axis cs: 1.7, 1.7) -- (axis cs: 2.0, 1.7) -- (axis cs: 2.0, 0);
	
	\end{axis}
	\end{tikzpicture}
	\caption{Tomamos el punto medio en lugar del extremo del rectángulo. Como puede observar, el excedente de \comillas{triangulitos} por sobre $f$ compensan la falta de estos que están bajo $f$.}
	\label{05::CasoVariado:Cuadratura}
\end{figure}

Para este caso se tiene que:
\begin{align*}
\int_{a}^{b}f\nparentesis{x}\mathit{dx}\approx\nparentesis{n-a}f\nparentesis{\dfrac{a-b}{2}}
\end{align*}
Expandimos usando \textit{series de Taylor}:
\begin{align*}
F\nparentesis{a+h}&=F\nparentesis{a}+F'\nparentesis{a}h+\dfrac{1}{2}F''\nparentesis{a}h^2+\dfrac{1}{6}F'''\nparentesis{a}h^3+O\nparentesis{h^4}\\
&=f\nparentesis{a}h+\dfrac{1}{2}f'\nparentesis{a}h^2+\dfrac{1}{6}f''\nparentesis{a}h^3+O\nparentesis{h^4}
\end{align*}
Si $b=a+h$, tenemos (expandiendo $f\nparentesis{a+\frac{h}{2}}$) para el error:
\begin{align*}
E&=\int_{a}^{a+h}f\nparentesis{x}\mathit{dx}-hf\nparentesis{a+\dfrac{h}{2}}\\
&=hf\nparentesis{a}+\dfrac{h^2}{2}f'\nparentesis{a}+\dfrac{h^3}{6}f''\nparentesis{a}+O\nparentesis{h^4}-h\nparentesis{f\nparentesis{a}+\dfrac{h}{2}f'\nparentesis{a}+\dfrac{h^2}{8}f''\nparentesis{a}+O\nparentesis{h^3}}\\
&=hf\nparentesis{a}+\dfrac{h^2}{2}f'\nparentesis{a}+\dfrac{h^3}{6}f''\nparentesis{a}+O\nparentesis{h^4}-\nparentesis{hf\nparentesis{a}+\dfrac{h^2}{2}f\nparentesis{a}+\dfrac{h^3}{8}f''\nparentesis{a}+O\nparentesis{h^4}}\\
&=\dfrac{1}{24}f''\nparentesis{a}h^3+O\nparentesis{h^4}\numberthis\label{05::ErrrObtenido}
\end{align*}
Sorprendentemente, el error es de carácter cúbico. Si nos dan la segunda derivada de $f$ en $a$, estamos listos.

\subsection{Teorema del Valor Intermedio Ad Hoc}
Considerando el error obtenido en la ecuación \eqref{05::ErrrObtenido}:
\begin{equation}
E=\dfrac{1}{24}f''\nparentesis{\zeta}h^3\qquad \text{con }a\leq \zeta \leq a+h
\end{equation}
Suponiendo intervalitos $a=x_0, x_1, \ldots, x_n =b$ con $x_{j+1}-x_j=h$, tenemos para cada uno:
\begin{align*}
E_j=\dfrac{1}{24}f''\nparentesis{\zeta_j}h^3, \qquad x_j\leq \zeta _j \leq x_{j+1}
\end{align*}
Si $m\leq f''\nparentesis{x}\leq M$ en $\ncorchetes{a, b}$
\begin{align*}
E=\sum_{j}E_j=\dfrac{h^3}{24}\sum_j f''\nparentesis{\zeta _j}=\dfrac{nh^3}{24}f''\nparentesis{\zeta}
\end{align*}
porque
\begin{align*}
nm&\leq \sum_j f''\nparentesis{\zeta_j} \leq nM\\
m&\leq \dfrac{1}{n}\sum_j f''\nparentesis{\zeta_j}\leq M
\end{align*}
Lo que nos dice que hay $\zeta \in \ncorchetes{a, b}$ tal que:
\begin{align*}
\dfrac{1}{n}\sum_j f''\nparentesis{\zeta_j}=f''\nparentesis{\zeta}
\end{align*}








\end{document}


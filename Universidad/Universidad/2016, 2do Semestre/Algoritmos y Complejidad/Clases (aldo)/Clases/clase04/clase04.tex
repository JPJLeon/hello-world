\documentclass[english, spanish, fleqn, 10pt]{article}
\usepackage[top = 2.5cm, bottom = 2cm, left = 2.5cm, right = 2.5cm]{geometry}
\usepackage[es-noindentfirst]{babel}
\usepackage[utf8]{inputenc}
\usepackage{amsthm, amsmath, amssymb, amsfonts, environ}
\usepackage{caption}
\usepackage{subcaption} % para las subfigures
\usepackage{mathrsfs}
\usepackage[colorlinks, urlcolor=blue,
pdfauthor={Aldo Berrios Valenzuela}]{hyperref}
\usepackage{fourier}
\usepackage{colortbl}
\usepackage{relsize}
\usepackage{color}
\usepackage{graphicx}
\usepackage{enumitem}
%\renewcommand{\familydefault}{\sfdefault}
%\setlength{\parindent}{0pt}					%Espacio de sangria
\setlength{\parskip}{0.5mm}						%Interlinado entre párrafos
\usepackage{fancyhdr, blindtext}		%Paquete de encabezado
\usepackage{listings}
\usepackage[framemethod=tikz]{mdframed}

\definecolor{webred}{rgb}{0.75,0,0}
\definecolor{mblue}{rgb}{0.0705,0.345,0.7529}
\definecolor{newmblue}{HTML}{004D99}
\usepackage{longtable, colortbl, array}

\definecolor{superGreenNew}{HTML}{169F01}
\hypersetup{
	colorlinks   = true,
	citecolor    = superGreenNew
}


%==============================
%Modificador de Titulos de secciones, subsecciones, etc
\usepackage[explicit, noindentafter]{titlesec}

\titlespacing\subsubsection{0pt}{8pt plus 4pt minus 2pt}{4pt plus 2pt minus 2pt}


%==================================
%Diseno para insertar codigos de programacion
\definecolor{newGray}{HTML}{F8F8F8}
\definecolor{anotherGray}{HTML}{DDDFFF}
\lstdefinestyle{customc}{
	belowcaptionskip=1\baselineskip,
	breaklines=true,
	backgroundcolor=\color{newGray},
	frame=single,
	rulecolor=\color{anotherGray},
	xleftmargin=\parindent,
	language=bash,
	showstringspaces=false,
	basicstyle=\small\ttfamily,
	keywordstyle=\bfseries\color{green!40!black},
	commentstyle=\itshape\color{purple!40!black},
	identifierstyle=\color{black},
	stringstyle=\color{mblue},
	tabsize=2,
	literate={\ \ }{{\ }}1	
}
\lstset{style=customc}

%==================================
%Macros útiles
\newcommand{\comillas}[1]{``#1''}
\newcommand{\comentarioc}[1]{\texttt{\textcolor{webred}{/* #1 */}}}

\numberwithin{equation}{section}

%=================================
%comandos matemáticos personalizados de rápido acceso
\newcommand{\nparentesis}[1]{\left( #1 \right)}
\newcommand{\llaves}[1]{\left \{ #1 \right \}}
\newcommand{\nabsoluto}[1]{\left| #1 \right|}
\newcommand{\ncorchetes}[1]{\left[ #1 \right]}
%=================================

%cambia el espaciado entre el parrafo y antes del itemize
\setlist[itemize]{itemsep=1mm, topsep=1.5mm}
\setlist[enumerate]{itemsep=1mm, topsep=1.5mm}
%===========================

\theoremstyle{definition}
\newtheorem{teorema}{Teorema}[section]
\newtheorem{definition}{Definición}[section]
\newtheorem{beforeExample}{Ejemplo}[section]
\newtheorem{preposicion}{Preposición}[section]
\newtheorem{corolario}{Corolario}[teorema]

\captionsetup{justification=centering, margin=2.3cm}

\newenvironment{ejemplo}[1][]{\begin{beforeExample}[#1]\renewcommand{\qedsymbol}{$\blacksquare$}}{\qed\end{beforeExample}}

\captionsetup{justification=centering, margin=2.3cm}

% Para poder hacer "quote" con estilo github

\newenvironment{newquote}{\begin{mdframed}[topline=false,
		rightline=false, bottomline=false, linewidth=.3em,backgroundcolor=black!5, linecolor=black!60]\color{black!70}}{\end{mdframed}}
%================================


\usepackage{fancyhdr}

\pagestyle{fancy}
\lhead{INF221\quad Algoritmos y Complejidad}
\rhead{\textit{Clase \#4\quad Interpolación}}


\author{Aldo Berrios Valenzuela}
\title{INF221 -- Algoritmos y Complejidad\\[.4\baselineskip]Clase \#4\\\textit{Interpolación}}
\date{Miércoles 10 de Agosto de 2016}

\usepackage{tikz,ifthen}
\usepackage{pgfplots}
\usepackage{tikz,ifthen}
\usetikzlibrary{automata,positioning, babel}
\usetikzlibrary{calc}
\newcommand\numberthis{\addtocounter{equation}{1}\tag{\theequation}}

\begin{document}
\maketitle
\section{Interpolación}
\paragraph{Idea:} Nos dan los valores exactos de una función desconocida en $n+1$ puntos $f\nparentesis{x_0}, \ldots, f\nparentesis{x_n}$, queremos hallar una función que tome esos valores (para calcular valores intermedios). El caso más común es utilizar \textit{polinomios}. Para esta ocasión sabemos que hay exactamente \underline{un} polinomio de grado $\leq n$ que pasa por $n+1$ puntos.

Supongamos que hay dos polinomios distintos, $p\nparentesis{x}$ y $q\nparentesis{x}$, de grado $\leq n$ que pasan los $n+1$ puntos (véase la Figura \ref{04::EjemploIdeaInterpolacion}). 
\begin{figure}[!h]
	\centering
	\begin{tikzpicture}
	\begin{axis}[
	axis lines = center,
	xlabel = $x$,
	ylabel = {$y$},
	every axis y label/.style={at=(current axis.above origin),anchor=south},
	every axis x label/.style={at=(current axis.right of origin),anchor=west},
	xmin=0, 
	xmax=3,
	ymin=-0.6,
	ymax=2.3,
	xtick={0.257, 0.72, 1.315, 1.85},
	xticklabels = {$x_0$, $x_1$, $x_2$, $x_3$},
	ytick = {0},
	yticklabels = {},
	height = 6cm, width = 9cm
	]
	
	\addplot [black!30!blue!90, samples = 100, domain = 0.15:2.4] {ln(2*x+0.8)+0.2} node [xshift = 10] (f) {$q\nparentesis{x}$};
	\addplot [black!10!red!90, smooth, tension = 0.8] coordinates {(0.2, 0.2) (0.55, 1.3) (0.9, 0.6) (1.2, 1.7) (1.5, 0.8) (1.9, 1.9)} node [yshift = 5pt, xshift = 4pt] (p) {$p\nparentesis{x}$};
	
	\draw [dashed, blue] (axis cs:0.257, 0) -- (axis cs:0.257, 0.5);
	\draw [dashed, blue] (axis cs:0.72, 0) -- (axis cs:0.72, 1);
	\draw [dashed, blue] (axis cs:1.315, 0) -- (axis cs:1.315, 1.43);
	\draw [dashed, blue] (axis cs:1.85, 0) -- (axis cs:1.85, 1.7);
	\draw [fill = black!30] (axis cs:0.257, 0.47) circle (0.7mm);
	\draw [fill = black!30] (axis cs:0.72, 1) circle (0.7mm);
	\draw [fill = black!30] (axis cs:1.315, 1.43) circle (0.7mm);
	\draw [fill = black!30] (axis cs:1.85, 1.7) circle (0.7mm);
	\end{axis}
	\end{tikzpicture}
	\caption{Sabemos que $p\nparentesis{x}$ y $q\nparentesis{x}$ se interceptan en los puntos $x_0$, $x_1$, $x_2$ y $x_3$. Las formas que tengan estos polinomios no importa.}
	\label{04::EjemploIdeaInterpolacion}
\end{figure}
Entonces $p\nparentesis{x}-q\nparentesis{x}$ es un polinomio de grado $\leq n$ que tiene $n+1$ ceros $x_0, \ldots, x_n$, implicando que $p\nparentesis{x}-q\nparentesis{x}$ es exactamente igual a cero. 

Entre las cosas que podemos hacer para hallar la interpolación de $f\nparentesis{x}$ tenemos:
\begin{itemize}
	\item De forma implícita.
	\item Con un sistema de ecuaciones.
\end{itemize}

\subsection{De forma implícita}
Para encontrar la interpolación de $f$ de manera implícita simplemente inventamos un polinomio que pase por los puntos. Para ello podemos usar dos métodos:
\begin{itemize}
	\item Lagrange.
	\item Newton.
\end{itemize}

\subsubsection{Por Lagrange}
Consiste en usar el polinomio:
\begin{align*}
p\nparentesis{x}=&f\nparentesis{x_0}\dfrac{\nparentesis{x-x_1}\nparentesis{x-x_2}\cdots\nparentesis{x-x_n}}{\nparentesis{x_0-x_1}\nparentesis{x_0-x_2}\ldots \nparentesis{x_0-x_n}} + f\nparentesis{x_1}\dfrac{\nparentesis{x-x_0}\nparentesis{x-x_2}\ldots \nparentesis{x-x_n}}{\nparentesis{x_1-x_0}\nparentesis{x_1-x_2}\ldots \nparentesis{x_1-x_n}}+\cdots+\\
& f\nparentesis{x_n}\dfrac{\nparentesis{x-x_0}\ldots\nparentesis{x-x_{n+1}}}{\nparentesis{x_n-x_0}\ldots\nparentesis{x_n-x_{n+1}}}\numberthis\label{04::Lagrange:1}\\
=&\sum_{k\leq n}f\nparentesis{x_k}\prod_{\substack{j\leq n\\j\ne k}}\dfrac{x-x_j}{x_k-x_j}\numberthis\label{04::Lagrange:2}
\end{align*}
como interpolación de $f$. Es claro que si evalúa $p$ en alguno de los $x_k$ con $k\in\llaves{1, 2, \ldots, n}$ que nos entregan, se tiene que $p\nparentesis{x_k}=f\nparentesis{x_k}$.

Si los puntos están casi igualmente espaciados, las ecuaciones \eqref{04::Lagrange:1} y \eqref{04::Lagrange:2} resultan bastante agradables.

\subsubsection{Por Newton}
Podemos escribir:
\begin{align*}
Q_0\nparentesis{x}&=f\nparentesis{x_0}=a_0\\
Q_1\nparentesis{x}&=f\nparentesis{x_0}+\nparentesis{x_1-x_0}\underbrace{\dfrac{f\nparentesis{x_1}-f\nparentesis{x_0}}{x_1-x_0}}_{Q_0\nparentesis{x_1}}\\
Q_2\nparentesis{x}&=f\nparentesis{x_0}+\nparentesis{x-x_0}\dfrac{f\nparentesis{x_1}-f\nparentesis{x_0}}{x_1-x_0}+\nparentesis{x-x_0}\nparentesis{x-x_1}\dfrac{f\nparentesis{x_2}-Q_1\nparentesis{x_2}}{\nparentesis{x_2-x_0}\nparentesis{x_2-x_1}}\\
&\vdots\\
a_k&=\dfrac{f\nparentesis{x_k}-Q_{k-1}\nparentesis{x_k}}{\mathlarger\prod_{0\leq i\leq k-1}\nparentesis{x_k-x_i}}, \qquad Q_k\nparentesis{x}=Q_{k-1}\nparentesis{x}+\nparentesis{x-x_0}\ldots\nparentesis{x-x_{k-1}}a_k
\end{align*}
donde $Q_k$ corresponde a la interpolación de grado $k$.


Como comentario al margen, la fórmula prohibida que vimos en la primera clase (ecuación \eqref{04::formula:prohibida}) que asegura VTR 2 hace que la precisión se vaya a las pailas.
\begin{equation}\label{04::formula:prohibida}
x_2=\dfrac{f\nparentesis{x_1}\cdot x_0-f\nparentesis{x_0}\cdot x_1}{f\nparentesis{x_1}-f\nparentesis{x_0}}
\end{equation}

\subsection{Por sistema de ecuaciones}
Suponiendo $p\nparentesis{x}=a_0 + a_1 x + \cdots + a_n x^n$, creamos un sistema de ecuaciones de la forma:
\begin{equation*}
\begin{cases}
p\nparentesis{x_0}=f\nparentesis{x_0}=a_0+a_1x_0+a_2x_0^2+\cdots a_nx_0^n\\
p\nparentesis{x_1}=f\nparentesis{x_1}=a_0+a_1x_1+a_2x_1^2+\cdots a_nx_1^n\\
\hspace{9.5mm}\vdots\\
p\nparentesis{x_n}=f\nparentesis{x_n}=a_0+a_1x_n+a_2x_n^2+\cdots a_nx_n^n
\end{cases}
\end{equation*}
que matricialmente puede ser escrito como:
\begin{equation}\label{04::Sistema:de:ecuaciones:1}
	\begin{pmatrix}
	1 & x_0 & x_0^2 & \cdots & x_0^n\\
	1 & x_1 & x_1^2 & \cdots & x_1^n\\
	\vdots &\vdots &\vdots &\ddots&\vdots &\\
	1 & x_n & x_n^2 & \cdots & x_n^n
	\end{pmatrix}
	\begin{pmatrix}
	a_0\\
	a_1\\
	\vdots\\
	a_n
	\end{pmatrix}=
	\begin{pmatrix}
	f\nparentesis{x_0}\\
	f\nparentesis{x_1}\\
	\vdots\\
	f\nparentesis{x_n}
	\end{pmatrix}
\end{equation}
Con igual cantidad de ecuaciones e incógnitas, el sistema de ecuaciones \eqref{04::Sistema:de:ecuaciones:1} tiene solución si el determinante es distinto de cero 0:
\begin{equation}\label{04::Determinante:de:Vandermonde=0}
\nabsoluto{
	\begin{matrix}
	1 & \cdots & x_0^n\\
	\vdots & \ddots & \vdots \\
	1 & \cdots & x_n^n
	\end{matrix}}\ne 0
\end{equation}

La determinante que se muestra en la ecuación \eqref{04::Determinante:de:Vandermonde=0} resulta ser la \textit{determinante de Vandermonde}, la que resulta ser:
\begin{equation}\label{04::Definicion:Vandermonde:Matrix}
\nabsoluto{\begin{matrix}
	1 & x_0 & \cdots & x_0^n\\
	\vdots & \vdots  & \ddots & \vdots \\
	1 & x_n & \cdots & x_n^n
	\end{matrix}}=\prod_{i>j}\nparentesis{x_i-x_j}
\end{equation}
donde $x_i\ne x_j$ para $i\ne j$. Por lo tanto, si reemplazamos \eqref{04::Definicion:Vandermonde:Matrix} en \eqref{04::Determinante:de:Vandermonde=0}, observamos que el sistema de ecuaciones \eqref{04::Sistema:de:ecuaciones:1} tiene solución si:
\begin{equation}
\prod_{i>j}\nparentesis{x_i-x_j}\ne 0
\end{equation}

Para demostrar la igualdad \eqref{04::Definicion:Vandermonde:Matrix} podemos usar inducción.








\end{document}
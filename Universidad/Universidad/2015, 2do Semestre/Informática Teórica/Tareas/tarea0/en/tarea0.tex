\documentclass[spanish, fleqn]{article}
\usepackage{babel}
\usepackage[utf8]{inputenc}
\usepackage{amsmath}
\usepackage{amsfonts}
\usepackage{mathrsfs}
\usepackage{dcolumn}
\usepackage[colorlinks, urlcolor=blue]{hyperref}
\usepackage{fourier}

\newcommand{\num}{0}

\renewcommand{\thefootnote}{\fnsymbol{footnote}}
\newcommand{\cycle}[2]{\genfrac{[}{]}{0pt}{}{#1}{#2}}	 % Stirling 1a especie

\title{INF-155: Introducción a la Informática Teórica\\
       ILI-225: Informática Teórica \\[0.4\baselineskip]
       Tarea \#\num \\
       \emph{``¡Hagamos monitos!''}
      }
\author{\href{mailto:vonbrand@inf.utfsm.cl}{Horst von Brand}
	\and {Alondra Rojas Ruz}
	\and {Cristóbal Galleguillos}
	\and {Renato Sanhueza}}
\date{21 de octubre 2015}

\begin{document}
\maketitle

\thispagestyle{empty}

\section*{Preguntas}

  Usando \href{http://www.texample.net/tikz}{TikZ},
  dibuje los siguientes autómatas finitos deterministas:
  \begin{itemize}
  \item % 20152t0p1
    \(M = (Q, \Sigma, \delta, q_0, F)\),
    con \(Q = \{q_0, q_1, q_2\}\),
    \(\Sigma = \{0, 1\}\),
    \(F = \{q_2\}\),
    con la función de transición dada por la siguiente tabla:
    \begin{center}
      \begin{tabular}{>{\(}c<{\)}|>{\(}c<{\)}>{\(}c<{\)}}
	    &  0  &  1	\\
	\hline
	q_0 & q_0 & q_1 \\
	q_1 & q_0 & q_2 \\
	q_2 & q_2 & q_2 
      \end{tabular}
    \end{center}
  \item % 20152t0p2
    \(M = (Q, \Sigma, \delta, S_0, F)\),
    con \(Q = \{S_0, S_1, S_2, S_3\}\),
    \(\Sigma = \{a, b\}\),
    \(F = \{S_3\}\),
    en el cual la función de transición está dada por la siguiente tabla:
    \begin{center}
      \begin{tabular}{>{\(}c<{\)}|>{\(}c<{\)}>{\(}c<{\)}}
	    &  a  &  b	\\
	\hline
	S_0 & S_1 & S_0 \\
	S_1 & S_1 & S_2 \\
	S_2 & S_1 & S_3 \\
	S_3 & S_1 & S_0
      \end{tabular}
    \end{center}
  \item % 20152t0p3
    Diseñe y dibuje un DFA que acepte palabras sobre \(\Sigma = \{a, b, c\}\)
    que contengan \(a a\).
  \end{itemize}

% Condiciones generales de tarea 0 de INF-155/ILI-255 2015/2
\section*{Condiciones Generales}

  \begin{itemize}
  \item
    La tarea se realizará \emph{individualmente}
    (esto es grupos de una persona),
    sin excepciones.
  \item
    En caso de que se descubra copia,
    equivale a nota 0 para los estudiantes implicados.
  \item
    Cada respuesta debe estar correctamente justificada, 
    en caso contrario el puntaje obtenido queda 
    sujeto al criterio de los ayudantes.
  \item
    Deberá subir los fuentes {\LaTeX} de su solución
    en el área designada al efecto
    en \href{http://moodle.inf.utfsm.cl}{Moodle}
    bajo el formato
    \texttt{tarea\num-\emph{rol}.tar.gz}.
    El archivo debe contener el directorio \texttt{tarea\num-\emph{rol}},
    en el cual están los archivos pedidos.
  \item
    Por cada día de atraso se descontarán 20 puntos.
    A partir del tercer día de atraso
    no se reciben más tareas y la nota es automáticamente cero.
  \item
    La nota de la tarea puede ser según lo entregado,
    o (en el caso de algunos estudiantes elegidos al azar)
    el resultado de una interrogación en que deberá explicar lo entregado.
    No presentarse a la interrogación significa automáticamente
    nota cero.

    Sobre la nota de la interrogación se aplican los descuentos por atraso.
  \end{itemize}

  \vfill\hfill AR/CG/RS/HvB/\LaTeXe
\end{document}

%%% Local Variables:
%%% mode: latex
%%% TeX-master: t
%%% End:

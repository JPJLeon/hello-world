% Condiciones generales de tareas de Fundamentos de Informática II, 2013
\section*{Condiciones Generales}

  \begin{itemize}
  \item
    La tarea se realizará \emph{individualmente}
    (esto es grupos de una persona),
    sin excepciones.
    \item En caso de que se descubra copia, equivale a nota 0 para los estudiantes implicados.

  \item
	Cada respuesta debe estar correctamente justificada, 
	en caso contrario el puntaje obtenido queda 
	sujeto al criterio de los ayudantes.
  \item
    La tarea debe ser realizada en \LaTeX{}.
  \item
    La tarea debe ser entregada impresa en
    la Secretaría Docente de Informática
    (Piso 1, edificio F3)
    el día indicado en \href{http://moodle.inf.utfsm.cl}{Moodle}. No se revisarán tareas que no hayan sido entregadas impresas.
  \item Deberá subir los fuentes \LaTeX{} de su solución y el pdf en un \emph{tarball}
    en el área designada al efecto
    en \href{http://moodle.inf.utfsm.cl}{Moodle}
    bajo el formato
    \texttt{tarea2-\emph{rol}.tar.gz}.
    El archivo debe contener el directorio \texttt{tarea2-\emph{rol}},
    en el cual están los archivos pedidos (fuentes y pdf o los que sean requeridos).


  \item
    Por cada día de atraso se descontarán 20 puntos.
    A partir del tercer día de atraso
    no se reciben más tareas y la nota es automáticamente cero.
  \item
    La nota de la tarea puede ser según lo entregado,
    o (en el caso de algunos estudiantes elegidos al azar)
    el resultado de una interrogación en que deberá explicar lo entregado.
    No presentarse a la interrogación significa automáticamente
    nota cero.

    Sobre la nota de la interrogación se aplican los descuentos por atraso.

  \end{itemize}

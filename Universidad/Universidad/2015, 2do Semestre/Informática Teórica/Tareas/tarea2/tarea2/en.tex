\documentclass{article}
\usepackage[utf8]{inputenc}
\usepackage{fourier}
\usepackage{amsmath}
\usepackage{amssymb}
\usepackage{amsfonts}
\usepackage[colorlinks]{hyperref}
\usepackage[top = 2.5cm, bottom = 2cm, left = 2cm, right = 2cm]{geometry}

\usepackage[spanish, es-noshorthands]{babel}
\usepackage{tikz}
\usetikzlibrary{automata, positioning}
\usepackage{graphicx}


\definecolor{red}{RGB}{200,0,0}

\title{INF-155: Introducción a la Informática Teórica\\ILI-225: Informática Teórica \\
       Tarea \#2 \\
       \emph{``\#YOLO''}
      }
\author{{Horst von Brand} \and {Alondra Rojas Ruz} \and {Cristobal Galleguillos} \and {Renato Sanhueza Ulsen}}
\date{9 de noviembre 2015}


\begin{document}
\maketitle

\section*{Problemas}
\begin{enumerate}

\item Dada la expresión regular $ R = (c(0|1)^*c)^* $ con $\Sigma=\{0,1,c\}$ y el NFA A:

\begin{center}
\begin{tikzpicture}[shorten >=1pt,node distance=1.5cm,on grid,>=stealth, initial text=Inicio,
every state/.style={draw=red!50,very thick,fill=red!20},
bend angle=35]]

\node[state,initial]  (q_0)                       {$q_0$};
\node[state]          (q_1) [right=of q_0]        {$q_1$};
\node[state]          (q_2) [right=of q_1]        {$q_2$};
\node[state]          (q_3) [right=of q_2]        {$q_3$};
\node[state]          (q_4) [right=of q_3]        {$q_4$};
\node[state]          (q_5) [right=of q_4]   	  {$q_5$};
\node[state]          (q_6) [right=of q_5]        {$q_6$};
\node[state]          (q_7) [right=of q_6]        {$q_7$};
\node[state]          (q_8) [right=of q_7]        {$q_8$};
\node[state,accepting](q_9) [right=of q_8]        {$q_9$};
\path[->] 
(q_0)  edge              node[below]        {\(x\)} (q_1)

(q_1)  edge              node[below]        {\(=\)} (q_2)

(q_2)  edge              node[above]        {\(\epsilon\)} (q_3)
(q_2)  edge   [bend right]           node[below]        {\(\epsilon\)} (q_5) 

(q_3)  edge              node[below]        {\(y\)} (q_4)

(q_4)  edge              node[above]        {\(\epsilon\)} (q_5)
(q_4)  edge   [bend right]           node[above]        {\(\epsilon\)} (q_3)


(q_5)  edge              node[above]        {\(+\)} (q_6)

(q_6)  edge              node[below]        {\(\epsilon\)} (q_7)
(q_6)  edge      [bend left]        node[above]        {\(\epsilon\)} (q_9)


(q_7)  edge              node[above]        {\(z\)} (q_8)

(q_8)  edge              node[below]        {\(\epsilon\)} (q_9)
(q_8)  edge     [bend left]         node[below]        {\(\epsilon\)} (q_7);

\end{tikzpicture}
\end{center}
Con $\Sigma'=\{x,y,z,=,+\}$ se pide lo siguiente:


\begin{itemize}
\item Transformar la expresión regular R en un NFA utilizando el algoritmo de construcción de Thompson. Se evalúa el desarrollo por lo que debe 
incluir todos los pasos del algoritmo. Tanto el desarrollo como el resultado debe ser dibujado usando TikZ. 
\item Transformar el NFA A en un DFA utilizando el algoritmo de construcción de subconjuntos. Se evalúa el desarrollo. El DFA resultante debe
ser dibujado usando TikZ.
\end{itemize}

\item Probar la regularidad o no regularidad del lenguaje \(\mathcal{L}= \big\{pp^{R}q \colon p,q \in \big\{0,1\big\}^{+}\big\} \).\\
\textbf{Hint:} Puede ser una buena idea usar alguna propiedad de clausura para simplificar la demostración. 

\item A una víctima \(x\) de un ramo \(x\) (conocido como TALF) se le pide determinar si el lenguaje \(\mathcal{L} = \big\{ 1^{3}0^{n} : n \geq 1 \big\} \) es regular o no. La víctima sospecha que el lenguaje no es regular (obviando el hecho de que podría describirse como una expresión regular, o mediante un DFA o NFA), por lo que intenta demostrar su idea mediante el \emph{Lema de Bombeo}. Primero, toma la constante N del lema  para una subcadena \(\alpha\beta = 1^{3}0^{x}\), tal que \(|\alpha\beta| \leq N\). Sea \(\alpha = 1\) y \(\beta = 1^{2}0^{x}\) entonces \(\gamma=b^{N-x}\). Cuando \(k = 0\), \(\sigma = \alpha\beta^0\gamma = \alpha\gamma = ab^{N-x}\), que ya no pertenece a \(\mathcal{L}\) y por lo tanto no es regular.
Sin embargo, como nosotros somos más astutos sabemos que su conslusión es errada.

Explique detalladamente con sus palabras cual fue el error cometido por la víctima al aplicar el lema de bombeo, lo que le llevo a tomar una conclusión equivocada.

\end{enumerate}
\newpage
% Condiciones generales de tarea 0 de INF-155/ILI-255 2015/2
\section*{Condiciones Generales}

  \begin{itemize}
  \item
    La tarea se realizará \emph{individualmente}
    (esto es grupos de una persona),
    sin excepciones.
  \item
    En caso de que se descubra copia,
    equivale a nota 0 para los estudiantes implicados.
  \item
    Cada respuesta debe estar correctamente justificada, 
    en caso contrario el puntaje obtenido queda 
    sujeto al criterio de los ayudantes.
  \item
    Deberá subir los fuentes {\LaTeX} de su solución
    en el área designada al efecto
    en \href{http://moodle.inf.utfsm.cl}{Moodle}
    bajo el formato
    \texttt{tarea\num-\emph{rol}.tar.gz}.
    El archivo debe contener el directorio \texttt{tarea\num-\emph{rol}},
    en el cual están los archivos pedidos.
  \item
    Por cada día de atraso se descontarán 20 puntos.
    A partir del tercer día de atraso
    no se reciben más tareas y la nota es automáticamente cero.
  \item
    La nota de la tarea puede ser según lo entregado,
    o (en el caso de algunos estudiantes elegidos al azar)
    el resultado de una interrogación en que deberá explicar lo entregado.
    No presentarse a la interrogación significa automáticamente
    nota cero.

    Sobre la nota de la interrogación se aplican los descuentos por atraso.
  \end{itemize}

  \vfill\hfill CG/AR/RS/HvB/\LaTeXe
\end{document}


